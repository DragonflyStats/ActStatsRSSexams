\documentclass[]{report}

\voffset=-1.5cm
\oddsidemargin=0.0cm
\textwidth = 480pt

\usepackage{framed}
\usepackage{subfiles}
\usepackage{graphics}
\usepackage{newlfont}
\usepackage{eurosym}
\usepackage{amsmath,amsthm,amsfonts}
\usepackage{amsmath}
\usepackage{color}
\usepackage{amssymb}
\usepackage{multicol}
\usepackage[dvipsnames]{xcolor}
\usepackage{graphicx}
\begin{document}
\section*{Discrete Random Variables}
	\begin{enumerate}
		\item  On a roulette wheel there are 37 numbers {0,1,…,36}. 18 numbers are black. If I bet €1 on black, I win €1 if a black number comes up, otherwise I lose my stake. Let X denote my winnings on one bet.
		\begin{itemize}
		\item[(i)]	Calculate E(X) and Var(X)
		\end{itemize}
		
		
		Suppose I make 6 such bets. Let Y denote my total winnings. 
	\begin{itemize}
		\item[(ii)]	Derive the distribution of Y.
	\item[(iii)]	Calculate E(Y) and Var(Y)
	\end{itemize}	

\item The probability distribute of discrete random variable $X$ is tabulated below. There are 5 possible outcome of $X$, i.e. 1, 2, 3, 4 and 5.
	\begin{center}
		\begin{tabular}{|c||c|c|c|c|c|}
			\hline
			$x_i$  & 1 & 2 & 3 & 4 & 5  \\\hline
			$p(x_i)$ & 0.30 & 0.20 & k & 0.10 & 0.20 \\
			\hline
		\end{tabular}
	\end{center}
	
	\begin{itemize}
		\item[a.] (1 Mark) Compute the value of $k$.
		\item[b.] (1 Mark) What is the expected value of X?
		%\item[c.] (1 Mark) Compute the value of $E(X^2)$
		\item[c.] (1 Mark) Given that $E(X^2) = 9.5$, compute the variance of $X$.
	\end{itemize}
	\end{enumerate}

%----------------------------------------------------------%
\begin{enumerate}
	\item  The probability distribution of discrete random variable $X$ is tabulated below. There are 6 possible outcome of $X$, i.e. 0, 1, 2, 4 ,8 and 10.
	\begin{center}
		\begin{tabular}{|c||c|c|c|c|c|c|}
			\hline
			$x_i$  & 0 & 1 & 2 & 4 & 8 & 10 \\\hline
			$P(x_i)$ & 0.25 & 0.15 & 0.25 & 0.15 & k & 0.10\\
			\hline
		\end{tabular}
	\end{center}
	
	\begin{itemize}
		\item[i.] (1 marks) Compute the value for $k$.
		\item[ii.] (3 marks) Determine the expected value $E(X)$.
		\item[iii.] (2 marks) Evaluate $E(X^2)$.
		\item[iv.] (3 marks) Compute the variance of random variable $X$.
	\end{itemize}
	%---------------------------------------------------------- %
	%Question 2
	\item 
	Suppose $X$ is a random variable with 
	\begin{itemize}
		\item $E(X^2)=3.6$
		\item $P(X=2)=0.6$
		\item $P(X=3)=0.1$
	\end{itemize}
	
	\begin{itemize}
		\item[(a)] The random variable takes just one other value besides 2 and 3. This value is greater than 0. What is this value?
		\item[(b)] What is the variance of $X$?
	\end{itemize}
	
	
	%Question 6
	\item Consider the random variables $X$ and $Y$. Both $X$ and $Y$ take the values 0,$\;$1 and 2. 
	The joint probabilities for each pair are given by the following table.
	\begin{center}
		\begin{tabular}{|c|c|c|c|}
			\hline  & $X=0$ & $X=1$ & $X=2$ \\ 
			\hline $Y=0$ & 0.1 & 0.15 & 0.1 \\ 
			\hline $Y=1$ & 0.1 & 0.1 & 0.1 \\ 
			\hline $Y=2$ & 0.2 & 0.05 & 0.1 \\ 
			\hline 
		\end{tabular} 
	\end{center}
	Compute the $E(U)$ expected value of $U$, where $U=X-Y$.
\end{enumerate}
	
	
	
	
	
	
	\vspace{-0.5cm}
	\textbf{Given:}\\
	Suppose $X$ is a random variable with 
	\begin{itemize}
		\item $E(X^2)=3.6$
		\item $P(X=2)=0.6$
		\item $P(X=3)=0.1$
	\end{itemize}
	
	
	\textbf{Questions:}
	\begin{itemize}
		\item[(a)] The random variable takes just one other value besides 2 and 3. This value is greater than 0. What is this value?\\
		\item[(b)] What is the variance of $X$?
	\end{itemize}
	
	%-------------------------------------------------------------%
	
	
	
	
	\textbf{Part a}
	\begin{itemize}
		\item Determine the missing value (let's call it $k$).\\
		\item First we determine the probability of that value. 
		\item We know that $E(X^2)=3.6$. Let use the approach for computing $E(X^2)$.
		
	\end{itemize}
	
	\begin{center}
		\begin{tabular}{|c|c|c|c|}
			\hline
			$x_i$ & \phantom{sp}2\phantom{sp} & \phantom{sp}3\phantom{sp} & \phantom{sp}k\phantom{sp} \\ \hline
			$x^2_i$ & 4 & 9 & $k^2$ \\ \hline
			$p(x_i)$ & 0.6 &  0.1 &  \\ \hline 
		\end{tabular}
	\end{center}
	
	%-------------------------------------------------------------%
	
	
	
	
	\textbf{Part a}
	\begin{itemize}
		\item Determine the missing value (let's call it $k$).\\
		\item First we determine the probability of that value. 
		\item We know that $E(X^2)=3.6$. Let use the approach for computing $E(X^2)$.
		
	\end{itemize}
	
	\begin{center}
		\begin{tabular}{|c|c|c|c|}
			\hline
			$x_i$ & \phantom{sp}2\phantom{sp} & \phantom{sp}3\phantom{sp} & \phantom{sp}k\phantom{sp} \\ \hline
			$x^2_i$ & 4 & 9 & $k^2$ \\ \hline
			$p(x_i)$ & 0.6 &  0.1 & \textbf{0.3} \\ \hline 
		\end{tabular}
	\end{center}
	
	
	
	
	
	
	
	\begin{center}
		\begin{tabular}{|c|c|c|c|}
			\hline
			$x_i$ & \phantom{sp}2\phantom{sp} & \phantom{sp}3\phantom{sp} & \phantom{sp}k\phantom{sp} \\ \hline
			$x^2_i$ & 4 & 9 & $k^2$ \\ \hline
			$p(x_i)$ & 0.6 &  0.1 & 0.3 \\ \hline 
		\end{tabular}
	\end{center}
	
	
	
	\[ E(X^2) =  \sum  x^2_i \cdot p(x_i)  = 3.6 \]
	
	
	
	
	
	
	
	
	
	\begin{center}
		\begin{tabular}{|c|c|c|c|}
			\hline
			$x_i$ & \phantom{sp}2\phantom{sp} & \phantom{sp}3\phantom{sp} & \phantom{sp}k\phantom{sp} \\ \hline
			$x^2_i$ & 4 & 9 & $k^2$ \\ \hline
			$p(x_i)$ & 0.6 &  0.1 & 0.3 \\ \hline 
		\end{tabular}
	\end{center}
	
	
	
	\[ E(X^2) =  \sum  x^2_i \cdot p(x_i)  = 3.6 \]
	
	\[(4 \times 0.6) + (9\times 0.1) + (k^2\times 0.3) = 3.6 \]
	
	
	
	
	
	
	
	\begin{center}
		\begin{tabular}{|c|c|c|c|}
			\hline
			$x_i$ & \phantom{sp}2\phantom{sp} & \phantom{sp}3\phantom{sp} & \phantom{sp}k\phantom{sp} \\ \hline
			$x^2_i$ & 4 & 9 & $k^2$ \\ \hline
			$p(x_i)$ & 0.6 &  0.1 & 0.3 \\ \hline 
		\end{tabular}
	\end{center}
	
	
\begin{itemize}
\item $ E(X^2) =  \sum  x^2_i \cdot p(x_i)  = 3.6 $
\item $(4 \times 0.6) + (9\times 0.1) + (k^2\times 0.3) = 3.6 $
\item $2.4 + 0.9 + (k^2\times 0.3) = 3.6 $
\item $(4 \times 0.6) + (9\times 0.1) + (k^2\times 0.3) = 3.6$
\item $2.4 + 0.9 + (k^2\times 0.3) = 3.6 $
\item $ 3.3 + 0.3k^2 = 3.6$

\item $(4 \times 0.6) + (9\times 0.1) + (k^2\times 0.3) = 3.6 $
\item $2.4 + 0.9 + (k^2\times 0.3) = 3.6$
\item $ 3.3 + 0.3k^2 = 3.6$
\item $0.3k^2 =0.3$
\end{itemize}	
	
	
	
	
	
	
	\[(4 \times 0.6) + (9\times 0.1) + (k^2\times 0.3) = 3.6 \]
	
	\[2.4 + 0.9 + (k^2\times 0.3) = 3.6 \]
	
	\[ 3.3 + 0.3k^2 = 3.6\]
	
	\[0.3k^2 =0.3\]
	\begin{center}
		$k^2 = 1$  \qquad Therefore $k=1$
	\end{center}
	
	%-------------------------------------------------------------%
	
	
	
	
	\textbf{Part b}\\
	Compute the variance of $X$
	
	\[ \mbox{Var}(x) = E(X^2) - \{E(X)\}^2 \]
	
	
	\begin{itemize}
		\item We already know $E(X^2) =3.6$
		\item Need to compute $E(X)$.
	\end{itemize}
	
	%-------------------------------------------------------------%
	
	
	
	
	\textbf{Computing $E(X)$}
	\begin{center}
		\begin{tabular}{|c|c|c|c|}
			\hline
			$x_i$ & 2 & 3 & 1 \\ \hline 
			$p(x_i)$ & 0.6 &  0.1 & 0.3 \\ \hline
		\end{tabular}
	\end{center}
	
	\[ E(X) =  \sum  x_i \cdot p(x_i)   \]
	
	
	
	
	
	
	
	
	
	
	\[E(X) = (2\times 0.6) + (3 \times 0.1) + (1 \times 0.3) = 1.8 \]
	
	
	
	%-------------------------------------------------------------%
	
	
	
	
	
	\textbf{Part b}\\
	Compute the variance of $X$
	
	\[ \mbox{Var}(X) = E(X^2) - \{E(X)\}^2 \]
	
	\[ \mbox{Var}(X) = 3.6 - \{1.8\}^2 \]

	
	\[ \mbox{Var}(X) = 3.6 - 3.24  = \boldsymbol{0.36} \]
	
	%-----------------------------------------------------------------------------------%
	
	
	
	Consider the random variables $X$ and $Y$. $X$ takes the values 0,1 and 2. $Y$ takes the values $0$ and $1$.
	The joint probabilities for each pair are given by the following table.
	\begin{center}
		\begin{tabular}{|c|c|c|c|}
			\hline  & $X=0$ & $X=1$ & $X=2$  \\ 
			\hline $Y=0$ & 0.1  & 0.4 & 0.1 \\ 
			\hline  $Y=1$ & 0.1 & 0.1 & 0.2 \\ 
			\hline 
		\end{tabular} 
	\end{center}
	\begin{itemize}
		\item Compute the expected values of $X$ and $Y$.
		\item Compute the $E(X|Y=1)$
	\end{itemize}
	
	
	
	%-----------------------------------------------------------------------------------%
	
	
	
	
	\textbf{Compute $E(X)$ and $E(Y)$.}\\
	First compute the marginal distributions.
	
	\begin{center}
		\begin{tabular}{|c|c|c|c||c|}
			\hline  & $X=0$ & $X=1$ & $X=2$  &\phantom{spaces}\\ 
			\hline $Y=0$ & 0.1  & 0.4 & 0.1 &\\ 
			\hline  $Y=1$ & 0.1 & 0.1 & 0.2 & \\ \hline
			\hline & & & & \\
			\hline 
		\end{tabular} 
	\end{center}
	
	
	
	%-----------------------------------------------------------------------------------%
	
	
	\textbf{Compute $E(X)$ and $E(Y)$.}\\
	First compute the marginal distributions.
	
	\begin{center}
		\begin{tabular}{|c|c|c|c||c|}
			\hline  & $X=0$ & $X=1$ & $X=2$  &\phantom{spaces}\\ 
			\hline $Y=0$ & 0.1  & 0.4 & 0.1 & 0.6\\ 
			\hline  $Y=1$ & 0.1 & 0.1 & 0.2 & 0.4\\ \hline
			\hline & 0.2& 0.5& 0.3& \\
			\hline 
		\end{tabular} 
	\end{center}
	
	\textbf{Compute $E(X)$}
	\begin{center}
		\begin{tabular}{|c|c|c|c|}
			\hline $x_i$ & $0$ & $1$ & $2$  \\ 
			\hline $p(x_i)$& 0.20 & 0.50 & 0.30 \\
			\hline 
		\end{tabular} 
	\end{center}
	\[ E(X) =  \sum  x_i \cdot p(x_i)   \]
	
	
	\textbf{Compute $E(X)$}
	\begin{center}
		\begin{tabular}{|c|c|c|c|}
			\hline $x_i$ & $0$ & $1$ & $2$  \\ 
			\hline $p(x_i)$& 0.20 & 0.50 & 0.30 \\
			\hline 
		\end{tabular} 
	\end{center}
	
	\[ E(X) =  \sum  x_i \cdot p(x_i)   \]
	
	\[E(X) =(0\times 0.2) + (1 \times 0.5) + (2 \times 0.3)\;  = 1.1\]
	
	\textbf{Compute $E(X)$}
	\begin{center}
		\begin{tabular}{|c|c|c|}
			\hline $y_i$ & $0$ & $1$  \\ 
			\hline $p(y_i)$& 0.60 & 0.40 \\
			\hline 
		\end{tabular} 
	\end{center}
	
	\[ E(Y) =  \sum  y_i \cdot p(y_i)   \]
	
	
	
	\textbf{Compute $E(X)$}
	\begin{center}
		\begin{tabular}{|c|c|c|}
			\hline $y_i$ & $0$ & $1$  \\ 
			\hline $p(y_i)$& 0.60 & 0.40 \\
			\hline 
		\end{tabular} 
	\end{center}
	
	\[ E(Y) =  \sum  y_i \cdot p(y_i)   \]
	
	\[ E(Y) = (0 \times 0.6) +  (1 \times 0.4)  =0.4 \]
	
	
	%-----------------------------------------------------------------------------------%
	
	
	
	Consider the random variables $X$ and $Y$. Both $X$ and $Y$ take the values 0,$\;$1 and 2. 
	The joint probabilities for each pair are given by the following table.
	\begin{center}
		\begin{tabular}{|c|c|c|c|}
			\hline  & $X=0$ & $X=1$ & $X=2$ \\ 
			\hline $Y=0$ & 0.1 & 0.15 & 0.1 \\ 
			\hline $Y=1$ & 0.1 & 0.1 & 0.1 \\ 
			\hline $Y=2$ & 0.2 & 0.05 & 0.1 \\ 
			\hline 
		\end{tabular} 
	\end{center}
	Compute the $E(U)$ expected value of $U$, where $U=X-Y$.
	
	
	%-----------------------------------------------------------------------------------%
	
	
	
	Compute $X-Y$
	\begin{center}
		\begin{tabular}{|c|cc|cc|cc|}
			\hline  & \phantom{space}&$X=0$ &\phantom{space} & $X=1$ & \phantom{space}& $X=2$ \\ 
			\hline $Y=0$& & 0.1 & & 0.15 & & 0.1 \\ 
			\hline $Y=1$& & 0.1 & & 0.1 & & 0.1 \\ 
			\hline $Y=2$& & 0.2 & & 0.05 & & 0.1 \\ 
			\hline 
		\end{tabular} 
	\end{center}
	
	%-----------------------------------------------------------------------------------%
	
	
	
	Compute $X-Y$
	\begin{center}
		\begin{tabular}{|c|cc|cc|cc|}
			\hline  & \phantom{sp}U \phantom{s}&$X=0$ &\phantom{sp}U \phantom{s} & $X=1$ & \phantom{sp}U\phantom{s}& $X=2$ \\ 
			\hline $Y=0$& 0 & 0.1 & 1 & 0.15 & 2& 0.1 \\ 
			\hline $Y=1$& -1& 0.1 & 0& 0.1 & 1& 0.1 \\ 
			\hline $Y=2$& -2 & 0.2 & -1 & 0.05 & 0 & 0.1 \\ 
			\hline 
		\end{tabular} 
	\end{center}
	\bigskip
	
	Determine the probability of each outcome of $U$.
	\\
	\begin{center}
		\begin{tabular}{|c|c|c|c|c|c|}
			\hline $u_i$ & -2 & -1  & 0 & 1 & 2 \\ 
			\hline p $(u_i)$ & \phantom{spaces} & \phantom{spaces} & \phantom{spaces} & \phantom{spaces} & \phantom{spaces} \\ 
			\hline 
		\end{tabular} 
	\end{center}
	
	%-----------------------------------------------------------------------------------%
	
	
	
	
	\[ E(U) =  \sum  u_i \cdot p(u_i)   \]
	\begin{center}
		\begin{tabular}{|c|c|c|c|c|c|}
			\hline $u_i$ & -2 & -1  & 0 & 1 & 2 \\ 
			\hline p $(u_i)$ & \phantom{s}0.20\phantom{s} & \phantom{s}0.15\phantom{s}  & \phantom{s}0.30\phantom{s}  & \phantom{s}0.25\phantom{s} & \phantom{s}0.10\phantom{s} \\ 
			\hline 
			$u_i \cdot p (u_i)$ & & & & & \\\hline
		\end{tabular} 
	\end{center}
	
	%-----------------------------------------------%
	
	
	\[ E(U) =  \sum  u_i \cdot p(u_i)   \]
	\begin{center}
		\begin{tabular}{|c|c|c|c|c|c|}
			\hline $u_i$ & -2 & -1  & 0 & 1 & 2 \\ 
			\hline p $(u_i)$ & \phantom{s}0.20\phantom{s} & \phantom{s}0.15\phantom{s}  & \phantom{s}0.30\phantom{s}  & \phantom{s}0.25\phantom{s} & \phantom{s}0.10\phantom{s} \\ 
			\hline 
			$u_i \cdot p (u_i)$ & -0.40 & -0.15 & 0.00 & 0.25 & 0.20 \\\hline
		\end{tabular} 
	\end{center}
	
	
	\[ E(U) =  \sum  u_i \cdot p(u_i)   \]
	\begin{center}
		\begin{tabular}{|c|c|c|c|c|c|}
			\hline $u_i$ & -2 & -1  & 0 & 1 & 2 \\ 
			\hline p $(u_i)$ & \phantom{s}0.20\phantom{s} & \phantom{s}0.15\phantom{s}  & \phantom{s}0.30\phantom{s}  & \phantom{s}0.25\phantom{s} & \phantom{s}0.10\phantom{s} \\ 
			\hline 
			$u_i \cdot p (u_i)$ & -0.40 & -0.15 & 0.00 & 0.25 & 0.20 \\\hline
		\end{tabular} 
		
		
	\end{center}
	
	\[ E(U) =  \sum  u_i \cdot p(u_i)   \]
	\begin{center}
		\begin{tabular}{|c|c|c|c|c|c|}
			\hline $u_i$ & -2 & -1  & 0 & 1 & 2 \\ 
			\hline p $(u_i)$ & \phantom{s}0.20\phantom{s} & \phantom{s}0.15\phantom{s}  & \phantom{s}0.30\phantom{s}  & \phantom{s}0.25\phantom{s} & \phantom{s}0.10\phantom{s} \\ 
			\hline 
			$u_i \cdot p (u_i)$ & -0.40 & -0.15 & 0.00 & 0.25 & 0.20 \\\hline
		\end{tabular} 
		
		
	\end{center}
	
	\[E(U) = -0.10\]
	
	\[ E(U) =  \sum  u_i \cdot p(u_i)   \]
	\begin{center}
		\begin{tabular}{|c|c|c|c|c|c|}
			\hline $u_i$ & -2 & -1  & 0 & 1 & 2 \\ 
			\hline p $(u_i)$ & \phantom{s}0.20\phantom{s} & \phantom{s}0.15\phantom{s}  & \phantom{s}0.30\phantom{s}  & \phantom{s}0.25\phantom{s} & \phantom{s}0.10\phantom{s} \\ 
			\hline 
			$u_i \cdot p (u_i)$ & -0.40 & -0.15 & 0.00 & 0.25 & 0.20 \\\hline
		\end{tabular} 
		
		
	\end{center}
	
	\[E(U) = -0.10\]
	
	
	
\end{document}