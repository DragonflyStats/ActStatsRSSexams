\documentclass[]{report}

\voffset=-1.5cm
\oddsidemargin=0.0cm
\textwidth = 480pt

\usepackage{framed}
\usepackage{subfiles}
\usepackage{graphics}
\usepackage{newlfont}
\usepackage{eurosym}
\usepackage{amsmath,amsthm,amsfonts}
\usepackage{amsmath}
\usepackage{color}
\usepackage{amssymb}
\usepackage{multicol}
\usepackage[dvipsnames]{xcolor}
\usepackage{graphicx}
\begin{document}
	
	Probability Theory
	Tutorial Sheet 1

\begin{enumerate}
	\item I pick 3 cards from a pack of 52. Calculate the probability that 
	i)	I pick exactly one spade
	ii)	I pick at least one spade
	iii)	I pick exactly one spade, given that I pick at least one spade.
	
\item The probability that a new born child is a boy is 0.51. Calculate the probability that in a family with 3 children
	i)	there are two girls and one boy
	ii)	all the children are boys, given that the eldest and the youngest child are of the same sex.
	
\item	A coin is thrown 4 times. Calculate 
	i)	The probability of throwing 4 heads
	ii)	The probability of throwing 3 heads 
	iii)	The probability of throwing 3 heads, given that the result of the first roll is tails.
\end{enumerate}	

	
	4.	The following contingency table gives the results of operations in a hospital according to the complexity of the operation
	
	Simple	Complex
	Successful	1990	950
	Unsuccessful	10	50
	
	Let A be the event that an operation is simple and B the event that an operation is successful. 
	
	Calculate P(A), P(B), P(A|B), P(A|BC ), P(B|A), P(B|AC ).
	
	5. Prove the law of total probability, i.e. when A1 , A2 , …, An form a partition
	
	P(B) = P(B|A1)P(A1) + P(B|A2)P(A2) + … + P(B|An)P(An)
	
	6. A survey of students was carried out before an exam. 20% of students stated that they were very confident, 50% stated that they were confident and 30% were unconfident. 80% of those who said they were very confident, 50% of those who said they were confident and 20% of those who said they were unconfident got at least a B1 grade. Calculate 
	i)	the probability that a randomly picked student  got at least a B1 grade.
	ii)	given that the student got at least a B1 grade, what is the probability that he/she was very confident.
	iii)	given that the student got less than a B1 grade, what is the probability that he/she was unconfident.
	
\begin{enumerate}
	\item One in 10 000 people suffer from a particular disease. Given a person has the disease, a test for the disease is always positive (indicates that the person has the disease). Given a person does not have the disease, a test for the disease is positive with probability 0.01.
	i) calculate the probability that when a randomly chosen person is tested, the result is positive. 
	ii) calculate the probability that an individual has the disease, given that the test result was positive.
	
	\item A machine is composed of 3 components, which function independently of each other with probabilities p1, p2 and p3, respectively. Calculate the probability that the machine works when
	a)	the machine only works when all the components are working
	b)	the machine works when at least one of the components works.
	
	\item  A die is thrown twice. A is the event that the sum is 7. B is the event that the first die roll results in a 1. C is the event that the second die roll results in a 6. 
	i)	Are the events A, B and C independent?
	ii)	Are the events A, B and C pairwise independent?
\item A coin is tossed until it falls on the same side twice in a row.
\begin{itemize}
	\item[(i)] Define the set of elementary events of such an experiment.
	\item[(ii)]  Calculate the probability that the coin is thrown exactly 5 times.
	\item[(iii)]  Calculate the probability that the number of throws is even.
\end{itemize}
	
\end{enumerate}	
	
\newpage	
%================================================================%
	
 
	
	
	
	Probability Theory 
	Tutorial Sheet 2

\begin{enumerate}	
\item a) I throw a coin 10 times. Calculate the probability that 
	i) I throw exactly 3 heads
	ii) I throw at least 2 heads
	


%--------------------------------------------------%
	
\item During the day, cars pass along a point on a remote road at an average rate of one per 20 minutes. Calculate the probability that 
\begin{itemize}
	\item[(i)]	in the course of an hour no car passes. 
	\item[(ii)]	in the course of 30 minutes exactly 4 cars pass
	\item[(iii)]	in the course of 30 minutes at least two cars pass
\end{itemize}


%--------------------------------------------------%	
\item	In a newspaper on average 1 in 10 000 characters is incorrectly printed. Suppose the paper contains 50 000 characters. Calculate the exact probability that 
 \begin{itemize}
\item[(i)]	no printing errors are made
\item[(ii)]	at least 3 errors are made
 \end{itemize}	
	Using the appropriate approximation, estimate these two probabilities. 
%--------------------------------------------------%	
\item In the British national lottery 6 numbers are chosen without replacement from 49. Calculate the probability of 
	i)	winning the jackpot (choosing all 6 numbers correctly)
	ii)	winning the smallest prize (choosing 3 of the 6 numbers correctly)
	iii)	choosing at least one of the numbers correctly.
	
\item	I obtain a hand of 13 cards. Calculate the probability of
	i) obtaining 2 aces.
	ii) obtaining 2 aces and 2 kings.
	iii) obtaining 2 aces or 2 kings.
	
	7.  A computer chooses a number at random n times from the set {1, 2, 3, 4, 5} (with replacement). Let S denote the sum of the numbers chosen. Show that
	E(S) = 3n and Var(S) = 2n.
	
\item	Let Y=aX+b. Prove that i) E(Y)=aE(X)+b, ii) Var(Y)=a2Var(X).
	
\item	a) Suppose X has a geometric distribution with parameter p. From the standard interpretation of the geometric distribution, conditioning on whether X=1 or not and using the memoryless property of the geometric distribution
	i)	calculate E(X)
	ii)	calculate Var(X). [Hint: first calculate E(X2)].
	
\item	The probabilities with which Liverpool win, draw or lose a premier league match are 0.5, 0.3 and 0.2, respectively. Calculate the probability that in 8 matches
	i)	Liverpool lose 2 matches.
	ii)	Liverpool win 5 matches and draw 2.
	
\item	A die is thrown until either a six is obtained or five rolls are done (a truncated geometric distribution). Let X be the number of rolls.
	i)	Define the cumulative distribution function of X.
	ii)	Find the median of X.

\end{enumerate}
%====================================================%	
\newpage	
	
	Probability Theory
	Tutorial Sheet 3
\begin{enumerate}	
\item During the day, cars pass along a point on a remote road at an average rate of one per 20 minutes. Calculate the probability that 
	i)	The time between the arrival of 2 cars is greater than 1 hour.
	ii)	The time between the arrival of 2 cars is less than 10 minutes 
	iii)	The time between the arrival of 2 cars is greater than 20 minutes, but less than 40 minutes. 
	
\item 	Suppose X has an Exp(λ) distribution.
	i)	Derive E(X) and Var(X). [Use the fact that limx->∞ xk e-λx = 0, for any positive integer k].
	ii)	Using induction, show that the k-th moment of X is given by k!/λk .
	iii) 	      Show that X has the memoryless property.
	

	
\item IQ  is defined to have a normal distribution with mean 100 and standard deviation 15. 
	a) Calculate the probability that a person’s IQ is
	i) greater than 130
	ii) less than 110
	iii) between 82 and 120
	
	b) Calculate the IQ that is exceeded by 15\% of the population.
	
\item  An elevator can lift 600kg. 4 men and 4 women are in the lift. The mean mass of males is 80kg with a standard deviation of 20kg, the mean mass of females is 65kg with a standard deviation of 15kg. Assuming the weights of these individuals are independent and approximately normally distributed, estimate the probability that the elevator will lift these passengers. 
	
	Note: 1) Use the results regarding the sum of independent, normally distributed random variables. 
	2)	The sum of the masses of the males is the sum of 4 random variables.
	
\item 	The lengths of Padraig Harrington's drives are normally distributed with mean of 250m and standard deviation of 15m. The lengths of Rory McIlroy's drives are normally distributed with a mean of 245m and a standard deviation of 20m. Calculate the probability that Rory drives further than Padraig.
	
\item 	Suppose X has the following cumulative distribution function. F(x)=0, for x≤0, 
	F(x)=1, for x≥5 and F(x) = x3/125 for 0≤ x ≤ 5. Derive E(X) and Var(X).
	
	
	
\item  Suppose calls come into a call centre randomly at a rate of one per 30 seconds.
	i) What is the distribution of the time to the second call?
	ii) Using this distribution, calculate the probability that the second call arrives within a minute. 
	iii) Using the appropriate discrete distribution, calculate the probability that at least 2 calls are received in a minute (note this probability has to be the same as above).
	iv) What is the exact distribution of the time to the 200th call?
	v) Using the central limit theorem, give the normal distribution which approximates the distribution from iv).
	vi) Using your answer from v), estimate the probability that the time to the 200th call is less than 102 minutes.
	
\item A coin is tossed 100 times. Using the appropriate approximating distribution, estimate the probability that
	a) exactly 46 heads are thrown
	b) between 48 and 59 heads (inclusively) are thrown.

\end{enumerate}	
\newpage


\section{Probability Tutorial Questions -Independent and Dependent Events}

\textbf{Example 1:} What is the probability of rolling two consecutive fives on a six-sided die?
\begin{itemize}
	\item You know that the probability of rolling one five is 1/6, and the probability of rolling another five with the same die is also 1/6.
	\item These are independent events, because what you roll the first time does not affect what happens the second time; you can roll a 3, and then roll a 3 again.
\end{itemize}

%----------------------------------------------------------------------%

\subsection{Example 2:} Two cards are drawn randomly from a deck of cards. What is the likelihood that both cards are clubs?
\begin{itemize}
	\item The likelihood that the first card is a club is 13/52, or 1/4. (There are 13 clubs in every deck of cards.) Now, the likelihood that the second card is a club is 12/51.
	\item You are measuring the probability of dependent events. This is because what you do the first time affects the second; if you draw a 3 of clubs and don't put it back, there will be one less fewer club and one less card in the deck (51 instead of 52).
\end{itemize}
%-----------------------------------------------------------------------%


\subsection{Independent and Dependent Events}

\textbf{Example 3:} A jar contains 4 blue marbles, 5 red marbles and 11 white marbles. If three marbles are drawn from the jar at random, what is the probability that the first marble is red, the second marble is blue, and the third is white?
\begin{itemize}
	\item The probability that the first marble is red is 5/20, or 1/4. 
	\item The probability of the second marble being blue is 4/19, since we have one fewer marble, but not one fewer blue marble. 
	\item And the probability that the third marble is white is 11/18, because we've already chosen two marbles. This is another measure of a dependent event.
\end{itemize}
\subsubsection{Example:}
Of 200 employees of a company, a total of 120 smoke cigarettes:
60\% of the smokers are male and 80\% of the non smokers are
male. What is the probability that an employee chosen at random:
\begin{itemize}
	\item[1.]  is male or smokes cigarettes
	\item[2.] is female or does not smoke cigarettes
	\item[3.] either smokes or does not smoke
\end{itemize}




Question 1
Events  (Important the event names are explained)

F: 		Student plays football			P(F) = 50%
S: 		Student does Swimming		P(S) = 20%
F and S:            Student takes part in both swimming and football  
P(S and F) = 15%

Find P (F or S)
Use addition rule

P (F or S)  =  P(F) + P(S) – P( Fand S)
=  50% + 20 % - 15%
=   55%

(We subtract 15% to stop the “boths” getting counted twice)

Probability of playing neither
This is the complement event of playing one or both sports.
P(Neither) = 1 –  P( F or S)  = 45%

Question 2 
2 components A and B.
P(A) = event that A is working		P(A) = 0.98
P(B) = event that B is working			P(B) = 0.95
P(A and B) = event that both A and B are working = P(A) x P(B) = 0.98 x 0.95 = 0.931

Question 3
Lots of useless information.
Complement event of at least one working is that they are both broken.

Answer  100 – 4\% = 96%

Question 4


\begin{verbatim}
Events 
A components from supplier A  	P (A) = 0.8 
B components from supplier B  	P (B) = 0.2 
F = resistor fails test

1\% Probability of a Failure given that component is from Supplier A.      P(F|A) = 0.01
3\% Probability of a Failure given that component is from Supplier B.      P(F|B) = 0.03

\end{verbatim}
Probability of flaw : P(F)
Failed resistors either come from A or B
\[	P( F) =  P ( F \mbox{ and } A)  +  P( F \mbox{ and } B)\]
Use conditional Probability  rule		
\[P(F) = P(F|A)\times P(A)  + P(F|B)\times P(B)\]

\[P(F)  =  ( 0.01 x 0.8 ) + ( 0.03 x 0.2) = (0.008) + (0.06) = 0.014	\]
	Answer 1.4%

Part ii
Given that a component failed, what was the probability of coming from A
P(A|F) 

P(A|F) = P(A and F)  / P(F)		We found P(A and F) earlier ; 0.008

P(A|F) = 0.008/0.014 =  0.57			[answer : 57%]
















Question 6

\begin{verbatim}
A = coming from supplier A		P(A) = 50\%
B = coming from supplier B		P(B) = 30\%	
C = coming from supplier C		P( C)= 20\%

D = being defective

Given P(D|A) = 0.01	P(D|B) = 0.03	P(D|C) = 0.04


P(D) = P( D and A) + P( D and B) + P(D and C)
= P(D|A).P(A) +  P(D|B).P(B) + P(D|C).P(C)
= (0.01 X 0.5)  + (0.03 X 0.30) + (0.04 X 0.20)
= 0.005 + 0.009 +  0.008
= 0.022 [I.E. 2.2%]


P(B|D) = P(B and D) / P(D)  
= 0.009/0.022 =  0.409		[41%]
\end{verbatim}
% Question 5 in lectures
% Answer P(D) =	0.0395 (3.95%)
% P(C|D) = 0.0506		(5.06%)






\subsection{Example: Factorials }

\noindent \textbf{Examples:}

\begin{itemize}
	\item $4! = 4 \times 3 \times 2 \times 1 = 24$
	\item $7! = 7 \times 6 \times 5 \times 4 \times 3 \times 2 \times 1 = 5,040$
	\item $1! = 1$
	\item $0! = 1 $
\end{itemize}
Importantly 
\[n! = n \times (n-1)!  = n \times (n-1) \times (n-2)! \]
For Example
\[6! = 6 \times 5!  = 6 \times 5 \times 4! \]

%%- \frametitle{Factorials Numbers}


\begin{itemize}
	\item $3!  = 3 \times 2  \times 1 = 6 $
	
	\item $4!  = 4 \times 3! = 4 \times 3 \times 2 \times 1 = 24$
	
	\item factorials 
	\[ n! = (n)\times (n-1)\times(n-2) \times \ldots \times 1 \]
	\begin{itemize}
		\item $5! = 5 \times 4 \times 3 \times 2 \times 1 = 120 $
		\item $3! = 3 \times 2 \times 1$
	\end{itemize}
	\item Zero factorial : Remark $0! = 1$ not $0$.
	\[ 0! =  1 \]
\end{itemize}	

\subsection{Example 1}

\[ \binom 5 2  = \frac{5!}{2!\;(5-2)!} = \frac{5.4.3!}{2! .3!} = \frac{5.4}{2.1} = 10\]


\[ \binom 5 0   = \frac{5!}{0!\;(5-0)!} = \frac{5!}{0! .5!} = \frac{5!}{2!} = 1\]
Recall $0! =1$


\subsection{Addition Rule: Worked Example}
Suppose we wish to find the probability of drawing either a Queen or a Heart
in a single draw from a pack of 52 playing cards. We define the events $Q$ =
`draw a queen' and $H$ = `draw a heart'.
\begin{itemize}
	\item $P(Q)$ probability that a random selected card is a Queen
	\item  $P(H)$ probability that a randomly selected card is a Heart.
	\item  $P(Q\cap H)$ probability that a randomly selected card is the Queen of
	Hearts.
	\item  $P(Q\cup H)$ probability that a randomly selected card is a Queen or a Heart.
\end{itemize}



\noindent \textbf{Probability: Worked Example }
An electronics assembly subcontractor receives resistors from two suppliers: Deltatech provides
70\% of the subcontractors's resistors while another company, Echelon, supplies the remainder.
\\
1\% of the resistors provided by Deltatech fail the quality control test, while 2\% of the
resistors from Echelon also fail the quality control test.

\begin{enumerate}
	\item What is the probability that a resistor will fail the quality control test?
	\item What is the probability that a resistor that fails the quality control test was supplied by Echelon?
\end{enumerate}


\noindent \textbf{Probability: Worked Example}
Firstly, let's assign names to each event.
\begin{itemize}
	\item $D$ : a randomly chosen resistor comes from Deltatech.
	\item $E$ : a randomly chosen resistor comes from Echelon.
	\item $F$ : a randomly chosen resistor fails the quality control test.
	\item $P$ : a randomly chosen resistor passes the quality control test.
\end{itemize}
\bigskip
We are given (or can deduce) the following probabilities:
\begin{itemize}
	\item $P(D) = 0.70$,
	\item $P(E) = 0.30$.
\end{itemize}



{
	\noindent \textbf{Probability: Worked Example}
	
	We are given two more important pieces of information:
	\begin{itemize}
		\item The probability that a randomly chosen resistor fails the quality control test, given that it comes from Deltatech: $P(F|D) = 0.01 $.
		\item The probability that a randomly chosen resistor fails the quality control test, given that it comes from Echelon: $P(F|E) = 0.02$.
	\end{itemize}
	
	
	\noindent \textbf{Probability: Worked Example}
	
	The first question asks us to compute the probability that a randomly chosen resistor fails the quality control test. i.e. $P(F)$.\\
	\bigskip
	All resistors come from either Deltatech or Echelon. So, using the \textbf{\emph{law of total probability}}, we can express $P(F)$ as follows:
	
	\[ P(F)  = P(F \cap D) + P(F \cap E) \]
	
	
	
	Using the \textbf{\emph{multiplication rule}}  i.e. $P(A \cap B) = P(A|B) \times P(B)$, we can re-express the formula as follows
	
	\[ P(F)  = P(F|D) \times P(D) + P(F|E) \times P(E) \]
	
	We have all the necessary probabilities to solve this.
	
	\[ P(F)  = 0.01 \times 0.70 + 0.02 \times 0.30   = 0.007 + 0.006  = 0.013\]
	
	
	
	\begin{itemize}
		\item
		The second question asks us to compute probability that a resistor that fails the quality control test was supplied by Echelon.
		\item In other words; of the resistors that did fail the quality test only, what is the probability that a randomly selected resistor was supplied by Echelon?
		\item We can express this mathematically as $P(E|F)$.
		\item We can use \textbf{\emph{Bayes' theorem}} to compute the answer.
	\end{itemize}
	
	
	Recall Bayes' theorem
	\[ P(A|B) = \frac{P(B|A)\times P(A)}{P(B)} \]
	\bigskip
	
	\[ P(E|F) = \frac{P(F|E)\times P(E)}{P(F)}  =  \frac{0.02 \times 0.30}{0.013} = 0.46\]
	
}




\subsection{Probability: Worked Example }
An electronics assembly subcontractor receives resistors from two suppliers: Deltatech provides
70\% of the subcontractors's resistors while another company, Echelon, supplies the remainder.
\\
1\% of the resistors provided by Deltatech fail the quality control test, while 2\% of the
resistors from Echelon also fail the quality control test.

\begin{enumerate}
	\item What is the probability that a resistor will fail the quality control test?
	\item What is the probability that a resistor that fails the quality control test was supplied by Echelon?
\end{enumerate}

Firstly, let's assign names to each event.
\begin{itemize}
	\item $D$ : a randomly chosen resistor comes from Deltatech.
	\item $E$ : a randomly chosen resistor comes from Echelon.
	\item $F$ : a randomly chosen resistor fails the quality control test.
	\item $P$ : a randomly chosen resistor passes the quality control test.
\end{itemize}
\bigskip
We are given (or can deduce) the following probabilities:
\begin{itemize}
	\item $P(D) = 0.70$,
	\item $P(E) = 0.30$.
\end{itemize}



We are given two more important pieces of information:
\begin{itemize}
	\item The probability that a randomly chosen resistor fails the quality control test, given that it comes from Deltatech: $P(F|D) = 0.01 $.
	\item The probability that a randomly chosen resistor fails the quality control test, given that it comes from Echelon: $P(F|E) = 0.02$.
\end{itemize}


The first question asks us to compute the probability that a randomly chosen resistor fails the quality control test. i.e. $P(F)$.\\
\bigskip
All resistors come from either Deltatech or Echelon. So, using the \textbf{\emph{law of total probability}}, we can express $P(F)$ as follows:

\[ P(F)  = P(F \cap D) + P(F \cap E) \]



Using the \textbf{\emph{multiplication rule}}  i.e. $P(A \cap B) = P(A|B) \times P(B)$, we can re-express the formula as follows

\[ P(F)  = P(F|D) \times P(D) + P(F|E) \times P(E) \]

We have all the necessary probabilities to solve this.

\[ P(F)  = 0.01 \times 0.70 + 0.02 \times 0.30   = 0.007 + 0.006  = 0.013\]





{
	\noindent \textbf{Probability: Worked Example}
	
	\begin{itemize}
		\item
		The second question asks us to compute probability that a resistor that fails the quality control test was supplied by Echelon.
		\item In other words; of the resistors that did fail the quality test only, what is the probability that a randomly selected resistor was supplied by Echelon?
		\item We can express this mathematically as $P(E|F)$.
		\item We can use \textbf{\emph{Bayes' theorem}} to compute the answer.
	\end{itemize}
	
	
	
	\noindent \textbf{Probability: Worked Example}
	Recall Bayes' theorem
	\[ P(A|B) = \frac{P(B|A)\times P(A)}{P(B)} \]
	\bigskip
	
	\[ P(E|F) = \frac{P(F|E)\times P(E)}{P(F)}  =  \frac{0.02 \times 0.30}{0.013} = 0.046\]
	
}

\subsection{Independent Events : example}
Competitors A and B fire at their respective targets. The probability that A hits a target is 1/3 and the probability that B hits a target is 1/5. Find the probability that:
\begin{itemize}
	\item[(i)] (2 marks) A does not hit the target,
	\item[(ii)](2 marks)  both hit their respective targets,
	\item[(iii)](2 marks)  only one of them hits a target,
	\item[(iv)](2 marks) neither A nor B hit their targets.
\end{itemize}


\subsection*{Q2. Probability (4 Marks)} % 5 Marks

The following contingency table illustrates the number of 200 students in different
departments according to gender.

\begin{center}
	\begin{tabular}{|c|c|c|c|c|}
		\hline
		% after \\: \hline or \cline{col1-col2} \cline{col3-col4} ...
		& Physics & Biology & Chemistry & Total \\\hline
		Males & 30 & 20 & 50 & 100 \\  \hline
		Females & 20 & 50 & 30 & 100 \\ \hline
		Total & 50 & 70 & 80 & 200 \\
		\hline
	\end{tabular}
\end{center}

\begin{itemize}
	\item[a.] (1 mark) What is the probability that a randomly chosen person from the sample is a
	Chemistry student?
	\item[b.] (1 mark) What is the probability that a randomly chosen person from the sample is both female and studying Biology?
	\item[c.] (1 mark) Given that the student is female, what is the probability that she is an
	Biology student?
	\item[d.] (1 mark) Given that a student studies Biology, what is the probability that the student is female?
\end{itemize}

\bigskip
\subsection*{Question 12. (8 marks) } % 10 Marks
On completion of a programming project, four programmers from a
team submit a collection of subroutines to an acceptance group. The
following table shows the percentage of subroutines each programmer
submitted and the probability that a subroutine submitted by each
programmer will pass the certification test based on historical data.
\begin{center}
	\begin{tabular}{|c|c|c|c|c|}
		\hline
		% after \\: \hline or \cline{col1-col2} \cline{col3-col4} ...
		Programmer & 1 & 2 & 3 & 4 \\
		Proportion of subroutines submitted & 0.10 & 0.20 & 0.40 & 0.30 \\
		Probability of acceptance & .55 & .60 & .95 & .75 \\
		\hline
	\end{tabular}
\end{center}
\begin{itemize}
	\item[a.] What is the proportion of subroutines that pass the acceptance test?
	\item[b.] After the acceptance tests are completed, one of the subroutines is
	selected at random and found to have passed the test. What is the
	probability that it was written by Programmer l?
\end{itemize}






%
%
% x=seq(2,18,length=1600)
% y=dnorm(x,10,2)
% plot(x,y,type="l",  col="black")
% x=seq(10,12.4,length=240)
% y=dnorm(x,10,2)
% polygon(c(10,x,12.4),c(0,y,0),col="wheat")
% abline(h=0)
% text(14,0.15,"P(10 < X < 12.4)")
%
%
%
%
% x=seq(2,18,length=1600)
% y=dnorm(x,10,2)
% plot(x,y,type="l",  col="black")
% L=2
% U=9.7
% x=seq(L,U,length=((U-L)*100) )
% y=dnorm(x,10,2)
% polygon(c(L,x,U),c(0,y,0),col="wheat")
% abline(h=0)
% text(14,0.15,"P(10 < X < 12.4)")
%
%
%abline(v=10)


\end{document}