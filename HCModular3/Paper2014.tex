EXAMINATIONS OF THE HONG KONG STATISTICAL SOCIETY
HIGHER CERTIFICATE IN STATISTICS, 2008
(Modular format)
MODULE 3 : Basic statistical methods
Time allowed: One and a half hours
Candidates should answer THREE questions.
Each question carries 20 marks.
The number of marks allotted for each part-question is shown in brackets.
Graph paper and Official tables are provided.
Candidates may use calculators in accordance with the regulations published in
the Society's "Guide to Examinations" (document Ex1).
The notation log denotes logarithm to base e.
Logarithms to any other base are explicitly identified, e.g. log10.
Note also that is the same as n
Cr. ⎟
⎟
⎠
⎞
⎜
⎜
⎝
⎛
r
n
 1 HC Module 3 2008
This examination paper consists of 3 printed pages each printed on one side only.
This front cover is page 1.
Question 1 starts on page 2.
There are 4 questions altogether in the paper.
©RSS 2008 
1. A population consists of the five values 1, 4, 9, 16 and 25.
(i) Calculate the population mean and variance. (3)
(ii) Write down all the samples of size two that may be drawn, with replacement,
from this population, and calculate the sample mean of each. (6)
(iii) Let X denote the mean of a random sample of size two drawn, with
replacement, from this population. Write down the expected value and
variance of X . (3)
(iv) For X as in part (iii), find P( X > 16.5). Find also an approximation to this
quantity, using an appropriate Normal approximation [use of a continuity
correction is not expected], and comment briefly on your results. (8)
2. (i) Two independent random samples are taken from separate populations. The
first sample, of size n1, is taken from a population in which the proportion of
members with a certain attribute A is p1. The second sample, of size n2, is
from a population containing proportion p2 with attribute A. There are R1
members of the first sample, and R2 of the second, with attribute A.
Write down a Normal approximation to the distribution of the difference
1
1 2
R R2 D
n n = − , and deduce an approximate test of the null hypothesis that the
proportion who possess attribute A is the same in each population, against a
two-sided alternative hypothesis, when n1 and n2 are large.
(6)
(ii) Random samples of 1000 males and 1000 females were assessed for colourblindness,
and the results are shown in the following table.
Males Females
Colour-blind 80 6
Not colour-blind 920 994
Use the result of part (i) to test at the 0.1% level the hypothesis of equality of
the proportions of colour-blind among all males and among all females, against
the alternative that the proportions differ. State your conclusions clearly.
(7)
(iii) Use the data given in part (ii) to construct an approximate 95% confidence
interval for the difference in the proportions of colour-blind amongst all the
males and amongst all the females. Interpret this confidence interval.
(7)
2
Turn over 
3. In a survey of borrowers of books from two public libraries, a random sample of
borrowing transactions (all of independent individuals) is taken from each library and
the borrowing times are noted. Data for each library are put in ascending order for
convenience.
Borrowing Times (days) Total
Library 1 2, 3, 5, 10, 15, 21, 26, 38 120
Library 2 1, 4, 6, 7, 12, 18, 27, 36, 42 153
You are given that the respective sums of squares of the observations for libraries 1
and 2 are 2924 and 4359.
(i) A trainee librarian who once attended a statistics course suggests using a t test
to examine the hypothesis that the mean borrowing times for the two libraries
are equal, against the alternative that they differ. State the assumptions
necessary for the validity of this test. Perform the test and report your
conclusions.
(8)
(ii) Perform a suitable non-parametric test of the hypothesis that the median
borrowing times for the two libraries are equal, against the alternative that they
differ. Report your conclusions and state the assumptions necessary for the
validity of this test.
(8)
(iii) Discuss critically which test you regard as more appropriate. Provide
graphical and/or other evidence to support your argument.
(4)
4. The weights in grams of a random sample of 10 cherry tomatoes taken from the plants
in a large greenhouse are 37, 39, 40, 33, 41, 39, 36, 46, 35 and 44.
(i) Calculate the mean and variance of the observed weights.
(4)
(ii) Assuming that the distribution underlying the data may be taken as
approximately Normal, give 95% confidence intervals for (a) the mean and (b)
the variance of the weights of cherry tomatoes in the greenhouse. State clearly
any formulae you use for these calculations.
(9)
(iii) In order to be graded as acceptable for sale, the cherry tomatoes should not be
too small or very variable in size. Test, in each case at the 5% level of
significance, (a) the hypothesis that the mean weight is 42 g against the
hypothesis that it is less than 42 g, (b) the hypothesis that the variance of
weight is 12 g2
, against the hypothesis that it is greater than 12 g2
.
(7)
3 
