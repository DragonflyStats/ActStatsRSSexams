EXAMINATIONS OF THE HONG KONG STATISTICAL SOCIETY
GRADUATE DIPLOMA, 2012
MODULE 1 : Probability distributions
Time allowed: Three Hours
Candidates should answer FIVE questions.
All questions carry equal marks.
The number of marks allotted for each part-question is shown in brackets.
Graph paper and Official tables are provided.
Candidates may use calculators in accordance with the regulations published in
the Society's "Guide to Examinations" (document Ex1).
The notation log denotes logarithm to base e.
Logarithms to any other base are explicitly identified, e.g. log 10 .
Note also that
n
is the same as n C r .
r
1
GD Module 1 2012
This examination paper consists of 6 printed pages, each printed on one side only.
This front cover is page 1.
Question 1 starts on page 2.
There are 8 questions altogether in the paper.
© RSS 20121.
1
,
n
An experiment has n possible outcomes labelled 1, 2, ..., n, each with probability
for some n > 1.
(a)
Suppose that n = 8, and A = {1, 2, 3, 4}, B = {1, 2, 5, 6} and C = {1, 3, 7, 8}.
(i) Show that P(A
B) = P(A)P(B), and that P(A
B
C) = P(A)P(B)P(C).
(4)
(ii) By considering another intersection, show that A, B and C are not
independent.
(2)
(iii) Construct an event D so that A, B and D are independent.
(4)
(b)
By considering the cases n = 6m + r for r = 0, 1, 2, 3, 4, 5, find the values of n
for which the events
E = {Outcome divisible by 2} and F = {Outcome divisible by 3}
are independent.
(10)

###############################################
2.
Suppose X has a Poisson distribution with parameter .
(i)
Given an integer r
1, show that E ( X ( X 1)( X
Deduce the coefficient of skewness of X.
2)
( X r 1) )
r
.
(8)
[The coefficient of skewness of a random variable Y with mean
E ( Y
) 3
2
0 is
.]
3
(ii)
Find E
1
X 1
and variance
.
(4)
(iii)
Suppose first that > 1 is not an integer. Show that the sequence p 0 , p 1 , p 2 , ...,
where p k denotes the probability P(X = k), increases to a unique maximum,
then decreases. Describe the behaviour of this sequence when
1 is an
integer.
(8)
2
Turn ove

######################################################################

3.
(i) A random variable has the continuous uniform distribution over the interval
( b a ) 2
[a, b]. Write down its mean and show that its variance is
.
12
(6)
(ii) Let X be a random variable with mean , variance 2 , and 0 X 1. Given X,
the independent random variables V and W are uniformly distributed over the
intervals [0, X] and [X, 1] respectively. Write Y = W – V. Find the mean and
variance of Y, in terms of and 2 .
(14)
[You may use the results
E(Y) = E ( E(Y | X) ) and Var(Y) = E ( Var(Y | X) ) + Var ( E(Y | X) ) .]

Sure, let's break it down:

### (i) Mean and Variance of Continuous Uniform Distribution

For a random variable \(X\) uniformly distributed over the interval \([a, b]\):

- The probability density function (pdf) is:
  \[
  f(x) = \frac{1}{b - a}, \quad \text{for } a \leq x \leq b
  \]

- The mean (expected value) \(E(X)\) is:
  \[
  E(X) = \frac{a + b}{2}
  \]

- The variance \( \text{Var}(X) \) is given by:
  \[
  \text{Var}(X) = E(X^2) - [E(X)]^2
  \]

  Let's calculate \(E(X^2)\):
  \[
  E(X^2) = \int_{a}^{b} x^2 f(x) dx = \int_{a}^{b} x^2 \frac{1}{b - a} dx = \frac{1}{b - a} \int_{a}^{b} x^2 dx
  \]

  The integral of \(x^2\) over \([a, b]\) is:
  \[
  \int_{a}^{b} x^2 dx = \left[ \frac{x^3}{3} \right]_{a}^{b} = \frac{b^3}{3} - \frac{a^3}{3}
  \]

  Thus,
  \[
  E(X^2) = \frac{1}{b - a} \left( \frac{b^3}{3} - \frac{a^3}{3} \right) = \frac{b^3 - a^3}{3(b - a)}
  \]

  Using the difference of cubes,
  \[
  b^3 - a^3 = (b - a)(b^2 + ab + a^2)
  \]

  Therefore,
  \[
  E(X^2) = \frac{(b - a)(b^2 + ab + a^2)}{3(b - a)} = \frac{b^2 + ab + a^2}{3}
  \]

  The variance is then:
  \[
  \text{Var}(X) = E(X^2) - [E(X)]^2 = \frac{b^2 + ab + a^2}{3} - \left(\frac{a + b}{2}\right)^2
  \]

  Simplifying,
  \[
  \left(\frac{a + b}{2}\right)^2 = \frac{a^2 + 2ab + b^2}{4}
  \]

  Therefore,
  \[
  \text{Var}(X) = \frac{b^2 + ab + a^2}{3} - \frac{a^2 + 2ab + b^2}{4}
  \]

  Combine the fractions:
  \[
  \text{Var}(X) = \frac{4(b^2 + ab + a^2) - 3(a^2 + 2ab + b^2)}{12}
  \]
  \[
  \text{Var}(X) = \frac{4b^2 + 4ab + 4a^2 - 3a^2 - 6ab - 3b^2}{12} = \frac{b^2 - 2ab + a^2}{12} = \frac{(b - a)^2}{12}
  \]

Thus, the variance of a continuous uniform distribution over \([a, b]\) is:
\[ \text{Var}(X) = \frac{(b - a)^2}{12} \]

### (ii) Mean and Variance of Y

Let \(X\) be a random variable with mean \( \mu \), variance \( \sigma^2 \), and \( 0 \leq X \leq 1 \).
Given \(X\), the independent random variables \(V\) and \(W\) are uniformly distributed over \([0, X]\) and \([X, 1]\) respectively.
Define \(Y = W - V\).

#### Mean of Y

Using the law of total expectation:
\[ E(Y) = E(E(Y|X)) \]

Given \(X\), the mean of \(V\) and \(W\) are:
\[ E(V|X) = \frac{X}{2} \]
\[ E(W|X) = \frac{X + 1}{2} \]

Therefore:
\[ E(Y|X) = E(W|X) - E(V|X) = \frac{X + 1}{2} - \frac{X}{2} = \frac{1}{2} \]

Thus,
\[ E(Y) = E\left(\frac{1}{2}\right) = \frac{1}{2} \]

#### Variance of Y

Using the law of total variance:
\[ \text{Var}(Y) = E(\text{Var}(Y|X)) + \text{Var}(E(Y|X)) \]

First, calculate the variance given \(X\):
\[ \text{Var}(V|X) = \frac{X^2}{12} \]
\[ \text{Var}(W|X) = \frac{(1 - X)^2}{12} \]

Since \(V\) and \(W\) are independent given \(X\):
\[ \text{Var}(Y|X) = \text{Var}(W|X) + \text{Var}(V|X) = \frac{(1 - X)^2}{12} + \frac{X^2}{12} = \frac{(1 - X)^2 + X^2}{12} = \frac{1 - 2X + X^2 + X^2}{12} = \frac{1 - 2X + 2X^2}{12} \]

Next, calculate the expectation:
\[ E(\text{Var}(Y|X)) = E\left(\frac{1 - 2X + 2X^2}{12}\right) = \frac{1}{12} - \frac{2}{12}E(X) + \frac{2}{12}E(X^2) \]

Given \(E(X) = \mu\) and \(\text{Var}(X) = \sigma^2 = E(X^2) - E(X)^2\):
\[ E(X^2) = \sigma^2 + \mu^2 \]

Therefore:
\[ E(\text{Var}(Y|X)) = \frac{1}{12} - \frac{2\mu}{12} + \frac{2(\sigma^2 + \mu^2)}{12} = \frac{1}{12} - \frac{2\mu}{12} + \frac{2\sigma^2}{12} + \frac{2\mu^2}{12} = \frac{1 + 2\sigma^2 - 2\mu + 2\mu^2}{12} \]

Finally, calculate the variance of the expectation:
\[ \text{Var}(E(Y|X)) = \text{Var}\left(\frac{1}{2}\right) = 0 \]

Thus,
\[ \text{Var}(Y) = E(\text{Var}(Y|X)) = \frac{1 + 2\sigma^2 - 2\mu + 2\mu^2}{12} \]

Let me know if you have any more questions!
########################################################################
4.
A sequence of Bernoulli trials is conducted, in which the probability of success is
p > 0. Let W denote the number of trials needed to obtain the first success. Write
down the distribution of W, and find its probability generating function (pgf). Hence
1
1 p
or otherwise show that its mean and variance are
and 2 respectively.
p
p
(10)
Deduce the mean and variance of the number of trials needed to obtain k successes,
where k 2.
(6)
By using pgfs or otherwise, find the probability that exactly n trials are required to
obtain k successes.
(4)
[You may quote standard properties of pgfs without proof.]
3
Turn over5.
(i)
Let U have a continuous uniform distribution over the interval [0, 1], and let
f (.) be a continuous function defined on that interval. Write
E ( f ( U ) ) and
2
1
Var ( f ( U ) ) . Show that
0
f ( u ) du and
2
1
0
( f ( u ) ) 2 du
2
.
(2)
(ii)
Let g ( U )
f ( U )
f (1 U )
. Deduce that
2
2
E ( g(U) ) = , and that Var ( g ( U ) )
2
,
where is the covariance of f ( U ) and f (1 U ) .
(4)
(iii)
Consider the particular case when f ( U )
(a)
1
. You are given that μ = log 2.
1 U
Evaluate τ.
(6)
(b)
Let {U i : i = 1, 2, ..., 2n} be independent, all having the same
2 n
n
f ( U i )
distribution as U.
Write
i 1
A n
2 n
g ( U i )
and
Show that E(A n ) = E(B n ) = log 2, and evaluate the ratio
significant figures, given that
2
1
2
B n
i 1
n
.
Var( B n )
to two
Var( A n )
(log 2) 2 .
(8)
4
Turn over6.
Let the random variable Z have a standard Normal distribution. You are given that its
moment generating function (mgf) is exp
(i)
1 2
t
2
, and that of Z 2 is (1 2 t )
1
2
for t
1
.
2
Use the mgf of Z 2 to find the mean and variance of Z 2 .
(4)
7.
(ii) Let Z 1 , Z 2 , Z 3 , ... be independent random variables, all having the standard
Z n 2 . Show that E(W n ) = n,
Normal distribution, and define W n Z 1 2 Z 2 2
W n n
Var(W n ) = 2n, and find g n (t), the mgf of X n
.
2 n
(8)
(iii) Show that log ( g n (t) ) converges to 12 t 2 as n
. Hence deduce the
approximate distribution of W n for large n, and thus, approximately, the
probability that W 50 exceeds 60.
(8)
(i) Let U have the continuous uniform distribution over the interval [0, 1]. Show
n log U has the exponential distribution with probability density
that X
exp( x / n )
function g ( x )
on x 0.
n
(4)
(ii) The density function of a random variable having a gamma distribution is
f ( x )
e x x n 1
given by f ( x )
on x 0 for integer n > 1. Show that
reaches
g ( x )
( n 1)!
its maximum when x = n, and that the maximum value is k
n n e 1 n
.
( n 1)!
(10)
[Hint: Take the log of
(iii)
f ( x )
.]
g ( x )
You have a supply of values from independent random variables U 1 , U 2 , U 3 , ...,
all having the same distribution as U in part (i). Describe how to use them, as
well as the results of parts (i) and (ii), in a rejection method to generate a
stream of values from a random variable having the density f (x) given in
part (ii).
(6)
5
Turn over8.
Let C be a circle with unit radius so that the length of the side of an inscribed
equilateral triangle is 3 . Three possible ways of describing a "random chord" are
suggested.
(i) Select two points independently and uniformly distributed on the
circumference, and join them.
(ii) Select a point P within the interior of the circle at random (i.e. uniformly
distributed), and join P to the centre of the circle along a radius. The chord is
the line through P perpendicular to this radius.
(iii) First select a radius at random (i.e. uniformly distributed over all directions),
then choose a point Q uniformly distributed along this radius. The chord is the
line through Q perpendicular to this radius.
In each case, find the probability that the length of the "random chord" exceeds
3 .
(8, 6, 6)
6
