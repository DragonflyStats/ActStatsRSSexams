EXAMINATIONS OF THE HONG KONG STATISTICAL SOCIETY
GRADUATE DIPLOMA, 2009
(Modular format)
MODULE 1 : Probability Distributions
Time Allowed: Three Hours
Candidates should answer FIVE questions.
All questions carry equal marks.
The number of marks allotted for each part-question is shown in brackets.
Graph paper and Official tables are provided.
Candidates may use calculators in accordance with the regulations published in
the Society's "Guide to Examinations" (document Ex1).
The notation log denotes logarithm to base e.
Logarithms to any other base are explicitly identified, e.g. log 10 .
⎛ n ⎞
Note also that ⎜ ⎜ ⎟ ⎟ is the same as n C r .
⎝ r ⎠
1
GD Module 1 2009
This examination paper consists of 5 printed pages, each printed on one side only.
This front cover is page 1.
Question 1 starts on page 2.
There are 8 questions altogether in the paper.
©RSS 20091.

%%%%%%%%%%%%%%%%%%%%%%%%%%%%%%%%%%%%%%%%%%%%%%%%%%%%%%%%%
(i)
The events { E i : i = 1, 2, ..., n } partition a sample space, and A is an event with
non-zero probability P ( A ) . Write down Bayes' Theorem for the conditional
probability P ( E 1 | A ) in terms of conditional probabilities P ( A | E i ) and
unconditional probabilities P ( E i ) , i = 1, 2, ..., n.
(4)
(ii)
There is a dispute as to whether customer Brian gave barmaid Anna a £10 note
or a £20 note to pay for his drink. Both are honest, but may make mistakes.
Consider the following three pieces of information.
(A) The till contains ten £10 notes and twenty £20 notes, all mixed up at
random.
(B) Anna correctly identifies notes 90% of the time, and states that Brian
used a £10 note.
(C) Brian, who correctly identifies £20 notes 80% of the time, and correctly
identifies £10 notes 70% of the time, claims to have paid using a £20
note.
Explaining your reasoning, estimate the probability that Brian used a £20 note,
if you are given each of the following.

%%%%%%%%%%%%%%%%%%%%%%%%%%%%%%%%%%%%%%%%%%%%%%%%%%%%%%%%%
2.
(a) Information (A) alone. (2)
(b) Both (A) and (B). (6)
(c) All of (A), (B) and (C). (8)
(i) Suppose A and B are independent events, and A , B denote the complementary
events to A, B. Show that A and B are independent; deduce that A and B
are independent.
(4)
(ii) The point (X, Y) has the uniform distribution over the unit square
{ 0 ≤ x ≤ 1, 0 ≤ y ≤ 1} , and u is a given value with 0 < u < 0.5. Events A and B
are defined as A :{ X ≥ 2 Y } and B :{ Y ≤ X ≤ Y + u } .
(a) Sketch three separate diagrams of the unit square, illustrating the events
A, B and A ∩ B respectively.
(4)
(b) Find the probabilities of these three events, and show that A and B are
independent if, and only if, u = 2/5.
(10)
(c) Deduce the value of P( A ∩ B ) when u = 2/5.
(2)
2
Turn over3.
Suppose X and Y are independent random variables having Poisson distributions with
respective means λ ( > 0) and μ ( > 0).
(i)
Show that X + Y also follows a Poisson distribution.
(5)

%%%%%%%%%%%%%%%%%%%%%%%%%%%%%%%%%%%%%%%%%%%%%%%%%%%%%%%%%
4.
(ii) Find P ( X = k | X + Y = n ) when k and n are integers with 0 ≤ k ≤ n . For
given fixed n > 0, name the distribution you have obtained.
(7)
(iii) Telephone calls arriving at a computer helpline are classed as urgent or
standard; urgent calls average 8 per hour, standard calls average 24 per hour.
Ten calls arrive within 30 minutes; find (to two significant figures) the
probability that at most two of them are urgent, stating any assumptions you
make.
(8)
##########################
(i) Consider a family of non-negative random variables with the property that the
variance of any member of the family is proportional to the square of its
expected value. Let X be a member of this family, with E(X) = μ . Use a
Taylor series expansion to show that, if Y = log(X), then the variance of Y is
approximately constant no matter which X is chosen from the original family
of random variables.
(8)
(ii) Show that, when X has the Gamma distribution Γ ( α , ν ) , i.e. its probability
ν α x α − 1 − ν x
Γ ( k + α )
e for x > 0 , then E ( X k ) = k
density function is f ( x ) =
when
Γ ( α )
ν Γ ( α )
k > 0 . Hence verify that, for fixed α > 0 , this family of random variables has
the property described in (i) as ν varies.
(12)
∞
[You may use without proof the results Γ ( α ) = ∫ e − t t α − 1 dt and Γ ( α + 1) = α Γ ( α ). ]
0
3

##################################
Turn over5.
The continuous random variables X and Y have joint probability density function
f ( x , y ) = kxy if 0 < x < y < 1, with f ( x , y ) = 0 elsewhere, where k is a constant.
(i) Evaluate k, and find the marginal probability densities of X and Y. Say, with a
reason, whether or not X and Y are independent.
(10)
(ii) Show that, for all non-negative integers r and s, E ( X r Y s ) =
8
.
( r + 2 )( r + s + 4 )
Hence find the correlation between X and Y.
(10)
%%%%%%%%%%%%%%%%%%%%%%%%%%%%%%%%%%%%%%%%%%%%%%%%%%%%%%%%%
Sure, let's tackle this step-by-step.

### Part (i)

We start by considering a family of non-negative random variables with the property that the variance of any member is proportional to the square of its expected value.

Let \( X \) be a member of this family, with \( \mathbb{E}(X) = \mu \) and \( \text{Var}(X) = k\mu^2 \) where \( k \) is a constant.

We need to show that if \( Y = \log(X) \), then \( \text{Var}(Y) \) is approximately constant.

Using the Taylor series expansion, for \( \log(X) \) around \( \mu \):
\[ \log(X) \approx \log(\mu) + \frac{X - \mu}{\mu} - \frac{(X - \mu)^2}{2\mu^2} + \cdots \]

Since \( Y = \log(X) \):
\[ \mathbb{E}(Y) \approx \log(\mu) \]
\[ \text{Var}(Y) \approx \mathbb{E}\left[\left(\frac{X - \mu}{\mu} - \frac{(X - \mu)^2}{2\mu^2}\right)^2\right] \]

Ignoring higher-order terms, we get:
\[ \text{Var}(Y) \approx \frac{1}{\mu^2}\text{Var}(X) \]

Using \( \text{Var}(X) = k\mu^2 \):
\[ \text{Var}(Y) \approx \frac{1}{\mu^2} \cdot k\mu^2 = k \]

This shows that the variance of \( Y \) is approximately constant, as \( k \) is a constant.

### Part (ii)

Given that \( X \) has the Gamma distribution \( \Gamma(\alpha, \nu) \), its probability density function is:
\[ f(x) = \frac{\nu^\alpha x^{\alpha-1} e^{-\nu x}}{\Gamma(\alpha)} \]

We need to show that:
\[ \mathbb{E}(X^k) = \frac{\Gamma(k + \alpha)}{\nu^k \Gamma(\alpha)} \text{ for } k > 0 \]

Using the property of the Gamma function:
\[ \Gamma(\alpha) = \int_0^\infty t^{\alpha-1} e^{-t} dt \]

For \( X \) following \( \Gamma(\alpha, \nu) \):
\[ \mathbb{E}(X^k) = \int_0^\infty x^k f(x) dx \]

Substitute \( f(x) \):
\[ \mathbb{E}(X^k) = \int_0^\infty x^k \frac{\nu^\alpha x^{\alpha-1} e^{-\nu x}}{\Gamma(\alpha)} dx \]

\[ \mathbb{E}(X^k) = \frac{\nu^\alpha}{\Gamma(\alpha)} \int_0^\infty x^{k+\alpha-1} e^{-\nu x} dx \]

Using the substitution \( t = \nu x \):
\[ \mathbb{E}(X^k) = \frac{\nu^\alpha}{\Gamma(\alpha)} \cdot \frac{1}{\nu^{k+\alpha}} \int_0^\infty t^{k+\alpha-1} e^{-t} dt \]

\[ \mathbb{E}(X^k) = \frac{1}{\nu^k \Gamma(\alpha)} \Gamma(k+\alpha) \]

Hence,
\[ \mathbb{E}(X^k) = \frac{\Gamma(k+\alpha)}{\nu^k \Gamma(\alpha)} \]

For fixed \( \alpha > 0 \), as \( \nu \) varies, we see that:
\[ \text{Var}(X) = \text{E}(X^2) - (\text{E}(X))^2 \]

Using \( \text{E}(X) = \frac{\alpha}{\nu} \) and \( \text{E}(X^2) = \frac{\Gamma(2 + \alpha)}{\nu^2 \Gamma(\alpha)} \):

\[ \text{Var}(X) = \frac{\alpha(\alpha+1)}{\nu^2} - \left(\frac{\alpha}{\nu}\right)^2 \]

\[ \text{Var}(X) = \frac{\alpha}{\nu^2} = \left(\frac{1}{\nu^2}\right)\left(\frac{\Gamma(2 + \alpha)}{\Gamma(\alpha)}\right) = \left(\frac{\Gamma(1 + \alpha)}{\Gamma(\alpha)}\right)^2 \left(\frac{1}{\nu^2}\right) \]

This shows that the family of random variables \( X \) has the property that the variance is proportional to the square of the expected value as \( \nu \) varies.

I hope this helps you understand the concepts better! If you have any more questions or need further clarification, feel free to ask.
###################################
6.
(i) State the Central Limit Theorem for sums of independent identically
distributed random variables.
(4)
(ii) Explain carefully what is meant by the phrase continuity correction when
using a continuous distribution to approximate the probability P(X ≤ k) when X
takes integer values only, including the value k.
(4)
(iii) Give a detailed argument involving the Central Limit Theorem that justifies
the use of a Normal distribution to give an approximation to a probability of
the form P ( a ≤ Y ≤ b ) in the following cases.
(a) Y has the binomial distribution with parameters n and p (0 < p < 1),
with n large.
(6)
(b) Y has a Poisson distribution with parameter λ , when λ is large.
(6)
[You may use without proof standard properties of binomial or Poisson random
variables, and need not mention any continuity correction in your answers to (iii).]
4
Turn over7.
(i)
Suppose X and Y are independent random variables, each following the chi-
squared distribution with four degrees of freedom; this distribution has
probability density function (pdf) we − w / 2 / 4 on w > 0 .
Define new random variables U = X / Y and V = Y. Obtain the joint pdf of U
and V, and hence show that U has pdf h ( u ) = 6 u /( 1 + u ) 4 for u > 0 .
(10)
[You may use without proof the result
(ii)
8.
∫
∞
0
t k e − t dt = k ! when k is a positive integer.]
Name the distribution followed by U. Using the density function, show that
E(U) = 2, and hence verify that E(X / Y) > E(X) / E(Y). Explain briefly why this
is no surprise.
(10)

############################################

It is desired to use the rejection technique to simulate values from the beta distribution
with density f ( x ) = 6 x ( 1 − x ) over 0 < x < 1.
(i) Sketch the graph of the function f (x ) and, on the same diagram, draw the
trapezium whose four vertices are A = (0, 0), B = (0.25, 1.5), C = (0.75, 1.5)
and D = (1, 0). Prove that the given density function is entirely contained
within this trapezium.
(6)
(ii) Given a method that generates points uniformly distributed inside the
trapezium ABCD, show that, in the long run, 1/9 of these points will be
rejected when the rejection technique is used.
(4)
(iii) Give the outcomes when the rejection technique is applied to the four points
(0.452, 0.697), (0.120, 0.066), (0.914, 0.871) and (0.657, 1.345). In each case,
state the reason for your answer.
(4)
(iv) Describe an efficient method (i.e. one that wastes no input) for using a stream
{y i : i = 1, 2, 3, ...} of independent random numbers, each uniformly distributed
over the interval (0, 1), to generate points uniformly distributed inside ABCD.
(6)
5

#############

Let's break this down step-by-step:

### Part (i)

First, we need to sketch the function \( f(x) = 6x(1-x) \) over \( 0 < x < 1 \) and draw the trapezium whose vertices are A = (0, 0), B = (0.25, 1.5), C = (0.75, 1.5), and D = (1, 0).

1. **Sketch \( f(x) = 6x(1-x) \)**:
   The function \( f(x) = 6x(1-x) \) is a quadratic function which opens downwards. It reaches its maximum value at \( x = 0.5 \).

2. **Trapezium ABCD**:
   - A = (0, 0)
   - B = (0.25, 1.5)
   - C = (0.75, 1.5)
   - D = (1, 0)

To prove that \( f(x) \) is entirely contained within this trapezium, observe the following:
- At \( x = 0 \) and \( x = 1 \), \( f(x) = 0 \) and these points lie on the lines AD and AB.
- At \( x = 0.25 \) and \( x = 0.75 \), the value of \( f(x) = 6(0.25)(1-0.25) = 6(0.75)(1-0.75) = 1.5 \).

Thus, the function \( f(x) \) is bounded above by the horizontal line at \( y = 1.5 \) and the vertices of the trapezium. The function lies entirely within the trapezium ABCD.

### Part (ii)

We need to show that 1/9 of the points generated uniformly inside the trapezium ABCD will be rejected when the rejection technique is used.

The area of the trapezium ABCD is:
\[ \text{Area} = \frac{1}{2} \times (\text{base}_1 + \text{base}_2) \times \text{height} \]
\[ \text{Area} = \frac{1}{2} \times (1 + 1) \times 1.5 = 1.5 \]

The area under the density function \( f(x) \) is:
\[ \int_0^1 6x(1-x) \, dx = 6 \left[ \frac{x^2}{2} - \frac{x^3}{3} \right]_0^1 = 6 \left( \frac{1}{2} - \frac{1}{3} \right) = 6 \times \frac{1}{6} = 1 \]

Thus, the probability of accepting a point is \( \frac{\text{Area under } f(x)}{\text{Area of trapezium}} = \frac{1}{1.5} = \frac{2}{3} \).

The proportion of points rejected is \( 1 - \frac{2}{3} = \frac{1}{3} \). Therefore, 1/9 of the points are rejected.

### Part (iii)

Applying the rejection technique to the four points:

1. \( (0.452, 0.697) \):
   - \( f(0.452) = 6 \times 0.452 \times (1 - 0.452) = 6 \times 0.452 \times 0.548 \approx 1.485 \).
   - Since 0.697 < 1.485, it is **accepted**.

2. \( (0.120, 0.066) \):
   - \( f(0.120) = 6 \times 0.120 \times (1 - 0.120) = 6 \times 0.120 \times 0.880 \approx 0.6336 \).
   - Since 0.066 < 0.6336, it is **accepted**.

3. \( (0.914, 0.871) \):
   - \( f(0.914) = 6 \times 0.914 \times (1 - 0.914) = 6 \times 0.914 \times 0.086 \approx 0.471 \).
   - Since 0.871 > 0.471, it is **rejected**.

4. \( (0.657, 1.345) \):
   - \( f(0.657) = 6 \times 0.657 \times (1 - 0.657) = 6 \times 0.657 \times 0.343 \approx 1.351 \).
   - Since 1.345 < 1.351, it is **accepted**.

### Part (iv)

An efficient method to generate points uniformly distributed inside ABCD is as follows:

1. Generate a random variable \( U \) uniformly distributed over \( (0,1) \).
2. If \( U \leq 0.25 \), set \( X = U \times 4 \times 1.5 \).
3. If \( 0.25 < U < 0.75 \), set \( X = 1.5 \times \left(\frac{U - 0.25}{0.5} \right) \).
4. If \( U \geq 0.75 \), set \( X = 1.5 \times (1 - (U - 0.75) \times 4) \).


##################
Certainly! Here's the R code to sketch the function \( f(x) = 6x(1-x) \) over \( 0 < x < 1 \) and draw the trapezium whose vertices are A = (0, 0), B = (0.25, 1.5), C = (0.75, 1.5), and D = (1, 0).

```R
# Define the function f(x)
f <- function(x) {
  6 * x * (1 - x)
}

# Define the x values
x <- seq(0, 1, length.out = 100)

# Calculate the y values
y <- f(x)

# Plot the function f(x)
plot(x, y, type = "l", col = "blue", lwd = 2, 
     xlab = "x", ylab = "f(x)", 
     main = "Function f(x) = 6x(1-x) and Trapezium ABCD",
     ylim = c(0, 2))

# Add the trapezium ABCD
polygon(c(0, 0.25, 0.75, 1), c(0, 1.5, 1.5, 0), border = "red", lwd = 2, col = rgb(1, 0, 0, 0.2))

# Add points A, B, C, D
points(c(0, 0.25, 0.75, 1), c(0, 1.5, 1.5, 0), pch = 19, col = "red")
text(c(0, 0.25, 0.75, 1), c(0.1, 1.6, 1.6, 0.1), labels = c("A", "B", "C", "D"), col = "red", pos = 4)
```

This code defines the function \( f(x) = 6x(1-x) \), plots it over the interval \( 0 < x < 1 \), and adds the trapezium with vertices A, B, C, and D. The trapezium is shown in red, and the function is shown in blue.

Feel free to try this code in R and visualize the function and the trapezium! If you have any other questions or need further assistance, I'm here to help.

This method ensures that points are uniformly distributed inside the trapezium ABCD.

I hope this helps clarify things! If you have more questions or need further assistance, feel free to ask!