EXAMINATIONS OF THE HONG KONG STATISTICAL SOCIETY
GRADUATE DIPLOMA, 2010
MODULE 1 : Probability Distributions
Time Allowed: Three Hours
Candidates should answer FIVE questions.
All questions carry equal marks.
The number of marks allotted for each part-question is shown in brackets.
Graph paper and Official tables are provided.
Candidates may use calculators in accordance with the regulations published in
the Society's "Guide to Examinations" (document Ex1).
The notation log denotes logarithm to base e.
Logarithms to any other base are explicitly identified, e.g. log 10 .
Note also that
( nr ) is the same as
n
C r .
1
GD Module 1 2010
This examination paper consists of 6 printed pages, each printed on one side only.
This front cover is page 1.
Question 1 starts on page 2.
There are 8 questions altogether in the paper.
© RSS 20101.
Let X be a random variable with probability mass function
P ( X = n ) =
K
,
n ( n + 2 )
n = 1, 2, 3, ...
for some constant K.
(i)
By noting that
1
1 ⎛ 1
1 ⎞
4
= ⎜ −
⎟ , or otherwise, show that K = .
n ( n + 2 ) 2 ⎝ n n + 2 ⎠
3
(6)
(ii)
Calculate the probability that the value of X is an odd number.
(6)
(iii)
Suppose Y is independent of X and has the same distribution as X. Find the
probability that
(a)
X + Y is even,
(4)
(b)
X is even, given that X + Y is even.
(4)
2
Turn over2.
[In this question, all random variables take non-negative integer values only, and you
may quote standard properties of probability generating functions (pgfs) without
proof.]
Let N be a random variable with pgf g(z), and let X 1 , X 2 , X 3 , ... be independent random
variables all having the same distribution whose pgf is f (z).
(i)
Define S N = X 1 + X 2 + ... + X N for N ≥ 1, and S 0 = 0. Show that the pgf of S N
is the composition g ( f ( z ) ) .
(6)
(ii) Suppose that N has the Poisson distribution with mean λ > 0, while each X i
takes only the values 0 and 1 with respective probabilities 1 – p and p
(0 < p < 1). Find the corresponding pgfs, and deduce the distribution of S N .
(7)
(iii) Suppose instead that, for m = 0, 1, 2, ..., p m denotes the probability that, in any
given road accident, m casualties will need hospital treatment. Assume that the
numbers of casualties in different accidents are independent, and that the
number of road accidents relevant to a particular hospital on a Sunday
afternoon has the Poisson distribution with mean 1. Show that the probability
that this hospital will have to treat at most two road accident casualties next
Sunday afternoon is
⎛
p 1 2 ⎞
⎜ 1 + p 1 + p 2 +
⎟ exp ( p 0 − 1 ) .
2 ⎠
⎝
(7)
3.
Suppose that X and Y are independent random variables with the same probability
density function (pdf) f (x). Write down, without proof, a formula for the pdf of X + Y.
(2)
Suppose that f (x) = x/2 for 0 < x < 2 (and f (x) = 0 elsewhere).
(i)
Find the pdf of W = X + Y for 0 < w < 2 and for 2 < w < 4.
(12)
(ii)
Find the pdf of V = (X – 1) 2 .
(6)
3
Turn over4.
A sequence of independent Bernoulli trials, in which the probability of success is p,
with 0 < p < 1, is carried out.
Let X denote the number of trials up to and including the first success. Find the
distribution of X and obtain its expected value and variance.
(8)
A further sequence of x trials is carried out, where x is the observed value of the
random variable X. Let Y denote the number of successes in these trials. Show that Y
has expected value 1 and find its variance.
(8)
[You may use the result Var(Y) = E(Var(Y | X)) + Var(E(Y | X)).]
Explain briefly why the result E(Y) = 1 should cause no surprise, and why it might be
anticipated that the variance of Y decreases as p increases.
(4)
5.
Suppose {X n } (n = 1, 2, 3, ...) are independent random variables, all having the same
continuous distribution. When observations are made sequentially, a record occurs
whenever a value exceeds all previous values (note that the first value is itself a
record). Let A n denote the event that X n is a record, and let S n denote the number of
records in the sequence {X 1 , X 2 , ..., X n }.
(i)
Show that the probability of A n is 1/n.
(2)
(ii)
Explain why the events A n and A m are independent when m ≠ n.
(4)
(iii)
Find the mean and variance of S n .
(6)
(iv)
Use the central limit theorem, with a continuity correction, to estimate the
probability of at least 10 records in the first 100 values of {X n }.
(8)
2
n
n
1
1 π
[You may use the results ∑ ≈ log n and ∑ 2 ≈
.]
6
r = 1 r
r = 1 r
4
Turn over6.
Suppose that U has the uniform distribution over the interval (0, 1).
Write V = exp(U 2 ), W = V + K(U – 0.5) where K is a constant, and θ = ∫ exp ( t 2 ) dt .
1
0
(i)
Show that E(W) = E(V) = θ .
(5)
(ii)
Show that the covariance of U and V is (e – 1 – θ )/2.
(5)
(iii)
Deduce the value of K that minimises the variance of W.
(5)
(iv)
7.
(i)
The mean of 1000 simulated values of V yields a preliminary estimate of θ as
approximately 1.46. Assume that you have an unlimited supply of values from
independent random variables all of which have the same distribution as U.
Describe an efficient simulation method, based on (i) and (iii), that would be
expected to lead to the estimation of the value of θ to a high degree of
precision.
(5)
Show that, if U has the uniform distribution on the interval (0, 1), then –2logU
has the exponential distribution with mean 2.
(5)
Suppose that X and Y are independent standard Normal random variables. Let (R, Θ )
be the corresponding polar coordinates, i.e.
R = X 2 + Y 2 and cos Θ = X / R , sin Θ = Y / R .
(ii)
Show that the joint density of ( R , Θ ) is g ( r , θ ) =
⎛ r 2 ⎞
r
exp ⎜ − ⎟ over 0 < r < ∞ ,
2 π
⎝ 2 ⎠
0 < θ < 2 π . Deduce the density function of R .
(10)
(iii) Hence or otherwise show that R 2 has the exponential distribution with mean 2.
(5)
5
Turn over8.
Let X denote the smallest and Y the largest of four independent random variables, each
having the continuous uniform distribution over the interval (0, θ ) where θ > 0.
(i)
For given 0 < x < y < θ , show that P ( X > x , Y < y )
( y − x )
=
θ 4
4
. Deduce the
joint density of X and Y .
(4)
(ii) Write U = Y – X and V = Y + X . Find the joint density of U and V . Sketch the
region over which this density is non-zero.
(6)
(iii) Deduce the marginal densities of U and V .
(4)
(iv)
You are given that E ( X ) = θ /5, Var( X ) = 2 θ 2 /75, E ( Y ) = 4 θ /5, Var( Y ) = 2 θ 2 /75,
Var( U ) = θ 2 /25 and Var( V ) = θ 2 /15. Find constants c 1 , c 2 , c 3 , c 4 such that
E ( c 1 X ) = E ( c 2 Y ) = E ( c 3 U ) = E ( c 4 V ) = θ . Which of the four quantities c 1 X , c 2 Y ,
c 3 U , c 4 V has the smallest variance?
(6)
6
