\documentclass[a4paper,12pt]{article}

%%%%%%%%%%%%%%%%%%%%%%%%%%%%%%%%%%%%%%%%%%%%%%%%%%%%%%%%%%%%%%%%%%%%%%%%%%%%%%%%%%%%%%%%%%%%%%%%%%%%%%%%%%%%%%%%%%%%%%%%%%%%%%%%%%%%%%%%%%%%%%%%%%%%%%%%%%%%%%%%%%%%%%%%%%%%%%%%%%%%%%%%%%%%%%%%%%%%%%%%%%%%%%%%%%%%%%%%%%%%%%%%%%%%%%%%%%%%%%%%%%%%%%%%%%%%

\usepackage{eurosym}
\usepackage{vmargin}
\usepackage{amsmath}
\usepackage{graphics}
\usepackage{epsfig}
\usepackage{enumerate}
\usepackage{multicol}
\usepackage{subfigure}
\usepackage{fancyhdr}
\usepackage{listings}
\usepackage{framed}
\usepackage{graphicx}
\usepackage{amsmath}
\usepackage{chngpage}

%\usepackage{bigints}
\usepackage{vmargin}

% left top textwidth textheight headheight

% headsep footheight footskip

\setmargins{2.0cm}{2.5cm}{16 cm}{22cm}{0.5cm}{0cm}{1cm}{1cm}

\renewcommand{\baselinestretch}{1.3}

\setcounter{MaxMatrixCols}{10}

\begin{document}
\begin{enumerate}
1
A bag contains 8 black and 6 white balls. Two balls are drawn out at random, one
after the other and without replacement.
Calculate the probabilities that:
2
(i) The second ball drawn out is black.
(ii) The first ball drawn out was white, given that the second ball drawn out is
black.
[2]
[Total 3]
Let A and B denote independent events.
Show that A and B , the complement of event B, are also independent events.
3
[1]
[3]
Consider 12 independent insurance policies, numbered 1, 2, 3, , 12, for each of
which a maximum of 1 claim can occur. For each policy, the probability of a claim
occurring is 0.1.
Find the probability that no claims arise on the group of policies numbered 1, 2, 3, 4,
5 and 6, and exactly 1 claim arises in total on the group of policies numbered 7, 8, 9,
10, 11, and 12.
[3]

%%%%%%%%%%%%%%%%%%%%%%%%%%%%%%%%%%%%%%%%%%%%%%%%%%%%%%%%%%%%%%%%%%%%%%%%%%%%%%%%%
1
(i)
P (second ball drawn is B ) = P (first ball drawn is B ) = 8/14 = 0.571
OR P (1 st B and 2 nd B ) + P (1 st W and 2 nd B )
= (8/14) × (7/13) + (6/14) × (8/13) = 8/14
(ii)
2
P (1 st W | 2 nd B ) = P (1 st W and 2 nd B )/ P (2 nd B ) = (6/14) × (8/13)/(8/14)
= 6/13 = 0.462
Since A and B are independent, P ( A ) = P ( A | B ) = P ( A | B )
(
Noting that ( B ) = B , it follows immediately that P ( A ) = P ( A | B ) = P A | ( B )
)
and so A and B are independent.
[OR
Since A and B are independent P (A ∩ B) = P(A)P(B).
Thus,
P(A ∩ B ) = P(A) − P(A ∩ B) = P(A) − P(A)P(B) = P(A){1 − P(B)} = P(A)P( B )
∴ A and B are independent.
3
P(no claims on 6 policies) = 0.5314 (from tables p186 — or using 0.9 6 )
P(1 claim on 6 policies) = 0.8857 − 0.5314 = 0.3543 (or using 6(0.1)(0.9 5 ))
So required probability = 0.5314 × 0.3543 = 0.188.
