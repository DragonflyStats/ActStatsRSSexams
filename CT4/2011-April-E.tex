\documentclass[a4paper,12pt]{article}

%%%%%%%%%%%%%%%%%%%%%%%%%%%%%%%%%%%%%%%%%%%%%%%%%%%%%%%%%%%%%%%%%%%%%%%%%%%%%%%%%%%%%%%%%%%%%%%%%%%%%%%%%%%%%%%%%%%%%%%%%%%%%%%%%%%%%%%%%%%%%%%%%%%%%%%%%%%%%%%%%%%%%%%%%%%%%%%%%%%%%%%%%%%%%%%%%%%%%%%%%%%%%%%%%%%%%%%%%%%%%%%%%%%%%%%%%%%%%%%%%%%%%%%%%%%%

\usepackage{eurosym}
\usepackage{vmargin}
\usepackage{amsmath}
\usepackage{graphics}
\usepackage{epsfig}
\usepackage{enumerate}
\usepackage{multicol}
\usepackage{subfigure}
\usepackage{fancyhdr}
\usepackage{listings}
\usepackage{framed}
\usepackage{graphicx}
\usepackage{amsmath}
\usepackage{chngpage}

%\usepackage{bigints}
\usepackage{vmargin}

% left top textwidth textheight headheight

% headsep footheight footskip

\setmargins{2.0cm}{2.5cm}{16 cm}{22cm}{0.5cm}{0cm}{1cm}{1cm}

\renewcommand{\baselinestretch}{1.3}

\setcounter{MaxMatrixCols}{10}

\begin{document}
\begin{enumerate}
CT4 A2011—410
At Miracle Cure hospital a pioneering new surgery was tested to replace human lungs with synthetic implants. Operations were carried out throughout June 2010. Patients who underwent the surgery were monitored daily until the end of August 2010, or until they died or left hospital if sooner. The results are shown below. Where no date is given, the patient was alive and still in hospital at the end of August.
(i)
Patient Date of surgery Date of leaving
observation Reason for
leaving
observation
A
B
C
D
E
F
G
H
I
J
K
L
M
N June 1
June 3
June 5
June 8
June 9
June 12
June 16
June 17
June 22
June 24
June 25
June 26
June 29
June 30 June 3
July 2 Died
Left Hospital
July 11 Died
June 21
Aug 12 Died
Left Hospital
June 29
Aug 20 Died
Died
Aug 6 Left Hospital
Explain whether each of the following types of censoring is present and for
those present explain where they occur:
•
•
•
right censoring
left censoring
informative censoring
(ii) Calculate the Kaplan-Meier estimate of the survival function for these patients, stating all assumptions that you make.
(iii) Sketch, on a suitably labelled graph, the Kaplan-Meier estimate of the survival function.
(iv) Estimate the probability that a patient will die within four weeks of surgery.

%%%%%%%%%%%%%%%%%%%%%%%%%%%%%%%%%%%%%%%%%%%%%%%%%%%%%%%%%%
11
An historian has investigated the force of mortality from tuberculosis in a particular town in a developed country in the 1860s using a sample of records from a cemetery. He wishes to test whether the underlying mortality from tuberculosis in the town is
the same as the national force of mortality from this cause of death, as reported in death registration data. The data are shown in the table below.

Deaths in
sample Central exposed to
risk in sample National force
of mortality
5–14
15–24
25–34
35–44
45–54
55–64
65–74
75–84 13
47
52
50
33
23
13
3 3,685
2,540
1,938
1,687
1,386
1,018
663
260 0.0051
0.0199
0.0309
0.0316
0.0286
0.0230
0.0202
0.0070
(i) Carry out an overall test of the null hypothesis that the underlying mortality
from tuberculosis in the town is the same as the national force of mortality,
and state your conclusion.
[6]
(ii) (a)
Identify two differences between the experience of the sample
and the national experience which the test you performed in (i)
might not detect.
(b)
Carry out a test for each of the differences in (ii)(a).
(iii)
12
Age-group
Comment on the results from all the tests carried out in (i) and (ii).
[7]
%%%%%%%%%%%%%%%%%%%%%%%%%%%%%%%%%%%%%%%%%%%%%%%%%%%%%%%%%%%%%%%%%%%%%%%%%%%%%%%%
Question 10
(i)
Right censoring is present
for those still alive and in hospital at the end of August
OR
for those who left hospital while still alive
Left censoring is not present
The censoring is likely to be informative, since those leaving hospital are likely to be in much better health than those who remain. (The idea of going home to die when you have had a lung transplant is a little tenuous.)
(ii)
The durations and outcomes are shown in the table below.
Patient Died/Censored Duration
A
G
J
B
E
M
H
K
N
L
I
F
D
C Died
Died
Died
Censored
Died
Censored
Censored
Died
Censored
Censored
Censored
Censored
Censored
Censored 2
5
5
29
32
38
56
56
62
66
70
80
84
87
Page 13Subject CT4 (Models Core Technical) — Examiners’ Report, April 2011
EITHER ALTERNATIVE 1
Assuming that at duration 56 the death occurred before the life was censored, the
Kaplan-Meier estimate is as follows:
t j n j d j c j
0
2
5
32
56
+1⁄2 14
14
13
10
8
+1⁄2 0
1
2
1
1
+1⁄2 0
0
1
1
7
+1⁄2
λ j =
d j
n j
0
1/14
2/13
1/10
1/8
The Kaplan-Meier estimate at duration t is given by the product of 1 −
d j
n j
over
durations up to and including t. Thus the Kaplan-Meier estimate of the survival
function is
^
t S ( t )
0≤ t < 2
2≤ t < 5
5≤ t < 32
32≤ t < 56
56≤ t < 92 1.0000
0.9286
0.7857
0.7071
0.6188
OR
OR
OR
OR
13/14
11/14
99/140
99/160
OR ALTERNATIVE 2
Assuming that at duration 56 the death occurred after the life was censored, the Kaplan-Meier estimate is as follows:

t j n j d j c j
0
2
5
32
56 14
14
13
10
7 0
1
2
1
1 0
0
1
2
6
λ j =
d j
n j
0
1/14
2/13
1/10
1/7

%%%%%%%%%%%%%%%%%%%%%%%%%%%%%%%%%%%%%%%%%%%%%%%%%%%%%%%%%%%%%%%%%%%%%%%%%%
The Kaplan-Meier estimate at duration t is given by the product of 1 −
d j
n j
over
durations up to and including t. Thus the Kaplan-Meier estimate of the survival
function is
^
t S ( t )
0≤ t < 2
2≤ t < 5
5≤ t < 32
32≤ t < 56
56≤ t < 92 1.0000
0.9286
0.7857
0.7071
0.6061
OR 13/14
OR 11/14
OR 99/140
OR 297/490
(iii)
(iv)
The probability of death within 4 weeks is 1 – S(28) = 0.2143.
In part (i) candidates could receive credit for saying that left censoring was present IF they gave a valid reason (which typically involved the imprecise measurement of the times of surgery or of events – the left censoring arising as a special case of interval censoring).
In part (ii) each error was only penalised once. Correct calculations which carried forward earlier errors were given full credit. However, candidates who did not list the durations they were using, but then presented incorrect estimates of the survival function, were more heavily penalised, as it was not clear how many errors they had made.
In part (ii) candidates who assume that the death at duration 56 takes place after the
censoring at the same duration (ALTERNATIVE 2) were required to state this assumption for full credit. For ALTERNATIVE 1, the assumption that the death at duration 56 takes place before the censoring does not need to be stated for full credit, as it is the convention when
calculating Kaplan-Meier estimates.
Page 15Subject CT4 (Models Core Technical) — Examiners’ Report, April 2011
In part (iii) the plotted function should be consistent with the answer to part (ii). If the answer to part (ii) was incorrect but the incorrect answer to part (ii) was correctly plotted in part (iii), full credit could be awarded to part (iii).
Question 11
(i)
The chi-squared test is a suitable overall test.
Let μ x be the force of mortality in age-group x in the sample.
Let μ x s be the force of mortality in age group x in the national population.
Let E x c be the central exposed to risk in the sample.
Then if z x =
E x c μ x − E x c μ s x
E x c μ s x
the test statistic is
∑ z x 2 ∼ χ 2 m ,
x
THEN EITHER
where m is the number of age groups, which in this case is 8.
The calculations are shown below.
Age-group Expected deaths z x z x 2
5–14
15–24
25–34
35–44
45–54
55–64
65–74
75–84 18.7935
50.5460
59.8842
53.3092
39.6396
23.4140
13.3926
1.8200 –1.3364
–0.4988
–1.0188
–0.4532
–1.0546
–0.0856
–0.1073
0.8747 1.7860
0.2488
1.0380
0.2054
1.1121
0.0073
0.0115
0.7651
Therefore the value of the test statistic is 5.1742.
The critical value of the chi-squared distribution at the 5% level of significance with 8
degrees of freedom is 15.51.
Since 5.1742 < 15.51 we do not reject the null hypothesis that the mortality rate from tuberculosis in the sample is the same as that in the national population.
%%%%%%%%%%%%%%%%%%%%%%%%%%%%%%%%%%%%%%%%%%%%%%%%%%%%%%%%%%%%%%%%%
OR
where m is the number of age groups, which in this case is 7, because we should combine age groups 65–74 and 75–84 as the expected number of deaths in age group 75–84 years is less than 5
The calculations are shown below.
Age-group Expected deaths z x z x 2
5–14
15–24
25–34
35–44
45–54
55–64
65–84 18.7935
50.5460
59.8842
53.3092
39.6396
23.4140
15.2126 –1.3364
–0.4988
–1.0188
–0.4532
–1.0546
–0.0856
0.2019 1.7860
0.2488
1.0380
0.2054
1.1121
0.0073
0.0408
Therefore the value of the test statistic is 4.438.
The critical value of the chi-squared distribution at the 5\% level of significance with 7
degrees of freedom is 14.07.
Since 4.438 < 14.07 we do not reject the null hypothesis that the mortality rate from tuberculosis in the sample is the same as that in the national population
(ii)
(a)
Small bias which is not great enough for the chi-squared test to detect.
EITHER
(b)
Signs test
Under the null hypothesis that the mortality rate from tuberculosis in the sample is the same as that in the national population,
the number of positive signs is distributed Binomial (m, 0.5), where m is the number of ages.
We have 1 positive sign.
The probability of 1 or fewer positive signs is given by
⎛ 8 ⎞ 8 ⎛ 8 ⎞ 8
⎜ ⎟ 0.5 + ⎜ ⎟ 0.5 = 0.0352 .
⎝ 0 ⎠
⎝ 1 ⎠
Page 17Subject CT4 (Models Core Technical) — Examiners’ Report, April 2011
OR (if only 7 age groups are being used)
⎛ 7 ⎞ 7 ⎛ 7 ⎞ 7
⎜ ⎟ 0.5 + ⎜ ⎟ 0.5 = 0.0625 .
⎝ 0 ⎠
⎝ 1 ⎠
We use a two-tailed test (since too few or too many positive signs would be a
problem)
so we reject the null hypothesis if the probability of 1 or fewer positive signs
is less than 0.025.
Since 0.0352 (or 0.0625) > 0.025
we do not reject the null hypothesis.
OR
(b)
Cumulative deviations test
Under the null hypothesis that the mortality rate from tuberculosis in the sample is the same as that in the national population
the test statistic
∑ ( E x c μ x − E x c μ s x )
x
∑ E x c μ s x
∼ Normal(0,1)
x
The calculations are shown in the table below
Age-group
5–14
15–24
25–34
35–44
45–54
55–64
65–74
75–84
Σ
E x c μ x − E x c μ s x E x c μ s x
–5.7935
–3.5460
–7.8842
–3.3092
–6.6396
–0.4140
–0.3926
1.1800 18.7935
50.5460
59.8842
53.3092
39.6396
23.4140
13.3926
1.8200
–26.7991 260.7991
So the value of the test statistic is
Page 18
− 26.7991
= 1.6595 .
260.7991Subject CT4 (Models Core Technical) — Examiners’ Report, April 2011
Using a 5% level of significance, we see that −1.96 < 1.6596 < 1.96.
We do not reject the null hypothesis.
(a) Individual ages at which there are unusually large differences between the sample and the national experience.
(b) Individual standardised deviations
Under the null hypothesis that the mortality rate from tuberculosis in the sample is the same as that in the national population
we would expect the individual deviations to be distributed Normal (0,1) and therefore only 1 in 20 z x s should have absolute magnitudes greater than
1.96
OR
none should lie outside the range (–3, +3)
OR
diagram showing split of deviations actual versus expected.
Since the largest deviation is less in absolute magnitude than 1.96 we do not reject the null hypothesis.
(a)
Sections of the data where there is appreciable bias, revealed by runs or clumps of signs of the same type.
EITHER
(b)
Grouping of signs test
Under the null hypothesis that the mortality rate from tuberculosis in the
sample is the same as that in the national population
G = Number of groups of positive zs = 1
m = number of deviations = 8 (or 7 if last two age groups combined)
n 1 = number of positive deviations = 1
n 2 = number of negative deviations = 7 (or 6 if last two age groups combined)
THEN EITHER
We want k* the largest k such that
⎛ n 1 − 1 ⎞⎛ n 2 + 1 ⎞
k ⎜
⎟⎜
⎟
⎝ t − 1 ⎠⎝ t ⎠
⎛ m ⎞
t = 1
⎜ ⎟
⎝ n 1 ⎠
∑
< 0.05
The test fails at the 5% level if G ≤ k *.
Page 19Subject CT4 (Models Core Technical) — Examiners’ Report, April 2011
In the table in the Gold Book a value for k * is not given,
OR
The table in the Gold Book shows that k * = 0,
so we are not able to reject the null hypothesis
OR
so there is no evidence of clumping.
OR
For t = 1
⎛ n 1 − 1 ⎞ ⎛ 0 ⎞
⎜
⎟ = ⎜ ⎟
⎝ t − 1 ⎠ ⎝ 0 ⎠
which is 1
So this test is automatically passed
OR
There is no evidence of clumping
OR
We cannot reject the null hypothesis.
OR
(b)
Serial correlations (lag 1)
The calculations are shown in the tables below.
EITHER USING SEPARATE MEANS FOR THE z x AND z x + 1
Age
group z x z x A = z x − z
5–14
15–24
25–34
35–44
45–54
55–64
65–74
75–84 –1.336
–0.499
–1.019
–0.453
–1.055
–0.086
–0.107 –0.499
–1.019
–0.453
–1.055
–0.086
–0.107
0.875 –0.686
0.152
–0.368
0.197
–0.404
0.565
0.543
z –0.651 –0.335
0.595/√(1.446*2.604) = 0.307
Page 20
B = z x + 1 − z
–0.164
–0.684
–0.118
–0.720
0.249
0.228
1.210
Sum
A 2 B 2
0.342
–0.155
0.167
–0.208
0.035
–0.061
0.475 0.470
0.023
0.136
0.039
0.163
0.319
0.295 0.027
0.468
0.014
0.518
0.062
0.052
1.463
0.595 1.446 2.604
ABSubject CT4 (Models Core Technical) — Examiners’ Report, April 2011
Test 0.307 (√8) = 0.868 against Normal (0,1), and, since
0.868 < 1.645, we do not reject the null hypothesis.
that the mortality rate from tuberculosis in the sample is the same as that in the
national population
OR USING THE FORMULA IN THE GOLD BOOK
Age
group z x z x A = z x − z
5–14
15–24
25–34
35–44
45–54
55–64
65–74
75–84 –1.336
–0.499
–1.019
–0.453
–1.055
–0.086
–0.107
0.875 –0.499
–1.019
–0.453
–1.055
–0.086
-0.107
0.875 –0.876
–0.039
–0.559
0.007
–0.595
0.374
0.353
1.335
z –0.460
B = z x + 1 − z
–0.039
–0.559
0.007
–0.595
0.374
0.353
1.335
Sum
A 2
AB
0.034
0.022
–0.004
–0.004
–0.223
0.132
0.471 0.767
0.002
0.312
0.000
0.354
0.140
0.125
1.782
0.428 3.481
1
(0.428)
7
= 0.141
1
(3.481)
8
Test 0.141 (√8) = 0.397 against Normal (0,1), and, since
0.397 < 1.645, we do not reject the null hypothesis.
that the mortality rate from tuberculosis in the sample is the same as that in the national population
(iii)
In none of the tests we have performed do we reject the null hypothesis.
Therefore it seems that the mortality from tuberculosis in the town is the same as the national force of mortality.
In part (ii) the null hypothesis should be stated somewhere for each test. It could be stated at
the beginning, or in the conclusion. As long as it is correctly stated somewhere, full credit was given. In part (iii), the comment should be consistent with the results of the tests performed in parts (i) and (ii) to gain credit.
Most candidates made a good attempt at part (i). Attempts at part (ii) were more varied. In particular, most candidates did not point out that the chi-squared test only fails to detect SMALL (but consistent) bias. If the bias is large and consistent, the chi-squared test will
detect it.
\end{document}
