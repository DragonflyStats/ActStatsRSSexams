\documentclass[a4paper,12pt]{article}

%%%%%%%%%%%%%%%%%%%%%%%%%%%%%%%%%%%%%%%%%%%%%%%%%%%%%%%%%%%%%%%%%%%%%%%%%%%%%%%%%%%%%%%%%%%%%%%%%%%%%%%%%%%%%%%%%%%%%%%%%%%%%%%%%%%%%%%%%%%%%%%%%%%%%%%%%%%%%%%%%%%%%%%%%%%%%%%%%%%%%%%%%%%%%%%%%%%%%%%%%%%%%%%%%%%%%%%%%%%%%%%%%%%%%%%%%%%%%%%%%%%%%%%%%%%%

\usepackage{eurosym}
\usepackage{vmargin}
\usepackage{amsmath}
\usepackage{graphics}
\usepackage{epsfig}
\usepackage{enumerate}
\usepackage{multicol}
\usepackage{subfigure}
\usepackage{fancyhdr}
\usepackage{listings}
\usepackage{framed}
\usepackage{graphicx}
\usepackage{amsmath}
\usepackage{chngpage}

%\usepackage{bigints}
\usepackage{vmargin}

% left top textwidth textheight headheight

% headsep footheight footskip

\setmargins{2.0cm}{2.5cm}{16 cm}{22cm}{0.5cm}{0cm}{1cm}{1cm}

\renewcommand{\baselinestretch}{1.3}

\setcounter{MaxMatrixCols}{10}

\begin{document}
\large
\begin{enumerate}[(a)]
\item 
Explain what is meant by a time-homogeneous Markov chain.

\item Consider the time-homogeneous two-state Markov chain with transition matrix:

\[
\begin{pmatrix}
1 - a & a \\
b & 1 - b \\
\end{pmatrix}
\]
\noindent Explain the range of values that a and b can take which result in this being a
valid Markov chain which is:
\begin{enumerate}[(i)]
\item  irreducible
\item periodic
\end{enumerate}
\end{enumerate}
%%%%%%%%%%%%%%%%%%%%

\newpage


\begin{itemize}
\item[(a)]
A Markov chain is a stochastic process with discrete states operating in discrete time in which the probabilities of moving from one state to another
are dependent only on the present state of the process.
\begin{itemize}
    \item If the transition probabilities are also independent of time.
\item Also: If the $n-$step transition probabilities are dependent only on the time lag, the
chain is said to be time-homogeneous.
\end{itemize}

\item[(b)]

\begin{enumerate}[(i)]
\item In this case the chain is irreducible if the transition probability out of each state is non-zero (or, equivalently, if it is possible to
reach the other state from both states)
\\\\
So this requires $0 < a \leq 1$ and $0 < b \leq 1$
\item 
The chain is only periodic if the chain must alternate between
the states.
So $a = 1$ and $b = 1$.
\end{enumerate}
\end{itemize}
\end{document}

