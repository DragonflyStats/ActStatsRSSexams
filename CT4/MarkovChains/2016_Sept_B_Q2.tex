\documentclass[a4paper,12pt]{article}

%%%%%%%%%%%%%%%%%%%%%%%%%%%%%%%%%%%%%%%%%%%%%%%%%%%%%%%%%%%%%%%%%%%%%%%%%%%%%%%%%%%%%%%%%%%%%%%%%%%%%%%%%%%%%%%%%%%%%%%%%%%%%%%%%%%%%%%%%%%%%%%%%%%%%%%%%%%%%%%%%%%%%%%%%%%%%%%%%%%%%%%%%%%%%%%%%%%%%%%%%%%%%%%%%%%%%%%%%%%%%%%%%%%%%%%%%%%%%%%%%%%%%%%%%%%%

\usepackage{eurosym}
\usepackage{vmargin}
\usepackage{amsmath}
\usepackage{graphics}
\usepackage{epsfig}
\usepackage{enumerate}
\usepackage{multicol}
\usepackage{subfigure}
\usepackage{fancyhdr}
\usepackage{listings}
\usepackage{framed}
\usepackage{graphicx}
\usepackage{amsmath}
\usepackage{chngpage}

%\usepackage{bigints}
\usepackage{vmargin}

% left top textwidth textheight headheight

% headsep footheight footskip

\setmargins{2.0cm}{2.5cm}{16 cm}{22cm}{0.5cm}{0cm}{1cm}{1cm}

\renewcommand{\baselinestretch}{1.3}

\setcounter{MaxMatrixCols}{10}

\begin{document}
%%%%%%%%%%%%%%%%%%%%%%%%%%%%%%%%%%%%%%%%%%%%%%5
\large
\noindent The diagrams below show three Markov chains, where arrows indicate a non-zero
transition probability.\\ 
State whether each of the chains is:
\begin{itemize}
	\item irreducible.
	\item periodic, giving the period.
\end{itemize}
\begin{framed}
	\noindent 	
A Markov chain is said to be irreducible if it is possible to get to any state from any state. 
\end{framed}

\begin{framed}
\noindent 	A state in a discrete-time Markov chain is periodic if the chain can return to the state only at multiples of some specific integer larger than 1
\end{framed}

\begin{framed}
\noindent A Markov chain is aperiodic if every state is aperiodic.
\end{framed}
\begin{figure}[h!]
\centering
\includegraphics[width=0.7\linewidth]{Plot1}
\end{figure}
\begin{figure}[h!]
	\centering
	\includegraphics[width=0.7\linewidth]{Plot2}
\end{figure}
\begin{figure}[h!]
	\centering
	\includegraphics[width=0.7\linewidth]{Plot3}
\end{figure}


\newpage
% %- Q2

\noindent \textbf{Markov Chain 1}

\begin{itemize}
\item Irreducible
\item Aperiodic

\end{itemize}



\noindent \textbf{Markov Chain 2}
\begin{itemize}
\item Irreducible
\item Periodic with period 2
\end{itemize}


\noindent \textbf{Markov Chain 3}
\begin{itemize}
	\item Reducible
	\item Aperiodic
\end{itemize}



% [Total 3]
% Many candidates did well on this question. The most common errors were to regard Markov Chain 1 as periodic (this is incorrect because return to any state is possible in 2 or 3 steps, and 2 and 3 have no common factor higher than 1), and to regard Markov Chain 2 as being periodic with period 4 rather than 2.

\end{document}
