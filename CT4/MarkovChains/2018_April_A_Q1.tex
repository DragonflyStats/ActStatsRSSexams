\documentclass[a4paper,12pt]{article}

%%%%%%%%%%%%%%%%%%%%%%%%%%%%%%%%%%%%%%%%%%%%%%%%%%%%%%%%%%%%%%%%%%%%%%%%%%%%%%%%%%%%%%%%%%%%%%%%%%%%%%%%%%%%%%%%%%%%%%%%%%%%%%%%%%%%%%%%%%%%%%%%%%%%%%%%%%%%%%%%%%%%%%%%%%%%%%%%%%%%%%%%%%%%%%%%%%%%%%%%%%%%%%%%%%%%%%%%%%%%%%%%%%%%%%%%%%%%%%%%%%%%%%%%%%%%

\usepackage{eurosym}
\usepackage{vmargin}
\usepackage{amsmath}
\usepackage{graphics}
\usepackage{epsfig}
\usepackage{enumerate}
\usepackage{multicol}
\usepackage{subfigure}
\usepackage{fancyhdr}
\usepackage{listings}
\usepackage{framed}
\usepackage{graphicx}
\usepackage{amsmath}
\usepackage{chngpage}

%\usepackage{bigints}
\usepackage{vmargin}

% left top textwidth textheight headheight

% headsep footheight footskip

\setmargins{2.0cm}{2.5cm}{16 cm}{22cm}{0.5cm}{0cm}{1cm}{1cm}

\renewcommand{\baselinestretch}{1.3}

\setcounter{MaxMatrixCols}{10}

\begin{document}
	%% Institute and Faculty of Actuaries
	%% Question 1
	\large
	\noindent A Markov Chain has the following transition matrix:
	
	
	$$\bordermatrix{ & A & B &C \cr
		A & 0  & 0.6 & 0.4 \cr
		B & 0.75  &  0 &  0.25\cr
		C &  0.5  &    0.5 &0 }$$
	
	\begin{enumerate}[(a)]
		\item 
		Draw a transition graph for this Markov Chain, including the transition rates.
		
		\item 
		Explain whether the Markov Chain is:
		\begin{enumerate}[(i)]
			\item irreducible
			\item periodic
		\end{enumerate}
	\end{enumerate}
	
	
	%%%%%%%%%%%%%%%%%%%%%%%%%
	%%IMAGE
	%%--- [Total for part (i): 2]
	
	\newpage
	\noindent \textbf{Part (b)}\\
	\begin{itemize}
		\item The chain is irreducible 
		because every state can eventually be reached from every other state. 
		\item The chain is aperiodic 
		because it can return to each state in multiples of 2 or 3, giving no overall
		period. 
	\end{itemize}
	
	% This straightforward question was well answered. In part (ii), a minority of candidates thought that the Markov chain was periodic
	% with period 2. Some candidates wrote that the Markov chain was aperiodic but gave vague or incorrect explanations.
\end{document}
