\documentclass[a4paper,12pt]{article}

%%%%%%%%%%%%%%%%%%%%%%%%%%%%%%%%%%%%%%%%%%%%%%%%%%%%%%%%%%%%%%%%%%%%%%%%%%%%%%%%%%%%%%%%%%%%%%%%%%%%%%%%%%%%%%%%%%%%%%%%%%%%%%%%%%%%%%%%%%%%%%%%%%%%%%%%%%%%%%%%%%%%%%%%%%%%%%%%%%%%%%%%%%%%%%%%%%%%%%%%%%%%%%%%%%%%%%%%%%%%%%%%%%%%%%%%%%%%%%%%%%%%%%%%%%%%

\usepackage{eurosym}
\usepackage{vmargin}
\usepackage{amsmath}
\usepackage{graphics}
\usepackage{epsfig}
\usepackage{enumerate}
\usepackage{multicol}
\usepackage{subfigure}
\usepackage{fancyhdr}
\usepackage{listings}
\usepackage{framed}
\usepackage{graphicx}
\usepackage{amsmath}
\usepackage{chngpage}

%\usepackage{bigints}
\usepackage{vmargin}

% left top textwidth textheight headheight

% headsep footheight footskip

\setmargins{2.0cm}{2.5cm}{16 cm}{22cm}{0.5cm}{0cm}{1cm}{1cm}

\renewcommand{\baselinestretch}{1.3}

\setcounter{MaxMatrixCols}{10}

\begin{document}9 A life insurance company is developing a new class of annuity business. It has
conducted a study of mortality among lives it believes represent this new business. It
wishes to graduate the data so that they are suitable for use in financial calculations.
It decides to use a standard table as a basis for graduation and the function:
s 0.01
\[\mux = \mux + ��
where x \mu\]
��
are the graduated rates and s
$\mu_x$ are the rates from the standard table.
The table below gives some results from the graduation.
Age x Crude rates Graduated rates Exposed to risk
%% ˆ x \mu x \mu
%% ��
\begin{center}
\begin{tabular}{|c|c|c|c|}
70 &  0.0167 &  0.022661 & 1,200 \\ \hline
71 &  0.0209 &  0.024783 & 1,194 \\ \hline
72 &  0.0236 &  0.027204 & 973 \\ \hline
73 &  0.0324 &  0.029956 & 956 \\ \hline
74 &  0.0362 &  0.033072 & 912 \\ \hline
75 &  0.0402 &  0.036587 & 845 \\ \hline
76 &  0.0561 &  0.040357 & 820 \\ \hline
77 &  0.0623 &  0.044962 & 369 \\ \hline
78 &  0.0552 &  0.049899 & 489 \\ \hline
79 &  0.0640 &  0.055390 & 500 \\ \hline
\end{tabular}
\end{center}

\begin{enumerate}[(a)]
\item (i) Carry out an overall test of the goodness-of-fit of this graduation to the crude
rates. 
\item (ii) List three defects of a graduation which the test you conducted in (i) may not
detect. 
\item (iii) Perform, for each of two of the defects listed in (ii), an additional test which
can detect the defect. 
\item (iv) Comment on the results of the tests carried out in parts (i) and (iii). 
\end{enumerate}
%%-- [Total 17]
%%-- END OF PAPER

%%%%%%%%%%%%%%%%%%%%%%%%%%%%%%%%%%%%%%%%%%%%%%%%%%%%%%%%%%%%%%%%%%%%%%5
9 (i) To test for overall goodness of fit we use the 2 test.
The null hypothesis is that the graduated rates are the same as the true
underlying rates applying to the new class of business.
The test statistic 2 2
x m
x
z   where m is the degrees of freedom.
The calculations are shown in the table below.
\begin{verbatim}
Age x Crude rates
Graduated
rates
Exposed
to risk
Observed
deaths
Expected
deaths zx 2
zx
70 0.0167 0.022661 1,200 20 27.193 1.379 1.903
71 0.0209 0.024783 1,194 25 29.591 0.844 0.712
72 0.0236 0.027204 973 23 26.469 0.674 0.455
73 0.0324 0.029956 956 31 28.638 0.441 0.195
74 0.0362 0.033072 912 33 30.162 0.517 0.267
75 0.0402 0.036587 845 34 30.916 0.555 0.308
76 0.0561 0.040357 820 46 33.093 2.244 5.034
77 0.0623 0.044962 369 23 16.591 1.573 2.476
78 0.0552 0.049899 489 27 24.401 0.526 0.277
79 0.0640 0.055390 500 32 27.695 0.818 0.669
\end{verbatim}
\begin{itemize}
\item The observed test statistic is 12.295
\item The number of age groups is 10, but we lose some degrees of freedom
because of the process of graduation
one for the parameter of the function linking the graduated rates to the
standard table rates, and at least one more for the choice of standard table, so
m = 7 or 8, say.
%%Subject CT4 (Models Core Technical) – September 2014 – Examiners’ Report
% Page 19
\item The critical value of the chi-squared distribution with 7 degrees of
freedom at the 5\% level is 14.07, and with 8 degrees of freedom is 15.51.
\item Since 12.295 < 14.07 (or 15.51)
we do not reject the null hypothesis at the 95\% confidence level. 
\end{itemize}
%%%%%%%%%%%%%%%%%%%%%%%%%%%%%%%%%%%%%%%%%%%%%%%%%%%%%%%%%%%%%%%%%%%%%%%%%%%%%
\newpage
\noindent \textbf{Part (b)}
(ii) 
\begin{itemize}
    \item There may be one or two large deviations at individual ages, the effect of
which are insufficient to raise the chi-squared value above the critical level.
\item Small but consistent bias across the whole of the age range.
\item The graduation might be the wrong shape, in that the graduated rates might be
higher than the crude rates in one part of the age range, and systematically
lower in another part of the age range.
\item This will lead to runs or clumps of
deviations of the same sign.
\end{itemize}
The rates may not progress smoothly from age to age.
[Only three of these were required for full credit] 
%%%%%%%%%%%%%%%%%%%%%%%%%%%%%%%%%%%%%%%%%%%%%%%%%%%%%%%%%%%%%%%%%%%%%%%%%%%%%
\newpage
\noindent \textbf{Part (c)}\\

(iii) Large deviations
For large deviations, use the Individual Standardised Deviations test.
The null hypothesis is the same as in part (i), that the graduated rates are the
true underlying rates for the new class of business.
We would expect the individual deviations to be distributed Normal (0,1)
and therefore only 1 in 20 zxs should have absolute magnitude greater than
1.96 (or none should be outside 3 to +3)
OR table showing split of deviations, actual versus expected
as below


\begin{center}
\begin{tabular}{ccccccc}
Range  & $–\infty,–2 $ & $  –2,–1 $ & $ –1,0$ & $ 0,1 $ & $1,2 $ & $2,+–\infty$ \\
Expected & 0.2 & 1.4 & 3.4 & 3.4 & 1.4 & 0.2 \\
Actual & 0 & 1 & 2 & 5 & 1 & 1\\
\end{tabular}
\end{center}

\begin{itemize}
    \item Looking at the zxs we see that the largest one is 2.24 and the next is 1.57.
\item This test is therefore inconclusive (1 deviation out of 10 ages is greater
than 1.96).
\item Small but consistent bias
\item For small but consistent bias use the Signs test or the Cumulative Deviations
test.
%%Subject CT4 (Models Core Technical) – September 2014 – Examiners’ Report
%%Page 20
\item The null hypothesis is the same as in part (i), that the graduated rates are the
true underlying rates for the new class of business.
\end{itemize}

%%%%%%%%%%%%%%%%%%%%%%%%%%%%%%%%%%%%%%%%%%%%%%%%%%%%%%%%%%%%%%%%%%%%%%%%%%%%%
\newpage
\noindent \textbf{EITHER SIGNS TEST}

Under the null hypothesis, the number of positive signs amongst the zx is
distributed Binomial (10, ½ )
We observe 7 positive signs.
The probability of observing 7 or more positive signs in 10 observations is
0.1719
OR
the probability of observing exactly 7 positive signs is 0.1172.
either of which implies that Pr[observing 6 or more] > 0.025 (a two-tailed
test),
so we have no evidence to reject the null hypothesis
OR CUMULATIVE DEVIATIONS TEST
Under the null hypothesis
the test statistic
(Observed deaths Expected deaths)
Expected deaths
x
x
 

~ Normal(0,1)
So, using the results in the table in the solution to part (a) the value of the test
statistic is
294 274.75 1.16
274.75


Since –1.96 < test statistic < +1.96
we have insufficient evidence to reject the null hypothesis.
Shape of graduation/runs or clumps
For the existence of runs or clumps of deviations of the same sign, we use the
Grouping of Signs test or the Serial Correlations Test
The null hypothesis is the same as before, that the graduated rates are the true
underlying rates for the new class of business.
%%Subject CT4 (Models Core Technical) – September 2014 – Examiners’ Report Page 21
EITHER GROUPING OF SIGNS TEST
\begin{itemize}
    \item G = Number of groups of positive deviations = 1
    \item m = number of deviations = 10
    \item n1 = number of positive deviations = 7
    \item n2 = number of negative deviations = 3
\end{itemize}

THEN EITHER
We want k* the largest k such that
1 2
1
1 1
1
1
0.05
n n
k
t t
m
t n
    
      
 
  
 
 
The test fails at the 5% level if G ≤ k*.
From the Gold Book k* = 1, so we reject the null hypothesis.
OR
For t = 1
1 2 1 6 1 4
1 and 4
1 0 1
n n
t t
         
                     1
10
and 120
7
m
n
   
     
   
So Pr[t = 1] if the null hypothesis is true is
4/120 = 0.0333, which is less than 5% so we reject the null hypothesis.
OR SERIAL CORRELATIONS TEST
The calculations are shown in the table below.
\begin{center}
\begin{tabular}{cccc|cccc}
Age & zx & zx+1 & A = zx – z'x & B = zx+1 – z'x+1& AB & A2&  B2 \\
70 &–1.379 &–0.844 &–1.708 &–1.417 & 2.420 & 2.917 & 2.008\\
71 &–0.844 &–0.674& –1.173 &–1.247 & 1.462 & 1.375 & 1.555\\
72 & –0.674 & 0.441 &–1.003 &–0.132 & 0.132 & 1.006 & 0.017\\
73 & 0.441 & 0.517 & 0.112 &–0.056& –0.006 & 0.013 & 0.003\\
74 & 0.517 & 0.555 & 0.188 &–0.018& –0.003 & 0.035 & 0.000\\
75 & 0.555 & 2.244 & 0.226 & 1.671 & 0.378 & 0.051 & 2.793\\
76 & 2.244 & 1.573 & 1.915 & 1.000 & 1.915& 3.668 & 1.000\\
77 & 1.573 & 0.526 & 1.244 –0.047 –0.058 & 1.548 & 0.002\\
78 & 0.526 & 0.818 & 0.194 & 0.245 & 0.048 & 0.039 & 0.060\\
\end{tabular}
\end{center}
%%-- z' & 0.329 & 0.573 Sums \\6.288 \\10.652 \\7.438\\



z' 0.329 0.573 Sums 6.288 10.652 7.438
6.288/√(10.652*7.438) = 0.706
Subject CT4 (Models Core Technical) – September 2014 – Examiners’ Report
Page 22
Test 0.706 (√10) = 2.232 against Normal (0,1), and, since
2.232 > +1.645, we have sufficient evidence to reject the null hypothesis (NB
one-sided test)
Smoothness of the graduated rates
To check the smoothness of the graduated rates we do the Third Differences
test.
\begin{verbatim}
Graduated rates First difference Second
difference
Third Difference
0.022661 0.002122 0.000299 0.000032
0.024783 0.002421 0.000331 0.000033
0.027204 0.002752 0.000364 0.000034
0.029956 0.003116 0.000399 –0.000144
0.033072 0.003515 0.000255 0.000580
0.036587 0.003770 0.000835 –0.000503
0.040357 0.004605 0.000332 0.000222
0.044962 0.004937 0.000554
0.049899 0.005491
0.055390
\end{verbatim}

\begin{itemize}
    \item These should be small in magnitude compared with the rates themselves and
progress regularly which does not seem to be the case here.
\item So we conclude that the graduated rates are not sufficiently smooth. 
\item (iv) Although the overall fit of the graduated rates to the crude rates is
acceptable,
\item EITHER
The result of the grouping of signs/serial correlation test suggests that the
graduated rates are the wrong shape: too high at younger ages and too low at
older ages.
\end{itemize}

OR
\begin{itemize}
\item The results of the individual standardised deviation test indicate that there might be an outlier at age 76
\item The graduation should be carried out again with either a different standard
table or a different link functions
\item It seems a multiplicative function such as s
x  ax  might be better than just
adding a constant to the s
x s.
\end{itemize}

%%-- Subject CT4 (Models Core Technical) – September 2014 – Examiners’ Report
Page 23
IF SMOOTHNESS TEST NOT DONE
The graduated rates should be sufficiently smooth to use for financial
calculations.
OR, IF SMOOTHNESS TEST DONE
The graduated rates do not seem to be especially smooth.
Because the shape of the graduated rates seems incorrect, the company
would be unwise to use these rates for financial calculations. 
%%-- [Total 17]

% In part (i) the null hypothesis was often vaguely expressed. Statements that the graduated rates are “a good fit”, “consistent with”, “similar to” or “representative of” the underlying rates were not given full credit. In part (iii) it was acceptable if candidates did not use whole numbers of actual deaths but obtained the actual deaths as the result of multiplying the (rounded) crude rates by the exposed to risk. In part (iii) the null hypothesis should be stated for each test, but candidates could obtain credit by a statement that the null hypotheses for each test are the same as than in part (i). To obtain credit, the comments in part (iv) had to reflect the tests actually carried out by the candidate. Many candidates scored highly on parts (i) and (ii). Performance on part (iii) was less convincing. In part (iv) few candidates made comments beyond noting the immediate implications of the test results (i.e. that although the overall fit is satisfactory the graduation failed the Grouping of Signs test or that there was “clumping of signs”).

%%-- END OF EXAMINERS’ REPORT
\end{document}
