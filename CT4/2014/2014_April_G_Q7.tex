\documentclass[a4paper,12pt]{article}

%%%%%%%%%%%%%%%%%%%%%%%%%%%%%%%%%%%%%%%%%%%%%%%%%%%%%%%%%%%%%%%%%%%%%%%%%%%%%%%%%%%%%%%%%%%%%%%%%%%%%%%%%%%%%%%%%%%%%%%%%%%%%%%%%%%%%%%%%%%%%%%%%%%%%%%%%%%%%%%%%%%%%%%%%%%%%%%%%%%%%%%%%%%%%%%%%%%%%%%%%%%%%%%%%%%%%%%%%%%%%%%%%%%%%%%%%%%%%%%%%%%%%%%%%%%%

\usepackage{eurosym}
\usepackage{vmargin}
\usepackage{amsmath}
\usepackage{graphics}
\usepackage{epsfig}
\usepackage{enumerate}
\usepackage{multicol}
\usepackage{subfigure}
\usepackage{fancyhdr}
\usepackage{listings}
\usepackage{framed}
\usepackage{graphicx}
\usepackage{amsmath}
\usepackage{chngpage}

%\usepackage{bigints}
\usepackage{vmargin}

% left top textwidth textheight headheight

% headsep footheight footskip

\setmargins{2.0cm}{2.5cm}{16 cm}{22cm}{0.5cm}{0cm}{1cm}{1cm}

\renewcommand{\baselinestretch}{1.3}

\setcounter{MaxMatrixCols}{10}

\begin{document}
\begin{enumerate}
\item %%-- Question 7
A team of medical researchers is interested in assessing the effect of a certain
condition on mortality. The condition is, in itself, non-fatal and is curable, but is
believed to increase the risk of death from heart disease. The team proposes to use a
model with four states: 
\begin{itemize}
\item[(1)] “Alive, without condition”, 
\item[(2)] “Alive, with condition”,
\item[(3)] “Dead from heart disease” 
\item[(4)] “Dead from other causes”.
\end{itemize}

\begin{enumerate}
\item (i)
Draw a diagram showing the possible transitions between the four states.

Let the transition intensity between state i and state j at time x+t be \mu ijx + t . Let the
probability that a person in state i at time x will be in state j at time x+t be t p ij x .
\item (ii)
Show, from first principles, that
d
22 24
( t p x 24 ) = t p x 21 \mu 14
x + t + t p x \mu x + t .
dt
[5]
An empirical investigation using data for persons aged between 60 and 70 years produces the following results:
Waiting time in state “Alive, without condition” is 2,046 person-years
Waiting time in state “Alive, with condition” is 1,139 person-years
\begin{itemize}
\item 10 deaths from heart disease to persons “Alive, without condition”
\item 30 deaths from other causes to persons “Alive, without condition”
\item 25 deaths from heart disease to persons “Alive, with condition”
\item 20 deaths from other causes to persons “Alive, with condition”
\end{itemize}

%%%%%%%%%%%%%%%%%%%%%%%%%%
\item (iii)
%%% CT4 A2014–6
Show that there is a statistically significant difference (at the 95% confidence
level) between the death rates from heart disease for persons with and without
the condition.

\end{enumerate}
%%%%%%%%%%%%%%%%%%%%%%%%%%%%%%%%%%%%%%%%%%%%%%%%%%%%%%%%%%%%%%%%%%%%%%%%%%%%%%%%%%%%%%%%%%%%
6
(i)
Suppose we observe n individuals (i = 1,...,n), for a period E and the number
of events observed to happen to individual i is d i .
Then the Poisson likelihood is:
( \mu E ) d i e ( −\mu E )
∏ d i ! .
i = 1
n
Taking logarithms of the likelihood we have:

n n n n
i = 1 i = 1 i = 1 i = 1
log e L = ∑ d i log \mu + ∑ d i log E − ∑ E \mu − ∑ log e ( d i !).

%%-- Page 7
%%-- Subject CT4 (Models Core Technical) 
%% – April 2014 – Examiners’ Report
Differentiating with respect to $\mu$ we obtain:
n
∂ (log e L )
=
∂\mu
∑ d i
i = 1
\mu
− nE .
n
Setting this to zero and solving gives \mu ˆ =
∑ d i
i = 1
nE
.
n
Since the second derivative −
(ii)
∑ d i
i = 1
\mu 2
is negative, we have a maximum.
The number of minutes exposed to risk for each student is given in the table
below.
Student Exposed to risk (minutes)
1
2
3
4
5
6
7
8
9
10 5
25
10
5
10
5
10
20
30
60
The total length of time exposed to risk is thus 180 minutes, or 3 hours.
During this time, 4 buses arrived.
The maximum likelihood estimate is thus 4/3 = 1.33 buses per hour.
(iii)
\begin{itemize}
\item The Poisson model normally assumes a fixed exposed-to-risk for each person, but in this investigation the waiting times vary with the student.
\item This is not a problem provided we can regard the students as identical, and we replace each student who catches a bus with an identical student at the moment the bus leaves.
\item But in this study the students were not replaced.
\item In the investigation above, there are gaps when no student was at the bus stop (e.g. between 4.50 and 4.55 p.m.). Buses may have arrived during these gaps.
\item 
Had we observed these buses, our estimate of the rate may have been different.
The assumption that the arrival of buses follows a Poisson process may not be
\item valid as arrival times may not be independent due to traffic conditions, and/or
they may not be random due to timetabling.
\item Answers to this question were very disappointing, with a substantial minority of candidates
making only token attempts. In part (i) many candidates wrote answers in terms of a general
Poisson parameter \lambda, rather than \mu E . This attracted a modest penalty. Full credit could be
n
obtained in part (i) for candidates who used the total number of deaths, ∑ d i , and noted that
i = 1
n
∑ d i
\end{itemize}
%%%%%%%%%%%%%%%%%%
the sum of n independent Poisson variables is Poisson, so the likelihood L = e − n \mu E ( n \mu E ) i = 1 .


% Many candidates obtained credit for proceeding to derive a maximum likelihood estimator
% where the expression for the likelihood was incorrect.
% A very common error in part (ii) was to assume that six buses had arrived rather than four,
% two buses having arrived at 4.35 p.m. and two at 4.50 p.m. This was penalised, as the
% question explicitly stated that “only one bus arrived at any given time”. Less common, but
% still present in too many scripts, was the serious error of only including exposure for students
% who caught the bus, and omitting the exposure for those who were censored. In part (iii)
% most candidates’ comments were restricted to the appropriateness of the assumption that the
% arrival of buses followed a Poisson process.

%%%%%%%%%%%%%%%%%%%%%%%%%%%%%%%%%%%%%%%%%%%%%%%%%5
\newpage
7
(i)
1. Alive, without
condition 2. Alive, with
condition
3. Dead from heart
disease 4. Dead from other
causes
[2]
(ii)
With the state numbers in the diagram above we can write:
EITHER
Using the Markov assumption
%% Page 9Subject CT4 (Models Core Technical) – April 2014 – Examiners’ Report
OR
The Chapman Kolmogorov equation is
dt + t 22
24
23
34
24
44
p x 24 = t p x 21 dt p 14
x + t + t p x dt p x + t + t p x dt p x + t + t p x dt p x + t .
But dt
p 34
x + t = 0
and dt
p x 44 + t = 1 .
So:
dt + t
22
24
24
p x 24 = t p x 21 dt p 14
x + t + t p x dt p x + t + t p x .
Assuming that, for small dt
dt
p ij x + t = \mu ij x + t dt + o ( dt )
o ( dt )
= 0 ,
dt → 0 dt
where lim
then substituting, we have
dt + t
22 24
24
p x 24 = t p x 21 \mu 14
x + t dt + t p x \mu x + t dt + t p x + o ( dt )
so that
dt + t
and hence
(iii)
22 24
p x 24 − t p x 24 = t p x 21 \mu 14
x + t dt + t p x \mu x + t dt + o ( dt )
p 24 − p 24
d
22 24
( t p x 24 ) = lim t + dt x t x = t p x 21 \mu 14
x + t + t p x \mu x + t .
dt → 0
dt
dt
\begin{itemize}
\item The maximum likelihood estimate (MLE) of the death rate from heart disease
25
for persons with the condition is
= 0.02195 .
1,139
\item The MLE of the death rate from heart disease for persons without the
10
condition is
= 0.00489.
2, 046

\item An estimate of the variance of the maximum likelihood estimator of the death
rate from heart disease for persons with the condition is
0.02195
= 0.000019271 .
1,139
% Page 10
% Subject CT4 (Models Core Technical) – April 2014 – Examiners’ Report
\item An estimate of the variance of the maximum likelihood estimator of the death
rate from heart disease for persons without the condition is
0.00489
= 0.00000239.
2, 046
\item The null hypothesis H 0 is that there is no difference between the means.
\item The variance of the difference between the two estimates is therefore:
0.000019271 + 0.00000239 = 0.000021659.
THEN EITHER

A 95% confidence interval around the difference is 
therefore:
\begin{eqnarray*}
(0.02195 – 0.00489) \pm 1.96\sqrt{0.000021659}
&=& 0.01706 \pm 1.96\times 0.004654 \\
&=& 0.01706 \pm 0.009122\\
&=& (0.007938, 0.026182)\\
\end{eqnarray*}
which does not include zero
OR
Under the null hypothesis the difference ~ Normal (0, 0.000021659).
Our observed value of the difference is 0.021949 - 0.004888 = 0.01706
\item A z-score for the actual difference of 0.01706 is therefore
(0.01706/√0.000021659) = 3.67
and since this is greater than 1.96 we reject the null hypothesis at the 95%
level
THEN
so the difference is statistically significantly different from zero

%% \item Most candidates scored highly on part (i), though attempts at part (ii) were more variable.
%% For part (iii) most candidates correctly computed the estimated rates and a smaller (though
%% substantial) number computed the correct variances. Few candidates, however, attempted a
%% formal test of the statistical significance of the difference.
%% In part (iii) some candidates calculated 95% confidence intervals around each estimate and
%% argued that since these do not overlap, the difference between the two estimates is
%% statistically significant. This approach will not always produce the same conclusion as
%% testing the difference directly (because √X + √Y ≠ √(X + Y). It was given partial credit.


%%--- Page 11
%%--- Subject CT4 (Models Core Technical) 
%%--– April 2014 – Examiners’ Report
\end{itemize}
\end{document}
