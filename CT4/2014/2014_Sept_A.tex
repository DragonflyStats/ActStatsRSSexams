\documentclass[a4paper,12pt]{article}

%%%%%%%%%%%%%%%%%%%%%%%%%%%%%%%%%%%%%%%%%%%%%%%%%%%%%%%%%%%%%%%%%%%%%%%%%%%%%%%%%%%%%%%%%%%%%%%%%%%%%%%%%%%%%%%%%%%%%%%%%%%%%%%%%%%%%%%%%%%%%%%%%%%%%%%%%%%%%%%%%%%%%%%%%%%%%%%%%%%%%%%%%%%%%%%%%%%%%%%%%%%%%%%%%%%%%%%%%%%%%%%%%%%%%%%%%%%%%%%%%%%%%%%%%%%%

\usepackage{eurosym}
\usepackage{vmargin}
\usepackage{amsmath}
\usepackage{graphics}
\usepackage{epsfig}
\usepackage{enumerate}
\usepackage{multicol}
\usepackage{subfigure}
\usepackage{fancyhdr}
\usepackage{listings}
\usepackage{framed}
\usepackage{graphicx}
\usepackage{amsmath}
\usepackage{chngpage}

%\usepackage{bigints}
\usepackage{vmargin}

% left top textwidth textheight headheight

% headsep footheight footskip

\setmargins{2.0cm}{2.5cm}{16 cm}{22cm}{0.5cm}{0cm}{1cm}{1cm}

\renewcommand{\baselinestretch}{1.3}

\setcounter{MaxMatrixCols}{10}

\begin{document}
\begin{enumerate}
\item 
1 For each of the following processes:
\begin{itemize}
\item  counting process
\item simple random walk
\item compound Poisson process
\item  Markov jump process
\end{itemize}
%%%%%%%%%%%%%%%%%%%%%%%%%%%%%%%%%%%%
\begin{enumerate}[(i)]
\item State whether the state space is discrete, continuous or can be either. 
\item State whether the time set is discrete, continuous, or can be either. 
\end{enumerate}
%%%%%%%%%%%%%%%%%%%%%%%%%5
%% Question 2 
\item 
\begin{enumerate}[(i)](i) List eight factors which should be considered when assessing whether a model is suitable for a particular application. 
\item (ii) State, giving reasons, a factor which would be particularly important in each of the following applications:
\begin{itemize}
    \item  Calculating the pension contribution for a medium sized pension scheme.
\item Helping a friend construct a business case to secure a loan from a bank for his new ice-cream van venture.
\item Working out how much it will cost to buy each member of your team their favourite cake on your birthday in six months’ time.
\end{itemize}
\end{enumerate}

\end{enumerate}

%%%%%%%%%%%%%%%%%%%%%%%%%%%%%%%%%%%%%%%%%%%%%%%%%%%%%%%%%%
1 State Space Time Space
Counting process Discrete Can be either
Simple random walk Discrete Discrete
Compound Poisson Can be either Continuous
Markov jump process Discrete Continuous
%%-- This question was well answered, with an average mark of more than 3 out of 4. [4]
%%%%%%%%%%%%%%%%%%%%%%%%%%%%%%%%%%%%%%%%%%%%%%%%%%%%%%%%%%%%
2 (i) The objectives of the modelling exercise.
\begin{itemize}
    \item The validity of the model for the purpose to which it is to be put.
    \item The validity of the data to be used.
    \item The validity of assumptions used.
    \item The possible errors associated with the model or parameters used not being a perfect representation of the real world situation being modelled.
    \item The impact of correlations between the random variables that “drive” the model.
    \item The extent of correlations between the various results produced from the
model.
\end{itemize}

The current relevance of models written and used in the past.
The credibility of the data input.
The credibility of the results output.
The dangers of spurious accuracy.
Cost of buying or constructing, and of running the model.
Ease of use and availability of suitable staff to use it.
Risk of model being used incorrectly or with wrong inputs.
The ease with which the model and its results can be communicated.
Compliance with the relevant regulations.
Clear documentation. 
%%--- Subject CT4 (Models Core Technical) – September 2014 – Examiners’ Report
%%--- Page 4
(ii) Pension scheme for medium-sized client
Validity of data/assumptions. Compliance with legislation.
It is a financially significant figure which you cannot afford to be way off the mark and is likely to make a big difference to the company making the contributions, so accurate data and calculations are important and compliance with legislation essential.
Business case for a bank loan
Ease of communication.
You must explain it to your friend who in turn must explain it to the bank manager.
Cake list
Dangers of spurious accuracy.
The sum of money concerned is so small anything which is time-consuming or expensive is a waste. 
\end{document}
%%-- [Total 7]
%%-- In part (i) not all the points listed here were required for full credit. In part (ii) the 
%%-- suggestions given here are just examples. Factors other than those listed here were given
%%-- credit if sensible justifications were given. Most candidates scored reasonably well on part
%%-- (i) of this question but answers to part (ii) were very disappointing, with many candidates
%%-- appearing to treat part (ii) as completely unrelated to part (i). There is a tendency for
%%-- candidates to learn by rote lists such as those required for part (i) of this question, without
%%-- really thinking about the application of the lists to practical problems.

\end{document}
