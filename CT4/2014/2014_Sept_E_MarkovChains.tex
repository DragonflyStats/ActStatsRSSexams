\documentclass[a4paper,12pt]{article}

%%%%%%%%%%%%%%%%%%%%%%%%%%%%%%%%%%%%%%%%%%%%%%%%%%%%%%%%%%%%%%%%%%%%%%%%%%%%%%%%%%%%%%%%%%%%%%%%%%%%%%%%%%%%%%%%%%%%%%%%%%%%%%%%%%%%%%%%%%%%%%%%%%%%%%%%%%%%%%%%%%%%%%%%%%%%%%%%%%%%%%%%%%%%%%%%%%%%%%%%%%%%%%%%%%%%%%%%%%%%%%%%%%%%%%%%%%%%%%%%%%%%%%%%%%%%

\usepackage{eurosym}
\usepackage{vmargin}
\usepackage{amsmath}
\usepackage{graphics}
\usepackage{epsfig}
\usepackage{enumerate}
\usepackage{multicol}
\usepackage{subfigure}
\usepackage{fancyhdr}
\usepackage{listings}
\usepackage{framed}
\usepackage{graphicx}
\usepackage{amsmath}
\usepackage{chngpage}

%\usepackage{bigints}
\usepackage{vmargin}

% left top textwidth textheight headheight

% headsep footheight footskip

\setmargins{2.0cm}{2.5cm}{16 cm}{22cm}{0.5cm}{0cm}{1cm}{1cm}

\renewcommand{\baselinestretch}{1.3}

\setcounter{MaxMatrixCols}{10}

\begin{document}
%%CT4 S2014–5 PLEASE TURN OVER
%% Question 6 
A motor insurance company offers annually renewable policies. To encourage
policyholders to renew each year it offers a No Claims Discount system which
reduces the premiums for those people who claim less often. There are four levels of
premium:
\begin{itemize}
\item   0: no discount
\item 1: 15% discount
\item 2: 25% discount
\item 3: 40% discount
\end{itemize}

A policyholder who does not make a claim in the year, moves up one level of
discount the following year (or stays at the maximum level).
A policyholder who makes one or more claims in a year moves down one level of
discount if they did not claim in the previous year (or remains at the lowest level) but
if they made at least one claim in the previous year they move down two levels of
discount (subject to not going below the lowest level).
\begin{enumerate}
\item (i) (a) Explain how many states are required to model this as a Markov chain.
(b) Draw the transition graph of the process. \\
\smallskip


The probability, p, of making at least one claim in any year is constant and
independent of whether a claim was made in the previous year.
\item (ii) Calculate the proportion of policyholders who are at the 25\% discount level in
the long run given that the proportion at the 40\% level is nine times that at the
15\% level. 
\item (iii) (a) Explain how the state space of the process would change if the
probability of making a claim in any one year depended upon whether
a claim was made in the previous year.
\item (b) Write down the transition matrix for this new process. 
\end{enumerate}
\newpage

%%%%%%%%%%%%%%%%%%%%%%%%%%%%%%%%%%%%%%%%%%%%%

Stationary distribution  satisfies  = P
1  p(12 4) (1)
2 (1 p)1 p3 (2)
3 (1 p)2 (3)
4  p5 (4)
5 (1 p)(3 4 5) (5)
Also 1  2  3  4  5 1 (6)
Working in terms of 2
(3), (4) and (5) give 5  (1 p)(3  4  5)
p5 (1 p)3 (1 p)4
2
p5  (1 p) 2  (1 p) p5
2 2
p 5  (1 p) 2
2
5 2 2
(1 p)
p

  



%%%%%%%%%%%%%%%%%%%%%%%%%%%%%%%%%%%%%%%%%%%%
%%  Subject CT4 (Models Core Technical) – September 2014 – Examiners’ Report
Page 12
Substituting p = 0.25 gives 2  0.071006
2
3 4 2
0.75 0.75
0.25
        
 
 32
Thus the proportion of people at the 25\% discount level is 0.213018. 
%%%%%%%%%%%%%%%%%%%%%%%%%%%%%%%%%%%%%%%%%%
(iii) (a) 
\begin{itemize}
\item Six states are now required
because the probability of a person in discount level 1 moving to
discount level 2 depends upon whether a claim was made the previous
year or not.
\item Hence discount level 1 must be split into
1+ = no claim made previous year and
1 = claim made previous year
\item (b) Let the probability of a claim in any year if there was a claim in the
previous year be r(epeat) and the probability of a claim in any year if
there was not a claim in the previous year be n(ew), then the new
transition matrix is
1 0 0 0 0
0 0 1 0 0
0 0 1 0 0
0 0 0 0 1
0 0 0 0 1
0 0 0 0 1
r r
n n
r r
n n
r r
n n
  
    
  
 
  
  
    
where the levels are ordered 0, 1+, 1, 2+, 2, 3. [4]
\end{itemize}
%%[Total 13]
%Most candidates worked out that five states were required, identified the correct state spaceband drew the correct diagram. Many also produced a correct matrix in part (ii) but few were able to solve the equations. Attempts at part (iii) were patchy, and only a minority of candidates attempted to write down the expanded matrix in part (iii)(b).

\end{document}
