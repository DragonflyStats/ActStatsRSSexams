\documentclass[a4paper,12pt]{article}

%%%%%%%%%%%%%%%%%%%%%%%%%%%%%%%%%%%%%%%%%%%%%%%%%%%%%%%%%%%%%%%%%%%%%%%%%%%%%%%%%%%%%%%%%%%%%%%%%%%%%%%%%%%%%%%%%%%%%%%%%%%%%%%%%%%%%%%%%%%%%%%%%%%%%%%%%%%%%%%%%%%%%%%%%%%%%%%%%%%%%%%%%%%%%%%%%%%%%%%%%%%%%%%%%%%%%%%%%%%%%%%%%%%%%%%%%%%%%%%%%%%%%%%%%%%%

\usepackage{eurosym}
\usepackage{vmargin}
\usepackage{amsmath}
\usepackage{graphics}
\usepackage{epsfig}
\usepackage{enumerate}
\usepackage{multicol}
\usepackage{subfigure}
\usepackage{fancyhdr}
\usepackage{listings}
\usepackage{framed}
\usepackage{graphicx}
\usepackage{amsmath}
\usepackage{chngpage}

%\usepackage{bigints}
\usepackage{vmargin}

% left top textwidth textheight headheight

% headsep footheight footskip

\setmargins{2.0cm}{2.5cm}{16 cm}{22cm}{0.5cm}{0cm}{1cm}{1cm}

\renewcommand{\baselinestretch}{1.3}

\setcounter{MaxMatrixCols}{10}

\begin{document}
\begin{enumerate}
9
(i)
(a) State three features which are desirable when a graduation is
performed.
(b) Explain why they are desirable.

The actuary to a large pension scheme has attempted to graduate the scheme’s recent
mortality experience with reference to a table used for similar sized schemes in a
different industry. He has calculated the standardised deviations between the crude
and the graduated rates, z x , at each age and has sent you a printout of the figures over
a small range of ages. Unfortunately the dot matrix printer on which he printed the
results was very old and the dots which would form the minus sign in front of
numbers no longer function, so you cannot tell which of the standardised deviations is
positive and which negative. Below are the data which you have.
(ii)
(iii)
(iv)
CT4 A2014–8
Age Standardised
deviation
60
61
62
63
64
65
66
67
68
69
70 2.40
0.08
0.80
0.76
1.04
0.77
1.30
1.76
0.28
0.68
0.93
(a) Carry out an overall goodness-of-fit test on the data.
(b) Comment on your result.
(a) List four defects of a graduation which the test you have carried out
would fail to detect.
(b) Suggest, for each of the defects, a test which could be used to detect it.

Carry out one of the tests suggested in part (iii)(b).


[Total 15]

9
(i)
Smoothness,
EITHER because we are likely to want to use the data for financial
calculations, and clients expect these to progress smoothly.
OR because we believe the underlying quantities vary smoothly with age.
Adherence to data
because we want the graduated rates to reflect as closely as possible the
experience on which they are based.
Suitability for the purpose to hand
EITHER In life insurance work, losses result from premature deaths
(benefits are paid sooner than expected) so we must not underestimate
mortality,
OR In annuity work, losses result from delayed deaths
(benefits are paid for longer than expected) so we must not overestimate
mortality.
(ii)
(a)
To test for overall goodness of fit we use the \chi^2 test.
The null hypothesis is that the graduated rates are not significantly
different from the underlying rates in the new experiences.
2
The test statistic ∑ z x 2 ≈ χ m
where m is the degrees of freedom.
x
Page 14%%%%%%%%%%%%%%%%%%%%%%%%%%%%%%% – April 2014 – Examiners’ Report
The calculations are below:
Age
deviation Standardised
deviation Standardised
deviation (squared)
60
61
62
63
64
65
66
67
68
69
70 2.40
0.08
0.80
0.76
1.04
0.77
1.30
1.76
0.28
0.68
0.93 5.760
0.006
0.640
0.578
1.082
0.593
1.690
3.098
0.078
0.462
0.865
Sum 14.852
The observed value of the test statistic is 14.852.
We compare this with the critical value of the \chi^2 distribution with
degrees of freedom equal to the number of ages minus at least 1 for the
choice of standard table.
With 8 degrees of freedom the critical value at the 95\% significance
level is 15.51 [with 9 d.f. it is 16.92, and with 10 d.f. 18.31].
Since 14.852 < 15.51 [or 16.92]
we do not reject the null hypothesis.
(b)
(iii)
Overall, the graduated rates seem to represent the underlying rates in
the new experience.
There may be individual ages at which the graduated rates and the observed
rates differ substantially, but if these ages are a small proportion of the whole
the chi-squared test may not detect them.
Use Individual Standardised Deviations Test.
There may be a consistent but small bias in one direction, but the deviations
are not large enough to allow detection by the chi-squared test.
Use Signs Test or Cumulative Deviations Test.
There may be bias over sections of the age range.
Use Grouping of Signs Test or Serial Correlations test, or Cumulative.
Page 15%%%%%%%%%%%%%%%%%%%%%%%%%%%%%%% – April 2014 – Examiners’ Report
Deviations Test over sections of the age range (chosen independently of the
pattern of deviations).
The graduation may not be sufficiently smooth if the linking function is
complex.
Use Third Differences Test.
(iv)
We can carry out a modified version of the Individual Standardised
Deviations Test.
Under the null hypothesis we expect the individual standardised deviations to
have a Normal (0,1) distribution.
Only 1 in 20 of the z x s should lie above 1.96 in absolute value
OR
None should lie above 3 in absolute value.
OR
Range 0,1 1,2 2,3
Expected
Actual 7.5
7 3.1
3 0.4
1
We have one deviation with an absolute value of 2.4.
OR
We have no deviations above 3 in absolute value.
OR
The distribution of the deviations is close to that we might expect under a
Normal (0,1) distribution
therefore we have no strong reason to reject the null hypothesis
In part (i), most candidates mentioned smoothness and adherence to data, but fewer
mentioned suitability for the purpose to hand. Most candidates scored well on part (ii), the
most widespread error being a failure to deduct degrees of freedom for the choice of
standard table. Answers to part (iii) were sometimes vague. Few candidates realised that
the chi-squared test only fails to detect small bias (it will detect large bias), and few
mentioned smoothness. Some candidates used the terms “overgraduation” and
“undergraduation” incorrectly to refer to situations where the graduated rates are
systematically biased above and below the true rates respectively.
Page 16%%%%%%%%%%%%%%%%%%%%%%%%%%%%%%% – April 2014 – Examiners’ Report
Most candidates correctly identified the individual standardised deviations test as the only
test which could be carried out in part (iv).
\end{document}
