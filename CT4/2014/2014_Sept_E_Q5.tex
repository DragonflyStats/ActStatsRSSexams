\documentclass[a4paper,12pt]{article}

%%%%%%%%%%%%%%%%%%%%%%%%%%%%%%%%%%%%%%%%%%%%%%%%%%%%%%%%%%%%%%%%%%%%%%%%%%%%%%%%%%%%%%%%%%%%%%%%%%%%%%%%%%%%%%%%%%%%%%%%%%%%%%%%%%%%%%%%%%%%%%%%%%%%%%%%%%%%%%%%%%%%%%%%%%%%%%%%%%%%%%%%%%%%%%%%%%%%%%%%%%%%%%%%%%%%%%%%%%%%%%%%%%%%%%%%%%%%%%%%%%%%%%%%%%%%

\usepackage{eurosym}
\usepackage{vmargin}
\usepackage{amsmath}
\usepackage{graphics}
\usepackage{epsfig}
\usepackage{enumerate}
\usepackage{multicol}
\usepackage{subfigure}
\usepackage{fancyhdr}
\usepackage{listings}
\usepackage{framed}
\usepackage{graphicx}
\usepackage{amsmath}
\usepackage{chngpage}

%\usepackage{bigints}
\usepackage{vmargin}

% left top textwidth textheight headheight

% headsep footheight footskip

\setmargins{2.0cm}{2.5cm}{16 cm}{22cm}{0.5cm}{0cm}{1cm}{1cm}

\renewcommand{\baselinestretch}{1.3}

\setcounter{MaxMatrixCols}{10}

\begin{document}
[Total 10]
CT4 S2014–4
5 A sports league has two divisions {1,2} with Division 1 being the higher. 
\begin{itemize}
    \item Each
season the bottom team in Division 1 is relegated to Division 2, and the top team in
Division 2 is promoted to Division 1.
\item Analysis of the movements of teams between divisions indicates that the probabilities
of finishing top or bottom of a division differs if a team has just been promoted or
relegated, compared with the probabilities in subsequent seasons.
\end{itemize}
%%%%%%%%%%%%%%%%%%%%%%%%%%%%%%%%%%%%%%%%%%%%%%%%%%%%%%%%%%%%%%%%%%
The probabilities are as follows:
  Finishing
position
If promoted
previous season
If relegated
previous season
If neither promoted
nor relegated
previous season
Top 0.1 0.25 0.15
Bottom 0.3 0.25 0.15
Other 0.6 0.5 0.7
\begin{itemize}
\item (i) Write down the minimum number of states required to model this as a Markov
chain. 
\item (ii) Draw a transition graph for the Markov chain. 
\item (iii) Write down the transition matrix for the Markov chain. 
\item (iv) Explain whether the Markov chain is:
  (a) irreducible.
(b) aperiodic. 
Team A has just been promoted to Division 1.
\item (v) Calculate the minimum number of seasons before there is at least a 60%
probability of Team A having been relegated to Division 2. 
\end{itemize}
%% [Total 11]


%%%%%%%%%%%%%%%%%%%%%%%%%%%%%%%%%%%%%%%
% Subject CT4 (Models Core Technical) – September 2014 – Examiners’ Report
% Page 12
Substituting p = 0.25 gives 2  0.071006
2
3 4 2
0.75 0.75
0.25
        
 
 32

\begin{itemize}
    \item 
Thus the proportion of people at the 25\% discount level is 0.213018. 
\item (iii) (a) Six states are now required
because the probability of a person in discount level 1 moving to discount level 2 depends upon whether a claim was made the previous ear or not.
\item Hence discount level 1 must be split into
1+ = no claim made previous year and
1 = claim made previous year
%%%%%%%%%%%%%%%%%%%%%%%%%%%%%%%%%%%%%%%%%%%%%%%%%%%%%%%
\item (b) Let the probability of a claim in any year if there was a claim in the previous year be r(epeat) and the probability of a claim in any year if there was not a claim in the previous year be n(ew), then the new transition matrix is
1 0 0 0 0
0 0 1 0 0
0 0 1 0 0
0 0 0 0 1
0 0 0 0 1
0 0 0 0 1
r r
n n
r r
n n
r r
n n
  
    
  
 
  
  
    
where the levels are ordered 0, 1+, 1, 2+, 2, 3. 
[Total 13]
\end{itemize}


%%Most candidates worked out that five states were required, identified the correct state space and drew the correct diagram. Many also produced a correct matrix in part (ii) but few were able to solve the equations. Attempts at part (iii) were patchy, and only a minority of candidates attempted to write down the expanded matrix in part (iii)(b).

%Subject CT4 (Models Core Technical) – September 2014 – Examiners’ Report

\end{document}
