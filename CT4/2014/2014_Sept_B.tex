\documentclass[a4paper,12pt]{article}

%%%%%%%%%%%%%%%%%%%%%%%%%%%%%%%%%%%%%%%%%%%%%%%%%%%%%%%%%%%%%%%%%%%%%%%%%%%%%%%%%%%%%%%%%%%%%%%%%%%%%%%%%%%%%%%%%%%%%%%%%%%%%%%%%%%%%%%%%%%%%%%%%%%%%%%%%%%%%%%%%%%%%%%%%%%%%%%%%%%%%%%%%%%%%%%%%%%%%%%%%%%%%%%%%%%%%%%%%%%%%%%%%%%%%%%%%%%%%%%%%%%%%%%%%%%%

\usepackage{eurosym}
\usepackage{vmargin}
\usepackage{amsmath}
\usepackage{graphics}
\usepackage{epsfig}
\usepackage{enumerate}
\usepackage{multicol}
\usepackage{subfigure}
\usepackage{fancyhdr}
\usepackage{listings}
\usepackage{framed}
\usepackage{graphicx}
\usepackage{amsmath}
\usepackage{chngpage}

%\usepackage{bigints}
\usepackage{vmargin}

% left top textwidth textheight headheight

% headsep footheight footskip

\setmargins{2.0cm}{2.5cm}{16 cm}{22cm}{0.5cm}{0cm}{1cm}{1cm}

\renewcommand{\baselinestretch}{1.3}

\setcounter{MaxMatrixCols}{10}

\begin{document}
\begin{enumerate}


[Total 7]
3 
\begin{enumerate}
\item (i) Explain the census approximation for calculating the exposed to risk between
any two census dates. 
A mortality investigation bureau has collected the following information on number
of policies in-force each year from different companies.
\begin{verbatim}
  Age and year Year Company A Company B Company C
Age 54 2011 3,400 1,250 5,780
2012 3,350 1,450 5,500
2013 3,000 1,500 6,010
Age 55 2011 3,250 1,190 6,000
2012 3,390 1,300 5,960
2013 3,100 1,440 6,030
Age 56 2011 3,270 1,150 5,950
2012 3,020 1,300 5,980
2013 2,950 1,500 5,990  
\end{verbatim}

• Company A has provided in-force policy data as at the beginning of each calendar
year using age nearest birthday.
CT4 S2014–3 PLEASE TURN OVER
• Company B has provided in-force policy data as at the financial year closing date
(which was 31 March in each year) using age last birthday.
• Company C has provided in-force policy data as at the end of each calendar year
using age next birthday.
\item (ii) Calculate the contribution to central exposed to risk for lives aged 55 last
birthday for the calendar year 2012 for each of the companies. 
\end{enumerate}
%%%%%%%%%%%%%%%%%%%%%%%%%%%%%%%%%%%%%%%%%%%%%%5
\newpage
3 (i) In survival investigations, population counts will only be available at census
dates.
Define Px,t to be the number of lives under observation, aged x last birthday, at
any time t and suppose that we have the values of Px,t only if t is a census date.
We require the exposed to risk, c
Ex , over the interval between the first census
and the last.
This is
2
1
,
t
c
x x s
t
E   P ds , where t1 and t2 are the two census dates.
To evaluate this, we usually assume that Px,s is linear between census dates.
%%Subject CT4 (Models Core Technical) – September 2014 – Examiners’ Report
%%Page 5
If the censuses are one year apart this leads to the trapezium approximation:
  , 1 , 2
1( )
2
c
Ex  Px t  Px t . [(ii) 
\begin{itemize}
    \item Company A
Assume birthdays evenly distributed across calendar years.
Age 55 last = 0.5 * age 55 nearest + 0.5 * age 56 nearest.
The required exposed to risk is
1 (3,390 3,100 3,020 2,950) 3,115.
4
   
\item Company B
Data are based on age last birthday so no age adjustment needed.
Assuming population varies linearly between census dates, then
population on 1 January 2012 = 31,300 11,190 1,272.5
4 4
 
population on 1 January 2013 = 31,440 11,300 1,405
4 4
 
The required exposed to risk is then
3 1 (1,272.5 1,300) 9 1 1,300 1,405 1,335.9375
12 2 12 2
   
\item Company C
Age 55 last birthday is equivalent to age 56 next birthday.
Assume that data for 31 December in year t apply to 1 January in
year t + 1.
The required exposed to risk is then
1 (5,950 5,980) 5,965.
2
  [6]
[Total 8]
In part (i) equations were not required, explanations and diagrams are acceptable instead.
\end{itemize}
%% Candidates’ answers to this question varied considerably. Partial %% credit was given in part (ii) for a range of alternative 
%% approximations. A common error in part (ii) was to use the wrong %% age when averaging (in particular supposing that age 55 last 
%% could be obtained by averaging age 54 nearest and age 55 nearest).

%% - Subject CT4 (Models Core Technical) 
%% – September 2014 – Examiners’ Report
%% - Page 6
\end{document}
