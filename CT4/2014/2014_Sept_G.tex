\documentclass[a4paper,12pt]{article}

%%%%%%%%%%%%%%%%%%%%%%%%%%%%%%%%%%%%%%%%%%%%%%%%%%%%%%%%%%%%%%%%%%%%%%%%%%%%%%%%%%%%%%%%%%%%%%%%%%%%%%%%%%%%%%%%%%%%%%%%%%%%%%%%%%%%%%%%%%%%%%%%%%%%%%%%%%%%%%%%%%%%%%%%%%%%%%%%%%%%%%%%%%%%%%%%%%%%%%%%%%%%%%%%%%%%%%%%%%%%%%%%%%%%%%%%%%%%%%%%%%%%%%%%%%%%

\usepackage{eurosym}
\usepackage{vmargin}
\usepackage{amsmath}
\usepackage{graphics}
\usepackage{epsfig}
\usepackage{enumerate}
\usepackage{multicol}
\usepackage{subfigure}
\usepackage{fancyhdr}
\usepackage{listings}
\usepackage{framed}
\usepackage{graphicx}
\usepackage{amsmath}
\usepackage{chngpage}

%\usepackage{bigints}
\usepackage{vmargin}

% left top textwidth textheight headheight

% headsep footheight footskip

\setmargins{2.0cm}{2.5cm}{16 cm}{22cm}{0.5cm}{0cm}{1cm}{1cm}

\renewcommand{\baselinestretch}{1.3}

\setcounter{MaxMatrixCols}{10}

\begin{document}
8 An investigation was undertaken into the length of post-operative stay in hospital after
a particular type of surgery. 
\begin{itemize}
    \item All patients undergoing this surgery between 1 January
and 31 January 2013 were observed until either they left the hospital, died, or
underwent a second operation. 
\item The event of interest was leaving the hospital.
\item Patients who died or underwent a second operation during the period of investigation
were treated as censored at the date of death or second operation respectively. 
\item The
investigation ended on 28 February 2013, and patients who were still in the hospital at
that time were treated as censored.
\end{itemize}

\begin{enumerate}
    \item 

(i) State, with reasons, whether the following types of censoring are present in
this investigation:
\begin{itemize}
\item right
\item Type I
\item Type II
\item random
\end{itemize}


\item (ii) Comment on whether censoring in this investigation is likely to be
informative. 
The following data relate to 11 patients included in the investigation.

\begin{center}
\begin{tabular}{c|c|c} \hline
Date of operation & Date observation &  Reason that \\
& ended & observation Ended \\ \hline \hline
2 January &  30 January &  Second operation \\ \hline
5 January &  7 January &  Died \\ \hline
10 January &  24 January &  Left hospital \\ \hline
12 January &  12 February &  Left hospital \\ \hline
15 January &  29 January &  Left hospital \\ \hline
20 January &  4 February &  Left hospital \\ \hline
20 January &  21 January &  Died \\ \hline
23 January &  28 February &  End of investigation  \\ \hline
24 January &  31 January &  Second operation \\ \hline
27 January &  20 February &  Left hospital \\ \hline
31 January &  14 February &  Left hospital \\ \hline
\end{tabular}
\end{center}

\end{enumerate}

\begin{itemize}
\ite[(iii)] Calculate the Kaplan-Meier estimate of the survivor function for remaining in
the hospital.
\item[(iv)] Sketch the Kaplan-Meier estimate of the survivor function, labelling the axes.

\item[(v)] Comment on the results of the investigation. 
\end{itemize}
%% [Total 16]
%% CT4 S2014–8

\newpage
%%%%
  8 (i) Right censoring
Yes, of patients not experiencing the event of interest before 28 February
either because they died, or because they had a second operation, or because
they remained in the hospital until 28 February, each of which outcomes cut
short observations in progress.
Type I censoring
Yes, of those patients remaining in hospital on 28 February, since this date
was fixed in advance of the investigation.
%% Subject CT4 (Models Core Technical) – September 2014 – Examiners’ Report
% Page 16
Type II censoring
No, as the end of the investigation was determined by time, not by the number
of patients who had left hospital.
Random censoring
Yes, of patients who died or who had a second operation, the times of which
were not known in advance of the investigation and can be considered as
random variables. [4]
(ii) Censoring is likely to be informative.
Those patients who died or who underwent a second operation were
probably recovering less well than patients who left hospital.
Had they not died or undergone a second operation, they would probably
have remained in hospital for longer than those patients who were not
censored. 
(iii) We re-write the data as follows:
  Patient Duration (days) Experienced the event (1)
or censored (0)
1 28 0
2 2 0
3 14 1
4 31 1
5 14 1
6 15 1
7 1 0
8 36 0
9 7 0
10 24 1
11 14 1
The Kaplan-Meier estimate uses the table below:
  tj Nj dj cj dj/Nj j
j
d
1
N

0 11 0 3 0 0
14 8 3 0 0.375 0.625
15 5 1 0 0.2 0.8
24 4 1 1 0.25 0.75
31 2 1 1 0.5 0.5
%%Subject CT4 (Models Core Technical) – September 2014 – Examiners’ Report
%%Page 17
The Kaplan-Meier estimate is then given by
ˆ( ) 1 . j
j
t t
j
d
S t
 N
 
    
 
This produces
\begin{verbatim}
    t S(ˆt)
0 \leq t < 14 1.0000
14 \leq t < 15 0.6250
15 \leq t < 24 0.5000
24 \leq t < 31 0.3750
31 \leq t < 36 0.1875 [6]
\end{verbatim}

(iv) See the sketch below.
\newpage
(v) Deaths occur soon after the operation.
There is a high hazard of leaving the hospital after 14 days.
It may be that clinical protocols regard 14 days as the minimum period for
which patients who have had this operation should remain in hospital, no
matter how well they seem to be recovering.
The fact that censoring is informative is likely to bias the estimate.
The results may not be credible or may have a large variance because the
sample size is very small.
0
0.1
0.2
0.3
0.4
0.5
0.6
0.7
0.8
0.9
1
0 5 10 15 20 25 30 35 40
S(t)
Duration t

The data only allow us to make estimates of “survival” up to a duration of 36
days.

%% Most candidates correctly identified and described the types of censoring present in this investigation, although most candidates also thought that the censoring was likely to be informative. Explanations of why this was the case were often vague and woolly. Part (iii) was reasonably well answered. Full credit was given in part (iv) for plots which were correct given the answer to part (iii). Answers to part (v) were disappointing, but good candidates noticed the tendency for patients to be discharged from hospital after 14 days, and for deaths to occur soon after surgery.
\end{document}
