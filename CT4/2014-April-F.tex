\documentclass[a4paper,12pt]{article}

%%%%%%%%%%%%%%%%%%%%%%%%%%%%%%%%%%%%%%%%%%%%%%%%%%%%%%%%%%%%%%%%%%%%%%%%%%%%%%%%%%%%%%%%%%%%%%%%%%%%%%%%%%%%%%%%%%%%%%%%%%%%%%%%%%%%%%%%%%%%%%%%%%%%%%%%%%%%%%%%%%%%%%%%%%%%%%%%%%%%%%%%%%%%%%%%%%%%%%%%%%%%%%%%%%%%%%%%%%%%%%%%%%%%%%%%%%%%%%%%%%%%%%%%%%%%

\usepackage{eurosym}
\usepackage{vmargin}
\usepackage{amsmath}
\usepackage{graphics}
\usepackage{epsfig}
\usepackage{enumerate}
\usepackage{multicol}
\usepackage{subfigure}
\usepackage{fancyhdr}
\usepackage{listings}
\usepackage{framed}
\usepackage{graphicx}
\usepackage{amsmath}
\usepackage{chngpage}

%\usepackage{bigints}
\usepackage{vmargin}

% left top textwidth textheight headheight

% headsep footheight footskip

\setmargins{2.0cm}{2.5cm}{16 cm}{22cm}{0.5cm}{0cm}{1cm}{1cm}

\renewcommand{\baselinestretch}{1.3}

\setcounter{MaxMatrixCols}{10}

\begin{document}
\begin{enumerate}
10
An industrial kiln is used to produce batches of tiles and is run with a standard firing cycle. After each firing cycle is finished, a maintenance inspection is undertaken on the heating element which rates it as being in Excellent, Good or Poor condition, or
notes that the element has Failed.
The probabilities of the heating element being in each condition at the end of a cycle,
based on the condition at the start of the cycle are as follows:
START
END
Excellent
Excellent
Good
Poor
Failed
0.5
Good Poor Failed
0.2
0.5 0.2
0.3
0.5 0.1
0.2
0.5
1
(i) Write down the name of the stochastic process which describes the condition of a single heating element over time.

(ii) Explain whether the process describing the condition of a single heating
element is:
(a)
(b)
(iii)
irreducible.
periodic.

Derive the probability that the condition of a single heating element is assessed as being in Poor condition at the inspection after two cycles, if the heating element is currently in Excellent condition.

If the heating element fails during the firing cycle, the entire batch of tiles in the kiln is wasted at a cost of £1,000. Additionally a new heating element needs to be installed at a cost of £50 which will, of course, be in Excellent condition.
(iv) Write down the transition matrix for the condition of the heating element in the kiln at the start of each cycle, allowing for replacement of failed heating elements.
(v) Calculate the long term probabilities for the condition of the heating element in the kiln at the start of a cycle.

The kiln is fired 100 times per year.
(vi)
Calculate the expected annual cost incurred due to failures of heating
elements.
[2]
The company is concerned about the cost of ruined tiles and decides to change its policy to replace the heating element if it is rated as in Poor condition.
(vii)
Evaluate the impact of the change in replacement policy on the profitability of
the company.
%%%%%%%%%%%%%%%%%%%%%%%%%%%%%%%%%%%%%%%%%%%%%%%%%%%%%%%%%%%%%%%%%%%%%%%%%%%%%%%


10
(i) Markov chain.
(ii) (a)
It is not irreducible
because a heating element cannot move to a state of being in better
condition.
(b)
It is not periodic
because it can remain in each state (or any other suitable reason).
(iii)
EITHER
The second order transition matrix is:
0.25
0
0
0
0.2
0.25
0
0
0.26
0.3
0.25
0
0.29
0.45
0.75
1
Hence probability in Poor condition at the second inspection is 0.26.
OR
The required probability is equal to
Prob [Excellent to Excellent to Poor] +
Prob [Excellent to Good to Poor] +
Prob [Excellent to Poor to Poor]
which is \[(0.5 × 0.2) + (0.2 × 0.3) + (0.2 × 0.5) = 0.26.\]
(iv)
Excellent
Excellent
Good
Poor
(v)
0.6
0.2
0.5
Good
0.2
0.5
0
Poor
0.2
0.3
0.5
Long-term probabilities satisfy π = π P .
0.6 π E + 0.2 π G + 0.5 π P = π E (1)
0.2 π E + 0.5 π G = π G
(2)
0.2 π E + 0.3 π G + 0.5 π P = π P (3)
%%%%%%%%%%%%%%%%%%%%%%%%%%%%%%%%%%%%%%%%%%%%%%%%%%%%%%%%%%%%%%%%%%%%%%%%%%%%
%%%%%% Page 17Subject CT4 (Models Core Technical) – April 2014 – Examiners’ Report
Also π E + π G + π P = 1.
(2)–(3) gives:
5
π G = π P .
8
So π E =
25
π P .
16
⎛ 25 5 ⎞
Hence ⎜ + + 1 ⎟ π P = 1.
⎝ 16 8 ⎠
Stationary distribution π E =
(vi)
25
10
16
, π G = , π P = .
51
51
51
The expected number of failures of heating elements is:
\[(0.1 π E + 0.2 π G + 0.5 π P )*100 = 24.51.\]
The cost of each failure is £1,050 so the expected cost over a year is £25,735.
(vii)
The transition matrix for the condition of the element at the start of each cycle
will now be:
Excellent Good
0.8
0.5 0.2
0.5
Excellent
Good
The revised stationary distribution satisfies ρ = ρ P
0.8 ρ E + 0.5 ρ G = ρ E (1)
0.2 ρ E + 0.5 ρ G = ρ G (2)
0.2(1 − ρ G ) + 0.5 ρ G = ρ G
ρ E =
5
2
, ρ G =
7
7
Expected cost of failures is now:
(0.1 ρ E + 0.2 ρ G )*100*1050 = £13,500.
Page 18Subject CT4 (Models Core Technical) – April 2014 – Examiners’ Report
But we also now have extra heating element replacement costs of:
(0.2 ρ E + 0.3 ρ G )*100*50 = £1,143 .
So overall profits have improved by:
£25,735 − £13,500 − £1,143 = £11,092.

In part (i) “Markov jump chain” is not correct as the question makes reference to time, and the Markov jump chain loses the information about the timing of the transitions. Answers to part (ii) were very good. In part (iii) the full matrix was not required but some indication of where the numbers have come from or what they are was needed. In parts (iv) onwards many
candidates offered a 4 × 4 matrix as a solution as follows:

Excellent
Excellent
Good
Poor
Failed
0.5
0
0
1
Good
0.2
0.5
0
0
Poor
0.2
0.3
0.5
0
Failed
0.1
0.2
0.5
0
this is incorrect because it implicitly assumes that the kiln is run for a complete cycle with a failed element. It was penalised in part (iv) but full credit could be scored in part (v) for the correctly followed-through stationary distribution.
The better prepared candidates scored highly on this question.

\end{document}
