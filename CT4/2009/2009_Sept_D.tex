\documentclass[a4paper,12pt]{article}

%%%%%%%%%%%%%%%%%%%%%%%%%%%%%%%%%%%%%%%%%%%%%%%%%%%%%%%%%%%%%%%%%%%%%%%%%%%%%%%%%%%%%%%%%%%%%%%%%%%%%%%%%%%%%%%%%%%%%%%%%%%%%%%%%%%%%%%%%%%%%%%%%%%%%%%%%%%%%%%%%%%%%%%%%%%%%%%%%%%%%%%%%%%%%%%%%%%%%%%%%%%%%%%%%%%%%%%%%%%%%%%%%%%%%%%%%%%%%%%%%%%%%%%%%%%%

\usepackage{eurosym}
\usepackage{vmargin}
\usepackage{amsmath}
\usepackage{graphics}
\usepackage{epsfig}
\usepackage{enumerate}
\usepackage{multicol}
\usepackage{subfigure}
\usepackage{fancyhdr}
\usepackage{listings}
\usepackage{framed}
\usepackage{graphicx}
\usepackage{amsmath}
\usepackage{chngpage}

%\usepackage{bigints}
\usepackage{vmargin}

% left top textwidth textheight headheight

% headsep footheight footskip

\setmargins{2.0cm}{2.5cm}{16 cm}{22cm}{0.5cm}{0cm}{1cm}{1cm}

\renewcommand{\baselinestretch}{1.3}

\setcounter{MaxMatrixCols}{10}

\begin{document}
\begin{enumerate}
8
Sketch, using your answers to parts (i) and \item (ii), a graph showing the
probability that a car currently located at the Airport is subsequently at the
Airport, Beach or City against the number of times the car has been rented. 
[Total 10]
A researcher is studying a certain incurable disease. The disease can be fatal, but
often sufferers survive with the condition for a number of years. The researcher
wishes to project the number of deaths caused by the disease by using a multiple state
model with state space:
{ H – Healthy, I – Infected, D (from disease) – Dead (caused by the disease), D (not from disease)
– Dead (not caused by the disease)}.
The transition rates, dependent on age x , are as follows:
\item a mortality rate from the Healthy state of \mu ( x )
\item a rate of infection with the disease σ ( x )
\item a mortality rate from the Infected state of υ ( x ) of which ρ ( x ) relates to Deaths
caused by the disease
(i) Draw a transition diagram for the multiple state model.
\item (ii) Write down Kolmogorov’s forward equations governing the transitions by
specifying the transition matrix.

\item (iii) Determine integral expressions, in terms of the transition rates and any
expressions previously determined, for:
(a) P HH (x, x + t)
(b) P HI (x, x + t)
(c) P HD(from disease) (x, x + t)


[Total 10]
CT4 S2009—5
%%%%%%%%%%%%%%%%%%%%%%%%%%%%%%%%%%%%%%%%%

8
(i)
 ( x )
 ( x )
Healthy Infected
 ( x )  ( x )   ( x )
Dead (not
from
disease)
\item (ii)
d
P ( x )  P ( x ) A ( x )
dt
where with order of state space
{Healthy, Infected, Dead (not disease), Dead(from disease)}
 ( x )
0 
  ( x )   ( x )  ( x )


0
 ( x )  ( x )   ( x )  ( x ) 

A(x)=

0
0
0
0 


0
0
0
0 

10
Dead
(from
disease)  — %%%%%%%%%%%%%%%%%%%%%%%%%%%%%%%%%%%%%%%%%%%% — Examiners’ Report
\item (iii)
t
a. P HH (x, x+t)= exp[ 

(  ( x  w )   ( x  w )) dw ]
w  0
t

b. P HI (x, x+t)=
t
P HH ( x , x  w ).  ( x  w ).exp[ 
w  0

 ( x  u ) du ]. dw
u  w
c. EITHER
t

P HD(from disease) (x, x+t)=
P HI ( x , x  w ).  ( x  w ). dw
w  0
OR (backwards alternative)
P HD(from disease) (x, x+t)
t
 P
=
HH
( x  w ).  ( x  w ). P ID ( fromdiseas e ) ( x  w , x  t ). dw .
w  0
t
Now P ID ( fromdiseas e ) ( x  w , x  t ) 
 P
II
( x  w , x  s ).  ( x  s ). 1 . ds
s  w
 s

and P II ( x  w , x  s )  exp     ( x  u ) du  .
 u  w

So P HD(from disease) (x, x+t)
 s

  P HH ( x  w ).  ( x  w ).  exp     ( x  u ) du  .  ( x  s ). ds . dw
w  0
s  w
 u  w

t
t
This question was considerably better answered than were similar questions in
previous examinations. In particular, the proportion of candidates making serious
attempts at part (iii) was greater than has been the case for similar questions in the
past.


6
(i) A Poisson process is a continuous-time integer valued process
N t , t  0 with
N 0 = 0
independent increments
EITHER
increments follow a Poisson distribution
OR
P [ N t  N s  n ] 
[  ( t  s )] n exp[  ( t  s )]
,
n !
for s < t, n = 0, 1, 2, ....
\item (ii) Average work created by a complaint is
60%* 1⁄2+ 30%* 1 + 10%*4 = 1 day.
Complaints arrive at a rate 1.25 per working day
So, work expected to be generated is 1.25*1*5 = 6.25 person-days.
\item (iii)As the time to handle complaints follows an exponential (memoryless)
distribution, only need to know how many unanswered complaints there are –
7  — %%%%%%%%%%%%%%%%%%%%%%%%%%%%%%%%%%%%%%%%%%%% — Examiners’ Report
but do need to know how many of each type. If cases are allocated randomly
rather than in order, then the state space consists of (in terms of complaints not
resolved):
r – straightforward,
s – medium,
t – complicated.
where r = 0,1,2,3,4,5,....
s = 0,1,2,3,4,5,......
t = 0,1,2,3,4,5,.....
(iv)
EITHER The model will only give an approximation.
OR The model is not suitable for this purpose.
The model could not be used to do this without extending the state space to
consider the time the complaint has been in the queue. There are only two employees, so holidays and sickness are important factors not taken into
account.
The model assumes complaints are time-homogeneous. We do not know the
nature of the business, but for some industries complaints would be seasonal
e.g. holiday companies.
The model assumes that complaint arrivals are independent, but more
complaints might be expected if the company has had a quality control
problem at a particular time. If struggling to meet the service standard, action
would be. Taken, such as overtime, or prioritising easy cases. Staff may be
able to deal with complaints which are similar to other recent complaints very quickly, using standard „template‟ responses.
The memoryless property is unlikely to be realistic as the work required to
complete the case could be assessed and then worked through to a schedule.
The Markov jump process could be used to estimate the probability that a complaint is responded to within a given number of days of receipt.
So the model could be used to estimate the probability of a complaint not
being responded to in the stated time, that is the failure to meet the service
standard.
[1⁄2 mark was awarded for each point up to a maximum of 3 marks]
Answers to this question were disappointing. Most candidates were able to tackle the
calculation in part (ii) but few correctly identified the state space in part (iii), and
most only made a cursory attempt at part (iv).
7
(i) Two step transition matrix
8  — %%%%%%%%%%%%%%%%%%%%%%%%%%%%%%%%%%%%%%%%%%%% — Examiners’ Report
 0.5 0.25 0.25   0.5 0.25 0.25   0.375 0.375 0.25 

 
 

0  .  0.25 0.75
0  =  0.3125 0.625 0.0625 
=  0.25 0.75
 0.25 0.25 0.5   0.25 0.25 0.5   0.3125 0.375 0.3125 


 
 
 0.5 0.25 0.25 


0 
\item (ii)     0.25 0.75
 0.25 0.25 0.5 


 1  0.5  1  0.25  2  0.25  3
 2  0.25  1  0.75  2  0.25  3
 3  0.25  1  0.5  3
and  1   2   3  1
 1  2  3
 2  3  3
 1  1
3
 2  1
 3  1
2
6
\item (iii)The stationary distribution gives the long run probability that a particular car
will be at each location. However this does not take into account the demand
for hiring vehicles at each location, or the amount of space available at each
location. These factors are likely to be more important in determining how
many cars to base at each site.
1.2
1
0.8
Airport
Beach
City
0.6
0.4
0.2
0
0
9
1
2
3
4
5
Number of rentals
6
7
8
9  — %%%%%%%%%%%%%%%%%%%%%%%%%%%%%%%%%%%%%%%%%%%% — Examiners’ Report
(iv) A starts at 1, B and C at zero
Asymptote to the stationary distribution probs.
B and C same after 1 period
A and B same after 2 periods.
\begin{itemize}
\itme The calculations in parts (i) and (ii) were, as is usually the case in CT4 examinations,
successfully completed by the vast majority of candidates. However only a minority
made the point that, whereas the stationary distribution gives the long run probability
that cars will be returned to each location, the company would be better advised to
position cars at the three locations to reflect the demand for rentals. In part (iv), some
candidates drew a set of histograms. 
\item Credit was given for this, provided that
histograms were presented for 1 rental, 2 rentals, and the long run distribution,
together with a statement that at 0 rentals the car must be at the Airport.
\end{itemize}

\end{document}
