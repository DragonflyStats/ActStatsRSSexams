\documentclass[a4paper,12pt]{article}

%%%%%%%%%%%%%%%%%%%%%%%%%%%%%%%%%%%%%%%%%%%%%%%%%%%%%%%%%%%%%%%%%%%%%%%%%%%%%%%%%%%%%%%%%%%%%%%%%%%%%%%%%%%%%%%%%%%%%%%%%%%%%%%%%%%%%%%%%%%%%%%%%%%%%%%%%%%%%%%%%%%%%%%%%%%%%%%%%%%%%%%%%%%%%%%%%%%%%%%%%%%%%%%%%%%%%%%%%%%%%%%%%%%%%%%%%%%%%%%%%%%%%%%%%%%%

\usepackage{eurosym}
\usepackage{vmargin}
\usepackage{amsmath}
\usepackage{graphics}
\usepackage{epsfig}
\usepackage{enumerate}
\usepackage{multicol}
\usepackage{subfigure}
\usepackage{fancyhdr}
\usepackage{listings}
\usepackage{framed}
\usepackage{graphicx}
\usepackage{amsmath}
\usepackage{chngpage}

%\usepackage{bigints}
\usepackage{vmargin}

% left top textwidth textheight headheight

% headsep footheight footskip

\setmargins{2.0cm}{2.5cm}{16 cm}{22cm}{0.5cm}{0cm}{1cm}{1cm}

\renewcommand{\baselinestretch}{1.3}

\setcounter{MaxMatrixCols}{10}

\begin{document}

6
The complaints department of a company has two employees, both of whom work
five days per week.
The company models the arrival of complaints using a Poisson process with rate 1.25
per working day.
\begin{enumerate}
\item (i)
List the assumptions underlying the Poisson process model.
\medskip
On receipt of a complaint, it is immediately assessed as being straightforward, of medium difficulty or complicated. 60\% of cases are assessed as straightforward and
10\% are assessed as complicated. The time taken in person-days’ effort to prepare responses is assumed to follow an exponential distribution, with parameters 2 for
straightforward complaints, 1 for medium difficulty complaints and 0.25 for complicated complaints.

\item (ii) Calculate the average number of person-days’ work expected to be generated by complaints arriving during a five-day working week.

\item (iii) Define a state space under which the number of outstanding complaints can be modelled as a Markov jump process.
\medskip
The company has a service standard of responding to complaints within a fixed number of days of receipt. It is considering using this Markov jump process to model
the probability of failing to meet this service standard.

\item (iv)
7
Discuss the appropriateness of using the model for this purpose, with reference
to the assumptions being made.
\end{itemize}
%%%%%%%%%%%%%%%%%%%%%%%%%%%%%%%%%%%%%%%%%%%%%%%%%%%%%%%%%%%%%%%%%%%%%%%%%%%
\newpage

6
\begin{itemize}
\item (i) A Poisson process is a continuous-time integer valued process
N t , t  0 with
N 0 = 0
independent increments
EITHER
increments follow a Poisson distribution
OR
P [ N t  N s  n ] 
[  ( t  s )] n exp[  ( t  s )]
,
n !
for s < t, n = 0, 1, 2, ....
\item (ii) Average work created by a complaint is
60%* 1⁄2+ 30%* 1 + 10%*4 = 1 day.
Complaints arrive at a rate 1.25 per working day
So, work expected to be generated is 1.25*1*5 = 6.25 person-days.
\item (iii)As the time to handle complaints follows an exponential (memoryless)
distribution, only need to know how many unanswered complaints there are –
7  — %%%%%%%%%%%%%%%%%%%%%%%%%%%%%%%%%%%%%%%%%%%% — Examiners’ Report
but do need to know how many of each type. If cases are allocated randomly
rather than in order, then the state space consists of (in terms of complaints not
resolved):
r – straightforward,
s – medium,
t – complicated.
where r = 0,1,2,3,4,5,....
s = 0,1,2,3,4,5,......
t = 0,1,2,3,4,5,.....
\item (iv)
EITHER The model will only give an approximation.
OR The model is not suitable for this purpose.
The model could not be used to do this without extending the state space to
consider the time the complaint has been in the queue. There are only two employees, so holidays and sickness are important factors not taken into
account.
\item The model assumes complaints are time-homogeneous. We do not know the
nature of the business, but for some industries complaints would be seasonal
e.g. holiday companies.
\item The model assumes that complaint arrivals are independent, but more
complaints might be expected if the company has had a quality control
problem at a particular time. If struggling to meet the service standard, action
would be. Taken, such as overtime, or prioritising easy cases. Staff may be
able to deal with complaints which are similar to other recent complaints very quickly, using standard „template‟ responses.
\item The memoryless property is unlikely to be realistic as the work required to
complete the case could be assessed and then worked through to a schedule.
The Markov jump process could be used to estimate the probability that a complaint is responded to within a given number of days of receipt.
\item So the model could be used to estimate the probability of a complaint not
being responded to in the stated time, that is the failure to meet the service
standard.
\item [1⁄2 mark was awarded for each point up to a maximum of 3 marks]
\item Answers to this question were disappointing. Most candidates were able to tackle the
calculation in part (ii) but few correctly identified the state space in part (iii), and
most only made a cursory attempt at part (iv).
\end{itemize}
                
                
\end{document}
