Total 16]
CT4 A2009—612
A motor insurer operates a no claims discount system with the following levels of
discount {0%, 25%, 50%, 60%}.
The rules governing a policyholder’s discount level, based upon the number of claims
made in the previous year, are as follows:
• Following a year with no claims, the policyholder moves up one discount level, or
remains at the 60% level.
• Following a year with one claim, the policyholder moves down one discount level,
or remains at 0% level.
• Following a year with two or more claims, the policyholder moves down two
discount levels (subject to a limit of the 0% discount level).
The number of claims made by a policyholder in a year is assumed to follow a
Poisson distribution with mean 0.30.
\item (i) Determine the transition matrix for the no claims discount system. 
\item (ii) Calculate the stationary distribution of the system, \pi . 
\item (iii) Calculate the expected average long term level of discount. 
The following data shows the number of the insurer’s 130,200 policyholders in the
portfolio classified by the number of claims each policyholder made in the last year.
This information was used to estimate the mean of 0.30.
No claims
96,632
One claim
28,648
Two claims
4,400
Three claims
476
Four claims
36
Five claims
8
(iv) Test the goodness of fit of these data to a Poisson distribution with mean 0.30.

(v) Comment on the implications of your conclusion in (iv) for the average level
of discount applied.

[Total 16]
END OF PAPER
CT4 A2009—7

%%%%%%%%%%%%%%%%%%%%%%%%%%%%%%%%%%%%%%%%%%%%%%%%%%%
\newpage


12
\item (i)
The probability of making the relevant number of claims is:
P[0 claims] = exp(−0.3) = 0.740818
P[1 claim] = 0.3exp(−0.3) = 0.222245
So P[2 or more claims] = 1 − 0.740818 − 0.222245 = 0.036936
Therefore the transition matrix P is given by:
0
0
⎛ 0.259182 0.740818
⎞
⎜
⎟
0
0.740818
0
⎜ 0.259182
⎟
⎜ 0.036936 0.222245
0
0.740818 ⎟
⎜
⎟
0
0.036936 0.222245 0.740818 ⎠
⎝
\item (ii)
\pi = \pi P
\pi_1 = 0.259182 \pi_1 + 0.259182 \pi_2 + 0.036936 \pi 3
\pi_2 = 0.740818 \pi_1 + 0.222245 \pi 3 + 0.036936 \pi 4
\pi 3 = 0.740818 \pi_2 + 0.222245 \pi 4
\pi 4 = 0.740818 \pi 3 + 0.740818 \pi 4
\pi_1 + \pi_2 + \pi 3 + \pi 4 = 1
(1)
(2)
(3)
(4)
Page 17Subject CT4 — %%%%%%%%%%%%%%%%%%%%%%%%%%%%%%%%%%%%%%%%%%%%%%% — Examiners’ Report
Using (4)
\pi 3 = [(1 − 0.740818) / 0.740818]* \pi 4 = 0.349859 \pi 4 .
In (3)
\pi_2 = [(0.349859 − 0.222245) / 0.740818]* \pi 4 = 0.17226 \pi 4 .
Then in (2)
\pi_1 = [(0.17226 − 0.036936 − 0.222245*0.349859) / 0.740818]* \pi 4 = 0.07771 \pi 4
So
\pi 4 = 1/ (1+0.349859+0.17226+0.07771)=0.625067
\pi 3 = 0.218685
\pi_2 = 0.107674
\pi_1 = 0.048574
\item (iii)
Average discount =
60%*0.625067+50%*0.218685+25%*0.107674 = 51.13%
(iv)
The total number of policyholders shown is 130,200.
Number of
claims
0
1
2
3
4
5
Probability
0.740818221
0.222245466
0.03333682
0.003333682
0.000250026
1.50016E−05
Expected
Number
96454.53
28936.35
4340.45
434.05
32.55
1.95
Observed
96632
28648
4400
476
36
8
(O − E) 2 /E
0.327
2.873
0.817
4.054
0.366
18.771
Null hypothesis: the data come from a source where the underlying
distribution of number of claims follows a Poisson distribution with mean
0.30.
The test statistic z =
∑ ( O i − E i ) 2
i
E i
is distributed as chi-square
with (6 − 1(parameter) − 5 degrees of freedom under the null hypothesis.
This is a one-tailed test, and the upper 5% point of the chi-squared distribution
with 5 degrees of freedom is 11.07.
The observed value of the test statistic is 27.2.
Page 18Subject CT4 — %%%%%%%%%%%%%%%%%%%%%%%%%%%%%%%%%%%%%%%%%%%%%%% — Examiners’ Report
As 27.2 > 11.07 we reject the null hypothesis.
(v)
As the goodness of test fails, the discount level calculated assuming the
Poisson distribution may be incorrect.
The goodness-of-fit test fails due to a larger number of multiple
claims than expected.
Conversely a higher number of policyholders make no claims than expected
(within the mean of 0.30), so the average discount level may be understated.
The average discount level calculated from the data could usefully be
compared with that estimated using the Poisson distribution.
END OF EXAMINERS’ REPORT
Page 19
