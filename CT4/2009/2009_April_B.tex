\documentclass[a4paper,12pt]{article}

%%%%%%%%%%%%%%%%%%%%%%%%%%%%%%%%%%%%%%%%%%%%%%%%%%%%%%%%%%%%%%%%%%%%%%%%%%%%%%%%%%%%%%%%%%%%%%%%%%%%%%%%%%%%%%%%%%%%%%%%%%%%%%%%%%%%%%%%%%%%%%%%%%%%%%%%%%%%%%%%%%%%%%%%%%%%%%%%%%%%%%%%%%%%%%%%%%%%%%%%%%%%%%%%%%%%%%%%%%%%%%%%%%%%%%%%%%%%%%%%%%%%%%%%%%%%

\usepackage{eurosym}
\usepackage{vmargin}
\usepackage{amsmath}
\usepackage{graphics}
\usepackage{epsfig}
\usepackage{enumerate}
\usepackage{multicol}
\usepackage{subfigure}
\usepackage{fancyhdr}
\usepackage{listings}
\usepackage{framed}
\usepackage{graphicx}
\usepackage{amsmath}
\usepackage{chngpage}

%\usepackage{bigints}
\usepackage{vmargin}

% left top textwidth textheight headheight

% headsep footheight footskip

\setmargins{2.0cm}{2.5cm}{16 cm}{22cm}{0.5cm}{0cm}{1cm}{1cm}

\renewcommand{\baselinestretch}{1.3}

\setcounter{MaxMatrixCols}{10}

\begin{document}
\begin{enumerate}

5
Suppose we have a set of n crude mortality rates for a given age range x to x + n − 1,
and we wish to compare them to a standard set of n mortality rates for the same age
range.
If the mortality underlying the crude rates is the same as that of the standard set of rates (the null hypothesis), then we should expect the difference between the two sets
of rates to be due only to sampling variability.
The grouping of signs test tests the null hypothesis by examining the number of groups of consecutive positive deviations among the n ages, where a positive
deviation occurs when the crude rate exceeds the corresponding rate in the standard set.
Suppose there are a total of m positive deviations, n – m negative deviations and G positive groups.
Then the number of possible ways to arrange t positive groups among n – m negative ⎛ n − m + 1 ⎞
deviations is ⎜
⎟ .
⎝ t
⎠
Page 7%%%%%%%%%%%%%%%%%%%%%%%%%%%— %%%%%%%%%%%%%%%%%%%%%%%%%%%%%%%%%%%%%%%%%%%%%%% — Examiners’ Report
⎛ m − 1 ⎞
There are ⎜
⎟ ways to arrange m positive signs into t positive groups.
⎝ t − 1 ⎠
⎛ n ⎞
There are ⎜ ⎟ ways to arrange m positive and n – m negative signs.
⎝ m ⎠
Therefore the probability of exactly t positive groups is
⎛ n − m + 1 ⎞ ⎛ m − 1 ⎞
⎜
⎟⎜
⎟
t
t − 1 ⎠
⎝
⎠
⎝
Pr[ G = t ] =
⎛ n ⎞
⎜ ⎟
⎝ m ⎠
The grouping of signs test then evaluates Pr[ t ≤ G ] under the null hypothesis.
If this is less than 0.05 we reject the null hypothesis at the 5\%  level.
6
\item (i)
(a)
The relevant recovery rates can be estimated as
r x =
d x
E x c
, x = 0, 1, 2, ... months
where d x is the number of persons recovering in the calendar month that was x months after the calendar month of their operation, and E x c is
the central exposed to risk.
(b)
We need to ensure that the E x c correspond to the data on persons recovering
The hospital’s data imply a calendar month rate interval for the recoveries, running from the first day of each month until the last day
of each month.
Using the monthly “census” data, a definition of E x c which corresponds to the deaths data can be obtained as follows.
We observe P x , t = number of lives under observation for whom the time elapsing since the operation was between x − 1⁄2 and x + 1⁄2
months, where t is the time in months since 1 January 2008.
Page 8%%%%%%%%%%%%%%%%%%%%%%%%%%%— %%%%%%%%%%%%%%%%%%%%%%%%%%%%%%%%%%%%%%%%%%%%%%% — Examiners’ Report
Therefore, using the census formula:
E x c =
12 11
0 0
\int P * x , t dt = ∑ 1
2 (
P * x , t + P * x + 1, t + 1 ) ,
where P * x , t = 1 ( P x − 1, t + P x , t ) .
2
We assume all months are the same length, and that the numbers in the
hospital vary linearly across each month.
\item (ii)
At the start of the rate interval, durations since the operation range from x − 1
to x months, so the average duration is x − 1⁄2, assuming operations take place
evenly across the month.
r x estimates the recovery rate at the mid-point of the rate interval .
This is exactly x months since the operation, so f = 0.
\end{document}
