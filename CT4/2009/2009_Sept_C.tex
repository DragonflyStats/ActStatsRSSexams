\documentclass[a4paper,12pt]{article}

%%%%%%%%%%%%%%%%%%%%%%%%%%%%%%%%%%%%%%%%%%%%%%%%%%%%%%%%%%%%%%%%%%%%%%%%%%%%%%%%%%%%%%%%%%%%%%%%%%%%%%%%%%%%%%%%%%%%%%%%%%%%%%%%%%%%%%%%%%%%%%%%%%%%%%%%%%%%%%%%%%%%%%%%%%%%%%%%%%%%%%%%%%%%%%%%%%%%%%%%%%%%%%%%%%%%%%%%%%%%%%%%%%%%%%%%%%%%%%%%%%%%%%%%%%%%

\usepackage{eurosym}
\usepackage{vmargin}
\usepackage{amsmath}
\usepackage{graphics}
\usepackage{epsfig}
\usepackage{enumerate}
\usepackage{multicol}
\usepackage{subfigure}
\usepackage{fancyhdr}
\usepackage{listings}
\usepackage{framed}
\usepackage{graphicx}
\usepackage{amsmath}
\usepackage{chngpage}

%\usepackage{bigints}
\usepackage{vmargin}

% left top textwidth textheight headheight

% headsep footheight footskip

\setmargins{2.0cm}{2.5cm}{16 cm}{22cm}{0.5cm}{0cm}{1cm}{1cm}

\renewcommand{\baselinestretch}{1.3}

\setcounter{MaxMatrixCols}{10}

\begin{document}
\begin{enumerate}
6
The complaints department of a company has two employees, both of whom work
five days per week.
The company models the arrival of complaints using a Poisson process with rate 1.25
per working day.
(i)
List the assumptions underlying the Poisson process model.

On receipt of a complaint, it is immediately assessed as being straightforward, of medium difficulty or complicated. 60\% of cases are assessed as straightforward and
10\% are assessed as complicated. The time taken in person-days’ effort to prepare responses is assumed to follow an exponential distribution, with parameters 2 for
straightforward complaints, 1 for medium difficulty complaints and 0.25 for complicated complaints.

\item (ii) Calculate the average number of person-days’ work expected to be generated by complaints arriving during a five-day working week.

\item (iii) Define a state space under which the number of outstanding complaints can be modelled as a Markov jump process.

The company has a service standard of responding to complaints within a fixed number of days of receipt. It is considering using this Markov jump process to model
the probability of failing to meet this service standard.

(iv)
7
Discuss the appropriateness of using the model for this purpose, with reference
to the assumptions being made.
%%%%%%%%%%%%%%%%%%%%%%%%%%%%%%%%%%%%%%%%%%%%%%%%%%%%%%%%%%%%%%%%%%%%%%%%%%%
\newpage
[Total 9]
A firm rents cars and operates from three locations — the Airport, the Beach and the
City. Customers may return vehicles to any of the three locations.
The company estimates that the probability of a car being returned to each location is
as follows:
Car hired from Car returned to
Airport Beach City
Airport
Beach
City 0.5
0.25
0.25
0.25
0.75
0.25
0.25
0
0.5
\begin{enumerate}[(i)]
\item (i) Calculate the 2-step transition matrix. 
\item (ii) Calculate the stationary distribution $\pi$ . 
It is suggested that the cars should be based at each location in proportion to the
stationary distribution.
\item (iii)
Comment on this suggestion.
%%CT4 S2009—4
(iv)
\end{enumerate}

$$\bordermatrix{ &c_1&c_2&\ldots &c_n\cr
                r_1&a_{11} &  0  & \ldots & a_{1n}\cr
                r_2& 0  &  a_{22} & \ldots & a_{2n}\cr
                r_4& 0  &   0       &\ldots & a_{nn}}$$
                
                
\end{document}
