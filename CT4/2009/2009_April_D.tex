\documentclass[a4paper,12pt]{article}

%%%%%%%%%%%%%%%%%%%%%%%%%%%%%%%%%%%%%%%%%%%%%%%%%%%%%%%%%%%%%%%%%%%%%%%%%%%%%%%%%%%%%%%%%%%%%%%%%%%%%%%%%%%%%%%%%%%%%%%%%%%%%%%%%%%%%%%%%%%%%%%%%%%%%%%%%%%%%%%%%%%%%%%%%%%%%%%%%%%%%%%%%%%%%%%%%%%%%%%%%%%%%%%%%%%%%%%%%%%%%%%%%%%%%%%%%%%%%%%%%%%%%%%%%%%%

\usepackage{eurosym}
\usepackage{vmargin}
\usepackage{amsmath}
\usepackage{graphics}
\usepackage{epsfig}
\usepackage{enumerate}
\usepackage{multicol}
\usepackage{subfigure}
\usepackage{fancyhdr}
\usepackage{listings}
\usepackage{framed}
\usepackage{graphicx}
\usepackage{amsmath}
\usepackage{chngpage}

%\usepackage{bigints}
\usepackage{vmargin}

% left top textwidth textheight headheight

% headsep footheight footskip

\setmargins{2.0cm}{2.5cm}{16 cm}{22cm}{0.5cm}{0cm}{1cm}{1cm}

\renewcommand{\baselinestretch}{1.3}

\setcounter{MaxMatrixCols}{10}

\begin{document}
\begin{enumerate}


8
There is a population of ten cats in a certain neighbourhood. Whenever a cat which has fleas meets a cat without fleas, there is a 50% probability that some of the fleas
transfer to the other cat such that both cats harbour fleas thereafter. Contacts between two of the neighbourhood cats occur according to a Poisson process with rate $\mu$, and these meetings are equally likely to involve any of the possible pairs of individuals.
Assume that once infected a cat continues to have fleas, and that none of the cats’ owners has taken any preventative measures.
\item (i) If the number of cats currently infected is x, explain why the number of possible pairings of cats which could result in a new flea infection is x(10 – x).

\item (ii) Show how the number of infected cats at any time, X(t), can be formulated as
a Markov jump process, specifying:
(a)
(b)
the state space
the Kolmogorov differential equations in matrix form

9
\item (iii) State the distribution of the holding times of the Markov jump process.
(iv) Calculate the expected time until all the cats have fleas, starting from a single
flea-infected cat.

[Total 9]


%%%%%%%%%%%%%%%%%%%%%%%%%%%%%%%%

\item (i) Prove that, under Gompertz’s Law, the probability of survival from age x to
age x + t, t p x , is given by:
⎡
⎛ − B ⎞ ⎤
t p x = ⎢ exp ⎜
⎟ ⎥
⎝ ln c ⎠ ⎦
⎣

c x ( c t − 1)
.

For a certain population, estimates of survival probabilities are available as follows:
= 0.995
2 p 50 = 0.989 .
1 p 50
\item (ii) Calculate values of B and c consistent with these observations.
\item (iii) Comment on the calculation performed in \item (ii) compared with the usual process
for estimating the parameters from a set of crude mortality rates.

[Total 9]
CT4 A2009—4
10
Let T x be a random variable denoting future lifetime after age x, and let T be
another random variable denoting the lifetime of a new-born person.
\item (i)
\item (ii)
(a) Define, in terms of probabilities, S_x(t) , which represents the survival
function of T x .
(b) Derive an expression relating S_x(t) to S ( t ) , the survival function of T.

Define, in terms of probabilities involving T x , the force of mortality, \mu x + t .

The Weibull distribution has a survival function given by
(
)
S_x(t) = exp − ( \lambda t ) \beta ,
where \lambda and \beta are parameters (\lambda, \beta > 0).
\item (iii)
Derive an expression for the Weibull force of mortality in terms of \lambda and \beta.

(iv)
Sketch, on the same graph, the Weibull force of mortality for 0 ≤ t ≤ 5 for the
following pairs of values of \lambda and \beta:
\lambda = 1, \beta = 0.5
\lambda = 1, \beta = 1.0
\lambda = 1, \beta = 1.5

[Total 10]
CT4 A2009—5

%%%%%%%%%%%%%%%%%%%%%%%%%%%%%%%%%%%%%%%%%%%%%%%%%%%%%%%%%%%%%%%%%%%%%%%%%%%%


9
\item (i)
Under Gompertz’s Law
\mu x = Bc x .
Since
⎛ t
⎞
⎜
⎟ ,
=
−
\mu
p
exp
dw
t x
⎜ \int x + w ⎟
⎝ 0
⎠
⎛
⎛ t
⎞
Bc x c w
we have t p x = exp ⎜ − \int Bc x + w dw ⎟ = exp ⎜ −
⎜
⎟
⎜
ln c
⎝ 0
⎠
⎝
⎛ ⎡ Bc x c t − Bc x ⎤ ⎞
⎦ ⎟ =
which is exp ⎜ − ⎣
⎜
⎟
ln c
⎝
⎠
\item (ii)
⎡
⎛ − B ⎞ ⎤
Define Q = ⎢ exp ⎜
⎟ ⎥
⎝ ln c ⎠ ⎦
⎣
⎡
⎛ − B ⎞ ⎤
⎢ exp ⎜ ln c ⎟ ⎥
⎝
⎠ ⎦
⎣
⎞
⎟ ,
⎟
0 ⎠
t
c x ( c t − 1)
.
c 50
ln 0.995 = (c − 1) ln Q
ln 0.989 = (c 2 − 1) ln Q
( c 2 − 1) ( c − 1)( c + 1)
=
= 2.20665
( c − 1)
( c − 1)
c = 1.20665
Therefore Q = 0.976036128
1.20665 50
⎡
− B
⎛
⎞ ⎤
⎢ exp ⎜ ln1.20665 ⎟ ⎥
⎝
⎠ ⎦
⎣
= 0.976036128
B = 3.797*10 − 7 .
\item (iii)
In this example, only two observations are provided so there is an analytical solution to the Gompertz model.
This is unrealistic as in general a graduation process would be used to provide a fit to a set of crude rates.
This could be done by weighted least squares or maximum likelihood.

\newpage
Page 12%%%%%%%%%%%%%%%%%%%%%%%%%%%— %%%%%%%%%%%%%%%%%%%%%%%%%%%%%%%%%%%%%%%%%%%%%%% — Examiners’ Report
The more general graduation process allows the fitting of more complex
models from the Gompertz-Makeham family which have the form
\mu x = polynomial(1) + exp(polynomial(2))
the parameters of which cannot always so easily be estimated by the method
used in part \item (ii).
10
\item (i)
(a) S_x(t) = Pr[ T x > t ]
(b) EITHER
Since Pr[ T x > t ] = Pr[ T > x + t | T > x ] =
Pr[ T > x + t ]
Pr[ T > x ]
and S ( t ) = Pr[ T > t ] ,
then S_x(t) =
S ( x + t )
.
S ( x )
OR
Since S_x(t) = t p x , then using the consistency principle
x + t p 0 = t p x . x p 0
Therefore t p x = S_x(t) =
\item (ii)
p 0 S ( x + t )
=
.
S ( x )
x p 0
x + t
EITHER
\mu x + t = −
1
d
[Pr( T x > t )]
Pr[ T x > t ] dt
OR
\mu x + t = lim
h → 0
\item (iii)
+
1
( Pr[ T x ≤ t + h | T x > t )
h
EITHER
If the density function of T x is f x ( t ) , then we can write
f x ( t ) = S_x(t) \mu x + t = −
d
S_x(t)
dt
Page 13%%%%%%%%%%%%%%%%%%%%%%%%%%%— %%%%%%%%%%%%%%%%%%%%%%%%%%%%%%%%%%%%%%%%%%%%%%% — Examiners’ Report
Therefore \mu x + t = −
1 d
S_x(t)
S_x(t) dt
(
)
If S_x(t) = exp − ( \lambda t ) \beta , therefore, we have
\mu x + t = −
\mu x + t = −
(
1
exp − ( \lambda t ) \beta
)
(
d
exp − ( \lambda t ) \beta
dt
)
( exp ( − ( \lambda t ) ) ) ( −\lambda \beta t ) = \lambda \beta t
exp ( − ( \lambda t ) )
1
\beta
\beta
OR
⎡ t
⎤
S_x(t) = exp ⎢ − \int \mu x + s ds ⎥ = exp ⎡ − ( \lambda t ) \beta ⎤ .
⎣
⎦
⎢ ⎣ 0
⎥ ⎦
So
t
⎤
d ⎡
d
⎢ \int \mu x + s ds ⎥ = \mu x + t = ⎡ ( \lambda t ) \beta ⎤ ,
⎦
dt ⎢
dt ⎣
⎥ ⎦
⎣ 0
and hence
\mu x + t = \beta\lambda \beta t \beta− 1 .
(iv)
Page 14
\beta
\beta− 1
\beta
\beta− 1%%%%%%%%%%%%%%%%%%%%%%%%%%%— %%%%%%%%%%%%%%%%%%%%%%%%%%%%%%%%%%%%%%%%%%%%%%% — Examiners’ Report

