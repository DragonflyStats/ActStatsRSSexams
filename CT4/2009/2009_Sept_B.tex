\documentclass[a4paper,12pt]{article}

%%%%%%%%%%%%%%%%%%%%%%%%%%%%%%%%%%%%%%%%%%%%%%%%%%%%%%%%%%%%%%%%%%%%%%%%%%%%%%%%%%%%%%%%%%%%%%%%%%%%%%%%%%%%%%%%%%%%%%%%%%%%%%%%%%%%%%%%%%%%%%%%%%%%%%%%%%%%%%%%%%%%%%%%%%%%%%%%%%%%%%%%%%%%%%%%%%%%%%%%%%%%%%%%%%%%%%%%%%%%%%%%%%%%%%%%%%%%%%%%%%%%%%%%%%%%

\usepackage{eurosym}
\usepackage{vmargin}
\usepackage{amsmath}
\usepackage{graphics}
\usepackage{epsfig}
\usepackage{enumerate}
\usepackage{multicol}
\usepackage{subfigure}
\usepackage{fancyhdr}
\usepackage{listings}
\usepackage{framed}
\usepackage{graphicx}
\usepackage{amsmath}
\usepackage{chngpage}

%\usepackage{bigints}
\usepackage{vmargin}

% left top textwidth textheight headheight

% headsep footheight footskip

\setmargins{2.0cm}{2.5cm}{16 cm}{22cm}{0.5cm}{0cm}{1cm}{1cm}

\renewcommand{\baselinestretch}{1.3}

\setcounter{MaxMatrixCols}{10}

\begin{document}
\begin{enumerate}

[Total 7]4
(i)
In the context of mortality investigations describe the principle of
correspondence and give an example of a situation in which it may be hard to
adhere to this principle.

On 1 January 2005 a country introduced a comprehensive system of death
registration, which classified deaths by age last birthday on the date of death.
The government of the country wishes to obtain estimates of the force of mortality,
\mu x , by single years of age x for the period between 1 January 2005 and 1 January
2008. Annual population censuses have been taken on 30 June each year since 2004,
which classify the population by age last birthday. However the only copy of the data
from the population census of 30 June 2006 was lost when the computer disc on
which it was stored was being transferred between government departments.
Let the population aged x last birthday on 30 June in year t be denoted by the symbol
P x , t , and the number of deaths during the period of investigation of persons aged x be
denoted by the symbol d x .
\item (ii)
Derive an expression in terms of P x , t and d x which may be used to estimate
\mu x .
5
(i)
[6]
[Total 8]
State the Markov property.

A stochastic process X ( t ) operates with state space S .
\item (ii) Prove that if the process has independent increments it satisfies the Markov
property.

\item (iii) (a)
Describe the difference between a Markov chain and a Markov jump
process.
(b)
Explain what is meant by a Markov chain being irreducible.

An actuarial student can see the office lift (elevator) from his desk. The lift has an
indicator which displays on which of the office’s five floors it is at any point in time.
For light relief the student decides to construct a model to predict the movements of
the lift.
(iv)
Explain whether it would be appropriate to select a model which is:
(a)
(b)
irreducible
has the Markov property

[Total 9]
CT4 S2009—3

%%%%%%%%%%%%%%%%%%%%%%%%%
4
(i) The principle of correspondence states that a life alive at time t should be
included in the exposure at age x at time t if and only if, were that life to die
immediately, he or she would be counted in the deaths data at age x. Problems
in adhering to this can arise when the deaths data and the exposed-to-risk data
come from two different sources. These may classify lives differently.
\item (ii) Since deaths are classified by age last birthday at date of death, a central
exposed to risk which corresponds to the deaths data is given by
t  3
E x c

 P x , t
t  0
where P x , t is the population aged x last birthday at time t, and t is measured in
years since 1 January 2005. We have censuses on 30 June 2004, 30 June 2005,
30 June 2007 and 30 June 2008.
Assuming that the population varies linearly across the period between each
successive census for which we have data the population aged x last birthday
on 1 January 2005 is equal to
1 ( P
 P x ,30 / 6 / 2005 )
2 x ,30 / 6 / 2004
and the population aged x last birthday on 1 January 2008 is equal to
1 ( P
 P x ,30 / 6 / 2008 ) .
2 x ,30 / 6 / 2007
Dividing the period of the investigation into three sub-periods
from 1 January 2005 to 30 June 2005
from 30 June 2005 to 30 June 2007
from 30 June 2007 to 1 January 2008
and applying the trapezium rule to each sub-period produces the following
exposed to risk for persons aged x last birthday
For the sub-period between 1 January 2005 and 30 June 2005
Page 5  — %%%%%%%%%%%%%%%%%%%%%%%%%%%%%%%%%%%%%%%%%%%% — Examiners’ Report
1  1 ( P
 P x ,30/6/2005 ) 
2  2 x ,1/1/2005

 1  1 ( 1 ( P x ,30/6/2004  P x ,30/6/2005 )  P x ,30/6/2005 ) 
2  2 2

For the sub-period between 30 June 2005 and 30 June 2007
2  1 ( P x ,30/6/2005  P x ,30/6/2007 ) 
 2

For the sub-period between 30 June 2007 and 1 January 2008
1  1 ( P
 P x ,1/1/2008 ) 
2  2 x ,30/6/2007

 1  1 ( P x ,30/6/2007  1 ( P x ,30/6/2007  P x ,30/6/2008 )) 
2  2
2

Summing these gives
E x c  1 P x ,30/6/2004  1 P x ,30/6/2005  1 P x ,30/6/2005  P x ,30/6/2005
8
8
4
 P x ,30/6/2007  1 P x ,30/6/2007  1 P x ,30/6/2007  1 P x ,30/6/2008
4
8
8
which simplifies to
E x c  1 P x ,30/6/2004  11 P x ,30/6/2005  11 P x ,30/6/2007  1 P x ,30/6/2008 .
8
8
8
8
The force of mortality may be estimated using the formula
 x 
d x
E x c
,
where d x denotes deaths to persons aged x last birthday when they died.
This was very poorly answered. It was perhaps rather more difficult than some
exposed-to-risk questions in previous examination papers, but nevertheless the
standard of most attempts was disappointing. In part \item (ii) credit was given for various
alternative approximations provided that they were explained clearly.
5
(i) The Markov property states that the future development of a process can be
predicted from its present state alone without reference to its past history.
\item (ii) Formally, for times s 1  s 2  ...  s n  s  t and for states x 1 , x 2 ,..., x n , x in the
state space S and all subsets A of S, the Markov property can be written
Pr[ X ( t )  A | X ( s 1 )  x 1 , X ( s 2 )  x 2 ,...., X ( s n )  x n , X ( s )  x ]  Pr[ X t  A | X ( s )  x ]
For independent increments we can write
Pr[ X ( t )  A | X ( s 1 )  x 1 , X ( s 2 )  x 2 ,...., X ( s n )  x n , X ( s )  x ]
 Pr[ X ( t )  X ( s )  x  A | X ( s 1 )  x 1 , X ( s 2 )  x 2 ,...., X ( s n )  x n , X ( s )  x ]
 Pr[ X ( t )  X ( s )  x  A | X ( s )  x ]
 Pr[ X ( t )  A | X ( s )  x ]
Page 6  — %%%%%%%%%%%%%%%%%%%%%%%%%%%%%%%%%%%%%%%%%%%% — Examiners’ Report
\item (iii)
a. A Markov chain is a stochastic process with the Markov property
which has a discrete time set with a discrete state space. A Markov
jump process is a stochastic process with the Markov property which
has a continuous time set with a discrete state space.
b.A Markov chain is irreducible if any state can be reached from any
other state.
(iv)
a. A lift could not serve its purpose unless it could return to each of the
floors which it serves. This means an irreducible model would be
appropriate.
b.Suppose, for example, the lift is currently at the third floor, with its last
two states being the fourth floor and the fifth floor. In such a case the
lift is more likely to be heading downwards than upwards. So the past
history is likely to provide information on the likely future movement
of the lift, unless the state space is very complicated (involving a
number of past floors as well as the current floor). Therefore a Markov
model is unlikely to be appropriate.
This question was generally well answered, apart from section (iv)(b) in which few
candidates spotted the point that the direction of travel of the lift as well as its current
floor will influence its next location.
