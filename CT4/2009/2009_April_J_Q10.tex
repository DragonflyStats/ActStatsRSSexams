\documentclass[a4paper,12pt]{article}

%%%%%%%%%%%%%%%%%%%%%%%%%%%%%%%%%%%%%%%%%%%%%%%%%%%%%%%%%%%%%%%%%%%%%%%%%%%%%%%%%%%%%%%%%%%%%%%%%%%%%%%%%%%%%%%%%%%%%%%%%%%%%%%%%%%%%%%%%%%%%%%%%%%%%%%%%%%%%%%%%%%%%%%%%%%%%%%%%%%%%%%%%%%%%%%%%%%%%%%%%%%%%%%%%%%%%%%%%%%%%%%%%%%%%%%%%%%%%%%%%%%%%%%%%%%%

\usepackage{eurosym}
\usepackage{vmargin}
\usepackage{amsmath}
\usepackage{graphics}
\usepackage{epsfig}
\usepackage{enumerate}
\usepackage{multicol}
\usepackage{subfigure}
\usepackage{fancyhdr}
\usepackage{listings}
\usepackage{framed}
\usepackage{graphicx}
\usepackage{amsmath}
\usepackage{chngpage}

%\usepackage{bigints}
\usepackage{vmargin}

% left top textwidth textheight headheight

% headsep footheight footskip

\setmargins{2.0cm}{2.5cm}{16 cm}{22cm}{0.5cm}{0cm}{1cm}{1cm}

\renewcommand{\baselinestretch}{1.3}

\setcounter{MaxMatrixCols}{10}

\begin{document}
[Total 9]
CT4 A2009—4
10
Let $T_x$ be a random variable denoting future lifetime after age x, and let T be
another random variable denoting the lifetime of a new-born person.
\item (i)
\item (ii)
(a) Define, in terms of probabilities, $S_x(t)$ , which represents the survival
function of $T_x$ .
(b) Derive an expression relating $S_x(t)$ to $S ( t )$ , the survival function of $T$.

Define, in terms of probabilities involving T x , the force of mortality, \mu x + t .

The Weibull distribution has a survival function given by

\[
S_x(t) = exp − ( \lambda t ) \beta ,
\]
where $\lambda$ and $\beta$ are parameters ($\lambda, \beta > 0$).
\item (iii)
Derive an expression for the Weibull force of mortality in terms of $\lambda$ and $\beta$.

(iv)
Sketch, on the same graph, the Weibull force of mortality for $0 \leq t \leq 5$ for the
following pairs of values of $\lambda$ and $\beta$:
\lambda = 1, \beta = 0.5
\lambda = 1, \beta = 1.0
\lambda = 1, \beta = 1.5

[Total 10]
CT4 A2009—5
10
\item (i)
(a) $S_x(t) = Pr[ T x > t ]$
(b) EITHER
Since 
\begin{eqnarray*} Pr[ T x > t ] &=& Pr[ T > x + t | T > x ] \\ &=&
Pr[ T > x + t ] \times
Pr[ T > x ]
\end{eqnarray*}
and S ( t ) = Pr[ T > t ] ,
then S_x(t) =
S ( x + t )
.
S ( x )
OR
Since S_x(t) = t p x , then using the consistency principle
x + t p 0 = t p x . x p 0
Therefore t p x = S_x(t) =
\item (ii)
p 0 S ( x + t )
=
.
S ( x )
x p 0
x + t
EITHER
\mu x + t = −
1
d
[Pr( T x > t )]
Pr[ T x > t ] dt
OR
\mu x + t = lim
h → 0
\item (iii)
+
1
( Pr[ T x ≤ t + h | T x > t )
h
EITHER
If the density function of T x is $f_x ( t )$ , then we can write
f x ( t ) = S_x(t) \mu x + t = −
d
S_x(t)
dt
Page 13%%%%%%%%%%%%%%%%%%%%%%%%%%%— %%%%%%%%%%%%%%%%%%%%%%%%%%%%%%%%%%%%%%%%%%%%%%% — Examiners’ Report
Therefore \mu x + t = −
1 d
S_x(t)
S_x(t) dt
(
)
If $S_x(t) = exp − ( \lambda t ) \beta$ , therefore, we have
\mu x + t = −
\mu x + t = −
(
1
exp − ( \lambda t ) \beta
)
(
d
exp − ( \lambda t ) \beta
dt
)
( exp ( − ( \lambda t ) ) ) ( −\lambda \beta t ) = \lambda \beta t
exp ( − ( \lambda t ) )
1
\beta
\beta
OR
⎡ t
⎤
S_x(t) = exp ⎢ − \int \mu x + s ds ⎥ = exp ⎡ − ( \lambda t ) \beta ⎤ .
⎣
⎦
⎢ ⎣ 0
⎥ ⎦
So
t
⎤
d ⎡
d
⎢ \int \mu x + s ds ⎥ = \mu x + t = ⎡ ( \lambda t ) \beta ⎤ ,
⎦
dt ⎢
dt ⎣
⎥ ⎦
⎣ 0
and hence
\mu x + t = \beta\lambda \beta t \beta− 1 .
(iv)
Page 14
\beta
\beta− 1
\beta
\beta− 1%%%%%%%%%%%%%%%%%%%%%%%%%%%— %%%%%%%%%%%%%%%%%%%%%%%%%%%%%%%%%%%%%%%%%%%%%%% — Examiners’ Report

\end{document}
