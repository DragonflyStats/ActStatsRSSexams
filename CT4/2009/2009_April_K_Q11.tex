\documentclass[a4paper,12pt]{article}

%%%%%%%%%%%%%%%%%%%%%%%%%%%%%%%%%%%%%%%%%%%%%%%%%%%%%%%%%%%%%%%%%%%%%%%%%%%%%%%%%%%%%%%%%%%%%%%%%%%%%%%%%%%%%%%%%%%%%%%%%%%%%%%%%%%%%%%%%%%%%%%%%%%%%%%%%%%%%%%%%%%%%%%%%%%%%%%%%%%%%%%%%%%%%%%%%%%%%%%%%%%%%%%%%%%%%%%%%%%%%%%%%%%%%%%%%%%%%%%%%%%%%%%%%%%%

\usepackage{eurosym}
\usepackage{vmargin}
\usepackage{amsmath}
\usepackage{graphics}
\usepackage{epsfig}
\usepackage{enumerate}
\usepackage{multicol}
\usepackage{subfigure}
\usepackage{fancyhdr}
\usepackage{listings}
\usepackage{framed}
\usepackage{graphicx}
\usepackage{amsmath}
\usepackage{chngpage}

%\usepackage{bigints}
\usepackage{vmargin}

% left top textwidth textheight headheight

% headsep footheight footskip

\setmargins{2.0cm}{2.5cm}{16 cm}{22cm}{0.5cm}{0cm}{1cm}{1cm}

\renewcommand{\baselinestretch}{1.3}

\setcounter{MaxMatrixCols}{10}

\begin{document}
\begin{enumerate}


%%--- Question 11
An investigation into mortality by cause of death used the four-state Markov model
shown below.
1 Alive
\mu 12
x + t
2 Dead from
heart disease
\item (i)
\mu 14
x + t
\mu 13
x + t
3 Dead from
cancer
4 Dead from
other causes
Show from first principles that
\partial 12
12
11
t p x = \mu x + t t p x .
\partial t

The investigation was carried out separately for each year of age, and the transition
intensities were assumed to be constant within each single year of age.
\item (ii)
(a) Write down, defining all the terms you use, the likelihood for the transition intensities.
(b) Derive the maximum likelihood estimator of the force of mortality
from heart disease for any single year of age.

The investigation produced the following data for persons aged 64 last birthday:
Total waiting time in the state Alive 1,065 person-years
Number of deaths from heart disease
Number of deaths from cancer
Number of deaths from other causes 34
36
42
\item (iii)
(iv)
(a) Calculate the maximum likelihood estimate (MLE) of the force of
mortality from heart disease at age 64 last birthday.
(b) Estimate an approximate 95\% confidence interval for the MLE of the
force of mortality from heart disease at age 64 last birthday.

Discuss how you might use this model to analyse the impact of risk factors on
the death rate from heart disease and suggest, giving reasons, a suitable
alternative model.

[

%%%%%%%%%%%%%%%%%%%%%%%%%%%%%%%%%%%%%%%%%%%%%%%%%%%%%%%%%%%%%%%%%%%%%%%%%%%%%%%%%%%%%%%%%%%%%%%%%%%%%%%%%

11
\item (i)
Condition on the state occupied at t.
We have
t + dt
11
12
12
22
p 12
x = t p x dt p x + t + t p x dt p x + t .
since it is impossible to leave states 3 and 4 once entered.
Also,
dt
p x 22 + t = 1,
since state 2 is an absorbing state.
We now assume that, for small dt,
dt
12
p 12
x + t = \mu x + t dt + o ( dt )
where o(dt) is the probability that a life makes two or more transitions in the
time interval dt, and
o ( dt )
= 0 .
dt → 0 dt
lim
Substituting for
t + dt
dt
p 12
x + t gives
12
11
12
p 12
x = \mu x + t t p x dt + t p x + o ( dt )
Thus
t + dt
12
12
11
p 12
x − t p x = \mu x + t t p x dt + o ( dt )
and
\partial 12
t p x = lim +
\partial t
dt → 0
\item (ii)
(a)
t + dt
12
p 12
11
x − t p x
= \mu 12
x + t t p x
dt
Suppose we observe d 12 deaths from heart disease, d 13 deaths from
cancer and d 14 deaths from other causes.
Suppose also that we observe the waiting time for each life, and that
the total observed waiting time is V, being the sum of the waiting times
for each life.
15%%%%%%%%%%%%%%%%%%%%%%%%%%%— %%%%%%%%%%%%%%%%%%%%%%%%%%%%%%%%%%%%%%%%%%%%%%% — Examiners’ Report
Then the likelihood of the data is given by
(
)
L ∝ exp ⎡ − \mu 12 + \mu 13 + \mu 14 V ⎤ ( \mu 12 ) d ( \mu 13 ) d ( \mu 14 ) d .
⎣
⎦
(b)
12
13
14
The maximum likelihood estimator of \mu 12 is obtained by
differentiating this expression (or its logarithm) with respect to \mu 12
and setting the derivative equal to zero.
Taking logarithms produces
log L = − ( \mu 12 + \mu 13 + \mu 14 ) V + d 12 log \mu 12 + d 13 log \mu 13 + d 14 log \mu 14 + K
(where K is a constant )
Partially differentiating this with respect to \mu 12 leads to
\partial log L
\partial\mu 12
= − V +
d 12
\mu 12
,
and setting the partial derivative equal to zero leads to the solution
\mu ˆ 12 =
d 12
.
V
\partial 2 log L
=−
d 12
, the second derivative is always negative
( \partial\mu 12 ) 2
( \mu 12 ) 2
and so we have a maximum.
Since
\item (iii)
(a) The maximum likelihood estimate of the force of mortality from heart disease is 34/1,065 = 0.0319249
(b) The variance of the maximum likelihood estimator of \mu 12 is
asymptotically
\mu 12
, where E[V] is the expected waiting time in the
E [ V ]
state “alive” and \mu 12 is the “true” population value of the force of
mortality from heart disease.
This may be approximated by using the observed force of mortality
and the observed waiting time, so that an estimate of the variance is
0.0319249
= 0.000029976 .
1, 065
16%%%%%%%%%%%%%%%%%%%%%%%%%%%— %%%%%%%%%%%%%%%%%%%%%%%%%%%%%%%%%%%%%%%%%%%%%%% — Examiners’ Report
The estimated standard error is therefore
0.000029976 = 0.00547507 .
The 95\% confidence interval is therefore
0.0319249 \pm (1.96)0.00547507 = 0.0319249 \pm 0.0107311
= (0.0212, 0.0427).
(iv)
Using the four state model, the lives in the investigation would have to be stratified according to the risk factors and the transition intensities estimated separately for each stratum.
This is likely to run into problems of small numbers.
Using a Cox regression model with death from heart disease as the event of interest and the risk factors as covariates would avoid this problem.
Lives who died from other causes could be treated as censored at the durations when they died.
\end{document}
