\documentclass[a4paper,12pt]{article}

%%%%%%%%%%%%%%%%%%%%%%%%%%%%%%%%%%%%%%%%%%%%%%%%%%%%%%%%%%%%%%%%%%%%%%%%%%%%%%%%%%%%%%%%%%%%%%%%%%%%%%%%%%%%%%%%%%%%%%%%%%%%%%%%%%%%%%%%%%%%%%%%%%%%%%%%%%%%%%%%%%%%%%%%%%%%%%%%%%%%%%%%%%%%%%%%%%%%%%%%%%%%%%%%%%%%%%%%%%%%%%%%%%%%%%%%%%%%%%%%%%%%%%%%%%%%

\usepackage{eurosym}
\usepackage{vmargin}
\usepackage{amsmath}
\usepackage{graphics}
\usepackage{epsfig}
\usepackage{enumerate}
\usepackage{multicol}
\usepackage{subfigure}
\usepackage{fancyhdr}
\usepackage{listings}
\usepackage{framed}
\usepackage{graphicx}
\usepackage{amsmath}
\usepackage{chngpage}

%\usepackage{bigints}
\usepackage{vmargin}

% left top textwidth textheight headheight

% headsep footheight footskip

\setmargins{2.0cm}{2.5cm}{16 cm}{22cm}{0.5cm}{0cm}{1cm}{1cm}

\renewcommand{\baselinestretch}{1.3}

\setcounter{MaxMatrixCols}{10}

\begin{document}

%%CT4 S2009—6
%%-Qeustion 10
An investigation into the mortality of men engaged in a hazardous occupation was
carried out. The following is an extract from the results.
\begin{verbatim}
Age x Initial
exposed-to-risk E x
30
31
32
33
34
35
36
37
38 950
1,200
1,200
900
1,000
1,100
800
1,250
1,400
Observed
deaths θ x
12
14
16
9
11
15
10
16
17
q ˆ x
0.0126
0.0117
0.0133
0.0100
0.0110
0.0136
0.0125
0.0128
0.0121
\end{verbatim}
It was decided to graduate the results with reference to English Life Table 15 (males).
o
The formula used for the graduation was q x = 10 q x s .
\begin{enumerate}
\item (i) Using a test of the overall fit of the graduated rates to the data, test the
hypothesis that the underlying mortality of men in the hazardous occupation is
in accordance with the graduation formula given above.
[6]
\item (ii) Test the graduation using two other tests which detect different features of the
graduation. For each test you apply:
(a)
(b)
(c)
State the feature of the graduation it is designed to detect.
Carry out the test.
State your conclusion.
\end[enumerate}


%%%%%%%%%%%%%%%%%%%%%%%%%%%%%%%%%%%%%%%%%%%%%%%%%%%%%%%%%%%%%%%%%%%%%%%%%%%%%%%%%%%
\newpage

10
\begin{itemize}
\item (i) The chi-squared test is for the overall fit of the graduated rates to the data
The test statistic is
 z x 2 , where
o
z x 
 x  E x q x
o
o
.
E x q x (1  q x )
o
\item The calculations are shown in the table below (since q x is
o
small we use the approximation z x 
 x  E x q x
o
.
E x q x
\item % Page 13
 x o o
Age x q x E x q x
30 12 0.0091 31 14 0.0094 32 16 33
8.645
z x
z x 2
1.141 1.302
11.28 0.810 0.656
0.0097 11.64 1.278 1.633
9 0.0099 8.91 0.030 0.001
34 11 0.0106 10.60 0.123 0.015
35 15 0.0116 12.76 0.627 0.393
36 10 0.0127 10.16 -0.050 0.003
37 16 0.0138 17.25 -0.301 0.091  — %%%%%%%%%%%%%%%%%%%%%%%%%%%%%%%%%%%%%%%%%%%% — Examiners’ Report
38
17
0.0149
20.86
-0.845
0.714
∑
4.808
\item The test statistic has a chi-squared distribution with degrees of freedom (d.f.)
given by number of ages
o
– 1 (for parameter of function linking q x and q x s )
– some d.f. for constraints imposed by choice of standard table
\item The critical value of the chi-squared distribution is
11.07 with 5 d.f.
12.59 with 6 d.f.
14.07 with 7 d.f.
15.51 with 8 d.f.
16.92 with 9 d.f. at the 5\% level (from tables)
\item Since 4.808 < 11.07 (or 12.59 etc.) there is no evidence to reject the null
hypothesis that the graduated rates are the true rates underlying the crude
rates.
\item (ii)
EITHER
Signs test
a. The Signs test looks for overall bias.
b. The number of positive signs among the z x s
is distributed Binomial (9, 0.5).
\item We observe 6 positive signs.
The probability of obtaining 6 or more positive signs is
(from tables)
1 – 0.7461 = 0.2539.
\item [Alternatively, candidates could calculate the probability of obtaining exactly 6
positive signs, which is 0.1641]
\item Since this is greater than 0.025 (two-tailed test)
c. we cannot reject the null hypothesis and we conclude that the
graduated rates are not systematically higher or lower than the crude
rates.
OR
Cumulative Deviations test
Page 14  — %%%%%%%%%%%%%%%%%%%%%%%%%%%%%%%%%%%%%%%%%%%% — Examiners’ Report
a. When applied over the whole age range, the Cumulative Deviations
test looks for overall bias
b. The test statistic is

o

    x  E x q x  
x
o
 Normal(0,1)
 E x q x
x
o
o
Age x  x 30 12 31 14 11.28 2.72
32 16 11.64 4.36
33 9 8.91 0.09
34 11 10.60 0.40
35 15 12.76 2.24
36 10 10.16 -0.16
37 16 17.25 -1.25
38 17 20.86 -3.86
∑ 112.105 7.895
E x q x
8.645
So the value of the test statistic is
 x  E x q x
3.355
7.895
 0.7457
112.105
\item Using a 5\% level of significance, we see that
1.96 < 0.7457 < 1.96
c. We accept the null hypothesis at the 5\% level of significance and
conclude there is no overall bias in the graduation.
\item Grouping of Signs test
a. The Grouping of Signs test looks for runs or clumps of deviations of
the same sign OR the grouping of signs test tests for overgraduation.
b. We have:
9 ages in total
6 positive deviations
3 negative deviations
We have 1 positive run
Pr[1 positive run] is therefore equal to
Page 15  — %%%%%%%%%%%%%%%%%%%%%%%%%%%%%%%%%%%%%%%%%%%% — Examiners’ Report
 5  4 
  
4
4
 0   1  

 0.0476
 9 
 9.8.7  84


 
 3.2 
 6 
\item Since this is less than 0.05 (using a one-tailed
test)
\ned{itemize}
\begin{itemize}
\item c. We reject the null hypothesis that the graduated rates are the true rates
underlying the crude rates (OR we conclude that the graduation is
unsatisfactory OR there is evidence of over-graduation).
Individual Standardised Deviations test
\item a. The Individual Standardised Deviations tests looks for individual large
deviations at particular ages.
\item b. If the graduated rates were the true rates underlying the observed rates
we would expect the individual deviations to be distributed Normal (0,1) and therefore only 1 in 20 z x s should have absolute magnitudes
greater than 1.96. Looking at the z x s we see that the largest individual
deviation is 1.278. Since this is less in absolute magnitude than 1.96
\item c. we cannot reject the null hypothesis that the graduated rates are the
true rates underlying the crude rates.
Answers to this question were disappointing compared with previous years. A
common error was for candidates to misread the question and to try to compare the
observed number of deaths with an ‘expected’ number computed on the basis of the
^
q x given in the question. These candidates were, in effect, examining deviations
based solely on rounding! 
\item Candidates who made this error were penalised in part (i),
but could gain credit for some of the alternative tests in part \item (ii) provided that they
performed the tests correctly.
\end{itemize}
\end{document}
