\documentclass[a4paper,12pt]{article}

%%%%%%%%%%%%%%%%%%%%%%%%%%%%%%%%%%%%%%%%%%%%%%%%%%%%%%%%%%%%%%%%%%%%%%%%%%%%%%%%%%%%%%%%%%%%%%%%%%%%%%%%%%%%%%%%%%%%%%%%%%%%%%%%%%%%%%%%%%%%%%%%%%%%%%%%%%%%%%%%%%%%%%%%%%%%%%%%%%%%%%%%%%%%%%%%%%%%%%%%%%%%%%%%%%%%%%%%%%%%%%%%%%%%%%%%%%%%%%%%%%%%%%%%%%%%

\usepackage{eurosym}
\usepackage{vmargin}
\usepackage{amsmath}
\usepackage{graphics}
\usepackage{epsfig}
\usepackage{enumerate}
\usepackage{multicol}
\usepackage{subfigure}
\usepackage{fancyhdr}
\usepackage{listings}
\usepackage{framed}
\usepackage{graphicx}
\usepackage{amsmath}
\usepackage{chngpage}

%\usepackage{bigints}
\usepackage{vmargin}

% left top textwidth textheight headheight

% headsep footheight footskip

\setmargins{2.0cm}{2.5cm}{16 cm}{22cm}{0.5cm}{0cm}{1cm}{1cm}

\renewcommand{\baselinestretch}{1.3}

\setcounter{MaxMatrixCols}{10}

\begin{document}


6 An investigation by a hospital into rates of recovery after a specific type of operation
collected the following data for each month of the calendar year 2008:
•
number of persons who recovered from the operation during the month (defined as
being discharged from the hospital) classified by the month of their operation.
You may assume that there were no deaths.
On the first day of each month from January 2008 to January 2009, the hospital listed
all in-patients who were yet to recover from this operation, classified according to the
length of time elapsing since their operation, to the nearest month.
\begin{enumerate}[(i)]
\item (i)
7
(a) Write down an expression which will enable the hospital to calculate
rates of recovery, r x , during 2008 at various durations x since the
operation using the available data.
(b) Derive a formula for the exposed to risk based on the information in
the hospital’s monthly lists of in-patients which corresponds to the data
on recovery from the operation.

\item (ii) Determine the value of f such that the expression in \item (i)(a) applies to an actual
duration x + f since the operation.
\end{enumerate}


%%%%%%%%%%%%%%%%%%%%%
6
(i)
(a)
The relevant recovery rates can be estimated as
r x =
d x
E x c
, x = 0, 1, 2, ... months
where d x is the number of persons recovering in the calendar month
that was x months after the calendar month of their operation, and E x c is
the central exposed to risk.
(b)
We need to ensure that the E x c correspond to the data on persons
recovering
The hospital’s data imply a calendar month rate interval for the
recoveries, running from the first day of each month until the last day
of each month.
Using the monthly “census” data, a definition of E x c which corresponds
to the deaths data can be obtained as follows.
We observe P x , t = number of lives under observation for whom the time elapsing since the operation was between x − 1⁄2 and x + 1⁄2
months, where t is the time in months since 1 January 2008.



Therefore, using the census formula:
E x c =
12 11
0 0
∫ P * x , t dt = ∑ 1
2 (
P * x , t + P * x + 1, t + 1 ) ,
where P * x , t = 1 ( P x − 1, t + P x , t ) .
2
We assume all months are the same length, and that the numbers in the hospital vary linearly across each month.
(ii)
At the start of the rate interval, durations since the operation range from x − 1
to x months, so the average duration is x − 1⁄2, assuming operations take place
evenly across the month.
r x estimates the recovery rate at the mid-point of the rate interval .
This is exactly x months since the operation, so f = 0.
%%%%%%%%%%%%%%%%%%%%%%%%%%%%%%%%%%%%%%%%%%%%%%%%%%%%%%%%%%%%%%%%%%%%%%%%%%%%%
\end{document}
