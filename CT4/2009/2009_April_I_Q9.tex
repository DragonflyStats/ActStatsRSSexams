\documentclass[a4paper,12pt]{article}

%%%%%%%%%%%%%%%%%%%%%%%%%%%%%%%%%%%%%%%%%%%%%%%%%%%%%%%%%%%%%%%%%%%%%%%%%%%%%%%%%%%%%%%%%%%%%%%%%%%%%%%%%%%%%%%%%%%%%%%%%%%%%%%%%%%%%%%%%%%%%%%%%%%%%%%%%%%%%%%%%%%%%%%%%%%%%%%%%%%%%%%%%%%%%%%%%%%%%%%%%%%%%%%%%%%%%%%%%%%%%%%%%%%%%%%%%%%%%%%%%%%%%%%%%%%%

\usepackage{eurosym}
\usepackage{vmargin}
\usepackage{amsmath}
\usepackage{graphics}
\usepackage{epsfig}
\usepackage{enumerate}
\usepackage{multicol}
\usepackage{subfigure}
\usepackage{fancyhdr}
\usepackage{listings}
\usepackage{framed}
\usepackage{graphicx}
\usepackage{amsmath}
\usepackage{chngpage}

%\usepackage{bigints}
\usepackage{vmargin}

% left top textwidth textheight headheight

% headsep footheight footskip

\setmargins{2.0cm}{2.5cm}{16 cm}{22cm}{0.5cm}{0cm}{1cm}{1cm}

\renewcommand{\baselinestretch}{1.3}

\setcounter{MaxMatrixCols}{10}

\begin{document}
\begin{enumerate}


%%%%%%%%%%%%%%%%%%%%%%%%%%%%%%%%

\item (i) Prove that, under Gompertz’s Law, the probability of survival from age x to
age x + t, t p x , is given by:
⎡
⎛ − B ⎞ ⎤
t p x = ⎢ exp ⎜
⎟ ⎥
⎝ ln c ⎠ ⎦
⎣

c x ( c t − 1)
.

For a certain population, estimates of survival probabilities are available as follows:
= 0.995
2 p 50 = 0.989 .
1 p 50
\item (ii) Calculate values of B and c consistent with these observations.
\item (iii) Comment on the calculation performed in \item (ii) compared with the usual process
for estimating the parameters from a set of crude mortality rates.



%%%%%%%%%%%%%%%%%%%%%%%%%%%%%%%%%%%%%%%%%%%%%%%%%%%%%%%%%%%%%%%%%%%%%%%%%%%%


9
\item (i)
Under Gompertz’s Law
\mu x = Bc x .
Since
⎛ t
⎞
⎜
⎟ ,
=
−
\mu
p
exp
dw
t x
⎜ \int x + w ⎟
⎝ 0
⎠
⎛
⎛ t
⎞
Bc x c w
we have t p x = exp ⎜ − \int Bc x + w dw ⎟ = exp ⎜ −
⎜
⎟
⎜
ln c
⎝ 0
⎠
⎝
⎛ ⎡ Bc x c t − Bc x ⎤ ⎞
⎦ ⎟ =
which is exp ⎜ − ⎣
⎜
⎟
ln c
⎝
⎠
\item (ii)
⎡
⎛ − B ⎞ ⎤
Define Q = ⎢ exp ⎜
⎟ ⎥
⎝ ln c ⎠ ⎦
⎣
⎡
⎛ − B ⎞ ⎤
⎢ exp ⎜ ln c ⎟ ⎥
⎝
⎠ ⎦
⎣
⎞
⎟ ,
⎟
0 ⎠
t
c x ( c t − 1)
.
c 50
ln 0.995 = (c − 1) ln Q
ln 0.989 = (c 2 − 1) ln Q
( c 2 − 1) ( c − 1)( c + 1)
=
= 2.20665
( c − 1)
( c − 1)
c = 1.20665
Therefore Q = 0.976036128
1.20665 50
⎡
− B
⎛
⎞ ⎤
⎢ exp ⎜ ln1.20665 ⎟ ⎥
⎝
⎠ ⎦
⎣
= 0.976036128
B = 3.797*10 − 7 .
\item (iii)
In this example, only two observations are provided so there is an analytical solution to the Gompertz model.
This is unrealistic as in general a graduation process would be used to provide a fit to a set of crude rates.
This could be done by weighted least squares or maximum likelihood.

\newpage
Page 12%%%%%%%%%%%%%%%%%%%%%%%%%%%— %%%%%%%%%%%%%%%%%%%%%%%%%%%%%%%%%%%%%%%%%%%%%%% — Examiners’ Report
The more general graduation process allows the fitting of more complex
models from the Gompertz-Makeham family which have the form
\[\mu x = polynomial(1) + exp(polynomial(2))\]
the parameters of which cannot always so easily be estimated by the method
used in part \item (ii).

\end{document}
