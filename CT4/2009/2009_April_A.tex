\documentclass[a4paper,12pt]{article}

%%%%%%%%%%%%%%%%%%%%%%%%%%%%%%%%%%%%%%%%%%%%%%%%%%%%%%%%%%%%%%%%%%%%%%%%%%%%%%%%%%%%%%%%%%%%%%%%%%%%%%%%%%%%%%%%%%%%%%%%%%%%%%%%%%%%%%%%%%%%%%%%%%%%%%%%%%%%%%%%%%%%%%%%%%%%%%%%%%%%%%%%%%%%%%%%%%%%%%%%%%%%%%%%%%%%%%%%%%%%%%%%%%%%%%%%%%%%%%%%%%%%%%%%%%%%

\usepackage{eurosym}
\usepackage{vmargin}
\usepackage{amsmath}
\usepackage{graphics}
\usepackage{epsfig}
\usepackage{enumerate}
\usepackage{multicol}
\usepackage{subfigure}
\usepackage{fancyhdr}
\usepackage{listings}
\usepackage{framed}
\usepackage{graphicx}
\usepackage{amsmath}
\usepackage{chngpage}

%\usepackage{bigints}
\usepackage{vmargin}

% left top textwidth textheight headheight

% headsep footheight footskip

\setmargins{2.0cm}{2.5cm}{16 cm}{22cm}{0.5cm}{0cm}{1cm}{1cm}

\renewcommand{\baselinestretch}{1.3}

\setcounter{MaxMatrixCols}{10}

\begin{document}
\begin{enumerate}

\item % Question 1
Graduation by reference to a standard table might be appropriate, if a suitable standard table could be found.
However the fact that the company insures non-standard lives makes it unlikely that a suitable standard table would exist.
Graphical graduation might be used if no suitable standard table can be found.
However it is a last resort as it is difficult to obtain results which are smooth and which adhere to the data.
Graduation using a parametric formula is unlikely to be appropriate as the amount of data in this investigation is likely to be small and it is unlikely that the company will
want to produce a standard table.
2
\item (i)
A Markov chain is a stochastic process with discrete states operating in discrete time in which the probabilities of moving from one state to another
are dependent only on the present state of the process.
EITHER
If the transition probabilities are also independent of time.
OR
If the l-step transition probabilities are dependent only on the time lag, the
chain is said to be time-homogeneous.
\item (ii)
(a)
In this case the chain is irreducible if the transition probability out of each state is non-zero (or, equivalently, if it is possible to
reach the other state from both states)

So requires $0 < a \leq 1 and 0 < b \leq 1$
(b)
The chain is only periodic if the chain must alternate between
the states.
So a = 1 and b = 1.
4%%%%%%%%%%%%%%%%%%%%%%%%%%%— %%%%%%%%%%%%%%%%%%%%%%%%%%%%%%%%%%%%%%%%%%%%%%% — Examiners’ Report
3
Benefits
Complex systems with stochastic elements can be studied.
Different future policies or possible actions can be compared.
In models of complex systems we can control the experimental conditions and thus
reduce the variance of the results without upsetting their mean values.
Can calibrate to observed data and hence model interdependencies between
outcomes.
Often models are the only practicable means of answering actuarial questions.
Systems with a long time-frame can be studied and results obtained relatively quickly.
Limitations
Time or cost or resources required for model development.
In a stochastic model, many independent runs of the model are needed to obtain
results for a given set of inputs.
Models can look impressive and there is a danger this results in false sense of
confidence.
Poor or incredible data input or assumptions will lead to flawed output.
Users need to understand the model and the uses to which it can safely be put — the
model is not a “black box”.
It is not possible to include all future events in a model (e.g. change in legislation).
Interpreting the results can be a challenge.
Any model will be an approximation.
Models are better for comparing the impact of input variations than for optimising
outputs.
5%%%%%%%%%%%%%%%%%%%%%%%%%%%— %%%%%%%%%%%%%%%%%%%%%%%%%%%%%%%%%%%%%%%%%%%%%%% — Examiners’ Report
4
\item (i)
(a)
Under UDD the number of deaths between exact ages 30 and 35 years
is half the number of deaths between exact ages 30 and 40 years.
So the number of deaths between exact ages 30 and 35 years is
1⁄2(98,617 – 97,952) = 332.5
and 5 q 30 =
(b)
332.5
= 0.0033716 .
98, 617
Let the constant force of mortality be \mu.
⎛ t
⎞
Then, since t p x = exp ⎜ − \int \mu x + s ds ⎟ ,
⎜
⎟
⎝ 0
⎠
10 p 30 = exp ( − 10 \mu )
so
\mu=
− log e ( 10 p 30 )
5 q 30
10
=
− log e ( 97,952 / 98, 617 )
10
= 0.0006766 .
= 1 − 5 p 30 = 1 − exp( − 5 \mu )
= 1 − exp[( − 5)(0.0006766)] = 0.0033773 .
\item (ii)
EITHER
The number of survivors to exact age 35 years is
98, 617 5 p 30 = 98, 617(1 − 5 q 30 ) ,
so for UDD this is
98, 617(1 − 0.0033716) = 98, 284.5 ,
and under a constant force of mortality this is
98, 617(1 − 0.0033773) = 98, 283.9 .
OR
Under UDD the number of survivors to exact age 35 years is
(98,617 + 97,952)/2 = 98,284.5.
6%%%%%%%%%%%%%%%%%%%%%%%%%%%— %%%%%%%%%%%%%%%%%%%%%%%%%%%%%%%%%%%%%%%%%%%%%%% — Examiners’ Report
Under a constant force of mortality the number of survivors to
exact age 35 years is given by
98, 617 *97,952 = 98, 283.9
\item (iii)
The actual number of survivors to exact age 35 years is higher (or,
equivalently, mortality is lighter) than that under either the UDD or the
constant force assumptions.
The actual number of survivors implies that there were 258 deaths between
ages 30 and 35 years and 407 deaths between ages 35 and 40 years.
The actual data reveal that the force of mortality is higher between ages 35 and
40 years than it is between ages 30 and 35 years for females in English Life
Table 15, which suggests that the force of mortality is increasing over this age
range.
The assumption of UDD implies an increasing force of mortality.
The actual force of mortality seems to be increasing even faster than is implied
by UDD.
A constant force of mortality is unlikely to be realistic for this age range.
Used over a 10-year age span the assumption of UDD is unlikely to be
appropriate, whereas used over single years of age it is acceptable.


%%%%%%%%%%%%%%%%%%%%%%%%%%%%%%%%%%%%%%%%%%%%%%%%%%%%%
\newpage

© Institute of Actuaries1
A life insurance company has a small group of policies written on impaired lives and
has conducted an investigation into the mortality of these policyholders. It is
proposed that the crude mortality rates be graduated for use in future premium
calculations.
Discuss the suitability of two methods of graduation that the insurance company could
use.

2
\item (i)
Explain what is meant by a time-homogeneous Markov chain.

Consider the time-homogeneous two-state Markov chain with transition matrix:
a ⎞
⎛ 1 − a
⎜
⎟
⎝ b 1 − b ⎠
\item (ii)
Explain the range of values that a and b can take which result in this being a
valid Markov chain which is:
(a)
(b)
irreducible
periodic

[Total 5]
3 List the benefits and limitations of modelling in actuarial work.
4 Below is an extract from English Life Table 15 (females).
\item (i)
Number of survivors to
exact age x out of
100,000 births
30
40 98,617
97,952
Calculate 5 q 30 under each of the two following alternative assumptions:
(a)
(b)
\item (ii)
Age x
(years)

a uniform distribution of deaths (UDD) between ages 30 and 40 years
a constant force of mortality between ages 30 and 40 years

Calculate the number of survivors to exact age 35 years out of 100,000 births
under each of the assumptions in \item (i) above.

English Life Table 15 (females) was originally calculated using data classified by
single years of age. The number of survivors to exact age 35 years was 98,359.
\item (iii)
Comment on the appropriateness of the assumptions of UDD and a constant
force of mortality between ages 30 and 40 years in this example.

[Total 7]
CT4 A2009—25 Explain the basis underlying the grouping of signs test, and derive the formula for the
probability of exactly t positive groups by considering the possible arrangements of a
set of positive and negative signs.
