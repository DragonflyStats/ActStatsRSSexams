\documentclass[a4paper,12pt]{article}

%%%%%%%%%%%%%%%%%%%%%%%%%%%%%%%%%%%%%%%%%%%%%%%%%%%%%%%%%%%%%%%%%%%%%%%%%%%%%%%%%%%%%%%%%%%%%%%%%%%%%%%%%%%%%%%%%%%%%%%%%%%%%%%%%%%%%%%%%%%%%%%%%%%%%%%%%%%%%%%%%%%%%%%%%%%%%%%%%%%%%%%%%%%%%%%%%%%%%%%%%%%%%%%%%%%%%%%%%%%%%%%%%%%%%%%%%%%%%%%%%%%%%%%%%%%%

\usepackage{eurosym}
\usepackage{vmargin}
\usepackage{amsmath}
\usepackage{graphics}
\usepackage{epsfig}
\usepackage{enumerate}
\usepackage{multicol}
\usepackage{subfigure}
\usepackage{fancyhdr}
\usepackage{listings}
\usepackage{framed}
\usepackage{graphicx}
\usepackage{amsmath}
\usepackage{chngpage}

%\usepackage{bigints}
\usepackage{vmargin}

% left top textwidth textheight headheight

% headsep footheight footskip

\setmargins{2.0cm}{2.5cm}{16 cm}{22cm}{0.5cm}{0cm}{1cm}{1cm}

\renewcommand{\baselinestretch}{1.3}

\setcounter{MaxMatrixCols}{10}

\begin{document}


%%--- Question 9
 An electronics company developed a revolutionary new battery which it believed
would make it enormous profits. It commissioned a sub-contractor to estimate the
survival function of battery life for the first 12 prototypes. The sub-contractor
inserted each prototype battery into an identical electrical device at the same time and
measured the duration elapsing between the time each device was switched on and the time its battery ran out. The sub-contractor was instructed to terminate the test
immediately after the failure of the 8th battery, and to return all 12 batteries to the
company.
When the test was complete, the sub-contractor reported that he had terminated the
test after 150 days. He further reported that:
\begin{itemize}
    \item two batteries had failed after 97 days
\item three further batteries had failed after 120 days
\item two further batteries had failed after 141 days
\item one further battery had failed after 150 days

\end{itemize}
However, he reported that he was only able to return 11 batteries, as one had exploded
after 110 days, and he had treated this battery as censored at that duration when
working out the Kaplan-Meier estimate of the survival function.
\begin{enumerate}
\item (i) State, with reasons, the forms of censoring present in this study.

\item (ii) Calculate the Kaplan-Meier estimate of the survival function based on the
information supplied by the sub-contractor.

In his report, the sub-contractor claimed that the Kaplan-Meier estimate of the
survival function at the duration when the investigation was terminated was 0.2727.
\item (iii)
Explain why the sub-contractor’s Kaplan-Meier estimate would be consistent
with him having stolen the battery he claimed had exploded.
\end{enumerate}
\newpage
%%%%%%%%%%%%%%%%%%%%%%%%%%%%%%%%%%%%%%
9
\begin{itemize]
\item 
(i) Type II censoring as the study was terminated after a pre-determined number
of failures. Random censoring of the device which exploded.
\item (ii) According to the information supplied by the sub-contractor, the Kaplan-
Meier estimate of the survival function should be calculated as follows:
Page 11
j t j N j
0 0 12
1 97 12
d j c j d j /N j 1 – d j /N j
2 1 2/12 10/12  — %%%%%%%%%%%%%%%%%%%%%%%%%%%%%%%%%%%%%%%%%%%% — Examiners’ Report
2 120 9 3 0 3/9 6/9
3 141 6 2 0 2/6 4/6
4 150 4 1 3 1/4 3/4
The Kaplan-Meier estimate is then
 d j
S ˆ ( t )    1 
 N j
t j \leqt 




so we have
\begin{center}
\begin{tabular}{cc}
t & \hat{S} ( t )\\
0 \leq t < 97 &  1\\
97 \leq t < 120 & 5/6\\
120 \leq t < 141 & 5/9\\
141 \leq t < 150 & 10/27\\
150 \leq t & 5/18 = 0.2778\\
\end{tabular}
\end{center}
\item (iii)Since 5/18 is not equal to 0.2727, the sub-contractor‟s story is internally
inconsistent. The Kaplan-Meier estimate of the survival function after the
failure of the 8th battery of 0.2727 would be obtained had only 11 batteries
been tested at the start, and no battery being censored, as shown in the
following table.
\begin{verbatim}
d j c j d j /N j 1 – d j /N j
11 2 0 2/11 9/11
120 9 3 0 3/9 6/9
3 141 6 2 0 2/6 4/6
4 150 4 1 0 1/4 3/4
+1⁄2 +1⁄2
j t j N j
0 0 11
1 97 2
\end{verbatim}
The Kaplan-Meier estimate is then
 d j
S ˆ ( t )    1 
 N j
t j \leqt 




so we have
t
%%Page 12
S ˆ ( t )  — %%%%%%%%%%%%%%%%%%%%%%%%%%%%%%%%%%%%%%%%%%%% — Examiners’ Report
\begin{verbatim}
0 \leqt < 97  1
97 \leqt < 120 9/11
120 \leqt < 141 6/11
141 \leqt < 150 4/11
150 \leqt 3/11 = 0.2727
\end{verbatim}
Therefore the value of S ˆ (150) reported by the sub-contractor is consistent with
him having stolen the last battery.

\item Many candidates scored highly on this question. Credit was given in part (i) for other
types of censoring provided that a sensible reason was given. In part \item (iii), for full
credit some kind of calculation of an alternative survival function was needed,
together with an explanation of why this provided evidence to support the suggestion
that the sub-contractor has stolen the battery.
\end{itemzie}

\end{document}
