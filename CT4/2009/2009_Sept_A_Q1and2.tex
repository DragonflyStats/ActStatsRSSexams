
Skip to content
Pull requests
Issues
Marketplace
Explore
@DragonflyStats

1
0

    0

DragonflyStats/ActStatsRSSexams
Code
Issues 0
Pull requests 0
Projects 0
Wiki
Insights
Settings
ActStatsRSSexams/CT4/2009/2009_Sept_A.tex
@ShinyMCS ShinyMCS corrections 2 2b53cfb on Mar 24
@ShinyMCS
@DragonflyStats
220 lines (202 sloc) 7.6 KB
\documentclass[a4paper,12pt]{article}

%%%%%%%%%%%%%%%%%%%%%%%%%%%%%%%%%%%%%%%%%%%%%%%%%%%%%%%%%%%%%%%%%%%%%%%%%%%%%%%%%%%%%%%%%%%%%%%%%%%%%%%%%%%%%%%%%%%%%%%%%%%%%%%%%%%%%%%%%%%%%%%%%%%%%%%%%%%%%%%%%%%%%%%%%%%%%%%%%%%%%%%%%%%%%%%%%%%%%%%%%%%%%%%%%%%%%%%%%%%%%%%%%%%%%%%%%%%%%%%%%%%%%%%%%%%%

\usepackage{eurosym}
\usepackage{vmargin}
\usepackage{amsmath}
\usepackage{graphics}
\usepackage{epsfig}
\usepackage{enumerate}
\usepackage{multicol}
\usepackage{subfigure}
\usepackage{fancyhdr}
\usepackage{listings}
\usepackage{framed}
\usepackage{graphicx}
\usepackage{amsmath}
\usepackage{chngpage}

%\usepackage{bigints}
\usepackage{vmargin}

% left top textwidth textheight headheight

% headsep footheight footskip

\setmargins{2.0cm}{2.5cm}{16 cm}{22cm}{0.5cm}{0cm}{1cm}{1cm}

\renewcommand{\baselinestretch}{1.3}

\setcounter{MaxMatrixCols}{10}

\begin{document}
\begin{enumerate}
© Institute of Actuaries1
Describe the difference between the following assumptions about mortality between
any two ages, x and y (y > x):
\item
\item
uniform distribution of deaths
constant force of mortality
In your answer, explain the shape of the survival function between ages x and y under
each of the two assumptions.

2
(i)
List the key steps in constructing a new actuarial model.

You work for an actuarial consultancy which is taking over responsibility for a
modelling process which has previously been conducted in house by a client.
3
\item (ii) Discuss the extent to which the steps required for this task differ from those
listed in your answer to (i).



%%%%%%%%%%%%%%%%%%%%%%%%%%%%%%%%%%%%
1
A uniform distribution of deaths means
EITHER
that deaths are evenly spaced between the ages x and y.
OR
that t q x  tq x
( t  y  x )
OR
that t p x  x  t is constant for t  y  x .
It also means that the survival function decreases linearly between ages x and y. The
assumption of a constant force of mortality between any two ages means
EITHER
that the hazard does not change with age over this age range.
OR
that t p x  ( p x ) t .
This implies that the survival function decreases exponentially between ages x and y.
Answers to this straightforward bookwork question were disappointing. Although
most candidates could describe the difference between a constant force of mortality
and the increasing force implied by a uniform distribution of deaths, few made correct reference to the form of the survival function. An alarming number of candidates
referred to survival functions which increased with age! Credit was given for graphs which correctly depicted the shape of the survival function under the two
assumptions.
%%%%%%%%%%%%%%%%%%%%%
\newpage
2
(i) Define objectives of modelling process.
Plan the modelling process and how it will be validated.
Collect and validate the data required.
2  — %%%%%%%%%%%%%%%%%%%%%%%%%%%%%%%%%%%%%%%%%%%% — Examiners’ Report
\begin{itemize}
\item Define the form of the model.
\item Involve experts on the real world system/get feedback on validity.
\item Decide on software to be used, choose random number generator etc.
\item Write the computer program.
\item Debug the program.
\item Analyse the output
\item Test the reasonableness of the output.
\item Consider appropriateness of response of the model to small changes in input
parameters.
\item Communicate and document results.
\end{itemize}
%%[1⁄2 mark was awarded for each point up to a maximum of 4 marks]

\item (ii) Whilst in theory all steps are still required, some may take the form of
reviewing the appropriateness of existing decisions made, such as how the
form of the model was determined.
Extent of work will depend on whether the existing model is to be used,
adapted or superseded.
An understanding of how results compare with those previously used by the
company will be required.
Process maps for the existing approach, or discussions with the people running the process about what they do, may be helpful.
The scope needs to be tightly defined up front to ensure it is clear what is expected of the consultancy.
Data sources may already be established.
[1⁄2 mark was awarded for each point up to a maximum of 2 marks]
Part (i) of this question was basic bookwork and was extremely well
answered. part (ii) required more thought, but many candidates were able to
write down some relevant points.

\end{document}
