\documentclass[a4paper,12pt]{article}

%%%%%%%%%%%%%%%%%%%%%%%%%%%%%%%%%%%%%%%%%%%%%%%%%%%%%%%%%%%%%%%%%%%%%%%%%%%%%%%%%%%%%%%%%%%%%%%%%%%%%%%%%%%%%%%%%%%%%%%%%%%%%%%%%%%%%%%%%%%%%%%%%%%%%%%%%%%%%%%%%%%%%%%%%%%%%%%%%%%%%%%%%%%%%%%%%%%%%%%%%%%%%%%%%%%%%%%%%%%%%%%%%%%%%%%%%%%%%%%%%%%%%%%%%%%%

\usepackage{eurosym}
\usepackage{vmargin}
\usepackage{amsmath}
\usepackage{graphics}
\usepackage{epsfig}
\usepackage{enumerate}
\usepackage{multicol}
\usepackage{subfigure}
\usepackage{fancyhdr}
\usepackage{listings}
\usepackage{framed}
\usepackage{graphicx}
\usepackage{amsmath}
\usepackage{chngpage}

%\usepackage{bigints}
\usepackage{vmargin}

% left top textwidth textheight headheight

% headsep footheight footskip

\setmargins{2.0cm}{2.5cm}{16 cm}{22cm}{0.5cm}{0cm}{1cm}{1cm}

\renewcommand{\baselinestretch}{1.3}

\setcounter{MaxMatrixCols}{10}

\begin{document}
\begin{enumerate}
© Institute of Actuaries1
Describe the difference between the following assumptions about mortality between
any two ages, x and y (y > x):
\item
\item
uniform distribution of deaths
constant force of mortality
In your answer, explain the shape of the survival function between ages x and y under
each of the two assumptions.

2
(i)
List the key steps in constructing a new actuarial model.

You work for an actuarial consultancy which is taking over responsibility for a
modelling process which has previously been conducted in house by a client.
3
\item (ii) Discuss the extent to which the steps required for this task differ from those
listed in your answer to (i).

%%%%%%%%%%%%%%%%%%%%%%%%%%%%%%%%%%%%%%%%%%%%%%%%%%%%%%%%%%%%%%%%%%%%%%%%%%%%%%%%%%%%%%%%%%%%%%%%%
(i) List the data needed for the exact calculation of a central exposed to risk
depending on age.

An investigation studied the mortality of persons aged between exact ages 40 and 41
years. The investigation began on 1 January 2008 and ended on 31 December 2008.
The following table gives details of 10 lives involved in the investigation.
Life Date of 40th birthday Date of death
1
2
3
4
5
6
7
8
9
10 1 March 2007
1 May 2007
1 July 2007
1 October 2007
1 December 2007
1 February 2008
1 April 2008
1 June 2008
1 August 2008
1 December 2008 –
1 October 2008
–
–
1 February 2008
–
–
1 November 2008
–
–
Persons with no date of death given were still alive when the investigation ended.
\item (ii)
\item (iii)
Calculate a central exposed to risk using the data for the 10 lives in the
sample.

(a) Calculate the maximum likelihood estimate of the hazard of death at
age 40 last birthday.
(b) Hence, or otherwise, estimate q 40 .
CT4 S2009—2


%%%%%%%%%%%%%%%%%%%%%%%%%%%%%%%%%%%%
1
A uniform distribution of deaths means
EITHER
that deaths are evenly spaced between the ages x and y.
OR
that t q x  tq x
( t  y  x )
OR
that t p x  x  t is constant for t  y  x .
It also means that the survival function decreases linearly between ages x and y. The
assumption of a constant force of mortality between any two ages means
EITHER
that the hazard does not change with age over this age range.
OR
that t p x  ( p x ) t .
This implies that the survival function decreases exponentially between ages x and y.
Answers to this straightforward bookwork question were disappointing. Although
most candidates could describe the difference between a constant force of mortality
and the increasing force implied by a uniform distribution of deaths, few made correct reference to the form of the survival function. An alarming number of candidates
referred to survival functions which increased with age! Credit was given for graphs which correctly depicted the shape of the survival function under the two
assumptions.
2
(i) Define objectives of modelling process.
Plan the modelling process and how it will be validated.
Collect and validate the data required.
2  — %%%%%%%%%%%%%%%%%%%%%%%%%%%%%%%%%%%%%%%%%%%% — Examiners’ Report
Define the form of the model.
Involve experts on the real world system/get feedback on validity.
Decide on software to be used, choose random number generator etc.
Write the computer program.
Debug the program.
Analyse the output
Test the reasonableness of the output.
Consider appropriateness of response of the model to small changes in input
parameters.
Communicate and document results.
%%[1⁄2 mark was awarded for each point up to a maximum of 4 marks]

\item (ii) Whilst in theory all steps are still required, some may take the form of
reviewing the appropriateness of existing decisions made, such as how the
form of the model was determined.
Extent of work will depend on whether the existing model is to be used,
adapted or superseded.
An understanding of how results compare with those previously used by the
company will be required.
Process maps for the existing approach, or discussions with the people running the process about what they do, may be helpful.
The scope needs to be tightly defined up front to ensure it is clear what is expected of the consultancy.
Data sources may already be established.
[1⁄2 mark was awarded for each point up to a maximum of 2 marks]
Part (i) of this question was basic bookwork and was extremely well
answered. part (ii) required more thought, but many candidates were able to
write down some relevant points.
3
(i) For each life we need
EITHER date of birth OR exact age at entry into observation OR exact age at
exit from observation
Date of entry into observation
Date of exit from observation
3  — %%%%%%%%%%%%%%%%%%%%%%%%%%%%%%%%%%%%%%%%%%%% — Examiners’ Report
[Alternatives were given full credit, provided the information given allowed the
calculation of the date of entry into and exit from observation and the life’s age]
\item (ii) The contribution of each life to the central exposed to risk is the number of
months between STARTDATE and ENDDATE, where STARTDATE is the
latest of date of 40th birthday 1 January 2008 and ENDDATE is the earliest of
date of 41st birthday date of death 31 December 2008
Life
STARTDATE
ENDDATE
number of months
between
STARTDATE
and ENDDATE
1 1 January 2008 1 March 2008 2
2 1 January 2008 1 May 2008 4
3 1 January 2008 1 July 2008 6
4 1 January 2008 1 October 2008 9
5 1 January 2008 1 February 2008 1
6 1 February 2008 31 December 2008 11
7 1 April 2008 31 December 2008 9
8 1 June 2008 1 November 2008 5
9 1 August 2008 31 December 2008 5
10 1 December 2008 31 December 2008 1
Summing the number of months over the 10 lives gives a total of 53 months,
which is 4.42 years, which is the central exposed to risk.
\item (iii)
a. The total number of deaths during the period of observation is 2. So the
maximum likelihood estimate of the hazard of death is 2/4.42 =
0.4528.
b. ALTERNATIVE 1
If the hazard of death at age 40 years is  40 , then
q 40  1  p 40  1  exp(  40 )
= 1  exp(  0.4528)  1  0.6358  0.3642.
ALTERNATIVE 2
If the central exposed to risk is E 40 c , then if we work in years
q 40 
4
d 40
E  0.5 d 40
c
40  — %%%%%%%%%%%%%%%%%%%%%%%%%%%%%%%%%%%%%%%%%%%% — Examiners’ Report
=
2
2

 0.3690.
4.42  1 5.42
This was well answered. A common error was to count 3 deaths rather than 2.
Although 3 deaths are mentioned in the data given in the question, one of these
occurred after the life’s 41st birthday and so should not be included in the estimation
of \mu 40 . Another common error was to forget that exposure ends at exact age 41 years.
Each of these errors was only penalised once, so that calculations which followed
through correctly in \item (iii) were awarded full marks for part (iii). Note also that
candidates who made BOTH the above errors were only penalised for one, as if
exposure is assumed to continue past exact age 41 years, it is consistent to count 3
deaths!
\end{document}
