\documentclass[a4paper,12pt]{article}

%%%%%%%%%%%%%%%%%%%%%%%%%%%%%%%%%%%%%%%%%%%%%%%%%%%%%%%%%%%%%%%%%%%%%%%%%%%%%%%%%%%%%%%%%%%%%%%%%%%%%%%%%%%%%%%%%%%%%%%%%%%%%%%%%%%%%%%%%%%%%%%%%%%%%%%%%%%%%%%%%%%%%%%%%%%%%%%%%%%%%%%%%%%%%%%%%%%%%%%%%%%%%%%%%%%%%%%%%%%%%%%%%%%%%%%%%%%%%%%%%%%%%%%%%%%%

\usepackage{eurosym}
\usepackage{vmargin}
\usepackage{amsmath}
\usepackage{graphics}
\usepackage{epsfig}
\usepackage{enumerate}
\usepackage{multicol}
\usepackage{subfigure}
\usepackage{fancyhdr}
\usepackage{listings}
\usepackage{framed}
\usepackage{graphicx}
\usepackage{amsmath}
\usepackage{chngpage}

%\usepackage{bigints}
\usepackage{vmargin}

% left top textwidth textheight headheight

% headsep footheight footskip

\setmargins{2.0cm}{2.5cm}{16 cm}{22cm}{0.5cm}{0cm}{1cm}{1cm}

\renewcommand{\baselinestretch}{1.3}

\setcounter{MaxMatrixCols}{10}

\begin{document}
[Total 9]
A firm rents cars and operates from three locations — the Airport, the Beach and the
City. Customers may return vehicles to any of the three locations.
The company estimates that the probability of a car being returned to each location is
as follows:
Car hired from Car returned to
Airport Beach City
Airport
Beach
City 0.5
0.25
0.25
0.25
0.75
0.25
0.25
0
0.5

$$\bordermatrix{ &c_1&c_2&\ldots &c_n\cr
    r_1&a_{11} &  0  & \ldots & a_{1n}\cr
    r_2& 0  &  a_{22} & \ldots & a_{2n}\cr
    r_4& 0  &   0 &\ldots & a_{nn}}$$
    
\begin{enumerate}[(i)]
\item (i) Calculate the 2-step transition matrix. 
\item (ii) Calculate the stationary distribution $\pi$ . 
It is suggested that the cars should be based at each location in proportion to the
stationary distribution.
\item (iii)
Comment on this suggestion.
%%CT4 S2009—4
(iv)
\end{enumerate}


  \newpage
  7
  \begin{itemize}
\item (i) Two step transition matrix
8  — %%%%%%%%%%%%%%%%%%%%%%%%%%%%%%%%%%%%%%%%%%%% — Examiners’ Report
 0.5 0.25 0.25   0.5 0.25 0.25   0.375 0.375 0.25 

 
 

0  .  0.25 0.75
0  =  0.3125 0.625 0.0625 
=  0.25 0.75
 0.25 0.25 0.5   0.25 0.25 0.5   0.3125 0.375 0.3125 


 
 
 0.5 0.25 0.25 


0 
\item (ii)     0.25 0.75
 0.25 0.25 0.5 


 1  0.5  \pi_{1} 0.25  \pi_{2} 0.25  3
 2  0.25  \pi_{1} 0.75  \pi_{2} 0.25  3
 3  0.25  \pi_{1} 0.5  3
and  \pi_{1}  \pi_{2}  3  1
 1  2  3
 2  3  3
 1  1
3
 2  1
 3  1
2
6
\item (iii)The stationary distribution gives the long run probability that a particular car
will be at each location. However this does not take into account the demand
for hiring vehicles at each location, or the amount of space available at each
location. These factors are likely to be more important in determining how
many cars to base at each site.
1.2
1
0.8
Airport
Beach
City
0.6
0.4
0.2
0
0
9
1
2
3
4
5
Number of rentals
6
7
8
9  — %%%%%%%%%%%%%%%%%%%%%%%%%%%%%%%%%%%%%%%%%%%% — Examiners’ Report
\item (iv) A starts at 1, B and C at zero
Asymptote to the stationary distribution probs.
B and C same after 1 period
A and B same after 2 periods.
\begin{itemize}
\item The calculations in parts (i) and (ii) were, as is usually the case in CT4 examinations,
successfully completed by the vast majority of candidates. However only a minority
made the point that, whereas the stationary distribution gives the long run probability
that cars will be returned to each location, the company would be better advised to
position cars at the three locations to reflect the demand for rentals. In part (iv), some
candidates drew a set of histograms. 
\item Credit was given for this, provided that
histograms were presented for 1 rental, 2 rentals, and the long run distribution,
together with a statement that at 0 rentals the car must be at the Airport.
\end{itemize}
\end{itemize}
    \end{document}
