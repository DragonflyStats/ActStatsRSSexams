\documentclass[a4paper,12pt]{article}

%%%%%%%%%%%%%%%%%%%%%%%%%%%%%%%%%%%%%%%%%%%%%%%%%%%%%%%%%%%%%%%%%%%%%%%%%%%%%%%%%%%%%%%%%%%%%%%%%%%%%%%%%%%%%%%%%%%%%%%%%%%%%%%%%%%%%%%%%%%%%%%%%%%%%%%%%%%%%%%%%%%%%%%%%%%%%%%%%%%%%%%%%%%%%%%%%%%%%%%%%%%%%%%%%%%%%%%%%%%%%%%%%%%%%%%%%%%%%%%%%%%%%%%%%%%%

\usepackage{eurosym}
\usepackage{vmargin}
\usepackage{amsmath}
\usepackage{graphics}
\usepackage{epsfig}
\usepackage{enumerate}
\usepackage{multicol}
\usepackage{subfigure}
\usepackage{fancyhdr}
\usepackage{listings}
\usepackage{framed}
\usepackage{graphicx}
\usepackage{amsmath}
\usepackage{chngpage}

%\usepackage{bigints}
\usepackage{vmargin}

% left top textwidth textheight headheight

% headsep footheight footskip

\setmargins{2.0cm}{2.5cm}{16 cm}{22cm}{0.5cm}{0cm}{1cm}{1cm}

\renewcommand{\baselinestretch}{1.3}

\setcounter{MaxMatrixCols}{10}

\begin{document}
\begin{enumerate}

6 An investigation by a hospital into rates of recovery after a specific type of operation
collected the following data for each month of the calendar year 2008:
•
number of persons who recovered from the operation during the month (defined as
being discharged from the hospital) classified by the month of their operation.
You may assume that there were no deaths.
On the first day of each month from January 2008 to January 2009, the hospital listed
all in-patients who were yet to recover from this operation, classified according to the
length of time elapsing since their operation, to the nearest month.
\begin{enumerate}[(i)]
\item (i)
7
(a) Write down an expression which will enable the hospital to calculate
rates of recovery, r x , during 2008 at various durations x since the
operation using the available data.
(b) Derive a formula for the exposed to risk based on the information in
the hospital’s monthly lists of in-patients which corresponds to the data
on recovery from the operation.

\item (ii) Determine the value of f such that the expression in \item (i)(a) applies to an actual
duration x + f since the operation.
\end{enumerate}
[Total 7]
\item (i) Explain how the classification of stochastic processes according to the nature
of their state space and time space leads to a four way classification.

\item (ii) For each of the four types of process:
(a) give an example of a statistical model
(b) write down a problem of relevance to the operation of:
•
•
a food retailer
a general insurance company


%%%%%%%%%%%%%%%%%%%%%%%%%%%%%%%%%%%%%%%%%%%%%%%%%%%%%%%%%%%%%%%%%%%%%%
\newpage

7
\item (i)
Processes can be classified, first, according to whether their state space (i.e.
the range of states they can possibly occupy) is discrete or continuous
For processes operating in both discrete and continuous state space the time
domain can either be discrete or continuous
Therefore we have four possible types of process
EITHER
2 types of state space × 2 types of time domain
OR
State space Time domain
Discrete
Discrete
Continuous
Continuous Discrete
Continuous
Discrete
Continuous
Page 9%%%%%%%%%%%%%%%%%%%%%%%%%%%— %%%%%%%%%%%%%%%%%%%%%%%%%%%%%%%%%%%%%%%%%%%%%%% — Examiners’ Report
\item (ii)
Type of process Statistical model
SS Discrete/
T Discrete Markov chain
Markov jump chain
Counting process
Random walk
Counting process
Poisson process
Markov jump process
Compound Poisson
process
ARIMA time series
model
General random walk
White noise
Compound Poisson
process
Brownian motion
Ito process
SS Discrete/
T Continuous
SS Continuous/
T Discrete
SS Continuous/
T Continuous
8
\item (i)
Problem of relevance to food retailer Whether or not particular product out of stock at the end of each day Rate of arrival of
customers in shop Problem of relevance to a general insurer
No claims bonus Value of goods in
stock at the end of each day Total amount insured
on a certain type of policy valued at the end of each month Value of claims
arriving monitored continuously 
Volume (or value) of trade in shop over a
continuous period of time 
Number of claims
received monitored
continuously
There are x infected cats and hence 10 – x uninfected cats.
Flea transmission requires one of the x infected cats to meet one of the (10 − x )
uninfected cats.
\item (ii)
⎛ 10 ⎞
The total number of pairings of cats is ⎜ ⎟ = 45.
⎝ 2 ⎠
So the probability of a meeting resulting in an increase in the number of cats
with fleas is 0.5 x (10 − x )/45.
As this depends only on the number of cats currently infected, and meetings
occur according to a Poisson process, the number of infected cats over time
follows a Markov jump process.
(a)
The state space is the number of cats infected {0,1,2,,.....10}%%%%%%%%%%%%%%%%%%%%%%%%%%%— %%%%%%%%%%%%%%%%%%%%%%%%%%%%%%%%%%%%%%%%%%%%%%% — Examiners’ Report
(b)
The generator matrix is
⎛ 0 0
⎞
⎜
⎟
− 9 9
⎜
⎟
⎜
⎟
− 16 16
⎜
⎟
− 21 21
⎜
⎟
⎜
⎟
− 24 24
⎟
\mu ⎜
A = ⎜
− 25 25
⎟
90 ⎜
⎟
− 24 24
⎜
⎟
− 21 21
⎜
⎟
⎜
⎟
− 16 16
⎜
⎟
⎜
− 9 9 ⎟
⎜
⎟
0 ⎠
⎝
Kolmogorov’s equations:
EITHER
forward form
d
P ( t ) = P ( t ) A
dt
OR
backward form
\item (iii)
Holding times are exponentially distributed.
With mean
(iv)
d
P ( t ) = AP ( t )
dt
90
\mu x (10 − x )
OR parameter
.
\mu x (10 − x )
90
Total expected time is the sum of the mean holding times.
=
90 9
1
90 ⎛ 1 1 1
1
1
1
1 1 1 ⎞
= ⎜ + + +
+
+
+ + + ⎟
∑
\mu x = 1 x (10 − x ) \mu ⎝ 9 16 21 24 25 24 21 16 9 ⎠
= 50.92/

\mu
Page 11%%%%%%%%%%%%%%%%%%%%%%%%%%%— %%%%%%%%%%%%%%%%%%%%%%%%%%%%%%%%%%%%%%%%%%%%%%% — Examiners’ Report
