\documentclass[a4paper,12pt]{article}

%%%%%%%%%%%%%%%%%%%%%%%%%%%%%%%%%%%%%%%%%%%%%%%%%%%%%%%%%%%%%%%%%%%%%%%%%%%%%%%%%%%%%%%%%%%%%%%%%%%%%%%%%%%%%%%%%%%%%%%%%%%%%%%%%%%%%%%%%%%%%%%%%%%%%%%%%%%%%%%%%%%%%%%%%%%%%%%%%%%%%%%%%%%%%%%%%%%%%%%%%%%%%%%%%%%%%%%%%%%%%%%%%%%%%%%%%%%%%%%%%%%%%%%%%%%%

\usepackage{eurosym}
\usepackage{vmargin}
\usepackage{amsmath}
\usepackage{graphics}
\usepackage{epsfig}
\usepackage{enumerate}
\usepackage{multicol}
\usepackage{subfigure}
\usepackage{fancyhdr}
\usepackage{listings}
\usepackage{framed}
\usepackage{graphicx}
\usepackage{amsmath}
\usepackage{chngpage}

%\usepackage{bigints}
\usepackage{vmargin}

% left top textwidth textheight headheight

% headsep footheight footskip

\setmargins{2.0cm}{2.5cm}{16 cm}{22cm}{0.5cm}{0cm}{1cm}{1cm}

\renewcommand{\baselinestretch}{1.3}

\setcounter{MaxMatrixCols}{10}

\begin{document}
\begin{enumerate}
8
Sketch, using your answers to parts (i) and \item (ii), a graph showing the
probability that a car currently located at the Airport is subsequently at the
Airport, Beach or City against the number of times the car has been rented. 
[Total 10]
A researcher is studying a certain incurable disease. The disease can be fatal, but
often sufferers survive with the condition for a number of years. The researcher
wishes to project the number of deaths caused by the disease by using a multiple state
model with state space:
{ H – Healthy, I – Infected, D (from disease) – Dead (caused by the disease), D (not from disease)
– Dead (not caused by the disease)}.
The transition rates, dependent on age x , are as follows:
\item a mortality rate from the Healthy state of \mu ( x )
\item a rate of infection with the disease σ ( x )
\item a mortality rate from the Infected state of υ ( x ) of which ρ ( x ) relates to Deaths
caused by the disease
(i) Draw a transition diagram for the multiple state model.
\item (ii) Write down Kolmogorov’s forward equations governing the transitions by
specifying the transition matrix.

\item (iii) Determine integral expressions, in terms of the transition rates and any
expressions previously determined, for:
(a) P HH (x, x + t)
(b) P HI (x, x + t)
(c) P HD(from disease) (x, x + t)


[Total 10]
CT4 S2009—5
%%%%%%%%%%%%%%%%%%%%%%%%%%%%%%%%%%%%%%%%%

8
(i)
 ( x )
 ( x )
Healthy Infected
 ( x )  ( x )   ( x )
Dead (not
from
disease)
\item (ii)
d
P ( x )  P ( x ) A ( x )
dt
where with order of state space
{Healthy, Infected, Dead (not disease), Dead(from disease)}
 ( x )
0 
  ( x )   ( x )  ( x )


0
 ( x )  ( x )   ( x )  ( x ) 

A(x)=

0
0
0
0 


0
0
0
0 

10
Dead
(from
disease)  — %%%%%%%%%%%%%%%%%%%%%%%%%%%%%%%%%%%%%%%%%%%% — Examiners’ Report
\item (iii)
t
a. P HH (x, x+t)= exp[ 

(  ( x  w )   ( x  w )) dw ]
w  0
t

b. P HI (x, x+t)=
t
P HH ( x , x  w ).  ( x  w ).exp[ 
w  0

 ( x  u ) du ]. dw
u  w
c. EITHER
t

P HD(from disease) (x, x+t)=
P HI ( x , x  w ).  ( x  w ). dw
w  0
OR (backwards alternative)
P HD(from disease) (x, x+t)
t
 P
=
HH
( x  w ).  ( x  w ). P ID ( fromdiseas e ) ( x  w , x  t ). dw .
w  0
t
Now P ID ( fromdiseas e ) ( x  w , x  t ) 
 P
II
( x  w , x  s ).  ( x  s ). 1 . ds
s  w
 s

and P II ( x  w , x  s )  exp     ( x  u ) du  .
 u  w

So P HD(from disease) (x, x+t)
 s

  P HH ( x  w ).  ( x  w ).  exp     ( x  u ) du  .  ( x  s ). ds . dw
w  0
s  w
 u  w

t
t
This question was considerably better answered than were similar questions in
previous examinations. In particular, the proportion of candidates making serious
attempts at part (iii) was greater than has been the case for similar questions in the
past.


\end{document}
