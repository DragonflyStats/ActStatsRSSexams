\documentclass[a4paper,12pt]{article}

%%%%%%%%%%%%%%%%%%%%%%%%%%%%%%%%%%%%%%%%%%%%%%%%%%%%%%%%%%%%%%%%%%%%%%%%%%%%%%%%%%%%%%%%%%%%%%%%%%%%%%%%%%%%%%%%%%%%%%%%%%%%%%%%%%%%%%%%%%%%%%%%%%%%%%%%%%%%%%%%%%%%%%%%%%%%%%%%%%%%%%%%%%%%%%%%%%%%%%%%%%%%%%%%%%%%%%%%%%%%%%%%%%%%%%%%%%%%%%%%%%%%%%%%%%%%

\usepackage{eurosym}
\usepackage{vmargin}
\usepackage{amsmath}
\usepackage{graphics}
\usepackage{epsfig}
\usepackage{enumerate}
\usepackage{multicol}
\usepackage{subfigure}
\usepackage{fancyhdr}
\usepackage{listings}
\usepackage{framed}
\usepackage{graphicx}
\usepackage{amsmath}
\usepackage{chngpage}

%\usepackage{bigints}
\usepackage{vmargin}

% left top textwidth textheight headheight

% headsep footheight footskip

\setmargins{2.0cm}{2.5cm}{16 cm}{22cm}{0.5cm}{0cm}{1cm}{1cm}

\renewcommand{\baselinestretch}{1.3}

\setcounter{MaxMatrixCols}{10}

\begin{document}

\item (i) Explain how the classification of stochastic processes according to the nature
of their state space and time space leads to a four way classification.

\item (ii) For each of the four types of process:
(a) give an example of a statistical model
(b) write down a problem of relevance to the operation of:
•
•
a food retailer
a general insurance company


%%%%%%%%%%%%%%%%%%%%%%%%%%%%%%%%%%%%%%%%%%%%%%%%%%%%%%%%%%%%%%%%%%%%%%
\newpage

7
\item (i)
Processes can be classified, first, according to whether their state space (i.e.
the range of states they can possibly occupy) is discrete or continuous
For processes operating in both discrete and continuous state space the time
domain can either be discrete or continuous
Therefore we have four possible types of process
EITHER
2 types of state space × 2 types of time domain
OR
State space Time domain
Discrete
Discrete
Continuous
Continuous Discrete
Continuous
Discrete
Continuous
Page 9%%%%%%%%%%%%%%%%%%%%%%%%%%%— %%%%%%%%%%%%%%%%%%%%%%%%%%%%%%%%%%%%%%%%%%%%%%% — Examiners’ Report
\item (ii)



%%%%%%%%%%%%%%%%%%%%%%%%%%%%%%%%%%%%%%%%%
\noindent \textbf{1. SS Discrete/
T Discrete}
\begin{itemize}
\item Markov chain
\item Markov jump chain 
\item Counting process
\item Random walk
\end{itemize}

\noindent \textbf{Relevance to a  to food retailer }
\begin{itemize}
\item Whether or not
particular product out
of stock at the end of
each day
\end{itemize}
\noindent \textbf{Relevance to a general insurer}
\begin{itemize}
\item No claims bonus
\end{itemize}

%%%%%%%%%%%%%%%%%%%%%%%%%%%%%%%%%%%%%%%%%
\noindent \textbf{2. SS Discrete/
T Continuous }
\begin{itemize}
\item Counting process
\item Poisson process
\item Markov jump process
\item Compound Poisson process
\end{itemize}

\noindent \textbf{Relevance to a  to food retailer }
\begin{itemize}
\item Rate of arrival of
customers in shop
\end{itemize}
\noindent \textbf{Relevance to a general insurer}
\begin{itemize}
\item Number of claims
received monitored
continuously
\end{itemize}

%%%%%%%%%%%%%%%%%%%%%%%%%%%%%%%%%%%%%%%%%
\noindent \textbf{3. SS Continuous/
T Discrete}
\begin{itemize}
\item ARIMA time series model 
\item General random walk 
\item White noise 
\end{itemize}
%----------------------%
\noindent \textbf{Relevance to a  to food retailer }
\begin{itemize}
\item Value of goods in
stock at the end of
each day
\end{itemize}
\noindent \textbf{Relevance to a general insurer}
\begin{itemize}
\item Total amount insured
on a certain type of
policy valued at the
end of each month
\end{itemize}
%%%%%%%%%%%%%%%%%%%%%%%%%%%%%%%%%%%%%%%%%
\noindent \textbf{4. SS Continuous/
T Continuous }
\begin{itemize}
Compound Poisson process \\ Brownian motion \\ Ito process
\end{itemize}

\noindent \textbf{Relevance to a  to food retailer }
\begin{itemize}
\item Volume (or value) of
trade in shop over a
continuous period of
time
\end{itemize}

\noindent \textbf{Relevance to a general insurer}
\begin{itemize}
\item Value of claims
arriving monitored
continuously
\end{itemize}
%%%%%%%%%%%%%%%%%%%%%%%%%%%%%%%%%%%%%%%%
