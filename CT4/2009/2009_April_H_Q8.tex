\documentclass[a4paper,12pt]{article}

%%%%%%%%%%%%%%%%%%%%%%%%%%%%%%%%%%%%%%%%%%%%%%%%%%%%%%%%%%%%%%%%%%%%%%%%%%%%%%%%%%%%%%%%%%%%%%%%%%%%%%%%%%%%%%%%%%%%%%%%%%%%%%%%%%%%%%%%%%%%%%%%%%%%%%%%%%%%%%%%%%%%%%%%%%%%%%%%%%%%%%%%%%%%%%%%%%%%%%%%%%%%%%%%%%%%%%%%%%%%%%%%%%%%%%%%%%%%%%%%%%%%%%%%%%%%

\usepackage{eurosym}
\usepackage{vmargin}
\usepackage{amsmath}
\usepackage{graphics}
\usepackage{epsfig}
\usepackage{enumerate}
\usepackage{multicol}
\usepackage{subfigure}
\usepackage{fancyhdr}
\usepackage{listings}
\usepackage{framed}
\usepackage{graphicx}
\usepackage{amsmath}
\usepackage{chngpage}

%\usepackage{bigints}
\usepackage{vmargin}

% left top textwidth textheight headheight

% headsep footheight footskip

\setmargins{2.0cm}{2.5cm}{16 cm}{22cm}{0.5cm}{0cm}{1cm}{1cm}

\renewcommand{\baselinestretch}{1.3}

\setcounter{MaxMatrixCols}{10}

\begin{document}

There is a population of ten cats in a certain neighbourhood. Whenever a cat which
has fleas meets a cat without fleas, there is a 50% probability that some of the fleas
transfer to the other cat such that both cats harbour fleas thereafter. Contacts between
two of the neighbourhood cats occur according to a Poisson process with rate μ, and
these meetings are equally likely to involve any of the possible pairs of individuals.
Assume that once infected a cat continues to have fleas, and that none of the cats’
owners has taken any preventative measures.
\begin{enumerate}
\item (i) If the number of cats currently infected is x, explain why the number of
possible pairings of cats which could result in a new flea infection is x(10 – x).
\item 
(ii) Show how the number of infected cats at any time, X(t), can be formulated as
a Markov jump process, specifying:
(a)
(b)
the state space
the Kolmogorov differential equations in matrix form
\item 
(iii) State the distribution of the holding times of the Markov jump process.
\item (iv) Calculate the expected time until all the cats have fleas, starting from a single
flea-infected cat.
\end{enumerate}

%%%%%%%%%%%%%%%%%%%%%%%%%%%%%%%%%%%%%%%%%%%%%%%%%%%%%%%%%%%%%%%%%%%%%%%%%%%%%%%5
8
\item (i)

There are x infected cats and hence 10 – x uninfected cats.
Flea transmission requires one of the x infected cats to meet one of the (10 − x )
uninfected cats.
\item (ii)
⎛ 10 ⎞
The total number of pairings of cats is ⎜ ⎟ = 45.
⎝ 2 ⎠
So the probability of a meeting resulting in an increase in the number of cats
with fleas is 0.5 x (10 − x )/45.
As this depends only on the number of cats currently infected, and meetings
occur according to a Poisson process, the number of infected cats over time
follows a Markov jump process.
(a)
The state space is the number of cats infected {0,1,2,,.....10}

%%%%%%%%%%%%%%%%%%%%%%%%%%%— %%%%%%%%%%%%%%%%%%%%%%%%%%%%%%%%%%%%%%%%%%%%%%% — Examiners’ Report
(b)
The generator matrix is
⎛ 0 0
⎞
⎜
⎟
− 9 9
⎜
⎟
⎜
⎟
− 16 16
⎜
⎟
− 21 21
⎜
⎟
⎜
⎟
− 24 24
⎟
\mu ⎜
A = ⎜
− 25 25
⎟
90 ⎜
⎟
− 24 24
⎜
⎟
− 21 21
⎜
⎟
⎜
⎟
− 16 16
⎜
⎟
⎜
− 9 9 ⎟
⎜
⎟
0 ⎠
⎝
Kolmogorov’s equations:
EITHER
forward form
d
P ( t ) = P ( t ) A
dt
OR
backward form
\item (iii)
Holding times are exponentially distributed.
With mean
(iv)
d
P ( t ) = AP ( t )
dt
90
\mu x (10 − x )
OR parameter
.
\mu x (10 − x )
90
Total expected time is the sum of the mean holding times.
=
90 9
1
90 ⎛ 1 1 1
1
1
1
1 1 1 ⎞
= ⎜ + + +
+
+
+ + + ⎟
∑
\mu x = 1 x (10 − x ) \mu ⎝ 9 16 21 24 25 24 21 16 9 ⎠
= 50.92/

\mu
Page 11%%%%%%%%%%%%%%%%%%%%%%%%%%%— %%%%%%%%%%%%%%%%%%%%%%%%%%%%%%%%%%%%%%%%%%%%%%% — Examiners’ Report


\end{document}
