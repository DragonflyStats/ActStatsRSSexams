\documentclass[a4paper,12pt]{article}

%%%%%%%%%%%%%%%%%%%%%%%%%%%%%%%%%%%%%%%%%%%%%%%%%%%%%%%%%%%%%%%%%%%%%%%%%%%%%%%%%%%%%%%%%%%%%%%%%%%%%%%%%%%%%%%%%%%%%%%%%%%%%%%%%%%%%%%%%%%%%%%%%%%%%%%%%%%%%%%%%%%%%%%%%%%%%%%%%%%%%%%%%%%%%%%%%%%%%%%%%%%%%%%%%%%%%%%%%%%%%%%%%%%%%%%%%%%%%%%%%%%%%%%%%%%%

\usepackage{eurosym}
\usepackage{vmargin}
\usepackage{amsmath}
\usepackage{graphics}
\usepackage{epsfig}
\usepackage{enumerate}
\usepackage{multicol}
\usepackage{subfigure}
\usepackage{fancyhdr}
\usepackage{listings}
\usepackage{framed}
\usepackage{graphicx}
\usepackage{amsmath}
\usepackage{chngpage}

%\usepackage{bigints}
\usepackage{vmargin}

% left top textwidth textheight headheight

% headsep footheight footskip

\setmargins{2.0cm}{2.5cm}{16 cm}{22cm}{0.5cm}{0cm}{1cm}{1cm}

\renewcommand{\baselinestretch}{1.3}

\setcounter{MaxMatrixCols}{10}

\begin{document}

4
(i)
Describe what is meant by the following terms:
(a) discrete state space
(b) stochastic model
(c) continuous time model
(d)
stochastic process of mixed type

(ii)



Q4
(i)
(a) The set of possible states that the process can take in a case where the
process can only take a countable number of different values,
OR
the set of states that the process can take where it can take only distinct
states. 
(b) A model in which at least one of the components is random in nature,
OR
a collection of random variables, one for each time point. 
(c) A model in which changes in state may take place at any point in time
(between the start and end times).

(d) Processes which operate in continuous time but which can also change
value at predetermined discrete instants.

[Total for part (i): 4]
(ii)
Whether outputs from the model are only required at discrete points in time.

The objectives of the modelling
OR
the accuracy required. 
The nature of the input data (which may override the nature of the
process). 
The expertise of the analyst. 
Time, cost, IT resources. 
The nature of previous models. 
If simulation is required it may be easier to make the time step discrete. 
Continuous time models are ultimately more flexible than discrete time
models. 
Some results for continuous time models cannot be obtained by discrete
simulation at all. 
Regulatory requirements. 
Page 7
The need to explain the model to a non-technical audience.

[Total for part (ii): max. 3]
[Total 7]
Answers to this question were generally poor. In part (i)(a) and
(c), a substantial minority of candidates provided unnecessary
answers rather than offering descriptions of what the terms mean:
thus “a continuous time model is a model in which time is
continuous”. No credit was awarded for such answers. In part
(i)(d) many candidates incorrectly wrote that a stochastic process
of mixed type is one in which the state space is discrete and the time
domain is continuous, or vice versa. In part (ii), full credit could be
obtained for less than is written in this Examiners’ Report, though
candidates generally did not offer factors beyond the first three
points listed in the solution above.
