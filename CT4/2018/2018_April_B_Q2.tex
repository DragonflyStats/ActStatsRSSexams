\documentclass[a4paper,12pt]{article}

%%%%%%%%%%%%%%%%%%%%%%%%%%%%%%%%%%%%%%%%%%%%%%%%%%%%%%%%%%%%%%%%%%%%%%%%%%%%%%%%%%%%%%%%%%%%%%%%%%%%%%%%%%%%%%%%%%%%%%%%%%%%%%%%%%%%%%%%%%%%%%%%%%%%%%%%%%%%%%%%%%%%%%%%%%%%%%%%%%%%%%%%%%%%%%%%%%%%%%%%%%%%%%%%%%%%%%%%%%%%%%%%%%%%%%%%%%%%%%%%%%%%%%%%%%%%

\usepackage{eurosym}
\usepackage{vmargin}
\usepackage{amsmath}
\usepackage{graphics}
\usepackage{epsfig}
\usepackage{enumerate}
\usepackage{multicol}
\usepackage{subfigure}
\usepackage{fancyhdr}
\usepackage{listings}
\usepackage{framed}
\usepackage{graphicx}
\usepackage{amsmath}
\usepackage{chngpage}

%\usepackage{bigints}
\usepackage{vmargin}

% left top textwidth textheight headheight

% headsep footheight footskip

\setmargins{2.0cm}{2.5cm}{16 cm}{22cm}{0.5cm}{0cm}{1cm}{1cm}

\renewcommand{\baselinestretch}{1.3}

\setcounter{MaxMatrixCols}{10}

\begin{document}

[Total 4]
2
A football match between two teams, Team A and Team B, is being decided by a
penalty competition. Each team takes one penalty alternately. Team A goes first.
Let X i be the total number of penalties scored by team A minus the total number of
penalties scored by team B after the ith penalty has been taken. If X i = 2, team A wins
and the competition stops. If X i = –2, team B wins and the competition stops.
(i)
Determine the possible sample paths for the process X i for i = 1, 2, 3, 4.

Suppose the chance of team A scoring each of its penalties is 0.5, and the chance of
team B scoring each of its penalties is 0.4.


Q2
(i)
Team A goes first, so at i = 1 the process can have the values 1 (if Team A
scores) or 0 (if Team A misses).

Page 3
Team B then has a go. If Team B scores, then X 2 = X 1 – 1.
If Team B misses, then X 2 = X 1 .

Team A then has another go. If Team A scores, then X 3 = X 2 + 1.
If Team A misses, then X 3 = X 2 .

Hence possible sample paths for X i (i = 1, 2, 3, 4) are:
0, 0, 0, 0
0, 0, 0, –1
0, 0, 1, 0
0, 0, 1, 1
0, –1, 0, 0
0, –1, 0, –1
0, –1, –1, –1
0, –1, –1, –2
1, 0, 0, –1
1, 0, 0, 0
1, 0, 1, 0
1, 0, 1, 1
1, 1, 1, 0
1, 1, 1, 1
1, 1, 2, process ends at i = 3
(ii)
Taking the paths in (i) and considering only the first three penalties we can
compute the probabilities as follows:
0, 0, 0
0, 0, 1
0, -1, 0
0, -1, -1
1, 0, 0
1, 0, 1
1, 1, 1
1, 1, 2
0.5 x 0.6 x 0.5 = 0.15
0.5 x 0.6 x 0.5 = 0.15
0.5 x 0.4 x 0.5 = 0.10
0.5 x 0.4 x 0.5 = 0.10
0.5 x 0.4 x 0.5 = 0.10
0.5 x 0.4 x 0.5 = 0.10
0.5 x 0.6 x 0.5 = 0.15
0.5 x 0.6 x 0.5 = 0.15
so the distributions are as follows
Page 4

[Total for part (i): max. 3]


[Total for part (ii): 3]
[Total 6]
This question was on a part of the syllabus which had not been
tested for several sessions, but which is an important part of the
Core Reading. Performance was poor, with only a minority of
candidates managing to list the possible sample paths in part (i),
and fewer being able to compute the probabilities in part (ii). A
common error was to fail to read the question closely and to
consider pairs of penalties. Some credit was given for answers
which were correct on the basis of pairs of penalties.
