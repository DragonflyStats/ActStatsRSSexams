\documentclass[a4paper,12pt]{article}

%%%%%%%%%%%%%%%%%%%%%%%%%%%%%%%%%%%%%%%%%%%%%%%%%%%%%%%%%%%%%%%%%%%%%%%%%%%%%%%%%%%%%%%%%%%%%%%%%%%%%%%%%%%%%%%%%%%%%%%%%%%%%%%%%%%%%%%%%%%%%%%%%%%%%%%%%%%%%%%%%%%%%%%%%%%%%%%%%%%%%%%%%%%%%%%%%%%%%%%%%%%%%%%%%%%%%%%%%%%%%%%%%%%%%%%%%%%%%%%%%%%%%%%%%%%%

\usepackage{eurosym}
\usepackage{vmargin}
\usepackage{amsmath}
\usepackage{graphics}
\usepackage{epsfig}
\usepackage{enumerate}
\usepackage{multicol}
\usepackage{subfigure}
\usepackage{fancyhdr}
\usepackage{listings}
\usepackage{framed}
\usepackage{graphicx}
\usepackage{amsmath}
\usepackage{chngpage}

%\usepackage{bigints}
\usepackage{vmargin}

% left top textwidth textheight headheight

% headsep footheight footskip

\setmargins{2.0cm}{2.5cm}{16 cm}{22cm}{0.5cm}{0cm}{1cm}{1cm}

\renewcommand{\baselinestretch}{1.3}

\setcounter{MaxMatrixCols}{10}

\begin{document}

A small town is served by a single funeral director. The funeral director collects
corpses immediately following death and stores them in a refrigerator pending
embalming. The number of deaths per day in this town has the following probability
distribution:
Number of
deaths per day Probability
0 0.497
1 0.348
2 0.122
3 0.028
4 0.005
The embalmer can embalm exactly one corpse per day. He works on a corpse from
the refrigerator if there is one, but if the refrigerator is empty he works on the first
corpse to arrive that day. Corpses are removed from the refrigerator immediately
before being embalmed and are not returned there after embalming.
The refrigerator has room for four corpses. If more space is needed, the funeral
director has to ask the local hospital if there is spare capacity in the hospital’s
refrigerator.
(i)
Determine the transition matrix for the number of corpses in the funeral
director’s refrigerator.

(ii) Calculate the long-run probability of there being 0, 1, 2, 3 and 4 corpses in the
refrigerator.
(iii) Calculate the probability that the funeral director has to contact the hospital on
any given day.

The embalmer has not had a day off for years. The funeral director says that from now
on the embalmer must not work on Christmas Day.
(iv)

Calculate the probability that the funeral director will need to contact the
hospital on Christmas Day when the embalmer is not working. 

[Total 12]
END OF PAPER
CT4 S2018–9 
PLEASE TURN OVER


Q12
(i)
EITHER
The number of corpses in the refrigerator one morning is the number
the previous morning, plus the number of deaths that day less the one
the embalmer embalmed

OR
If no corpses are in the refrigerator, then
p(0,0) = 0.497 + 0.348 (since if there is
1 death the embalmer will embalm the one who dies)
p(0,1) = 0.122
p(0,2) = 0.028
p(0,3) = 0.005
p(0,4) = 0.
If one corpse is in the refrigerator, then
p(1,0) = 0.497
p(1,1) = 0.348
p(1,2) = 0.122
p(1,3) = 0.028
p(1,4) = 0.005

Using similar calculations for 2, 3 and 4 corpses, we obtain the
transition matrix, P, of the number of corpses in the refrigerator:
0  0.845 0.122 0.028 0.005
0 
1   0.497 0.348 0.122 0.028 0.005  
2  0
0.497 0.348 0.122 0.033  .


3  0
0
0.497 0.348 0.155 
4   0
0
0
0.497 0.503  
[+2]
[max. 3]
(ii)
If x i is the probability that the refrigerator contains i corpses at the
start of a day, then
Using the transition matrix in part (i) and π = π P we get
Page 24

=
x 0 0.845 x 0 + 0.497 x 1
x 1 = 0.122 x 0 + 0.348 x 1 + 0.497 x 2
x 2 = 0.028 x 0 + 0.122 x 1 + 0.348 x 2 + 0.497 x 3
[+2]
x 3 = 0.005 x 0 + 0.028 x 1 + 0.122 x 2 + 0.348 x 3 + 0.497 x 4
x 4 = 0.005 x 1 + 0.033 x 2 + 0.155 x 3 + 0.503 x 4
Proceeding recursively we obtain
x 1 = (0.155/0.497)x 0 = 0.312x 0
0.312x 0 = 0.122x 0 + 0.348(0.312x 0 ) + 0.497x 2
x 2 = (0.081/0.497)x 0 = 0.164x 0
0.164x 0 = 0.028x 0 + 0.122(0.312x 0 ) + 0.348(0.164x 0 ) + 0.497x 3
x 3 = (0.041/0.497)x 0 = 0.082x 0
0.082x 0 = 0.005x 0 + 0.028(0.312x 0 ) + 0.122(0.164x 0 ) + 0.348(0.082x 0 )
+ 0.497x 4
[+11⁄2]
x 4 = (0.020/0.497)x 0 = 0.040x 0
So we have
x 0 + 0.312x 0 + 0.164x 0 + 0.082x 0 + 0.040x 0 = 1
x 0
x 1
x 2
x 3
x 4
(iii)
= 0.626
= 0.195
= 0.103
= 0.051
= 0.025



The probability the funeral director has to contact the hospital is:
x 2 Pr[4 deaths] + x 3 Pr[3 or 4 deaths] + x 4 [Pr 2 or more deaths]

= 0.005x 2 + 0.033x 3 + 0.155x 4
= 0.005(0.103) + 0.033(0.051) + 0.155(0.025) = 0.006.
(iv)


Probability the funeral director has to contact the
hospital on Christmas Day is
Page 25
x 1 Pr[4 deaths] + x 2 Pr[3 or 4 deaths] + x 3 [Pr 2 or more deaths]
+ x 4 [Pr 1 or more deaths] 
= 0.005(0.195) + 0.033(0.103) + 0.155(0.051) +0.503(0.025) = 0.025. 

[Total 12]
This question was based on a practical problem faced by a funeral
director known to the Principal Examiner. It was poorly answered by
most candidates. In part (i) there was +2 marks for the correct matrix and
+1 for some explanation of how the matrix was determined. Few
candidates could correctly formulate the matrix. A common error was to
ignore the fact that the embalmer can embalm one corpse per day and will
always do so provided a corpse is available. Some candidates assumed
that once the fridge was full the funeral director would offload the entire contents to the hospital morgue, rather than just those corpses in excess of
4. Candidates whose matrices in part (i) were incorrect could score full
credit for part (ii) if they correctly calculated the stationary distribution
for the matrix they had produced in past (i). Few candidates attempted
parts (iii) and (iv) and most attempts were incorrect. Full credit could be
obtained in parts (iii) and (iv) for answers which applied the correct
method to incorrect matrices.
END OF EXAMINERS’ REPORT
Page 26
