\documentclass[a4paper,12pt]{article}

%%%%%%%%%%%%%%%%%%%%%%%%%%%%%%%%%%%%%%%%%%%%%%%%%%%%%%%%%%%%%%%%%%%%%%%%%%%%%%%%%%%%%%%%%%%%%%%%%%%%%%%%%%%%%%%%%%%%%%%%%%%%%%%%%%%%%%%%%%%%%%%%%%%%%%%%%%%%%%%%%%%%%%%%%%%%%%%%%%%%%%%%%%%%%%%%%%%%%%%%%%%%%%%%%%%%%%%%%%%%%%%%%%%%%%%%%%%%%%%%%%%%%%%%%%%%

\usepackage{eurosym}
\usepackage{vmargin}
\usepackage{amsmath}
\usepackage{graphics}
\usepackage{epsfig}
\usepackage{enumerate}
\usepackage{multicol}
\usepackage{subfigure}
\usepackage{fancyhdr}
\usepackage{listings}
\usepackage{framed}
\usepackage{graphicx}
\usepackage{amsmath}
\usepackage{chngpage}

%\usepackage{bigints}
\usepackage{vmargin}

% left top textwidth textheight headheight

% headsep footheight footskip

\setmargins{2.0cm}{2.5cm}{16 cm}{22cm}{0.5cm}{0cm}{1cm}{1cm}

\renewcommand{\baselinestretch}{1.3}

\setcounter{MaxMatrixCols}{10}

\begin{document}

PLEASE TURN OVER6
The National Statistics Office of a small, low income country wants to estimate recent
death rates. A death registration system has allowed the National Statistics Office to
estimate deaths by age nearest birthday for the ten-year period 1 January 2005 –
31 December 2014.
Censuses of this country are infrequent. A successful census was completed on
1 January 2015, but the previous reliable census was on 1 January 2002. Both
censuses collected data on the population aged x last birthday by single years of age.
(i)
Explain why the National Statistics Office should adjust the age definition in
the census data to correspond with that of the deaths data.

Let the census population at age x last birthday on 1 January in year t be P x,t .
(ii) Derive an expression, in terms of the P x,t , for the exposed to risk for the period
covering the years 2005 to 2014 inclusive which the National Statistics Office
could use to estimate the overall death rate at age x nearest birthday.
[5]
(iii) Set out any assumptions you make in your derivation in part (ii), indicating
where in the derivation they are needed.

The death registration system in this country is being maintained, but the next census
is not planned until 2025.
(iv)

CT4 A2018–4
Discuss how you might estimate death rates at age x nearest birthday for the
calendar years 2015 and 2016.



Q6
(i)
To ensure that they follow the principle of correspondence, 
which states that a life alive time t should be included in the exposure
at age x at time t if and only if, were that life to die immediately, he
or she would be counted in the death data at age x. 
The deaths data “carry most information” when mortality rates are small,
so we adjust the census data not the deaths data.

[Total for part (i): max. 2]
Page 9
(ii)
*
Let P x , t be the population in the census aged x nearest birthday at
time t.

Then
=
P x *, t
1
( P x , t + P x − 1, t ) .
2

The required exposed-to-risk is then given by
t 2
exposed-to-risk =
10
*
*
∫ P x , u du OR ∫ P x , u du
t 1

0
Where t 1 and t 2 are the start and end times.
We require the exposed to risk for a period in which t 1 = 2005 and
t 2 = 2015. 
Using the trapezium rule 
we can approximate these exposed-to-risks as
10 *
( P x ,2005 + P x * ,2015 ) .
2

Applying the same rule for the inter-censal period 2002–2015 we have
=
P x * ,2005
10 *
3
P x ,2002 + P x * ,2015 .
13
13

Hence the exposed-to-risk for x last birthday or the period 2005-2014 is:
10  10 *
3 *
16 *
 10  10 *

*
=
 P x ,2002 + P x ,2015 + P x ,2015

 P x ,2002 + P x ,2015  .
2  13
13
13
 2  13


So the required exposed-to-risk for x nearest birthday for the period
2005–2014 is:
5  10
16

 ( P x ,2002 + P x − 1,2002 ) + ( P x ,2015 + P x − 1,2015 )  .
2  13
13


[Total for part (ii): 5]
(iii)
When adjusting the age data
we need to assume that births are uniformly distributed across the
Page 10

calendar year. 
To use the trapezium rule 
we must assume that the population varies linearly between census
dates. 
We assume that the population enumerated in the census of 1 January
2015 can be taken to be the population at the end of the calendar year
2014.

[Total for part (iii): max. 2]
(iv)
The deaths data are available for the years 2015 and 2016 and will
not need adjusting. 
The exposed-to-risk will still need adjusting from an age last
birthday basis to an age nearest birthday basis. 
To compute the population aged x last birthday in each of the
calendar years 2015 and 2016 some kind of forecasting/modelling
will be required. 
Extrapolation of the linear change at each age x between
1 January 2002 and 1 January 2015 is one option. 
Better might be to use the deaths in 2015 to estimate an adjusted
population aged x last birthday for 1 January 2016 and to use this
to estimate the exposed-to-risk at age nearest birthday for the
calendar year 2015. 
The procedure could then be iterated to produce an exposed-to-risk
for the calendar year 2016. 
This should be fairly accurate for the first two years immediately
following a census. 
Data could be gathered on births or migration in and out to improve the
estimate of the exposed to risk.

[Total for part (iv): max. 3]
[Total 12]
Answers to part (i) of this question were generally satisfactory,
though only a minority of candidates pointed out that the deaths
data “carry most information” when mortality rates are small.
Most candidates had a general idea of how to approach part (ii)
but only a minority had the details of the estimation of the
population in 2005 correct. Credit was given for part (iii) where
candidates had included the assumptions in their answers to part
Page 11
(ii) at the appropriate point in their argument. Few candidates
offered extensive answers to part (iv). Most were content to say
that some kind of modelling or extrapolation of the population
would be required. In parts (iii) and (iv) full credit could be
obtained for somewhat less than is written in this Examiners;
Report.
