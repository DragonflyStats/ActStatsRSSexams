\documentclass[a4paper,12pt]{article}

%%%%%%%%%%%%%%%%%%%%%%%%%%%%%%%%%%%%%%%%%%%%%%%%%%%%%%%%%%%%%%%%%%%%%%%%%%%%%%%%%%%%%%%%%%%%%%%%%%%%%%%%%%%%%%%%%%%%%%%%%%%%%%%%%%%%%%%%%%%%%%%%%%%%%%%%%%%%%%%%%%%%%%%%%%%%%%%%%%%%%%%%%%%%%%%%%%%%%%%%%%%%%%%%%%%%%%%%%%%%%%%%%%%%%%%%%%%%%%%%%%%%%%%%%%%%

\usepackage{eurosym}
\usepackage{vmargin}
\usepackage{amsmath}
\usepackage{graphics}
\usepackage{epsfig}
\usepackage{enumerate}
\usepackage{multicol}
\usepackage{subfigure}
\usepackage{fancyhdr}
\usepackage{listings}
\usepackage{framed}
\usepackage{graphicx}
\usepackage{amsmath}
\usepackage{chngpage}

%\usepackage{bigints}
\usepackage{vmargin}

% left top textwidth textheight headheight

% headsep footheight footskip

\setmargins{2.0cm}{2.5cm}{16 cm}{22cm}{0.5cm}{0cm}{1cm}{1cm}

\renewcommand{\baselinestretch}{1.3}

\setcounter{MaxMatrixCols}{10}

\begin{document}


PLEASE TURN OVER11
Man Life is an insurance company which only sells life insurance to males. It has
recently bought another smaller company called Mixed Life which sells business to
both males and females. The company is reviewing the premium rates it charges for
life insurance.
Man Life has records of the number of policies in force at their year end, which is
30 September, recorded by age last birthday. Mixed Life has records of the number
of policies in force on 31 December each year recorded by age last birthday for males
and age nearest birthday for females.
These are the data for the most recent years.
Man Life
Age last
birthday
Number of policies Number of policies Number of policies
30 Sept. 2015
30 Sept. 2016
30 Sept. 2017
49 4,789 4,296 4,367
50 4,953 5,009 4,809
51 5,300 5,186 5,902
Mixed Life Males
Age last
birthday
Number of policies Number of policies Number of policies
31 Dec. 2015
31 Dec. 2016
31 Dec. 2017
49 1,832 1,650 1,698
50 1,800 1,750 1,550
51 1,966 1,756 1,569
Mixed Life Females
Age nearest Number of policies Number of policies Number of policies
birthday
31 Dec. 2015
31 Dec. 2016
31 Dec. 2017
49 1,602 1,568 1,639
50 1,506 1,497 1,508
51 1,610 1,587 1,411
(i) Calculate the central exposed to risk of the combined portfolio for males aged
50 last birthday for the calendar year 2016, stating each assumption you make
at the point where you make it.

(ii) Calculate the central exposed to risk of the combined male and female
portfolio for persons aged 50 last birthday for the calendar year 2016, stating
each assumption you make at the point where you make it.

CT4 S2018–8Legislation has been brought in which means that males and females must be charged
the same premium rates for life insurance. The company is considering basing its
future premium rates on the number of deaths across the whole male and female
portfolio of the two companies at each age divided by the exposed to risk across the
combined male and female portfolio at each age.
(iii)

12
Discuss the appropriateness of the company’s approach to determining its
future premium rates.	

[Total 11]

Q11
(i)
Man Life
See the diagram above. The required exposed to risk is represented
by Area X + Area Y
Assuming that the population varies linearly over inter-census periods,

and that the data for 31 December in a year can be taken to represent
the data for 1 January the following year
Number of policies in force on 1 January 2016 (A)
= (3⁄4 * 4,953) + (1⁄4 * 5,009) = 4,967

Number of policies in force on 1 January 2017 (B)
= (3⁄4 * 5,009) + (1⁄4 *4,809) = 4,959 
Area X = 9/24 * (4,967 + 5,009) = 3,741 
Area Y = 3/24 * (5,009 + 4,959) = 1,246 
Exposed to risk = 3,741 + 1,246 = 4,987 
Page 21
Mixed Life
Assuming that the population varies linearly over inter-census periods, and
that the data for 31 December in a year can be taken to represent
the data for 1 January the following year
(ii)
Exposed to risk = 1⁄2 (1,800 + 1,750) = 1,775 
Total male exposed to risk = 4,987 + 1,775 = 6,762 

We need to adjust the age definition for the female lives.
Assuming birthdays are spread evenly over calendar years,

and that the data for 31 December in a year can be taken to represent
the data for 1 January the following year,
the number of policies in force aged 50 last birthday is equal to
0.5 * number of policies in force aged 50 nearest birthday
+ 0.5 * number of policies in force aged 51 nearest birthday 
on 31 December 2015 this is 1⁄2 (1,506 + 1610) = 1,558 
on 31 December 2016 this is 1⁄2 (1,497 + 1,587) = 1,542 
so the exposed to risk for the female lives at age 50 last birthday is
1⁄2 (1,558 + 1,542) = 1,550

Total exposed to risk of the combined portfolio is therefore
6,762 + 1,550 = 8,312
Page 22


(iii)
This approach will only work if the mix of males and females
remains the same. 
It is not clear whether this will happen in the future. 
Need to know what competitors are doing. 
Other companies may base their rates on a different mix of
in force business, or some estimate of future mix. 
Consider mortality improvements going forward, and in particular
the future development of the ratio between male and female death rates. 
What demographic does the company want to target, e.g. only males? 
Some selection effects are nullified by the fact that all companies are
required to charge unisex rates. 
The overall mix of business by gender may alter temporarily as those who
are likely to lose out by the introduction of the new legislation may make
a dash to get cover before the legislation comes into force.

[max. 3]
[Total 11]
Answers to parts (i) and (ii) were, overall, better than answers to similar
questions on this part of the syllabus in previous sessions. Answers to part
(iii) were very poor. This was the other part question which led to the
Pass Mark being reduced to 58, as almost no candidates scored more than
+2 for this part. However, it was disappointing that most candidates
seemed not to have read the question. They wrote answers arguing that
customers would switch from the company to other companies who
charged different premiums to males and females, without realising that
all companies were required by law to charge the same premiums to males
and females. Assuming this applies to new business, not to existing
business, then the common premium will be determined by the sex ratio
applied to the pricing basis and profitability by how this compares with
the mix of sales in the future.
Page 23
