\documentclass[a4paper,12pt]{article}

%%%%%%%%%%%%%%%%%%%%%%%%%%%%%%%%%%%%%%%%%%%%%%%%%%%%%%%%%%%%%%%%%%%%%%%%%%%%%%%%%%%%%%%%%%%%%%%%%%%%%%%%%%%%%%%%%%%%%%%%%%%%%%%%%%%%%%%%%%%%%%%%%%%%%%%%%%%%%%%%%%%%%%%%%%%%%%%%%%%%%%%%%%%%%%%%%%%%%%%%%%%%%%%%%%%%%%%%%%%%%%%%%%%%%%%%%%%%%%%%%%%%%%%%%%%%

\usepackage{eurosym}
\usepackage{vmargin}
\usepackage{amsmath}
\usepackage{graphics}
\usepackage{epsfig}
\usepackage{enumerate}
\usepackage{multicol}
\usepackage{subfigure}
\usepackage{fancyhdr}
\usepackage{listings}
\usepackage{framed}
\usepackage{graphicx}
\usepackage{amsmath}
\usepackage{chngpage}

%\usepackage{bigints}
\usepackage{vmargin}

% left top textwidth textheight headheight

% headsep footheight footskip

\setmargins{2.0cm}{2.5cm}{16 cm}{22cm}{0.5cm}{0cm}{1cm}{1cm}

\renewcommand{\baselinestretch}{1.3}

\setcounter{MaxMatrixCols}{10}

\begin{document}

3
(ii)
 Determine the distribution of X i for i = 2 and i = 3.
(i) List the advantages of graduation:
• by parametric formula.
• with reference to a standard table.

[Total 6]
• using a graphical method.

(ii)

CT4 A2018–2
Outline the steps involved in graduating mortality rates with reference to a
standard table.

[Total 7]

Q3
(i)
Parametric formula
The resultant graduation will be sufficiently smooth
provided few parameters are used. 
It is a suitable method to produce standard tables. 
It can be useful to fit the same formula to several experiences to give insight
into the differences between experiences.

Reference to a standard table
It can be used to fit relatively small data sets in cases where a suitable standard
table exists.

The graduated rates should be smooth provided that a simple function is used.

The standard table can provide information at extreme ages where data may be
scanty
OR
Page 5
The shape of the table can be used to “fill in gaps” in the data

It can be useful to fit the same table to several experiences with the same link
function to give insight into how the experience differs over time.

Graphical graduation
It can be used for scanty data sets where no suitable standard table exists
OR
no more sophisticated method is justifiable.

It enables an experienced analyst to allow for known (or likely) features of the
data.

It can give a quick initial feel for the rates.
(ii)

[Total for part (i): max. 4]
Select a suitable standard table.

In making this selection, consider the nature of lives of involved, and compare
their characteristics with the description of data used in a range of standard
tables; and

the date range for information used in preparation of the standard tables, with
a general preference for using data closer in date to the period for the crude
rates if possible.

Select one or more link functions to try.

Exploratory graphical or regression analysis may help with the selection of the
link function.

Estimate the parameters 
using a method such as maximum likelihood or least squares. 
Compute the graduated rates. 
Perform statistical tests on the graduation to check adherence to the data. 
If necessary, repeat some or all steps until satisfied with graduation.

[Total for part (ii): max. 3]
[Total 7]
Page 6
This question was generally well answered. In both parts, full
credit could be gained for rather less than is written in this
Examiners’ Report.
