\documentclass[a4paper,12pt]{article}

%%%%%%%%%%%%%%%%%%%%%%%%%%%%%%%%%%%%%%%%%%%%%%%%%%%%%%%%%%%%%%%%%%%%%%%%%%%%%%%%%%%%%%%%%%%%%%%%%%%%%%%%%%%%%%%%%%%%%%%%%%%%%%%%%%%%%%%%%%%%%%%%%%%%%%%%%%%%%%%%%%%%%%%%%%%%%%%%%%%%%%%%%%%%%%%%%%%%%%%%%%%%%%%%%%%%%%%%%%%%%%%%%%%%%%%%%%%%%%%%%%%%%%%%%%%%

\usepackage{eurosym}
\usepackage{vmargin}
\usepackage{amsmath}
\usepackage{graphics}
\usepackage{epsfig}
\usepackage{enumerate}
\usepackage{multicol}
\usepackage{subfigure}
\usepackage{fancyhdr}
\usepackage{listings}
\usepackage{framed}
\usepackage{graphicx}
\usepackage{amsmath}
\usepackage{chngpage}

%\usepackage{bigints}
\usepackage{vmargin}

% left top textwidth textheight headheight

% headsep footheight footskip

\setmargins{2.0cm}{2.5cm}{16 cm}{22cm}{0.5cm}{0cm}{1cm}{1cm}

\renewcommand{\baselinestretch}{1.3}

\setcounter{MaxMatrixCols}{10}

\begin{document}

6
date of birth;
date of entry into observation (if entering after 1 January 2017);
date of death (if died between 1 January 2017 and 31 December 2017);
date of exit from observation while still alive (if leaving before 31 December
2017).
(i) Derive a maximum likelihood estimator of the hazard of death which could be
used with these data and which uses all the information available on the timing
of death.
[4]
(ii)
 Explain how these data can be used to estimate a life table.
[3]
[Total 7]
A Markov jump process has the following generator matrix:
A
B
C
(i)
⎛ −0.3 0.2 0.1 ⎞
⎟
⎜
⎜ 0.1 −0.5 0.4 ⎟
⎜ 0.3 0.1 −0.4 ⎟
⎠
⎝
Draw a transition graph for this process.
[2]
The process is in state A at time zero.
(ii)
Give Kolmogorov’s forward equations for
d
d
d
P AA ( t ) ,
P AB ( t ) and
P AC ( t ) .
dt
dt
dt
[2]
(iii) Calculate the probability that the process remains in state A throughout the
period t = 0 to t = 2.
[2]
(iv)
 Determine the probability that the third jump of the process is into state C.[3]
[Total 9]

%%%%%%%%%%%%%%%%%%%%%%%%%%%%%%%%%%%%%%%%%%%%%%%%%%%%%%%%%%%%%%%%%%%%%%%%%5

Q6
(i)
[+2]

d
P AA ( t ) =
− 0.3 P AA ( t ) + 0.1 P AB ( t ) + 0.3 P AC ( t )
dt
(ii)
d
P AB ( t ) = 0.2 P AA ( t ) − 0.5 P AB ( t ) + 0.1 P AC ( t )
dt
d
P AC ( t ) = 0.1 P AA ( t ) + 0.4 P AB ( t ) − 0.4 P AC ( t )
dt
(iii)
[+2]

EITHER
To stay in state A the equation reduces to:
d
P ( t ) = − 0.3 P AA ( t )
dt AA
which has solution
Page 10



So for t = 2 we have exp(-0.6) = 0.5488.
OR

We can model this as Poisson with parameter (0.1 + 0.2)*2 = 0.6
P ( Poi (0.6)
= 0)
=
− 0.6
= e =
0.5488
(iv)
e − 0.6 0.6 0
0!



The only paths under which the third jump is into state C are BAC, CAC
and CBC.

The probabilities of each jump are given by the ratio of the transition rates.
So the probabilities for each path are:
BAC = 
CAC = 
CBC = 
Sum = 7/36 = 0.194.


[Total 9]
Parts (i)-(iii) of this question were very well answered, with many
candidates scoring full marks. Part (iv) was only answered
correctly by a minority of candidates. An alternative solution
involving writing down the transition matrix, P, of the process, and
then pointing out that the correct probability would be found in
cell {1,3} of the matrix P3 was awarded credit.
Page 11
