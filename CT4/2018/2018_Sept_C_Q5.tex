\documentclass[a4paper,12pt]{article}

%%%%%%%%%%%%%%%%%%%%%%%%%%%%%%%%%%%%%%%%%%%%%%%%%%%%%%%%%%%%%%%%%%%%%%%%%%%%%%%%%%%%%%%%%%%%%%%%%%%%%%%%%%%%%%%%%%%%%%%%%%%%%%%%%%%%%%%%%%%%%%%%%%%%%%%%%%%%%%%%%%%%%%%%%%%%%%%%%%%%%%%%%%%%%%%%%%%%%%%%%%%%%%%%%%%%%%%%%%%%%%%%%%%%%%%%%%%%%%%%%%%%%%%%%%%%

\usepackage{eurosym}
\usepackage{vmargin}
\usepackage{amsmath}
\usepackage{graphics}
\usepackage{epsfig}
\usepackage{enumerate}
\usepackage{multicol}
\usepackage{subfigure}
\usepackage{fancyhdr}
\usepackage{listings}
\usepackage{framed}
\usepackage{graphicx}
\usepackage{amsmath}
\usepackage{chngpage}

%\usepackage{bigints}
\usepackage{vmargin}

% left top textwidth textheight headheight

% headsep footheight footskip

\setmargins{2.0cm}{2.5cm}{16 cm}{22cm}{0.5cm}{0cm}{1cm}{1cm}

\renewcommand{\baselinestretch}{1.3}

\setcounter{MaxMatrixCols}{10}

\begin{document}

[Total 6]5
The following data are available for a sample of lives that were alive for at least some
time between 1 January 2017 and 31 December 2017:
•
•
•
•
6
date of birth;
date of entry into observation (if entering after 1 January 2017);
date of death (if died between 1 January 2017 and 31 December 2017);
date of exit from observation while still alive (if leaving before 31 December
2017).
(i) Derive a maximum likelihood estimator of the hazard of death which could be
used with these data and which uses all the information available on the timing
of death.

(ii)
 Explain how these data can be used to estimate a life table.

[Total 7]
A Markov jump process has the following generator matrix:
A
B
C
(i)
⎛ −0.3 0.2 0.1 ⎞
⎟
⎜
⎜ 0.1 −0.5 0.4 ⎟
⎜ 0.3 0.1 −0.4 ⎟
⎠
⎝
Draw a transition graph for this process.

The process is in state A at time zero.
(ii)
Give Kolmogorov’s forward equations for
d
d
d
P AA ( t ) ,
P AB ( t ) and
P AC ( t ) .
dt
dt
dt

(iii) Calculate the probability that the process remains in state A throughout the
period t = 0 to t = 2.

(iv)
 Determine the probability that the third jump of the process is into state C.
[Total 9]
%%%%%%%%%%%%%%%%%%%%%%%%%%%%%%%%%%%%%%%%%%%%%%%%
Q5
(i)
Assume that the hazard of death (or force of mortality) is constant
between ages x and x+1 and takes the unknown value \mu.
Probability of observing all the data we actually observe – both
censored lives and those which died is the likelihood L, which is
∏
L =
S ( t i )
all censored lives
∏
f ( t i ) ,

all lives which died
where t i is the duration for which life i is observed, and S(t i ) and f(t i )
are the survival and probability density functions of the chosen survival
distribution.

^
To obtain \mu , define a variable δ i such that
δ i = 1 if life i died
δ i = 0 if life i was censored.

Then the likelihood can be written
n n
i = 1 i = 1
L = ∏ f ( t i ) δ i S ( t i ) 1 − δ i = ∏ \mu δ i exp( − \mu t i ) .
n
∑ δ
Thus
log L
=

n
log \mu − ∑ \mu t i
i
= i 1 = i 1

We differentiate this with respect to \mu to give
n
\partial log L
=
\partial \mu
∑ δ
i = 1
\mu
i
n
− ∑ t i .

i = 1
Page 7
Setting this equal to zero produces
n
∑ δ
i = 1
\mu
i
n
= ∑ t i ,

i = 1
so that
n
^
\mu =
∑ δ
i = 1
n
∑ t
i = 1
i
.

i
n
∑
i
\partial 2 log L
i = 1
We can check that this is a maximum by noting that
,
=
−
\mu 2
\partial \mu 2
which is negative.
(ii)
δ


[max. 4]
Obtain a series of separate estimates for the different hazards
in each year of age for the calendar year 2017. 
Suppose that the maximum likelihood estimate of the constant force
during the single year of age from x to x+1 is \mu x . 
Then the probability that a person alive at exact age x will still be alive at
exact age x+1 is just S x (1) . Given the constant force, then
^
S x =
(1) exp( − \mu x ) .

In general, therefore
 m − 1 ^ 

S x ( =
m ) m =
p x exp − ∑ \mu x + j  .
 j = 0



By ‘chaining’ together the probabilities in this way, we can create
a life table from our estimates and evaluate probabilities over
any relevant age range.
Page 8


[max. 3]
[Total 7]
Part (i) of this question was from the Core Reading, Unit 6, pages
8-11. A large number of candidates based their answers on the
Poisson likelihood. This is not quite correct as, strictly speaking, it
requires observation of all lives for the same fixed period, and the
question states clearly that this is not the case. Nevertheless,
candidates could score credit for correctly deriving the correct
maximum likelihood estimator from the Poisson likelihood. On the
other hand, Binomial likelihoods were given little credit, as they do
not furnish a maximum likelihood estimator of the hazard of death
(they allow one to construct an estimator of q x ). The answer to
part (ii) given above follows that in the Core Reading, Unit 6,
pages 11-12. Some candidates framed their answers to part (ii)
around the need for graduation and smoothing. Some credit was
given for such answers.
Page 9
