CT4 S2018–3 
PLEASE TURN OVER7
(i)
Define the following types of censoring in the context of a mortality
investigation:
• random censoring;
• right censoring;
• informative censoring.

Anwen received a bunch of 17 fresh red roses on the evening of her birthday from her
boyfriend. She arranged them in a vase and placed them on the table in the garden
for all to admire. She needed to do a project for school so decided to use them to
conduct an experiment as to how long roses live before they start to wilt. She checked
them very often, and noted down the date when any was showing signs of wilting, and
immediately removed the wilting rose from the vase. The following shows what she
discovered.
Day 2.	Very disappointing, already two roses wilting.
Day 3.	A neighbour passed with his goat which took a nibble at the bunch, so three
damaged, but otherwise fresh, roses had to be removed.
Day 5. One more wilting.
Day 7. Three more wilting.
Day 8. The boy down the road stole a fresh rose to give to his sweetheart.
Day 9.	Another one wilting and it is hard to make the remaining ones look good in
the vase, so the project is terminated.
(ii)
For each of the three types of censoring listed in part (i):
(a)
State which roses (if any) experience that censoring.
(b)
Explain why those roses (if any) experience that censoring.


[Total 9]


%%%%%%%%%%%%%%%%%%%%%%%%%%%%%%%%%%%%%%%%%%%%%%%%%%%%%%%%%%%%%%%%%%%%%%%%%%%%%%%%

Q7
(i)
Random
When a subject is removed from the investigation for
a reason other than by death, and the timing of the removal
can be considered a random variable.

Right
When a subject is removed from the investigation for a reason other
than by death, so that
EITHER
we do not know exactly when death will occur,
just that it occurs after the time of removal.
OR
censoring cuts short the observation in progress.

Informative
When the future mortality of a subject censored from the investigation is
likely to be different from those remaining in the investigation.


(ii)
Random
The three eaten by the goat on day 3 and the one stolen by the boy
down the road on day 8, 
as the times of these events could not have been known in advance. 
Arguably, those remaining fresh on day 9, if we did not know at
what remaining number it would be difficult to make them look good
(otherwise it would be type II). 
Right
Day 3, Day 8 and Day 9 as listed above,

as we do not know when they would have wilted if they had not been
removed from observation, just that it would have been after the day they were
removed.

Page 12
Informative.
Arguably the one stolen on day 8
as it is likely he took the freshest-looking one to present to his girlfriend
so it may have had a longer future fresh life than the others.



[max. 6]
[Total 9]
In part (i) many candidates did not include in their answers a
definition of censoring. This omission was penalised. The
understanding of many candidates about informative censoring is
still rather shaky. Many answers were vague, such as “censoring
gives information about the remaining lives” without specifying
what that information might be. In part (ii) most candidates
identified that the roses censored on Days 3 and 8 were examples
of random censoring, and the roses left on Day 9 were examples of
right censoring. Fewer candidates spotted that the roses censored
on Days 3 and 8 were also right censoring. Fewer still gave
persuasive examples of informative censoring.
