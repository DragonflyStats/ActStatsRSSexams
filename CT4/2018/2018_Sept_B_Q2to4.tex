\documentclass[a4paper,12pt]{article}

%%%%%%%%%%%%%%%%%%%%%%%%%%%%%%%%%%%%%%%%%%%%%%%%%%%%%%%%%%%%%%%%%%%%%%%%%%%%%%%%%%%%%%%%%%%%%%%%%%%%%%%%%%%%%%%%%%%%%%%%%%%%%%%%%%%%%%%%%%%%%%%%%%%%%%%%%%%%%%%%%%%%%%%%%%%%%%%%%%%%%%%%%%%%%%%%%%%%%%%%%%%%%%%%%%%%%%%%%%%%%%%%%%%%%%%%%%%%%%%%%%%%%%%%%%%%

\usepackage{eurosym}
\usepackage{vmargin}
\usepackage{amsmath}
\usepackage{graphics}
\usepackage{epsfig}
\usepackage{enumerate}
\usepackage{multicol}
\usepackage{subfigure}
\usepackage{fancyhdr}
\usepackage{listings}
\usepackage{framed}
\usepackage{graphicx}
\usepackage{amsmath}
\usepackage{chngpage}

%\usepackage{bigints}
\usepackage{vmargin}

% left top textwidth textheight headheight

% headsep footheight footskip

\setmargins{2.0cm}{2.5cm}{16 cm}{22cm}{0.5cm}{0cm}{1cm}{1cm}

\renewcommand{\baselinestretch}{1.3}

\setcounter{MaxMatrixCols}{10}

\begin{document}

2 Describe how you would determine the number of degrees of freedom to use in a
chi-square test when graduating a set of crude mortality rates.

3 For each of the following processes:
•
•
•
•
•
simple random walk
Markov jump process
compound Poisson process
Markov chain
counting process
(a)
State whether the state space is discrete, continuous, or can be either.
(b)
State whether the time set is discrete, continuous, or can be either.

4
(i)
(ii)

CT4 S2018–2
Suggest three types of information source which could be used in
recommending parameters to use in an actuarial model.

Comment on a practical difficulty which could arise with using each type of
information source.

%%%%%%%%%%%%%%%%%%%%%%%%%%%%%%%%%%%%%%%%%%%%%%%%%%%%%%%%%%%%%%%%%%%%%%%%%%


Q2
The starting point is the number of age groups used. 
If the age groups have been chosen with reference to the data,
an unknown number of degrees of freedom should be deducted. 
Then you deduct a number of degrees of freedom depending upon
the method of graduation used. 
If a standard table is used, deduct, say, 2 degrees of freedom for the
choice of standard table (though the exact number to deduct is not
determined easily). 
If a link function is used to a standard table or a parametric formula
is used, deduct one degree of freedom per parameter estimated. 
If graphical graduation is used, deduct 2 or 3 degrees of freedom
for every 10 or so ages. 
[Total max. 4]
Page 4
This question was well answered by most candidates.
Q3
Process State space Time set Simple random walk Discrete Discrete 
Markov jump process Discrete Continuous 
Compound Poisson process Either Continuous 
Markov chain Discrete Discrete 
Counting process Discrete Either 
[Total 5]
This question was well answered. Common errors were not to
realise that a Compound Poisson process can have either a
continuous or a discrete state space, and that a Counting process
can have either a continuous or a discrete time set.
Q4
(i)
Internal data for an old model performing the same/similar function. 
Internal data for a model performing a different function 
Market observable yields or rates. 
Expert opinion. 
Industry data, for example a standard table or surveys. 
Regulations set out by regulatory authorities. 
Government statistical data.

[max. 3]
Page 5
(ii)
Internal data for an old model performing the same/similar function
Past experience may not be representative of future experience.

Internal data on a model performing a different function
Subjective adjustments may be needed.

Market observable yields or rates
May be a time delay before they become available,
OR
different sources may give slightly different rates for the same item.

Expert opinion
May be hard or expensive to find a relevant expert,
OR
the expert’s advice may be theoretical and hard to adapt
into a pragmatic model.

Industry data, for example a standard table or surveys
May not be directly relevant to the situation to be modelled,
OR
survey may be expensive
OR
the experience likely to differ by firm due to distribution
approaches/target markets, etc.,
OR
Firms for whom the assumption is insignificant may take a high-level
approach and this may not be readily apparent.

Regulations set out by regulatory authorities.
Current regulations may change in the future.

Government statistical data
Will tend to apply to the population as a whole, and the
model may apply to a non-representative subset of the population.
Page 6

[max. 3]
[Total 6]
This higher skills question proved the most demanding of any
question on the examination paper. Many candidates did not
attempt it or made only token attempts. Credit was given for
sensible answers other than those listed above, but vague answers,
such as “past experience” did not receive full credit. Some
candidates offered alternative data sources such as “reinsurers’
data” or “data from other countries” and these were given credit.
The answers to part (ii) were expected to relate to the sources
listed in part (i).
