\documentclass[a4paper,12pt]{article}

%%%%%%%%%%%%%%%%%%%%%%%%%%%%%%%%%%%%%%%%%%%%%%%%%%%%%%%%%%%%%%%%%%%%%%%%%%%%%%%%%%%%%%%%%%%%%%%%%%%%%%%%%%%%%%%%%%%%%%%%%%%%%%%%%%%%%%%%%%%%%%%%%%%%%%%%%%%%%%%%%%%%%%%%%%%%%%%%%%%%%%%%%%%%%%%%%%%%%%%%%%%%%%%%%%%%%%%%%%%%%%%%%%%%%%%%%%%%%%%%%%%%%%%%%%%%

\usepackage{eurosym}
\usepackage{vmargin}
\usepackage{amsmath}
\usepackage{graphics}
\usepackage{epsfig}
\usepackage{enumerate}
\usepackage{multicol}
\usepackage{subfigure}
\usepackage{fancyhdr}
\usepackage{listings}
\usepackage{framed}
\usepackage{graphicx}
\usepackage{amsmath}
\usepackage{chngpage}

%\usepackage{bigints}
\usepackage{vmargin}

% left top textwidth textheight headheight

% headsep footheight footskip

\setmargins{2.0cm}{2.5cm}{16 cm}{22cm}{0.5cm}{0cm}{1cm}{1cm}

\renewcommand{\baselinestretch}{1.3}

\setcounter{MaxMatrixCols}{10}

\begin{document}CT4 S2018–48
A drug named Nimble is often prescribed to the elderly for periods greater than a year
to reduce the pain of arthritis. It has been proven that taking Nimble increases the
chance of death by heart disease. One local health authority is looking into whether
the increased risk continues once a patient stops taking the drug. It proposes to use
a model with five states: (1) Never taken Nimble, (2) Taking Nimble, (3) No longer
taking Nimble, (4) Dead through heart disease, (5) Dead through other causes.
(i)
Draw a diagram showing the possible transitions between the five states.

Let the transition intensity between state i and state j at time x + t be \mu ijx + t . Let the
probability that a person in state i at time x will be in state j at time x + t be t p xij .
(ii)
Show from first principles that
d
34
32 24
33 34
t p x = t p x \mu x + t + t p x \mu x + t
dt

(
)
The health authority wishes to calculate death rates aged 50 last birthday from each
cause for individuals in states 1, 2 and 3.
(iii)

CT4 S2018–5 
State what data would need to be extracted from the health authority’s medical
records in order to do this.

[Total 10]


%%%%%%%%%%%%%%%%%%%%%%%%%%%%%%%%%%%%%%%%%%%%%%%%%%%%%%%%%%%%%%%%%%%%%%%%%%%%%%%%%%%%%%

Page 13
Q8
(i)
[+2]

(ii)
Using the Markov assumption
OR
the Chapman Kolmogorov equation is
dt + t
32
24
33
34
34
44
35
54
p x 34 = t p x 31 dt p 14
x + t + t p x dt p x + t + t p x dt p x + t + t p x dt p x + t + t p x dt p x + t .
Since
dt + t

dt
54
p =
x + t
t
=
p x 31 0

34
44
p x 34 = t p x 32 dt p x 24 + t + t p x 33 dt p 34
x + t + t p x dt p x + t .
Given that
dt
p x 44 + t = 1


And assuming that, for small dt
dt
p ij x + t =
\mu ij x + t dt + o ( dt )
i ≠ j
o ( dt )
= 0 ,
dt → 0 dt
where lim


then substituting, we have
dt + t
33 34
34
p x 34 = t p x 32 \mu 24
x + t dt + t p x \mu x + t dt + t p x + o ( dt )
so that
Page 14
dt + t
33 34
p x 34 − t p x 34 = t p x 32 \mu 24
x + t dt + t p x \mu x + t dt + o ( dt )


p 34
− t p x 34
d
34
t
dt
x
+
and hence
( t p x ) = lim
=
dt → 0
dt
dt
t
33 34
p x 32 \mu 24
x + t + t p x \mu x + t . 
[max. 5]
(iii)
EITHER INDIVIDUAL-LEVEL DATA
If data are held at an individual level we would need , for
the period of the investigation 
dates the individual moved into and out of the local
health authority (if such movement took place) 
date the individual attained exact age 50 or 51 years
(or date of birth) 
dates the individual started taking Nimble 
dates the individual stopped taking Nimble 
date the individual died 
cause of death. 
whether the individual had taken Nimble before (or date the
individual first took Nimble). 
OR AGGREGATE-LEVEL DATA
If data is held at an aggregate level we would need
the amount of time spent within the investigation period 
by lives aged between ages 50 and 51 years exact 
for each of the states “Never taken Nimble”, “Taking Nimble”
and “No longer taking Nimble”
. 
The number of deaths 
from each of the three states “Never taken Nimble”,
“Taking Nimble” and “No longer taking Nimble” 
split by cause of death (heart disease or not)

[max. 3]
[Total 10]
Page 15
In part (i), most candidates wrote down a transition graph which
was either correct or close to being correct. Answers to part (ii)
were better than answers to similar questions in previous sessions.
In part (iii) most candidates provided answers assuming
aggregate-level data. Candidates could assume either individual-
level or aggregate-level data, but credit was not awarded from
both alternatives given in the solutions. Candidates were expected
to refer to the specific scenario in the question, so vague answers
of the form ‘number of transitions from i to j scored little credit.
