\documentclass[a4paper,12pt]{article}

%%%%%%%%%%%%%%%%%%%%%%%%%%%%%%%%%%%%%%%%%%%%%%%%%%%%%%%%%%%%%%%%%%%%%%%%%%%%%%%%%%%%%%%%%%%%%%%%%%%%%%%%%%%%%%%%%%%%%%%%%%%%%%%%%%%%%%%%%%%%%%%%%%%%%%%%%%%%%%%%%%%%%%%%%%%%%%%%%%%%%%%%%%%%%%%%%%%%%%%%%%%%%%%%%%%%%%%%%%%%%%%%%%%%%%%%%%%%%%%%%%%%%%%%%%%%

\usepackage{eurosym}
\usepackage{vmargin}
\usepackage

{amsmath}
\usepackage{graphics}
\usepackage{epsfig}
\usepackage{enumerate}
\usepackage{multicol}
\usepackage{subfigure}
\usepackage{fancyhdr}
\usepackage{listings}
\usepackage{framed}
\usepackage{graphicx}
\usepackage{amsmath}
\usepackage{chngpage}

%\usepackage{bigints}
\usepackage{vmargin}

% left top textwidth textheight headheight

% headsep footheight footskip

\setmargins{2.0cm}{2.5cm}{16 cm}{22cm}{0.5cm}{0cm}{1cm}{1cm}

\renewcommand{\baselinestretch}{1.3}

\setcounter{MaxMatrixCols}{10}

\begin{document}
[Total 11]10
A religious organisation maintains two lists of members:
•
•
a list of sick members, so that members may pray for them (the Sick List);
a list of recently deceased members (the Dead List).
Each list is published in a bulletin given to those attending the regular weekly
meetings of religious worship. The lists are updated each week half way between the
religious worship meetings.
A study was made of the mortality of sick members. A sample of members joining
the Sick List in the first quarter of 2016 was followed until they left the list. Those
who left the list but who did not move to the Dead List were assumed to have
recovered. The study terminated on 31 March 2017.
Below are given some data from the study. ‘Week first appeared on Sick List’ and
'Week last appeared on Sick List' are measured in weeks from the first week of 2016.
Member
number Week first
appeared on
Sick List Week last
appeared on
Sick List Outcome
1 1 1 Assumed recovered
2 1 3 Moved to Dead List
3 3 4 Moved to Dead List
4 3 65 Still on Sick List 31 March 2017
5 6 17 Moved to Dead List
6 7 14 Assumed recovered
7 9 11 Assumed recovered
8 10 60 Moved to Dead List
9 11 11 Moved to Dead List
10 12 65 Still on Sick List 31 March 2017
(i) Calculate the central death rate per week of these members using the exact
exposed to risk.

(ii) Determine the probability of survival for 52 weeks using your result from
part (i).

One member of the organisation has been studying statistics and recommends using
the Nelson-Aalen estimator to calculate the probability of survival for 52 weeks.
(iii) Calculate the Nelson-Aalen estimate of S(52).
(iv)
 Comment on your results in parts (ii) and (iii).
CT4 S2018–7 

[Total 12]



Q10
(i)
As the sick list is updated weekly, we can assume
that events (falling sick, recovering, dying) take place mid-week,
so that
duration of sickness = last week – first week + 1

The exposed to risk is therefore calculated as shown below
Member number
Duration
1
2
3
4
5
6
7
8
9
10
1
3
2
63
12
8
3
51
1
54
Total
(ii)
Outcome (needed for part
(iii))
Assumed recovered
Died
Died
Still alive
Died
Assumed recovered
Assumed recovered
Died
Died
Still alive
198
[+2]
There were 5 deaths, 
so the death rate is 5/198 = 0.02525 per week. 

 52

p = exp  − ∫ \mu dt  = e − 0.02525(52) = 0.26898.
 0



(iii)
The Nelson-Aalen estimate calculations are shown in the table below.
We adopt the convention that censoring happened immediately after
death where censored observations and deaths have the same duration.
t j
0
1
n j d j c j d j /n j ∑ ( d
10
10 1 1 1/10 0.1000
j
/ n j ) exp[ − ∑ ( d j / n j )]
0.9048
Page 19
2
3
12
51
8
7
4
3
1
1
1
1
0
2
0
2
1/8
1/7
1/4
1/3
0.2250
0.3679
0.6179
0.9512
0.7985
0.6922
0.5391
0.3863
[+4]
Since we have information up to week 65, 
the Nelson-Aalen estimate of S(52) is 0.3863. 

(iv)
The Nelson-Aalen estimate of the one-year survival probability is
higher than that obtained using the exact exposed to risk. 
The exact exposed to risk approach constrains the death rate
to be constant over the 52 weeks at the “average” rate implied by the
number of deaths and exposed to risk. 
The Nelson-Aalen estimate allows the death rate to vary with time
according to the data. 
The sample size is very small so the results are not likely to be
reliable. 
The group of lives being considered is very varied, so we do not
have a homogeneous group. 
The lives under observation are by definition sick, so the rates
we are coming out with are very high. 
[max. 2]
[Total 12]
In part (i) a common error was to fail to add +1 to the
difference between the last week and the first week. This
produced an exposed to risk of 188 weeks and a death rate
of 0.0266 per week. This was penalised by the loss of 1
mark. Candidates who made errors in part (i) which were
carried forward into parts (ii) and (iii) were not penalised
again in those later parts. Many candidates made only
token efforts at part (iv). This was one of the part questions
which led to the Pass Mark being reduced to 58, as almost
no candidates scored more than +1 for this part.
Page 20
