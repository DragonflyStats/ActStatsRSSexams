\documentclass[a4paper,12pt]{article}

%%%%%%%%%%%%%%%%%%%%%%%%%%%%%%%%%%%%%%%%%%%%%%%%%%%%%%%%%%%%%%%%%%%%%%%%%%%%%%%%%%%%%%%%%%%%%%%%%%%%%%%%%%%%%%%%%%%%%%%%%%%%%%%%%%%%%%%%%%%%%%%%%%%%%%%%%%%%%%%%%%%%%%%%%%%%%%%%%%%%%%%%%%%%%%%%%%%%%%%%%%%%%%%%%%%%%%%%%%%%%%%%%%%%%%%%%%%%%%%%%%%%%%%%%%%%

\usepackage{eurosym}
\usepackage{vmargin}
\usepackage{amsmath}
\usepackage{graphics}
\usepackage{epsfig}
\usepackage{enumerate}
\usepackage{multicol}
\usepackage{subfigure}
\usepackage{fancyhdr}
\usepackage{listings}
\usepackage{framed}
\usepackage{graphicx}
\usepackage{amsmath}
\usepackage{chngpage}

%\usepackage{bigints}
\usepackage{vmargin}

% left top textwidth textheight headheight

% headsep footheight footskip

\setmargins{2.0cm}{2.5cm}{16 cm}{22cm}{0.5cm}{0cm}{1cm}{1cm}

\renewcommand{\baselinestretch}{1.3}

\setcounter{MaxMatrixCols}{10}

\begin{document}
 Institute and Faculty of Actuaries1
A Markov Chain has the following transition matrix:
⎛
⎜
⎝
A ⎛ 0 0.6 0.4
B ⎜ 0.75 0 0.25
C ⎝ 0.5 0.5 0
(i)
Draw a transition graph for this Markov Chain, including the transition rates.

(ii)
Explain whether the Markov Chain is:
(a)
irreducible
(b)
periodic




%%%%%%%%%%%%%%%%%%%%%%%%%
Q1
(i)
0.6
A
B
0.75
0.4
0.5
0.5
0.25
C
[Total for part (i): 2]
(ii)
The chain is irreducible 
because every state can eventually be reached from every other state. 
The chain is aperiodic 
because it can return to each state in multiples of 2 or 3, giving no overall
period. 
[Total for part (ii): 2]
[Total 4]

% This straightforward question was well answered. In part (ii), a minority of candidates thought that the Markov chain was periodic
% with period 2. Some candidates wrote that the Markov chain was aperiodic but gave vague or incorrect explanations.
\end{document}
