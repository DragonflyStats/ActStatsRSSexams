\documentclass[a4paper,12pt]{article}

%%%%%%%%%%%%%%%%%%%%%%%%%%%%%%%%%%%%%%%%%%%%%%%%%%%%%%%%%%%%%%%%%%%%%%%%%%%%%%%%%%%%%%%%%%%%%%%%%%%%%%%%%%%%%%%%%%%%%%%%%%%%%%%%%%%%%%%%%%%%%%%%%%%%%%%%%%%%%%%%%%%%%%%%%%%%%%%%%%%%%%%%%%%%%%%%%%%%%%%%%%%%%%%%%%%%%%%%%%%%%%%%%%%%%%%%%%%%%%%%%%%%%%%%%%%%

\usepackage{eurosym}
\usepackage{vmargin}
\usepackage{amsmath}
\usepackage{graphics}
\usepackage{epsfig}
\usepackage{enumerate}
\usepackage{multicol}
\usepackage{subfigure}
\usepackage{fancyhdr}
\usepackage{listings}
\usepackage{framed}
\usepackage{graphicx}
\usepackage{amsmath}
\usepackage{chngpage}

%\usepackage{bigints}
\usepackage{vmargin}

% left top textwidth textheight headheight

% headsep footheight footskip

\setmargins{2.0cm}{2.5cm}{16 cm}{22cm}{0.5cm}{0cm}{1cm}{1cm}

\renewcommand{\baselinestretch}{1.3}

\setcounter{MaxMatrixCols}{10}

\begin{document}
%%PLEASE TURN OVER
10
ForLawn is a new treatment for lawns which, according to the manufacturer, kills
moss within days. A study was done in a small town in rural England in which
14 residents with mossy lawns participated. Seven, chosen at random, were given
ForLawn in a plain bottle and treated their lawns with it one morning. The other
seven were given the most popular moss treatment already on the market, also in a
plain bottle, and treated their lawns with it on the same morning. All 14 were asked
to assess their lawns each morning following treatment until all the moss had gone.
The study ended after 16 days, at which time any lawn which still had moss was
considered to be censored.
Unfortunately a flock of sheep escaped from the fields adjacent to the town on days 5
and 8 and made such a mess of five of the lawns in the study that these lawns had to
be withdrawn. These five lawns are also considered to be censored.
The table below shows how many days elapsed before all the moss disappeared for
each of the 14 lawns, or until censoring. Censoring is denoted by an *.
ForLawn group:
Alternative treatment group:
5, 6, 8, 8*, 8*, 11, 16*
3, 4, 5, 5*, 5* 5*, 10
(i) Describe THREE types of censoring present in this study, giving examples of
how they occur.

(ii) Calculate the Kaplan-Meier estimate of the survival functions of still having
moss for each of the two groups separately.

(iii) Sketch the two estimated survival functions on the same graph.

(iv) Comment on your results.

A local statistician suggests that it would be a good idea to use Cox regression to
compare the effectiveness of ForLawn with the alternative treatment. She thinks that
using a dummy variable, X, with the value 1 for ForLawn and 0 for the alternative
treatment would be suitable.
(v) Determine the equations for the hazard function for the lawns treated with
ForLawn and those treated with the alternative treatment, defining all the terms
you use.

(vi)
 Derive the partial likelihood of the data in this Cox model.
END OF PAPER
CT4 A2018–8

[Total 19]

%%%%%%%%%%%%%%%%%%%%%%%%%%%%%%%%%%%%%%%%%%%%%%%%%%%%%%%%%%%%%%%%%%%%%%%%%%%%5
Q10
(i)
Random censoring when the time at which the life is censored is a random
variable

of the lawns damaged by sheep. 
Right censoring when the censoring mechanism cuts short the
observations in progress 
such as when the sheep damage the lawns or any lawns still mossy when
the study ends after 16 days. 
Type 1 censoring when the censoring times are known in advance 
of the lawns still mossy after 16 days. 
Interval censoring, 
because lawns are only checked once per day, so we do not know
when in the day the moss disappeared. 
Non-informative censoring,
if the destruction of lawns by sheep gives no information about when they
might have been free of moss.
Page 20
Informative censoring,
if sheep are attracted to lawns where the moss has started to die.
[Total for part (i): max. 3]
(ii)
N j
t j
d j c j \lambda j (1 – \lambda j )
0
0
2
1 1/7
1/6
1/5
1/2 6/7
5/6
4/5
1/2
ForLawn
5
6
8
11 7
6
5
2 1
1
1
1
  

[+2]
Alternative treatment
3
4
5
10 7
6
5
1 1
1
1
1
  
0
0
3
0
1/7
1/6
1/5
1
6/7
5/6
4/5
0

[+2]
From which we obtain the survival functions as follows:
ForLawn
t S(t)
0 ≤ t < 5
5 ≤ t < 6
6 ≤ t < 8
8 ≤ t < 11
11 ≤ t < 16 1
6/7
5/7
4/7
2/7
 

Alternative treatment
t S(t)
0 ≤ t < 3
3 ≤ t < 4
4 ≤ t < 5
5 ≤ t < 10
10 ≤ t < 16 1
6/7
5/7
4/7
0
Page 21



[Total for part (ii): 6]
(iii)
1.2
1
0.8
0.6
For Lawn
0.4
Alternative
treatment
0.2
0
0
5
10
15
20
t (days)
[+2]
[Total for part (iii): 2]
(iv)
ForLawn does not seem to be better than the alternative treatment. 
Indeed, the survival function for the alternative treatment is below
that of ForLawn at most durations, indicating that moss lasts less
long under the alternative treatment than it does when treated
with For Lawn. 
However the study is small, so the difference may not be statistically
significant at conventional levels.

[Total for part (iv): 2]
(v)
For ForLawn we have
h ( t | X = 1) = h 0 ( t ) exp β .

For the alternative treatment we have
h ( t | X = 0) = h 0 ( t ).

Where h(t) is the hazard, h 0 (t) is the baseline hazard, β is the
coefficient measuring the impact of the different treatments and X is a
dummy variable taking the value 1 for ForLawn and 0 for the alternative
treatment.

[Total for part (v): 2]
(vi)
Page 22
The contributions to the partial likelihood, L, at durations where events occur
are:
1
t = 3

7 + 7e β
1
t = 4

6 + 7e β
e β
t = 5
(using Breslow correction for ties)
(5 + 7 e β ) 2

e β
t = 6

1 + 6 e β
e β
t = 8

1 + 5 e β
1
t = 10

1 + 2e β
e β
t = 11
β
2 e
=
1
2

Multiplying these elements together gives
L =
1
β
7 + 7e
.
1
β
6 + 7e
.
e β
β 2
(5 + 7 e )
.
e β
β
1 + 6 e
.
e β
β
1 + 5 e
.
1
. .

1 + 2e 2
[Total for part (vi): max. 4]
[Total 19]
1
β
This question was generally well answered, with many candidates
scoring 12 marks or more. Answers to parts (i) – (iii) were especially
good: many candidates scored full credit for these parts. In part (ii),
some candidates did not realise that we have no information after t =
16, so the ranges for the last interval of the survival function for
ForLawn should be 11 ≤ t < 16, not just 11 ≤ t. In part (v) many
candidates simply wrote down the general form of the Cox model,
rather than determining the equations of the model for the two groups
in this specific study. Answers to part (vi) fell into two groups:
candidates who made little or no attempt, and candidates whose
derivations were largely correct. Some of those who had the correct
terms did not state to which time interval each term applied, and a
small number of marks were lost for this.
Page 23
END OF EXAMINERS’ REPORT
Page 24
\end{document}
