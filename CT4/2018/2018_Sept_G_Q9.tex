\documentclass[a4paper,12pt]{article}

%%%%%%%%%%%%%%%%%%%%%%%%%%%%%%%%%%%%%%%%%%%%%%%%%%%%%%%%%%%%%%%%%%%%%%%%%%%%%%%%%%%%%%%%%%%%%%%%%%%%%%%%%%%%%%%%%%%%%%%%%%%%%%%%%%%%%%%%%%%%%%%%%%%%%%%%%%%%%%%%%%%%%%%%%%%%%%%%%%%%%%%%%%%%%%%%%%%%%%%%%%%%%%%%%%%%%%%%%%%%%%%%%%%%%%%%%%%%%%%%%%%%%%%%%%%%

\usepackage{eurosym}
\usepackage{vmargin}
\usepackage{amsmath}
\usepackage{graphics}
\usepackage{epsfig}
\usepackage{enumerate}
\usepackage{multicol}
\usepackage{subfigure}
\usepackage{fancyhdr}
\usepackage{listings}
\usepackage{framed}
\usepackage{graphicx}
\usepackage{amsmath}
\usepackage{chngpage}

%\usepackage{bigints}
\usepackage{vmargin}

% left top textwidth textheight headheight

% headsep footheight footskip

\setmargins{2.0cm}{2.5cm}{16 cm}{22cm}{0.5cm}{0cm}{1cm}{1cm}

\renewcommand{\baselinestretch}{1.3}

\setcounter{MaxMatrixCols}{10}

\begin{document}

PLEASE TURN OVER9
(i) Describe why a mortality experience would need to be graduated.
(ii) Describe how smoothness is achieved when using the following graduation
methods:

(a)
parametric formula;
(b)
graphical;
(c)
with reference to a standard table.

An insurance company conducts an investigation into the mortality rates of
policyholders who choose to retire at a relatively young age.
The following table shows data from the investigation, together with graduated rates
o
s
q x which were fitted with reference to standard table rates, q x using a link function
o
s
q x = q x + constant.
(iii)

CT4 S2018–6
o
Age x Exposed to risk Deaths q x
55
56
57
58
59
60
61
62
63
64 1,550
2,100
2,300
2,450
2,700
3,250
3,100
3,450
3,600
3,750 15
18
15
21
18
29
25
30
45
41 0.00673
0.00689
0.00709
0.00736
0.00770
0.00820
0.00891
0.00978
0.01084
0.01210
Test the goodness-of-fit of the graduated rates using a chi-square test.
Q9
(i)
We believe that mortality varies smoothly with age
OR
evidence from large experiences suggests mortality varies
smoothly with age
. 
Therefore the crude estimate of mortality at any age carries
information about mortality at adjacent ages. 
By smoothing the experience, we can make use of data at
adjacent ages to improve the estimates at each age. 
This reduces sampling (or random) errors. 
The mortality experience may be used in financial calculations. 
Irregularities, jumps and anomalies in financial quantities
(such as premiums for life insurance contracts) are hard to justify to
customers
OR
jumps and anomalies in financial quantities may be taken advantage of
by customers.
(ii)


Parametric formula
Rates are automatically smooth provided that a formula with
sufficiently few parameters is used.

Graphical
Reliance is placed on the skill of the practitioner
to draw a sufficiently smooth line through the crude rates. 
The third differences test for smoothness is useful here. 
Page 16
It is usually necessary to make several attempts,
and to adjust the results by hand (rather than re-drawing the curve),
a process called hand-polishing.

With reference to a standard table
A standard table will already be smooth. 
Provided a link function is selected with few parameters,
this smoothness should be preserved in the graduated rates. 
[max. 3]
(iii)
The null hypothesis is that the graduated rates are the same as the true
underlying mortality rates for this block of business.
The test statistic

∑ z x 2 ≈ χ 2 m where m is the degrees of freedom.
x
Age Exposed
to risk
55
56
57
58
59
60
61
62
63
64 1550
2100
2300
2450
2700
3250
3100
3450
3600
3750
Observed
deaths Graduated
rates Expected
deaths zx
15
18
15
21
18
29
25
30
45
41 0.00673
0.00689
0.00709
0.00736
0.00770
0.00820
0.00891
0.00978
0.01084
0.01210 10.432
14.469
16.307
18.032
20.790
26.650
27.621
33.741
39.024
45.375 1.41449
0.92828
-0.32366
0.69894
-0.61190
0.45522
-0.49871
-0.64403
0.95663
-0.64949
Total
zx 2
2.00079
0.86170
0.10476
0.48852
0.37442
0.20722
0.24871
0.41478
0.91514
0.42183
6.0378
[+11⁄2]
The observed test statistic is 6.0378 
The degrees of freedom are 10 minus an unknown number for the
choice of standard table (say 2) and a further one for the parameter in
the link function. 
So m = 7 say (but could also use 6 or 8 degrees of freedom) 
Page 17
The critical value of the χ 2 distribution with 7 degrees of
freedom at the 95% significance level is 14.07 (6 d.f.12.59, 8 d.f. 15.51) 
Since 6.0378 < 14.07 (or 12.59, or 15.51 
We have insufficient evidence to reject the null hypothesis. 

[Total 11]
This question was reasonably well answered by most
candidates. The weakest section was part (ii). Few
candidates realised that a parametric formula would
automatically furnish smoothness if the number of parameters
was small. Similarly, few mentioned that a standard table will
already be smooth, so the requirement is to find a link
function capable of transferring that smoothness to the
graduated rates. In part (iii), many candidates did not apply
the approach outlined in the solution to Q2 to the
determination of the number of degrees of freedom.
Page 18
