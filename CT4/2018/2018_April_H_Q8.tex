\documentclass[a4paper,12pt]{article}

%%%%%%%%%%%%%%%%%%%%%%%%%%%%%%%%%%%%%%%%%%%%%%%%%%%%%%%%%%%%%%%%%%%%%%%%%%%%%%%%%%%%%%%%%%%%%%%%%%%%%%%%%%%%%%%%%%%%%%%%%%%%%%%%%%%%%%%%%%%%%%%%%%%%%%%%%%%%%%%%%%%%%%%%%%%%%%%%%%%%%%%%%%%%%%%%%%%%%%%%%%%%%%%%%%%%%%%%%%%%%%%%%%%%%%%%%%%%%%%%%%%%%%%%%%%%

\usepackage{eurosym}
\usepackage{vmargin}
\usepackage{amsmath}
\usepackage{graphics}
\usepackage{epsfig}
\usepackage{enumerate}
\usepackage{multicol}
\usepackage{subfigure}
\usepackage{fancyhdr}
\usepackage{listings}
\usepackage{framed}
\usepackage{graphicx}
\usepackage{amsmath}
\usepackage{chngpage}

%\usepackage{bigints}
\usepackage{vmargin}

% left top textwidth textheight headheight

% headsep footheight footskip

\setmargins{2.0cm}{2.5cm}{16 cm}{22cm}{0.5cm}{0cm}{1cm}{1cm}

\renewcommand{\baselinestretch}{1.3}

\setcounter{MaxMatrixCols}{10}

\begin{document}CT4 A2018–5 
PLEASE TURN OVER8
(i)
State why the Gompertz model is often used in analyses of human mortality.

The following data are taken from an investigation of the mortality of males aged
60–70 years inclusive in a developed country.
Age (years) x\mu x Deaths Exposed-to-risk
60
61
62
63
64
65
66
67
68
69
70 0.02029
0.02230
0.02466
0.02721
0.02937
0.03102
0.03194
0.03055
0.04297
0.04405
0.04749 49
51
55
68
70
67
69
66
84
88
83 2,415
2,287
2,230
2,499
2,383
2,160
2,160
2,160
1,955
1,998
1,748
(ii) Determine the parameters of the Gompertz model using the data for ages 60
and 70 years only.

(iii) Test the overall fit of the model you estimated in part (ii) using data from
ages 61–69 years only.
(iv)

CT4 A2018–6
Comment on your results in part (iii).


%%%%%%%%%%%%%%%%%%%%%%%%%%%%%%%%%%%%%%%%%%%%%%%%%%%%%%%%%%%%%%%%%%%%%%%%%%%%%%
Q8
(i)
Page 14
The Gompertz model has been shown to approximate human
mortality closely in the middle and/or older ages in human populations. 
The Gompertz model is simple to understand, 
and is easy to fit.
(ii)

[Total for part (i): 2]
EITHER
The Gompertz model is
Bc x .
\mu x =
Hence
\mu x Bc x
= =
c x − y .
y
\mu y Bc 
\mu 70
= c 10
\mu 60 
Using the values for ages 60 and 70 years we have
c = 10
0.04749
0.02029
So c = 1.08876.

Hence
log e B log e 0.02029 − 60(0.08504)
=
and B = 0.0001234.

OR
The Gompertz model is
\mu
=
x exp( α 0 + α 1 x )
Hence
log e\mu x = α 0 + α 1 x

Using the values for ages 60 and 70 years we have
log e 0.02029 =α 0 + 60 α 1
log e 0.04749 =α 0 + 70 α 1

Page 15
Hence
log e =
0.04749 log e 0.02029 + 10 α 1
=
α 1
log e (0.04749 / 0.02029)
= 0.085039 .
10

So that
α 0 = log e 0.02029 − 60(0.08504) =− 9.0000
(iii)

[Total for part (ii): 3]
We compute the actual and expected deaths in the table below.
EITHER using\mu x
Age x Actual
deaths Gompertz
\mu x Expected
deaths z x z x 2
61
62
63
64
65
66
67
68
69 51
55
68
70
67
69
66
84
88 0.02209
0.02405
0.02619
0.02851
0.03104
0.03380
0.03680
0.04006
0.04362 50.52
53.64
65.44
67.94
67.05
73.00
79.48
78.32
87.15 0.0673
0.1863
0.3164
0.2498
–0.0061
–0.4683
–1.5121
0.6416
0.0911 0.0045
0.0347
0.1001
0.0624
0.0000
0.2193
2.2864
0.4116
0.0083
Actual Gompertz Expected z x z x 2
deaths\mu x + 1/2 deaths 51
55
68
70
67
69
66
84
88 0.02305
0.02510
0.02732
0.02975
0.03239
0.03526
0.03839
0.04180
0.04551 52.72
55.97
68.28
70.89
69.96
76.17
82.93
81.72
90.94 -0.2364
-0.1290
-0.0342
-0.1060
–0.3541
–0.8217
–1.8594
0.2517
-0.3078 0.0559
0.0166
0.0012
0.0112
0.1254
0.6752
3.4572
0.0634
0.0947
OR using\mu x + 1/2
Age x
61
62
63
64
65
66
67
68
69
[+2]
Page 16
The test is the chi-squared test, and the test statistic is
∑ z x
2

x
(iv)
The null hypothesis is that the observed deaths come from a population
in which the underlying mortality is described by the fitted model in (ii). 
The calculated value of the test statistic is 3.1267.
OR,
using\mu x + 1/2 it is 4.5009 
We compare this with the critical value of the chi-squared distribution
with 9 degrees of freedom at the 95% level, 
because we have not used the data involved in the test
to estimate the expected deaths. 
The critical value is 16.92. 
Since 3.1267(4.5009) < 16.92, 
we have no reason to reject the null hypothesis. 
[Total for part (iii): 6]
The Gompertz model seems to fit the data well. 
However there is a relatively large negative deviations at age 67 years

The exposed-to-risk is exactly the same at ages 65 to 67 years.
Could there be a transcription error in the data?

The model might be a better fit if MLE or weighted least squares had
been used to fit the parameters.

[Total for part (iv): max. 1]
[Total 12]
Parts (i) and (ii) of this question were well answered, with the majority
of candidates correctly determining the parameters of the Gompertz
model. Answers to part (iii) varied. Common errors included using the
data given in the question, rather than the fitted model, to estimate the
expected deaths, and hence effectively carrying out a chi-squared test
on the rounding errors (unsurprisingly, the fit was found to be
extremely good). Few candidates realised that, as the data for ages 61-
69 years had not been used to estimate the parameters of the Gompertz
distribution, and we can suppose the estimated \mu x s for each age to have
been obtained independently, it was not necessary to deduct two
degrees of freedom when carrying out the chi - squared test (it would
Page 17
have been necessary had data for all ages been used to estimate the
parameters).
