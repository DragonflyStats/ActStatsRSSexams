\documentclass[a4paper,12pt]{article}

%%%%%%%%%%%%%%%%%%%%%%%%%%%%%%%%%%%%%%%%%%%%%%%%%%%%%%%%%%%%%%%%%%%%%%%%%%%%%%%%%%%%%%%%%%%%%%%%%%%%%%%%%%%%%%%%%%%%%%%%%%%%%%%%%%%%%%%%%%%%%%%%%%%%%%%%%%%%%%%%%%%%%%%%%%%%%%%%%%%%%%%%%%%%%%%%%%%%%%%%%%%%%%%%%%%%%%%%%%%%%%%%%%%%%%%%%%%%%%%%%%%%%%%%%%%%

\usepackage{eurosym}
\usepackage{vmargin}
\usepackage{amsmath}
\usepackage{graphics}
\usepackage{epsfig}
\usepackage{enumerate}
\usepackage{multicol}
\usepackage{subfigure}
\usepackage{fancyhdr}
\usepackage{listings}
\usepackage{framed}
\usepackage{graphicx}
\usepackage{amsmath}
\usepackage{chngpage}

%\usepackage{bigints}
\usepackage{vmargin}

% left top textwidth textheight headheight

% headsep footheight footskip

\setmargins{2.0cm}{2.5cm}{16 cm}{22cm}{0.5cm}{0cm}{1cm}{1cm}

\renewcommand{\baselinestretch}{1.3}

\setcounter{MaxMatrixCols}{10}

\begin{document}

5
Describe the factors which should be considered when deciding whether to
consider time in a discrete or continuous way for a model.

[Total 7]
The calculation of the daily unit price for a fund investing in commercial properties
can be done on one of two bases:
• Bid price – reflecting the price at which the properties could be sold, allowing for
the transaction costs for selling properties.
• Offer price – reflecting the price at which properties could be purchased, again
allowing for the transaction costs which would apply.
Whether the fund is priced on the Bid or Offer basis depends on whether there is net
investment into or redemption from the fund.
Movements between the states Bid (B) and Offer (O) pricing basis are to be modelled
using a Markov Jump Process with constant transition rates from Bid to Offer of l and
Offer to Bid of \mu.
(i) Give the generator matrix of the Markov Jump Process. 
(ii) State the distribution of holding times in each state. 
Let t P sij be the probability that the process is in state j at time s + t given that it was in
state i at time s (i, j = B, O).
(iii)
Write down Kolmogorov’s forward equations for
d
d
BB
BO
. 
t P s and
t P s
dt
dt

The fund was being priced on a Bid price basis at time s.
(iv)

Solve the Kolmogorov equations to obtain an expression for t P sBB . 
CT4 A2018–3 

[Total 9]


Q5
(i)
Writing the state space in the order {Bid (B), Offer (O)},

the generator matrix is:
B  −\lambda \lambda 

 .
O \mu −\mu 

[Total for part (i): 1]
(ii)
The holding times are exponentially distributed with 
parameter \lambda in state B, 
and\mu in state O.
(iii)

[Total for part (ii): 2]
\partial
BB
= −\lambda . t P s BB +\mu . t P s BO .
t P s
\partial t 
\partial
BO
= \lambda . t P s BB −\mu . t P s BO .
t P s
\partial t 
[Total for part (iii): 3]
(iv)
Page 8
We have a two-state model so:
BB
t P s
+ t P s BO =
1 .

Substituting:
\partial
BB
= −\lambda . t P s BB +\mu .(1 − t P s BB ) ;
t P s
\partial t 
\partial 
exp(( \lambda +\mu ) t ). t P s BB  =\mu .exp(( \lambda +\mu ) t ) ;

\partial t  
and hence
exp(( \lambda + \mu
=
) t ). t P s BB
\mu
.exp(( \lambda +\mu ) t ) + constant.
\lambda+\mu

Since the process is in state Bid at time s (i.e. t = 0),
\lambda
,
the constant is
\mu+\lambda
BB
and thus t P
=
s

\mu
\lambda
+
.exp( − ( \lambda +\mu ) t ) .
\lambda+\mu \lambda+\mu

[Total for part (iv): 4]
[Total 9]
This question was well answered, with many candidates scoring full
marks, or close to full marks. In part (iv) a common error was to
BB
determine exp(( \lambda +\mu ) t ). t P s to be 0 for t = 0, which led to the
−\mu
constant being evaluated as \lambda +\mu . Where candidates wrote down
incorrect Kolmogorov forward equations in part (iii), credit was
given in part (iv) for sensible attempts to solve the equations that
had been written.
