\documentclass[a4paper,12pt]{article}

%%%%%%%%%%%%%%%%%%%%%%%%%%%%%%%%%%%%%%%%%%%%%%%%%%%%%%%%%%%%%%%%%%%%%%%%%%%%%%%%%%%%%%%%%%%%%%%%%%%%%%%%%%%%%%%%%%%%%%%%%%%%%%%%%%%%%%%%%%%%%%%%%%%%%%%%%%%%%%%%%%%%%%%%%%%%%%%%%%%%%%%%%%%%%%%%%%%%%%%%%%%%%%%%%%%%%%%%%%%%%%%%%%%%%%%%%%%%%%%%%%%%%%%%%%%%

\usepackage{eurosym}
\usepackage{vmargin}
\usepackage{amsmath}
\usepackage{graphics}
\usepackage{epsfig}
\usepackage{enumerate}
\usepackage{multicol}
\usepackage{subfigure}
\usepackage{fancyhdr}
\usepackage{listings}
\usepackage{framed}
\usepackage{graphicx}
\usepackage{amsmath}
\usepackage{chngpage}

%\usepackage{bigints}
\usepackage{vmargin}

% left top textwidth textheight headheight

% headsep footheight footskip

\setmargins{2.0cm}{2.5cm}{16 cm}{22cm}{0.5cm}{0cm}{1cm}{1cm}

\renewcommand{\baselinestretch}{1.3}

\setcounter{MaxMatrixCols}{10}

\begin{document}[Total 12]9
In a three-state model where the states are:
1. Healthy
2. Sick
3. Dead
let t p xij be the probability that a person in state i at age x is in state j at age x + t ,
and let\mu ij x + t be the transition intensity from state i to state j at age x + t.
(i)
Show from first principles that:
d 13
11 13
12 23

t p x = t p x\mu x + t + t p x\mu x + t
dt

Doctors have been investigating the incidence of a sickness called Wadles. This
sickness is debilitating and can last for many years. It can sometimes kill the patient,
but not always. Doctors have established that if a patient recovers from Wadles they
are immune to further infection from this sickness.
(ii)
Sketch a diagram showing the states required for this sickness to be modelled
as a Markov process and the possible transitions between the states.

During the investigation, the doctors collected data on a sample of people. The data
included the total length of time each person was under investigation, and the length
of time each person was observed to be afflicted with Wadles. The doctors also
recorded each occasion on which a person contracted Wadles, recovered from Wadles
and died, noting whether the death was caused by Wadles or another reason.
(iii)
(iv)

Express the likelihood for the transition intensities in terms of the data
collected, defining all the terms you use.

Derive the maximum likelihood estimator of the death rate from Wadles. 
[Total 12]
CT4 A2018–7 


%%%%%%%%%%%%%%%%%%%%%%%%%%%%%%%%%%%%%%%%%%%%%%%%%%%%%%%%%%%%%%%%%%%%%%%%%%



Q9
(i)
Using the Markov assumption or from the Chapman-Kolmogorov
equations we can write
dt + t 11
13
12
23
13
33
p 13
x = t p x dt p x + t + t p x dt p x + t + t p x dt p x + t .
But dt


p 33
x + t = 1 .

Assuming that, for small dt,
dt
p ij x + t =
\mu ij x + t dt + o ( dt ) ,

o ( dt )
= 0 ;
dt → 0 dt
where lim
then substituting, we have
dt + t
11 13
12 23
13
p 13
x = t p x\mu x + t dt + t p x\mu x + t dt + t p x + o ( dt ) ,
so that
dt + t
and hence

13
11 13
12 23
p 13
x − t p x = t p x\mu x + t dt + t p x\mu x + t dt + o ( dt ) ,
d
( t p 13
x ) =
dt
lim
t + dt
dt → 0
13
p 13
x − t p x
=
dt
t

13
12 23
p 11
x\mu x + t + t p x\mu x + t .

[Total for part (i): 4]
(ii)
\mu NDO
Never had
Wadles
(N)
Dead from
other causes
(DO)
\mu SDO
\mu NS
Suffering from
Wadles
(S)
Page 18
Dead from
Wadles (DW)
\mu SDW
\mu IDO
Recovered
and immune
(I)
\mu SI
[+2]
OR
[Total for part (ii): 2]
(iii)
The likelihood is
{ (
) } { (
(\mu ) (\mu )
) } {
(\mu ) (\mu )
L ∝ exp −\mu NS −\mu NDO ν N exp −\mu SDW −\mu SI −\mu SDO ν S exp −\mu IDO ν I
(\mu ) (\mu )
NS
d NS
NDO
d NDO
d SDW
SDW
SDO
d SDO
SI
d SI
IDO
d IDO
}
.
[+2]
Here
ν I is the waiting time in state I, 
d IJ is the number of transitions from state I to state J, and 
and\mu IJ is the intensity of the transition from state I to state J .

[Total for part (iii): max. 3]
(iv)
Taking logarithms of the likelihood we have:
ln L =
( −\mu ) ν
SDW
S
(
+ d SDW ln\mu SDW
)
plus terms not dependent on\mu SDW .

Differentiating with respect to\mu SDW gives:
d ( ln L )
d\mu SDW
= −ν S +
d SDW
\mu SDW
,

and setting this to zero gives a maximum likelihood estimate of\mu SDW
\mu ˆ
SDW
d SDW
.
=
ν S
This is a maximum as the second derivative


d 2 ( ln L )
( d\mu )
SDW
2
= −
d SDW
(\mu )
SDW
2

Page 19
must be negative.

[Total for part (iv): 3]
[Total 12]
Part (i) was reasonably well answered by most candidates. Identifying
the state space in part (ii) proved challenging for many candidates.
Credit was given for an alternative four-state solution {Never had
Wadles, Sick with Wadles, Recovered and Immune, Dead}. This is not
ideal for estimating the death rate from Wadles, as a person who has
Wadles may die from a cause other than Wadles, but is a reasonable
answer to part (ii) as asked in the Examination Paper. A common
error was to label the first state “Healthy” rather than “Never had
Wadles”. This is ambiguous, as a person who has recovered from
Wadles is also “Healthy”. In part (iii) most candidates successfully
wrote down a likelihood consistent with the state space and transitions
they had sketched in part (ii); they also managed to derive the
maximum likelihood estimator of the death rate from Wadles in part
(iv). Where the diagram in part (ii) did not include the death rate from
Wadles (for example, it might include the death rate of any sick person,
regardless of whether that person had Wadles or not), credit was given
in part (iv) for a derivation of the maximum likelihood estimator of the
transition rate which was closest to the death rate from Wadles .
