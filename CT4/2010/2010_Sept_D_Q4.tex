\documentclass[a4paper,12pt]{article}

%%%%%%%%%%%%%%%%%%%%%%%%%%%%%%%%%%%%%%%%%%%%%%%%%%%%%%%%%%%%%%%%%%%%%%%%%%%%%%%%%%%%%%%%%%%%%%%%%%%%%%%%%%%%%%%%%%%%%%%%%%%%%%%%%%%%%%%%%%%%%%%%%%%%%%%%%%%%%%%%%%%%%%%%%%%%%%%%%%%%%%%%%%%%%%%%%%%%%%%%%%%%%%%%%%%%%%%%%%%%%%%%%%%%%%%%%%%%%%%%%%%%%%%%%%%%

\usepackage{eurosym}
\usepackage{vmargin}
\usepackage{amsmath}
\usepackage{graphics}
\usepackage{epsfig}
\usepackage{enumerate}
\usepackage{multicol}
\usepackage{subfigure}
\usepackage{fancyhdr}
\usepackage{listings}
\usepackage{framed}
\usepackage{graphicx}
\usepackage{amsmath}
\usepackage{chngpage}

%\usepackage{bigints}
\usepackage{vmargin}

% left top textwidth textheight headheight

% headsep footheight footskip

\setmargins{2.0cm}{2.5cm}{16 cm}{22cm}{0.5cm}{0cm}{1cm}{1cm}

\renewcommand{\baselinestretch}{1.3}

\setcounter{MaxMatrixCols}{10}

\begin{document}
\begin{enumerate}
%%%%%%%%%%%%%%%%%%%%%%%%%%%%%%%%%%%%%%%%%%%%%%%%%%%%%%%%%%%%%%%%%%%%%%%%%%%%%%%%%%%%%%%55
%%--- Question 2

Compare the characteristics of deterministic and stochastic models, by considering the
relationship between inputs and outputs.
%%%%%%%%%%%%%%%%%%%%%%%%%%%%%%%%%%%%%%%%%%%%%%%%%%%%%%%%%%%%%%%%%%%%%%%%%%%%%%%%%%%%%%%55
%%--- Question 3
3 The government of a small island state intends to set up a model to analyse the
mortality of the island’s population over the past 50 years.
Describe the process that would be followed to carry out the analysis.
%%%%%%%%%%%%%%%%%%%%%%%%%%%%%%%%%%%%%%%%%%%%%%%%%%%%%%%%%%%%%%%%%%%%%%%%%%%%%%%%%%%%%%%55
%%--- Question 4
[6]
A large pension scheme conducts an investigation into the mortality of its younger
male pensioners. The crude mortality rates are graduated using a standard table by
subtracting a constant from the rates given in the table.
A trainee has been asked to test the goodness-of-fit of the proposed graduation using
a chi-squared test. The trainee’s workings are reproduced below:
“Test H 0 : good fit against H 1 : bad fit.
Age Actual Deaths Expected Deaths (Actual Deaths –
Expected Deaths) 2
/Actual Deaths
60
61
62
63
64
65
Test Statistic 8
8
10
12
14
13 8.23
10.01
10.52
14.80
14.21
17.37 0.00661
0.50501
0.02704
0.65333
0.00315
1.46899
2.66413
Age range is 65–60 = 5 years so 5 degrees of freedom.
Two-tailed test so take 2 * 2.66413 = 5.32826 and compare against tabulated value of
chi-square distribution with 5 degrees of freedom at 2.5% level, which is 12.833.
So we accept the null hypothesis.”
Identify the errors in the trainee’s workings, without performing any detailed
calculations.
CT4 S2010—2


\newpage













\newpage

Page 3 %%%%%%%%%%%%%%%%%%%%%%%%%%%%%%%%%%%%%%%%%%% — 
Question 4
The null hypothesis is poorly expressed – should be “underlying rates are the graduated rates”
or similar.
The test statistic is incorrect – the denominator should be expected deaths.
Cannot comment on figures in table as no access to workings.
Number of ages is 6 not 5.
However fewer than 6 degrees of freedom is appropriate because should deduct 1 for
estimated parameter and some for choice of standard table
This is a one-tailed test not two-tailed.
Even if it were two-tailed, multiplying test statistic by 2 is inappropriate.
The trainee has not stated the level of significance to which he or she is working (presumably
5 per cent)
Does not explain that the reason for conclusion is 12.833 > 5.32826.
The null hypothesis should never be “accepted” rather it is “not rejected”.
The trainee has not stated his or her conclusion in terms of the null hypothesis
All the graduated rates are above the crude rates so although the graduation has been
accepted it is suspect.
This question was reasonably well answered.
\end{document}
