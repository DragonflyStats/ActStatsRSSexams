\documentclass[a4paper,12pt]{article}

%%%%%%%%%%%%%%%%%%%%%%%%%%%%%%%%%%%%%%%%%%%%%%%%%%%%%%%%%%%%%%%%%%%%%%%%%%%%%%%%%%%%%%%%%%%%%%%%%%%%%%%%%%%%%%%%%%%%%%%%%%%%%%%%%%%%%%%%%%%%%%%%%%%%%%%%%%%%%%%%%%%%%%%%%%%%%%%%%%%%%%%%%%%%%%%%%%%%%%%%%%%%%%%%%%%%%%%%%%%%%%%%%%%%%%%%%%%%%%%%%%%%%%%%%%%%

\usepackage{eurosym}
\usepackage{vmargin}
\usepackage{amsmath}
\usepackage{graphics}
\usepackage{epsfig}
\usepackage{enumerate}
\usepackage{multicol}
\usepackage{subfigure}
\usepackage{fancyhdr}
\usepackage{listings}
\usepackage{framed}
\usepackage{graphicx}
\usepackage{amsmath}
\usepackage{chngpage}

%\usepackage{bigints}
\usepackage{vmargin}

% left top textwidth textheight headheight

% headsep footheight footskip

\setmargins{2.0cm}{2.5cm}{16 cm}{22cm}{0.5cm}{0cm}{1cm}{1cm}

\renewcommand{\baselinestretch}{1.3}

\setcounter{MaxMatrixCols}{10}

\begin{document}
\begin{enumerate}
[6]5
(i)
Write down a formula for t q x (0\leq t \leq 1) under each of the following
assumptions:
(a)
(b)
(c)
uniform distribution of deaths
constant force of mortality
the Balducci assumption

6
(ii) Calculate 0.5 p 60 to six decimal places under each assumption given q 60 = 0.05.

(iii) Comment on the relative magnitude of your answers to part (ii).
(i) Outline the circumstances under which graphical graduation of crude
mortality rates might be useful.
(ii)
7
List the steps involved in graphical graduation.

[Total 6]



%%%%%%%%%%%%%%%%%%%%%%%%%%%%%%%%%%%%%%%%%%%%%%%%%%%%%%%%%%%%%%%%%%%%%%%%%%%%%%%%
\newpage
Question 5
(i)
(ii)
= t × q x
(a) t q x
(b) EITHER t q x = 1 − e −\mu t OR t q x = 1 − (1 − q x ) t
(c) t q x
(a) 1⁄2 q 60
=
tq x
1 − ( 1 − t ) q x
= 0.025
therefore 1⁄2 p 60 = 0.975000
Page 4 %%%%%%%%%%%%%%%%%%%%%%%%%%%%%%%%%%%%%%%%%%% — 
(b)
1 – e –\mu = 0.05
so −\mu = ln 0.95 and \mu = 0.051293
1⁄2 p 60
(c)
= e −0.5\mu = 0.974679
1 (0.05)
2
1 q 60 =
2
1 − 12 (0.05)
= 0.025641025
so
(iii)
1⁄2 p 60
= 0.974359
The Balducci assumption has the smallest value, and the uniform distribution of
deaths (UDD) the largest value
This is because the UDD implies an increasing force of mortality over the year of age,
whereas the Balducci assumption implies a decreasing force and a constant force is
clearly constant.
The higher the force of mortality in the second half of the year of age relative to its
magnitude in the first half of the year of age, the higher the probability of survival to
age 60.5 years
The difference between the three values of
0.5
q 60 is very small in this case.
Most candidates answered the parts relating to the uniform distribution of deaths and the
constant force of mortality correctly. Far fewer correctly worked out the formula for
t q x under the Balducci assumption. Instead, many candidates simply wrote down
1 − t q x + t = (1 − t ) q x in answer to (i)(c), which was not given credit, as it is not a formula for
t q x and hence is not answering the question set. However, credit was given to such
candidates in (ii)(c) if they calculated the correct numerical value for 1⁄2 p 60 . Some candidates
did not calculate the quantities in (ii) to six decimal places, and this was penalised.

Question 6
(i)
Graphical graduation might be used when EITHER a quick visual impression OR a
rough estimate is all that is required,
This is useful when the data are scanty and
EITHER
there is very little prior knowledge about the class of lives being analysed so that
a suitable standard table cannot be found
OR
the experience of a professional person can be called upon
Page 5 %%%%%%%%%%%%%%%%%%%%%%%%%%%%%%%%%%%%%%%%%%% — 
(ii)
Plot the crude data,
preferably on a logarithmic scale.
If data are scanty, group ages together,
choosing evenly spaced groups and making sure there are a reasonable number of
deaths (e.g. at least 5) in each group.
Plot approximate confidence limits or error bars around the plotted crude rates.
Draw the curve as smoothly as possible, trying to capture the overall shape of the
crude rates.
Test the graduation for goodness-of-fit and
EITHER test for smoothness OR examine third differences
If the graduation fails the test, re-draw the curve.
“Hand polishing” individual ages may be necessary to ensure adequate smoothness.
Many answers to this question were very sketchy and missed several of the points listed
above. In (i) simply saying “when data are scanty” was not sufficient for credit, as
graduation with reference to a standard table can also be used with scanty data sets provided
a suitable standard table can be found. In (ii) credit was given for additional points,
including noting that the curve can go outside the 95% confidence intervals at one out of
every 20 or so ages, and mentioning that the analyst might want to look at obvious outliers
before drawing the curve, as these may indicate data errors. A maximum of 5 marks was
available for (ii).
