\documentclass[a4paper,12pt]{article}

%%%%%%%%%%%%%%%%%%%%%%%%%%%%%%%%%%%%%%%%%%%%%%%%%%%%%%%%%%%%%%%%%%%%%%%%%%%%%%%%%%%%%%%%%%%%%%%%%%%%%%%%%%%%%%%%%%%%%%%%%%%%%%%%%%%%%%%%%%%%%%%%%%%%%%%%%%%%%%%%%%%%%%%%%%%%%%%%%%%%%%%%%%%%%%%%%%%%%%%%%%%%%%%%%%%%%%%%%%%%%%%%%%%%%%%%%%%%%%%%%%%%%%%%%%%%

\usepackage{eurosym}
\usepackage{vmargin}
\usepackage{amsmath}
\usepackage{graphics}
\usepackage{epsfig}
\usepackage{enumerate}
\usepackage{multicol}
\usepackage{subfigure}
\usepackage{fancyhdr}
\usepackage{listings}
\usepackage{framed}
\usepackage{graphicx}
\usepackage{amsmath}
\usepackage{chngpage}

%\usepackage{bigints}
\usepackage{vmargin}

% left top textwidth textheight headheight

% headsep footheight footskip

\setmargins{2.0cm}{2.5cm}{16 cm}{22cm}{0.5cm}{0cm}{1cm}{1cm}

\renewcommand{\baselinestretch}{1.3}

\setcounter{MaxMatrixCols}{10}

\begin{document}
\begin{enumerate}

7
A government has introduced a two-tier driving test system. Once someone applies for a provisional licence they are considered a Learner driver. Learner drivers who score 90% or more on the primary examination (which can be taken at any time)
become Qualified. Those who score between 50% and 90% are obliged to sit a secondary examination and are given driving status Restricted. Those who score 50% or below on the primary examination remain as Learners. Restricted drivers who pass
the secondary examination become Qualified, but those who fail revert back to Learner status and are obliged to start again.
\begin{enumerate}
\item (i) Sketch a diagram showing the possible transitions between the states.
 
\item (ii) Write down the likelihood of the data, assuming transition rates between states
are constant over time, clearly defining all terms you use.
\end{enumerate}
%%%%%%%%%%%%%%%%%%%%%%%%%%%%%%%%%
Figures over the first year of the new system based on those who applied for a
provisional licence during that time in one area showed the following:
Person-months in Learner State
Person-months in Restricted State
Number of transitions from Learner to Restricted
Number of transitions from Restricted to Learner
Number of transitions from Restricted to Qualified
Number of transitions from Learner to Qualified
\item (iii)
1,161
1,940
382
230
110
217
(a) Derive the maximum likelihood estimator of the transition rate from
Restricted to Learner.
(b) Estimate the constant transition rate from Restricted to Learner.
 
[Total 8]


%%%%%%%%%%%%%%%%%%%%%%%%%%%%%%%%%%%%%%%%%%%%%%%%%%%%%%%%%%%%%%%%%%%%%%%%%%%%%%%
\newpage



%%-- Page 8Subject CT4 %%%%%%%%%%%%%%%%%%%%%%%%%%%%%%%%%%%%%%%%%%%%%%%%%5— Examiners’ Report
\newpage
7
\item (i)
Learner
P
α
γ
δ
Restricted
Qualified
Q
\item (ii)
R
\beta

\begin{itemize}
\item  $\alpha$ be the transition rate R to P
\item $\beta$ be the transition rate R to Q
\item $\gamma$ be the transition rate P to Q
\item $\delta$ be the transition rate P to R
\end{itemize}
%%%%%%%%%%%%%%%%%%%%%%%5
Let P be the time spent in Learner state
R be the time spent in Restricted state
Let 
\begin{itemize}
\item[a] the number of transitions from Restricted to Learner
\item[b]the number of transitions from Restricted to Qualified
\item[c] the number of transitions from Learner to Qualified
\item[d] the number of transitions from Learner to Restricted
\end{itemize}

%%%%%%%%%%%%%%%%%%%%%%%%%%%%%%%%%%%%%%%%%%%%%%%%%%%%%%5
\[ L \sim exp{(-\delta - \gamma) P}exp{(-\alpha -\beta)R}\alphaa\betab\gammac\deltad \]

(iii) Take the logarithm of the likelihood
\[\log_e L = k - P(\delta + \gamma) - R(\alpha + \beta) + a \ln \alpha + b \ln\beta + c \ln \gamma + d \ln \delta \]
Differentiate with respect to \alpha
\[ \frac{d}{d \alpha} \left(\log_e L  \right) = -R  + \frac{a}{\alpha} \]


Set equal to zero to get estimator:
\begin{itemize}
\item ${ \displaystyle R = \frac{a}{\alpha} }$
\item ${ \displaystyle \hat{\alpha} = \frac{a}{R} }$
\end{itemize}

Set equal to zero to get estimator:
a
= R
α
α ˆ = a .
R
%%--- Page 9Subject CT4 %%%%%%%%%%%%%%%%%%%%%%%%%%%%%%%%%%%%%%%%%%%%%%%%%5— Examiners’ Report
Differentiate a second time:
d ln L
2
d α
=−
a
α 2
.
which is always negative, so that we have a maximum.
Thus $\hat{\alpha}  = 230 /1940 = 0.1186$
