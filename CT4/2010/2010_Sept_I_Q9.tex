\documentclass[a4paper,12pt]{article}

%%%%%%%%%%%%%%%%%%%%%%%%%%%%%%%%%%%%%%%%%%%%%%%%%%%%%%%%%%%%%%%%%%%%%%%%%%%%%%%%%%%%%%%%%%%%%%%%%%%%%%%%%%%%%%%%%%%%%%%%%%%%%%%%%%%%%%%%%%%%%%%%%%%%%%%%%%%%%%%%%%%%%%%%%%%%%%%%%%%%%%%%%%%%%%%%%%%%%%%%%%%%%%%%%%%%%%%%%%%%%%%%%%%%%%%%%%%%%%%%%%%%%%%%%%%%

\usepackage{eurosym}
\usepackage{vmargin}
\usepackage{amsmath}
\usepackage{graphics}
\usepackage{epsfig}
\usepackage{enumerate}
\usepackage{multicol}
\usepackage{subfigure}
\usepackage{fancyhdr}
\usepackage{listings}
\usepackage{framed}
\usepackage{graphicx}
\usepackage{amsmath}
\usepackage{chngpage}

%\usepackage{bigints}
\usepackage{vmargin}

% left top textwidth textheight headheight

% headsep footheight footskip

\setmargins{2.0cm}{2.5cm}{16 cm}{22cm}{0.5cm}{0cm}{1cm}{1cm}

\renewcommand{\baselinestretch}{1.3}

\setcounter{MaxMatrixCols}{10}

\begin{document}
\begin{enumerate}


[Total 7]
A researcher is reviewing a study published in a medical journal into survival after a
certain major operation. The journal only gives the following summary information:
\begin{itemize}
\item the study followed 16 patients from the point of surgery
\item  the patients were studied until the earliest of five years after the operation, the end
of the study or the withdrawal of the patient from the study
\item the Nelson-Aalen estimate, S(t), of the survival function was as follows:
\end{itemize}

%%%%%%%%%%%%%%%%%%%%%%%%%%%%%
Duration since operation t (years) S(t)
0\leqt<1
1\leqt<3
3\leqt<4
4\leqt<5 1
0.9355
0.7122
0.6285

\begin{enumerate}
\item (i) Describe the types of censoring which are present in the study.
\item (ii) Calculate the number of deaths which occurred, classified by duration since
the operation.
\item 
(iii) Calculate the number of patients who were censored.
%%CT4 S2010—4
\end{itemize}

\newpage


%%%%%%%%%%%%%%%%%%%%%%%%%%%%%%%%%%%%%%%%%%%%%%%%%%%%%%%%%%%%5


Question 9
(i)
Type I (right censoring) of patients who survive to duration 5 years.
Random censoring of patients who withdraw from the study.
(ii)
⎛ d j
Since S(t) = exp(−\lambda t ) where \lambda t = ∑ ⎜
⎜
t j \leq t ⎝ n j
\lambda t = −ln [S(t)]
Page 10
⎞
⎟
⎟
⎠ %%%%%%%%%%%%%%%%%%%%%%%%%%%%%%%%%%%%%%%%%%% — 
So
Duration since operation t (years)
0 \leq t < 1
1 \leq t < 3
3 \leq t < 4
4 \leq t < 5
S(t) \lambda t
1
0.9355
0.7122
0.6285 0
0.0667
0.3394
0.4644
Let d j and n j be the number of deaths and the number in the risk set at the jth point at
which events occur.
Consider t = 1
d 1
= 0 . 0667
n 1
Since there can be no more than 16 patients at risk at t = 1, the only possible
combination is d 1 = 1 and n 1 = 15
Consider t = 3 (the second point at which events occur)
d 2
= 0.3394 – 0.0667=0.2727
n 2
Recognising this as 3/11, and that there are at most 14 patients at risk, this implies that
d 2 = 3 and n 2 = 11.
Consider t = 4 (the third point at which events occur)
d 3
= 0.4644-0.0667 − 0.2727 = 0.125
n 3
Recognising this as 1/8, and that there are at most 11 patients at risk, this implies that
d 3 = 1 and n 3 = 8.
So the answer is:
1 death at duration
3 deaths at duration
1 death at duration
1 year
3 years
4 years
Page 11 %%%%%%%%%%%%%%%%%%%%%%%%%%%%%%%%%%%%%%%%%%% — 
(iii)
Patients either die or are censored. As the total number of patients is 16 and 5 die the
number censored is 16 − 5 = 11.

%%%%%%%%%%%%%%%%%%%%%%%%%%%%%%%%%%%%%%%%%%%%%
\newpage
This was the best answered question on the examination paper. In (i) “right” censoring was
awarded credit, as was some explanation of whether the censoring was informative or non-
informative. In (ii) a common error was to state the durations as ranges (i.e. 1 death at
durations between 1 and 3 years, 3 deaths at durations between 3 and 4 years, and 1 death at
durations over 4 years). This reveals a misunderstanding of the estimator, and was penalised
by the loss of 1 mark. Candidates who calculated an incorrect number of deaths in (ii) were
given credit for (iii) if their answer to (iii) was consistent with their answer to (ii).
\end{document}
