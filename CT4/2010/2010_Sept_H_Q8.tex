\documentclass[a4paper,12pt]{article}

%%%%%%%%%%%%%%%%%%%%%%%%%%%%%%%%%%%%%%%%%%%%%%%%%%%%%%%%%%%%%%%%%%%%%%%%%%%%%%%%%%%%%%%%%%%%%%%%%%%%%%%%%%%%%%%%%%%%%%%%%%%%%%%%%%%%%%%%%%%%%%%%%%%%%%%%%%%%%%%%%%%%%%%%%%%%%%%%%%%%%%%%%%%%%%%%%%%%%%%%%%%%%%%%%%%%%%%%%%%%%%%%%%%%%%%%%%%%%%%%%%%%%%%%%%%%

\usepackage{eurosym}
\usepackage{vmargin}
\usepackage{amsmath}
\usepackage{graphics}
\usepackage{epsfig}
\usepackage{enumerate}
\usepackage{multicol}
\usepackage{subfigure}
\usepackage{fancyhdr}
\usepackage{listings}
\usepackage{framed}
\usepackage{graphicx}
\usepackage{amsmath}
\usepackage{chngpage}

%\usepackage{bigints}
\usepackage{vmargin}

% left top textwidth textheight headheight

% headsep footheight footskip

\setmargins{2.0cm}{2.5cm}{16 cm}{22cm}{0.5cm}{0cm}{1cm}{1cm}

\renewcommand{\baselinestretch}{1.3}

\setcounter{MaxMatrixCols}{10}

\begin{document}
\begin{enumerate}

[Total 6]
Two neighbouring small countries have for many years taken annual censuses of their
populations on 1 January in which each inhabitant must give his or her age. Country
A uses an “age last birthday” definition of age, whereas Country B uses an “age
nearest birthday” definition. Each country has also operated a system in which deaths
are recorded on an “age nearest birthday at date of death” basis.
On 30 June 2009 Country A invaded Country B and the two countries became one
state. The new government wishes to estimate a single set of age-specific death rates,
\mu x , for the new unified state using the census data taken in the years before the
invasion.
Derive a formula which the new government may use to estimate \mu x in terms of the
recorded number of deaths in each country, and the population of each country
recorded as being aged x in the censuses. State any assumptions you make.
[8]
CT4 S2010—3
PLEASE TURN OVER8
Rocky Bay is a small seaside town in the north of Europe. In a leaflet advertising the
town, the tourist office has claimed that “in August, Rocky Bay has a Mediterranean
climate”. An actuarial student spent August 2009 on holiday in Rocky Bay with his
family, and became sceptical of this claim. When he returned home, he thought it
might be interesting to examine the claim by applying some of the methods he had
learned while studying for the Core Technical subjects. For each of the 31 days in
August 2009 he collected data recorded by various meteorological offices on the
maximum temperature in Rocky Bay and the mean of the maximum temperatures
reported on the same day at a range of places in the Mediterranean region.
The data are shown below, where, for each of the days in August, “+” means that
Rocky Bay had the higher maximum temperature and “–“ means that the
Mediterranean average was higher.
\begin{verbatim}
1 2 3 4 5 6 7 8 9 10 11 12 13 14 15 16 17 18 19 20
- - - - - - - - - - - - + + + + - - - -
21 22 23 24 25 26 27 28 29 30 31
- - - - - - - - + + +
\end{verbatim}

9
\begin{enumerate}
\item (i) Carry out a statistical test to examine the tourist office’s claim.
\item 
(ii) Suggest reasons why the test might not be an appropriate way to examine the
tourist office’s claim.
\end{enumerate}

%%%%%%%%%%%%%%%%%%%%%%%%%%%%%%%%%%%%%%%%%%%%%%%%%%%%%%5


Question 8
(i)
The null hypothesis, H 0 , is that the climate – or the underlying (long-run average)
temperature – in Rocky Bay in August is the same as that in the Mediterranean.
EITHER
Signs Test
Let P be the number of days for which the maximum temperature in Rocky Bay is
greater than that expected in the Mediterranean.
Under H 0 , P ∼ Binomial(31, 0.5) .
THEN EITHER NORMAL APPROXIMATION
Using the Normal approximation as we have more than 20 days,
⎛ 31 31 ⎞
P ∼ Normal ⎜ , ⎟
⎝ 2 4 ⎠
In the observations P = 7,
The value of the test statistic is therefore
Z =
7 − 15.5
= − 3.05
7.75
Since Z > 1.96 we reject H 0 at the 5 per cent level of significance
OR EXACT CALCULATION
We have P = 7
Page 8 %%%%%%%%%%%%%%%%%%%%%%%%%%%%%%%%%%%%%%%%%%% — 
The probability of obtaining 7 or fewer positive signs is
⎛ 31 ⎞ 31 ⎛ 31 ⎞ 31
⎛ 31 ⎞ 31
⎜ ⎟ 0.5 + ⎜ ⎟ 0.5 + ... + ⎜ ⎟ 0.5
⎝ 7 ⎠
⎝ 6 ⎠
⎝ 0 ⎠
which is 0.00122 + 0.00034 + 0.00008 + ... + 0.00000 = 0.00166
since this is less than 0.025 (two-tailed test)
we reject H 0 at the 5 per cent level of significance
and conclude that the climate of Rocky Bay is not the same as that in the
Mediterranean.
OR
Grouping of Signs Test
Let P be the number of days for which the maximum temperature in Rocky Bay is
greater than that expected in the Mediterranean.
Let Q (= 31 – P) be the number of days for which the maximum temperature in Rocky
Bay is less than that expected in the Mediterranean.
To test the null hypothesis, we need to calculate the maximum number
⎛ P − 1 ⎞ ⎛ Q + 1 ⎞
⎜
⎟⎜
⎟
t − 1 ⎠ ⎝ t ⎠
⎝
of positive runs, g, for which ∑
< 0.05 .
⎛ P + Q ⎞
t = 1
⎜
⎟
⎝ P ⎠
g
since P = 7 and Q = 24,
THEN EITHER
using the table on p. 189 of the Formulae and Tables for Examinations,
we find that g = 3.
OR
using the normal approximation we have
G ~ Normal(5.64, 0.95),
so, using a one-tailed test, the critical value at the 5% level is 5.645 – 1.645*√0.947 =
4.04.
Page 9 %%%%%%%%%%%%%%%%%%%%%%%%%%%%%%%%%%%%%%%%%%% — 
Since we only have 2 positive runs in the data we reject H 0 at the 5 per cent level of
significance and conclude that the climate of Rocky Bay is not the same as that in the
Mediterranean.
(ii)
Runs of consecutive days with the same sign are likely since the weather tends to be
determined by atmospheric conditions lasting more than one day.
The Mediterranean averages are averages for the month of August 2009,
not long-run averages.
August 2009 might have been an unusually hot month in the Mediterranean region.
Maximum temperature is not the only measure of climate, also consider mean
temperature, hours of sunshine, windiness, etc.
Choice of locations used for Mediterranean data could be important.
Also tests just look at whether one is higher or lower – the difference in each case
could be negligible (e.g. 25.001 degrees vs 25.002 degrees)
A non-standard measurement method might have been used in Rocky Bay,
which confounds the comparison.
For the signs test the continuity correction was not required, but if done has to be correct.
Candidates were given credit for a one-sided signs test in (i) provided that they set the null
hypothesis up correctly – i.e. that the average maximum temperature in Rocky Bay in August
is no lower than that in the Mediterranean. In (ii) other sensible comments were given
credit, and the maximum score of 2 marks could be obtained for making four sensible points
– not all the points listed above were required.
