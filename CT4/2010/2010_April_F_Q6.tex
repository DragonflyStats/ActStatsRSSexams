\documentclass[a4paper,12pt]{article}

%%%%%%%%%%%%%%%%%%%%%%%%%%%%%%%%%%%%%%%%%%%%%%%%%%%%%%%%%%%%%%%%%%%%%%%%%%%%%%%%%%%%%%%%%%%%%%%%%%%%%%%%%%%%%%%%%%%%%%%%%%%%%%%%%%%%%%%%%%%%%%%%%%%%%%%%%%%%%%%%%%%%%%%%%%%%%%%%%%%%%%%%%%%%%%%%%%%%%%%%%%%%%%%%%%%%%%%%%%%%%%%%%%%%%%%%%%%%%%%%%%%%%%%%%%%%

\usepackage{eurosym}
\usepackage{vmargin}
\usepackage{amsmath}
\usepackage{graphics}
\usepackage{epsfig}
\usepackage{enumerate}
\usepackage{multicol}
\usepackage{subfigure}
\usepackage{fancyhdr}
\usepackage{listings}
\usepackage{framed}
\usepackage{graphicx}
\usepackage{amsmath}
\usepackage{chngpage}

%\usepackage{bigints}
\usepackage{vmargin}

% left top textwidth textheight headheight

% headsep footheight footskip

\setmargins{2.0cm}{2.5cm}{16 cm}{22cm}{0.5cm}{0cm}{1cm}{1cm}

\renewcommand{\baselinestretch}{1.3}

\setcounter{MaxMatrixCols}{10}

\begin{document}
\begin{enumerate}

%%-- Question 6
 
An oil company has discovered a vast deposit of oil in an equatorial swamp. The area is extremely unhealthy and inhabited by venomous spiders. There is an antidote to bites from these spiders but it is expensive. The antidote acts
instantly but does not provide future immunity. The company commissions a study to estimate the rate of being bitten by the spiders among its employees, in order to determine the amount of antidote to provide.
Employees of the company are posted to the swamp for six month tours of duty starting on 1 January, 1 April, 1 July or 1 October. The first employees to be posted arrived on 1 January 2008. The swamp is so inaccessible that no employees are
allowed to leave before their six month tours of duty are completed.
Accidental deaths are common in this dangerous location. The table below gives some data from the study.
Quarter
beginning Number of new
arrivals at start
of quarter Number of
accidental deaths
during quarter Number of
spider bites
during quarter
1 January 2008
1 April 2008
1 July 2008
1 October 2008 90
80
114
126 10
8
10
13 15
25
30
40
\item (i)
\item (ii)
Estimate the quarterly rate of being bitten by a spider for each quarter of
2008, stating any assumptions you make.
 
Suggest reasons why the assumptions you made in \item (i) might not be valid.  
[Total 8]


%%%%%%%%%%%%%%%%%%%%%%%%%%%%%%%%%%%%%%%%%%%%%%%%%%%%%%%%%%%%%%%%%%%%%%%%%%%%%%%
\newpage


6
\item (i)
A central exposed to risk for each quarter in person-quarters can be constructed as follows.
In the first quarter there are 90 employees in the first three months of their six-
month tour of duty. Of these 10 will die during the quarter, and these contribute 0.5 each to the exposed to risk.
Therefore the total exposed to risk for the first quarter is 80 + (10 \times 0.5) = 85 person-quarters.
This assumes that accidental deaths occur on average half way through the quarter in which they were reported. OR that accidental
deaths are uniformly distributed across quarters. In the second quarter there are 80 new employees in the first three months of
their six-month tour, and 80 (90 minus the 10 who have died) employees in the second three months of their six-month tour. Of these 8 die during the quarter, and these contribute 0.5 each to the exposed to risk.
Therefore the total exposed to risk for the second quarter
is 152 + (8 \times 0.5) = 156 person-quarters
Page 6Subject CT4 %%%%%%%%%%%%%%%%%%%%%%%%%%%%%%%%%%%%%%%%%%%%%%%%%5— Examiners’ Report
In the third quarter there are 114 new employees in the first three months of
their six-month tour, and 76 (the 80 who were new on 1 April 2009 minus half of the 8 who died in the second quarter) employees in the second three months of their six-month tour. Of these 10 die during the quarter, and these
contribute 0.5 months each to the exposed to risk.
This assumes that accidental deaths are equally likely for employees
in the first quarter of their tour of duty, and those in the second quarter
of their tour of duty.
Therefore the total exposed to risk for the third quarter
is 180 + (10 \times 0.5) = 185 person-quarters
Finally, in the fourth quarter there are 126 new employees in the first three
months of their six-month tour, and 108 (the 114 who were new on 1 April
2009 minus a proportion equal to 114/(114 + 76) = 0.6 of the 10 who died in
the third quarter) employees in the second three months of their six-month
tour.
Of these 13 died during the quarter, and these contribute
0.5 quarters each to the exposed to risk.
Therefore the total exposed to risk for the fourth quarter is
221 + (13 \times 0.5) = 227.5 person-quarters.
We assume there are no deaths apart from accidental deaths.
These calculations are summarised in the table below.
Quarter
beginning Employees in
first quarter
of tour Employees in
second quarter
of tour Less 0.5 \times
accidental
deaths Central
exposed
to risk in
quarters
1 January
1 April
1 July
1 October 90
80
114
126 0
80
76
108 5
4
5
6.5 85
156
185
227.5
The quarterly rates of being bitten are therefore as follows:
Quarter
beginning Spider bites
1 January
1 April
1 July
1 October 15
25
30
40
Exposed to
risk
85
156
185
227.5
Rate of
being bitten
15/85
25/156
30/185
40/227.5
= 0.176
= 0.160
= 0.162
= 0.176
Page 7Subject CT4 %%%%%%%%%%%%%%%%%%%%%%%%%%%%%%%%%%%%%%%%%%%%%%%%%5— Examiners’ Report
We assume that all spider bites are treated.
\item (ii)
The assumption that there are no deaths apart from accidental deaths is
unlikely to be true, and probably the company would have data on these
which could be included in the calculations.
Accidental deaths may be more likely among employees in their first quarter
than their second, as those in their second quarter have more experience.
Accidental deaths may be more likely at the beginning of a quarter, when
there are newly arrived employees.
The experience of the quarter beginning 1 January may be different from that
of other quarters because that is the first quarter that any employees are
stationed in the swamp, and they may not know about the spiders when they
arrive. In subsequent quarters they may be able to adjust their
arrangements to reduce the possibility of being bitten.
Several alternatives to part (i) were also given credit. For example assuming spider bites are
all fatal produces the following solution to part (i):
Quarter
beginning Employees in
first quarter
of tour
1 January
1 April
1 July
1 October 90
80
114
126
Employees in
second quarter
of tour
0
65
62
88
Less 0.5 \times
total
deaths Central
exposed
to risk in
quarters
12.5
16.5
20
26.5 77.5
128.5
156.0
187.5
The quarterly rates of being bitten are therefore as follows:
Quarter
beginning Spider bites
1 January
1 April
%%-- 1 July
1 October 15
25
30
40
Exposed to
risk
77.5
128.5
156
187.5
Rate of
being bitten
15/77.5
25/128.5
30/156
40/187.5
= 0.194
= 0.195
= 0.192
= 0.213
In part (ii) credit was only given if the points made related to one of the assumptions stated in
the answer to part (i).

%%-- Page 8Subject CT4 %%%%%%%%%%%%%%%%%%%%%%%%%%%%%%%%%%%%%%%%%%%%%%%%%5— Examiners’ Report
