\documentclass[a4paper,12pt]{article}

%%%%%%%%%%%%%%%%%%%%%%%%%%%%%%%%%%%%%%%%%%%%%%%%%%%%%%%%%%%%%%%%%%%%%%%%%%%%%%%%%%%%%%%%%%%%%%%%%%%%%%%%%%%%%%%%%%%%%%%%%%%%%%%%%%%%%%%%%%%%%%%%%%%%%%%%%%%%%%%%%%%%%%%%%%%%%%%%%%%%%%%%%%%%%%%%%%%%%%%%%%%%%%%%%%%%%%%%%%%%%%%%%%%%%%%%%%%%%%%%%%%%%%%%%%%%

\usepackage{eurosym}
\usepackage{vmargin}
\usepackage{amsmath}
\usepackage{graphics}
\usepackage{epsfig}
\usepackage{enumerate}
\usepackage{multicol}
\usepackage{subfigure}
\usepackage{fancyhdr}
\usepackage{listings}
\usepackage{framed}
\usepackage{graphicx}
\usepackage{amsmath}
\usepackage{chngpage}

%\usepackage{bigints}
\usepackage{vmargin}

% left top textwidth textheight headheight

% headsep footheight footskip

\setmargins{2.0cm}{2.5cm}{16 cm}{22cm}{0.5cm}{0cm}{1cm}{1cm}

\renewcommand{\baselinestretch}{1.3}

\setcounter{MaxMatrixCols}{10}

\begin{document}
\begin{enumerate}
CT4 A2010—812
\item (i)
State three different methods of graduating raw mortality data and for each method give an example of a situation wheN_the method would be
appropriate.
 
A life insurance company last priced its whole of life contract 30 years ago using a
standard mortality table. The company wishes to establish whether recent mortality experience iN_the portfolio of business is in line with the pricing basis. These are the
data:
Recent Experience
Extract from the standard table
used for pricing the product
Age last
birthday Exposed to
Risk during
2009 Deaths during
2009 X_number of
survivors to age
x
50
51
52
53
54
55
56
57
58
59 2,381
3,177
3,460
1,955
3,122
3,485
2,781
3,150
3,651
3,991 16
21
22
15
24
29
26
31
39
48 50
51
52
53
54
55
56
57
58
59
60 32,669
32,513
32,338
32,143
31,926
31,685
31,417
31,121
30,795
30,435
30,039
\item (ii) Test the goodness of fit of these data with the pricing basis and comment on
your results.
[8]
\item (iii) (a)
State, with reasons, one further test which you would deem appropriate
to perform oN_these data.
(b)
Carry out that test.

[Total 15]
END OF PAPER
CT4 A2010—9

%%%%%%%%%%%%%%%%%%%%%%%%%%%%%%%%%%%%%%%%%%%%%%
12
\item (i)
By reference to a standard table – appropriate if data are scanty or a table of
similar lives exists.
Graphical graduation – appropriate if a “quick and dirty” result needed OR
for scanty data where no other method is appropriate
By parametric formula, if the experience is large.
\item (ii)
Standard table data
Age x
Number of survivors p x q x
32,669
32,513
32,338
32,143
31,926
31,685
31,417
31,121
30,795
30,435
30,039 0.99522
0.99462
0.99397
0.99325
0.99245
0.99154
0.99058
0.98952
0.98831
0.98699 0.00478
0.00538
0.00603
0.00675
0.00755
0.00846
0.00942
0.01048
0.01169
0.01301
50
51
52
53
54
55
56
57
58
59
60
Calculations:
Age last Exposed
to risk Expected
deaths (E) Observed
O-E
Deaths (O) (O-E) 2 /E (O-E) 2
E(1-q)
50
51
52
53
54
55
56
57
58
59 2,381
3,177
3,460
1,955
3,122
3,485
2,781
3,150
3,651
3,991 11.3697
17.1001
20.8640
13.1984
23.5671
29.4770
26.2016
32.9970
42.6810
51.9282 16
21
22
15
24
29
26
31
39
48 4.6303
3.8999
1.1360
1.8016
0.4329
–0.4770
–0.2016
–1.9970
–3.6810
–3.9282 1.8857
0.8894
0.0619
0.2459
0.0080
0.0077
0.0016
0.1209
0.3175
0.2972 1.8948
0.8942
0.0622
0.2476
0.0080
0.0078
0.0016
0.1221
0.3212
0.3011
Total 3.8356 3.8606
Page 20 %%%%%%%%%%%%%%%%%%%%%%%%%%%%%%%%%%%%%%%%%%%%%%%%%5— Examiners’ Report
The null hypothesis is that the data come from a population where the
mortalitY_is that represented by the standard table.
The test statistic
∑
( O − E ) 2
E
is distributed \chi^2 .
There are 10 age groups.
No degrees of freedom lost for choice of table, parameters or constraints on
data.
So we use 10 degrees of freedom.
This is a one-tailed test.
The upper 5\% point of the \chi^2 with 10 degrees of freedom is 18.31.
The observed test statistic is 3.84.
Since 3.84 < 18.31.
We have insufficient evidence to reject the null hypothesis.
\item (iii)
ALTERNATIVE 1
(a) The data easily pass the chi squared test, but there does seem to be a
gradual drift of ( O – E ) figures from strongly positive to strongly
negative. I would do a grouping of signs test to see if the data
display runs or “clumps” of deviations of the same sign.
(b) G = Number of groups of positive z s = 1
m = number of deviations = 10
n 1 = number of positive deviations = 5
n 2 = number of negative deviations = 5
THEN EITHER
We want k * the largest k such that
⎛ n 1 − 1 ⎞⎛ n 2 + 1 ⎞
k ⎜
⎟⎜
⎟
⎝ t − 1 ⎠⎝ t ⎠
⎛ m ⎞
t = 1
⎜ ⎟
⎝ n 1 ⎠
∑
< 0.05
The test fails at the 5\% level if G \leq k *.
From the Gold Book k * = 1, so we reject the null hypothesis.
Page 21 %%%%%%%%%%%%%%%%%%%%%%%%%%%%%%%%%%%%%%%%%%%%%%%%%5— Examiners’ Report
OR
For t = 1
⎛ m ⎞ ⎛ 10 ⎞
⎛ n 1 − 1 ⎞ ⎛ 4 ⎞
⎛ n 2 + 1 ⎞ ⎛ 6 ⎞
⎜
⎟ = ⎜ ⎟ = 1 and ⎜
⎟ = ⎜ ⎟ = 6 and ⎜ n ⎟ = ⎜ ⎟ = 252
⎝ t − 1 ⎠ ⎝ 0 ⎠
⎝ t ⎠ ⎝ 1 ⎠
⎝ 1 ⎠ ⎝ 5 ⎠
So Pr[ t = 1] if the null hypothesis is true is 6/252 = 0.0238, which is less than
5\% so we reject the null hypothesis.
ALTERNATIVE 2
(a) The data easily pass the chi squared test, but there does seem to be
a gradual drift of ( O – E ) figures from strongly positive to strongly
negative. I would do a serial correlatioN_test to see if the data displays
runs or clumps” of deviations of the same sign.
(b) The calculations are shown iN_the table below
x z x
z x +1
A = z x − z
B = z x + 1 − z
AB
A 2
B 2
50
51
52
53
54
55
56
57
58
59 1.373
0.943
0.249
0.496
0.089
0.088
0.039
0.348
0.536
0.545
0.943
0.249
0.496
0.089
0.088
0.039
0.348
0.563
0.545
0.908
0.478
–0.217
0.031
–0.376
–0.378
–0.426
–0.118
0.098
0.570
–0.125
0.123
–0.284
–0.286
–0.334
–0.026
0.190
0.172
0.517
–0.060
–0.027
–0.009
0.107
0.126
0.011
–0.022
0.017
0.824
0.228
0.047
0.001
0.142
0.143
0.181
0.014
0.010
0.325
0.016
0.015
0.081
0.082
0.112
0.001
0.036
0.029
0.465
0.373
Sum
0.661
1.589
0.695
−
z
−
−
0.661/(1.589*0.695) 0.5 = 0.629
Test 0.629 (9 0.5 ) = 1.887 against Normal (0,1), and, since
0.629 (9 0.5 ) = 1.887 > 1.645, we reject the null hypothesis.
ALTERNATIVE 3
Page 22
(a) Do the signs test to detect overall bias.
(b) Under the null hypothesis, the number of positive signs
amongst the z x s is distributed Binomial (10, 1⁄2 ). %%%%%%%%%%%%%%%%%%%%%%%%%%%%%%%%%%%%%%%%%%%%%%%%%5— Examiners’ Report
We observe 5 positive signs.
The probability of obtaining 5 or more positive signs is 0.623
OR
The probability of obtaining exactly 5 positive signs is 0.246
Since this is greater than 0.025 (two-tailed test), we cannot reject the
null hypothesis.
Note that because this test is not really appropriate in a case where there are five negative
and five positive deviations, no marks were awarded for part (a) to candidates who chose the
Signs Test unless earlier errors meant that the number of negative and positive signs were
unequal.
ALTERNATIVE 4
(a)
(b)
Do the cumulative deviations test to detect overall bias.
o ⎞
⎛
E
q
θ
−
∑ ⎜ ⎝ x x x ⎟ ⎠
x
The test statistic is
o
∼ Normal(0,1)
∑ E x q x
x
o
o
Age x θ x E x q x θ x − E x q x
50
51
52
53
54
55
56
57
58
59 16
21
22
15
24
29
26
31
39
48
∑ 11.37
17.10
20.86
13.20
23.57
29.48
26.20
33.00
42.68
51.93
269.38 4.63
3.90
1.14
1.80
0.43
–0.48
–0.20
–2.00
–3.68
–3.93
1.62
So the value of the test statistic is
1.62
= 0.09846 .
269.38
Using a 5\% level of significance,
we see that $− 1.96 < 0.09846 < 1.96$.
We do not reject the null hypothesis.
Page 23 %%%%%%%%%%%%%%%%%%%%%%%%%%%%%%%%%%%%%%%%%%%%%%%%%5— Examiners’ Report
ALTERNATIVE 5
(a) To check for outliers we do the individual standardised
deviations test.
(b) If the standard table rates were the true rates underlying the
observed rates
we would expect the individual deviations to be distributed Normal
(0,1)
and therefore only 1 in 20 z x s should have absolute magnitudes
greater than 1.96
OR
none should lie outside the range $(–3, +3)$
OR
or diagram showing split of deviations actual versus expected.
Looking at the z x s we see that the largest individual deviation
is 1.373.
Since this is less in absolute magnitude than 1.96 we cannot reject the
null hypothesis.
In part (ii) credit was only given for the null hypothesis if the wording used by the candidate
indicates that (s)he understands that it is the mortality underlying the observed data that is
not significantly different from that iN_the standard table, or that the standard table
“represents” the mortalitY_iN_the observed data. The null hypothesis is not that the mortality
iN_the observed data is the same as that iN_the standard table – as it will normally not be.
END OF EXAMINERS’ REPORT
Page 24
