\documentclass[a4paper,12pt]{article}

%%%%%%%%%%%%%%%%%%%%%%%%%%%%%%%%%%%%%%%%%%%%%%%%%%%%%%%%%%%%%%%%%%%%%%%%%%%%%%%%%%%%%%%%%%%%%%%%%%%%%%%%%%%%%%%%%%%%%%%%%%%%%%%%%%%%%%%%%%%%%%%%%%%%%%%%%%%%%%%%%%%%%%%%%%%%%%%%%%%%%%%%%%%%%%%%%%%%%%%%%%%%%%%%%%%%%%%%%%%%%%%%%%%%%%%%%%%%%%%%%%%%%%%%%%%%

\usepackage{eurosym}
\usepackage{vmargin}
\usepackage{amsmath}
\usepackage{graphics}
\usepackage{epsfig}
\usepackage{enumerate}
\usepackage{multicol}
\usepackage{subfigure}
\usepackage{fancyhdr}
\usepackage{listings}
\usepackage{framed}
\usepackage{graphicx}
\usepackage{amsmath}
\usepackage{chngpage}

%\usepackage{bigints}
\usepackage{vmargin}

% left top textwidth textheight headheight

% headsep footheight footskip

\setmargins{2.0cm}{2.5cm}{16 cm}{22cm}{0.5cm}{0cm}{1cm}{1cm}

\renewcommand{\baselinestretch}{1.3}

\setcounter{MaxMatrixCols}{10}

\begin{document}
\begin{enumerate}


[Total 7]
A researcher is reviewing a study published in a medical journal into survival after a
certain major operation. The journal only gives the following summary information:
• the study followed 16 patients from the point of surgery
• the patients were studied until the earliest of five years after the operation, the end
of the study or the withdrawal of the patient from the study
• the Nelson-Aalen estimate, S(t), of the survival function was as follows:
Duration since operation t (years) S(t)
0\leqt<1
1\leqt<3
3\leqt<4
4\leqt<5 1
0.9355
0.7122
0.6285
(i) Describe the types of censoring which are present in the study.
(ii) Calculate the number of deaths which occurred, classified by duration since
the operation.
[6]
(iii) Calculate the number of patients who were censored.
CT4 S2010—4


[Total 9]10
11
A study is undertaken of marriage patterns for women in a country where bigamy is
not permitted. A sample of women is interviewed and asked about the start and end
dates of all their marriages and where the marriages had ended, whether this was due
to death or divorce (all other reasons can be ignored). The investigators are interested
in estimating the rate of first marriage for all women and the rate of re-marriage
among widows.
(i) Draw a diagram illustrating a multiple-state model which the investigators
could use to make their estimates, using the four states: “Never married”,
“Married”, “Widowed” and “Divorced”.

(ii) Derive from first principles the Kolmogorov differential equation for first
marriages.

(iii) Write down the likelihood of the data in terms of the waiting times in each
state, the numbers of transitions of each type, and the transition intensities,
assuming the transition intensities are constant.
[3]
(iv) Derive the maximum likelihood estimator of the rate of first marriage.



%%%%%%%%%%%%%%%%%%%%%%%%%%%%%%%%%%%%%%%%%%%%%%%%%%%%%%%%%%%%5


Question 9
(i)
Type I (right censoring) of patients who survive to duration 5 years.
Random censoring of patients who withdraw from the study.
(ii)
⎛ d j
Since S(t) = exp(−\lambda t ) where \lambda t = ∑ ⎜
⎜
t j \leq t ⎝ n j
\lambda t = −ln [S(t)]
Page 10
⎞
⎟
⎟
⎠ %%%%%%%%%%%%%%%%%%%%%%%%%%%%%%%%%%%%%%%%%%% — 
So
Duration since operation t (years)
0 \leq t < 1
1 \leq t < 3
3 \leq t < 4
4 \leq t < 5
S(t) \lambda t
1
0.9355
0.7122
0.6285 0
0.0667
0.3394
0.4644
Let d j and n j be the number of deaths and the number in the risk set at the jth point at
which events occur.
Consider t = 1
d 1
= 0 . 0667
n 1
Since there can be no more than 16 patients at risk at t = 1, the only possible
combination is d 1 = 1 and n 1 = 15
Consider t = 3 (the second point at which events occur)
d 2
= 0.3394 – 0.0667=0.2727
n 2
Recognising this as 3/11, and that there are at most 14 patients at risk, this implies that
d 2 = 3 and n 2 = 11.
Consider t = 4 (the third point at which events occur)
d 3
= 0.4644-0.0667 − 0.2727 = 0.125
n 3
Recognising this as 1/8, and that there are at most 11 patients at risk, this implies that
d 3 = 1 and n 3 = 8.
So the answer is:
1 death at duration
3 deaths at duration
1 death at duration
1 year
3 years
4 years
Page 11 %%%%%%%%%%%%%%%%%%%%%%%%%%%%%%%%%%%%%%%%%%% — 
(iii)
Patients either die or are censored. As the total number of patients is 16 and 5 die the
number censored is 16 − 5 = 11.
This was the best answered question on the examination paper. In (i) “right” censoring was
awarded credit, as was some explanation of whether the censoring was informative or non-
informative. In (ii) a common error was to state the durations as ranges (i.e. 1 death at
durations between 1 and 3 years, 3 deaths at durations between 3 and 4 years, and 1 death at
durations over 4 years). This reveals a misunderstanding of the estimator, and was penalised
by the loss of 1 mark. Candidates who calculated an incorrect number of deaths in (ii) were
given credit for (iii) if their answer to (iii) was consistent with their answer to (ii).
Question 10
(i)
(ii)
1 Never married 2 Married
3 Widowed 4 Divorced
Using the numbering of the states above, let the probability that a women who is in
state i at time x will be in state j at time x + t be t p ij x .
Using the Markov property, and conditioning on the state occupied
at time x + t,
and noting that for first marriages return from the widowed or
divorced state is not possible, we can write
t + dt
11
12
12
22
p 12
x = t p x dt p x + t + t p x dt p x + t
Using the law of total probability,
dt
p x 22 + t = 1 − dt p x 23 + t − dt p x 24 + t ,
so that
t + dt
11
12
12
23
24
p 12
x = t p x dt p x + t + t p x (1 − dt p x + t − dt p x + t )
Let the transition rate from state i to state j at time x+t be \mu ijx + t .
Assume that
Page 12
dt
p ij x + t = \mu ij x + t dt + o ( dt ) , i ≠ j %%%%%%%%%%%%%%%%%%%%%%%%%%%%%%%%%%%%%%%%%%% — 
o ( dt )
= 0 .
dt → 0 + dt
where lim
Substituting for the
t + dt
t
p ij x in the equation above produces
11 12
12
23
24
p 12
x = t p x \mu x + t dt + t p x (1 − \mu x + t dt − \mu x + t dt ) + o ( dt )
Therefore
t + dt
12
11 12
12 23
12 24
p 12
x − t p x = t p x \mu x + t dt − t p x \mu x + t dt − t p x \mu x + t dt + o ( dt )
and, taking limits, we have
lim
dt → 0 +
t + dt
12
p 12
12
12 23
12 24
x − t p x
= t p 11
x \mu x + t − t p x \mu x + t − t p x \mu x + t
dt
So
d 12
11
12
12 23
12 24
t p x = t p x \mu x + t − t p x \mu x + t − t p x \mu x + t
dt
(iii)
Let the waiting time in state i be v i ,
and the number of transitions from state i to state j be d ij
and the transition intensity from state i to state j is \mu ij
Then the likelihood, L, may be written
L = K exp[ − v 1 \mu 12 − v 2 ( \mu 23 + \mu 24 ) − v 3 \mu 32 − v 4 \mu 42 ] \mu 12 d 12 \mu 23 d 23 \mu 24 d 24 \mu 32 d 32 \mu 42 d 42 .
(iv)
The logarithm of the likelihood is
log e L = log e K − v 1 \mu 12 − v 2 ( \mu 23 + \mu 24 ) − v 3 \mu 32 − v 4 \mu 42
+ d 12 log e \mu 12 + d 23 log e \mu 23 + d 24 log e \mu 24 + d 32 log e \mu 32 + d 42 log e \mu 42
Differentiating with respect to \mu 12 gives
d
\partial L
= − v 1 + 12 .
\partial\mu 12
\mu 12
Page 13 %%%%%%%%%%%%%%%%%%%%%%%%%%%%%%%%%%%%%%%%%%% — 
Setting this equal to 0 and solving for \mu 12 gives
^
\mu 12 =
d 12
.
v 1
This is a maximum because
\partial 2 L
\partial\mu 12
2
=−
d 12
\mu 12 2
which is negative.
Answers to (i), (iii) and (iv) were generally good. In (i) the arrows from Widowed to Married
and Divorced to Married were not required for full marks, as the question is about first
marriages. Answers to (ii) were more disappointing, with many candidates omitting steps in
32
14 42
the argument. In (ii), some candidates included the extra terms + t p 13
x \mu x + t + t p x \mu x + t . Since it
is just about possible to interpret the question in a way such that these should be included,
this was not heavily penalised.
