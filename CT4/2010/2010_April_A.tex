\documentclass[a4paper,12pt]{article}

%%%%%%%%%%%%%%%%%%%%%%%%%%%%%%%%%%%%%%%%%%%%%%%%%%%%%%%%%%%%%%%%%%%%%%%%%%%%%%%%%%%%%%%%%%%%%%%%%%%%%%%%%%%%%%%%%%%%%%%%%%%%%%%%%%%%%%%%%%%%%%%%%%%%%%%%%%%%%%%%%%%%%%%%%%%%%%%%%%%%%%%%%%%%%%%%%%%%%%%%%%%%%%%%%%%%%%%%%%%%%%%%%%%%%%%%%%%%%%%%%%%%%%%%%%%%

\usepackage{eurosym}
\usepackage{vmargin}
\usepackage{amsmath}
\usepackage{graphics}
\usepackage{epsfig}
\usepackage{enumerate}
\usepackage{multicol}
\usepackage{subfigure}
\usepackage{fancyhdr}
\usepackage{listings}
\usepackage{framed}
\usepackage{graphicx}
\usepackage{amsmath}
\usepackage{chngpage}

%\usepackage{bigints}
\usepackage{vmargin}

% left top textwidth textheight headheight

% headsep footheight footskip

\setmargins{2.0cm}{2.5cm}{16 cm}{22cm}{0.5cm}{0cm}{1cm}{1cm}

\renewcommand{\baselinestretch}{1.3}

\setcounter{MaxMatrixCols}{10}

\begin{document}
\begin{enumerate}
 Institute of Actuaries1 List four factors often used to subdivide life insurance mortality statistics.  
2 Write down integral equations for the mean and variance of the complete future
lifetime at age x, T x .  
3
For each of the following processes:
counting process;
general random walk;
compound Poisson process;
Poisson process;
Markov jump chain.
(a) State whether the state space is discrete, continuous or can be either.
(b) State whether the time set is discrete, continuous, or can be either.
 
4
A Markov Chain with state space {A, B, C} has the following properties:
\item it is irreducible
\item it is periodic
\item the probability of moving from A to B equals the probability of moving from A
to C
\item (i) Show that these properties uniquely define the process.
\item (ii) Sketch a transition diagram for the process.
CT4 A2010—2

 
[Total 5]5
Ten years ago, a confectionery manufacturer launched a new product, the Scrummy
Bar. The product has been successful, with a rapid increase in consumption since the product was first sold. In order to plan future investment in production capacity, the
manufacturer wishes to forecast the future demand for Scrummy Bars. It has data on age-specific consumption rates for the past ten years, together with projections of the
population by age over the next twenty years. It proposes the following modelling strategy:
\item extrapolate past age-specific consumption rates to forecast age-specific consumption rates for the next 20 years
\item apply the forecast age-specific consumption rates to the projected population by
age to obtain estimated total consumption of the product by age for each of the
next 20 years
\item sum the results to obtaiN_the total demand for each year
Describe the advantages and disadvantages of this strategy.

%%%%%%%%%%%%%%%%%%%%%%%%%%%%%%%%%%%%%%%%%%%%%%%%%%%%%%%%%%%%%%%%%%%%%%%%%%%%%%%%%%%%%%%%%%%%%%%%%%%%%%%%%%%%%%%
\newpage

Page 3 %%%%%%%%%%%%%%%%%%%%%%%%%%%%%%%%%%%%%%%%%%%%%%%%%5— Examiners’ Report
1
Sex
Age
Type of policy
Smoker/non-smoker status
Level of underwriting OR lifestyle/participation in dangerous sports
Duration in force
Sales channel
Policy size
Occupation of policyholder
Known impairments
Post code OR region/county/country OR address
Marks were given for up to four factors from the list above.
2
E [ T x ] = e x = \int
Var [ T x ] =
{ \int
\omega− x
0
t
\omega− x 2
t t
0
p x dt
OR
E [ T x ] = e x = \int
\omega− x
0
t t p x \mu x + t dt
}
p x \mu x + t dt − e x 2
The upper limits to the integrals can also be anything above \omega-x, for example $\omega$ or $\infty$, since
any age above \omega-x just adds zero to the summation .
3
Counting Process
General Random Walk
Compound Poisson Process
Poisson Process
Markov Jump Chain
Page 4
State Space Time Set
Discrete
Discrete or Continuous
Discrete or Continuous
Discrete
Discrete Discrete or Continuous
Discrete
Continuous
Continuous
Discrete %%%%%%%%%%%%%%%%%%%%%%%%%%%%%%%%%%%%%%%%%%%%%%%%%5— Examiners’ Report
4
\item (i)
As periodic and irreducible then all states are periodic, hence
probability of staying in any state is zero.
By law of total probability, P AA + P AB + P AC = 1.
But P AB = P AC and P AA = 0 so P AB = P AC = 0.5.
To be irreducible at least one of P BA or P CA must be greater than zero.
If P BA > 0 thento be periodic must have P CB = 0,
and to be irreducible P CA > 0,
and if P CA > 0 thento be periodic must have P BC = 0, and to be
irreducible P BA > 0.
So must have P BC = P CB = 0 and P BA = P CA = 1.
\item (ii)
1.0
B
0.5
A
0.5
5
C
1.0
Advantages
The model is simple to understand and to communicate.
The model takes account of one major source of variation in consumption
rates, specifically age.
The model is easy and cheap to implement.
The past data on consumption rates by age are likely to be fairly accurate.
The model can be adapted easily to different projected populations OR takes
into account future changes iN_the population.
Disadvantages
Past trends in consumption by age may not be a good guide to future trends.
Extrapolation of past age-specific consumption rates may be complex or
difficult and can be done in different ways.
Consumption of chocolate may be affected by the state of the economy,
e.g. whether there is a recession.
Page 5 %%%%%%%%%%%%%%%%%%%%%%%%%%%%%%%%%%%%%%%%%%%%%%%%%5— Examiners’ Report
Factors other than age may be important in determining consumption,
e.g. expenditure on advertising.
Consumption may be sensitive to pricing, which may change iN_the future.
A rapid increase in consumption rates is unlikely to be sustained
for a long period as there is likely to be an upper limit to
the amount of Scrummy Bars a person can eat.
The projections of the future population by age may not be accurate, as
they depend on future fertility, mortality and migration rates.
The proposed strategy does not include any testing of the
sensitivity of total demand to changes iN_the projected population,
or variations in future consumptioN_trends from that used iN_the model.
Unforeseen events such as competitors launching new products, or the nation
becoming increasingly health-aware, may affect future consumption.
The consumption of Scrummy Bars may vary with cohort rather than age, and
the model does not capture cohort effects.
Not all the points listed above were required for full credit. Other advantages, for example
those related to business prospects, were also given credit.
\end{document}
