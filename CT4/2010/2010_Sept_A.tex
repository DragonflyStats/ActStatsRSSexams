\documentclass[a4paper,12pt]{article}

%%%%%%%%%%%%%%%%%%%%%%%%%%%%%%%%%%%%%%%%%%%%%%%%%%%%%%%%%%%%%%%%%%%%%%%%%%%%%%%%%%%%%%%%%%%%%%%%%%%%%%%%%%%%%%%%%%%%%%%%%%%%%%%%%%%%%%%%%%%%%%%%%%%%%%%%%%%%%%%%%%%%%%%%%%%%%%%%%%%%%%%%%%%%%%%%%%%%%%%%%%%%%%%%%%%%%%%%%%%%%%%%%%%%%%%%%%%%%%%%%%%%%%%%%%%%

\usepackage{eurosym}
\usepackage{vmargin}
\usepackage{amsmath}
\usepackage{graphics}
\usepackage{epsfig}
\usepackage{enumerate}
\usepackage{multicol}
\usepackage{subfigure}
\usepackage{fancyhdr}
\usepackage{listings}
\usepackage{framed}
\usepackage{graphicx}
\usepackage{amsmath}
\usepackage{chngpage}

%\usepackage{bigints}
\usepackage{vmargin}

% left top textwidth textheight headheight

% headsep footheight footskip

\setmargins{2.0cm}{2.5cm}{16 cm}{22cm}{0.5cm}{0cm}{1cm}{1cm}

\renewcommand{\baselinestretch}{1.3}

\setcounter{MaxMatrixCols}{10}

\begin{document}
\begin{enumerate}

© Institute of Actuaries1
Following a review of the results of a stochastic model run, an actuary requests that a
parameter is changed. The change is not expected to alter the results significantly,
but results on the final basis are required in order to complete a report. Unfortunately
the actuarial student who produced the original model run is away on study leave, and
so the revised run is assigned to a different student.
When the revised results are produced, they are significantly different from the
original results.
Discuss possible reasons why the results are different.
[3]
2 Compare the characteristics of deterministic and stochastic models, by considering the
relationship between inputs and outputs.
[4]
3 The government of a small island state intends to set up a model to analyse the
mortality of the island’s population over the past 50 years.
Describe the process that would be followed to carry out the analysis.
4
[6]
A large pension scheme conducts an investigation into the mortality of its younger
male pensioners. The crude mortality rates are graduated using a standard table by
subtracting a constant from the rates given in the table.
A trainee has been asked to test the goodness-of-fit of the proposed graduation using
a chi-squared test. The trainee’s workings are reproduced below:
“Test H 0 : good fit against H 1 : bad fit.
Age Actual Deaths Expected Deaths (Actual Deaths –
Expected Deaths) 2
/Actual Deaths
60
61
62
63
64
65
Test Statistic 8
8
10
12
14
13 8.23
10.01
10.52
14.80
14.21
17.37 0.00661
0.50501
0.02704
0.65333
0.00315
1.46899
2.66413
Age range is 65–60 = 5 years so 5 degrees of freedom.
Two-tailed test so take 2 * 2.66413 = 5.32826 and compare against tabulated value of
chi-square distribution with 5 degrees of freedom at 2.5% level, which is 12.833.
So we accept the null hypothesis.”
Identify the errors in the trainee’s workings, without performing any detailed
calculations.
CT4 S2010—2


\newpage














Question 1
One or both of the runs (the original or the new) may have been incorrect as, for example, the
second trainee may not have been fully aware of the set-up (for example he or she may not
have followed the procedure correctly, or may have used different assumptions)
The difference between the two runs may not have only been the parameter change, for
example the two runs may have used different random seeds, or the second run may have had
fewer simulations.
The expectation that the model was not sensitive to this parameter could have been incorrect.
Other valid points were given credit, for example that some parameters might be linked to
live data, which will necessarily have changed; or that there may have been other
amendments to the data in the meantime. However, the maximum number of marks
attainable on this question was 3.
Question 2
A deterministic model is a model which does not contain any random components.
The output is determined once the fixed inputs and the relationships between inputs and
outputs have been defined.
A stochastic model is one that recognises the random nature of the input components.
The inputs to a stochastic model are random variables, and hence for any given values of the
inputs the outputs are an estimate of the characteristics of the model.
Several independent iterations of the model are required for each set of inputs to study their
implications.
The output of a stochastic model gives the distribution of relevant results for a distribution
of scenarios.
A deterministic model can be seen as a special case of a stochastic model.
The output of a stochastic model can be reproduced if the same random seed is used.
The output of a deterministic model is only a snap shot or an estimate of the characteristics
of the model for a given set of inputs.
Full marks could be obtained for rather less than is written above. The maximum number
of marks attainable was 4 even if all the above points were made.
Page 2 %%%%%%%%%%%%%%%%%%%%%%%%%%%%%%%%%%%%%%%%%%% — 
Question 3
Define the objectives of the model – what aspects of mortality are to be analysed (e.g.
average mortality rates, split male/female, analysis of trends over 50 years).
Plan the model.
Establish what data are available – collect data
Evaluate the accuracy of data and the consistency of the data over time (e.g. there may have
been changes to the way deaths and census data were recorded over a 50-year period)
Try to identify the main features of the mortality, and measure them.
Involve experts – e.g. there may be a national census office or Government department who
can advise.
Decide between simulation package or general purpose language, or use of spreadsheet
package.
Set up computer program and input data.
Debug program.
Test the output for reasonableness – is the model faithful to the actual mortality. experience of
the island over the required time frame?
Check the sensitivity of model to small changes to input parameters.
Analyse the model output.
Communicate and document the results.
This question was generally well answered, though many candidates simply reproduced the
list in the Core Reading, Unit 1, pages 2 and 3, without any reference to the specific problem
in the question – the analysis of mortality. These candidates did not gain full credit. A
minority of candidates interpreted this question as being about a mortality investigation,
making reference to the estimation of mortality rates and their subsequent graduation.
Credit was given to such candidates.
Page 3 %%%%%%%%%%%%%%%%%%%%%%%%%%%%%%%%%%%%%%%%%%% — 
Question 4
The null hypothesis is poorly expressed – should be “underlying rates are the graduated rates”
or similar.
The test statistic is incorrect – the denominator should be expected deaths.
Cannot comment on figures in table as no access to workings.
Number of ages is 6 not 5.
However fewer than 6 degrees of freedom is appropriate because should deduct 1 for
estimated parameter and some for choice of standard table
This is a one-tailed test not two-tailed.
Even if it were two-tailed, multiplying test statistic by 2 is inappropriate.
The trainee has not stated the level of significance to which he or she is working (presumably
5 per cent)
Does not explain that the reason for conclusion is 12.833 > 5.32826.
The null hypothesis should never be “accepted” rather it is “not rejected”.
The trainee has not stated his or her conclusion in terms of the null hypothesis
All the graduated rates are above the crude rates so although the graduation has been
accepted it is suspect.
This question was reasonably well answered.
