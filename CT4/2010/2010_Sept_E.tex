\documentclass[a4paper,12pt]{article}

%%%%%%%%%%%%%%%%%%%%%%%%%%%%%%%%%%%%%%%%%%%%%%%%%%%%%%%%%%%%%%%%%%%%%%%%%%%%%%%%%%%%%%%%%%%%%%%%%%%%%%%%%%%%%%%%%%%%%%%%%%%%%%%%%%%%%%%%%%%%%%%%%%%%%%%%%%%%%%%%%%%%%%%%%%%%%%%%%%%%%%%%%%%%%%%%%%%%%%%%%%%%%%%%%%%%%%%%%%%%%%%%%%%%%%%%%%%%%%%%%%%%%%%%%%%%

\usepackage{eurosym}
\usepackage{vmargin}
\usepackage{amsmath}
\usepackage{graphics}
\usepackage{epsfig}
\usepackage{enumerate}
\usepackage{multicol}
\usepackage{subfigure}
\usepackage{fancyhdr}
\usepackage{listings}
\usepackage{framed}
\usepackage{graphicx}
\usepackage{amsmath}
\usepackage{chngpage}

%\usepackage{bigints}
\usepackage{vmargin}

% left top textwidth textheight headheight

% headsep footheight footskip

\setmargins{2.0cm}{2.5cm}{16 cm}{22cm}{0.5cm}{0cm}{1cm}{1cm}

\renewcommand{\baselinestretch}{1.3}

\setcounter{MaxMatrixCols}{10}

\begin{document}
\begin{enumerate}
[Total 11]
At a certain airport, taxis for the city centre depart from a single terminus. The taxis
are all of the same make and model, and each can seat four passengers (not including
the driver). The terminus is arranged so that empty taxis queue in a single line, and
passengers must join the front taxi in the line. As soon as it is full, each taxi departs.
A strict environmental law forbids any taxi from departing unless it is full. Taxis are
so numerous that there is always at least one taxi waiting in line.
Customers arrive at the terminus according to a Poisson process with a rate \beta per
minute.
(i) Explain how that the number of passengers waiting in the front taxi can be
modelled as a Markov jump process.

(ii) Write down, for this process:
(a)
(b)
the generator matrix
Kolmogorov’s forward equations in component form
[4]
(iii)
Calculate the expected time a passenger arriving at the terminus will have to
wait until his or her taxi departs.
[4]
CT4 S2010—5
PLEASE TURN OVERThe four-passenger taxis were highly polluting, and the government instituted a
“scrappage” scheme whereby taxi drivers were given a subsidy to replace their old
four-passenger taxis with new “greener” models. Two such models were on the
market, one of which had a capacity of three passengers and the other of which had a
capacity of five passengers (again, not including the driver in each case). Half the
taxis were replaced with three-passenger models, and half with five-passenger
models.
Assume that, after the replacement, three-passenger and five-passenger models arrive
randomly at the terminus.
12
(iv) Write down the transition matrix of the Markov jump chain describing the
number of passengers in the front taxi after the vehicle replacement.

(v) Calculate the expected waiting time for a passenger arriving at the terminus
after the vehicle scrappage scheme and compare this with your answer to part
(iii).
[3]

%%%%%%%%%%%%%%%%%%%%%%%%%%%%%%%%%%%%%%%%%%%%%%%%%%%%%%%%%%%%%%%%%%%%%%%%%%%%%%%%%%%%%%
\newpage
Question 11
(i)
A Markov jump process is a continuous-time Markov process with a discrete state
space.
For a process to be Markov, the future development of the process must depend only
on its current state.
This is the case here, as the future of the process depends only on the number of
passengers currently in the front taxi.
The number of passengers in the front taxi also has a discrete state space {0, 1, 2, 3}.
(Note that immediately a fourth passenger arrives the taxi will depart so the front taxi
in the queue will never have four passengers in it.)
(ii)
(a)
The generator matrix A is
0 ⎞
⎛ −\beta \beta 0
⎜
⎟
⎜ 0 −\beta \beta 0 ⎟
⎜ 0
0 −\beta \beta ⎟
⎜
⎟
0 −\beta ⎠
⎝ \beta 0
(b)
Kolmogorov’s forward equations can be written in compact form as
d
P ( t ) = P ( t ) A ,
dt
Which are, for j = 0
d
p i 0 ( t ) = \beta p i 3 ( t ) − \beta p i 0 ( t )
dt
Page 14 %%%%%%%%%%%%%%%%%%%%%%%%%%%%%%%%%%%%%%%%%%% — 
and, for j = 1,2,3
d
p ij ( t ) = \beta p i , j − 1 ( t ) − \beta p ij ( t ) .
dt
(iii)
Since the waiting times under a Poisson process are exponential the expected waiting
1
time between the arrival of passengers at the terminus is minutes.
\beta
Successive waiting times are independent, therefore the expected waiting time for a
passenger arriving at the terminus is
3
E [ t ] = ∑ p i
i = 0
3 − i
,
\beta
where p i is the probability that the front taxi has exactly i previous passengers waiting
in it when the passenger arrives.
Since the p i s are all equal for i = 0, 1, 2, 3
⎛ 3 2 1 0 ⎞ 3
E [ t ] = 0.25 ⎜ + + + ⎟ =
minutes.
⎝ \beta \beta \beta \beta ⎠ 2 \beta
(iv)
The transition matrix, P , is
⎛ 0 1 0
⎜
⎜ 0 0 1
⎜ 1
0 0
⎜
2
⎜
⎜ 0 0 0
⎜ 1 0 0
⎝
(v)
0 0 ⎞
⎟
0 0 ⎟
⎟
1
0 ⎟ .
2
⎟
0 1 ⎟
0 0 ⎟ ⎠
The expected waiting time if the front taxi is a three-passenger model is
2
E [ t | 3 − passenger model] = ∑ p i
i = 0
2 − i 1 ⎛ 2 1 0 ⎞ 1
= ⎜ + + ⎟ =
3 ⎝ \beta \beta \beta ⎠ \beta
\beta
The expected waiting time if the front taxi is a five-passenger model is
4
E [ t | 5 − passenger model] = ∑ p i
i = 0
4 − i 1 ⎛ 4 3 2 1 0 ⎞ 2
= ⎜ + + + + ⎟ = .
\beta
5 ⎝ \beta \beta \beta \beta \beta ⎠ \beta
Page 15 %%%%%%%%%%%%%%%%%%%%%%%%%%%%%%%%%%%%%%%%%%% — 
But 5-passenger models must expect to wait
5
times as long at the front of the queue
3
than do 3-passenger models.
5
of the time the taxi at the front of the
8
3
queue will be a five-passenger model and only of the time will is be a three-
8
passenger model.
So when a passenger arrives at the terminus,
So the overall expected waiting time in minutes is
3
5
13
.
( E [ t | 3 − passenger model]) + ( E [ t | 5 − passenger model]) =
8
8
8 \beta
As this is longer than
3
, the service provided to the passengers has deteriorated.
2\beta
Many candidates struggled with this question. A common error in (ii) was to draw a matrix
with five states rather than four, failing to recognise that taxis with four passengers in do not
wait at the front of the queue, but depart as soon as the fourth passenger arrives. Most
candidates who attempted (iv) wrote down a generator matrix, whereas the question asked
for a transition matrix.
Page 16 %%%%%%%%%%%%%%%%%%%%%%%%%%%%%%%%%%%%%%%%%%% — 
