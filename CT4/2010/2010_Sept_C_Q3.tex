\documentclass[a4paper,12pt]{article}

%%%%%%%%%%%%%%%%%%%%%%%%%%%%%%%%%%%%%%%%%%%%%%%%%%%%%%%%%%%%%%%%%%%%%%%%%%%%%%%%%%%%%%%%%%%%%%%%%%%%%%%%%%%%%%%%%%%%%%%%%%%%%%%%%%%%%%%%%%%%%%%%%%%%%%%%%%%%%%%%%%%%%%%%%%%%%%%%%%%%%%%%%%%%%%%%%%%%%%%%%%%%%%%%%%%%%%%%%%%%%%%%%%%%%%%%%%%%%%%%%%%%%%%%%%%%

\usepackage{eurosym}
\usepackage{vmargin}
\usepackage{amsmath}
\usepackage{graphics}
\usepackage{epsfig}
\usepackage{enumerate}
\usepackage{multicol}
\usepackage{subfigure}
\usepackage{fancyhdr}
\usepackage{listings}
\usepackage{framed}
\usepackage{graphicx}
\usepackage{amsmath}
\usepackage{chngpage}

%\usepackage{bigints}
\usepackage{vmargin}

% left top textwidth textheight headheight

% headsep footheight footskip

\setmargins{2.0cm}{2.5cm}{16 cm}{22cm}{0.5cm}{0cm}{1cm}{1cm}

\renewcommand{\baselinestretch}{1.3}

\setcounter{MaxMatrixCols}{10}

\begin{document}
\begin{enumerate}

%%%%%%%%%%%%%%%%%%%%%%%%%%%%%%%%%%%%%%%%%%%%%%%%%%%%%%%%%%%%%%%%%%%%%%%%%%%%%%%%%%%%%%%55
%%--- Question 3
3 The government of a small island state intends to set up a model to analyse the
mortality of the island’s population over the past 50 years.
Describe the process that would be followed to carry out the analysis.
%%%%%%%%%%%%%%%%%%%%%%%%%%%%%%%%%%%%%%%%%%%%%%%%%%%%

\newpage














Page 2 %%%%%%%%%%%%%%%%%%%%%%%%%%%%%%%%%%%%%%%%%%% — 
Question 3
Define the objectives of the model – what aspects of mortality are to be analysed (e.g.
average mortality rates, split male/female, analysis of trends over 50 years).
Plan the model.
Establish what data are available – collect data
Evaluate the accuracy of data and the consistency of the data over time (e.g. there may have
been changes to the way deaths and census data were recorded over a 50-year period)
Try to identify the main features of the mortality, and measure them.
Involve experts – e.g. there may be a national census office or Government department who
can advise.
Decide between simulation package or general purpose language, or use of spreadsheet
package.
Set up computer program and input data.
Debug program.
Test the output for reasonableness – is the model faithful to the actual mortality. experience of
the island over the required time frame?
Check the sensitivity of model to small changes to input parameters.
Analyse the model output.
Communicate and document the results.
This question was generally well answered, though many candidates simply reproduced the
list in the Core Reading, Unit 1, pages 2 and 3, without any reference to the specific problem
in the question – the analysis of mortality. These candidates did not gain full credit. A
minority of candidates interpreted this question as being about a mortality investigation,
making reference to the estimation of mortality rates and their subsequent graduation.
Credit was given to such candidates.
Page 3 %%%%%%%%%%%%%%%%%%%%%%%%%%%%%%%%%%%%%%%%%%% — 

\end{document}
