\documentclass[a4paper,12pt]{article}

%%%%%%%%%%%%%%%%%%%%%%%%%%%%%%%%%%%%%%%%%%%%%%%%%%%%%%%%%%%%%%%%%%%%%%%%%%%%%%%%%%%%%%%%%%%%%%%%%%%%%%%%%%%%%%%%%%%%%%%%%%%%%%%%%%%%%%%%%%%%%%%%%%%%%%%%%%%%%%%%%%%%%%%%%%%%%%%%%%%%%%%%%%%%%%%%%%%%%%%%%%%%%%%%%%%%%%%%%%%%%%%%%%%%%%%%%%%%%%%%%%%%%%%%%%%%

\usepackage{eurosym}
\usepackage{vmargin}
\usepackage{amsmath}
\usepackage{graphics}
\usepackage{epsfig}
\usepackage{enumerate}
\usepackage{multicol}
\usepackage{subfigure}
\usepackage{fancyhdr}
\usepackage{listings}
\usepackage{framed}
\usepackage{graphicx}
\usepackage{amsmath}
\usepackage{chngpage}

%\usepackage{bigints}
\usepackage{vmargin}

% left top textwidth textheight headheight

% headsep footheight footskip

\setmargins{2.0cm}{2.5cm}{16 cm}{22cm}{0.5cm}{0cm}{1cm}{1cm}

\renewcommand{\baselinestretch}{1.3}

\setcounter{MaxMatrixCols}{10}

\begin{document}
\begin{enumerate}


6

(i) Outline the circumstances under which graphical graduation of crude
mortality rates might be useful.
(ii)
7
List the steps involved in graphical graduation.

[Total 6]



%%%%%%%%%%%%%%%%%%%%%%%%%%%%%%%%%%%%%%%%%%%%%%%%%%%%%%%%%%%%%%%%%%%%%%%%%%%%%%%%
\newpage

Question 6
(i)
Graphical graduation might be used when EITHER a quick visual impression OR a
rough estimate is all that is required,
This is useful when the data are scanty and
EITHER
there is very little prior knowledge about the class of lives being analysed so that
a suitable standard table cannot be found
OR
the experience of a professional person can be called upon
Page 5 %%%%%%%%%%%%%%%%%%%%%%%%%%%%%%%%%%%%%%%%%%% — 
(ii)
Plot the crude data,
preferably on a logarithmic scale.
If data are scanty, group ages together,
choosing evenly spaced groups and making sure there are a reasonable number of
deaths (e.g. at least 5) in each group.
Plot approximate confidence limits or error bars around the plotted crude rates.
Draw the curve as smoothly as possible, trying to capture the overall shape of the
crude rates.
Test the graduation for goodness-of-fit and
EITHER test for smoothness OR examine third differences
If the graduation fails the test, re-draw the curve.
“Hand polishing” individual ages may be necessary to ensure adequate smoothness.

Many answers to this question were very sketchy and missed several of the points listed above. In (i) simply saying “when data are scanty” was not sufficient for credit, as graduation with reference to a standard table can also be used with scanty data sets provided a suitable standard table can be found. In (ii) credit was given for additional points, including noting that the curve can go outside the 95\% confidence intervals at one out of
every 20 or so ages, and mentioning that the analyst might want to look at obvious outliers before drawing the curve, as these may indicate data errors. A maximum of 5 marks was available for (ii).
\end{document}
