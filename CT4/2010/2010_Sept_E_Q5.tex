\documentclass[a4paper,12pt]{article}

%%%%%%%%%%%%%%%%%%%%%%%%%%%%%%%%%%%%%%%%%%%%%%%%%%%%%%%%%%%%%%%%%%%%%%%%%%%%%%%%%%%%%%%%%%%%%%%%%%%%%%%%%%%%%%%%%%%%%%%%%%%%%%%%%%%%%%%%%%%%%%%%%%%%%%%%%%%%%%%%%%%%%%%%%%%%%%%%%%%%%%%%%%%%%%%%%%%%%%%%%%%%%%%%%%%%%%%%%%%%%%%%%%%%%%%%%%%%%%%%%%%%%%%%%%%%

\usepackage{eurosym}
\usepackage{vmargin}
\usepackage{amsmath}
\usepackage{graphics}
\usepackage{epsfig}
\usepackage{enumerate}
\usepackage{multicol}
\usepackage{subfigure}
\usepackage{fancyhdr}
\usepackage{listings}
\usepackage{framed}
\usepackage{graphicx}
\usepackage{amsmath}
\usepackage{chngpage}

%\usepackage{bigints}
\usepackage{vmargin}

% left top textwidth textheight headheight

% headsep footheight footskip

\setmargins{2.0cm}{2.5cm}{16 cm}{22cm}{0.5cm}{0cm}{1cm}{1cm}

\renewcommand{\baselinestretch}{1.3}

\setcounter{MaxMatrixCols}{10}

\begin{document}
\begin{enumerate}
[6]5
(i)
Write down a formula for t q x (0\leq t \leq 1) under each of the following
assumptions:
\begin{enumerate}
\item uniform distribution of deaths
\item constant force of mortality
\item the Balducci assumption
\end{enumerate}

(ii) Calculate 0.5 p 60 to six decimal places under each assumption given q 60 = 0.05.

(iii) Comment on the relative magnitude of your answers to part (ii).



%%%%%%%%%%%%%%%%%%%%%%%%%%%%%%%%%%%%%%%%%%%%%%%%%%%%%%%%%%%%%%%%%%%%%%%%%%%%%%%%
\newpage
Question 5
(i)
(ii)
= t × q x
(a) t q x
(b) EITHER t q x = 1 − e −\mu t OR t q x = 1 − (1 − q x ) t
(c) t q x
(a) 1⁄2 q 60
=
tq x
1 − ( 1 − t ) q x
= 0.025
therefore 1⁄2 p 60 = 0.975000
Page 4 %%%%%%%%%%%%%%%%%%%%%%%%%%%%%%%%%%%%%%%%%%% — 
(b)
1 – e –\mu = 0.05
so −\mu = ln 0.95 and \mu = 0.051293
1⁄2 p 60
(c)
= e −0.5\mu = 0.974679
1 (0.05)
2
1 q 60 =
2
1 − 12 (0.05)
= 0.025641025
so
(iii)
1⁄2 p 60
= 0.974359
The Balducci assumption has the smallest value, and the uniform distribution of
deaths (UDD) the largest value
This is because the UDD implies an increasing force of mortality over the year of age,
whereas the Balducci assumption implies a decreasing force and a constant force is
clearly constant.
The higher the force of mortality in the second half of the year of age relative to its
magnitude in the first half of the year of age, the higher the probability of survival to
age 60.5 years
The difference between the three values of
0.5
q 60 is very small in this case.
%%%%%%%%%%%%%%%%%%%%%%%%%%%%%%%%%%%%%%%%%%%%%%%%%%%%%%%%%%%%
\newpage
Most candidates answered the parts relating to the uniform distribution of deaths and the
constant force of mortality correctly. Far fewer correctly worked out the formula for
t q x under the Balducci assumption. Instead, many candidates simply wrote down
1 − t q x + t = (1 − t ) q x in answer to (i)(c), which was not given credit, as it is not a formula for
t q x and hence is not answering the question set. However, credit was given to such
candidates in (ii)(c) if they calculated the correct numerical value for 1⁄2 p 60 . Some candidates
did not calculate the quantities in (ii) to six decimal places, and this was penalised.
\end{document}
