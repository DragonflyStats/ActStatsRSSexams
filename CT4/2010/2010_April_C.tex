\documentclass[a4paper,12pt]{article}

%%%%%%%%%%%%%%%%%%%%%%%%%%%%%%%%%%%%%%%%%%%%%%%%%%%%%%%%%%%%%%%%%%%%%%%%%%%%%%%%%%%%%%%%%%%%%%%%%%%%%%%%%%%%%%%%%%%%%%%%%%%%%%%%%%%%%%%%%%%%%%%%%%%%%%%%%%%%%%%%%%%%%%%%%%%%%%%%%%%%%%%%%%%%%%%%%%%%%%%%%%%%%%%%%%%%%%%%%%%%%%%%%%%%%%%%%%%%%%%%%%%%%%%%%%%%

\usepackage{eurosym}
\usepackage{vmargin}
\usepackage{amsmath}
\usepackage{graphics}
\usepackage{epsfig}
\usepackage{enumerate}
\usepackage{multicol}
\usepackage{subfigure}
\usepackage{fancyhdr}
\usepackage{listings}
\usepackage{framed}
\usepackage{graphicx}
\usepackage{amsmath}
\usepackage{chngpage}

%\usepackage{bigints}
\usepackage{vmargin}

% left top textwidth textheight headheight

% headsep footheight footskip

\setmargins{2.0cm}{2.5cm}{16 cm}{22cm}{0.5cm}{0cm}{1cm}{1cm}

\renewcommand{\baselinestretch}{1.3}

\setcounter{MaxMatrixCols}{10}

\begin{document}
\begin{enumerate}
CT4 A2010—48
A certain profession admits new members to the status of student. Students may
qualify as fellows of the profession by virtue of passing a series of examinations.
Normally student members sit the examinations whilst working for an employer.
There are two sessions of the examinations each year.
An employer provides study support to student members of the profession. It wishes
to assess the cost of providing this study support and therefore wishes to know the
average time it can expect to take for its students to qualify.
The employer has maintained records for 23 of its students who all sat their first
examination in the first session of 2003. The students’ progress has been recorded up
to and including the last session of 2009. The following data records the number of
sessions which had been held before the specified event occurred for a student in this
cohort:
Qualified
Stopped studying
6, 8, 8, 9, 9, 9, 11, 11, 13, 13, 13
4, 5, 8, 11, 14
The remaining seven students were still studying for the examinations at the end of
2009.
\begin{enumerate}[(i)]
\item (i) Determine the median number of sessions taken to qualify for those students
who qualified during the period of observation.
\item  
(ii) Calculate the Kaplan-Meier estimate of the survival function, S(t), for the
“hazard” of qualifying, where t is the number of sessions of examinations
since 1 January 2003.
\item  
(iii) Hence estimate the median number of sessions to qualify for the students of
this employer.
\item  
(iv) Explain the difference between the results in (i) and (iii) above.
\end{enumerate}
%%%%%%%%%%%%%%%%%%%%%%%%%%%%%%%%%%%%%%%%%%%%%%%%%%%%%%%%%%%%%%%%%%%%%%%%%%%%%%%%%%%%%%%%%%%%%%%%%%%
\newpage
[Total 11]
9
(i)
Write down the hazard function for the Cox proportional hazards model defining all the terms that you use.
 
A farmer is concerned that he is losing a lot of his birds to a predator, so he decides to build a new enclosure using taller fencing. This fencing is expensive and he cannot afford to build a large enough area for all his birds. He therefore decides to put half
his birds in the new enclosure and leave the others in the existing enclosure. He is convinced that the new enclosure is an improvement, but has asked an actuarial student to determine whether the new enclosure will result in an increase in the life
expectancy of his birds. The student has fitted a Cox proportional hazards model to
data on the duration until a bird is killed by a predator and calculated the following
figures relating to the regression parameters:
Bird Chicken
Duck
Goose
Enclosure New
Old
Sex Male
Female
Parameter estimate Variance
0
–0.210
0.075 0
0.002
0.004
0.125
0 0.0015
0
0.2
0 0.0026
0
(ii) State the features of the bird to which the baseline hazard applies.
(iii) For each regression parameter:
(a) Define the associated covariate.
(b) Calculate the 95\% confidence interval based on the standard error.
 
 
(iv)
(v)
Comment on the farmer’s belief that the new enclosure will result in an
increase in his birds’ life expectancy.
 
Calculate, using this model, the probability that a female duck in the new
enclosure has been killed by a predator at the end of six months, given that the
probability that a male goose in the old enclosure has been killed at the end of
the same period is 0.1 (all other decrements can be ignored).

[Total 12]


%%%%%%%%%%%%%%%%%%%%%%%%%%%%%%%%%%%%%%%%%%%%%%%%%%%%%%%%%%%%%%%%%%%%%%%%%%%%%%%%%%%%%%%%%%%%%%%%%%%%%%%%%%%%%%%%%%%%%%%%%%%

8
(i)
11 students qualified during the period of observation, so the median is the
number of sessions taken to qualify by the sixth student to qualify.
This is 9 sessions.
(ii)
Define t as the number of sessions which have taken place since 1 Jan 2003.
Stopped studying implies recorded after the session number reported.
tj Nj Dj Cj
0
6
8
9
11
13 23
21
20
17
14
11 0
1
2
3
2
3 2
0
1
0
1
0
D j
1 −
N j
–
1/21
2/20
3/17
2/14
3/11
D j
N j
1
20/21
18/20
14/17
12/14
8/11
The Kaplan-Meier estimate is given by product of 1 −
D j
N j
Then the Kaplan-Meier estimate of the survival function is
t
0≤ t < 6
6≤ t < 8
8≤ t < 9
9 ≤ t < 11
11≤ t < 13
13≤ t < 14
(iii)
1
0.9524
0.8571
0.7059
0.6050
0.4400
The median time to qualify as estimated by the Kaplan-Meier estimate
is the first time at which S ( t ) is below 0.5.
Therefore the estimate is 13 sessions.
Page 10
^
S ( t )Subject CT4 %%%%%%%%%%%%%%%%%%%%%%%%%%%%%%%%%%%%%%%%%%%%%%%%%5— Examiners’ Report
(iv)
The estimate based on students qualifying during the period
is a biased estimate because it does not contain information
about students still studying at the end of the period, or about
those who dropped out (stopped studying without qualifying).
The students still studying at the end of 2009 have (by definition) a longer
period to qualification than those who qualified in the period.
Hence the Kaplan-Meier estimate is higher than the median using
only students who qualified during the period.
In part (i) the question said “determine” so some explanation of where the answer comes
from was required for full credit. In part (ii) the question said “calculate” so the correct S(t)
and associated ranges of t scored full marks.
%%%%%%%%%%%%%%%%%%%%%%%%%%%%%%%%%%%%%%%%%%%%%%%%%%%%%%%%%%%%%%%%%%%%%%%%%%%%%%%%%%%%%%%%%%%%%%%%%%%%%%%5
\newpage

9
(i)
\[ h(z,t) = h_0(t) \mbox{exp}(\beta Z_i^{T})\]

\begin{itemize}
\item $h(z,t)$ is the hazard at time $t$ ( simplifed as $h(t)$).
\item $h_0(t)$ is the baselines hazard.
\item $z_i$ are the covariates.
\item $\beta$ is a vector of regression parameters
\end{itemize}

(ii) The baseline hazard refers to a female chicken in the old enclosure
(iii) The 95 per cent confidence interval for a parameter \beta is given by
the formula
\beta ± 1.96(SE[ \beta ]) = \beta ± 1.96 Var ( \beta ) ,
where SE[\beta] is the standard error of the parameter \beta.
Thus, for the covariate z 1 =1 if Duck 0 otherwise, we have
95 per cent confidence interval =
− 0.210 ± 1.96 0.002 = − 0.210 ± 0.088 = { − 0.298, − 0.122}
95\% C.I.
z 1 = 1 if Duck 0 otherwise \beta 1 = (–0.298, -0.122)
z 2 = 1 if Goose 0 otherwise \beta 2 = (–0.049, 0.199)
z 3 = 1 if New enclosure 0 otherwise \beta 3 = (0.049, 0.201)
z 4 = 1 if Male 0 otherwise \beta 4 = (0.100, 0.300)
Page 11Subject CT4 %%%%%%%%%%%%%%%%%%%%%%%%%%%%%%%%%%%%%%%%%%%%%%%%%5— Examiners’ Report
(iv)
The parameter for the new enclosure is 0.125 so the
ratio of the hazard for two otherwise identical birds is
exp(0.125) = 1.133.
So the hazard appears to have got worse.
The 95\% confidence interval is entirely positive OR does not
include zero
so at the 95\% level the deterioration is statistically significant.
(v)
ALTERNATIVE 1
Hazard for a Male, Goose in the Old enclosure is
h 0 ( t ) exp (0.2 + 0.075 + 0) = h 0 ( t ) exp (0.275)
%-------------------------------------------------------%

Hazard for a Male, Goose in the Old enclosure is

\[h_0 (t) exp(0.2 + 0.075 + 0) = h_0 (t) exp(0.275)\]
Therefore the probability of still being alive in 6 months is
\begin{eqnarray*}
S_{\mbox{Goose}}
&=& exp \left[-\int^{6}_{0} h_{0}(t) exp(0.275) dt \\
&=& exp \left[-1.31653\int^{6}_{0} h_{0}(t) dt \\
\end{eqnarray*}

This is equal to 0.9 so

\[  frac{\ln 0.9}{1.31653} = - \int^{6}_{0} h_{0}(t) dt\


\[ \int^{6}_{0} h_{0}(t) dt = 0.080028951\]

Hazard of a Female, Duck in the New enclosure is

\[h_0 (t) exp(0 - 0.210 + 0.125) = h_0 (t) exp(-0.085)\]


So, the probability she is alive after 6 months is
6
S Duck = exp ⎡ ⎢ − \int h 0 ( t ) exp ( − 0.085 ) dt ⎤ ⎥
⎣ 0
⎦
= exp { − 0.080028951(0.918512284) }
= exp { − 0.073507574 }
= 0.929129
So the probability she’s dead is 0.07087
Page 12Subject CT4 %%%%%%%%%%%%%%%%%%%%%%%%%%%%%%%%%%%%%%%%%%%%%%%%%5— Examiners’ Report
ALTERNATIVE 2
Hazard for a Male, Goose in the Old enclosure is
h 0 ( t ) exp (0.2 + 0.075 + 0) = h 0 ( t ) exp (0.275)
Therefore the probability of still being alive in 6 months is
6
S Goose = exp ⎡ ⎢ − \int h 0 ( t ) exp ( 0.275 ) dt ⎤ ⎥
⎣ 0
⎦
Similarly, the probability of still being alive in 6 months for
A Female Duck in the New enclosure is
6
S Duck = exp ⎡ ⎢ − \int h 0 ( t ) exp ( − 0.085 ) dt ⎤ ⎥ .
⎣ 0
⎦
Therefore we can write
6
exp ⎡ − \int h 0 ( t ) exp ( 0.275 ) dt ⎤
S Goose
⎢ ⎣ 0
⎥ ⎦
=
,
6
S Duck exp ⎡ − h ( t ) exp ( − 0.085 ) dt ⎤
⎢ ⎣ \int 0 0
⎥ ⎦
whence
6
− \int h 0 ( t ) exp ( 0.275 ) dt
log e S Goose
exp(0.275)
0
=
=
.
6
log e S Duck − h ( t ) exp ( − 0.085 ) dt exp( − 0.085)
\int 0
0
Hence
log e S Duck =
log e S Goose [exp( − 0.085)]
.
exp(0.275)
Since S Goose = 0.9, therefore
log e S Duck =
log e 0.9[exp( − 0.085)]
= − 0.07351
exp(0.275)
So S Duck = 0.929129
So the probability she’s dead is 0.07087
\end{document}
