\documentclass[a4paper,12pt]{article}

%%%%%%%%%%%%%%%%%%%%%%%%%%%%%%%%%%%%%%%%%%%%%%%%%%%%%%%%%%%%%%%%%%%%%%%%%%%%%%%%%%%%%%%%%%%%%%%%%%%%%%%%%%%%%%%%%%%%%%%%%%%%%%%%%%%%%%%%%%%%%%%%%%%%%%%%%%%%%%%%%%%%%%%%%%%%%%%%%%%%%%%%%%%%%%%%%%%%%%%%%%%%%%%%%%%%%%%%%%%%%%%%%%%%%%%%%%%%%%%%%%%%%%%%%%%%

\usepackage{eurosym}
\usepackage{vmargin}
\usepackage{amsmath}
\usepackage{graphics}
\usepackage{epsfig}
\usepackage{enumerate}
\usepackage{multicol}
\usepackage{subfigure}
\usepackage{fancyhdr}
\usepackage{listings}
\usepackage{framed}
\usepackage{graphicx}
\usepackage{amsmath}
\usepackage{chngpage}

%\usepackage{bigints}
\usepackage{vmargin}

% left top textwidth textheight headheight

% headsep footheight footskip

\setmargins{2.0cm}{2.5cm}{16 cm}{22cm}{0.5cm}{0cm}{1cm}{1cm}

\renewcommand{\baselinestretch}{1.3}

\setcounter{MaxMatrixCols}{10}

\begin{document}
\begin{enumerate}

[Total 15]
A pet shop has four glass tanks in which snakes for sale are held. The shop can stock
at most four snakes at any one time because:
• if more than one snake were held in the same tank, the snakes would attempt to
eat each other and
• having snakes loose in the shop would not be popular with the neighbours
The number of snakes sold by the shop each day is a random variable with the
following distribution:
Number of Snakes Potentially Sold
in Day (if stock is sufficient) Probability
None
One
Two 0.4
0.4
0.2
If the shop has no snakes in stock at the end of a day, the owner contacts his snake
supplier to order four more snakes. The snakes are delivered the following morning
before the shop opens. The snake supplier makes a charge of C for the delivery.
(i) Write down the transition matrix for the number of snakes in stock when the
shop opens in a morning, given the number in stock when the shop opened the
previous day.

(ii) Calculate the stationary distribution for the number of snakes in stock when
the shop opens, using your transition matrix in part (i).
[4]
(iii) Calculate the expected long term average number of restocking orders placed
by the shop owner per trading day.

CT4 S2010—6If a customer arrives intending to purchase a snake, and there is none in stock, the sale
is lost to a rival pet shop.
(iv)
Calculate the expected long term number of sales lost per trading day.

The owner is unhappy about losing these sales as there is a profit on each sale of P.
He therefore considers changing his restocking approach to place an order before he
has run out of snakes. The charge for the delivery remains at C irrespective of how
many snakes are delivered.
(v) Evaluate the expected number of restocking orders, and number of lost sales
per trading day, if the owner decides to restock if there are fewer than two
snakes remaining in stock at the end of the day.

(vi) Explain why restocking when two or more snakes remain in stock cannot
optimise the shop’s profits.

The pet shop owner wishes to maximise the profit he makes on snakes.
(vii)
Derive a condition in terms of C and P under which the owner should change
from only restocking where there are no snakes in stock, to restocking when
there are fewer than two snakes in stock.

[Total 19]
END OF PAPER
CT4 S2010—7

%%%%%%%%%%%%%%%%%%%%%%%%%%%%%%%%%%%%%%%%%%%%%%%%%%%%%%%%%%%%%%%%%%%%%%%%%%%%%%%%%%%%%%%%%
Question 12
( i)
Start
previous
day
1
2
3
4
(ii)
Start morning
1
2
0.4
0.4
0.2
0
If stationary distribution is π = ( π 1 π 2
0
0.4
0.4
0.2
π 3
3 4
0
0
0.4
0.4 0.6
0.2
0
0.4
π 4 )
Then π A = π where A is the matrix in (i)
0.4 π 1 + 0.4 π 2 + 0.2 π 3 = π 1
0.4 π 2 + 0.4 π 3 + 0.2 π 4 = π 2
0.4 π 3 + 0.4 π 4 = π 3
0.6 π 1 + 0.2 π 2 + 0.4 π 4 = π 4
(a)
(b)
(c)
(d)
From (c) π 3 = 0.666666 π 4
From (b) π 2 = 0.7778 π 4
From (a) π 1 = 0.7407 π 4
π 1 + π 2 + π 3 + π 4 = 1 = (0.7407 + 0.7778 + 0.6666 + 1) π 4
Implies π 1 = 0.2325 , π 2 = 0.2442 , π 3 = 0.2093 , π 4 = 0.31395
OR π 1 =
(iii)
10
21
9
27
, π 2 =
, π 3 = , π 4 =
.
43
86
43
86
Probability of restocking is 0.6 if in π 1 and 0.2 if in π 2
So long term rate = 0.6 * 0.2325 + 0.2 * 0.2442 = 0.1884 per trading day
(iv)
Probability of losing a sale is 0.2 if in π 1
So expected lost sales per day = 0.2 * 0.2325 = 0.0465
Page 17 %%%%%%%%%%%%%%%%%%%%%%%%%%%%%%%%%%%%%%%%%%% — 
(v)
If restock when fewer than two in stock then transition matrix changes to:
Start previous
day
2
3
4
Start morning
2
3
0.4
0.4
0.2
0
0.4
0.4
4
0.6
0.2
0.4
Label stationary distribution \lambda . Then
0.4 \lambda 2 + 0.4 \lambda 3 + 0.2 \lambda 4 = \lambda 2
0.4 \lambda 3 + 0.4 \lambda 4 = \lambda 3
0.6 \lambda 2 + 0.2 \lambda 3 + 0.4 \lambda 4 = \lambda 4
(b1)
(c1)
(d1)
From (c1) \lambda 3 = 0.666666 \lambda 4
From (b1) \lambda 2 = 0.7778 \lambda 4
\lambda 2 = 0.3182 OR 7/22
\lambda 3 = 0.2727 OR 3/11
1
\lambda 4 =
= 0.4091 OR 9/22
(1 + 2 / 3 + 7 / 9)
As no more than two snakes sell per day,
there are no lost sales.
Probability of restocking 0.6 if in \lambda 2 and 0.2 in \lambda 3 = 0.2455
(vi)
Restocking at two or more snakes would not result in fewer lost sales than restocking
at 1.
Because the probability of selling more than 2 snakes is zero.
It would, however, result in more restocking charges than restocking at 1.
Therefore it must result in lower profits than restocking at 1 so is not optimal.
Page 18 %%%%%%%%%%%%%%%%%%%%%%%%%%%%%%%%%%%%%%%%%%% — 
(vii)
Costs if restock at 0
0.1884C + 0.0465 P
Costs if restock at 1
0.24546 C
So should change restocking approach if
0.24546 C < 0.1884 C + 0.0465 P
C < 0.8148 P
In this question, many candidates answered (i) and (ii), but made no further progress.
Candidates who wrote down the wrong matrix in (i) but evaluated the stationary distribution
correctly for the matrix they had written down were given full credit for (ii), and gained
credit in (iii) and (iv) if these parts were answered correctly given the matrix which had been
written down in (i). A common error was to write down a five-state model in (i). Few
candidates attempted the later sections of this question.
END OF 
Page 19
