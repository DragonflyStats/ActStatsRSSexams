\documentclass[a4paper,12pt]{article}

%%%%%%%%%%%%%%%%%%%%%%%%%%%%%%%%%%%%%%%%%%%%%%%%%%%%%%%%%%%%%%%%%%%%%%%%%%%%%%%%%%%%%%%%%%%%%%%%%%%%%%%%%%%%%%%%%%%%%%%%%%%%%%%%%%%%%%%%%%%%%%%%%%%%%%%%%%%%%%%%%%%%%%%%%%%%%%%%%%%%%%%%%%%%%%%%%%%%%%%%%%%%%%%%%%%%%%%%%%%%%%%%%%%%%%%%%%%%%%%%%%%%%%%%%%%%

\usepackage{eurosym}
\usepackage{vmargin}
\usepackage{amsmath}
\usepackage{graphics}
\usepackage{epsfig}
\usepackage{enumerate}
\usepackage{multicol}
\usepackage{subfigure}
\usepackage{fancyhdr}
\usepackage{listings}
\usepackage{framed}
\usepackage{graphicx}
\usepackage{amsmath}
\usepackage{chngpage}

%\usepackage{bigints}
\usepackage{vmargin}

% left top textwidth textheight headheight

% headsep footheight footskip

\setmargins{2.0cm}{2.5cm}{16 cm}{22cm}{0.5cm}{0cm}{1cm}{1cm}

\renewcommand{\baselinestretch}{1.3}

\setcounter{MaxMatrixCols}{10}

\begin{document}
\begin{enumerate}

© Institute of Actuaries1
Following a review of the results of a stochastic model run, an actuary requests that a
parameter is changed. The change is not expected to alter the results significantly,
but results on the final basis are required in order to complete a report. Unfortunately
the actuarial student who produced the original model run is away on study leave, and
so the revised run is assigned to a different student.
When the revised results are produced, they are significantly different from the
original results.
Discuss possible reasons why the results are different.

Question 1
One or both of the runs (the original or the new) may have been incorrect as, for example, the
second trainee may not have been fully aware of the set-up (for example he or she may not
have followed the procedure correctly, or may have used different assumptions)
The difference between the two runs may not have only been the parameter change, for
example the two runs may have used different random seeds, or the second run may have had
fewer simulations.
The expectation that the model was not sensitive to this parameter could have been incorrect.
Other valid points were given credit, for example that some parameters might be linked to
live data, which will necessarily have changed; or that there may have been other
amendments to the data in the meantime. However, the maximum number of marks
attainable on this question was 3.
\end{document}
