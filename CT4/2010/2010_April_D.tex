CT4 A2010—610
An airline runs a frequent flyer scheme with four classes of member: in ascending
order Ordinary, Bronze, Silver and Gold. Members receive benefits according to their
class. Members who book two or more flights in a given calendar year move up one
class for the following year (or remain Gold members), members who book exactly
one flight in a given calendar year stay at the same class, and members who book no
flights in a given calendar year move down one class (or remain Ordinary members).
Let the proportions of members booking 0, 1 and 2+ flights in a given year be p 0 , p 1
and p 2+ respectively.
(i)
(a) Explain how this scheme can be modelled as a Markov chain.
(b) Explain why there must be a unique stationary distribution for the
proportion of members in each class.
 
(ii)
Write down the transition matrix of the process.
 
The airline’s research has shown that in any given year, 40% of members book no
flights, 40% book exactly one flight, and 20% book two or more flights.
(iii)
Calculate the stationary probability distribution.
 
The cost of running the scheme per member per year is as follows:
Ordinary members
Bronze members
Silver members
Gold members
\$0
\$10
\$20
\$30
The airline makes a profit of \$10 per passenger for every flight before taking into
account costs associated with the frequent flyer scheme.
(iv)
Assess whether the airline makes a profit on the members of the scheme. 
[Total 13]
CT4 A2010—7
%%%%%%%%%%%%%%%%%%%%%%%%%%%%%%%%%%%%%%%%%%%%%%%%%%%%%%%%%%%%%%%%%%%
Page 13Subject CT4 %%%%%%%%%%%%%%%%%%%%%%%%%%%%%%%%%%%%%%%%%%%%%%%%%5— Examiners’ Report
10
(i)
(a)
The state space is discrete (with four states: O – ordinary passenger,
B – bronze member, S – silver member and G – gold member)
The probability that a passenger has a particular membership
status next year depends only on their membership status in the
current year (i.e. the status in previous years is not relevant).
Therefore the process is Markov.
(b)
The state space is finite and therefore there is at least one stationary
probability distribution.
Since any state can be reached from any other state, the
Markov chain is irreducible.
Therefore the stationary probability distribution is unique.
(ii)
The transition matrix P is:
O ⎛ p 0 + p 1
⎜
B ⎜ p 0
S ⎜ 0
⎜
G ⎝ 0
p 2 + 0
p 1 p 2 +
p 0 p 1
0 p 0
⎞
⎟
0
⎟
p 2 + ⎟
⎟
p 1 + p 2 + ⎠
0
where the probability that a passenger books i flights in a year is p i .
(iii)
Let the probability that a passenger is in state j according to the stationary
distribution be π j ( j = O , B , S , G ).
The π j are given by the general formula
π = π P .
With p 0 = 0.4 , p 1 = 0.4 and p 2 + = 0.2 , we therefore have the equations
π O = 0.8 π O + 0.4 π B
π B = 0.2 π O + 0.4 π B + 0.4 π S
π S = 0.2 π B + 0.4 π S + 0.4 π G
π G = 0.2 π S + 0.6 π G
We also know that
π O + π B + π S + π G = 1 .
Page 14
(1)
(2)
(3)
(4)Subject CT4 %%%%%%%%%%%%%%%%%%%%%%%%%%%%%%%%%%%%%%%%%%%%%%%%%5— Examiners’ Report
Using equation (1) we have
0.2 π O = 0.4 π B
so that
π O = 2 π B .
Substituting in equation (2) this yields
π B = 0.2(2 π B ) + 0.4 π B + 0.4 π S ,
so that
0.2 π B = 0.4 π S
and hence
π S = 0.5 π B .
Finally, substituting in equation (3) yields
0.5 π B = 0.2 π B + 0.4(0.5 π B ) + 0.4 π G ,
so that
0.1 π B = 0.4 π G
and hence
π G = 0.25 π B .
We therefore have
2 π B + π B + 0.5 π B + 0.25 π B = 1 ,
whence
π B =
1
4
= = 0.2667 ,
3.75 15
and the stationary distribution is
π O =
8
= 0.5333
15
Page 15Subject CT4 %%%%%%%%%%%%%%%%%%%%%%%%%%%%%%%%%%%%%%%%%%%%%%%%%5— Examiners’ Report
(iv)
π B = 4
= 0.2667
15
π S = 2
= 0.1333
15
π G = 1
= 0.0667
15
EITHER
The expected cost of the scheme per member per year is
(0 \times 0.5333) + (\$10 \times 0.2667) + (\$20 \times 0.1333) + (\$30 \times 0.0667) = \$7.33
For the scheme to be worth running, therefore, the average profit
per member per year must exceed \$7.33.
The profit per member is 0 for the 40% who book no flights,
\$10 for the 40% who book one flight, and \$10 m for the 20% who
book two or more flights, where m is the average number of flights
booked by those in the latter category.
For there to be a profit, we must have
(0.4 \times 0) + (0.4 \times \$10) + (0.2 \times \$10 m ) > 7.33
or
4 + 2 m > 7.33
2 m > 3.33
m > 1.67
This must be the case since m cannot be less than 2.
Therefore the airline makes a profit on the members of the scheme.
OR
Assuming that the distribution of the number of flights taken
is the same for all membership statuses, then for an Ordinary
member the expected profit is
(0.4 \times 0) + (0.4 \times 10) + (0.2 \times 20) = \$8
Similarly for the other classes of member the expected profit
is
Page 16Subject CT4 %%%%%%%%%%%%%%%%%%%%%%%%%%%%%%%%%%%%%%%%%%%%%%%%%5— Examiners’ Report
Bronze: (0.4 \times –10) + (0.4 \times 0) + (0.2 \times 10)
= \$ –2
= \$ –12
Silver: (0.4 \times –20) + (0.4 \times –10) + (0.2 \times 0)
Gold:
(0.4 \times –30) + (0.4 \times –20) + (0.2 \times –10) = \$ –22
In any one year, the proportions of members in each category
are given by the stationary distribution,
so the expected profit per member is
8
4
2
1
(\$8) + (\$ − 2) + (\$ − 12) + (\$ − 22) = \$0.667
15
15
15
15
This assumes no member makes more than 2 flights per year, so
is a minimum estimate of the profit.
This minimum estimate is positive, so the airline makes a profit.
