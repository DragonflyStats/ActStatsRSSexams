\documentclass[a4paper,12pt]{article}

%%%%%%%%%%%%%%%%%%%%%%%%%%%%%%%%%%%%%%%%%%%%%%%%%%%%%%%%%%%%%%%%%%%%%%%%%%%%%%%%%%%%%%%%%%%%%%%%%%%%%%%%%%%%%%%%%%%%%%%%%%%%%%%%%%%%%%%%%%%%%%%%%%%%%%%%%%%%%%%%%%%%%%%%%%%%%%%%%%%%%%%%%%%%%%%%%%%%%%%%%%%%%%%%%%%%%%%%%%%%%%%%%%%%%%%%%%%%%%%%%%%%%%%%%%%%

\usepackage{eurosym}
\usepackage{vmargin}
\usepackage{amsmath}
\usepackage{graphics}
\usepackage{epsfig}
\usepackage{enumerate}
\usepackage{multicol}
\usepackage{subfigure}
\usepackage{fancyhdr}
\usepackage{listings}
\usepackage{framed}
\usepackage{graphicx}
\usepackage{amsmath}
\usepackage{chngpage}

%\usepackage{bigints}
\usepackage{vmargin}

% left top textwidth textheight headheight

% headsep footheight footskip

\setmargins{2.0cm}{2.5cm}{16 cm}{22cm}{0.5cm}{0cm}{1cm}{1cm}

\renewcommand{\baselinestretch}{1.3}

\setcounter{MaxMatrixCols}{10}

\begin{document}

 
[Total 5]5
Ten years ago, a confectionery manufacturer launched a new product, the Scrummy
Bar. The product has been successful, with a rapid increase in consumption since the product was first sold. In order to plan future investment in production capacity, the
manufacturer wishes to forecast the future demand for Scrummy Bars. It has data on age-specific consumption rates for the past ten years, together with projections of the
population by age over the next twenty years. It proposes the following modelling strategy:
\begin{enumerate}
\item extrapolate past age-specific consumption rates to forecast age-specific consumption rates for the next 20 years
\item apply the forecast age-specific consumption rates to the projected population by
age to obtain estimated total consumption of the product by age for each of the
next 20 years
\item sum the results to obtaiN_the total demand for each year
Describe the advantages and disadvantages of this strategy.
\end{enumerate}

%%%%%%%%%%%%%%%%%%%%%%%%%%%%%%%%%%%%%%%%%%%%%%%%%%%%%%%%%%%%%%%%%%%%%%%%%%%%%%%%%%%%%%%%%%%%%%%%%%%%%%%%%%%%%%%
\newpage

4
\item (i)
As periodic and irreducible then all states are periodic, hence
probability of staying in any state is zero.
By law of total probability, P AA + P AB + P AC = 1.
But P AB = P AC and P AA = 0 so P AB = P AC = 0.5.
To be irreducible at least one of P BA or P CA must be greater than zero.
If P BA > 0 thento be periodic must have P CB = 0,
and to be irreducible P CA > 0,
and if P CA > 0 thento be periodic must have P BC = 0, and to be
irreducible P BA > 0.
So must have P BC = P CB = 0 and P BA = P CA = 1.
\item (ii)
1.0
B
0.5
A
0.5
5
C
1.0
%%%%%%%%%%%%%%%%%%%%%%%%%%%%%
Advantages
\begin{itemize}
\item The model is simple to understand and to communicate.
\item The model takes account of one major source of variation in consumption
rates, specifically age.
\item The model is easy and cheap to implement.
\item The past data on consumption rates by age are likely to be fairly accurate.
\item The model can be adapted easily to different projected populations OR takes
into account future changes in the population.
\end{itemize}
Disadvantages
\begin{itemize}
\item Past trends in consumption by age may not be a good guide to future trends.
\item Extrapolation of past age-specific consumption rates may be complex or
difficult and can be done in different ways.
\item Consumption of chocolate may be affected by the state of the economy,
e.g. whether there is a recession.
%%--- Page 5 %%%%%%%%%%%%%%%%%%%%%%%%%%%%%%%%%%%%%%%%%%%%%%%%%5— Examiners’ Report
\item Factors other than age may be important in determining consumption,
e.g. expenditure on advertising.
\item Consumption may be sensitive to pricing, which may change in the future.
\end{itemize}
A rapid increase in consumption rates is unlikely to be sustained
for a long period as there is likely to be an upper limit to
the amount of Scrummy Bars a person can eat.
\begin{itemize}
\item The projections of the future population by age may not be accurate, as
they depend on future fertility, mortality and migration rates.
\item The proposed strategy does not include any testing of the sensitivity of total demand to changes iN_the projected population, or variations in future consumption trends from that used in the model.
\item Unforeseen events such as competitors launching new products, or the nation becoming increasingly health-aware, may affect future consumption.
\item The consumption of Scrummy Bars may vary with cohort rather than age, and
the model does not capture cohort effects.
\end{itemize}
%Not all the points listed above were required for full credit. Other advantages, for example those related to business prospects, were also given credit.
\end{document}
