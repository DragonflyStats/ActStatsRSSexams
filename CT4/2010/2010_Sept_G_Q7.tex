\documentclass[a4paper,12pt]{article}

%%%%%%%%%%%%%%%%%%%%%%%%%%%%%%%%%%%%%%%%%%%%%%%%%%%%%%%%%%%%%%%%%%%%%%%%%%%%%%%%%%%%%%%%%%%%%%%%%%%%%%%%%%%%%%%%%%%%%%%%%%%%%%%%%%%%%%%%%%%%%%%%%%%%%%%%%%%%%%%%%%%%%%%%%%%%%%%%%%%%%%%%%%%%%%%%%%%%%%%%%%%%%%%%%%%%%%%%%%%%%%%%%%%%%%%%%%%%%%%%%%%%%%%%%%%%

\usepackage{eurosym}
\usepackage{vmargin}
\usepackage{amsmath}
\usepackage{graphics}
\usepackage{epsfig}
\usepackage{enumerate}
\usepackage{multicol}
\usepackage{subfigure}
\usepackage{fancyhdr}
\usepackage{listings}
\usepackage{framed}
\usepackage{graphicx}
\usepackage{amsmath}
\usepackage{chngpage}

%\usepackage{bigints}
\usepackage{vmargin}

% left top textwidth textheight headheight

% headsep footheight footskip

\setmargins{2.0cm}{2.5cm}{16 cm}{22cm}{0.5cm}{0cm}{1cm}{1cm}

\renewcommand{\baselinestretch}{1.3}

\setcounter{MaxMatrixCols}{10}

\begin{document}
\begin{enumerate}

[Total 6]
Two neighbouring small countries have for many years taken annual censuses of their
populations on 1 January in which each inhabitant must give his or her age. Country
A uses an “age last birthday” definition of age, whereas Country B uses an “age
nearest birthday” definition. Each country has also operated a system in which deaths
are recorded on an “age nearest birthday at date of death” basis.
On 30 June 2009 Country A invaded Country B and the two countries became one
state. The new government wishes to estimate a single set of age-specific death rates,
\mu x , for the new unified state using the census data taken in the years before the
invasion.
Derive a formula which the new government may use to estimate \mu x in terms of the
recorded number of deaths in each country, and the population of each country
recorded as being aged x in the censuses. State any assumptions you make.
[8]
CT4 S2010—3
PLEASE TURN OVER8
Rocky Bay is a small seaside town in the north of Europe. In a leaflet advertising the
town, the tourist office has claimed that “in August, Rocky Bay has a Mediterranean
climate”. An actuarial student spent August 2009 on holiday in Rocky Bay with his
family, and became sceptical of this claim. When he returned home, he thought it
might be interesting to examine the claim by applying some of the methods he had
learned while studying for the Core Technical subjects. For each of the 31 days in
August 2009 he collected data recorded by various meteorological offices on the
maximum temperature in Rocky Bay and the mean of the maximum temperatures
reported on the same day at a range of places in the Mediterranean region.
The data are shown below, where, for each of the days in August, “+” means that
Rocky Bay had the higher maximum temperature and “–“ means that the
Mediterranean average was higher.
1 2 3 4 5 6 7 8 9 10 11 12 13 14 15 16 17 18 19 20
- - - - - - - - - - - - + + + + - - - -
21 22 23 24 25 26 27 28 29 30 31
- - - - - - - - + + +
9
(i) Carry out a statistical test to examine the tourist office’s claim.

(ii) Suggest reasons why the test might not be an appropriate way to examine the
tourist office’s claim.



Question 7
We adjust the exposed to risk to correspond to the deaths data.
Deaths are recorded on an “age nearest birthday” basis. Let the number of deaths to persons
aged x in countries A and B respectively in year t be θ x A , t and θ Bx , t .
This means that the estimated rate \mu x will apply to exact age x, no further adjustment being
required.
Let the populations recorded in the censuses of the two countries as being aged x in the
census on 1 January in year t be P x A , t and P x B , t .
Page 6 %%%%%%%%%%%%%%%%%%%%%%%%%%%%%%%%%%%%%%%%%%% — 
A central exposed to risk for each country for year t which corresponds to the deaths data is
s = 1
E c x A , t =
∫
s = 1
P * x A , t + s ds
and
E cB x , t =
s = 0
∫
P * B x , t + s ds ,
s = 0
where P * x A , t + s and P * Bx , t + s are the populations aged x nearest birthday in countries A and B at
time t + s.
This central exposed to risk can be approximated by
E c x A , t =
1
( P * x A , t + P * x A , t + 1 )
2
and
E cB x , t =
1
( P * B x , t + P * B x , t + 1 ) ,
2
assuming the population varies linearly between census dates.
But in country A the census does not collect P * x A , t , but P x A , t , the population aged x last
birthday.
Assuming birthdays are evenly distributed across the calendar year, however, we can write
P * x A , t =
1 A
( P x , t + P x A − 1, t ) .
2
We also know that P * Bx , t = P x B , t .
Therefore an exposed to risk for the two countries combined which corresponds to the deaths
data is
1
1
E c x A , t + E cBx , t = ( P * x A , t + P * x A , t + 1 ) + ( P * B x , t + P * B x , t + 1 )
2
2
1 1
1
1
= ( ( P x A , t + P x A − 1, t ) + ( P x A , t + 1 + P x A − 1, t + 1 )) + ( P x B , t + P x B , t + 1 ) ,
2 2
2
2
and hence the combined age specific death rate can be estimated as
\mu x =
1 A 1 A
P x , t + P x − 1, t
4
4
θ x A , t + θ B x , t
.
1 A
1 A
1 B 1 B
+ P x , t + 1 + P x − 1, t + 1 + P x , t + P x , t + 1
4
4
2
2
Page 7 %%%%%%%%%%%%%%%%%%%%%%%%%%%%%%%%%%%%%%%%%%% — 
The solution above assumes that the estimates of \mu x are to be made using a single calendar
year t. Additional credit was given to candidates who stated that the appropriate time over
which the estimates are to be made should be defined at the outset, and that if this period is
longer than one year the deaths and exposed to risk for all relevant calendar years should be
summed and the total deaths divided by the total exposed to risk (subject to a maximum of 8
marks being available). The Examiners were looking for understanding of the process that
must be gone through in order to obtain the required estimates. Answers consisting mainly of
“disembodied” statements, without a coherent argument received limited credit.
