\documentclass[a4paper,12pt]{article}

%%%%%%%%%%%%%%%%%%%%%%%%%%%%%%%%%%%%%%%%%%%%%%%%%%%%%%%%%%%%%%%%%%%%%%%%%%%%%%%%%%%%%%%%%%%%%%%%%%%%%%%%%%%%%%%%%%%%%%%%%%%%%%%%%%%%%%%%%%%%%%%%%%%%%%%%%%%%%%%%%%%%%%%%%%%%%%%%%%%%%%%%%%%%%%%%%%%%%%%%%%%%%%%%%%%%%%%%%%%%%%%%%%%%%%%%%%%%%%%%%%%%%%%%%%%%

\usepackage{eurosym}
\usepackage{vmargin}
\usepackage{amsmath}
\usepackage{graphics}
\usepackage{epsfig}
\usepackage{enumerate}
\usepackage{multicol}
\usepackage{subfigure}
\usepackage{fancyhdr}
\usepackage{listings}
\usepackage{framed}
\usepackage{graphicx}
\usepackage{amsmath}
\usepackage{chngpage}

%\usepackage{bigints}
\usepackage{vmargin}

% left top textwidth textheight headheight

% headsep footheight footskip

\setmargins{2.0cm}{2.5cm}{16 cm}{22cm}{0.5cm}{0cm}{1cm}{1cm}

\renewcommand{\baselinestretch}{1.3}

\setcounter{MaxMatrixCols}{10}

\begin{document}
\begin{enumerate}
5
An investigation has been performed into risk factors for liver disease in persons
currently resident in the United Kingdom (UK) and aged over 50 years. It considered
the impact of three covariates: age at the start of the investigation, weekly alcohol
consumption and previous residence in a tropical country.
The investigation used a Cox regression model for the hazard of developing the
disease, h(t), with three parameters, β A , β C , and β T , as follows:
h ( t ) = h 0 ( t ) exp( β A A + β C C + β T T ).
A was defined as exact age at the start of the investigation less 50 years.
C represented weekly alcohol consumption, and took the value 1 if the person
consumed more than the recommended maximum per week (a heavy drinker) and 0
otherwise.
T represented previous residence in a tropical country, and took the value 1 if the
person had lived in a tropical country for more than 12 months and 0 otherwise.
(i)
State the characteristics of a person to whom the baseline hazard, h 0 (t),
applies.
[1]
The results of the investigation revealed that the hazard was:
• twice as high for a heavy drinker aged 60 years exact at the start of the
investigation than for a person aged 50 years exact at the start of the investigation
who was not a heavy drinker, where neither had previously lived in a tropical
country.
• four times as high for a heavy drinker who had previously lived in a tropical
country for more than 12 months than for a non-heavy drinker of the same age
who had not previously lived in a tropical country.
• three times as high for a person who had lived in a tropical country for more than
12 months than for a person of the same age and drinking habits who had always
lived in the UK.
(ii)
Calculate β A , β C , and β T .
[5]
The probability of a person aged 50 years exact at the start of the investigation, who
does not drink heavily and has always lived in the UK remaining free of the disease
for 10 years is 0.8.
(iii)
CT4 A2014–4
Show that the probability of a person of the same age and drinking habits, who
has lived for more than 12 months in a tropical country, remaining free of the
disease for 10 years is slightly over one half.
[4]
[Total 10]6
In the Poisson model, if the average number of events occurring to each member of a
population in a given period of time is λ, then the probability of observing exactly d
events occurring to any one individual in the same period of time is:
exp( −λ ) λ d
Pr[ D = d ] =
.
d !
(i)
Derive the maximum likelihood estimator under the Poisson model of the
average rate at which events occur, μ, in a population where the exposed to
risk for each person i is E.
[4]
A university runs a bus service between its teaching campus and its student halls of
residence. Traffic conditions mean that the arrival of buses at the bus stop on the
teaching campus can be considered to follow a Poisson process.
The university decided to commission a study of how long students typically have to
wait at the bus stop for a bus to arrive. Students were asked to record the time they
arrive at the stop, and the time the next bus arrived. Students who became tired of
waiting at the stop and left before the next bus arrived were asked to record the time
they left. Below are given data from 10 students.
Student Time arrived
1
2
3
4
5
6
7
8
9
10 4.00 p.m.
4.10 p.m.
4.20 p.m.
4.30 p.m.
4.40 p.m.
4.45 p.m.
4.55 p.m.
5.00 p.m.
5.10 p.m.
5.10 p.m.
Time left (if
left before
next bus
arrived)
Time next
bus arrived
4.05 p.m.
4.35 p.m.
4.30 p.m.
4.35 p.m.
4.50 p.m.
4.50 p.m.
5.05 p.m.
5.20 p.m.
5.40 p.m.
6.10 p.m.
(ii) Calculate the maximum likelihood estimate of the hourly rate at which buses
arrive at the bus stop, using the Poisson estimator, and assuming that only one
bus arrived at any given time.
[3]
(iii) Comment on the use of the Poisson model for this investigation.
CT4 A2014–5
[3]
[Total 10]
PLEASE TURN OVER

\newpage

4
(i) A life alive at time t should be included in the exposed-to-risk at age x at time
if and only if, were that life to die immediately, he or she would be included
in the deaths data d x at age x.
(ii) The death rate at age 45 relates to the deaths reported aged 46 next.
Dealing with the exposure in country A first. Census data are at age last, so
census data at age x correspond to the age next deaths data for age x+1.
E x c
1/1/13
=
∫
P x , t dt
1/1/12
Let P An be the population aged 45 last in country A on the census date in
year n.
Page 4Subject CT4 (Models Core Technical) – April 2014 – Examiners’ Report
Assuming the population varies linearly between census dates the central
exposed to risk aged 45 last in the calendar year 2012 can be approximated by:
1/12 * 1/2 * (P A 12 + (P A 11 + 11/12*(P A 12 − P A 11 ))) +
11/12 * 1/2 * (P A 12 + (P A 12 + 11/12*(P A 13 − P A 12 )))
= 1/24 * (381,000 + 380,417) + 11/24 * (381,000 + 384,667)
= 31,725.7 + 350,930.7
= 382,656
In country B we need to make the age definition of the exposure data match
the deaths data.
Assuming that birthdays are spread uniformly over the calendar year, half of
those aged 45 last will be aged 45 nearest and half will be aged 46 nearest.
Let P Bn be the population of country B aged 45 last at the census date in year
n then:
P B 11 = 0.5 (374,000 + 354,000) = 364,000
P B 12 = 0.5 (381,000 + 372,000) = 376,500
P B 13 = 0.5 (385,000 + 375,000) = 380,000
Assuming the population varies linearly between census dates the central
exposed to risk aged 45 last in the calendar year 2012 is:
7/12 * 1/2 * (P B 12 + (P B 11 +5/12*(P B 12 - P B 11 ))) +
5/12 * 1/2 * (P B 12 + (P B 12 +7/12*(P B 13 - P B 12 )))
= 7/24 * (369,208 + 376,500) + 5/24 * (376,500 + 377,958)
= 217,498.2 + 157,178.8
= 374,677.
Assuming the force of mortality is constant within each year of age
μ 45 =
(iii)
4,800
= 0.006338 for the calendar year 2012.
382, 656 + 374, 677
The rate interval is the life year starting at age 45 exact.
The estimate relates to the age in the middle of the rate interval, which is 45.5
years.
In part (i) the phrase “if and only if” was required for the full mark. Part (ii) was
demanding, and candidates struggled to obtain many marks. For full credit, the assumptions
needed to be stated at the correct place in the argument, and not just listed at the end. In
part (ii) full credit could be obtained for estimates of q 45 , provided the initial exposed to risk
was used.
A common error was to use the wrong age (i.e. 44 years last birthday). This was penalised in
part (ii), but full credit could be scored for part (iii) if the answer to part (iii) was consistent
with the age used in part (ii), and whether q 45 or μ 45 were estimated in part (ii).
Page 5Subject CT4 (Models Core Technical) – April 2014 – Examiners’ Report
A (slightly simpler) alternative answer to the exposed to risk for country A was:
1/12 * 1/2 * (P A11 + P A12 ) + 11/12 * 1/2 * (P A12 + P A13 )
= 1/24 (374,000 + 381,000) + 11/24 * (381,000 + 385,000)
= 31,458.3 + 351,083.3
= 382,542;
and for country B the corresponding was:
7/12 * 1/2 * (P A11 + P A12 ) + 5/12 * 1/2 * (P A12 + P A13 )
= 7/24 (364,000 + 376,500) + 5/24 * (376,500 + 380,000)
= 215,979.2 + 157,604.2
= 373,583,
giving a final answer of 0.00635. Because it is slightly simpler than the correct solution it
did not score full credit.
5
(i) A person who is aged 50 years at the start of the investigation, is not a heavy
drinker,
and has not lived for 12 months or more in a tropical country.
(ii) From the third result we have:
exp( β T ) = 3 ,
so that
β T = log e 3 = 1.099.
From the second result we have:
exp( β C + β T ) = exp( β C )exp( β T ) = 4 ,
hence
exp( β C ) =
4
,
3
and
β C = log e
4
= 0.2877.
3
Finally, using the first result we have:
exp( β C + 10 β A ) = exp( β C ) exp(10 β A ) = 2 ,
Page 6Subject CT4 (Models Core Technical) – April 2014 – Examiners’ Report
hence exp(10 β A ) =
and β A =
(iii)
6
,
4
1
6
log e = 0.0405.
10
4
The chance of a 50-year old non-heavy-drinking person who has always lived
in the UK remaining free of the disease for 10 years is:
⎛ 10
⎞
S T = 0 (10) = exp ⎜ − ∫ h 0 ( t ) dt ⎟ = 0.8.
⎜
⎟
⎝ 0
⎠
The chance of a person of the same age and drinking habits who has lived for
more than 12 months in a tropical country remaining free of the disease for 10
years is therefore:
⎛ 10
⎞
β T
S T = 1 (10) = exp ⎜ − ∫ h 0 ( t ) e β T dt ⎟ = 0.8 e = 0.8 3 = 0.512.
⎜
⎟
⎝ 0
⎠
Many candidates scored highly on this question, particularly on part (ii). In part (i) very few
candidates wrote that the baseline hazard referred to a person aged 50 at the start of the
investigation. Candidates who missed this important point were penalised. In part (iii) full
10
credit was given to answers that evaluated − ∫ h 0 ( t ) dt , though this was not necessary.
0
However, candidates who assumed that the baseline hazard was constant were penalised, as
this assumption is neither necessary nor correct for the Cox regression model.
