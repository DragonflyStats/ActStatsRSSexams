\documentclass[a4paper,12pt]{article}

%%%%%%%%%%%%%%%%%%%%%%%%%%%%%%%%%%%%%%%%%%%%%%%%%%%%%%%%%%%%%%%%%%%%%%%%%%%%%%%%%%%%%%%%%%%%%%%%%%%%%%%%%%%%%%%%%%%%%%%%%%%%%%%%%%%%%%%%%%%%%%%%%%%%%%%%%%%%%%%%%%%%%%%%%%%%%%%%%%%%%%%%%%%%%%%%%%%%%%%%%%%%%%%%%%%%%%%%%%%%%%%%%%%%%%%%%%%%%%%%%%%%%%%%%%%%

\usepackage{eurosym}
\usepackage{vmargin}
\usepackage{amsmath}
\usepackage{graphics}
\usepackage{epsfig}
\usepackage{enumerate}
\usepackage{multicol}
\usepackage{subfigure}
\usepackage{fancyhdr}
\usepackage{listings}
\usepackage{framed}
\usepackage{graphicx}
\usepackage{amsmath}
\usepackage{chngpage}

%\usepackage{bigints}
\usepackage{vmargin}

% left top textwidth textheight headheight

% headsep footheight footskip

\setmargins{2.0cm}{2.5cm}{16 cm}{22cm}{0.5cm}{0cm}{1cm}{1cm}

\renewcommand{\baselinestretch}{1.3}

\setcounter{MaxMatrixCols}{10}

\begin{document}
\begin{enumerate}

\item A survey of first marriage patterns among women in a remote population in central Asia collected the following data for a sample of women:
•
•
[6]
calendar year of birth
calendar year of first marriage

Data are also available about the population of never-married women on 1 January each year, classified by age last birthday.
You have been asked to estimate the intensity, λ x , of first marriage for women aged x.
\begin{enumerate}[(i)] 
\item State the rate interval implied by the first marriages data. 
\item Derive an appropriate exposed to risk which corresponds to the first marriages data. State any assumptions that you make. 
\item Explain to what age x your estimate of λ x applies. State any assumptions that you make.
\end{enumerate}
%%%%%%%%%%%%%%%%%%%%%%%%%%%%%%%%%%%%%%%%%%%%%%%%%%%%%%%%%%%%%%%%%%%%%%%%%%%%%%%%

\item 
(i) Calendar year rate interval starting on 1 January each year.
(ii) The first marriages data may be described as m x = number of first marriages, age x on the birthday in the
calendar year of marriage, during a defined period of investigation of length N years A definition of the population data which is compatible with these data on first marriages is
P x,t = number of lives under observation at time t since the start of the investigation who were aged x next birthday on the 1 January
immediately preceding t Since we follow each cohort of lives through each calendar year, this exposed
to risk is
N
E x c
=
∫ P x , t dt
0
which may be approximated as
E x c =
N − 1
∑ 2 ( P x , t + P x + 1, t + 1 )
1
0
(where the summation considers just integer values of t). This assumes that the population varies linearly across the
calendar year. However, we have data classified by age last birthday so we need to make a further adjustment.
If the number of lives aged x last birthday on 1 January in year t is P x,t * then
\[P x,t = P x-1,t *\]
and an appropriate exposed to risk in terms of the data we
have is
\[ E x c =
K + N
∑
t = K
1
( P x − 1, t * + P x , t + 1 * ) .\]

(iii)
The age range at the start of the rate interval is (x–1, x) exact.
So, assuming that birthdays are uniformly distributed across the calendar year the average age at the start of the rate interval is
x–1⁄2 and the average age in the middle of the rate interval is x.
Therefore the estimate of λ x applies to age x.
\end{enumerate}
%%%%%%%%%%%%%%%%%%%%%%%%%%%%%%%%%%%%%%%
6
(i)
Since we do not know the values of the rates in the crude experience but only the signs of the deviations the
tests we can carry out are limited. We can, however, perform the signs test and the grouping
of signs test.
(ii)
The signs test looks for overall bias. We have 25 ages, and at 18 of these the crude rates exceed the standard table rates (i.e. we have positive deviations) If the null hypothesis is true, then the observed number ofpositive deviations, P, will be such that P ~ Binomial (25, 1⁄2).
EITHER
We use the normal approximation to the Binomial distribution because we have a large number of ages (>20)
This means that, approximately, P ~ Normal (12.5, 6.25).
The z-score associated with the probability of getting 18 positive deviations if the null hypothesis is true is, therefore
17.5 − 12.5 − 5
=
= − 2.00 .
2.5
6.25
(using a continuity correction).
We use a two-tailed test, since both an excess of
positive and an excess of negative deviations are of interest.
Using a 5 % significance level, we have - 2.00 < - 1.96.
This means we have just sufficient evidence to reject the
null hypothesis.
Page 8Subject CT4 — Models Core Technical — April 2008 — Examiners’ Report
OR
Using the Binomial exactly we have
⎛ 25 ⎞ 25
⎟ 0.5 .
⎝ j ⎠
Pr[j positive deviations] = ⎜
So that the probability of obtaining 18 or more positive
25
deviations is
∑
j = 18
⎛ 25 ⎞ 25
⎜ ⎟ 0.5 .
⎝ j ⎠
This is equal to
(1 + 25 + 300 + 2,300 + 12,650 + 53,130 + 177,100 + 480,700)
× 0.0000000298
= 0.02164.
We apply a 2-tailed test, so we reject the null hypothesis at the 5\% level if this is less than 0.025
Since 0.02164 < 0.025 we reject the null hypothesis.
The grouping of signs test looks for long runs or clumps of ages with the same sign, indicating that the crude
experience is different from the standard experience over a substantial age range.
The number of runs of positive signs is 2 (65–72 years and 75–84 years).
We have 25 ages and 18 positive signs in total, which means 7 negative signs.
THEN EITHER
Using the table provided under n 1 = 18 and n 2 = 7, we find
that, under the null hypothesis, the greatest number of positive runs x for which the probability of x or fewer positive runs
is less than 0.05 is 3. Since we only have 2 runs, we conclude that the probability
of obtaining 2 or fewer runs is much less than 0.05.
Therefore we reject the null hypothesis.
%%%%%%%%%%%%%%%%%%%%%%%%%%%%%%%%%%%%%%%%%%%%%%%%%%%%%%%%%%%%%%%%%%%%%%%%%%%%%
OR
Using exact computation
⎛ 17 ⎞ ⎛ 8 ⎞
⎜ ⎟⎜ ⎟
0 1
8
Pr[1 positive run] = ⎝ ⎠ ⎝ ⎠ =
= 0.0000166
480, 700
⎛ 25 ⎞
⎜ ⎟
⎝ 18 ⎠
⎛ 17 ⎞ ⎛ 8 ⎞
⎜ ⎟⎜ ⎟
1
2
(17)(28)
Pr[2 positive runs] = ⎝ ⎠ ⎝ ⎠ =
= 0.000990
480, 700
⎛ 25 ⎞
⎜ ⎟
⎝ 18 ⎠
Therefore we conclude that the probability of obtaining 2 or fewer runs is much less than 0.05.
Therefore we reject the null hypothesis.
OR
Using the Normal approximation, the number of positive runs is distributed
⎛ (18)(8) [(18)(7)] 2 ⎞
N ⎜
,
⎟ = N ( 5.76,1.02 )
⎜ 25
(25) 3 ⎟ ⎠
⎝
so that the z-score associated with the probability of getting 2 runs
is
2 − 5.76
= − 3.722 .
1.02
which is much less than - 1.645 (using a 1-tailed test).
Therefore we conclude that the probability
of obtaining 2 or fewer runs is much less than 0.05.
Therefore we reject the null hypothesis.

\end{document}
