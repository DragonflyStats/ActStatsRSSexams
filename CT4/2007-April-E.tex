\documentclass[a4paper,12pt]{article}

%%%%%%%%%%%%%%%%%%%%%%%%%%%%%%%%%%%%%%%%%%%%%%%%%%%%%%%%%%%%%%%%%%%%%%%%%%%%%%%%%%%%%%%%%%%%%%%%%%%%%%%%%%%%%%%%%%%%%%%%%%%%%%%%%%%%%%%%%%%%%%%%%%%%%%%%%%%%%%%%%%%%%%%%%%%%%%%%%%%%%%%%%%%%%%%%%%%%%%%%%%%%%%%%%%%%%%%%%%%%%%%%%%%%%%%%%%%%%%%%%%%%%%%%%%%%

\usepackage{eurosym}
\usepackage{vmargin}
\usepackage{amsmath}
\usepackage{graphics}
\usepackage{epsfig}
\usepackage{enumerate}
\usepackage{multicol}
\usepackage{subfigure}
\usepackage{fancyhdr}
\usepackage{listings}
\usepackage{framed}
\usepackage{graphicx}
\usepackage{amsmath}
\usepackage{chngpage}

%\usepackage{bigints}
\usepackage{vmargin}

% left top textwidth textheight headheight

% headsep footheight footskip

\setmargins{2.0cm}{2.5cm}{16 cm}{22cm}{0.5cm}{0cm}{1cm}{1cm}

\renewcommand{\baselinestretch}{1.3}

\setcounter{MaxMatrixCols}{10}

\begin{document}
\begin{enumerate}
10
The members of a particular profession work exclusively in partnerships.
A certain partnership is concerned that it is losing trained technical staff to its
competitors. Informal debriefing interviews with individuals leaving the partnership
suggest that one reason for this is that the duration elapsing between becoming fully
qualified and being made a partner is longer in this partnership than in the profession
as a whole.
The partnership decides to investigate whether this claim is true using a multiple-state
model with three states: (1) fully qualified but not yet a partner, (2) fully qualified and
a partner, (3) working for another partnership. The period of the investigation is to
be 1 January 1997 to 31 December 2006.
(i)
(a) Draw and label a state-space diagram depicting the chosen model,
showing possible transitions between the three states.
(b) State any assumptions implied by the diagram you have drawn and
comment on their appropriateness.
[3]
(ii)
(a) State what data would be required in order to estimate the transition
intensity of moving from state (1) to state (2) for employees aged 30
years last birthday.
(b) Write down the likelihood of these data.
(c) Derive an expression for the maximum likelihood estimate of this
transition intensity.
The investigation assumes that all transition intensities are constant within
each year of age.
[7]
In order to estimate the corresponding transition intensity for competitors, the
partnership is compelled to rely on data kept by the relevant professional institute, of
which all fully qualified individuals must be members. The institute keeps data on the
numbers of members actively working on 1 January each year, classified by year of
birth, according to whether or not they are partners. It also keeps data on the number
of members who become partners each year, classified by age in completed years
upon election to partnership.
(iii)
Derive, using these data, an estimate for the profession as a whole of the
corresponding transition intensity of becoming a partner among persons aged
30 years last birthday during the period of the investigation. State any
assumptions you make.
%%%%%%%%%%%%%%%%%%%%%%%%%%%%%%%%%%%%%%%%%%%%%%%%%%%%%%%%%%%%%%%%%%%%%%%%%%%%%%%%%%%%%

10
(i)
(a)
A suitable diagram is shown below.
μ 21
x + t
1 Fully qualified but
not yet a partner
2 Fully qualified
and a partner
μ 12
x + t
μ 13
x + t
μ 31
x + t
μ 32
x + t
μ 23
x + t
3 Working for
another company
(b)
The chosen model ignores death among persons in the relevant age
groups. Since mortality in this age group among professional people is
likely to be low, this seems reasonable.
This diagram assumes that demotion is possible, i.e. some-one who has
become a partner can return to non-partnership status without leaving
the company.
The assumption is also made that a new employee joining from another
company can do so as a partner.
Credit was given for models based on alternative assumptions, provided these
were reasonable.
(ii)
(a)
Assume we have data on N individuals (i = 1, ..., N).
We should need to know for each individual:
• the total waiting time during the calendar years 1997–2006 in state
(1) when aged 30 last birthday
• whether or not the individual was made a partner between exact
ages 30 and 31 years during the calendar years 1997–2006 while
remaining in the company.
Page 15Subject CT4 — Models Core Technical — April 2007 — Examiners’ Report
(b)
The likelihood of the data is:
N
L = ∏ K exp[ − ( μ 13 + μ 12 ) v i ]( μ 12 ) d i
i = 1
where
v i is the waiting time at age 30 last birthday in state (1) for
individual i.
d i is an indicator variable such that d i = 1 if individual i was made a
partner while aged 30 last birthday during the period of the
investigation and d i = 0 otherwise.
K is a constant denoting terms that do not depend on μ 12 .
(c)
The logarithm of the likelihood is
N
log e L = ∑ log e K − ( μ 12 + μ 13 ) v i + d i log e μ 12
i = 1
Differentiating this with respect to μ 12 we obtain
N
∂ log e L
12
∂μ
N
= − ∑ v i +
i = 1
∑ d i
i = 1
12
μ
,
and setting this equal to zero and solving for μ 12 gives
N
12
μ ˆ
=
∑ d i
i = 1
N
∑ v i
.
i = 1
This is the maximum likelihood estimate, as can be seen by noting that
N
∂ 2 log e L
12 2
( ∂μ )
Page 16
=−
∑ d i
i = 1
12 2
( μ )
which must be negative.Subject CT4 — Models Core Technical — April 2007 — Examiners’ Report
(iii)
The data on becoming a partner are classified by age last birthday, which is
the same classification as used in the company’s own investigation, therefore
the relevant intensities will relate to the same age range.
For the correct exposed to risk we only consider those who are members of the
institute but not yet partners.
Let the number of such members in the census in year t who were born in year
s be P t , s .
All persons born in year s would be aged x last birthday on 1 January in year
s+x+1.
Therefore, assuming that the P t , s change linearly during each calendar year
the correct exposed to risk for the year 1997 is
1
( P 1997,1956 + P 1998,1957 )
2
and the exposed to risk for the entire 10-year period of the investigation is
t = 2006
1
( P t , t − 31 + P t + 1, t − 30 ) .
t = 1997 2
∑
If the number of persons becoming partners aged 30 last birthday in year t is
θ t , then an estimate of the relevant transition intensity is
t = 2006
∑
t = 2006
t = 1997
θ t
1
∑ 2 ( P t , t − 31 + P t + 1, t − 30 )
t = 1997
