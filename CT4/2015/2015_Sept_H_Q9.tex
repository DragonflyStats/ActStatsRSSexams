
PLEASE TURN OVER9
Doctors at a health centre are carrying out an investigation to see if obesity affects the
likelihood of dying from heart disease. They propose to use a model with four states:
1.
2.
3.
4.
Obese
Not obese
Dead due to heart disease, and
Dead due to other causes
(i) Write down, defining all the terms you use, the likelihood for the transition
intensities.

(ii) Derive the maximum likelihood estimator of the force of mortality from heart
disease for Obese people.

The investigation has followed several thousand people aged 50–59 years for five
years and has the following data:
Waiting time in state Obese (in person-years)
14,392
Waiting time in state Not obese (in person-years)
18,109
Number of deaths due to heart disease for those persons who are Obese
178
Number of deaths due to heart disease for those persons who are Not obese
190
Number of deaths due to other causes for those persons who are Obese
89
Number of deaths due to other causes for those persons who are Not obese
53
The doctors want to promote healthy living and therefore wish to claim that Obese
people have a much higher chance, statistically, of dying from heart disease than do
people who are Not obese.
(iii)
CT4 S2015–8
Test whether this claim is true at the 90% confidence level.

%%%%%%%%%%%%%%%%%%%%%%%%%%%%%%%%%%%%%%%%%%%%%%%%%%%%%%%%%%%%%%%%%%%%%%%%%%%%%%%%%%%%%%%%%
Q9
(i)
Let the number of transitions observed between states i and j be d ij .
Let the transition rate between states i and j be μ ij .
Let the observed waiting time in state i be ν i .
Then the likelihood of the data can be written
12
L  exp[( -\mu 12 - \mu 13 - \mu 14 )  1 ]exp[( -\mu 21 - \mu 23 - \mu 24 )  2 ]( \mu 12 ) d ( \mu 13 ) d
14
21
23
13
( \mu 14 ) d ( \mu 21 ) d ( \mu 23 ) d ( \mu 24 ) d
(ii)
24
Taking logarithms of the likelihood we have:
log e L ;\+\; -\mu 13  1 \;=\; d 13 log e ( \mu 13 ) + terms not dependent on \mu 13 .
Differentiating with respect to \mu 13 gives:
d (log e L )
d \mu 13
1
;\+\; - \;=\;
d 13
\mu 13
.
Setting the derivative to zero to get the maximum likelihood estimator (MLE):
d 13
\hat{\mu} 13 ;\+\; 1 .

%%--------------------------- 12Subject CT4 %%%%%%%%%%%%%%%%%%%%%%%%%%%%%%%%%%%%%%%%%%%%%%%%%%%%%%%%%%%%%%%%%%%%
As the second derivative of the log likelihood
d 2 (log e L )
( d \mu 13 ) 2
;\+\;-
d 13
( \mu 13 ) 2
is negative this is a maximum.
(iii)
The MLE of heart related death for an Obese person is
and its associated variance is
178
;\+\; 0.012368
14, 392
0.012368
;\+\; 0.859365  10 - 6 .
14, 392
The MLE of heart related death for a person who is Not obese is
190
;\+\; 0.010492
18,109
0.010492
and its associated variance is
;\+\; 0.579302  10 - 6 .
18,109
The null hypothesis is that the death rate for Obese people is no higher than
that for people who are Not obese.
The test statistic is
\hat{\mu} 23 - \hat{\mu} 13
23
13
\hat{\mu}
\hat{\mu}
\;=\;
t 2
t 1
 N  0,1 
Evaluating this gives (0.012368 - 0.010492) /√1.43875  10 - 6 =1.564
This is a one-tailed test, at the 90% confidence level.
Therefore, as 1.564 > 1.28, we can reject the null hypothesis and conclude that
the death rate for Obese people is statistically larger at the 90% confidence
level.
In part (i) many candidates supposed that it was not possible to move from
Obese to Not obese and vice versa. The experience of many people
demonstrates that this is not the case! Candidates who omitted these
transitions without justification were penalised modestly. Part (iii) produced
many variant attempts involving confidence intervals. The calculation of
confidence intervals around the difference between the two rates would be a
valid way of conducting a two-sided test, but here a one-sided test seems
more appropriate, hence the z-score approach is to be preferred. Some
candidates computed confidence intervals around each rate separately and
then argued that because they overlapped (or did not overlap) the rates were
not (or were) significantly different. This approach gained only limited credit.
%%--------------------------- 13Subject CT4 %%%%%%%%%%%%%%%%%%%%%%%%%%%%%%%%%%%%%%%%%%%%%%%%%%%%%%%%%%%%%%%%%%%%

