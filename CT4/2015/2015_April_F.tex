\documentclass[a4paper,12pt]{article}

%%%%%%%%%%%%%%%%%%%%%%%%%%%%%%%%%%%%%%%%%%%%%%%%%%%%%%%%%%%%%%%%%%%%%%%%%%%%%%%%%%%%%%%%%%%%%%%%%%%%%%%%%%%%%%%%%%%%%%%%%%%%%%%%%%%%%%%%%%%%%%%%%%%%%%%%%%%%%%%%%%%%%%%%%%%%%%%%%%%%%%%%%%%%%%%%%%%%%%%%%%%%%%%%%%%%%%%%%%%%%%%%%%%%%%%%%%%%%%%%%%%%%%%%%%%%

\usepackage{eurosym}
\usepackage{vmargin}
\usepackage{amsmath}
\usepackage{graphics}
\usepackage{epsfig}
\usepackage{enumerate}
\usepackage{multicol}
\usepackage{subfigure}
\usepackage{fancyhdr}
\usepackage{listings}
\usepackage{framed}
\usepackage{graphicx}
\usepackage{amsmath}
\usepackage{chngpage}

%\usepackage{bigints}
\usepackage{vmargin}

% left top textwidth textheight headheight

% headsep footheight footskip

\setmargins{2.0cm}{2.5cm}{16 cm}{22cm}{0.5cm}{0cm}{1cm}{1cm}

\renewcommand{\baselinestretch}{1.3}

\setcounter{MaxMatrixCols}{10}

\begin{document}
\begin{enumerate}

12
(iv) Derive an expression (in terms of the transition intensities) for the probability
that an islander who has never suffered from the disease will still be alive in
three years’ time.
[4]
(v) Set out the information which the students would need when they returned
three years later in order to calculate the rate of sickness from the disease. [2]
[Total 14]
A study was made of a group of people seeking jobs. 700 people who were just
starting to look for work were followed for a period of eight months in a series of
interviews after exactly one month, two months, etc. If the job seeker found a job
during a month, the job was assumed to have started at the end of the month.
Unfortunately, the study was unable to maintain contact with all the job seekers.
The data from the study are shown in the table below:
(i)
CT4 A2015–6
Months since
start of study Found employment Contact lost
1
2
3
4
5
6
7
8 100
70
50
40
20
20
12
6 50
0
20
20
30
60
38
0
(a) Describe two types of censoring present in the investigation.
(b) Describe an example of a person to whom each type applies.
[3](ii)
Calculate the Kaplan-Meier estimate of the function for “remaining without
employment”.
[6]
A Weibull distribution with a rate h(t) given by the formula h(t) = λ β  t 1 was fitted
to these data. The estimated value of λ was 0.18 and the estimated value of β was 0.3.
(iii)
Test the goodness-of-fit of the data to this Weibull distribution.
END OF PAPER
CT4 A2015–7
[6]
[Total 15]


12
(i)
Right censoring
The exact duration of the event is not known, but only that it
duration.
exceeds some
Example: job seekers with whom contact was lost during the investigation (or
those still seeking jobs at the end of the investigation)
Random censoring
The time at which contact was lost may be regarded as a random variable.
Example: a job seeker with whom contact was lost during the investigation.
Type I censoring
The censoring times were known in advance (as they were determined by the
fixed period of the investigation).
Example: a person still without work after 8 months.
Interval censoring
The censoring mechanism prevents us from knowing exactly when the event
of interest took place, only that it fell within a certain period.
Example: EITHER a person who actually found a job after 5.5 months (say) is
recorded as having found a job after 6 months;
OR a person who was still seeking work at the end of the investigation found a
job within the interval [8,∞)
Informative censoring
Censoring gives information about the lifetimes of those who remain (survival
function for each censored observation for t greater than the time of censoring
is the same as that for the non-censored observations).
Example: a person lost to the investigation will have a greater or lesser chance
of finding a job than those who remained.
[3]
Page 21 %%%%%%%%%%%%%%%%%%%%%%%%%%%%%%%%%%%%%%%%%%%%%%%%%%%% – April 2015 – Examiners’ Report
(ii)
The calculations are shown in the table below.
t j N j
0
1
2
3
4
5
6
7
8 700
700
550
480
410
350
300
220
170
d j
100
70
50
40
20
20
12
6
d j /N j 1  d j /N j
0.1429
0.1273
0.1042
0.0976
0.0571
0.0667
0.0545
0.0353 0.8571
0.8727
0.8958
0.9024
0.9429
0.9333
0.9455
0.9647
c j
50
0
20
20
30
60
38
164
 d j
The Kaplan-Meier estimate is S(t) =   1 
t j  t 
 n j
(iii)

 .


t Kaplan-Meier estimate of S(t)
0 ≤ t < 1
1 ≤ t < 2
2 ≤ t < 3
3 ≤ t < 4
4 ≤ t < 5
5 ≤ t < 6
6 ≤ t < 7
7 ≤ t < 8
t = 8 1.0000
0.8571
0.7480
0.6701
0.6047
0.5702
0.5321
0.5031
0.4854
The null hypothesis is that the durations at which job seekers find
work follow a Weibull distribution with parameters λ = 0.18 and β = 0.3.
Using the chi-squared test we have the following calculations:
t h(t) N j expected observed
1
2
3
4
5
6
7
8 0.1794
0.1104
0.0831
0.0680
0.0581
0.0512
0.0459
0.0418 700
550
480
410
350
300
220
170 125.55
60.72
39.90
27.86
20.35
15.35
10.11
7.11 100
70
50
40
20
20
12
6
z x
 2.28
1.19
1.60
2.30
 0.08
1.19
0.60
 0.42
The calculated value of the chi-squared statistic is 16.40.
Page 22
z x 2
5.20
1.42
2.56
5.29
0.01
1.41
0.36
0.17 %%%%%%%%%%%%%%%%%%%%%%%%%%%%%%%%%%%%%%%%%%%%%%%%%%%% – April 2015 – Examiners’ Report
This should be compared with the critical value at the 5% level
with 6 degrees of freedom (because we have eight ages and two parameters
have been fitted, and 8 – 6 = 2)
which is 12.59.
Since 16.40 > 12.59
we reject the null hypothesis that the time to employment follows the Weibull
distribution.
In part (i) credit was given for up to two different forms of censoring. For informative/non-
informative censoring, a candidate could decide that censoring is EITHER informative OR
non-informative and gain credit for a sensible explanation and example which are consistent
with this decision. In part (ii) a common error was to suggest that the estimate of S(t)
extended to values of t above 8. This is not the case as there is no information in the data
about what might happen after 8 months. In part (ii) some candidates decided that censoring
precedes the event. Provided they explained this, full credit was given.
A substantial proportion of candidates did not attempt part (iii). Of those who did, the
approach given above was the most common. Little credit was given to candidates who tried
to compare hazards or survival functions directly using the chi-squared test, as the
assumptions of the test are not met. A common error was to use an exposed-to-risk of 700 to
compute the expected deaths at all durations.
END OF EXAMINERS’ REPORT
Page 23
