\documentclass[a4paper,12pt]{article}

%%%%%%%%%%%%%%%%%%%%%%%%%%%%%%%%%%%%%%%%%%%%%%%%%%%%%%%%%%%%%%%%%%%%%%%%%%%%%%%%%%%%%%%%%%%%%%%%%%%%%%%%%%%%%%%%%%%%%%%%%%%%%%%%%%%%%%%%%%%%%%%%%%%%%%%%%%%%%%%%%%%%%%%%%%%%%%%%%%%%%%%%%%%%%%%%%%%%%%%%%%%%%%%%%%%%%%%%%%%%%%%%%%%%%%%%%%%%%%%%%%%%%%%%%%%%

\usepackage{eurosym}
\usepackage{vmargin}
\usepackage{amsmath}
\usepackage{graphics}
\usepackage{epsfig}
\usepackage{enumerate}
\usepackage{multicol}
\usepackage{subfigure}
\usepackage{fancyhdr}
\usepackage{listings}
\usepackage{framed}
\usepackage{graphicx}
\usepackage{amsmath}
\usepackage{chngpage}

%\usepackage{bigints}
\usepackage{vmargin}

% left top textwidth textheight headheight

% headsep footheight footskip

\setmargins{2.0cm}{2.5cm}{16 cm}{22cm}{0.5cm}{0cm}{1cm}{1cm}

\renewcommand{\baselinestretch}{1.3}

\setcounter{MaxMatrixCols}{10}

\begin{document}
4
Company A and Company B are two small insurance companies which have recently
merged to form Company C. Company C is reviewing its premium rates for a whole
of life product and so is conducting an analysis of mortality rates experienced.
Company A recorded the number of policies in force every 1 January using a
definition of age next birthday whereas Company B recorded the number of policies
in force every 1 April using an age definition of age last birthday. Both companies
recorded deaths as they happened using an age definition of age last birthday.
These are the data for the most recent years.
Age next
birthday
51
52
53
Age last
birthday
51
52
53
Company A
Number of
Number of
policies
policies
1 Jan. 2012
1 Jan. 2013
8,192
7,684
9,421
6,421
8,298
8,016
8,118
7,187
9,026
Company B
Number of
Number of
policies
policies
1 April 2012
1 April 2013
4,496
5,281
4,992
Number of
policies
1 Jan. 2014
3,817
5,218
5,076
Number of
policies
1 April 2014
4,872
3,812
5,076
In the calendar year 2013 Company A recorded 28 deaths of those aged 52 last
birthday and Company B recorded 17 deaths of those aged 52 last birthday.
(i)
(ii)
CT4 S2015–3
Estimate the force of mortality for the combined company for age 52 last
birthday, stating all assumptions that you make.
Explain the exact age to which your estimate applies.
[6]

%%%%%%%%%%%%%%%%%%%%%%%%%%%%%%%%%%%
\newpage

Q4
(i)
The deaths data carries more information, so the exposed to risk data must be
amended to correspond with the deaths data.
The exposed to risk may be calculated using the census formula
1
E x c ;\+\;  P x , t dt ,
0
where P x is the population aged x last birthday.
For Company A, 53 next corresponds to 52 last.
Assuming that the population varies linearly between census dates, this can be
approximated using the trapezium rule.
So E x c ;\+\;
1 
( P x \;=\; 1,1/1/13 \;=\; P^{\ast} x \;=\; 1,1/1/14 ) ,
2
where P^{\ast} x is the population aged x next birthday
= (8,016 + 9,026) / 2
= 8,521
For Company B the age definition does not need adjusting.
Again assuming the population varies linearly between census dates we can
calculate the population at 1/1/2013 as
5,218 + (3/12) (5,281  5,218) = 5,233.75,
%%--------------------------- 5Subject CT4 %%%%%%%%%%%%%%%%%%%%%%%%%%%%%%%%%%%%%%%%%%%%%%%%%%%%%%%%%%%%%%%%%%%%
and the exposed to risk for the first three months of 2013 is
(1/2) (3/12) (5,218 + 5,233.75) = 1,306.4688.
Similarly the population at 1/1/2014 is
5,218 - (9/12)(5,218  3,812) = 4,163.5
and the exposed to risk for the last nine months of 2013 is
(1/2)(9/12)(5,218 + 4,163.5) = 3,518.0625.
Assuming that the force of mortality is constant over the year of age,
we have \hat{\mu} 52 ;\+\;
(ii)
28 \;=\; 17
;\+\; 0.0034
8, 521 \;=\; 1, 306.47 \;=\; 3, 518.06
The estimate \mû 52 applies to the age at the middle of the rate interval, which is
age 52.5 exact.
The credit for stating the assumptions was only given if they were stated in
the right place in the argument. In fact, few candidates stated the correct
assumptions and many stated assumptions that were unnecessary (e.g. that
deaths should be uniformly distributed across the year of age). Very few
candidates realised that it was necessary to assume a constant force of
mortality across the year of age.
%%--------------------------- 20
\end{document}
