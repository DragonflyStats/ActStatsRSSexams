\documentclass[a4paper,12pt]{article}

%%%%%%%%%%%%%%%%%%%%%%%%%%%%%%%%%%%%%%%%%%%%%%%%%%%%%%%%%%%%%%%%%%%%%%%%%%%%%%%%%%%%%%%%%%%%%%%%%%%%%%%%%%%%%%%%%%%%%%%%%%%%%%%%%%%%%%%%%%%%%%%%%%%%%%%%%%%%%%%%%%%%%%%%%%%%%%%%%%%%%%%%%%%%%%%%%%%%%%%%%%%%%%%%%%%%%%%%%%%%%%%%%%%%%%%%%%%%%%%%%%%%%%%%%%%%

\usepackage{eurosym}
\usepackage{vmargin}
\usepackage{amsmath}
\usepackage{graphics}
\usepackage{epsfig}
\usepackage{enumerate}
\usepackage{multicol}
\usepackage{subfigure}
\usepackage{fancyhdr}
\usepackage{listings}
\usepackage{framed}
\usepackage{graphicx}
\usepackage{amsmath}
\usepackage{chngpage}

%\usepackage{bigints}
\usepackage{vmargin}

% left top textwidth textheight headheight

% headsep footheight footskip

\setmargins{2.0cm}{2.5cm}{16 cm}{22cm}{0.5cm}{0cm}{1cm}{1cm}

\renewcommand{\baselinestretch}{1.3}

\setcounter{MaxMatrixCols}{10}

\begin{document}

In a computer game a player starts with three lives. Events in the game which cause
the player to lose a life occur with a probability  dt  o ( dt ) in a small time interval dt.
However the player can also find extra lives. The probability of finding an extra life
in a small time interval dt is  dt  o ( dt ). The game ends when a player runs out of
lives.
\begin{enumerate}
\item (i) Outline the state space for the process which describes the number of lives a
player has.
\item
(ii) Draw a transition graph for the process, including the relevant transition rates.
\item 
(iii) Determine the generator matrix for the process.
\item 
(iv) Explain what is meant by a Markov jump chain.
\item 
(v) Determine the transition matrix for the jump chain associated with the process.
\item 
(vi) Determine the probability that a game ends without the player finding an extra
life.
\end{enumerate}
\newpage

%%%%%%%%%%%%%%%%%%%%%%%%%%%%%%%%%%%%%%%%%%%%%5
i)
10
{0,1,2,3,4....}
(ii)

0

1

(iii)

2

3


4


Generator matrix
Lives
0
1
2
3
4
0
0
0
0
 0


0
0
   (    )
 0

 (    )

0


 (    )

0
 0
 0

 (    )
0
0
 
 .
(iv)
...
. 







.  
EITHER
\begin{itemize}
\item If a Markov jump process X t is examined only at the times of transition, the
resulting process is called the jump chain associated with X t .
OR
\item 
A jump chain is each distinct state visited in the order visited where the time
set is the times when states are moved between.
(v)
Lives
0
1
2
3
4
...
1
0
0
0
0
etc. 



0
 / (    )
0
0
  / (    )



0
 / (    )
0
 / (    )
0


 / (    )
 / (    )
0
0
0




 / (    )
0
0
0
0
 
 
etc.


Page 15 %%%%%%%%%%%%%%%%%%%%%%%%%%%%%%%%%%%%%%%%%%%%%%%%%%%% – April 2015 – Examiners’ Report
3
(vi)
  


  
\item Many candidates scored respectably on this question. The most common error was to define the state space as {0, 1, 2, 3}, ignoring the possibility that the player could, at any time, have found more extra lives than (s)he has lost. Candidates who used this state space were
penalised in parts (i) and (ii) but could score full credit for later parts by carrying through their answer correctly. The other commonly occurring errors were to allow a transition out of state 0 (this is not possible, as the game ends when the player has no lives left), or to
ignore the absorbing state 0 completely.
\item Again, these errors were penalised in parts (i) and (ii) but credit was given for correctly following through into later parts. Candidates who ignored state 0 completely were unable to give a sensible answer to part (vi). In part (iv) a
disappointing number of candidates simply provided a general definition of a Markov chain (i.e. repeating the answer to Question 7, part (i)), rather than relating the Markov jump chain
to a Markov jump process.
\end{itemize}
\end{document}
