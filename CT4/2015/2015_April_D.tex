\documentclass[a4paper,12pt]{article}

%%%%%%%%%%%%%%%%%%%%%%%%%%%%%%%%%%%%%%%%%%%%%%%%%%%%%%%%%%%%%%%%%%%%%%%%%%%%%%%%%%%%%%%%%%%%%%%%%%%%%%%%%%%%%%%%%%%%%%%%%%%%%%%%%%%%%%%%%%%%%%%%%%%%%%%%%%%%%%%%%%%%%%%%%%%%%%%%%%%%%%%%%%%%%%%%%%%%%%%%%%%%%%%%%%%%%%%%%%%%%%%%%%%%%%%%%%%%%%%%%%%%%%%%%%%%

\usepackage{eurosym}
\usepackage{vmargin}
\usepackage{amsmath}
\usepackage{graphics}
\usepackage{epsfig}
\usepackage{enumerate}
\usepackage{multicol}
\usepackage{subfigure}
\usepackage{fancyhdr}
\usepackage{listings}
\usepackage{framed}
\usepackage{graphicx}
\usepackage{amsmath}
\usepackage{chngpage}

%\usepackage{bigints}
\usepackage{vmargin}

% left top textwidth textheight headheight

% headsep footheight footskip

\setmargins{2.0cm}{2.5cm}{16 cm}{22cm}{0.5cm}{0cm}{1cm}{1cm}

\renewcommand{\baselinestretch}{1.3}

\setcounter{MaxMatrixCols}{10}

\begin{document}
\begin{enumerate}
[Total 8]9
(i) Describe an example of a situation when graduation by parametric formula
would be used.
[1]
(ii) State two advantages and two disadvantages of graduation by parametric
formula.
(iii)
10
[4]
(a) Explain why the \chi^2 test is different when considering the goodness of fit of graduated data compared with when considering the similarity of two sets of data.
(b) Describe how this is dealt with when the graduation has been carried
out by parametric formula.

[Total 9]
%%%%%%%%%%%%%%%%%%%%%%%%%%%%%%%%%%%%%%%%%%%%%%%%%%%%%%%%%%%%%%%%%%%
9
(i)
When using a large experience
EITHER to produce a standard table.
OR where a suitable formula can be found to fit at all ages.
(ii)
Advantages
It is straightforward to extend the statistical theory of estimation from one parameter to several.
Provided a reasonably small number of parameters is used, the resulting graduation will be acceptably smooth.
When comparing several experiences, the same parametric formula can be fitted to all of them. Differences between the parameters, given their standard errors, give insight into the differences between the experiences.

Disadvantages
It can be difficult to find a single formula to fit at all ages.
Care is required when extrapolating. The fit of the curve will probably be best where there is most data, but results where data are scanty (e.g. at extreme ages) may be poor and require adjustment.
%%%%%%%%%%%%%%%%%%%%%%%%%%%%%%%%
(iii)
(a)
The $\chi^2$ test compares an “observed” experience with an “expected” experience.
It is essential when making this comparison that the two sets of experiences be independent.
This is normally the case when considering the similarity of two sets of data.
However when comparing the difference between an “observed” experience and that “expected”from graduated data, there is a problem
because the graduated data have been derived from the observed experience. Because of this, we need to make it easier to reject the null hypothesis, and we achieve this by reducing the number of degrees of freedom used in the chi-squared test.
(b)
When the graduation has been carried out by parametric formula, we reduce the degrees of freedom by one for each
parameter estimated from the observed data.

This was a bookwork question, to which answers were very disappointing. In part (ii) many
candidates were unable to reproduce the points made in Unit 12, page 7 of the Core Reading.
In part (iii) most candidates knew that the number of degrees of freedom should be reduced,

and many wrote that the reduction was equal to the number of parameters in the formula used for the graduation. But very few candidates explained why the reduction was needed (para. 5.4 in Unit 12, page 9 of the Core Reading).
%%--- Page 14 %%%%%%%%%%%%%%%%%%%%%%%%%%%%%%%%%%%%%%%%%%%%%%%%%%%% – April 2015 – Examiners’ Report
\end{document}
