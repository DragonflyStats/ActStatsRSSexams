\documentclass[a4paper,12pt]{article}

%%%%%%%%%%%%%%%%%%%%%%%%%%%%%%%%%%%%%%%%%%%%%%%%%%%%%%%%%%%%%%%%%%%%%%%%%%%%%%%%%%%%%%%%%%%%%%%%%%%%%%%%%%%%%%%%%%%%%%%%%%%%%%%%%%%%%%%%%%%%%%%%%%%%%%%%%%%%%%%%%%%%%%%%%%%%%%%%%%%%%%%%%%%%%%%%%%%%%%%%%%%%%%%%%%%%%%%%%%%%%%%%%%%%%%%%%%%%%%%%%%%%%%%%%%%%

\usepackage{eurosym}
\usepackage{vmargin}
\usepackage{amsmath}
\usepackage{graphics}
\usepackage{epsfig}
\usepackage{enumerate}
\usepackage{multicol}
\usepackage{subfigure}
\usepackage{fancyhdr}
\usepackage{listings}
\usepackage{framed}
\usepackage{graphicx}
\usepackage{amsmath}
\usepackage{chngpage}

%\usepackage{bigints}
\usepackage{vmargin}

% left top textwidth textheight headheight

% headsep footheight footskip

\setmargins{2.0cm}{2.5cm}{16 cm}{22cm}{0.5cm}{0cm}{1cm}{1cm}

\renewcommand{\baselinestretch}{1.3}

\setcounter{MaxMatrixCols}{10}

\begin{document}
\begin{enumerate}
PLEASE TURN OVER7
(i)
Describe what is meant by a Markov chain.
[2]
A simplified model of the internet consists of the following websites with links between the websites as shown in the diagram below.

N(ile) B(anana)
C(heep) H(andbook)

An internet user is assumed to browse by randomly clicking any of the links on the website he is on with equal probability.

(ii)
8

Calculate the transition matrix for the Markov chain representing which
website the internet user is on.

(iii) Calculate, of the total number of visits, what proportion are made to each
website in the long term.
%%%%%%%%%%%%%%%%%%%%%%%%%%%%%%%%%%%%%%%%%%%%%%%%%%%%%%%%%%%%%%%%%%%%
[Total 8]
(i) State why it is important to divide data into homogeneous classes when
undertaking mortality investigations.
(ii)

List four factors, apart from smoking behaviour, by which mortality data are
often classified by life insurance companies.

In a particular life insurance market, it has for many years been the practice for all companies to charge smokers higher premiums than non-smokers for the same term assurance policy. Suppose one company decides to switch to charging smokers and
non-smokers the same premiums for term assurance policies. The other companies
retain differential pricing for smokers and non-smokers.
(iii)
CT4 A2015–4
Discuss the likely implications for the company making the switch.
[4]



%%%%%%%%%%%%%%%%%%%%%%%%%%%%%%
7
(i)
A Markov chain is a stochastic process with discrete states operating in
discrete time in which
EITHER
P[X t  A  X s 1 = x 1 , X s 2 = x 2 , ..., X s n = x n , X s = x] = P[X t  A  X s = x]
for all times s 1 < s 2 < ... < s n < s < t, all states x 1 , x 2 , ..., x n , x in S and all
subsets A of S
OR
the probabilities of moving from one state to another depend only on the
present state of the process: the history of the process before the current state
is irrelevant.
%%--- Page 10 %%%%%%%%%%%%%%%%%%%%%%%%%%%%%%%%%%%%%%%%%%%%%%%%%%%% – April 2015 – Examiners’ Report
(ii)
This is based on the number of links to the site and where they go.
\[P = \bordermatrix{

  & N & B & C H H \cr
N  & 0 & 1/2 & 1/2 & 0 \cr
B  & 1/2 & 0 &1/2 0 \cr
C  & 1/3 & 1/3 0 1/3 \cr
H  & 0 & 0 & 1 & 0 \cr
}\]

%%%%%%%%%%%%%%%%%%%%%%%%%%%%%%%%%%%%%%%%%%%%%%%%%%%%%%%5
Stationary distribution satisfies $\pi \;=\; \pi P$

\begin{enumerate}
\item ${\displaystyle
\frac{1}{2} \pi_B + \frac{1}{3}\pi_C \;=\; \pi_N }$

\item ${\displaystyle
\frac{1}{2} \pi_N + \frac{1}{3}\pi_C \;=\; \pi_B }$

\item ${\displaystyle
\frac{1}{3} \pi_N + \frac{1}{2} \pi_B + \pi_H \;=\; \pi_C }$
\item ${\displaystyle
\frac{1}{3} \pi_C \;=\; \pi_H }$
\end{enumerate}

\[ \pi_N + \pi_B + \pi_H + \pi_C = 1\]


%%%%%%%%%%%%%%%%%%%%%%%%%%%%%%%%%%%%%%%%%%%%%%%%%%%%%5
From (1) and (2)
 N   B
From (3) and (4)
 N   B 
2
 C
3
1 
 2 2
So    1    C  1
3 
 3 3

%%--- Page 11 
%%%%%%%%%%%%%%%%%%%%%%%%%%%%%%%%%%%%%%%%%%%%%%%%%%%% – April 2015 – Examiners’ Report

Hence
 N 
1
1
3
1
,  B  ,  C  ,  H 
4
4
8
8

%% Both parts of this question were well answered by most candidates. In the final answer to part (ii), it was important for candidates to indicate which probability applied to which state, rather than just listing four numbers.

%%%%%%%%%%%%%%%%%%%%%%%%%%%%%%%%%%%%%%%%%%%%%%%%%%%%%%%%%%%%%%%%%%%%%%%%%%%%%%%%%%%%%%%%%
\newpage

8
(i)
All our models and analyses are based on the assumption that we can
observe groups of identical lives (or at least, lives whose mortality characteristics are the same).
Although in practice, this is never possible. We can at least subdivide our data according to characteristics known, from
experience, to have a significant effect on mortality.
This ought to reduce the heterogeneity of each class so formed.
(ii) Sex
Age
Type of policy
Level of underwriting
Duration in force
Sales channel
Policy size
Occupation or socio-economic group
Known impairments/medical history
Postcode/geographic location
Marital status
(iii) EITHER
If the company changing its policy charges both smokers and non-smokers a premium equal to the rate typically charged to smokers, then, relative to other companies, it will become poor value for non-smokers.
The company changing its policy will therefore lose business from non-smokers (whom it will charge more than an actuarially fair premium).
The portfolio will (eventually) be made up mostly of smokers (whom it will charge an actuarially fair premium).
The volume of business sold is likely to decrease, possibly to the extent that it does not cover the expenses estimated in the pricing basis.

%%--- Page 12 %%%%%%%%%%%%%%%%%%%%%%%%%%%%%%%%%%%%%%%%%%%%%%%%%%%% – April 2015 – Examiners’ Report
OR
If the company changing its policy charges both smokers and non-smokers a premium equal to the rate typically charged to non-smokers, then relative to other companies, it will become good value for smokers (and acceptable value for non-smokers).
The company changing its policy will therefore attract more business
from smokers (whom it will charge less than an actuarially fair premium). This is a form of anti-selection.
The smoker business is likely to be unprofitable, although the increase in business will reduce the overheads per policy
This is likely to lead to losses for the company changing its policy.
OR
If the company changing its policy charges both smokers and non-smokers a premium somewhere between the rate typically charged to smokers and the rate typically charged to non-smokers, then relative to other companies, it becomes good value for smokers and poor value for non-smokers.
The company changing its policy will therefore attract business from smokers and lose business from non-smokers (whom it will charge more than an actuarially fair premium). This is a form of anti-selection.
The smoker business is likely to be unprofitable, but any remaining non-smoker business will be profitable.
This may eventually lead to losses of the company changing its policy. This question was generally well answered, and part (iii) was very well answered. In part (i) the question asked about the reasons why data are subdivided when undertaking
investigations, so the answers were expected to reflect the assumptions underlying our models, rather than the convenience of users (e.g. pricing issues). In part (iii) most candidates implicitly supposed that the new single premium was between the previous smoker
and non-smoker premiums, though few explicitly stated this. For full credit the Examiners
were looking for consideration of at what level the single premium might be set and the
consequences of this: hence the range of alternative approaches given above.
Page 13 %%%%%%%%%%%%%%%%%%%%%%%%%%%%%%%%%%%%%%%%%%%%%%%%%%%% – April 2015 – Examiners’ Report
\end{document}
