\documentclass[a4paper,12pt]{article}

%%%%%%%%%%%%%%%%%%%%%%%%%%%%%%%%%%%%%%%%%%%%%%%%%%%%%%%%%%%%%%%%%%%%%%%%%%%%%%%%%%%%%%%%%%%%%%%%%%%%%%%%%%%%%%%%%%%%%%%%%%%%%%%%%%%%%%%%%%%%%%%%%%%%%%%%%%%%%%%%%%%%%%%%%%%%%%%%%%%%%%%%%%%%%%%%%%%%%%%%%%%%%%%%%%%%%%%%%%%%%%%%%%%%%%%%%%%%%%%%%%%%%%%%%%%%

\usepackage{eurosym}
\usepackage{vmargin}
\usepackage{amsmath}
\usepackage{graphics}
\usepackage{epsfig}
\usepackage{enumerate}
\usepackage{multicol}
\usepackage{subfigure}
\usepackage{fancyhdr}
\usepackage{listings}
\usepackage{framed}
\usepackage{graphicx}
\usepackage{amsmath}
\usepackage{chngpage}

%\usepackage{bigints}
\usepackage{vmargin}

% left top textwidth textheight headheight

% headsep footheight footskip

\setmargins{2.0cm}{2.5cm}{16 cm}{22cm}{0.5cm}{0cm}{1cm}{1cm}

\renewcommand{\baselinestretch}{1.3}

\setcounter{MaxMatrixCols}{10}

\begin{document}
%%-Question 1
For a simple random walk:

\begin{enumerate}
    \item (i) Define the process.
\item 
(ii) Write down the nature of the state space and time space in which it operates.
\item 
(iii) Describe an example of a practical application of the process.
\end{enumerate}
[Total 4]
The mortality of a rare form of flying beetle is being studied. It has been discovered
that beetles kept in a protected environment have a constant force of mortality, μ, but
that those in the wild have a force of mortality which is 50% higher. It has been
proven that the beetles revert immediately to the higher rate of mortality if they are
released from the protected environment.

A beetle born and always living in the wild has a 58% chance of living for eight days.
\begin{itemize}
\item Calculate the probability of living the same length of time for:
3
4
(a) a beetle born and reared in the protected environment.
(b) a beetle born in the protected environment which is scheduled to be released
into the wild after six days.
[4]
(i) Explain what is meant by a proportional hazards model.
(ii) Outline three reasons why the Cox proportional hazards model is widely used
in empirical work.

\begin{enumerate}
\item (i) List the stages you would go through in creating a model.
\item (ii) Discuss, for three of these stages, the specific issues that could arise when
creating a model to price a new sickness benefit product.
\end{enumerate}
%%%%%%%%%%%%%%%%%%%%%%%%%%%%%%%%%%%%%%%%%%%
\newpage
Question 1
\begin{itemize}
\item (i)
This is defined as X n = Y 1 + Y 2 +... Y n
where the random variables Y j (the steps of the walk) are mutually
independent with the common probability distribution:
Pr[Y j = 1] = p,
Pr[Y j = 1] = 1  p.
\item (ii) It operates in discrete time with a discrete state space.
\item (iii) Any reasonable practical application
e.g. cumulative results of the Oxford vs Cambridge boat race (net lead
of Cambridge over Oxford) measured annually.
\item OR how much a gambler has won or lost if he wins or loses
£1 on every bet.
OR number of cars in a car park controlled by a single entry/exit
barrier measured after each time the barrier goes up.
\item Most candidates answered parts (i) and (ii) well, though many missed the point about the Y t
being independent. Some of the examples in part (iii) were rather contrived.
\end{itemize}
%%%%%%%%%%%%%%%%%%%%%%%%%%%%%%%%%%%%%%%%%%%%%%%%%%%%%%%%%%%%%%%%%%%%%%
\newpage
%%-- Question 2
\begin{itemize}
\item (a)
A beetle in the wild has a force of mortality equal to 3μ/2.
\item So for a beetle in the wild
we have S (8)  exp[  8(3  / 2)]  exp(  12  )  0.58
\item Hence
12μ = 0.5447
μ = 0.0454.
\item Therefore a beetle reared in the protected environment will have an 8 day
survival probability of
exp(8*0.0454) = 0.6955.
Page 3 %%%%%%%%%%%%%%%%%%%%%%%%%%%%%%%%%%%%%%%%%%%%%%%%%%%% – April 2015 – Examiners’ Report
\item (b)
A beetle in the protected environment has a probability of surviving 6 days
equal to
exp(6*0.0454) = 0.7616,
and a probability of surviving 2 days in the wild of
exp (2*1.5*0.0454) = 0.8727
\item Therefore this beetle’s probability of surviving 8 days is
0.7615 × 0.8727 = 0.6646.
item This was the best answered question on the entire paper, with many candidates scoring full
marks. The most common error was to assume the rate in the wild was twice, rather than 1.5
times, the rate in the protected environment in part (a). If this was carried through correctly
into the rest of the answer then credit was given for subsequent calculations.
\end{itemize}
\newpage
%%%%%%%%%%%%%%%%%%%%%%%%%%%%%%%%%%%%%%5
%%-- Question 3
\begin{itemize}
    \item (i)
In a proportional hazards model the hazard of experiencing an event
may be factorised into two components:
one depending only on duration since some start event, which is known as the
baseline hazard, and the other depending only on a set of covariates and
associated parameters.
\item Thus the ratio between the hazards for any two individuals with different
values of the covariates is constant across all durations.
The baseline hazard applies to an individual with the value zero on all
covariates.
\item (ii)
The proportionality of the hazards makes estimating the impact of
covariates on the hazard straightforward (through partial likelihood).
Widely available statistical software packages have built-in routines for the
Cox model.
\item The Cox model is semi-parametric so the baseline hazard does not
need to be specified, and can be determined by the data (as with a Kaplan-
Meier hazard).
It ensures that the hazard is always positive.
It is easy to communicate.
\item There were some good attempts at this question. However, many candidates seemed to think
that the Cox model and the proportional hazards (PH) model were the same thing. In fact,
the Cox model is just one of a class of PH models. 
\item In part (i) we looked for knowledge of the
attractive characteristics of PH models in general, whereas in part (ii) we gave credit for
advantages of the Cox model in particular, as well as for general attributes of PH models
%%--Page 4 %%%%%%%%%%%%%%%%%%%%%%%%%%%%%%%%%%%%%%%%%%%%%%%%%%%% – April 2015 – Examiners’ Report
which are useful in practice. In part (i) some candidates simply wrote down a formula for a
PH model. If all the terms were defined, partial credit was given for this.
\end{itemize}
\end{document}
