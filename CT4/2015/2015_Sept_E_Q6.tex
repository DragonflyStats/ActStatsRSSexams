\documentclass[a4paper,12pt]{article}

%%%%%%%%%%%%%%%%%%%%%%%%%%%%%%%%%%%%%%%%%%%%%%%%%%%%%%%%%%%%%%%%%%%%%%%%%%%%%%%%%%%%%%%%%%%%%%%%%%%%%%%%%%%%%%%%%%%%%%%%%%%%%%%%%%%%%%%%%%%%%%%%%%%%%%%%%%%%%%%%%%%%%%%%%%%%%%%%%%%%%%%%%%%%%%%%%%%%%%%%%%%%%%%%%%%%%%%%%%%%%%%%%%%%%%%%%%%%%%%%%%%%%%%%%%%%

\usepackage{eurosym}
\usepackage{vmargin}
\usepackage{amsmath}
\usepackage{graphics}
\usepackage{epsfig}
\usepackage{enumerate}
\usepackage{multicol}
\usepackage{subfigure}
\usepackage{fancyhdr}
\usepackage{listings}
\usepackage{framed}
\usepackage{graphicx}
\usepackage{amsmath}
\usepackage{chngpage}

%\usepackage{bigints}
\usepackage{vmargin}

% left top textwidth textheight headheight

% headsep footheight footskip

\setmargins{2.0cm}{2.5cm}{16 cm}{22cm}{0.5cm}{0cm}{1cm}{1cm}

\renewcommand{\baselinestretch}{1.3}

\setcounter{MaxMatrixCols}{10}

\begin{document}
6
(i)
Describe what is meant by a proportional hazards model.

A pharmaceutical company is interested in testing a new treatment for a debilitating
but non-fatal condition in cows. A randomised trial was carried out in which a sample
of cows with the condition was assigned to either the new treatment or the previous
treatment. The event of interest was the recovery of a cow from the condition. The
results were analysed using a Cox regression model.
The final model estimated the hazard, h ( t , x ) as:
h(t, x) = h 0 (t) exp(\beta_0 z + \beta_1 x + \beta_2 xz),
where:
h 0 (t) is the baseline hazard;
z is a covariate taking the value 1 if the cow was assigned the new treatment and 0 if
the cow was assigned the previous treatment;
x is a covariate denoting the length of time (in days) for which the cow had been
suffering from the condition when treatment was started;
and t is the number of days since treatment started.
\beta_0 , \beta_1 and \beta_2 are parameters. Their estimated values were \beta_0 = 0.8, \beta_1 = 0.4
and \beta_2 = - 0.1 .
(ii)
Determine the characteristics of the baseline cow.

For a particular cow, the new treatment and the previous treatment have exactly the
same hazard.
(iii)
Calculate the number of days for which that cow had the condition before the
initiation of treatment.

Under the previous treatment, cows whose treatment began after they had been
suffering from the condition for three days had a median recovery time of 14 days
once treatment had started.
(iv)
CT4 S2015–5
Calculate the proportion of these cows which would still have had the
condition after 14 days if they had been given the new treatment.

Q6
(i)
A proportional hazards model is used to estimate the effect of covariates on
the hazard of experiencing an event.
In a proportional hazards model the hazard is assumed to factorise into two
components, one depending only on duration, and the other depending only on
the covariates.
The ratio between the hazards for persons with any two values of a covariate is
the same at all durations.
%%--------------------------- 7Subject CT4 %%%%%%%%%%%%%%%%%%%%%%%%%%%%%%%%%%%%%%%%%%%%%%%%%%%%%%%%%%%%%%%%%%%%
(ii) A cow who started the previous treatment immediately the condition appeared.
(iii) h 0 (t) exp(\beta_0 + \beta_1 x + \beta_2 x) = h 0 (t) exp(\beta_1 x)
exp(\beta_0 ) = exp(- \beta_2 x)
\beta_0 = - \beta_2 x
0.8 = 0.1x
So x = 8 days.
(iv)
The median recovery time is the value of t such that S(t) = 0.5.
For the previous treatment, we have
14
14




0.4(3)


S (14) ;\+\; exp  - e
h
(
t
)
dt
exp
3.320
h
(
t
)
dt
;\+\;
-
 0 
 0   ;\+\; 0.5 .


0
0




14
So
 h 0 ( t ) dt ;\+\;
0
log e (0.5)
;\+\; 0.209.
- 3.320
For the new treatment we have
14
S (14) ;\+\; exp( - e 0.8 \;=\; 1.2 - 0.3  h 0 ( t ) dt )
0
;\+\; exp[ - 5.474(0.209)] ;\+\; 0.319 .
Answers to part (ii) were disappointing. Some candidates wrote that the cow
has not suffered from the condition before, which may or may not have been
the case. Others wrote that the cow was not suffering from the condition
when the treatment started, which was absurd as the treatment would not
have been given were the cow not suffering from the condition. Part (iii) was
well answered by most candidates. Part (iv) was more demanding, but it was
disappointing that so many candidates framed their answers entirely in terms
of the hazard function, whereas the question clearly refers to the median
recovery time and hence implies that the survival function is required.
%%--------------------------- 20
\end{document}
