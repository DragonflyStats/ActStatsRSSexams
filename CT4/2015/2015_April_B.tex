\documentclass[a4paper,12pt]{article}

%%%%%%%%%%%%%%%%%%%%%%%%%%%%%%%%%%%%%%%%%%%%%%%%%%%%%%%%%%%%%%%%%%%%%%%%%%%%%%%%%%%%%%%%%%%%%%%%%%%%%%%%%%%%%%%%%%%%%%%%%%%%%%%%%%%%%%%%%%%%%%%%%%%%%%%%%%%%%%%%%%%%%%%%%%%%%%%%%%%%%%%%%%%%%%%%%%%%%%%%%%%%%%%%%%%%%%%%%%%%%%%%%%%%%%%%%%%%%%%%%%%%%%%%%%%%

\usepackage{eurosym}
\usepackage{vmargin}
\usepackage{amsmath}
\usepackage{graphics}
\usepackage{epsfig}
\usepackage{enumerate}
\usepackage{multicol}
\usepackage{subfigure}
\usepackage{fancyhdr}
\usepackage{listings}
\usepackage{framed}
\usepackage{graphicx}
\usepackage{amsmath}
\usepackage{chngpage}

%\usepackage{bigints}
\usepackage{vmargin}

% left top textwidth textheight headheight

% headsep footheight footskip

\setmargins{2.0cm}{2.5cm}{16 cm}{22cm}{0.5cm}{0cm}{1cm}{1cm}

\renewcommand{\baselinestretch}{1.3}

\setcounter{MaxMatrixCols}{10}

\begin{document}
\begin{enumerate}
5
(i)
State the principle of correspondence as it applies to death rates.
[1]
A nightclub opens at 10.00 p.m. and closes at 2.00 a.m. It admits only people aged
over 21 years on the production of an identity card giving date of birth.
The table below shows the number of people entering in various intervals between
10.00 p.m. and 2.00 a.m. on 30 June 2013. No-one was admitted after 1.00 a.m., and
you may assume that all those who enter the premises stay until 2.00 a.m.
Year of
birth 10.00–11.30
p.m. 11.30–12.00
p.m. 12.00 p.m.–1.00 a.m.
1989
1990
1991
1992 100
200
150
100 300
400
400
250 200
350
300
200
During the period of opening, 40 people aged 22 last birthday required medical
attention for heat exhaustion.
(ii)
6
Calculate the rate per person-hour at which those attending the night club aged
22 last birthday required medical attention for heat exhaustion, stating any
assumptions you make.
[6]
[Total 7]
A health insurance company has collected data on sickness rates during the calendar
year 2013 among a sample of its policyholders aged 40–64 years inclusive. It
compares these to the rates among its policyholders of the same age in 2012. It finds
that at ages 40–50 years inclusive, and at ages 56–61 years inclusive, the sickness
rates in 2013 are higher than those in 2012. At other ages, the sickness rates in 2013
were lower than those in 2012.
(i) Carry out two tests of the null hypothesis that the underlying sickness rates in
2013 are the same as those in 2012.
[6]
(ii) Comment on the implications of the results of your test for the company’s
sickness insurance business.
[2]
[Total 8]
CT4 A2015–3
%%%%%%%%%%%%%%%%%%%%%%%%%%%%%%%%%%%%%%%%%%
5
(i) A life alive at age x at time t should be included in the
exposed-to-risk if and only if, were that life to die immediately,
his or her death would be included in the deaths at age x, d x .
(ii) Those aged 22 last birthday on 30 June 2013 were born between 1 July 1990
and 30 June 1991, so half of them were born in 1990 and half in 1991.
Assuming that birthdays are evenly distributed across calendar years,
The number of persons aged 22 last birthday entering during each period is
10.00 – 11.30 p.m.
11.30 p.m. – 12.00 midnight
12.00 midnight – 1.00 a.m.
0.5(200 + 150) = 175
0.5(400 + 400) = 400
0.5(350 + 300) = 325
THEN EITHER
The number of persons aged 22 last birthday in the nightclub at 10.00 p.m.,
11.30 p.m., 12.00 midnight, 1.00 a.m. and 2.00 a.m. is therefore
10.00 p.m.
11.30 p.m.
12 midnight
1.00 a.m.
2.00 a.m.
0
175
575
900
900
Using the census approximation and assuming that arrivals are evenly
distributed across time,
the exposed to risk in person-hours is
 0  175 
 175  575   575  900 
1.5 
  0.5 
  1 
  900
2
2
 2 

 

 131.25  187.5  737.5  900
= 1,956.25
OR
Using the census approximation and assuming that arrivals are evenly
distributed across time,
the exposed-to-risk in person-hours is
175(3.25) + 400(2.25) + 325(1.5) = 1,956.25.
Page 7 %%%%%%%%%%%%%%%%%%%%%%%%%%%%%%%%%%%%%%%%%%%%%%%%%%%% – April 2015 – Examiners’ Report
AND HENCE
the rate of requiring medical attention is
40
 0.02045 per person hour.
1,956.25
Common errors were to use the wrong year or years of birth, to fail to cumulate the arrivals
(i.e. to realise that once inside the building, customers remained until 2.00 a.m.), and to
forget the final hour, during which the club was full and no more customers entered. In
general, answers to this question were rather better than answers to similar questions on
other recent examination papers. In part (ii) candidates were expected to relate the
assumptions to the specific stage of the derivation to which they applied. Candidates who
wrote down lists of assumptions – some relevant to the answer, others not – scored little
credit.
6
(i)
Signs Test
Under the null hypothesis, the number of positive deviations (2013 higher than
2012) is distributed Binomial (25, 0.5).
We have 17 positive deviations
ALTERNATIVE 1: NORMAL APPROXIMATION
As the number of ages is large enough, we can use the normal approximation,
 25 25 
in which the number of positive deviations is distributed Normal  ,  .
 2 4 
THEN EITHER
 17  12.5 
A z-score for 17 positive deviations is 
  1.8
 6.25 
OR
with a continuity correction a z-score for 17 positive deviations is
 16.5  12.5 

  1.6.
6.25 

AND HENCE
Since 1.8 (or 1.6) < 1.96 (2-tailed test)
we do not have sufficient evidence to reject the null hypothesis.
Page 8 %%%%%%%%%%%%%%%%%%%%%%%%%%%%%%%%%%%%%%%%%%%%%%%%%%%% – April 2015 – Examiners’ Report
ALTERNATIVE 2: EXACT TEST
 25 
Pr[exactly 17 positive signs] =   0.5 25  0.0322 .
 17 
Since 0.0322 > 0.025 (2-tailed test)
we do not have sufficient evidence to reject the null hypothesis.
Grouping of Signs Test
We have 25 age groups, 17 positive signs, and 2 positive groups
ALTERNATIVE 1: NORMAL APPROXIMATION
Using the Normal approximation (as we have more than 20 ages),
 17(8  1) (17 *8) 2 
the number of positive groups is distributed Normal 
,
 ,
(25) 3 
 25
which is Normal(6.12, 1.18).
 2  6.12 
We therefore compute a z-score for 2 runs as 
   3.79 .
 1.18 
Since Pr (z < 3.79) << 0.05 (one-tailed test) (or -3.79 < -1.645),
we reject the null hypothesis.
ALTERNATIVE 2: EXACT CALCULATION
Probability of getting 2 or fewer positive groups is
 16   9   16   9 
     
9
576
 0  1    1  2  

 0.000541
1, 081,575 1, 081,575
 25 
 25 
 
 
 17 
 17 
Since this is less than 0.05
we reject the null hypothesis.
ALTERNATIVE 3: USING THE TABLE IN THE “GOLD BOOK”
Using the table on p. 189 of the “Gold Book”, with n 1 = 17, n 2 = 8
the table shows that we reject the null hypothesis with 3 or fewer runs of
positive signs.
Page 9 %%%%%%%%%%%%%%%%%%%%%%%%%%%%%%%%%%%%%%%%%%%%%%%%%%%% – April 2015 – Examiners’ Report
Since we only have 2 positive runs and 2 < 3 we reject the null hypothesis.
(ii)
The results of the Signs Test suggest that the underlying rates in 2013 are not
systematically higher or lower than those in 2012.
The null hypothesis was rejected by the Grouping of Signs Test implying
that the shape of the distribution of sickness rates in 2013 is different from
that in 2012.
However this is only one year’s data and the company might wait to see if a
trend develops, or investigate whether there was a specific factor operating in
2012 or 2013 which caused the change.
If the shape of sickness rates makes them markedly different in 2013 from
2012 at ages where much business is sold, this will have implications for
profitability and pricing.
In part (i) most candidates used the Normal approximation for the Signs Test. If the exact
version was used, it is not necessary to compute Pr[17 or more positive signs] as Pr[exactly
17 positive signs] > 0.025. The correct value for Pr[17 or more positive signs] is 0.0538.
Answers to part (ii) were poor. Full credit was given for summarising the immediate
implications of the results obtained in part (i) for the comparison of the rates in 2013 and
2012 and for making some comment about the potential financial implications for the
company. Thus full credit could be obtained for less than is given in the model solution
above. Comments in part (ii) that were consistent with the actual results obtained by the
candidate in part (i) were given credit, even if the tests in part (i) were performed incorrectly
and reached conclusions different from those above.
