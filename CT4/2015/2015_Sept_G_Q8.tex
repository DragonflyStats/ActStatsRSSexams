\documentclass[a4paper,12pt]{article}

%%%%%%%%%%%%%%%%%%%%%%%%%%%%%%%%%%%%%%%%%%%%%%%%%%%%%%%%%%%%%%%%%%%%%%%%%%%%%%%%%%%%%%%%%%%%%%%%%%%%%%%%%%%%%%%%%%%%%%%%%%%%%%%%%%%%%%%%%%%%%%%%%%%%%%%%%%%%%%%%%%%%%%%%%%%%%%%%%%%%%%%%%%%%%%%%%%%%%%%%%%%%%%%%%%%%%%%%%%%%%%%%%%%%%%%%%%%%%%%%%%%%%%%%%%%%

\usepackage{eurosym}
\usepackage{vmargin}
\usepackage{amsmath}
\usepackage{graphics}
\usepackage{epsfig}
\usepackage{enumerate}
\usepackage{multicol}
\usepackage{subfigure}
\usepackage{fancyhdr}
\usepackage{listings}
\usepackage{framed}
\usepackage{graphicx}
\usepackage{amsmath}
\usepackage{chngpage}

%\usepackage{bigints}
\usepackage{vmargin}

% left top textwidth textheight headheight

% headsep footheight footskip

\setmargins{2.0cm}{2.5cm}{16 cm}{22cm}{0.5cm}{0cm}{1cm}{1cm}

\renewcommand{\baselinestretch}{1.3}

\setcounter{MaxMatrixCols}{10}

\begin{document}
8
(i)
Define a Markov Jump Process.

A company provides phones on contracts under which it is responsible for repairing
or replacing any phones which break down.
When a customer reports a fault with a phone, it is immediately taken to the
company’s repair shop and it is assessed whether it can be fixed (meaning fixable at
reasonable cost). Based on previous experience, it is estimated that the probability of
a phone being fixable is 0.75. If a phone is not fixable it is discarded and the
customer is provided with a new phone.
If a repaired phone breaks again the company, in line with its customer charter, will
not attempt to repair it again, and so discards the phone and replaces it with a new
one.
The status of a phone is to be modelled as a Markov Jump Process with state space
{Never Broken (NB), Repaired (R), Discarded (D)}.
The company considers the rate at which phones break down to vary according to
whether a phone has previously been repaired as follows:
(ii)
Status Probability of break down in small
interval of time, dt
Never Broken
Repaired 0.1dt + o(dt)
0.2dt + o(dt)
Draw a transition diagram for the possible transitions between the states,
including the associated transition rates.

Let P NB (t), P R (t) and P D (t) be the probabilities that a phone is in each state after time t
since it was provided as a new phone.
(iii) Determine Kolmogorov’s forward equations in component form for P NB (t),
P R (t) and P D (t).

(iv) Solve the equations in part (iii) to obtain P NB (t) and P R (t).
(v) Calculate the probability that a phone has not been discarded by time t.

[Total 10]
CT4 S2015–7
%%%%%%%%%%%%%%%%%%%%%%%%%%%%%%%%%%%%%%%%%%%%%%%%%%%%%%%%%%%%%%%%%%%%%%%%%%%%%%%%%
\newpage
Q8
(i)
A process with a continuous time space and discrete state space
satisfying the Markov property that
EITHER
the future progression of the process does not depend on the history of the
process prior to arrival in the current state.
OR
P[X t  A  X s 1 = x 1 , X s 2 = x 2 , ..., X s n = x n , X s = x] = P[X t  A  X s = x]
for all times s 1 < s 2 < ... < s n < s < t, all states x 1 , x 2 , ..., x n , x in S and all
subsets A of S.
%%--------------------------- 10Subject CT4 %%%%%%%%%%%%%%%%%%%%%%%%%%%%%%%%%%%%%%%%%%%%%%%%%%%%%%%%%%%%%%%%%%%%
(ii)
0.075
Repair-
ed
Never
Broken
0.2
0.025
(iii)
Discarded
d
P (t) = -0.1P NB (t)
dt NB
d
P (t) = 0.075P NB (t) - 0.2P R (t)
dt R
d
P D ( t ) ;\+\; 0.2 P R ( t ) \;=\; 0.025 P NB ( t )
dt
(iv)
d
P NB (t) = - 0.1P NB (t)
dt
d
[ln P NB (t)] = -0.1
dt
ln P NB (t) = -0.1t + const
P NB (0) = 1 so constant = 0
P NB (t) = exp(-0.1t)
d
P R (t) = 0.075 exp(-0.1t) - 0.2P R (t)
dt
 d

  dt P R ( t ) \;=\; 0.2 P R ( t )   exp(0.2t) = 0.075 exp(-0.1t) exp(0.2t)
d
[exp(0.2t) P R (t)] = 0.0750 exp(0.1t)
dt
%%--------------------------- 11Subject CT4 %%%%%%%%%%%%%%%%%%%%%%%%%%%%%%%%%%%%%%%%%%%%%%%%%%%%%%%%%%%%%%%%%%%%
exp(0.2t) P R (t) = 0.750 exp(0.1t) + const
P R (0) = 0 so constant = -0.75
P R (t) = 0.750 (exp(-0.1t) - exp(-0.2t))
(v)
This is P NB (t) + P R (t)
= 1.75 exp(-0.1t) - 0.75 exp(-0.2t)
Answers to parts (iii)-(v) of this question were very disappointing. Few
candidates worked out a constant of integration for P NB (t) (even though it was
equal to zero). Few candidates produced the correct transition diagram in
part (ii). Those candidates who wrote down incorrect transition diagrams
were given credit in parts (iii)-(v) for answers which correctly followed through
from the incorrect transition diagrams.
%%--------------------------- 20
\end{document}
