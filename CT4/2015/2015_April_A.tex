\documentclass[a4paper,12pt]{article}

%%%%%%%%%%%%%%%%%%%%%%%%%%%%%%%%%%%%%%%%%%%%%%%%%%%%%%%%%%%%%%%%%%%%%%%%%%%%%%%%%%%%%%%%%%%%%%%%%%%%%%%%%%%%%%%%%%%%%%%%%%%%%%%%%%%%%%%%%%%%%%%%%%%%%%%%%%%%%%%%%%%%%%%%%%%%%%%%%%%%%%%%%%%%%%%%%%%%%%%%%%%%%%%%%%%%%%%%%%%%%%%%%%%%%%%%%%%%%%%%%%%%%%%%%%%%

\usepackage{eurosym}
\usepackage{vmargin}
\usepackage{amsmath}
\usepackage{graphics}
\usepackage{epsfig}
\usepackage{enumerate}
\usepackage{multicol}
\usepackage{subfigure}
\usepackage{fancyhdr}
\usepackage{listings}
\usepackage{framed}
\usepackage{graphicx}
\usepackage{amsmath}
\usepackage{chngpage}

%\usepackage{bigints}
\usepackage{vmargin}

% left top textwidth textheight headheight

% headsep footheight footskip

\setmargins{2.0cm}{2.5cm}{16 cm}{22cm}{0.5cm}{0cm}{1cm}{1cm}

\renewcommand{\baselinestretch}{1.3}

\setcounter{MaxMatrixCols}{10}

\begin{document}
\begin{enumerate}

 Institute and Faculty of Actuaries1
2
For a simple random walk:
(i) Define the process.
[2]
(ii) Write down the nature of the state space and time space in which it operates.
[1]
(iii) Describe an example of a practical application of the process.
[1]
[Total 4]
The mortality of a rare form of flying beetle is being studied. It has been discovered
that beetles kept in a protected environment have a constant force of mortality, μ, but
that those in the wild have a force of mortality which is 50% higher. It has been
proven that the beetles revert immediately to the higher rate of mortality if they are
released from the protected environment.
A beetle born and always living in the wild has a 58% chance of living for eight days.
Calculate the probability of living the same length of time for:
3
4
(a) a beetle born and reared in the protected environment.
(b) a beetle born in the protected environment which is scheduled to be released
into the wild after six days.
[4]
(i) Explain what is meant by a proportional hazards model.
(ii) Outline three reasons why the Cox proportional hazards model is widely used
in empirical work.
[3]
[Total 6]
(i) List the stages you would go through in creating a model.
(ii) Discuss, for three of these stages, the specific issues that could arise when
creating a model to price a new sickness benefit product.
[3]
[Total 7]
CT4 A2015–2
[3]
[4]

%%%%%%%%%%%%%%%%%%%%%%%%%%%%%%%%%%%%%%%%%%%
1
(i)
This is defined as X n = Y 1 + Y 2 +... Y n
where the random variables Y j (the steps of the walk) are mutually
independent with the common probability distribution:
Pr[Y j = 1] = p,
Pr[Y j = 1] = 1  p.
(ii) It operates in discrete time with a discrete state space.
(iii) Any reasonable practical application
e.g. cumulative results of the Oxford vs Cambridge boat race (net lead
of Cambridge over Oxford) measured annually.
OR how much a gambler has won or lost if he wins or loses
£1 on every bet.
OR number of cars in a car park controlled by a single entry/exit
barrier measured after each time the barrier goes up.
Most candidates answered parts (i) and (ii) well, though many missed the point about the Y t
being independent. Some of the examples in part (iii) were rather contrived.
2
(a)
A beetle in the wild has a force of mortality equal to 3μ/2.
So for a beetle in the wild
we have S (8)  exp[  8(3  / 2)]  exp(  12  )  0.58
Hence
12μ = 0.5447
μ = 0.0454.
Therefore a beetle reared in the protected environment will have an 8 day
survival probability of
exp(8*0.0454) = 0.6955.
Page 3 %%%%%%%%%%%%%%%%%%%%%%%%%%%%%%%%%%%%%%%%%%%%%%%%%%%% – April 2015 – Examiners’ Report
(b)
A beetle in the protected environment has a probability of surviving 6 days
equal to
exp(6*0.0454) = 0.7616,
and a probability of surviving 2 days in the wild of
exp (2*1.5*0.0454) = 0.8727
Therefore this beetle’s probability of surviving 8 days is
0.7615 × 0.8727 = 0.6646.
This was the best answered question on the entire paper, with many candidates scoring full
marks. The most common error was to assume the rate in the wild was twice, rather than 1.5
times, the rate in the protected environment in part (a). If this was carried through correctly
into the rest of the answer then credit was given for subsequent calculations.
3
(i)
In a proportional hazards model the hazard of experiencing an event
may be factorised into two components:
one depending only on duration since some start event, which is known as the
baseline hazard, and the other depending only on a set of covariates and
associated parameters.
Thus the ratio between the hazards for any two individuals with different
values of the covariates is constant across all durations.
The baseline hazard applies to an individual with the value zero on all
covariates.
(ii)
The proportionality of the hazards makes estimating the impact of
covariates on the hazard straightforward (through partial likelihood).
Widely available statistical software packages have built-in routines for the
Cox model.
The Cox model is semi-parametric so the baseline hazard does not
need to be specified, and can be determined by the data (as with a Kaplan-
Meier hazard).
It ensures that the hazard is always positive.
It is easy to communicate.
There were some good attempts at this question. However, many candidates seemed to think
that the Cox model and the proportional hazards (PH) model were the same thing. In fact,
the Cox model is just one of a class of PH models. In part (i) we looked for knowledge of the
attractive characteristics of PH models in general, whereas in part (ii) we gave credit for
advantages of the Cox model in particular, as well as for general attributes of PH models
Page 4 %%%%%%%%%%%%%%%%%%%%%%%%%%%%%%%%%%%%%%%%%%%%%%%%%%%% – April 2015 – Examiners’ Report
which are useful in practice. In part (i) some candidates simply wrote down a formula for a
PH model. If all the terms were defined, partial credit was given for this.
4
(i)
Develop a well-defined set of objectives which need to be met by the
modelling process.
Plan the modelling process and how the model will be validated.
Collect and analyse the necessary data.
Define the parameters for the model and consider appropriate parameter
values.
Define the model initially by capturing the essence of the real world
system (refining the level of detail in the model can come at a later stage).
Involve experts on the real world system you are trying to imitate so as to get
feedback on the validity of the conceptual model.
Decide on whether a simulation package or a general purpose language is
appropriate for the implementation of the model.
Choose a statistically reliable random number generator that will perform
adequately in the context of the complexity of the model.
Write the computer program for the model.
Debug the program to make sure it performs the intended operations in the
model definition.
Test the reasonableness of the output of the model.
Review and carefully consider the appropriateness of the model in the light of
small changes in input parameters.
Analyse the output from the model.
Ensure that any relevant professional guidance has been complied with.
Communicate and document the results of the model.
(ii)
Objectives.
Is a single pricing table needed for a defined set of cover/deferred
period/definition of sickness?
Is a range of prices needed for different cover levels?
Is a simple price needed, or a confidence interval around profitability etc.?
Page 5 %%%%%%%%%%%%%%%%%%%%%%%%%%%%%%%%%%%%%%%%%%%%%%%%%%%% – April 2015 – Examiners’ Report
Data.
Look at national industry data and possibly international data on sickness and
recovery rates.
Take care as to the definitions of sickness.
Look at trends.
Is recovery rate affected by reduction in benefits?
Might future trends change due to economic influences, medical
developments, change in government policy to state benefits?
What might the likely level of expenses be? This will depend on expected
sales volumes (and commission levels).
It is a new product so perhaps the company has no experience of claim
monitoring. Look at industry data and in house data for new product launches.
Modelling process.
Start with a simple model reflecting sickness and recovery rates.
Build in expenses and reserving later.
Define the model.
Type of model will be determined by the objectives, deterministic for a simple
set of premium rates, stochastic if confidence intervals required.
Reasonableness of output.
Look at the premium rates generated and compare them with competitors in
the market.
Are they roughly where you would expect them to be?
Sensitivity to changes in input parameters.
Change each important parameter slightly; sickness rates, recovery rates,
expenses and ensure that the impact on the premium rates is not huge.
If it is, review the product design.
Can risk be reduced by introducing, say, annually reviewable premium rates.
Compliance with professional guidance and regulatory environment.
Look at professional guidance and legislative requirements, might
Solvency II impact the cost of capital to make the product uncompetitive or
uneconomic?
Part (i) was well answered by most candidates. In part (ii) other suggestions scored credit
provided they related to one of the stages identified in the answer to part (i) and dealt with a
model to price a new sickness benefit. No credit was given for comments which were not
specifically related to the pricing of a new sickness benefit product (many candidates made
general points which were applicable to any model). Note that the level of detail given above
is well in excess of that the Examiners required for full credit. Nevertheless, many
candidates gave responses which were too brief and vague to gain much credit for this part.
Page 6 %%%%%%%%%%%%%%%%%%%%%%%%%%%%%%%%%%%%%%%%%%%%%%%%%%%% – April 2015 – Examiners’ Report
