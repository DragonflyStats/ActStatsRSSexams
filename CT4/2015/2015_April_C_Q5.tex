\documentclass[a4paper,12pt]{article}

%%%%%%%%%%%%%%%%%%%%%%%%%%%%%%%%%%%%%%%%%%%%%%%%%%%%%%%%%%%%%%%%%%%%%%%%%%%%%%%%%%%%%%%%%%%%%%%%%%%%%%%%%%%%%%%%%%%%%%%%%%%%%%%%%%%%%%%%%%%%%%%%%%%%%%%%%%%%%%%%%%%%%%%%%%%%%%%%%%%%%%%%%%%%%%%%%%%%%%%%%%%%%%%%%%%%%%%%%%%%%%%%%%%%%%%%%%%%%%%%%%%%%%%%%%%%

\usepackage{eurosym}
\usepackage{vmargin}
\usepackage{amsmath}
\usepackage{graphics}
\usepackage{epsfig}
\usepackage{enumerate}
\usepackage{multicol}
\usepackage{subfigure}
\usepackage{fancyhdr}
\usepackage{listings}
\usepackage{framed}
\usepackage{graphicx}
\usepackage{amsmath}
\usepackage{chngpage}

%\usepackage{bigints}
\usepackage{vmargin}

% left top textwidth textheight headheight

% headsep footheight footskip

\setmargins{2.0cm}{2.5cm}{16 cm}{22cm}{0.5cm}{0cm}{1cm}{1cm}

\renewcommand{\baselinestretch}{1.3}

\setcounter{MaxMatrixCols}{10}

\begin{document}
\begin{enumerate}
\item 5

(i)
State the principle of correspondence as it applies to death rates.

\medksip
A nightclub opens at 10.00 p.m. and closes at 2.00 a.m. It admits only people aged
over 21 years on the production of an identity card giving date of birth.
The table below shows the number of people entering in various intervals between
10.00 p.m. and 2.00 a.m. on 30 June 2013. No-one was admitted after 1.00 a.m., and
you may assume that all those who enter the premises stay until 2.00 a.m.
\begin{verbatim}
Year of
birth 10.00–11.30
p.m. 11.30–12.00
p.m. 12.00 p.m.–1.00 a.m.
1989
1990
1991
1992 100
200
150
100 300
400
400
250 200
350
300
200
\end{verbatim}

During the period of opening, 40 people aged 22 last birthday required medical
attention for heat exhaustion.
\item (ii)
6
Calculate the rate per person-hour at which those attending the night club aged
22 last birthday required medical attention for heat exhaustion, stating any
assumptions you make.
\end{enumerate}
%%%%%%%%%%%%%%%%%%%%%%%%%%%%%%%%%%%%%%%%%%
\newpage
5
\begin{itemize}
    \item (i) A life alive at age x at time t should be included in the
exposed-to-risk if and only if, were that life to die immediately,
his or her death would be included in the deaths at age x, d x .
\item (ii) Those aged 22 last birthday on 30 June 2013 were born between 1 July 1990
and 30 June 1991, so half of them were born in 1990 and half in 1991.
\item Assuming that birthdays are evenly distributed across calendar years,
The number of persons aged 22 last birthday entering during each period is
10.00 – 11.30 p.m.
11.30 p.m. – 12.00 midnight
12.00 midnight – 1.00 a.m.
0.5(200 + 150) = 175
0.5(400 + 400) = 400
0.5(350 + 300) = 325
\item THEN EITHER
The number of persons aged 22 last birthday in the nightclub at 10.00 p.m.,
11.30 p.m., 12.00 midnight, 1.00 a.m. and 2.00 a.m. is therefore
10.00 p.m.
11.30 p.m.
12 midnight
1.00 a.m.
2.00 a.m.
0
175
575
900
900
\item Using the census approximation and assuming that arrivals are evenly
distributed across time,
the exposed to risk in person-hours is
 0  175 
 175  575   575  900 
1.5 
  0.5 
  1 
  900
2
2
 2 

 

 131.25  187.5  737.5  900
= 1,956.25
OR
\item Using the census approximation and assuming that arrivals are evenly
distributed across time,
the exposed-to-risk in person-hours is
175(3.25) + 400(2.25) + 325(1.5) = 1,956.25.
Page 7 %%%%%%%%%%%%%%%%%%%%%%%%%%%%%%%%%%%%%%%%%%%%%%%%%%%% – April 2015 – Examiners’ Report
AND HENCE
the rate of requiring medical attention is
40
 0.02045 per person hour.
1,956.25
\item 
Common errors were to use the wrong year or years of birth, to fail to cumulate the arrivals
(i.e. to realise that once inside the building, customers remained until 2.00 a.m.), and to
forget the final hour, during which the club was full and no more customers entered. In
general, answers to this question were rather better than answers to similar questions on
other recent examination papers. 
\item In part (ii) candidates were expected to relate the
assumptions to the specific stage of the derivation to which they applied. Candidates who
wrote down lists of assumptions – some relevant to the answer, others not – scored little
credit.
\end{itemize}
\end{document}
