[Total 7]
PLEASE TURN OVER5
(i)
Describe why models are used in actuarial work.

The following diagrams illustrate sample paths for four stochastic processes.
Sample Path A
Sample Path B
Time
Time
Sample Path D
Sample Path C
Time
(ii)
Time
Identify which sample path is most likely to correspond to a:




discrete time, discrete state process.
continuous time, discrete state process.
discrete time, continuous state process.
continuous time, continuous state process.

(iii)
CT4 S2015–4
Discuss the reasons why a discrete or continuous process may be selected for a
modelling exercise.

[Total 9]

%%%%%%%%%%%%%%%%%%%%%%%%%%%%%%%%%%%%%%%%%%%%%%%%%%%%%%%%%%%%%%%%%%%%
Q5
(i)
A model is an imitation of a real world system or process.
Different future policies or possible actions can be compared to see which best
suits the requirements of a user.
We can examine different scenarios without carrying them out in practice, or
to avoid potential costs or risks associated with trialling in real life.
Parameters can be sensitivity tested using a model so the effect of changing
certain input parameters can be studied before a decision is made to implement
the plans in the real world.
A model allows systems with long time-frames to be analysed in compressed
time.
Models also allow complex processes involving stochastic elements to be
investigated.
%%--------------------------- 6Subject CT4 %%%%%%%%%%%%%%%%%%%%%%%%%%%%%%%%%%%%%%%%%%%%%%%%%%%%%%%%%%%%%%%%%%%%
Models can be developed for many activities (e.g. the economy of a country,
future cash flows of a broker distribution channel).
(ii) Sample path C – discrete time, discrete state process
Sample path A – continuous time, discrete state process
Sample path B – discrete time, continuous state process
Sample path D – continuous time, continuous state process
(iii) The objectives of the study.
The real world process may only be able to change in a discrete fashion.
Outputs may only be required at discrete points.
To simulate a process may need to discretise the process (for example with
Monte Carlo simulation)
It may be easier to model certain situations with a probability density function,
which is therefore continuous.
Data may only be available at discrete points, for example the position at the
end of each day.
There may be other modelling constraints for example it may be that a limited number of states can be used so a discrete model would be preferred.
Answers to part (ii) were disappointing. Few candidates realised that in sample paths A and C the process could only be in one of a finite number of
states (hence discrete state), and that in sample paths B and C the observations were equally spaced on the time axis (hence discrete time). In
part (iii) other sensible comments were given credit. Unfortunately, many candidates simply gave examples of cases where a discrete or a continuous
process would be most appropriate without explaining the reasons why this would be the case. Such candidates were given only limited credit.
