 Institute and Faculty of Actuaries1
List four factors, other than age and sex, by which mortality statistics are often
subdivided. 
2 Describe the differences between a stochastic and a deterministic model. 


%%%%%%%%%%%%%%%%%%%%%%%%%%%%%%%%%%%%%%%%%%%%%%%%%%%%%%%%%%%%%%%%%%%%%%%%%%%%%5

Solutions
Q1
Type of policy (which often reflects the reason for insuring)
Smoker/non-smoker status
Level of underwriting
Duration in force
Sales channel
Policy size
Occupation of policyholder
Known impairments
Postcode/geographical region
Marital status
Most candidates scored full marks on this question.
Q2
A stochastic model is one which recognises the random nature of the input
components, whereas a deterministic model does not contain any random
components.
Running a stochastic model many times will produce a distribution of results for
possible scenarios, whereas a deterministic model will produce results for a single
scenario.
Thus a deterministic model can be seen as a special case of a stochastic model.
In a stochastic model the output of each run is one value from a distribution.
%%--------------------------- 3Subject CT4 %%%%%%%%%%%%%%%%%%%%%%%%%%%%%%%%%%%%%%%%%%%%%%%%%%%%%%%%%%%%%%%%%%%%
By contrast, in a deterministic model, the output is determined once the set of fixed
inputs and the relationships between them have been defined.
For many stochastic models, it is necessary to use numerical approximations in order
to integrate functions or solve differential equations.
The results for a deterministic model can often be obtained by direct calculations.
In a stochastic model, several independent runs are required for each set of inputs so
that statistical theory can be used to help study the implications of a set of inputs.
A deterministic model only requires one run.
Full marks could be obtained for rather less than is written in this solution.
