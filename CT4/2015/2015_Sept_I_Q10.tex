
[Total 11]10
A profession has examination papers in two subjects, A and B, each of which is
marked by a team of examiners. After each examination session, examiners are given
the choice of remaining on the same team, switching to the other team, or taking a
session’s holiday.
In recent sessions, 10% of subject A’s examiners have elected to switch to subject B
and 10% to take a holiday. Subject B is more onerous to mark than subject A, and in
recent sessions, 20% of subject B’s examiners have elected to take a holiday in the
next session, with 20% moving to subject A.
After a session’s holiday, the profession allocates examiners equally between subjects
A and B. No examiner is permitted to take holiday for two consecutive sessions.
(i) Sketch the transition graph for the process. 
(ii) Determine the transition matrix for this process. 
(iii) Calculate the proportion of the profession’s examiners marking for subjects A
and B in the long run.

The profession considers that in future, an equal number of examiners is likely to be
required for each subject. It proposes to try to ensure this by adjusting the proportion
of those examiners on holiday who, when they return to marking, are allocated to
subjects A and B.
(iv)
CT4 S2015–9
Calculate the proportion of examiners who, on returning from holiday, should
be allocated to subject B in order to have an equal number of examiners on
each subject in the long run.

[Total 12]
%%%%%%%%%%%%%%%%%%%%%%%%%%%%%%%%%%%%%%%%%%%%%%%%%%%%%%%%%%%%%%%%%%%%%%%%%%%%%%%%%%%%%%%
Q10
(i)
Marking Subject A
Marking Subject B
On holiday H
(ii)
Subject A  0.8 0.1 0.1 
Subject B   0.2 0.6 0.2   .
Holiday   0.5 0.5 0  
(iii)
We have P = .
The stationary distribution of examiners can be found as the solution of the set
of equations
 A = 0.8 A + 0.2 B + 0.5 H (1)
 B = 0.1 A + 0.6 B + 0.5 H (2)
 H = 0.1 A + 0.2 B
(3)
(1) gives
0.2 A = 0.2 B + 0.5 H
0.4 A = 0.4 B +  H
0.4 B = 0.4 A -  H
(2) gives
0.4 B = 0.1 A + 0.5 H
%%--------------------------- 14Subject CT4 %%%%%%%%%%%%%%%%%%%%%%%%%%%%%%%%%%%%%%%%%%%%%%%%%%%%%%%%%%%%%%%%%%%%
so
0.4 A -  H = 0.1 A + 0.5 H
0.3 A = 1.5 H
 A = 5 H
In (2) this gives
 B = 0.5 H + 0.6 B + 0.5 H
0.4 B =  H
 B = 2.5 H
So, since  A +  B +  H = 1,
the stationary distribution is {5 H , 2.5 H ,  H }
and hence
10
17
5
 B =
17
2
 H =
.
17
 A =
So in the long run 58.8% of examiners are marking subject A and 29.4% are
marking subject B.
(iv)
Let the new transition probability from H to A be x, and that from H to B be
1 - x.
The proportion we require is thus just x. The new transition matrix is
Subject A  0.8 0.1 0.1 
Subject B   0.2 0.6 0.2   .
Holiday   x 1 - x 0  
The stationary probability distribution is given by the three equations
\pi_A ;\+\; 0.8 \pi_A \;=\; 0.2 \pi_B \;=\; x \pi_H
\pi_B ;\+\; 0.1 \pi_A \;=\; 0.6 \pi_B \;=\; (1 - x ) \pi_H (1)
\pi_H ;\+\; 0.1 \pi_A \;=\; 0.2 \pi_B (3)
(2)
%%--------------------------- 15Subject CT4 %%%%%%%%%%%%%%%%%%%%%%%%%%%%%%%%%%%%%%%%%%%%%%%%%%%%%%%%%%%%%%%%%%%%
We also have \pi_A ;\+\; \pi_B .
EITHER
From (3) \pi_H ;\+\; 0.3 \pi_B ;\+\; 0.3 \pi_A
 10 10 3 
Therefore the new stationary probability distribution is  , ,  .
 23 23 23 
In (1) we have
10 = 8 + 2 + 3x
Hence x = 0.
OR
If \pi_A = \pi_B then in (1)
\pi_A ;\+\; 0.8 \pi_A \;=\; 0.2 \pi_B \;=\; x \pi_H ;\+\; 0.8 \pi_A \;=\; 0.2 \pi_A \;=\; x \pi_H ;\+\; \pi_A \;=\; x \pi_H
Hence x = 0.
AND HENCE
All those returning from holiday will have to be allocated to subject B.
There were a gratifying number of completely correct answers to this
question, and many candidates scored full marks on parts (i)–(iii). Part (iv)
was more demanding, and required candidates to invert the usual question
and establish what transition matrix would give rise to a particular stationary
distribution.
