\documentclass[a4paper,12pt]{article}

%%%%%%%%%%%%%%%%%%%%%%%%%%%%%%%%%%%%%%%%%%%%%%%%%%%%%%%%%%%%%%%%%%%%%%%%%%%%%%%%%%%%%%%%%%%%%%%%%%%%%%%%%%%%%%%%%%%%%%%%%%%%%%%%%%%%%%%%%%%%%%%%%%%%%%%%%%%%%%%%%%%%%%%%%%%%%%%%%%%%%%%%%%%%%%%%%%%%%%%%%%%%%%%%%%%%%%%%%%%%%%%%%%%%%%%%%%%%%%%%%%%%%%%%%%%%

\usepackage{eurosym}
\usepackage{vmargin}
\usepackage{amsmath}
\usepackage{graphics}
\usepackage{epsfig}
\usepackage{enumerate}
\usepackage{multicol}
\usepackage{subfigure}
\usepackage{fancyhdr}
\usepackage{listings}
\usepackage{framed}
\usepackage{graphicx}
\usepackage{amsmath}
\usepackage{chngpage}

%\usepackage{bigints}
\usepackage{vmargin}

% left top textwidth textheight headheight

% headsep footheight footskip

\setmargins{2.0cm}{2.5cm}{16 cm}{22cm}{0.5cm}{0cm}{1cm}{1cm}

\renewcommand{\baselinestretch}{1.3}

\setcounter{MaxMatrixCols}{10}

\begin{document}A health insurance company has collected data on sickness rates during the calendar
year 2013 among a sample of its policyholders aged 40–64 years inclusive. It
compares these to the rates among its policyholders of the same age in 2012. It finds
that at ages 40–50 years inclusive, and at ages 56–61 years inclusive, the sickness
rates in 2013 are higher than those in 2012. At other ages, the sickness rates in 2013
were lower than those in 2012.
\begin{enumerate}
\item (i) Carry out two tests of the null hypothesis that the underlying sickness rates in
2013 are the same as those in 2012.
\item  (ii) Comment on the implications of the results of your test for the company’s
sickness insurance business.
\end{enumerate}

\newpage
%%%%%%%%%%%%%%
6
\begin{itemize}
\item (i)
Signs Test
Under the null hypothesis, the number of positive deviations (2013 higher than
2012) is distributed Binomial (25, 0.5).
\item We have 17 positive deviations
\item ALTERNATIVE 1: NORMAL APPROXIMATION
As the number of ages is large enough, we can use the normal approximation,
 25 25 
in which the number of positive deviations is distributed Normal  ,  .
 2 4 
\item THEN EITHER
 17  12.5 
A z-score for 17 positive deviations is 
  1.8
 6.25 
OR
with a continuity correction a z-score for 17 positive deviations is
 16.5  12.5 

  1.6.
6.25 

AND HENCE
\item Since 1.8 (or 1.6) < 1.96 (2-tailed test)
we do not have sufficient evidence to reject the null hypothesis.
Page 8 %%%%%%%%%%%%%%%%%%%%%%%%%%%%%%%%%%%%%%%%%%%%%%%%%%%% – April 2015 – Examiners’ Report
\item ALTERNATIVE 2: EXACT TEST
 25 
Pr[exactly 17 positive signs] =   0.5 25  0.0322 .
 17 
\item 
Since 0.0322 > 0.025 (2-tailed test)
we do not have sufficient evidence to reject the null hypothesis.
Grouping of Signs Test
\item We have 25 age groups, 17 positive signs, and 2 positive groups
ALTERNATIVE 1: NORMAL APPROXIMATION
Using the Normal approximation (as we have more than 20 ages),
 17(8  1) (17 *8) 2 
the number of positive groups is distributed Normal 
,
 ,
(25) 3 
 25
which is Normal(6.12, 1.18).
 2  6.12 
\item We therefore compute a z-score for 2 runs as 
   3.79 .
 1.18 
\item Since Pr (z < 3.79) << 0.05 (one-tailed test) (or -3.79 < -1.645),
we reject the null hypothesis.
\item ALTERNATIVE 2: EXACT CALCULATION
Probability of getting 2 or fewer positive groups is
 16   9   16   9 
     
9
576
 0  1    1  2  

 0.000541
1, 081,575 1, 081,575
 25 
 25 
 
 
 17 
 17 
\item Since this is less than 0.05
we reject the null hypothesis.
item ALTERNATIVE 3: USING THE TABLE IN THE “GOLD BOOK”
Using the table on p. 189 of the “Gold Book”, with n 1 = 17, n 2 = 8
the table shows that we reject the null hypothesis with 3 or fewer runs of
positive signs.
Page 9 %%%%%%%%%%%%%%%%%%%%%%%%%%%%%%%%%%%%%%%%%%%%%%%%%%%% – April 2015 – Examiners’ Report
Since we only have 2 positive runs and 2 < 3 we reject the null hypothesis.
\item (ii)
The results of the Signs Test suggest that the underlying rates in 2013 are not
systematically higher or lower than those in 2012.
The null hypothesis was rejected by the Grouping of Signs Test implying
that the shape of the distribution of sickness rates in 2013 is different from
that in 2012.
\item However this is only one year’s data and the company might wait to see if a
trend develops, or investigate whether there was a specific factor operating in
2012 or 2013 which caused the change.
If the shape of sickness rates makes them markedly different in 2013 from
2012 at ages where much business is sold, this will have implications for
profitability and pricing.
\item In part (i) most candidates used the Normal approximation for the Signs Test. If the exact
version was used, it is not necessary to compute Pr[17 or more positive signs] as Pr[exactly
17 positive signs] > 0.025. The correct value for Pr[17 or more positive signs] is 0.0538.

\item Answers to part (ii) were poor. Full credit was given for summarising the immediate
implications of the results obtained in part (i) for the comparison of the rates in 2013 and
2012 and for making some comment about the potential financial implications for the
company. Thus full credit could be obtained for less than is given in the model solution
above. Comments in part (ii) that were consistent with the actual results obtained by the
candidate in part (i) were given credit, even if the tests in part (i) were performed incorrectly
and reached conclusions different from those above.
\end{itemize}
\end{document}
