\documentclass[a4paper,12pt]{article}

%%%%%%%%%%%%%%%%%%%%%%%%%%%%%%%%%%%%%%%%%%%%%%%%%%%%%%%%%%%%%%%%%%%%%%%%%%%%%%%%%%%%%%%%%%%%%%%%%%%%%%%%%%%%%%%%%%%%%%%%%%%%%%%%%%%%%%%%%%%%%%%%%%%%%%%%%%%%%%%%%%%%%%%%%%%%%%%%%%%%%%%%%%%%%%%%%%%%%%%%%%%%%%%%%%%%%%%%%%%%%%%%%%%%%%%%%%%%%%%%%%%%%%%%%%%%

\usepackage{eurosym}
\usepackage{vmargin}
\usepackage{amsmath}
\usepackage{graphics}
\usepackage{epsfig}
\usepackage{enumerate}
\usepackage{multicol}
\usepackage{subfigure}
\usepackage{fancyhdr}
\usepackage{listings}
\usepackage{framed}
\usepackage{graphicx}
\usepackage{amsmath}
\usepackage{chngpage}

%\usepackage{bigints}
\usepackage{vmargin}

% left top textwidth textheight headheight

% headsep footheight footskip

\setmargins{2.0cm}{2.5cm}{16 cm}{22cm}{0.5cm}{0cm}{1cm}{1cm}

\renewcommand{\baselinestretch}{1.3}

\setcounter{MaxMatrixCols}{10}

\begin{document}
\begin{enumerate}

\item (i)
Describe what is meant by a Markov chain.
[2]
\item A simplified model of the internet consists of the following websites with links between the websites as shown in the diagram below.

N(ile) B(anana)
C(heep) H(andbook)

An internet user is assumed to browse by randomly clicking any of the links on the website he is on with equal probability.
\item 
(ii)
8

Calculate the transition matrix for the Markov chain representing which
website the internet user is on.
\item 
(iii) Calculate, of the total number of visits, what proportion are made to each
website in the long term.
\end{enumerate}

%%%%%%%%%%%%%%%%%%%%%%%%%%%%%%
7
\begin{itemize}
\item (i)
A Markov chain is a stochastic process with discrete states operating in
discrete time in which
EITHER
P[X t  A  X s 1 = x 1 , X s 2 = x 2 , ..., X s n = x n , X s = x] = P[X t  A  X s = x]
for all times s 1 < s 2 < ... < s n < s < t, all states x 1 , x 2 , ..., x n , x in S and all
subsets A of S
OR
the probabilities of moving from one state to another depend only on the
present state of the process: the history of the process before the current state
is irrelevant.
%%--- Page 10 %%%%%%%%%%%%%%%%%%%%%%%%%%%%%%%%%%%%%%%%%%%%%%%%%%%% – April 2015 – Examiners’ Report
item (ii)
This is based on the number of links to the site and where they go.
\[P = \bordermatrix{

  & N & B & C H H \cr
N  & 0 & 1/2 & 1/2 & 0 \cr
B  & 1/2 & 0 &1/2 0 \cr
C  & 1/3 & 1/3 0 1/3 \cr
H  & 0 & 0 & 1 & 0 \cr
}\]

%%%%%%%%%%%%%%%%%%%%%%%%%%%%%%%%%%%%%%%%%%%%%%%%%%%%%%%5
\item Stationary distribution satisfies $\pi \;=\; \pi P$

\begin{enumerate}
\item ${\displaystyle
\frac{1}{2} \pi_B + \frac{1}{3}\pi_C \;=\; \pi_N }$

\item ${\displaystyle
\frac{1}{2} \pi_N + \frac{1}{3}\pi_C \;=\; \pi_B }$

\item ${\displaystyle
\frac{1}{3} \pi_N + \frac{1}{2} \pi_B + \pi_H \;=\; \pi_C }$
\item ${\displaystyle
\frac{1}{3} \pi_C \;=\; \pi_H }$
\end{enumerate}

\[ \pi_N + \pi_B + \pi_H + \pi_C = 1\]


%%%%%%%%%%%%%%%%%%%%%%%%%%%%%%%%%%%%%%%%%%%%%%%%%%%%%5
\item From (1) and (2)
 N   B
From (3) and (4)
 N   B 
2
 C
3
1 
 2 2
So    1    C  1
3 
 3 3

%%--- Page 11 
%%%%%%%%%%%%%%%%%%%%%%%%%%%%%%%%%%%%%%%%%%%%%%%%%%%% – April 2015 – Examiners’ Report

\item Hence
 N 
1
1
3
1
,  B  ,  C  ,  H 
4
4
8
8

%% Both parts of this question were well answered by most candidates. In the final answer to part (ii), it was important for candidates to indicate which probability applied to which state, rather than just listing four numbers.

\end{itemize}
\end{document}
