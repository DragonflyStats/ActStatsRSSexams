\documentclass[a4paper,12pt]{article}

%%%%%%%%%%%%%%%%%%%%%%%%%%%%%%%%%%%%%%%%%%%%%%%%%%%%%%%%%%%%%%%%%%%%%%%%%%%%%%%%%%%%%%%%%%%%%%%%%%%%%%%%%%%%%%%%%%%%%%%%%%%%%%%%%%%%%%%%%%%%%%%%%%%%%%%%%%%%%%%%%%%%%%%%%%%%%%%%%%%%%%%%%%%%%%%%%%%%%%%%%%%%%%%%%%%%%%%%%%%%%%%%%%%%%%%%%%%%%%%%%%%%%%%%%%%%

\usepackage{eurosym}
\usepackage{vmargin}
\usepackage{amsmath}
\usepackage{graphics}
\usepackage{epsfig}
\usepackage{enumerate}
\usepackage{multicol}
\usepackage{subfigure}
\usepackage{fancyhdr}
\usepackage{listings}
\usepackage{framed}
\usepackage{graphicx}
\usepackage{amsmath}
\usepackage{chngpage}

%\usepackage{bigints}
\usepackage{vmargin}

% left top textwidth textheight headheight

% headsep footheight footskip

\setmargins{2.0cm}{2.5cm}{16 cm}{22cm}{0.5cm}{0cm}{1cm}{1cm}

\renewcommand{\baselinestretch}{1.3}

\setcounter{MaxMatrixCols}{10}

\begin{document}
\begin{enumerate}
In a computer game a player starts with three lives. Events in the game which cause
the player to lose a life occur with a probability  dt  o ( dt ) in a small time interval dt.
However the player can also find extra lives. The probability of finding an extra life
in a small time interval dt is  dt  o ( dt ). The game ends when a player runs out of
lives.
(i) Outline the state space for the process which describes the number of lives a
player has.
[1]
(ii) Draw a transition graph for the process, including the relevant transition rates.
[3]
(iii) Determine the generator matrix for the process.
[2]
(iv) Explain what is meant by a Markov jump chain.
[1]
(v) Determine the transition matrix for the jump chain associated with the process.
[2]
(vi) Determine the probability that a game ends without the player finding an extra
life.
[1]
[Total 10]
CT4 A2015–5

%%%%%%%%%%%%%%%%%%%%%%%%%%%%%%%%%%%%%%%%%%%%%%%%%%%%%
PLEASE TURN OVER11
A new disease has been discovered which is transmitted by an airborne virus.
Anyone who contracts the disease suffers a high fever and then in 60% of cases dies
within an hour and in 40% of cases recovers. Having suffered from the disease once,
a person builds up antibodies to the disease and thereafter is immune.
(i) Draw a multiple state diagram illustrating the process, labelling the states and
possible transitions between states.
[2]
(ii) Express the likelihood of the process in terms of the transition intensities and
other observable quantities, defining all the terms you use
[4]
(iii) Derive the maximum likelihood estimator of the rate of first time sickness.
[2]
Three years ago medical students visited the island where the disease was first
discovered and found that of the population of 2,500 people, 860 had suffered from
the disease but recovered. They asked the leaders of the island to keep records of the
occurrence and the outcome of each incidence of the disease. The students intended
to return exactly three years later to collect the information.

%%%%%%%%%%%%%%%%%%%%%%%%%%%%%%%%%%%%%%%%%%%%%5
i)
10
{0,1,2,3,4....}
(ii)

0

1

(iii)

2

3


4


Generator matrix
Lives
0
1
2
3
4
0
0
0
0
 0


0
0
   (    )
 0

 (    )

0


 (    )

0
 0
 0

 (    )
0
0
 
 .
(iv)
...
. 







.  
EITHER
If a Markov jump process X t is examined only at the times of transition, the
resulting process is called the jump chain associated with X t .
OR
A jump chain is each distinct state visited in the order visited where the time
set is the times when states are moved between.
(v)
Lives
0
1
2
3
4
...
1
0
0
0
0
etc. 



0
 / (    )
0
0
  / (    )



0
 / (    )
0
 / (    )
0


 / (    )
 / (    )
0
0
0




 / (    )
0
0
0
0
 
 
etc.


Page 15 %%%%%%%%%%%%%%%%%%%%%%%%%%%%%%%%%%%%%%%%%%%%%%%%%%%% – April 2015 – Examiners’ Report
3
(vi)
  


  
Many candidates scored respectably on this question. The most common error was to define
the state space as {0, 1, 2, 3}, ignoring the possibility that the player could, at any time, have
found more extra lives than (s)he has lost. Candidates who used this state space were
penalised in parts (i) and (ii) but could score full credit for later parts by carrying through
their answer correctly. The other commonly occurring errors were to allow a transition out
of state 0 (this is not possible, as the game ends when the player has no lives left), or to
ignore the absorbing state 0 completely. Again, these errors were penalised in parts (i) and (ii) but credit was given for correctly following through into later parts. Candidates who
ignored state 0 completely were unable to give a sensible answer to part (vi). In part (iv) a
disappointing number of candidates simply provided a general definition of a Markov chain (i.e. repeating the answer to Question 7, part (i)), rather than relating the Markov jump chain
to a Markov jump process.
11
ALTERNATIVE 1
(i)
0.4ρ
Healthy
(H)
Recovered and
immune
(R)
0.6ρ
Dead from other
reason (O)
Dead from disease
(D)
(ii)
The likelihood is
 
L  exp  0.6   0.4   
 0.4  
d HR
     
HO
d HO
RO
HO
   exp        0.6  
H
RO
R
d RO
where
the four states are H – healthy, R – recovered, D – dead from
the disease and O – dead for some other reason, and
 I is the waiting time in state I (I = H, R)
Page 16
d HD %%%%%%%%%%%%%%%%%%%%%%%%%%%%%%%%%%%%%%%%%%%%%%%%%%%% – April 2015 – Examiners’ Report
d IJ is the number of transitions from state I to state J (J = R, D, O)
and  IJ is the intensity of the transition from state I to state J and ρ is
the rate of first time sickness
(iii)
Taking logarithms of the likelihood we have:
ln L    0.6   0.4    H  d HS ln  0.6    d HR ln  0.4   plus terms not
dependent on 
Differentiating with respect to  gives:
d  ln L 
d 
0.6 d HS 0.4 d HR
  

0.6 
0.4 
H
and setting this to zero gives a maximum likelihood estimate of 
 ˆ 
d HD  d HR
 H
This is a maximum as the second derivative
d 2  ln L 
 d   2

d HD  d HR
   2
must
be negative.
(iv)
We want a person in H at t = 0 not dead at t = 3.
So they can either be healthy throughout:
HH
t p 0
  3
 
  3
 
HO
 exp    0.4   0.6    du   exp       HO du 
  0
 
  0
 




or go H to R and stay there which is the integral between
0 and 3 of the product of
 
 
HO
u
surviving healthy for a period u i.e. exp  
getting sick and recovering at time u i.e. 0.4ρdu
  

RO
 3  u     ,
and staying recovered i.e. exp    

where we ignore the short time spent in the sick state.
3
 
 
  

So altogether exp      HO u  0.4  exp    RO  3  u   du .
 
 
 
 

0
Page 17 %%%%%%%%%%%%%%%%%%%%%%%%%%%%%%%%%%%%%%%%%%%%%%%%%%%% – April 2015 – Examiners’ Report
(v)
We need enough information to calculate the number of transitions from
Healthy to Sick and the waiting time in the “Healthy” state. Hence we will
need:
Date of birth of all births in last three years.
Date of death of all “Healthy” deaths in last three years.
Date of any other immigration or emigration of Healthy people in last three
years.
Date at which each person who contracted the disease fell ill
ALTERNATIVE 2
(i)
Suffering
from
disease
(S)
Healthy
(H)
Recovered and
immune
(R)
Dead
(D)
(ii)
The likelihood is
 
       
d
d
d
d
HD
RD
SR
SD




       
   
L  exp  HS   HD  H exp  RD  R exp  SR   SD  S  HS
HD
RD
SR
SD
where
 I is the waiting time in state I,
d IJ is the number of transitions from state I to state J,
and  IJ is the intensity of the transition from state I to state J.
The four states are H – healthy, S – suffering from disease,
R – recovered, D – dead.
Page 18
d HS %%%%%%%%%%%%%%%%%%%%%%%%%%%%%%%%%%%%%%%%%%%%%%%%%%%% – April 2015 – Examiners’ Report
(iii)
Taking logarithms of the likelihood we have:
 
ln L   HS  H  d HS ln  HS
plus terms not dependant on  HS
Differentiating with respect to  HS gives:
d  ln L 
d  HS
  H 
d HS
 HS
and setting this to zero gives a maximum likelihood estimate of  HS
 ˆ
HS

d HS
 H
This is a maximum as the second derivative
d 2  ln L 
 d  
HS
2

d HS
  
HS
2
must be
negative.
(iv)
We want a person in H at t = 0 neither suffering from the
disease nor dead at t = 3.
So they can either be healthy throughout:
HH
t p 0
  3
 
HS
HD
 exp       du 
  0
 


or to go H to S to R and stay there which is the integral between
0 and 3 of the product of:
 
 
HS
HD
u
surviving healthy for a period u i.e. exp   
getting sick at time u i.e.  HS du and recovering, i.e. 0.4
  
    ,
RD
 3  u 
and staying recovered i.e. exp    

where we ignore the short time spent in the sick state.
3
 
 
  

So altogether exp    HS   HD u   HS 0.4 exp    RD  3  u   du .
 
 
 
 

0
Page 19 %%%%%%%%%%%%%%%%%%%%%%%%%%%%%%%%%%%%%%%%%%%%%%%%%%%% – April 2015 – Examiners’ Report
(v)
We need enough information to calculate the number of transitions from
Healthy to Sick and the waiting time in the “Healthy” state. Hence we will
need:
Date of birth of all births in last three years.
Date of death of all “Healthy” deaths in last three years.
Date of any other immigration or emigration of Healthy people in last three
years.
Date at which each person who contracted the disease fell ill.
Various alternative (usually simpler) models were suggested by some candidates. For
example a much simpler model was proposed with only three states:
Healthy
0.4ρ
Recovered
from
disease
0.6ρ
Dead
from
disease
This was treated sympathetically for parts (i) to (iii) as the rate of first-time sickness can be
derived from this model. However in part (iv) the probability of remaining alive for three
years depends on the rate of death from causes other than the disease, even among islanders
who have never suffered from the disease, so this needs to be introduced. Moreover, the rate
of death from causes other than the disease may vary for persons who have never had the
disease and persons who have recovered from the disease.
Few candidates attempted part (iv). Candidates were expected to calculate the probability of
being alive in three years’ time for a person who had never had the disease at the start of that
three-year period, not the probability that a person who had not had the disease would be
alive and still not have had the disease in three years’ time. In part (iv) some candidates
assumed constant transition intensities and evaluate the integral, which was given full credit.
In part (v) many candidates stated that what is required is the number of healthy
persons at the beginning of the period (three years ago) and the current number of healthy
persons, together with the number of persons who had fallen sick. This would allow an
approximate sickness rate to be calculate using various assumptions and was given partial
credit.
Page 20 %%%%%%%%%%%%%%%%%%%%%%%%%%%%%%%%%%%%%%%%%%%%%%%%%%%% – April 2015 – Examiners’ Report
