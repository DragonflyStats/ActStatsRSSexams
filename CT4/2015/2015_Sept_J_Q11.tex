PLEASE TURN OVER11
(i)
Describe why an insurance company might want to compare the results of a
mortality investigation with previous experience.

A large life insurance company has undertaken an investigation of the mortality of its
policyholders. Currently it assumes that mortality at age x, \mu x , is equal to a standard
table. The company wishes to use the results from the investigation to see whether
the standard table is still appropriate. Below are shown some data from the
investigation.
Age x Number of
policies in force Actual death
claims Expected death
claims from
standard table
70
71
72
73
74
75
76
77
78
79 1,000
1,200
1,100
1,100
1,000
1,000
950
900
850
800 13
28
31
34
39
41
41
40
46
48 23.74
31.80
32.50
36.20
36.63
40.73
42.99
45.20
47.34
49.35
(ii) Perform an overall test of the hypothesis that the underlying mortality of the
company’s policyholders is, over this range of ages, represented by the
standard table.
[6]
(iii) Evaluate the suitability of the standard table for use in the company’s financial
modelling by performing two additional tests for different possible
inconsistencies between the actual death rates and those represented by the
standard table.
[6]
The company discovers that at age 70 years, one individual owns 25 of the policies in
the investigation, the remaining policies each being owned by different individuals.
(iv)
Assess the impact of this on the variance of the number of claims at age 70
years.

[Total 18]
END OF PAPER
CT4 S2015–10
%%%%%%%%%%%%%%%%%%%%%%%%%%%%%%%%%%%%%%%%%%%%%%%%%%%%%%%%%%%%%%%%%%%%%%%%%%%%%%%%%%%%%%%%%%%%%%%%%%%%%%%%%%%%%%%%%%%%%%%
Q11
(i)
If the previous experience is the recent experience of the policyholders of a
life insurance company, the comparison could be important for pricing life
insurance contracts.
Mortality rates are expected to change over time due to for example, improved
medical processes or change in the mix of the population.
It helps to validate the results of the investigation.
It is can indicate whether the office’s experience is out of line with the
population as a whole.
Unexpected changes in mortality may have an impact on the underwriting
process.
%%--------------------------- 16Subject CT4 %%%%%%%%%%%%%%%%%%%%%%%%%%%%%%%%%%%%%%%%%%%%%%%%%%%%%%%%%%%%%%%%%%%%
It is important for the company to know whether the investigation’s results are
consistent with published life tables, especially if the company plans to use
published tables for any financial calculations.
(ii)
The null hypothesis is that the company's underlying mortality experience is
the same as the standard table.
The calculations are shown in the table below.
Age x Number of
policies Actual
deaths Expected
deaths z x z x 2
70
71
72
73
74
75
76
77
78
79 1,000
1,200
1,100
1,100
1,000
1,000
950
900
850
800 13
28
31
34
39
41
41
40
46
48 23.74
31.80
32.50
36.20
36.63
40.73
42.99
45.20
47.34
49.35 -2.204
-0.674
-0.263
-0.366
0.392
0.042
-0.304
-0.773
-0.195
-0.192 4.859
0.454
0.069
0.134
0.153
0.002
0.092
0.598
0.038
0.037
Sum
6.436
An overall test of the hypothesis is the chi-squared test.
The test statistic is
 z x 2
x
where
z x ;\+\;
E x c \mu x - E x c \mu s x
E x c \mu s x
and
E x c is the central exposed to risk at age x, μ x is the observed death rate at age
x, and \mu sx is the death rate at age x in the standard table.
The calculated chi-squared statistic is 6.436.
We have 10 ages, so 10 degrees of freedom.
At the 95% level, the critical value is 18.31.
%%--------------------------- 17Subject CT4 %%%%%%%%%%%%%%%%%%%%%%%%%%%%%%%%%%%%%%%%%%%%%%%%%%%%%%%%%%%%%%%%%%%%
Since 6.436 < 18.31
we do not reject the null hypothesis that the underlying mortality of the
company’s policyholders is, overall, represented by the standard table.
(iii)
Overall, the data fit the standard table pretty well, but there are features which
the chi-squared test fails to detect such as small but consistent bias, or the
existence of outliers.
EITHER SIGNS TEST
This tests for bias.
We have only 2 positive signs out of 10 ages.
The probability of getting only 2 positive signs under the hypothesis that the
underlying mortality of the company’s policyholders is, overall, represented
by the standard table is equal to
 10  10

 0.5 ;\+\; 0.0439
 8 or 2 
which is greater than 0.025.
Therefore at the 95% significance level we can say that there is no bias.
OR CUMULATIVE DEVIATIONS TEST
This tests for bias.
 (Observed deaths - Expected deaths)
the test statistic
x
 Expected deaths
~ Normal(0,1).
x
So, using the results in the table, the value of the test statistic is
361 - 386.48
;\+\; - 1.30
386.48
Since –1.96 < test statistic < +1.96
Therefore at the 95% significance level we can say that there is no bias.
GROUPING OF SIGNS TEST
This tests to see if the shape of the mortality is the same, or whether there is
“clumping” of the deviations.
%%--------------------------- 18Subject CT4 %%%%%%%%%%%%%%%%%%%%%%%%%%%%%%%%%%%%%%%%%%%%%%%%%%%%%%%%%%%%%%%%%%%%
We also have just one run of positive signs with 2 positive signs and 8
negative signs.
We could try the Grouping of Signs test. The table on p. 189 of the Golden
Book shows no value for these data. So we cannot reject the null hypothesis
of no difference between the two sets of mortality rates.
The run of negatives, then positives, then negatives does look odd but we have
too few age groups to conclude that this is a problem.
INDIVIDUAL STANDARDISED DEVIATIONS TST
This is a test for outliers.
Under the null hypothesis we would expect the individual z x s to be distributed
Normal (0,1)
and therefore only 1 in 20 z x s should have absolute magnitude greater than
1.96 (or none should be outside -3 to +3)
OR
table showing split of deviations, actual versus expected as below
Range
-∞,-2
Expected 0.2
Actual
1
-2,-1
1.4
0
-1,0
3.4
7
0,1
3.4
2
1,2
1.4
0
2,+∞
0.2
0
In fact, z 70 = -2.20, which should give cause for concern.
Nevertheless this test is inconclusive (1 deviation out of 10 ages is greater than
1.96).
(iv)
The variance of the number of claims at age 70 years will increase in the ratio
 i 2  i
i
,
 i  i
i
where  i is the proportion of policyholders who have i policies.
If, at age 70 years, the other 975 policyholders each have one policy, then we
have
 1 ;\+\; 0.99898,  25 ;\+\; 0.00102 and  i = 0 for all other values of i.
%%--------------------------- 19Subject CT4 %%%%%%%%%%%%%%%%%%%%%%%%%%%%%%%%%%%%%%%%%%%%%%%%%%%%%%%%%%%%%%%%%%%%
OR
 1 ;\+\;
975
1
and  i ;\+\; 0 for all other values of I.
,  2 ;\+\;
976
976
Thus for age 70 the variance ratio is
(1*1* 0.99898) \;=\; (25* 25* 0.00102) 1.639
;\+\;
;\+\; 1.598
(1* 0.99898) \;=\; (25* 0.00102)
1.026
OR
(975 / 976) \;=\; 25* 25*(1/ 976)
;\+\; 1.6
(975 / 976) \;=\; 25(1/ 976)
So the variance of the number of claims will be inflated by a factor of 1.598
(or 1.6).
Most candidates made a fair effort at part (i) and the majority scored highly on
part (ii). Part (iii), however, was uncertainly answered by many candidates.
Some candidates carried out both the Signs Test and the Cumulative
Deviations Test, which were really checking for the same feature of the data.
In part (iii), credit was only given for two tests. Only a minority of candidates
attempted part (iv).
END OF EXAMINERS’ REPORT
%%--------------------------- 20
