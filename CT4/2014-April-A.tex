\documentclass[a4paper,12pt]{article}

%%%%%%%%%%%%%%%%%%%%%%%%%%%%%%%%%%%%%%%%%%%%%%%%%%%%%%%%%%%%%%%%%%%%%%%%%%%%%%%%%%%%%%%%%%%%%%%%%%%%%%%%%%%%%%%%%%%%%%%%%%%%%%%%%%%%%%%%%%%%%%%%%%%%%%%%%%%%%%%%%%%%%%%%%%%%%%%%%%%%%%%%%%%%%%%%%%%%%%%%%%%%%%%%%%%%%%%%%%%%%%%%%%%%%%%%%%%%%%%%%%%%%%%%%%%%

\usepackage{eurosym}
\usepackage{vmargin}
\usepackage{amsmath}
\usepackage{graphics}
\usepackage{epsfig}
\usepackage{enumerate}
\usepackage{multicol}
\usepackage{subfigure}
\usepackage{fancyhdr}
\usepackage{listings}
\usepackage{framed}
\usepackage{graphicx}
\usepackage{amsmath}
\usepackage{chngpage}

%\usepackage{bigints}
\usepackage{vmargin}

% left top textwidth textheight headheight

% headsep footheight footskip

\setmargins{2.0cm}{2.5cm}{16 cm}{22cm}{0.5cm}{0cm}{1cm}{1cm}

\renewcommand{\baselinestretch}{1.3}

\setcounter{MaxMatrixCols}{10}

\begin{document}
\begin{enumerate}
1 State the benefits of modelling in actuarial work.
2 (i) Explain why data are subdivided into homogeneous groups when mortality
investigations are conducted.
[2]
(ii) List four factors, other than age and sex, by which mortality statistics are often
subdivided.
[2]
[Total 4]
3
[4]
Explain what a stochastic model is and how it differs from a deterministic model. [4]
CT4 A2014–24
(i)
State the principle of correspondence as it relates to mortality investigations.
[1]
Two small countries conduct population censuses on an annual basis. Country A
records its population on 1 February every year based on an age definition of age last
birthday. Country B records its population on every 1 August using a definition of
age nearest birthday. Each country records deaths as they happen based on age
next birthday.
Below are some data from the last few years.
Age last
birthday
44
45
46
Age nearest
birthday
44
45
46
Country A
Population
Population
Population
1 February 2011 1 February 2012 1 February 2013
382,000
374,000
354,000
394,000
381,000
372,000
Country B
Population
Population
1 August 2011
1 August 2012
382,000
374,000
354,000
401,000
385,000
375,000
Population
1 August 2013
394,000
381,000
372,000
401,000
385,000
375,000
In the combined lands of Countries A and B in the calendar year 2012 there were
4,800 deaths of those aged 46 next birthday and 4,500 deaths of those aged 45 next
birthday.
The two countries decide to form an economic union, after which it will be mandatory
to offer the same rates for life insurance to residents of each country.
(ii)
(iii)
CT4 A2014–3
Estimate the death rate at age 45 years last birthday for the two countries
combined.
Explain the exact age to which your estimate relates.
[6]
[1]
[Total 8]
PLEASE TURN OVER
%%%%%%%%%%%%%%%%%%%%%%%%%%%%%%%%%%%%%%%%%%%%%%%%%%%%%%%%%%%%%%%%%%%%%%%%%%%%%%%%
\newpage

1
Systems with long time frames, such as the operation of a pension fund, can be
studied in compressed time.
Complex systems with stochastic elements, such as the operation of a life insurance
company, can be studied by simulation modelling when mathematical or logical
models cannot describe them in ways which are easy to interpret.
Different future policies of possible actions can be compared to see which best suits
the requirements or constraints of a user OR to test the sensitivity of profits under
different scenarios.
In a model of complex systems we can usually get control over experimental
conditions so that we can reduce the variance of the results from the model without
upsetting their mean values.
Alternative suggestions were also given credit, for example “to improve competitiveness by
testing out new underwriting approaches based on postcodes”, or “to help understand the
correlation between actions and decisions”, or “to allow the user to understand better the
potential impact of changes over which he or she may have little control”. Most candidates
scored fairly well on this question.
2
(i)
All our models and analyses are based on the assumption that we can observe
groups of identical lives (or at least, lives whose mortality characteristics are
the same).
Although in practice, this is never possible.
We can at least subdivide our data according to characteristics known, from
experience, to have a significant effect on mortality.
This ought to reduce the heterogeneity of each class so formed.
(ii)
Type of policy (which often reflects the reason for insuring)
Smoker/non-smoker status
Level of underwriting
Duration in force
Sales channel
Policy size
Occupation of policyholder OR socio-economic class
Known impairments
Postcode/geographical location
Marital status
Answers to part (i) were often unconvincing. The question asked specifically about the
conduct of mortality investigations: many candidates wrote about the problems of selection in
heterogeneous data, and this was not given credit. Most candidates scored full marks on
part (ii).
Page 3Subject CT4 (Models Core Technical) – April 2014 – Examiners’ Report
3
A stochastic model is one which recognises the random nature of the input
components, whereas a deterministic model does not contain any random
components.
In a stochastic model the output of each run is one value from a distribution. By
contrast, in a deterministic model, the output is determined once the set of fixed inputs
and the relationships between them have been defined.
In a stochastic model, several independent runs are required for each set of inputs so
that statistical theory can be used to help study the implications of a set of inputs. A
deterministic model only requires one run.
Running a stochastic model many times will produce a distribution of results for
possible scenarios, whereas a deterministic model will produce results for a single
scenario. Thus a deterministic model can be seen as a special case of a stochastic
model.
For many stochastic models, it is necessary to use numerical approximations in order
to integrate functions or solve differential equations. The results for a deterministic
model can often be obtained by direct calculations.
Monte Carlo simulation is an example of a stochastic model: a collection of
deterministic models each with an associated probability.
Most candidates made a reasonable attempt at this question. Some candidates wrote
answers that repeated the same point using different words. Credit was only given once for
each point. Comments that stochastic models are more costly or harder to interpret than
deterministic models were not given any credit as they do not explain what a stochastic
model is.
