CT4 A2008—49
An investigation into the mortality of patients following a specific type of major
operation was undertaken. A sample of 10 patients was followed from the date of the
operation until either they died, or they left the hospital where the operation was
carried out, or a period of 30 days had elapsed (whichever of these events occurred
first). The data on the 10 patients are given in the table below.
(i)
Patient number Duration of
observation
(days) Reason for
observation
ceasing
1
2
3
4
5
6
7
8
9
10 2
6
12
20
24
27
30
30
30
30 Died
Died
Died
Left hospital
Left hospital
Died
Study ended
Study ended
Study ended
Study ended
State whether the following types of censoring are present in this
investigation. In each case give a reason for your answer.
(a)
(b)
(c)
Type I
Type II
Random
[3]
(ii) State, with a reason, whether the censoring in this investigation is likely to be
informative.
[1]
(iii) Calculate the value of the Kaplan-Meier estimate of the survival function at
duration 28 days.
[5]
(iv) Write down the Kaplan-Meier estimate of the hazard of death at duration 8
days.
[1]
(v) Sketch the Kaplan-Meier estimate of the survival function.
CT4 A2008—5
[2]
[Total 12]
PLEASE TURN OVER10
An internet service provider (ISP) is modelling the capacity requirements for its
network. It assumes that if a customer is not currently connected to the internet
(“offline”) the probability of connecting in the short time interval [t,dt] is
0.2dt + o(dt). If the customer is connected to the internet (“online”) then it assumes
the probability of disconnecting in the time interval is given by 0.8dt + o(dt).
The probabilities that the customer is online and offline at time t are P ON (t) and
P OFF (t) respectively.
(i)
Explain why the status of an individual customer can be considered as a
Markov Jump Process. [2]
(ii) ′ ( t ) .
Write down Kolmogorov’s forward equation for P OFF [2]
(iii) Solve the equation in part (ii) to obtain a formula for the probability that a
customer is offline at time t, given that they were offline at time 0.
[3]
(iv) Calculate the expected proportion of time spent online over the period [0,t].
[HINT: Consider the expected value of an indicator function which takes the
value 1 if offline and 0 otherwise.]
[4]
(v) (a)
(b)
CT4 A2008—6
Sketch a graph of your answer to (iv) above.
Explain its shape.
[3]
[Total 14]11
An investigation was carried out into the relationship between sickness and mortality
in an historical population of working class men. The investigation used a three-state
model with the states:
1
2
3
Healthy
Sick
Dead
Let the probability that a person in state i at time x will be in state j at time x+t be
ij
ij
t p x . Let the transition intensity at time x+t between any two states i and j be μ x + t .
(i) Draw a diagram showing the three states and the possible transitions between
them.
[2]
(ii) Show from first principles that
∂
23
21 13
22 23
t p x = t p x μ x + t + t p x μ x + t .
∂ t
(iii)
[5]
Write down the likelihood of the data in the investigation in terms of the
transition rates and the waiting times in the Healthy and Sick states, under the
assumption that the transition rates are constant.
[3]
The investigation collected the following data:
•
•
•
•
man-years in Healthy state
man-years in Sick state
number of transitions from Healthy to Sick
number of transitions from Sick to Dead
265
140
20
40
(iv) Derive the maximum likelihood estimator of the transition rate from Sick to
Dead.
[3]
(v) Hence estimate:
(a)
(b)
the value of the constant transition rate from Sick to Dead
95 per cent confidence intervals around this transition rate
[4]
[Total 17]
END OF PAPER
CT4 A2008—7
