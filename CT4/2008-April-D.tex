\documentclass[a4paper,12pt]{article}

%%%%%%%%%%%%%%%%%%%%%%%%%%%%%%%%%%%%%%%%%%%%%%%%%%%%%%%%%%%%%%%%%%%%%%%%%%%%%%%%%%%%%%%%%%%%%%%%%%%%%%%%%%%%%%%%%%%%%%%%%%%%%%%%%%%%%%%%%%%%%%%%%%%%%%%%%%%%%%%%%%%%%%%%%%%%%%%%%%%%%%%%%%%%%%%%%%%%%%%%%%%%%%%%%%%%%%%%%%%%%%%%%%%%%%%%%%%%%%%%%%%%%%%%%%%%

\usepackage{eurosym}
\usepackage{vmargin}
\usepackage{amsmath}
\usepackage{graphics}
\usepackage{epsfig}
\usepackage{enumerate}
\usepackage{multicol}
\usepackage{subfigure}
\usepackage{fancyhdr}
\usepackage{listings}
\usepackage{framed}
\usepackage{graphicx}
\usepackage{amsmath}
\usepackage{chngpage}

%\usepackage{bigints}
\usepackage{vmargin}

% left top textwidth textheight headheight

% headsep footheight footskip

\setmargins{2.0cm}{2.5cm}{16 cm}{22cm}{0.5cm}{0cm}{1cm}{1cm}

\renewcommand{\baselinestretch}{1.3}

\setcounter{MaxMatrixCols}{10}

\begin{document}
\begin{enumerate}
PLEASE TURN OVER8
An education authority provides children with musical instrument tuition. The
authority is concerned about the number of children giving up playing their
instrument and is testing a new tuition method with a proportion of the children which
it hopes will improve persistency rates. Data have been collected and a Cox
proportional hazards model has been fitted for the hazard of giving up playing the
instrument. Symmetric 95% confidence intervals (based upon standard errors) for the
regression parameters are shown below.
Covariate
Confidence Interval
Instrument
Piano
Violin
Trumpet 0
[-0.05,0.19]
[0.07,0.21]
Tuition method
Traditional
New 0
[-0.15,0.05]
Sex
Male
Female
[-0.08,0.12]
0
(i) Write down a general expression for the Cox proportional hazards model,
defining all terms that you use.
[3]
(ii) State the regression parameters for the fitted model. [2]
(iii) Describe the class of children to which the baseline hazard applies. [1]
(iv) Discuss the suggestion that the new tuition method has improved the chances
of children continuing to play their instrument.
[3]
(v) Calculate, using the results from the model, the probability that a boy will still
be playing the piano after 4 years if provided with the new tuition method,
given that the probability that a girl will still be playing the trumpet after 4
years following the traditional method is 0.7.
[3]
[Total 12]

%%%%%%%%%%%%%%%%%%%%%%%%%%%%%%%%%%%%%%%%%%%%%%%%%%%%%%%%%%%%%%%%%%%%%%%%%%%%%%%%%%%%%%

8
(i)
h ( z , t ) = h 0 ( t ).exp( β . z i T )
where h(z,t) is the hazard at duration t
h o (t) is the baseline hazard
z i are the covariates
β is the vector of regression parameters
(ii)
(iii)
z 1 = 1 plays violin, 0 otherwise β 1 = 0.07
z 2 = 1 plays trumpet, 0 otherwise β 2 = 0.14
z 3 = 1 new tuition method, 0 otherwise β 3 = − 0.05
z 4 = 1 male, 0 otherwise β 4 = 0.02
Baseline hazard refers to
a female,
following traditional tuition method,
playing the piano
(iv)
The parameter associated with the new tuition
method is - 0.05. Because the parameter is negative, the hazard of dropping
out is reduced by the new tuition method.
Therefore the new tuition method does appear
to improve the chances of a child continuing
with his or her instrument.
However the 95% confidence interval for the parameter spans zero.
So at the 5% significance level it is not possible to conclude that the new
tuition method has improved the chances of children continuing to play their
instrument.
(v)
The hazard for a girl being taught the trumpet by the traditional
method giving up is h 0 ( t ) exp(0.14) .
Therefore the probability of her still playing after 4 years is
4
⎛ 4
⎞
⎛
⎞
S female (4) = exp ⎜ − ∫ h 0 ( t ) exp(0.14) dt ⎟ = exp ⎜ − 1.150274 ∫ h 0 ( t ) dt ⎟
0
⎝ 0
⎠
⎝
⎠
Page 13Subject CT4 — Models Core Technical — April 2008 — Examiners’ Report
Since this is equal to 0.7, we have
4
⎛
⎞
exp ⎜ − 1.150274 ∫ h 0 ( t ) dt ⎟ = 0.7, so that
0
⎝
⎠
4
log e 0.7 = − 1.150274 ∫ h 0 ( t ) dt ,
0
4
and hence
∫ h 0 ( t ) dt =
0
log e 0.7
= 0.310078.
− 1.150274
The hazard of giving up for a boy taught the piano by the new
method is h 0 ( t ) exp( − 0.05 + 0.02) = h 0 ( t ) exp( − 0.03).
Therefore the probability of him still playing after 4 years is
⎛ 4
⎞
S male (4) = exp ⎜ − ∫ h 0 ( t ) exp( − 0.03) dt ⎟ = exp [ − 0.310078(0.970446) ]
⎝ 0
⎠
which is exp( - 0.300914) = 0.74014.
ALTERNATIVELY
The hazard of giving up for a girl being taught the trumpet by the
traditional method is h 0 ( t ) exp( β 2 ) .
Therefore the probability of her still playing after 4 years is
4
⎛ 4
⎞
⎛
⎞
S female (4) = exp ⎜ − ∫ h 0 ( t ) exp( β 2 ) dt ⎟ = exp ⎜ − exp( β 2 ) ∫ h 0 ( t ) dt ⎟
0
⎝ 0
⎠
⎝
⎠
and hence
4
∫ h 0 ( t ) dt =
0
log e [ S female (4)]
− exp β 2
= − exp( − β 2 ) log e [ S female (4)] .
The hazard of a boy being taught the piano by the new
method giving up is h 0 ( t ) exp( β 3 + β 4 ) .
Page 14Subject CT4 — Models Core Technical — April 2008 — Examiners’ Report
Therefore the probability of him still playing after 4 years is
4
⎛
⎞
S male (4) = exp ⎜ − exp( β 3 + β 4 ) ∫ h 0 ( t ) dt ⎟ .
0
⎝
⎠
4
Substituting for
∫ h ( t ) dt produces
0
0
S male (4) = exp ( exp( β 3 + β 4 ) exp( − β 2 ) log e [ S female (4)] )
= exp[exp(-0.05+0.02)exp(-0.14)log e (0.7)]
= exp[0.970446 x 0.869358 x -0.356675)
= 0.74014.
