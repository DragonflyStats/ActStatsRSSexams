\documentclass[a4paper,12pt]{article}

%%%%%%%%%%%%%%%%%%%%%%%%%%%%%%%%%%%%%%%%%%%%%%%%%%%%%%%%%%%%%%%%%%%%%%%%%%%%%%%%%%%%%%%%%%%%%%%%%%%%%%%%%%%%%%%%%%%%%%%%%%%%%%%%%%%%%%%%%%%%%%%%%%%%%%%%%%%%%%%%%%%%%%%%%%%%%%%%%%%%%%%%%%%%%%%%%%%%%%%%%%%%%%%%%%%%%%%%%%%%%%%%%%%%%%%%%%%%%%%%%%%%%%%%%%%%

\usepackage{eurosym}
\usepackage{vmargin}
\usepackage{amsmath}
\usepackage{graphics}
\usepackage{epsfig}
\usepackage{enumerate}
\usepackage{multicol}
\usepackage{subfigure}
\usepackage{fancyhdr}
\usepackage{listings}
\usepackage{framed}
\usepackage{graphicx}
\usepackage{amsmath}
\usepackage{chng%% --- Page}

%\usepackage{bigints}
\usepackage{vmargin}

% left top textwidth textheight headheight

% headsep footheight footskip

\setmargins{2.0cm}{2.5cm}{16 cm}{22cm}{0.5cm}{0cm}{1cm}{1cm}

\renewcommand{\baselinestretch}{1.3}

\setcounter{MaxMatrixCols}{10}

\begin{document}

%%  ---  CT4 A2016–3
PLEASE TURN OVER6
An investigation was undertaken into the time spent waiting in check-out queues at a
supermarket. A random sample of customers was surveyed, and the times at which
they joined the check-out queue and completed their purchases were recorded. If they
left the check-out queue without completing a purchase, the time at which they left
was also recorded. Below are the data for 12 customers.
Customer
number
1
2
3
4
5
6
7
8
9
10
11
12
(i)
Time joined
10.00 a.m.
10.07 a.m.
10.10 a.m.
10.25 a.m.
10.30 a.m.
10.45 a.m.
11.10 a.m.
11.15 a.m.
11.35 a.m.
11.58 a.m.
12.10 p.m.
12.15 p.m.
Time purchase
completed
10.08 a.m.
10.09 a.m.
10.16 a.m.
10.31 a.m.
10.32 a.m.
10.49 a.m.
Time left without
making purchase
11.20 a.m.
11.21 a.m.
11.40 a.m.
12.09 p.m.
12.14 p.m.
12.22 p.m.
Calculate the Kaplan-Meier estimate of the survival function of the duration
between joining the queue and completing a purchase.
[6]
The supermarket decides to introduce a scheme under which any customer who has to
wait at a check-out for more than 10 minutes receives a $2 refund on the cost of their
shopping. The supermarket has 20,000 customers per day.
(ii) Give an estimate of the daily cost of the new scheme.
(iii) Comment on the assumptions that you have made in obtaining the estimate
in (ii).

[Total 9]
%%  ---  CT4 A2016–4

%%%%%%%%%%%%%%%%%%%%%%%%%%%%%%%%%%%%%%%%%%%%%%%%%%%%%%%%%%%%%%%%%%

Q6
(i)
Re-write the data as shown in the table below (* denotes a person who left
without making a purchase).
Customer number Duration
1
2
3
4
5
6
7
8
9
10
11
12 8
2
6
6
2
4
10*
6
5*
11
4
7*

Treating those who left without making a purchase as censored we create the
following table.
t j N j
0
2
4
6
8
11
 12
12
10
7
3
1

d j c j
0
2
2
3
1
1
 0
0
1
1
1
0

d j
N j
2/12
2/10
3/7
1/3
1

1 −
d j
N j
10/12
8/10
4/7
2/3
0


%% --- Page 12%%%%%%%%%%%%%%%%%%%%%%%%%%%%%%%%%%%%%%%%%%%5(Models Core Technical) – April 2016 – Examiners’ Report
The Kaplan-Meier estimate is thus
t S(t)
0 ≤ t < 2
2 ≤ t < 4
4 ≤ t < 6
6 ≤ t < 8
8 ≤ t < 11
11 ≤ t
 1
0.8333
0.6667
0.3810
0.2540
0


[Total 6]
(ii)
S (10) = 0.2540

so the daily cost of the scheme will be 0.2540 × 20,000 × $2 = $10,159. 
[Total 1]
(iii)
The survey data mainly relate to the morning. We assume that the staffing levels of the check-outs relative to customer flow remain the same in the
afternoons.

We assume that the introduction of the compensation scheme does not change
customers’ behaviour (for example discouraging customers from leaving the queue).

The sample size (12) is very small compared to the daily customer base
(20,000) which produces a very “steppy” result. We have had to use the value for S(10) which is also the value for S(8). A larger sample size may give a
smoother more accurate picture.

[Max 2]
[TOTAL 9]
Many candidates scored highly on part (i). A minority of candidates treated leaving without making a purchase as the decrement of interest, and received partial credit if they applied the method correctly. In part (ii) many candidates
did not use their estimate of S(10) from part (i). Answers to part (iii) were encouraging, with a gratifyingly large number of candidates making sensible
points. Some credit was given in part (iii) for comments about the
assumptions underlying the Kaplan Meier estimate, such as the presence of
non-informative censoring or the population being homogeneous.
%% --- Page 13%%%%%%%%%%%%%%%%%%%%%%%%%%%%%%%%%%%%%%%%%%%5(Models Core Technical) – April 2016 – Examiners’ Report
