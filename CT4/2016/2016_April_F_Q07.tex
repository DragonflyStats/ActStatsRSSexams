

\documentclass[a4paper,12pt]{article}

%%%%%%%%%%%%%%%%%%%%%%%%%%%%%%%%%%%%%%%%%%%%%%%%%%%%%%%%%%%%%%%%%%%%%%%%%%%%%%%%%%%%%%%%%%%%%%%%%%%%%%%%%%%%%%%%%%%%%%%%%%%%%%%%%%%%%%%%%%%%%%%%%%%%%%%%%%%%%%%%%%%%%%%%%%%%%%%%%%%%%%%%%%%%%%%%%%%%%%%%%%%%%%%%%%%%%%%%%%%%%%%%%%%%%%%%%%%%%%%%%%%%%%%%%%%%

\usepackage{eurosym}
\usepackage{vmargin}
\usepackage{amsmath}
\usepackage{graphics}
\usepackage{epsfig}
\usepackage{enumerate}
\usepackage{multicol}
\usepackage{subfigure}
\usepackage{fancyhdr}
\usepackage{listings}
\usepackage{framed}
\usepackage{graphicx}
\usepackage{amsmath}
\usepackage{chng%% --- Page}

%\usepackage{bigints}
\usepackage{vmargin}

% left top textwidth textheight headheight

% headsep footheight footskip

\setmargins{2.0cm}{2.5cm}{16 cm}{22cm}{0.5cm}{0cm}{1cm}{1cm}

\renewcommand{\baselinestretch}{1.3}
\setcounter{MaxMatrixCols}{10}

\begin{document}

%%%%%%%%%%%%%%%%%%%%%%%%%%%%%%%%%%%%%%%%%%%%%%%%%%%%%%%%%%%%%%%%%%%%%%%%%%%%%%%%%%%%

7
(i)
(ii)
State the condition needed for a Markov Jump Process to be time
inhomogeneous.

Describe the principal difficulties in modelling using a Markov Jump Process
with time inhomogeneous rates.

A multi-tasking worker at a children’s nursery observes whether children are being
“Good” or “Naughty” at all times. Her observations suggest that the probability of a
child moving between the two states varies with the time, t, since the child arrived at
the nursery in the morning. She estimates that the transition rates are:
From Good to Naughty: 0.2 + 0.04t
From Naughty to Good: 0.4  0.04t
where t is measured in hours from the time the child arrived in the morning,
0  t  8.
A child is in the “Good” state when he arrives at the nursery at 9 a.m.
\begin{enumerate}
\item (iii) Calculate the probability that the child is Good for all the time up until time t.
\item (iv) Calculate the time by which there is at least a 50% chance of the child having
been Naughty at some point.

Let P G (t) be the probability that the child is Good at time t.
\item (v)
%%  ---  CT4 A2016–5
Derive a differential equation just involving P G (t) which could be used to
determine the probability that the child is Good on leaving the nursery at
5 p.m.
\end{itemize}
%%--- [Total 10]

%%%%%%%%%%%%%%%%%%%%%%%%%%%%%%%%%%%%%%%%%%%%%%%%%%%%%%%%%%%%%%%%%%%%%%%%%%%%%%%%%%%%%%%%%
Q7
(i)
\begin{itemize}
\item Where the probabilities depend only on the length of time interval t – s , the process is called time-homogeneous.
\item 
If this condition is not met, the process is time-inhomogeneous.

\item ALTERNATIVE 1
If the probabilities do not depend solely upon the length of the time interval t -
s , the process is time-inhomogeneous.
\item 
OR ALTERNATIVE 2
The transition rates vary with time.
\item 
OR ALTERNATIVE 3
The transition rates depend upon the start and end times.
\end{itemize}

(ii)

%%--- [Max 1]
A model with time-inhomogeneous rates has more parameters, and there may
not be sufficient data available to estimate these parameters.

Also, the solutions to Kolmogorov’s equations may not be easy (or even
possible) to find analytically.

Time-inhomogeneous processes are computationally harder to simulate. 
[Max 2]
(iii)
We need P GG ( t ), i.e. probability the process remains in G throughout time 0 to
t .
This satisfies
d
P ( t ) = ( − 0.2 − 0.04 t ) P GG ( t ).
dt GG

Hence
1
d
P ( t ) = ( − 0.2 − 0.04 t );
P GG ( t ) dt GG
d
[ln P GG ( t )] = ( − 0.2 − 0.04 t ).
dt
%% --- Page 14
%%%%%%%%%%%%%%%%%%%%%%%%%%%%%%%%%%%%%%%%%%%5(Models Core Technical) – April 2016 – Examiners’ Report
Integrate both sides:
ln P GG ( s )
s = t
s = 0
= − 0.2 s − 0.02 s 2
s = t
s = 0
.

P GG (0) = 1, 
so P GG ( t ) = exp( − 0.2 t − 0.02 t 2 ). 
[Total 3]
(iv)
Occurs when P GG ( t ) = 0.5, so we have

0.5 = exp( − 0.2 t − 0.02 t 2 ),
0.02 t 2 + 0.2 t − 0.69315 = 0,

and solving using the quadratic equation formula produces
$t = 2.724$ or $t = − 12.724.$

The answer lies between 0 and 8, so we require t = 2.724.
(v)

[Total 2]
\[d
P G ( t ) = ( − 0.2 − 0.04 t ) P G ( t ) + (0.4 − 0.04 t ) P N ( t )
dt \]
But $P N ( t ) = 1 − P G ( t ),$ 
So
\[d
P G ( t ) = ( − 0.2 − 0.04 t ) P G ( t ) + (0.4 − 0.04 t ) (1 − P G ( t )),
dt\]

or
\[d
P G ( t ) = 0.4 − 0.04 t − 0.6 P G ( t ).
dt\]

%% [Total 2]
%% [TOTAL 10]
%% Answers to this question were disappointing, although most candidates made good attempts at parts (i) and (iii).
%% --- Page 15%%%%%%%%%%%%%%%%%%%%%%%%%%%%%%%%%%%%%%%%%%%5(Models Core Technical) – April 2016 – Examiners’ Report
\end{document}
