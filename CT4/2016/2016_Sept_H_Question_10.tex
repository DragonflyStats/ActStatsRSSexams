\documentclass[a4paper,12pt]{article}

%%%%%%%%%%%%%%%%%%%%%%%%%%%%%%%%%%%%%%%%%%%%%%%%%%%%%%%%%%%%%%%%%%%%%%%%%%%%%%%%%%%%%%%%%%%%%%%%%%%%%%%%%%%%%%%%%%%%%%%%%%%%%%%%%%%%%%%%%%%%%%%%%%%%%%%%%%%%%%%%%%%%%%%%%%%%%%%%%%%%%%%%%%%%%%%%%%%%%%%%%%%%%%%%%%%%%%%%%%%%%%%%%%%%%%%%%%%%%%%%%%%%%%%%%%%%

\usepackage{eurosym}
\usepackage{vmargin}
\usepackage{amsmath}
\usepackage{graphics}
\usepackage{epsfig}
\usepackage{enumerate}
\usepackage{multicol}
\usepackage{subfigure}
\usepackage{fancyhdr}
\usepackage{listings}
\usepackage{framed}
\usepackage{graphicx}
\usepackage{amsmath}
\usepackage{chngpage}

%\usepackage{bigints}
\usepackage{vmargin}

% left top textwidth textheight headheight

% headsep footheight footskip

\setmargins{2.0cm}{2.5cm}{16 cm}{22cm}{0.5cm}{0cm}{1cm}{1cm}

\renewcommand{\baselinestretch}{1.3}

\setcounter{MaxMatrixCols}{10}

\begin{document}PLEASE TURN OVER10
A researcher is investigating the contributing factors to the speed at which patients
recover from a common minor surgical procedure undertaken in hospitals across the
country. He has the questionnaires which each patient completed before the surgery
and the length of time the patient remained in hospital after surgery and is attempting
to fit a Cox proportional hazards model to the data.
He has fitted a model with what he assumes are the most common contributing factors
and has calculated the parameters as shown in the table below:
(i)
Covariate Category Parameter
Gender Male
Female 0
0.065
Smoker Non Smoker
Smoker 0.035
0
Drinker Non Drinker
Moderate Drinker
Heavy Drinker 0.06
0
0.085
Give the hazard function for this Cox proportional hazards model, defining all
the terms and covariates.
[4]
A male moderate drinker who does not smoke has a hazard of leaving hospital after
three days of 0.6.
(ii)
Calculate the probability that a female heavy drinker who smokes and who is
still in hospital after three days is NOT discharged at that point.
[3]
A colleague suggests that, in his experience, gender has no material impact on the
length of time in hospital after surgery.
(iii)
Explain how the researcher could test this suggestion statistically.
[2]
Another colleague suggests that the original model is good, but could be improved by
including an additional factor as to whether a patient is married or not.
(iv)
CT4 S2016–8
Set out how the researcher could establish whether an additional factor
representing marital status would improve the model.
[4]

%%%%%%%%%%%%%%%%%%%%%%%%%%%%%%%%%%%%%%%%%%%%%%%%%%%%%%%%%%%%%%%%%%%%%%%%%%%%%%%%%%%%%%%%%%%

Q10
(i)
\lambda(t:Z i ) = \lambda 0 (t) exp (0.065Z 1 – 0.035Z 2 – 0.06Z 3 + 0.085Z 4 ).
[2]
Here
\lambda(t:Z i ) is the hazard of being discharged at time t,
\lambda 0 (t) is the baseline hazard,

and
Z 1 is the gender covariate = 1 if the patient is female and 0 if the patient is male.
Z 2 is the smoker covariate = 1 if the patient is a non-smoker and 0 if the patient is a
smoker.
Z 3 is the non-drinker covariate = 1 if the patient does not drink and 0 if the patient
drinks.
Z 4 is the heavy drinker covariate = 1 if the patient is a heavy drinker and 0 otherwise.
[11⁄2]
(ii)
A male moderate drinker who does not smoke has hazard of leaving after 3 days of
\lambda 0 (3) exp (0 – 0.035 + 0 + 0). 
So 0.6 = \lambda 0 (3) exp (−0.035). 
A female heavy drinker who smokes has a hazard of leaving of
\lambda 0 (3) exp (0.065 + 0 + 0 + 0.085) = \lambda 0 (3) exp (0.15)
So the probability that the female is discharged is 0.6
%% ---  Page  16
[1]
exp(0.15)
= 0.7219,
exp( − 0.035)
Subject CT4 %%%%%%%%%%%%%%%%%%%%%%%%%%%%%%%%%%%%%5 – September 2016 – 
(iii)
(iv)
and the probability she is not discharged is 1 − 0.7219 = 0.2781 or 28%. 
The colleague’s null hypothesis is that the gender parameter is actually zero. 
From the data calculate the variance of the estimate of the parameter for gender. 
The 95\% confidence interval is then the parameter ± 1.96 * standard deviation. 
If this range does not include 0, we can be 95\% confident that the gender has a
material impact and we can reject the null hypothesis. 
A suitable statistical test is the log likelihood test. 
The null hypothesis is that the coefficient of the marital status term is zero. 
We compare the model with and without the extra parameter. 
If the log-likelihoods for the two models are L with and L without respectively, then the
test statistic is −2(L without − L with ).

This statistic has a chi-squared distribution with one degree of freedom (since the
marital status term involves one parameter).

If the test statistic is greater than 3.84 (the chi-squared critical value at 95% with 1
degree of freedom), we reject the null hypothesis
[1]
and conclude that the marital status term does improve the model.
\newpage

[Total 13]
In part (i) some candidates incorrectly defined the dummy variables Z 3 and Z 4
for drinking behaviour as “0 if the patient drinks and 1 if the patient is a
moderate drinker” and “0 if the patient is a heavy drinker and 1 if the patient is
a moderate drinker” respectively. This is incorrect as it leaves no category for
the heavy drinkers and the non-drinkers respectively. Other candidates failed
to mention the numerical values of the parameter estimates, which were
required for full credit as the question said “this ... model”. In part (ii) many
candidates interpreted the question as referring to a survival probability rather
than a hazard. Credit was given to all reasonable interpretations of the
question. In part (iii) it is, of course, possible to test the significance of the
gender covariate using the likelihood ratio test, and full credit was given for
this if fully explained.
%% ---  Page  17Subject CT4 %%%%%%%%%%%%%%%%%%%%%%%%%%%%%%%%%%%%%5 –September 2016 – 

\end{document}
