\documentclass[a4paper,12pt]{article}

%%%%%%%%%%%%%%%%%%%%%%%%%%%%%%%%%%%%%%%%%%%%%%%%%%%%%%%%%%%%%%%%%%%%%%%%%%%%%%%%%%%%%%%%%%%%%%%%%%%%%%%%%%%%%%%%%%%%%%%%%%%%%%%%%%%%%%%%%%%%%%%%%%%%%%%%%%%%%%%%%%%%%%%%%%%%%%%%%%%%%%%%%%%%%%%%%%%%%%%%%%%%%%%%%%%%%%%%%%%%%%%%%%%%%%%%%%%%%%%%%%%%%%%%%%%%

\usepackage{eurosym}
\usepackage{vmargin}
\usepackage{amsmath}
\usepackage{graphics}
\usepackage{epsfig}
\usepackage{enumerate}
\usepackage{multicol}
\usepackage{subfigure}
\usepackage{fancyhdr}
\usepackage{listings}
\usepackage{framed}
\usepackage{graphicx}
\usepackage{amsmath}
\usepackage{chngpage}

%\usepackage{bigints}
\usepackage{vmargin}

% left top textwidth textheight headheight

% headsep footheight footskip

\setmargins{2.0cm}{2.5cm}{16 cm}{22cm}{0.5cm}{0cm}{1cm}{1cm}

\renewcommand{\baselinestretch}{1.3}

\setcounter{MaxMatrixCols}{10}

\begin{document}
[Total 6]

PLEASE TURN OVER6

Brian worked in a large open-plan office with a communal kitchen in which the

workers made coffee. Each worker supplied his or her own coffee cup. For several

years Brian was annoyed by his coffee cups being taken away by colleagues and

never returned to the kitchen, so he decided to do an experiment. He brought into the

kitchen 20 cups which were distinguishable from the other cups in the kitchen. At the

end of each day for 15 days he counted the number of his 20 cups which remained.

The results were as follows:

Day Number of cups Day Number of cups

1

2

3

4

5

6

7

8 20

19

18

18

17

17

17

16 9

10

11

12

13

14

15 15

15

15

15

13

12

10

Brian noted that:

x the cup that “disappeared” during day 2 was taken home by Brian to be used by

his mother.

x the two cups that “disappeared” during day 13 were accidentally broken by Brian

when doing his daily check.

Let h(x) be the hazard that each of Brian’s cups is taken by colleagues during day x

and not returned, and let S(x) be the corresponding survival function.

(i)

(ii)

CT4 S2016–4

Determine an estimate of S(x) for Brian’s cups using the Nelson-Aalen

estimator.

Sketch a chart for your estimated S(x).

[6]

[2]



%%%%%%%%%%%%%%%%%%%%%%%%%%%%%%%%%%%%%%%%%%%%%%%%%%%%%%%%%%%%%%%%%%%%%%%%%%%%%%%%%%



Q6

(i)

The calculations are shown in the table below.

%------------------------------------------------%
% CT4 - 2016 - September - Question 6 % 
\begin{center}
\begin{tabular}{|c|c|c|c|c|c|}
$_j$j 	&	N j 	&	d j 	&	c j 	&	d j / N j 	&	$\sum ( d j / N j )	$ \\ \hline
2	&	20	&	1	&	1	&	0	&	0	\\ \hline
3	&	19	&	1	&	0	&	1/19	&	0.0526	\\ \hline
5	&	18	&	1	&	0	&	1/18	&	0.1082	\\ \hline
8	&	17	&	1	&	0	&	1/17	&	0.167	\\ \hline
9	&	16	&	0	&	0	&	1/16	&	0.2295	\\ \hline
13	&	15	&	1	&	2	&	0	&	0.2295	\\ \hline
14	&	13	&	2	&	0	&	1/13	&	0.3064	\\ \hline
15	&	12	&		&	0	&	02/12/19	&	0.4731	\\ \hline
\end{tabular}
\end{center}


\begin{center}
\begin{tabular}{|c|c|c|c|c|c|}
0 \leq  x < 3	&	1	\\ \hline
3 \leq  x < 5	&	0.9487	\\ \hline
5 \leq  x < 8	&	0.8974	\\ \hline
8 \leq  x < 9	&	0.8462	\\ \hline
9 \leq  x < 14	&	0.7949	\\ \hline
14 \leq  x < 15	&	0.7361	\\ \hline
x = 15 	&	0.6231	\\ \hline
\end{tabular}
\end{center}
%%%%%%%%%%%%%%%%%%%%%%%%%%%%%%%%%%%
^



d

The Nelson-Aalen estimator of S(x) is S ( x ) = exp  −  j

 t \leq x N j

 j



  .





%% ---  Page  7Subject CT4 %%%%%%%%%%%%%%%%%%%%%%%%%%%%%%%%%%%%%5 –September 2016 – 

So we have

(ii)

^

Range S ( x )
%%%%%%%%%%%%%%%%%%%%%%%%%%%%%%%%%%%%%%%%%%%%%%%%%%%%%%%%%%%%%%%%%%
A suitable sketch is shown below.

1

0.9

0.8

0.7

0.6

S(x) 0.5

0.4

0.3

0.2

0.1

0

0

2

4

6

8

10

12

14

Duration x

%%%%%%%%%%%%%%%%%%%%%%%%%%%%%%%%%%%%%%%%%%%%%%%%%%%%%%%%%%%%%%%%%%%%%%%%%%%5
% Many candidates scored highly on this straightforward application of the Nelson-Aalen estimator of the survival function. Some candidates assumed that events happened at the beginning of each day rather than the end.

% These candidates could score full credit (the only change to the solution above is that the t j s for the events should be 2, 4, 7, 8, 13 and 14). A common error was to quote the estimated S(t) for a duration more than 15 days. This is incorrect because we have no information about what happened after 15 days.

%% ---  Page  8Subject CT4 %%%%%%%%%%%%%%%%%%%%%%%%%%%%%%%%%%%%%5 – September 2016 – 

\end{document}
