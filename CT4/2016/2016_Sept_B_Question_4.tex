\documentclass[a4paper,12pt]{article}

%%%%%%%%%%%%%%%%%%%%%%%%%%%%%%%%%%%%%%%%%%%%%%%%%%%%%%%%%%%%%%%%%%%%%%%%%%%%%%%%%%%%%%%%%%%%%%%%%%%%%%%%%%%%%%%%%%%%%%%%%%%%%%%%%%%%%%%%%%%%%%%%%%%%%%%%%%%%%%%%%%%%%%%%%%%%%%%%%%%%%%%%%%%%%%%%%%%%%%%%%%%%%%%%%%%%%%%%%%%%%%%%%%%%%%%%%%%%%%%%%%%%%%%%%%%%

\usepackage{eurosym}
\usepackage{vmargin}
\usepackage{amsmath}
\usepackage{graphics}
\usepackage{epsfig}
\usepackage{enumerate}
\usepackage{multicol}
\usepackage{subfigure}
\usepackage{fancyhdr}
\usepackage{listings}
\usepackage{framed}
\usepackage{graphicx}
\usepackage{amsmath}
\usepackage{chngpage}

%\usepackage{bigints}
\usepackage{vmargin}

% left top textwidth textheight headheight

% headsep footheight footskip

\setmargins{2.0cm}{2.5cm}{16 cm}{22cm}{0.5cm}{0cm}{1cm}{1cm}

\renewcommand{\baselinestretch}{1.3}

\setcounter{MaxMatrixCols}{10}

\begin{document}
%%--- Question 4

An insurance company’s business consists only of policies covering a specified event

and which pay a sum assured of £Z immediately on occurrence of this event. Claims

on the portfolio of policies are considered to occur in accordance with a Poisson

process with annual rate O.

The insurance company currently has assets of £S (£Z > £S). It charges a premium

which is to be 50\% more than the expected outgo. Premiums can be assumed to be

received continuously. The insurance company’s expenses are small and can be ignored.

\begin{enumerate}
\item (i) Derive the total annual premium charged by the insurance company on the

portfolio.
\item  (ii) Show that the probability that the insurance company has insufficient assets to

pay the next claim made is given by:

a 1 § S · o

1  exp « 

 ̈ 1   ̧ »

¬ 1.5 © Z 1 1⁄4

[3]

\en{enumerate}
%%%%%%%%%%%%%%%%%%%%%%%%%%%%%%%%%%%%%%%55
\newpage

Q4

(i)

(ii)

The expected annual outgo is \lambda Z . 

The premium is 50\% more than this, hence 1.5 \lambda Z . 

Suppose the next claim happens at time T.

Then the company is unable to pay the claim if:

\[S + 1.5 \lambda ZT < Z .\]

[1]

This implies that

T <

Z − S

1  S 

=

 1 −  .

1.5 \lambda Z 1.5 \lambda  Z 

%%%%%%%%%%%%%%%%%%%%%%5

%%---- Question 4

Suppose the next claim happens at time $T$.
Then the company is unable to pay the claim if:
\[S +1.5\lambda ZT < Z . \]
This implies that
\[ T < \frac{Z-S}{1.5\lambda Z}\]

\[ T < \frac{1}{1.5\lambda Z}\left[ 1- \frac{S}{Z}\right]\]

We are told claims arrive in accordance with a Poisson process so
\[P(t \leq T ) = 1- exp(-\lambda T )  
= 1 - exp \left( -\frac{1}{1.5} \left[ 1- \frac{S}{Z}\right] \right), \]
as required.
%%%%%%%%%%%%%%%%%%%%%%%%%%%%%%





%% ---  Page  5Subject CT4 %%%%%%%%%%%%%%%%%%%%%%%%%%%%%%%%%%%%%5 –September 2016 – 

We are told claims arrive in accordance with a Poisson process so

 1  S  

\[P( t \leq T ) = 1 − exp( −\lambda T ) = 1 − exp  −

 1 −   , as\] required.

 1.5  Z  

[11⁄2]

[Total 4]

This was an unfamiliar application of a Poisson process. Many candidates

did not attempt this question. Of those that did, most only scored credit for

part (i). Few candidates seemed to know how to attempt part (ii), especially

the idea of examining the time of the next claim.

\end{document}
