4
An insurance company’s business consists only of policies covering a specified event
and which pay a sum assured of £Z immediately on occurrence of this event. Claims
on the portfolio of policies are considered to occur in accordance with a Poisson
process with annual rate O.
The insurance company currently has assets of £S (£Z > £S). It charges a premium
which is to be 50% more than the expected outgo. Premiums can be assumed to be
received continuously. The insurance company’s expenses are small and can be
ignored.
(i) Derive the total annual premium charged by the insurance company on the
portfolio.
[1]
(ii) Show that the probability that the insurance company has insufficient assets to
pay the next claim made is given by:
a 1 § S · o
1  exp « 
 ̈ 1   ̧ »
¬ 1.5 © Z 1 1⁄4
[3]
[Total 4]
Q4
(i)
(ii)
The expected annual outgo is \lambda Z . 
The premium is 50% more than this, hence 1.5 \lambda Z . 
Suppose the next claim happens at time T.
Then the company is unable to pay the claim if:
S + 1.5 \lambda ZT < Z .
[1]
This implies that
T <
Z − S
1  S 
=
 1 −  .
1.5 \lambda Z 1.5 \lambda  Z 

%% ---  Page  5Subject CT4 %%%%%%%%%%%%%%%%%%%%%%%%%%%%%%%%%%%%%5 –September 2016 – 
We are told claims arrive in accordance with a Poisson process so
 1  S  
P( t \leq T ) = 1 − exp( −\lambda T ) = 1 − exp  −
 1 −   , as required.
 1.5  Z  
[11⁄2]
[Total 4]
This was an unfamiliar application of a Poisson process. Many candidates
did not attempt this question. Of those that did, most only scored credit for
part (i). Few candidates seemed to know how to attempt part (ii), especially
the idea of examining the time of the next claim.
