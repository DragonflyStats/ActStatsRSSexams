\documentclass[a4paper,12pt]{article}

%%%%%%%%%%%%%%%%%%%%%%%%%%%%%%%%%%%%%%%%%%%%%%%%%%%%%%%%%%%%%%%%%%%%%%%%%%%%%%%%%%%%%%%%%%%%%%%%%%%%%%%%%%%%%%%%%%%%%%%%%%%%%%%%%%%%%%%%%%%%%%%%%%%%%%%%%%%%%%%%%%%%%%%%%%%%%%%%%%%%%%%%%%%%%%%%%%%%%%%%%%%%%%%%%%%%%%%%%%%%%%%%%%%%%%%%%%%%%%%%%%%%%%%%%%%%

\usepackage{eurosym}
\usepackage{vmargin}
\usepackage{amsmath}
\usepackage{graphics}
\usepackage{epsfig}
\usepackage{enumerate}
\usepackage{multicol}
\usepackage{subfigure}
\usepackage{fancyhdr}
\usepackage{listings}
\usepackage{framed}
\usepackage{graphicx}
\usepackage{amsmath}
\usepackage{chngpage}

%\usepackage{bigints}
\usepackage{vmargin}

% left top textwidth textheight headheight

% headsep footheight footskip

\setmargins{2.0cm}{2.5cm}{16 cm}{22cm}{0.5cm}{0cm}{1cm}{1cm}

\renewcommand{\baselinestretch}{1.3}

\setcounter{MaxMatrixCols}{10}

\begin{document}


%%-- PLEASE TURN OVER
%%-- Question 8
An analysis of the number of term assurance policies in force for three companies has
revealed the following information:
Age 50
Age 51
Age 52
Year Company A Company B Company C
2013
2014
2015 6,728
6,189
5,962 2,643
2,548
2,496 4,132
2013
2014
2015 5,987
6,002
5,056 2,333
2,417
2,213 4,012
2013
2014
2015 5,359
5,600
4,906 2,155
1,992
2,006 3,895
4,630
4,500
4,367
x Company A has reported the number of policies in force on 1 January each year
using age nearest birthday.
x Company B has reported the number of policies in force on 1 November each year
using age last birthday.
x Company C has reported the number of policies in force on 31 December each
year using age next birthday, but failed to provide data for 2014.
\begin{enumrate}[(a)]
\ite (i) Calculate the contribution to the central exposed to risk for lives age 51 last
birthday for the calendar year 2014 for each company individually.
\item (ii) (a)
State the assumptions you have made in order to perform
your calculations.
(b)
Explain why these assumptions were required.
\end{enumrate}
%%--- [6]
%%%%%%%%%%%%%%%%%%%%%%%%%%%%%%%%%%%%%%%%%%%%%%%%%%%%%%%%%
\newpage 
Q8
(i)
Company A
Age 51 last = 0.5 * age 51 nearest + 0.5 * age 52 nearest. 
The Exposed-to-risk = 0.25 * ( 6,002 + 5,600 + 5,056 + 4,906) = 5,391. [1]
1⁄2(6,002 + 5,600)
= 5,801
1/1/2014
1⁄2(5,056 + 4,906)
= 4,981
1/1/2015
Company B
Population on 1/1/2014 is (2/12 * 2,417 + 10/12 * 2,333) = 2,347, 
so the contribution before 1/11/2014 is 0.5 * (2,347 + 2,417) * (10/12) = 1,985. 
The population on 1/1/2015 is (10/12 * 2,417 + 2/12 * 2,213) = 2,383, 
so the contribution after 1/11/2014 is 0.5 * (2,417 + 2,383) * 2/12 = 400. 
%% ---  Page  11Subject CT4 %%%%%%%%%%%%%%%%%%%%%%%%%%%%%%%%%%%%%5 –September 2016 – 
Therefore the overall Exposed-to-risk is 2,385.
(10/12)2,333 + (2/12) 2,417
= 2,347

(2/12)2,213 + (10/12)2,417
= 2,383
1⁄2(10/12)*
(2,347 + 2,417)
= 1,985
1/11/2013
1/1/2014
1⁄2(2/12)*
(2,417 + 2,383)
= 400
1/11/2014
1/1/2015
1/11/2015
Company C
Population on 31/12/2014 is 0.5 * (3,895 + 4,367) = 4,131, 
so the Exposed-to-risk is 0.5 * (4,131 + 3,895) = 4,013 
1⁄2 (3,895 + 4,367)
= 4,131
4,367
3,895
1⁄2 (3,895+4,131)
=4,013
31/12/2013
(ii)
31/12/2014
31/12/2015
(a) Birthdays are evenly distributed across calendar years. [1]
(b) This is needed in order to average the data for 51 and 52 year olds or
to adjust the Exposed-to-risk from age nearest to age last birthday for
Company A. [1]
(a) 1 January data in year x can be taken as 31 December data in year x − 1. [1]
(b) This is needed in order to use start-2015 data as end-2014 data for Company
C.
[1]
(a) The population varies linearly between census dates.
%% ---  Page  12
[1]Subject CT4 %%%%%%%%%%%%%%%%%%%%%%%%%%%%%%%%%%%%%5 – September 2016 – 
(b)
This is needed in order to average between census dates (OR to apply the
trapezium rule).
[1]
[Total 11]
%%%%%%%%%%%%%%%%%%%%%%%%%%%%%%%%%%%%%%%%%%%%%%%%
\newpage
Answers to this question were variable. A minority of candidates did well,
scoring full marks on part (i). Others made a range of errors. Some of these
were simple, for example reading data for the wrong company from the
examination paper. Others included averaging ages 50 and 51 for Company
A, and simply averaging the estimated Exposed-to-risk figures for 1 January
2014 and 1 January 2015 for Company B (this is incorrect because we have
additional information about the Exposed-to-risk on 1 November 2014 which
we can use). In part (ii) there were a lot of marks available, and the
Examiners were looking for accuracy and clarity. So, for example, for full
credit candidates were required to state that the “population varies linearly
between census dates”. Many candidates failed to identify the second
assumption, that we will assume 31 December data to be equivalent to
1 January in the following year.
\end{document}
