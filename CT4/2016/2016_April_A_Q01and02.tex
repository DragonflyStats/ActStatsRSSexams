\documentclass[a4paper,12pt]{article}

%%%%%%%%%%%%%%%%%%%%%%%%%%%%%%%%%%%%%%%%%%%%%%%%%%%%%%%%%%%%%%%%%%%%%%%%%%%%%%%%%%%%%%%%%%%%%%%%%%%%%%%%%%%%%%%%%%%%%%%%%%%%%%%%%%%%%%%%%%%%%%%%%%%%%%%%%%%%%%%%%%%%%%%%%%%%%%%%%%%%%%%%%%%%%%%%%%%%%%%%%%%%%%%%%%%%%%%%%%%%%%%%%%%%%%%%%%%%%%%%%%%%%%%%%%%%

\usepackage{eurosym}
\usepackage{vmargin}
\usepackage{amsmath}
\usepackage{graphics}
\usepackage{epsfig}
\usepackage{enumerate}
\usepackage{multicol}
\usepackage{subfigure}
\usepackage{fancyhdr}
\usepackage{listings}
\usepackage{framed}
\usepackage{graphicx}
\usepackage{amsmath}
\usepackage{chng%% --- Page}

%\usepackage{bigints}
\usepackage{vmargin}

% left top textwidth textheight headheight

% headsep footheight footskip

\setmargins{2.0cm}{2.5cm}{16 cm}{22cm}{0.5cm}{0cm}{1cm}{1cm}

\renewcommand{\baselinestretch}{1.3}

\setcounter{MaxMatrixCols}{10}

\begin{document}
1
Write down the information required to compute the exact exposed to risk in an
investigation of mortality. 
2 List the advantages and disadvantages of using models in actuarial work. 

%%%%%%%%%%%%%%%%%%%%%%%%%%%%%%%%%%%%%%%%%555

Q1
Date of birth
OR
exact age at a specified date; [1]
date of entry into observation; and [1]
date of exit from observation.
[1]
[TOTAL 3]
Many candidates scored full marks on this question. A common error was to
write “date of death” rather than “date of exit”, ignoring exits for reasons other
than death. The question was about exposed to risk rather than transition
rates, so no credit was given for information needed to calculate the
numerator. A minority of candidates framed their answers in aggregate
terms: these candidates were awarded little or no credit.
Q2
Advantages
Systems with long time frames (such as the operation of a pension fund) can be
studied in compressed time.
[1⁄2]
Complex systems with stochastic elements (such as the operation of a life insurance
company) can be studied by simulation modelling.
[1⁄2]
Different future policies or possible actions can be compared to see which best suits
the requirements or constraints of a user.
[1⁄2]
Scenarios which could not be tested in real life can be examined.
[1⁄2]
In a model of a complex system we can usually get control over the experimental
conditions so that we can reduce the variance of the results output from the model
without upsetting their mean values.
[1⁄2]
Disadvantages
Model development requires a considerable investment of resources (time, money or
expertise).
[1⁄2]
In a stochastic model, for any given set of inputs each run gives only estimates of a
model’s outputs. So to study the outputs for any given set of inputs, several
independent runs of the model are needed.
[1⁄2]
Models can look impressive when run on a computer so that there is a danger that one
gets lulled into a false sense of confidence.
[1⁄2]
Page 4Subject CT4 (Models Core Technical) – April 2016 – Examiners’ Report
Models rely heavily on the data input. If the data quality is poor or lack credibility
then the output from the model is likely to be flawed.
[1⁄2]
There is a danger of using a model as a “black box” from which it is assumed that all
results are valid without considering the appropriateness of using that model for the
particular data input and the output expected.
[1⁄2]
It is not possible to include all future events in a model. For example, an unforeseen
change in legislation may invalidate the model.
[1⁄2]
It may be difficult to interpret/communicate some of the outputs of the model.
Models are better at comparing various inputs than optimising outputs.
[1⁄2]
[1⁄2]
[MAX 4]
Most candidates were able to make a good attempt at this question. The
instruction in the question was “list”, so not all the detail under each point
mentioned above was required for credit.
