\documentclass[a4paper,12pt]{article}

%%%%%%%%%%%%%%%%%%%%%%%%%%%%%%%%%%%%%%%%%%%%%%%%%%%%%%%%%%%%%%%%%%%%%%%%%%%%%%%%%%%%%%%%%%%%%%%%%%%%%%%%%%%%%%%%%%%%%%%%%%%%%%%%%%%%%%%%%%%%%%%%%%%%%%%%%%%%%%%%%%%%%%%%%%%%%%%%%%%%%%%%%%%%%%%%%%%%%%%%%%%%%%%%%%%%%%%%%%%%%%%%%%%%%%%%%%%%%%%%%%%%%%%%%%%%

\usepackage{eurosym}
\usepackage{vmargin}
\usepackage{amsmath}
\usepackage{graphics}
\usepackage{epsfig}
\usepackage{enumerate}
\usepackage{multicol}
\usepackage{subfigure}
\usepackage{fancyhdr}
\usepackage{listings}
\usepackage{framed}
\usepackage{graphicx}
\usepackage{amsmath}
\usepackage{chng%% --- Page}

%\usepackage{bigints}
\usepackage{vmargin}

% left top textwidth textheight headheight

% headsep footheight footskip

\setmargins{2.0cm}{2.5cm}{16 cm}{22cm}{0.5cm}{0cm}{1cm}{1cm}

\renewcommand{\baselinestretch}{1.3}

\setcounter{MaxMatrixCols}{10}

\begin{document}

%%%%%%%%%%%%%%%%%%%%%%%%%%%%%%%%%%%%%%%%%%%%%%%%%%%%%%%%%%%%%%%%%%%%%%%%%%%%%%%%%%%%

PLEASE TURN OVER10
In a small country there are only four authorised insurance companies, A, B, C and D.
The law in this country requires homeowners to take out buildings insurance from an
authorised insurance company. All policies provide cover for a period of one year.
Based on analysis of the compliance database used to check that every home is
insured, the probabilities of a homeowner transferring between the four companies at
the end of each year are considered to be described by the following transition matrix:
A  0 . 5

B  0 . 2
C  0 . 3

D   0
0 . 2
0 . 6
0 . 2
0 . 2
0 . 2
0 . 1
0 . 4
0 . 2
0 . 1 

0 . 1 
0 . 1 

0 . 6  
Yolanda has just bought her policy from Company C for the first time.
(i)
Calculate the probability that Yolanda will be covered by Company C for at
least five years before she changes provider.

Zachary took out a policy with Company A in January 2013. Unfortunately,
Zachary’s house burnt down on 12 March 2015.
(ii) Calculate the probability that Company A does NOT cover Zachary’s home at
the time of the fire.

(iii) Calculate the expected proportions of homeowners who are covered by each
insurance company in the long run.

Company A makes an offer to buy Company D. It bases its purchase price on the
assumption that homeowners who would previously have purchased policies from
Company A or Company D would now buy from the combined company, to be called
Addda.
(iv)
(v)
%%  ---  CT4 A2016–8
Determine the transition matrix which will apply after the takeover if
Company A’s assumption about homeowners’ behaviour is correct.
Comment on the appropriateness of Company A’s assumption.


[Total 12]


Q10 (i)
To provide cover for at least five years before she changes provider, Yolanda
must renew her policy at least four times.

The probability of renewing four times is 0.4 4 = 0.0256 (or 16/625).
(ii)

[Total 2]
The company covering the house on 12 March 2015 will be that securing
Zachary’s business at the second renewal.

The second order transition matrix is:
 0 . 5

 0 . 2
 0 . 3

 0

0 . 2
0 . 6
0 . 2
0 . 2
0 . 2
0 . 1
0 . 4
0 . 2
0 . 1   0 . 5
 
0 . 1   0 . 2
*
0 . 1   0 . 3
 
0 . 6     0
0 . 2
0 . 6
0 . 2
0 . 2
0 . 2
0 . 1
0 . 4
0 . 2
0 . 1   0 . 35
 
0 . 1   0 . 25
=
0 . 1   0 . 31
 
0 . 6     0 . 1
0 . 28
0 . 44
0 . 28
0 . 28
0 . 22 0 . 15 

0 . 16 0 . 15 
0 . 26 0 . 15 

0 . 22 0 . 4  

So the probability of being with Company A is 0.35,
and hence the probability of not being with Company A is 0.65.
(iii)
The long run probabilities satisfy
\piP = \pi

0.5\pi A + 0.2\pi B + 0.3\pi C = \pi A (1) 0.2\pi A + 0.6\pi B + 0.2\pi C + 0.2\pi D = \pi B (2) 0.2\pi A + 0.1\pi B + 0.4\pi C + 0.2\pi D = \pi C (3) 0.1\pi A + 0.1\pi B + 0.1\pi C + 0.6\pi D = \pi D (4) 
and \pi A + \pi B + \pi C + \pi D = 1. (5) 
(4) gives (using (5))
0.1\pi A + 0.1\pi B + 0.1\pi C + 0.1\pi D = 0.5\pi D = 0.1,
so
\pi D =
%% --- Page 22

[Total 2]
1
.
5%%%%%%%%%%%%%%%%%%%%%%%%%%%%%%%%%%%%%%%%%%%5(Models Core Technical) – April 2016 – Examiners’ Report
(3)–(2) gives
0.5\pi B + 0.2\pi C = \pi C
so \pi B =
8
\pi C .
5
(1) gives
\pi A =
31
\pi C
25
 31 40 
so  + + 1  \pi C + 0.2 = 1.
 25 25 

Hence \pi A = 31/120, \pi B = 1/3, \pi C = 5/24, and \pi D = 1/5.
So the long run probabilities are 0.2583, 0.3333, 0.2083 and 0.2
for companies A, B, C and D respectively.
(iv)


[Total 4]
The matrix would be:
Addda  0.6 0.2 0.2 


B  0.3 0.6 0.1 
C   0.4 0.2 0.4  
[Total 2]
(v)
There may be reasons customers of Company D do not want to use
Company A. 
Observe that currently the rate of customers going from Company D to
Company A is zero. 
Addda might merge its pricing system. This would change the relative pricing
of an individual’s cover from the different companies. To the extent that
pricing is a driver of the likelihood of customers moving this might change the
probabilities.

To the extent that customer service is a driver, it is not clear what the customer
service of Addda would be relative to Company A or Company D. This might
change the probabilities.

Reduction in competition might encourage a new entrant.

%% --- Page 23%%%%%%%%%%%%%%%%%%%%%%%%%%%%%%%%%%%%%%%%%%%5(Models Core Technical) – April 2016 – Examiners’ Report
It might be a valid assumption that customer behaviour continues unaltered
after the merger.

[Max 2]
[TOTAL 12]
Answers to this question were disappointing, particularly parts (i) and (ii). In
part (i), a very large number of candidates did not spot that five years’
continuous cover with the same provides only requires the decision not to
change to be made four times.
