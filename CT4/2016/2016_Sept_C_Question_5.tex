\documentclass[a4paper,12pt]{article}

%%%%%%%%%%%%%%%%%%%%%%%%%%%%%%%%%%%%%%%%%%%%%%%%%%%%%%%%%%%%%%%%%%%%%%%%%%%%%%%%%%%%%%%%%%%%%%%%%%%%%%%%%%%%%%%%%%%%%%%%%%%%%%%%%%%%%%%%%%%%%%%%%%%%%%%%%%%%%%%%%%%%%%%%%%%%%%%%%%%%%%%%%%%%%%%%%%%%%%%%%%%%%%%%%%%%%%%%%%%%%%%%%%%%%%%%%%%%%%%%%%%%%%%%%%%%

\usepackage{eurosym}
\usepackage{vmargin}
\usepackage{amsmath}
\usepackage{graphics}
\usepackage{epsfig}
\usepackage{enumerate}
\usepackage{multicol}
\usepackage{subfigure}
\usepackage{fancyhdr}
\usepackage{listings}
\usepackage{framed}
\usepackage{graphicx}
\usepackage{amsmath}
\usepackage{chngpage}

%\usepackage{bigints}
\usepackage{vmargin}

% left top textwidth textheight headheight

% headsep footheight footskip

\setmargins{2.0cm}{2.5cm}{16 cm}{22cm}{0.5cm}{0cm}{1cm}{1cm}

\renewcommand{\baselinestretch}{1.3}

\setcounter{MaxMatrixCols}{10}

\begin{document}
5

(i) Describe why the raw data gathered from a mortality investigation need to be

graduated.

[3]

(ii) Explain which method of graduation would be most suitable for each of the

following mortality investigations:

CT4 S2016–3

(a) the female population of a large European country

(b) a study of the mortality of rhinoceroses in the safari parks of South

Africa

(c) the pension scheme of a large company

%%%%%%%%%%%%%%%%%%%%%%%%%%%%%%%%%%%%%%%%%%%%%%%%%%%%%%%%%%%%%%%%%%%%
\newpage


Q5

(i)

We believe that mortality varies smoothly with age (and evidence from large experiences supports this belief).



Therefore the crude estimates of mortality at any age contains information about mortality at adjacent ages and by smoothing the experience we can make use of data at adjacent ages to improve the
estimate at each age.

[1]

This reduces sampling (or random) errors. 

The mortality experience may be used in financial calculations. 

Irregularities, jumps and anomalies in financial quantities (such as premiums for life assurance contracts) are hard to justify to customers.



[Max 3]

(ii)

(a)

Female population of a large European country

By parametric formula,



because the experience is large

OR because the graduated rates may be used to form a new standard table for

the country.



(b)

%% ---  Page  6

Mortality of rhinoceroses in the safari parks of South Africa

Graphical, 

because no suitable table is likely to exist and the experience is small. 

%%%--- Subject CT4 %%%%%%%%%%%%%%%%%%%%%%%%%%%%%%%%%%%%%5 – September 2016 – 

(c)

Members of a pension scheme of a large company

With reference to a standard table,



because there are many suitable tables in existence

OR it will provide help at high ages where data are scarce.



[Total 6]

%%Answers to this question were often weak, especially part (i) which was standard bookwork. In part (ii) many candidates recommended graduation with reference to a standard table for experience (a). While this is possible, the use of a parametric formula is better in this case. It was common also for candidates to recommend using a parametric formula for experience (c).

%Again, while this is possible, making reference to one of the many standard tables available for pensioners would be better in this case. In part (ii), even where the most suitable method of graduation was not chosen, credit was given for sensible reasoning behind the method that was chosen.
\end{document}
