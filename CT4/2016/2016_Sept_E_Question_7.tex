\documentclass[a4paper,12pt]{article}

%%%%%%%%%%%%%%%%%%%%%%%%%%%%%%%%%%%%%%%%%%%%%%%%%%%%%%%%%%%%%%%%%%%%%%%%%%%%%%%%%%%%%%%%%%%%%%%%%%%%%%%%%%%%%%%%%%%%%%%%%%%%%%%%%%%%%%%%%%%%%%%%%%%%%%%%%%%%%%%%%%%%%%%%%%%%%%%%%%%%%%%%%%%%%%%%%%%%%%%%%%%%%%%%%%%%%%%%%%%%%%%%%%%%%%%%%%%%%%%%%%%%%%%%%%%%

\usepackage{eurosym}
\usepackage{vmargin}
\usepackage{amsmath}
\usepackage{graphics}
\usepackage{epsfig}
\usepackage{enumerate}
\usepackage{multicol}
\usepackage{subfigure}
\usepackage{fancyhdr}
\usepackage{listings}
\usepackage{framed}
\usepackage{graphicx}
\usepackage{amsmath}
\usepackage{chngpage}

%\usepackage{bigints}
\usepackage{vmargin}

% left top textwidth textheight headheight

% headsep footheight footskip

\setmargins{2.0cm}{2.5cm}{16 cm}{22cm}{0.5cm}{0cm}{1cm}{1cm}

\renewcommand{\baselinestretch}{1.3}

\setcounter{MaxMatrixCols}{10}

\begin{document}


[Total 8]7

(i)

List EIGHT factors which should be considered when deciding whether a

model is suitable for a particular purpose.

[4]

A colleague has been asked to present a model which might be used to determine the

number of new schools required throughout a country over the next 40 years. He

forgot all about it until the last minute when he was reading an article in a newspaper

about immigration and education which provided some figures to back up the article.

Your colleague has the following suggestion for a model:

x Start with the number of children in the education system over the last twenty

years (as provided by the country’s central statistical office). Project these

forward using a straight line approach.

x Use the number of immigrants predicted to arrive in each of the next five years as

given in the newspaper article. Apply to this an estimate of “number and age of

children for each immigrant” also provided by the newspaper. Project this

forward also using a straight line approach.

x Add the two together to get the total number of children in the education system

for the next 40 years.

(ii)

CT4 S2016–5

Assess whether this model is suitable with regards to SIX of the factors which

you listed in your answer to part (i).

[6]

[Total 10]

%%%%%%%%%%%%%%%%%%%%%%%%%%%%%%%%%%%%%%%%%%%%%%%%%%%%%%%%%%%%%%%%%%%%%%%%%%%%%

Q7

(i)

The objectives of the modelling exercise 

The validity of the model for the purpose to which it is to be put 

The validity of the data to be used 

The validity of the assumptions used 

The possible errors associated with the model OR the fact that the parameters

used are not a perfect representation of the real world situation being modelled 

The impact of correlations between the random variables (or input variables) that

“drive” the model



The extent of correlations between the various results produced from the model 

The current relevance of models written and used in the past 

The credibility of the data input. 

The credibility of the results output 

The dangers of spurious accuracy 

The costs of buying or constructing, and of running the model 

Ease of use and availability of suitable staff to use it 

The risk of the model being used incorrectly or with wrong inputs 

The ease with which the model and its results can be communicated 

Compliance with the relevant regulations 

The existence of clear documentation

6

[Max 4]

(ii)

The objectives of the modelling exercise

The validity of the model for the purpose to which it is to be put

The model is not hugely valid as it does not address the number of schools directly,

for example by dividing the number of pupils by average school size or considering

when existing schools may become obsolete, the presence of competition etc.

[1]

The validity of the data to be used

The Central Statistical Office data will be fine, but that gained from the newspaper

will be of limited validity as estimates of future migrants arriving may be heavily

skewed by the political bias of the newspaper. Estimates of birth rates and migration

rates are generally valid data for this exercise.

[1]

%% ---  Page  9Subject CT4 %%%%%%%%%%%%%%%%%%%%%%%%%%%%%%%%%%%%%5 –September 2016 – 

The validity of the assumptions used

Straight line projection is dubious over 40 years, especially on immigration numbers.

[1]

The possible errors associated with the model OR the fact that the parameters

used are not a perfect representation of the real world situation being modelled

The total number of school children in 40 years’ time is very susceptible to errors in

the parameters, for example the difference between straight line projection following a

baby boom will give a rapidly increasing number, whereas if the baby boom is over,

the numbers may decline.

[1]

The impact of correlations between the random variables that “drive” the model.

It is quite likely that the estimate of new arrivals and the children per household of

new arrivals will be biased in the same direction, i.e. both overstated or understated.

[1]

The extent of correlations between the various results produced from the model.

If you overestimate the number of children in the education system in, say 5 years’

time, you will most likely overestimate the number of children in, say, 30 years’ time

as these latter will be the next generation, the children of those in the system

in 5 years’ time.

[1]

The current relevance of models written and used in the past

The government/local authorities should have models which are still relevant even if

they need parameters adjusting.

[1]

The credibility of the data input

The data from the newspaper may be of doubtful credibility. It would be worth

examining them in the light of past trends to see whether they fall within the range of

past data.

[1]

The credibility of the results output

This model will give a very crude answer which is pretty difficult to have much faith

in. Again, it will be worth examining the output in the light of recently past trends to

see whether they mark a break with the past.

[1]

The dangers of spurious accuracy

There is no point calculating the number of children to many significant figures when

the assumptions are so approximate and the size of individual schools so variable. [1]

The cost of buying or constructing, and of running the model

An advantage of the model is that it is very inexpensive. [1]

Ease of use and availability of suitable staff to use it

The model is very easy to use. [1]

The risk of the model being used incorrectly or with wrong inputs

This is low, as the model is so simple. [1]

%% ---  Page  10Subject CT4 %%%%%%%%%%%%%%%%%%%%%%%%%%%%%%%%%%%%%5 – September 2016 – 

The ease with which the model and its results can be communicated

Another advantage of the model is that it is very simple to communicate.

[1]

Compliance with the relevant regulations

Regulations are unlikely to be applicable in this case. However changes in legislation

concerning immigration might be an issue.

[1]

The existence of clear documentation

It should be easy to produce clear documentation.

[1]

[Max 6]
%%%%%%%%%%%%%%%%%%%%%%%%%%%%%%%%%%%%%%%%%%%%%%%%%%%%%%%%5
\newpage
[Total 10]

Part (i) of this question was well answered, with many candidates scoring full

marks. Only eight factors were required for full credit. The list above gives

the complete range of factors that were awarded marks. Answers to part (ii)

were more variable. The most able candidates produced thoughtful

comments demonstrating engagement with the particular application

described in the question, and a substantial minority of candidates scored

close to full marks. Credit was given for sensible comments other than those

listed above. On the other hand, some candidates wrote only general

comments which added little to part (i).
\end{document}
