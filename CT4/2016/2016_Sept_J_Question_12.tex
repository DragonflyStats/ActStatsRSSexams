PLEASE TURN OVER12
A large life insurance company is conducting an investigation into the mortality of its
policyholders to see if this has changed since the previous investigation ten years ago.
Below is a sample of the results:
Age
55
56
57
58
59
60
61
62
63
64
65
Current investigation
Exposed to risk Observed deaths
5,842
5,630
4,281
3,955
3,879
3,550
4,006
4,150
3,520
3,057
3,666
150
132
126
98
142
149
162
173
158
150
200
Previous investigation
mortality rates
0.0267
0.0278
0.0301
0.0325
0.0356
0.0387
0.0396
0.0410
0.0433
0.0458
0.0490
(i) Explain how many degrees of freedom would be used to conduct a chi-squared
test for goodness of fit on these data.
[2]
(ii) Carry out a chi-squared test on these data.
[5]
(iii) Perform a test to determine whether the shape of the mortality rates has
changed over the age range.
[3]
(iv) Comment on your results to parts (ii) and (iii).
[2]
(v) Suggest reasons why the mortality experience may have changed over the past
ten years.
[2]
[Total 14]
END OF PAPER
CT4 S2016–10

%%%%%%%%%%%%%%%%%%%%%%%%%%%%%%%%%%%%%%%%%%%%%%%%%%%%%%%%%%%%%%%%%%%%%%%%%%%%%%%%%%%%%%%%%%%%%%%%%%%%%%%%%%%%%%%%%%%%%%%%%%%%%%%%%%%5
Q12
(i)
There are 11 age groups:

EITHER
no parameters have been fitted and no table has been chosen
OR
we are not comparing our data with a set of graduated rates derived from our data [1]
so the number of degrees of freedom is 11.
(ii)

The null hypothesis is that the old rates are the true rates underlying the observed data.

Age ETR Deaths Mortality Expected
Deaths z z 2
55
56
57
58
59
60
61
62
63
64
65 5,842
5,630
4,281
3,955
3,879
3,550
4,006
4,150
3,520
3,057
3,666 150
132
126
98
142
149
162
173
158
150
200 0.0267
0.0278
0.0301
0.0325
0.0356
0.0387
0.0396
0.0410
0.0433
0.0458
0.0490 155.981
156.514
128.858
128.538
138.092
137.385
158.638
170.150
152.416
140.011
179.634 −0.479
−1.959
−0.252
−2.694
0.333
0.991
0.267
0.218
0.452
0.844
1.520 0.229
3.840
0.063
7.255
0.111
0.982
0.071
0.048
0.205
0.713
2.309
Total
The observed test statistic is 15.825.
%% ---  Page  20
15.825
[3]Subject CT4 %%%%%%%%%%%%%%%%%%%%%%%%%%%%%%%%%%%%%5 – September 2016 – 
The critical value of the chi squared distribution at the 5% level with 11 degrees of
freedom is 19.68.

(iii)
Since 15.85 < 19.68 
we do not reject the null hypothesis. 
To test whether the shape of the mortality rates has changed over the age range we use
the Grouping of Signs Test.
Under the null hypothesis that the old rates are the true rates underlying
the observed data, 
G = Number of groups of positive deviations = 1,
m = number of deviations = 11,
n 1 = number of positive deviations = 7, and
n 2 = number of negative deviations = 4 
EITHER
We want k * the largest k such that
 n 1 − 1  n 2 + 1 
k 


 t − 1  t 
 m 
t = 1
 
 n 1 

< 0.05 .
The test fails at the 5% level if G \leq k * . 
From the Gold Book the value of k * is 1. 
Since, therefore, G = k * in this case 
we have sufficient evidence to reject the null hypothesis at the 5% level. 
OR
 6  5 
  
0 1
5
Pr[t = 1] =    =
= 0.015152
330
 11 
 
 7 
Pr[t = 0] = 0 [1]
So Pr[t <= 1] < 0.05 
Hence we have evidence to reject the null hypothesis at the 5% level. 
%% ---  Page  21Subject CT4 %%%%%%%%%%%%%%%%%%%%%%%%%%%%%%%%%%%%%5 –September 2016 – 
(iv)
The Chi-Squared Test shows that overall the data are a good fit to the previous
investigation. 
However, the Grouping of Signs Test shows that the shape of the mortality has
changed, 
with the current investigation showing lower mortality at the lower end of the age
range and higher mortality at the higher end.

This change in shape was not picked up by the Chi-Squared test as it uses the square
of the standard deviations, and therefore has no regard for the direction of the
deviations.

(v)
Mortality may have improved/worsened due to changes in medical science, health
care provision, the environment or the standard of living.

This could have affects at all ages, for example when a new “cure all” drug is found,
or over particular age ranges if a procedure or drug is discovered which most notably
affects a certain age range of the population.

The composition of the lives may have changed (e.g. a different mix of nationalities,
or a different weighting between males and females)
[1]
There may be more underwritten lives in one investigation than the other.

One of the investigations could have included a period when an epidemic occurred.
OR There could be an error in one of the investigations.

There could be a general change in lifestyle or diet.

Underwriting practices may have changed for example a new “preferred lives”
category may have been created.
[1]
[Max 2]
[Total 14]
Answers to part (i) of this question were the weakest of any part of the
examination paper. The Core Reading clearly states that “[i]f we are
comparing an experience with a standard table, then [the test statistic] can be
assumed to have a χ 2 distribution with m degrees of freedom [where] m is just
the number of age groups”. It matters not how the previous table was
constructed. In this case, indeed, there is no standard table as such, just the
results of a previous investigation. Provided the current investigation is
independent of the previous investigation, then we do not need to deduct any
degrees of freedom for the “choice of standard table”. In part (iii) many
candidates incorrectly interpreted the question as referring to the “shape of
the distribution of the mortality rates” rather than the shape of the rates
themselves. Many candidates made sensible comments in part (v). Only a
subset of the points mentioned above was required for full credit.
END OF 
%% ---  Page  22
