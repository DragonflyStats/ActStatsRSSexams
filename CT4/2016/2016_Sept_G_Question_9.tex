\documentclass[a4paper,12pt]{article}

%%%%%%%%%%%%%%%%%%%%%%%%%%%%%%%%%%%%%%%%%%%%%%%%%%%%%%%%%%%%%%%%%%%%%%%%%%%%%%%%%%%%%%%%%%%%%%%%%%%%%%%%%%%%%%%%%%%%%%%%%%%%%%%%%%%%%%%%%%%%%%%%%%%%%%%%%%%%%%%%%%%%%%%%%%%%%%%%%%%%%%%%%%%%%%%%%%%%%%%%%%%%%%%%%%%%%%%%%%%%%%%%%%%%%%%%%%%%%%%%%%%%%%%%%%%%

\usepackage{eurosym}
\usepackage{vmargin}
\usepackage{amsmath}
\usepackage{graphics}
\usepackage{epsfig}
\usepackage{enumerate}
\usepackage{multicol}
\usepackage{subfigure}
\usepackage{fancyhdr}
\usepackage{listings}
\usepackage{framed}
\usepackage{graphicx}
\usepackage{amsmath}
\usepackage{chngpage}

%\usepackage{bigints}
\usepackage{vmargin}

% left top textwidth textheight headheight

% headsep footheight footskip

\setmargins{2.0cm}{2.5cm}{16 cm}{22cm}{0.5cm}{0cm}{1cm}{1cm}

\renewcommand{\baselinestretch}{1.3}

\setcounter{MaxMatrixCols}{10}

\begin{document}
[Total 11]9
(i)
Describe how transition rates can be estimated under multiple state models
with constant transition rates, including a statement of the data required. [3]
A specialist insurance policy provides cover only for the theft of valuable items (such
as jewellery) stored in safety deposit boxes in a bank vault. The premium for cover
for an item worth £C is paid in advance. If a claim is made, the cover ceases.
Claims are modelled using a two state model as follows, where \mu is a constant
transition rate:
\mu
Active
policy
Theft
claim
(ii) Give Kolmogorov’s forward equations for this process.
(iii) Determine the expected cost of claims incurred by time T.
[1]
[2]
If the item is no longer stored in the safety deposit box (for example, if the item is
sold) then the insurance cover lapses. The transition rate for lapses of such policies is
a constant O.
(iv) Draw a transition diagram for a revised process allowing for lapses.
(v) Derive the revised expected cost of claims incurred by time T.
CT4 S2016–7
[2]
[3]
[Total 11]

%%%%%%%%%%%%%%%%%%%%%%%%%%%%%%%%%%%%%%%%%%%%%%%%%%%%%%%%%%%%%%%%%%%%%%%%%%%%%%%%%%%%%%%%%%%%%%%%%%%%%%%%%%%%%%%%%%%%%%%%%%%%%
Q9
(i)
The maximum likelihood estimate of the transition intensity from state i to state j is
the number of transitions from state i to state j divided by the total waiting time in
state i.
[1]
To estimate the transition intensities exactly we therefore need
(ii)
the total time spent in each state
OR
entry and exit times for each individual for each state, [1]
and the total number of transitions of each type made. [1]
Define p AA ( s , t ) to be the probability of being in state Active at time s+t if Active at
time s.
Then EITHER
\partial
p AA ( s , t ) = − p AA ( s , t ) \mu
\partial t 
\partial
p AT ( s , t ) = p AA ( s , t ) \mu ,
\partial t 
%% ---  Page  13Subject CT4 %%%%%%%%%%%%%%%%%%%%%%%%%%%%%%%%%%%%%5 –September 2016 – 
OR
\partial
p ( s , t ) = p ( s , t ) M
\partial t 
 −\mu \mu 
where M = 
 in order Active, Theft,
 0 0  
OR
Integrated forward equations:
t
p AA ( s , t ) = exp   −  \mu du  
 u = s

p AT ( s , t ) = 
t
u = 0
(iii)
p AA ( s , u ). \mu .1 du .


Measure from time zero i.e. s = 0 and drop s from notation.
EITHER
\partial
p AA ( t ) = −\mu .
p AA ( t ) \partial t
1
\partial
(ln( p AA ( t ))) = −\mu ,
\partial t

hence p AA ( t ) = exp( −\mu t + C ) .
As p AA (0) = 1 , C = 0, so
p AA ( t ) = exp( −\mu t )

A claim occurs with cost £C if moves to state “Theft Claim”.
Hence the expected cost is C (1 − exp( −\mu T ))
[1]
OR
Solving for p AT , we have
\partial
p AT ( t ) = p AA ( t ) \mu = (1 − p AT ( t )) \mu (as the model has only two states).
\partial t
%% ---  Page  14
Subject CT4 %%%%%%%%%%%%%%%%%%%%%%%%%%%%%%%%%%%%%5 – September 2016 – 
Using an integrating factor, we can write
\partial
[exp( \mu t ) p AT ( t )) = \mu exp( \mu t ) ,
\partial t

exp( \mu t ) p AT ( t ) = exp( \mu t ) − 1 ,
p AT ( t ) = 1 − exp( −\mu t ) ,
and hence the expected cost is C (1 − exp( −\mu T )) .
[1]
OR
Solving the integrated forward equation
P AT ( T ) = 
T
s = 0
T
exp ( −\mu s ) \mu ds = [ − exp( \mu s ) ] 0 = 1 − exp( −\mu T ) ,
and hence the expected cost is C (1 − exp( −\mu T )) .
[1]
[1]
(iv)
\mu
Active
policy
Theft
claim
\lambda
Lapsed
[2]
(v)
We now have
\partial
p AA ( t ) = − p AA ( t )( \mu + \lambda ) .
\partial t

So p AA ( t ) = exp( − ( \mu + \lambda ) t ) .
We want

\partial
p AT ( t ) = p AA ( t ) \mu = \mu exp( − ( \mu + \lambda ) t )) .
\partial t
−\mu
Solving this produces p AT ( t ) =
exp( − ( \mu + \lambda ) t ))
( \mu + \lambda )

T
=
0
\mu
(1 − exp( − ( \mu + \lambda ) T )) .
\mu+\lambda
[1]
%% ---  Page  15Subject CT4 %%%%%%%%%%%%%%%%%%%%%%%%%%%%%%%%%%%%%5 –September 2016 – 
So claims become
\mu
C (1 − exp( − ( \mu + \lambda ) T )) .
\mu+\lambda

[Total 11]
Answers to this question were often weak. In part (i) a substantial minority of
candidates seemed not to have read the part of the question about a
statement of the data required, and so lost marks. Parts (iii) and (v) were
poorly answered, with part (v) not being attempted by many candidates.

\end{document}
