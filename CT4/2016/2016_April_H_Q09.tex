\documentclass[a4paper,12pt]{article}

%%%%%%%%%%%%%%%%%%%%%%%%%%%%%%%%%%%%%%%%%%%%%%%%%%%%%%%%%%%%%%%%%%%%%%%%%%%%%%%%%%%%%%%%%%%%%%%%%%%%%%%%%%%%%%%%%%%%%%%%%%%%%%%%%%%%%%%%%%%%%%%%%%%%%%%%%%%%%%%%%%%%%%%%%%%%%%%%%%%%%%%%%%%%%%%%%%%%%%%%%%%%%%%%%%%%%%%%%%%%%%%%%%%%%%%%%%%%%%%%%%%%%%%%%%%%

\usepackage{eurosym}
\usepackage{vmargin}
\usepackage{amsmath}
\usepackage{graphics}
\usepackage{epsfig}
\usepackage{enumerate}
\usepackage{multicol}
\usepackage{subfigure}
\usepackage{fancyhdr}
\usepackage{listings}
\usepackage{framed}
\usepackage{graphicx}
\usepackage{amsmath}
\usepackage{chng%% --- Page}

%\usepackage{bigints}
\usepackage{vmargin}

% left top textwidth textheight headheight

% headsep footheight footskip

\setmargins{2.0cm}{2.5cm}{16 cm}{22cm}{0.5cm}{0cm}{1cm}{1cm}

\renewcommand{\baselinestretch}{1.3}

\setcounter{MaxMatrixCols}{10}

\begin{document}

%%%%%%%%%%%%%%%%%%%%%%%%%%%%%%%%%%%%%%%%%%%%%%%%%%%%%%%%%%%%%%%%%%%%%%%%%%%%%%%%%%%%


9
Orange trees are susceptible to the disease Citrus Greening. There is no known cure
for this disease and, although trees often survive for some time with the disease, it can
ultimately be fatal.
A researcher decides to model the progression of the disease using a time-
homogeneous continuous-time Markov model with the following state space:
{Healthy (i.e. not infected with Citrus Greening);
Infected with Citrus Greening;
Dead (caused by Citrus Greening);
Dead (other causes)}.
The researcher chooses to label the transition rate parameters as follows:




a mortality rate from the Healthy state, 
a rate of infection with Citrus Greening, 
a total mortality rate from the Infected state, 
a mortality rate caused by Citrus Greening, 
(i) Draw a transition diagram for the chosen model, including the transition rates.

(ii) Determine Kolmogorov’s forward equations governing the transitions,
specifying the generator matrix.

Infected trees display clear symptoms of the disease. This has enabled the researcher
to record the following data on trees in the area of his study:
Tree-months in Healthy State
1,200
Tree-months in Infected State
600
Total number of deaths of trees
40
Number of deaths of Healthy trees
10
Number of deaths from Citrus Greening
30
(iii) Give the likelihood of these data. 
(iv) Derive the maximum likelihood estimator of the mortality rate caused by
Citrus Greening, . 
(v)
%%  ---  CT4 A2016–7
Estimate .

[Total 12]

%%%%%%%%%%%%%%%%%%%%%%%%%%%%%%%%%%%%%%%%%%%%%%%%%%%%%%%%%%%%%%%%%%%%%%%%%%%%%%%

Q9
(i)
σ
Infected
Healthy
\tau
Dead
(Citrus
Greening)
ρ – \tau
μ
Dead
(other
causes)
[Total 2]
(ii)
Forward equations are:
d
P ( t ) = P ( t ) A ,
dt

where A is the generator matrix:
μ 
 −σ − μ σ 0


−ρ \tau ρ − \tau 
0
A = 
,
 0
0 0
0 


0 0
0 
 0
with the order of states being
{Healthy, Infected, Dead (caused by Citrus Greening),
Dead (other causes)}.
%% --- Page 18
[11⁄2]

[Total 3]%%%%%%%%%%%%%%%%%%%%%%%%%%%%%%%%%%%%%%%%%%%5(Models Core Technical) – April 2016 – Examiners’ Report
(iii)
ALTERNATIVE 1
To estimate \tau we re-parameterise so that a new parameter ξ = ρ − \tau is the death
rate from other causes of infected trees.

The likelihood of the data can then be written
L ∝ exp  ( −μ − σ ) ν H  exp  ( −ξ − \tau ) ν I  ( σ ) d




HI
( μ ) d
HDOC
( \tau ) d
IDCG
( ξ ) d
IDOC

where v H and v I are the waiting times in the Healthy and Infected states
respectively,
d HI is the number of transitions from Healthy to Infected,
d HDOC is the number of transitions from Healthy to Dead from Other Causes
d IDCG is the number of transitions from Infected to Dead from Citrus
Greening
and d IDOC is the number of transitions from Infected to Dead from Other
Causes.

OR ALTERNATIVE 2
L ∝ exp  ( −μ − σ ) ν H  exp  −\tau − ( ρ − \tau ) ν I  ( σ ) d




HI
( μ ) d
HDOC
( \tau ) d
IDCG
( ρ − \tau ) d
IDOC

which equals
L ∝ exp  ( −μ − σ ) ν H  exp  −ρν I  ( σ ) d




HI
( μ ) d
HDOC
( \tau ) d
IDCG
( ρ − \tau ) d
IDOC
,

where v H and v I are the waiting times in the Healthy and Infected states
respectively,
d HI is the number of transitions from Healthy to Infected,
d HDOC is the number of transitions from Healthy to Dead from Other Causes
d IDCG is the number of transitions from Infected to Dead from Citrus
Greening
%% --- Page 19
,%%%%%%%%%%%%%%%%%%%%%%%%%%%%%%%%%%%%%%%%%%%5(Models Core Technical) – April 2016 – Examiners’ Report
and d IDOC is the number of transitions from Infected to Dead from Other
Causes.

OR ALTERNATIVE 3
Since we have 40 deaths in total, 10 of healthy trees and 30 from Citrus
Greening, then no infected tree dies from a cause other than Citrus Greening.

Hence ρ = \tau and the likelihood of the data can be written
L ∝ exp [ 1200( −μ − σ ) ] exp [ − 600 \tau ) ] ( σ ) d
HI
( μ ) 10 ( \tau ) 30 ,

where
d HI is the number of transitions from Healthy to Infected.
(iv)

[Max 3]
Taking logarithms of the likelihood we have:
log e L = −\tauν I + d IDCG log e ( \tau ) + terms not dependent on \tau.

Partially differentiating with respect to \tau gives:
d (log e L )
d IDCG
I
= −ν +
.
\tau
d \tau 
Setting the derivative to zero 
we obtain the maximum likelihood estimator:
\tau ˆ =
d IDCG
ν I
.

The second derivative of the log likelihood is
d 2 (log e L )
( d \tau ) 2
=−
d IDCG
( \tau ) 2
,
which is negative, so this is a maximum.
%% --- Page 20

%%%%%%%%%%%%%%%%%%%%%%%%%%%%%%%%%%%%%%%%%%%5(Models Core Technical) – April 2016 – Examiners’ Report
OR ALTERNATIVE 2
Taking logarithms of the likelihood we have:
log e L = d IDCG log e ( \tau ) + d IDOC log e ( ρ − \tau ) + terms not dependent on \tau.

Partially differentiating with respect to \tau gives:
d (log e L ) d IDCG d IDOC
=
−
.
d \tau
\tau
ρ−\tau 
Setting the derivative to zero 
we obtain the maximum likelihood estimator:
\tau ˆ =
ρ d IDCG
d IDOC + d IDCG
.

The second derivative of the log likelihood:
d 2 (log e L )
( d \tau ) 2
=−
d IDCG
( \tau ) 2
−
d IDOC
( ρ − \tau ) 2
,

is negative, therefore this is a maximum.
(v)
30/600 = 0.05 (per tree-month).

[Max 3]
[Total 1]
[TOTAL 12]
Several candidates tried to write the Kolmogorov equations in part (ii) in
component form. Credit was given for this if the resulting equations were
correct. ALTERNATIVE 1 in part (iv) follows from ALTERNATIVES 1 and 3 in
part (iii). ALTERNATIVE 2 in part (iv) follows from ALTERNATIVE 2 in part
(iii). To obtain a numerical estimate of \tau from ALTERNATIVE 2 in part (iv) it
was necessary also to differentiate the logarithm of the likelihood with respect
to ρ, set this derivative to zero, and solve the resulting simultaneous
equations. A few candidates did this, but it was not required for full credit.
Credit was given in part (v) for the correct numerical answer even if this did
not follow from the answers to previous sections.
%% --- Page 21%%%%%%%%%%%%%%%%%%%%%%%%%%%%%%%%%%%%%%%%%%%5(Models Core Technical) – April 2016 – Examiners’ Report
