\documentclass[a4paper,12pt]{article}

%%%%%%%%%%%%%%%%%%%%%%%%%%%%%%%%%%%%%%%%%%%%%%%%%%%%%%%%%%%%%%%%%%%%%%%%%%%%%%%%%%%%%%%%%%%%%%%%%%%%%%%%%%%%%%%%%%%%%%%%%%%%%%%%%%%%%%%%%%%%%%%%%%%%%%%%%%%%%%%%%%%%%%%%%%%%%%%%%%%%%%%%%%%%%%%%%%%%%%%%%%%%%%%%%%%%%%%%%%%%%%%%%%%%%%%%%%%%%%%%%%%%%%%%%%%%

\usepackage{eurosym}
\usepackage{vmargin}
\usepackage{amsmath}
\usepackage{graphics}
\usepackage{epsfig}
\usepackage{enumerate}
\usepackage{multicol}
\usepackage{subfigure}
\usepackage{fancyhdr}
\usepackage{listings}
\usepackage{framed}
\usepackage{graphicx}
\usepackage{amsmath}
\usepackage{chng%% --- Page}

%\usepackage{bigints}
\usepackage{vmargin}

% left top textwidth textheight headheight

% headsep footheight footskip

\setmargins{2.0cm}{2.5cm}{16 cm}{22cm}{0.5cm}{0cm}{1cm}{1cm}

\renewcommand{\baselinestretch}{1.3}

\setcounter{MaxMatrixCols}{10}

\begin{document}

5
(i)
Define the following types of stochastic process:
(a)
(b)
\begin{enumerate}[(i)]
\item a Poisson process
\item a compound Poisson process
\end{enumerate}
%%%%%%%%%%%%%%%%%%%%%%%%%%%

Consider the modelling of the following situations:
A the number of claims for motorcycle accidents received by an insurer’s
telephone claim line
B the number of breakfast bagels sold by a New York bagel bar
C the number of breakdowns of freezers in a large supermarket
D the cost of wasted food caused by breakdowns of freezers in a large
supermarket.
(ii) Comment on which of the following stochastic processes will be most suitable
for modelling each of the four situations above:




time-homogeneous Poisson process
time-inhomogeneous Poisson process
time-homogeneous compound Poisson process
time-inhomogeneous compound Poisson process
[6]
[Total 9]
\newpage
%%%%%%%%%%%%%%%%%%%%%%%%%%%%%%%%%%%%%%%%%%%%%%%%%%%%%%%%%%%%%%%%%%%%%%%%%%%%%%%%%%%%%%%%%%%%%%%%%%%%%%%
Q5
(i)
(a)
ALTERNATIVE 1
A Poisson process is a counting process in continuous time $\{ N t , t \geq 0\}$,
where N t records the number of occurrences of a type of event within
the time interval from 0 to t .

Events occur singly and may occur at any time.

The probability that an event occurs during the short time interval from time t to time t + h is approximately equal to λ h for small h , where the parameter λ is the rate of the Poisson process.
[1⁄2]
OR ALTERNATIVE 2
A Poisson process is an integer valued process in continuous time
{ N t , t ≥ 0} , where:
[1⁄2]
Pr[ N t + h − N t = 1| F t ] = λ h + o ( h ) ,
Pr[ N t + h − N t = 0 | F t ] = 1 − λ h + o ( h ) ,
Pr[ N t + h − N t ≠ 0,1| F t ] = o ( h )
[1⁄2]
%%%%%%%%%%%%%%%%%%%%%%%%%%%%%%%%%%%%%%%%%%%%%%%%%%%%%%%%%%%%%%%%%%%%%%%%%%%%%%%%%%%%%%%%%%%%%%%%%%%%%%%%%%%%%%
OR ALTERNATIVE 3
A Poisson process with rate λ is a continuous-time integer-valued
process N t , t ≥ 0 , with the following properties:
[1⁄2]
N 0 = 0,
N t has independent, Poisson distributed stationary increments
[ λ ( t − s )] n e −λ ( t − s )
,
P [ N t − N s = n ] =
n !
[1]
s < t , n = 0, 1, ...
[1⁄2]
OR ALTERNATIVE 4
{ N t , t ≥ 0} is a Poisson process with rate λ if the holding times T 0 , T 1 , ...
of
{ N t , t ≥ 0} are independent exponential random variables with
parameter λ and
N ( T 0 + T 1 + ... + T n–1 ) = n .
OR ALTERNATIVE 5
{ N t , t ≥ 0} is a Poisson process with rate λ if it is a Markov jump
process with independent increments and transition rates given by:
μ( i , j ) = –λ if j = i ,
μ( i , j ) = λ if j = i + 1,
μ( i , j ) = 0 otherwise.
%%%%%%%%%%%%%%%%%%%%%%%%%%%%%%%%%%%
(b)
Let N t be a Poisson process, t ≥ 0 [1⁄2]
and let Y 1 , Y 2 , ..., Y j , ..., be a sequence of independently and
identically distributed random variables. [1⁄2]
Then a compound Poisson process is defined by
X t =
N t
 Y j , t ≥ 0 .
[1⁄2]
j = 1
[Max 3]
Page 10 %%%%%%%%%%%%%%%%%%%%%%%%%%%%%%%%%%%%%%%%%%%%%%%%%%%%%%%%%%%%%%%%%%%%%%%%%%%%%%%%%%%%%%%%%%%%%%%%%%%%%%
(ii)
(a)
The number of claims for motorcycle accidents received on an
insurer’s telephone claim line
Time inhomogeneous Poisson process
[1⁄2]
Suitable reason, for example because motorcycle accidents are more
likely at certain times of year/of the week and are likely to occur
singly.

(b)
The number of breakfast bagels sold by a New York bagel bar
Time inhomogeneous compound Poisson process
[1⁄2]
Suitable reason, for example because customers wanting breakfast goods are likely to vary according to the time of day, and if customers
arrive following a Poisson process the number sold would follow a compound Poisson process as each customer might buy more than one.
(c)
The number of breakdowns of freezers in a large supermarket
Time homogeneous Poisson process
[1⁄2]
Suitable reason, for example because freezers will need to be left on continuously and no reason to expect failures at a particular time of day. Freezers are likely to break down individually.

(d)
The cost of wasted food caused by breakdowns of freezers in a large supermarket
Time homogeneous compound Poisson process
[1⁄2]
Suitable reason, for example if the number of failures is a time homogeneous Poisson process consistent with previous answer, the
cost of each failure will vary depending on what food is stored in each
freezer, how quickly the freezer is fixed, etc. Hence the cost would
follow a time homogeneous compound Poisson process.
[1]
[Max 6]
[TOTAL 9]
In part (ii) the marks were awarded for any suitable choice provided the explanation supported this. So, for example, in case B credit was given for a time inhomogeneous Poisson process if candidates made the point that this
assumed each customer only bought one bagel. The same process could have been selected for more than one example. However, where no reason was given the mark for the process was only awarded for the models suggested above. In case D a common error was to suppose that, because
the amount of food stored in freezers varied seasonally, the process was time
inhomogeneous. This is incorrect: variation in the average cost of food stored
means that the Y j which are summed in the compound Poisson process are
not identically distributed.

\end{document}
