1
2
State THREE advantages of the two-state model over the Binomial model for
estimating transition intensities where exact dates of entry into and exit from
observation are known.
[3]
The diagrams below show three Markov chains, where arrows indicate a non-zero
transition probability.
A
Markov Chain 1
State 1
State 2
B
State 3
Markov Chain 2
C
State 1 State 2
State 3 State 4
Markov Chain 3
State 1
State 2
State whether each of the chains is:
x
x
irreducible.
periodic, giving the period.
[3]
CT4 S2016–23
Describe the similarities and differences between the following processes:
x Markov Chain
x Markov Jump Chain
x Markov Jump Process
[4]

%%%%%%%%%%%%%%%%%%%%%%%%%%%%%%%%%%%%%%%%%%%%%%%%%%%%Solutions
Q1
We can calculate the Maximum Likelihood Estimate (MLE) of the transition intensities
directly using the two-state model, 
whereas the Binomial model requires additional assumptions. 
The variance of the Binomial estimator is greater than that of the estimate from the two-state
model, though the difference is tiny unless the transition intensities are large.
[1]
The MLE in the two-state model is consistent and unbiased, 
whereas the Binomial estimate is only consistent and unbiased if lives are observed for
exactly one year, which is rarely the case. 
The two-state model is easily extended to encompass increments and additional decrements,
whereas the Binomial model is not.
[1]
The two-state model uses the exact times of the transitions, whereas the Binomial model only
uses the number of transitions.
[1]
[Max 3]
Not all the points listed above were required for full credit. This was one of the
bookwork questions on which performance was relatively weak. Many
candidates were only able to make one or two of the points listed above.
%% ---  Page  3Subject CT4 %%%%%%%%%%%%%%%%%%%%%%%%%%%%%%%%%%%%%5 –September 2016 – 
Q2
A
Markov Chain 1
Irreducible
Aperiodic
B


Markov Chain 2
Irreducible
Periodic with period 2
C


Markov Chain 3
Reducible
Aperiodic


[Total 3]
Many candidates did well on this question. The most common errors were to
regard Markov Chain 1 as periodic (this is incorrect because return to any
state is possible in 2 or 3 steps, and 2 and 3 have no common factor higher
than 1), and to regard Markov Chain 2 as being periodic with period 4 rather
than 2.
Q3
All three processes have a discrete state space.
[1]
A Markov Chain and Markov Jump Chain both operate in discrete time but a Markov jump
Process operates in continuous time.
[1]
All have the Markov property which is

EITHER that the future development of the process can be predicted from its present state
alone, without reference to its past history.
OR that
P[X t ∈ A ⏐ X s 1 = x 1 , X s 2 = x 2 , ..., X s n = x n , X s = x] = P[X t ∈ A ⏐ X s = x]
for all times s 1 < s 2 < ... < s n < s < t, all states x 1 , x 2 , ..., x n , x in S and all subsets A of S.

EITHER If a Markov Jump Process X is examined only at the times of its transitions, the
resulting process is called the Jump Chain associated with X.
OR for a Jump Process X the Jump Chain X shows the states visited by X, taking an identical
path through the state space.

%% ---  Page  4Subject CT4 %%%%%%%%%%%%%%%%%%%%%%%%%%%%%%%%%%%%%5 – September 2016 – 
The Jump Chain obeys the Markov Property and behaves as a Markov Chain except when the
Jump Chain encounters an absorbing state. From that time it makes no further transitions,
implying that time stops for the Jump Chain.
[1]
The Jump Chain associated with X takes the same path through the state space as X does.
However questions about the times taken to visit a state are likely to have different answers
for X and for the Jump Chain associated with X.
[1]
The Markov Jump Chain and the Markov Chain are expressed in terms of probabilities
whereas the Markov Jump Process is expressed in terms of rates.

The Markov Chain can have loops in each state, the Markov Jump process cannot and the
Markov Jump Chain only has loops on absorbing states.

[Max 4]
Not all the above was required for full credit. Many candidates correctly
identified the fact that all three processes operated in discrete state space,
but a large proportion thought that the Markov Jump Chain was a continuous
time process. A Markov Jump Chain is a Markov Chain in its own right, and
hence operates in discrete time.
