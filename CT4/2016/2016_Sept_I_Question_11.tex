\documentclass[a4paper,12pt]{article}

%%%%%%%%%%%%%%%%%%%%%%%%%%%%%%%%%%%%%%%%%%%%%%%%%%%%%%%%%%%%%%%%%%%%%%%%%%%%%%%%%%%%%%%%%%%%%%%%%%%%%%%%%%%%%%%%%%%%%%%%%%%%%%%%%%%%%%%%%%%%%%%%%%%%%%%%%%%%%%%%%%%%%%%%%%%%%%%%%%%%%%%%%%%%%%%%%%%%%%%%%%%%%%%%%%%%%%%%%%%%%%%%%%%%%%%%%%%%%%%%%%%%%%%%%%%%

\usepackage{eurosym}
\usepackage{vmargin}
\usepackage{amsmath}
\usepackage{graphics}
\usepackage{epsfig}
\usepackage{enumerate}
\usepackage{multicol}
\usepackage{subfigure}
\usepackage{fancyhdr}
\usepackage{listings}
\usepackage{framed}
\usepackage{graphicx}
\usepackage{amsmath}
\usepackage{chngpage}

%\usepackage{bigints}
\usepackage{vmargin}

% left top textwidth textheight headheight

% headsep footheight footskip

\setmargins{2.0cm}{2.5cm}{16 cm}{22cm}{0.5cm}{0cm}{1cm}{1cm}

\renewcommand{\baselinestretch}{1.3}

\setcounter{MaxMatrixCols}{10}

\begin{document}


[Total 13]11
An individual’s marginal tax rate depends upon his or her total income during a
calendar year and may be 0% (that is, he or she is a non-taxpayer), 20% or 40%.
The movement in the individual’s marginal tax rate from year to year is believed to
follow a Markov Chain with a transition matrix as follows:
0% § 1  E  E 2
E
E 2 ·
 ̈
 ̧
20%  ̈
E
1  3 E
2 E  ̧
 ̧
40%  ̈ E 2
E
1  E  E 2 1
©

\begin{enumerate}
    \item (i) Draw the transition diagram of the process, including the transition rates.
\item (ii) Determine the range of values of E for which this is a valid transition matrix.
[3]
\item (iii) Explain whether the chain is:
(a)
(b)
[2]
irreducible.
periodic.
including whether this depends on the value of E.
[2]
The value of E has been estimated as 0.1.
\item (iv)
Calculate the long term proportion of taxpayers at each marginal rate.
[4]
Lucy pays tax at a marginal rate of 20% in 2011.
\item (v)
Calculate the probabilities that Lucy’s marginal tax rate in 2013 is:
(a)
(b)
(c)
0%.
20%.
40%.

\end{enumerate}



%%%%%%%%%%%%%%%%%%%%%%%%%%%%%%%%%%%%%%%%%%%%%%%%%%%%%%%%%%%%%%%%%%%%%%%%%


Q11
(i)
1 − 3\beta
20%
\beta
\beta
2\beta
\beta
1 − \beta −
\beta 2
0%
\beta 2
40%
1 − \beta − \beta 2
\beta 2
[2]
(ii)
We require each row of the transition matrix to sum to 1. 
Here this holds for all values of \beta . 
We require each of the following to lie between 0 and 1 inclusive:
\beta , \beta 2 , 2 \beta ,1 − 3 \beta ,1 − \beta − \beta 2 .
The first two require that 0 \leq \beta \leq 1 . 
The third requires that 0 \leq \beta \leq 1/ 2 . 
1
The fourth that \beta \leq , \beta ≥ 0 .
3 
The fifth implies \beta \leq
5 − 1
as the negative root is not viable.
2
1
So overall 0 \leq \beta \leq .
3
[1]

[Max 3]
(iii)
If \beta > 0 then it can reach any other state (so it is irreducible) [1]
and it has a loop on each state (so it is aperiodic). 
However if \beta = 0 it can never leave its current state so it is reducible. 
%% ---  Page  18Subject CT4 %%%%%%%%%%%%%%%%%%%%%%%%%%%%%%%%%%%%%5 – September 2016 – 
(iv)
 0.89 0.1 0.01 


In this case the matrix is P =  0.1 0.7 0.2 
 0.01 0.1 0.89 

 
The stationary distribution satisfies π = π P . 
We have:
π 0 = 0.89 π 0 + 0.1 π 20 + 0.01 π 40
π 20 = 0.1 π 0 + 0.7 π 20 + 0.1 π 40

π 40 = 0.01 π 0 + 0.2 π 20 + 0.89 π 40
and
π 0 + π 20 + π 40 = 1 .

The first and third equations give
0.11 π 40 − 0.22 π 0 = 0.01 π 0 − 0.02 π 40
0.23
π 0 = 1.769 π 0
0.13
0.1 + 0.1769
=
π 0 = 0.923 π 0 ,
0.3
π 40 =
π 20
so (1 + 0.923 + 1.769) π 0 = 1 , and hence
[1]
13
48
1
= 0.25 =
4
23
= 0.479 =
48
π 0 = 0.271 =
π 20
π 40
are the long term proportion of taxpayers at each marginal rate.
(v)
[1]
Looking for the rates two years’ later, these are given by P 2 , which is
 0.89 0.1 0.01   0.89 0.1 0.01   0.8022 0.16 0.0378 

 
 

 0.1 0.7 0.2  .  0.1 0.7 0.2  =  0.161 0.52 0.319 
 0.01 0.1 0.89   0.01 0.1 0.89   0.0278 0.16 0.8122 

 
 

%% ---  Page  19Subject CT4 %%%%%%%%%%%%%%%%%%%%%%%%%%%%%%%%%%%%%5 –September 2016 – 
So the required probabilities are:
(a)
(b)
(c)
0.161
0.52
0.319.
[2]
[Total 13]

% This question was generally well answered, except for part (ii). In part (ii) a large number of candidates considered that \beta could not take the values 0 and 1, even though it was a probability. In part (ii) candidates who gave the correct range of 0 \leq \beta \leq 1/3 scored 2 marks. The other mark was for reasoning leading to the correct range. Candidates who gave an incorrect range could score partial credit for sensible reasoning. The most common error in part (iv) was to base the answer on P 3 rather than P 2 .
\end{document}
