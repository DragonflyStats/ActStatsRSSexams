\documentclass[a4paper,12pt]{article}

%%%%%%%%%%%%%%%%%%%%%%%%%%%%%%%%%%%%%%%%%%%%%%%%%%%%%%%%%%%%%%%%%%%%%%%%%%%%%%%%%%%%%%%%%%%%%%%%%%%%%%%%%%%%%%%%%%%%%%%%%%%%%%%%%%%%%%%%%%%%%%%%%%%%%%%%%%%%%%%%%%%%%%%%%%%%%%%%%%%%%%%%%%%%%%%%%%%%%%%%%%%%%%%%%%%%%%%%%%%%%%%%%%%%%%%%%%%%%%%%%%%%%%%%%%%%

\usepackage{eurosym}
\usepackage{vmargin}
\usepackage{amsmath}
\usepackage{graphics}
\usepackage{epsfig}
\usepackage{enumerate}
\usepackage{multicol}
\usepackage{subfigure}
\usepackage{fancyhdr}
\usepackage{listings}
\usepackage{framed}
\usepackage{graphicx}
\usepackage{amsmath}
\usepackage{chng%% --- Page}

%\usepackage{bigints}
\usepackage{vmargin}

% left top textwidth textheight headheight

% headsep footheight footskip

\setmargins{2.0cm}{2.5cm}{16 cm}{22cm}{0.5cm}{0cm}{1cm}{1cm}

\renewcommand{\baselinestretch}{1.3}

\setcounter{MaxMatrixCols}{10}

\begin{document}


3 (i)
State the principle of correspondence in the context of a mortality
investigation.

A mortality investigation collects the following data:
n x (t) = total number of policies under which death claims are made when the
policyholder is aged x last birthday for each calendar year t.
P x (t) = number of in-force policies where the policyholder was aged x nearest
birthday on 1 January in year t.
(ii)
(a)
Derive an expression, in terms of P x (t), for the central exposed to risk,
E x c , corresponding to the claims data which may be used to estimate
the force of mortality in year t at each age x,  x .
(b)
4
State any assumptions that you make, indicating at which point in your
derivation each assumption is relevant.













Q3
(i) A life alive at time t should be included in the exposure at age x at time t if and
only if, were that life to die immediately, he or she would be counted in the
death data d x at age x.
[Total 1]
(ii) P x (t) is the number of policies under observation aged x nearest birthday on
1 January in year t.
To correspond with the claims data, we wish to have policies classified by age
last birthday.

Define the number of policies aged x last birthday on 1 January in year t to be
P x ′ ( t ).

Assuming that birthdays are evenly distributed over time,
P x ′ ( t ) =
1
[ P x ( t ) + P x + 1 ( t )] .
2


The central exposed to risk is given by
E x c =
1
 0 P x ′ ( t ) dt .

%% --- Page 5%%%%%%%%%%%%%%%%%%%%%%%%%%%%%%%%%%%%%%%%%%%5(Models Core Technical) – April 2016 – Examiners’ Report
Assuming that the population varies linearly between census dates, 
1
E x c ≈ [ P x ′ ( t ) + P x ′ ( t + 1)] .
2 
Substituting for the P x ′ ( t ) in terms of P x (t) from the equation above gives
1  1
1

E x c ≈  [ P x ( t ) + P x + 1 ( t )] + [ P x ( t + 1) + P x + 1 ( t + 1)]  .
2  2
2


[Total 5]
[TOTAL 6]
This was a straightforward exposed to risk question and was generally well
answered. The most common reasons that candidates lost marks in part (ii)
were the use of the incorrect age adjustment
P x ′ ( t ) =
1
[ P x ( t ) + P x − 1 ( t )] ,
2
and a failure to point out where in the argument the assumptions were
required.
