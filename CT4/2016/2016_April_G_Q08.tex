\documentclass[a4paper,12pt]{article}

%%%%%%%%%%%%%%%%%%%%%%%%%%%%%%%%%%%%%%%%%%%%%%%%%%%%%%%%%%%%%%%%%%%%%%%%%%%%%%%%%%%%%%%%%%%%%%%%%%%%%%%%%%%%%%%%%%%%%%%%%%%%%%%%%%%%%%%%%%%%%%%%%%%%%%%%%%%%%%%%%%%%%%%%%%%%%%%%%%%%%%%%%%%%%%%%%%%%%%%%%%%%%%%%%%%%%%%%%%%%%%%%%%%%%%%%%%%%%%%%%%%%%%%%%%%%

\usepackage{eurosym}
\usepackage{vmargin}
\usepackage{amsmath}
\usepackage{graphics}
\usepackage{epsfig}
\usepackage{enumerate}
\usepackage{multicol}
\usepackage{subfigure}
\usepackage{fancyhdr}
\usepackage{listings}
\usepackage{framed}
\usepackage{graphicx}
\usepackage{amsmath}
\usepackage{chng%% --- Page}

%\usepackage{bigints}
\usepackage{vmargin}

% left top textwidth textheight headheight

% headsep footheight footskip

\setmargins{2.0cm}{2.5cm}{16 cm}{22cm}{0.5cm}{0cm}{1cm}{1cm}

\renewcommand{\baselinestretch}{1.3}

\setcounter{MaxMatrixCols}{10}

\begin{document}

%%%%%%%%%%%%%%%%%%%%%%%%%%%%%%%%%%%%%%%%%%%%%%%%%%%%%%%%%%%%%%%%%%%%%%%%%%%%%%%%%%%%


PLEASE TURN OVER8
(i) List THREE different methods of graduating crude mortality rates. 
(ii) State the advantages of each method listed in part (i). 
A life insurance company has graduated its own mortality experience for term
assurance business over the past 15 years against a standard table using the following
equation:
q x  0.94 q x s  0.0001
where q x s is the mortality rate from the standard table.
The following is an extract from the data.
(iii)
%%  ---  CT4 A2016–6
Age x Exposed to Risk Deaths Graduated Rate
40
41
42
43
44
45
46
47
48
49
50 24,584
32,587
15,784
21,336
25,874
21,544
23,967
25,811
26,911
28,445
30,205 14
32
16
22
24
22
25
30
28
38
45 0.000625
0.000683
0.000748
0.000823
0.000908
0.001005
0.001114
0.001239
0.001378
0.001536
0.001713
Carry out a test for overall goodness of fit of the data, using a 95%
significance level.
[6]
[Total 12]

Q8
(i)
Graduation by parametric formula.
Graduation by reference to a standard table.
Graphical graduation.
[Total 2]
(ii)
Parametric formula
The resultant graduation will be sufficiently smooth provided few parameters
are used.

It is a suitable method for producing standard tables.

It can be useful to fit the same formula to several experiences to give insight
into the differences between experiences.

Reference to a standard table
It can be used to fit relatively small data sets where a suitable standard table
exists.

The graduated rates should be smooth provided that a simple function is used.

The standard table can provide information at extreme ages where data may be
scanty.

It can be useful to fit the same table to several experiences with the same link
function to give insight into how the experience changes over time.

Graphical graduation
It can be used for scanty data sets where no more sophisticated method is
justifiable.

It enables an experienced analyst to allow for known (or likely) features of the
data.

It can give a quick initial feel for the rates.
(iii)
To test for the overall goodness of fit use the χ 2 test.
The null hypothesis is that the graduated rates are the same as the true
underlying rates in the block of business.
%% --- Page 16

[Max 4]
%%%%%%%%%%%%%%%%%%%%%%%%%%%%%%%%%%%%%%%%%%%5(Models Core Technical) – April 2016 – Examiners’ Report
The test statistic
 z x 2 ≈ χ 2 m where m is the degrees of freedom.

x
Age Exposed to
Risk Observed
Deaths Graduated
Rates Expected
Deaths z x z x2
40
41
42
43
44
45
46
47
48
49
50 24,584
32,587
15,784
21,336
25,874
21,544
23,967
25,811
26,911
28,445
30,205 14
32
16
22
24
22
25
30
28
38
45 0.000625
0.000683
0.000748
0.000823
0.000908
0.001005
0.001114
0.001239
0.001378
0.001536
0.001713 15.37
22.26
11.81
17.56
23.49
21.65
26.70
31.98
37.08
43.69
51.74 − 0.34823
2.06521
1.22046
1.05968
0.10448
0.07485
− 0.32866
− 0.35010
− 1.49162
− 0.86105
− 0.93717 0.12126
4.26508
1.48953
1.12291
0.01092
0.00560
0.10815
0.12257
2.22492
0.74141
0.87828
Total
11.09063
[11⁄2]
The observed test statistic is 11.09. 
The number of age groups is 11, 
but we lose an unknown number of degrees
of freedom for the choice of the standard table, say 2, 
and a further two for the parameters in the link function. 
So m = 7 say (8 or 6 also acceptable). 
The critical value of the χ 2 distribution with 7 degrees of freedom at the 95%
significance level is 14.07 (6 d.f.12.59, 8 d.f. 15.51).

Since 11.09 < 14.07 (or 12.59, or 15.51)
We have insufficient evidence to reject the null hypothesis.


[Max 6]
[TOTAL 12]
Almost all candidates knew three methods of graduation. Part (ii) was poorly
answered, with a substantial minority of candidates simply describing the
three methods rather than their advantages. In part (iii) a common error was
to consider that the link function only involved one parameter, where there
were two (0.94 and 0.0001). However, the weakest element of candidates’
answers was the description of the null hypothesis, with many candidates
%% --- Page 17%%%%%%%%%%%%%%%%%%%%%%%%%%%%%%%%%%%%%%%%%%%5(Models Core Technical) – April 2016 – Examiners’ Report
writing incorrect formulations, such as “the crude rates were equal to the
graduated rates”, or that “the graduated rates were the same as the rates in
the standard table”.
