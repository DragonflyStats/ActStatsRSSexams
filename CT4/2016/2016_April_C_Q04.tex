\documentclass[a4paper,12pt]{article}

%%%%%%%%%%%%%%%%%%%%%%%%%%%%%%%%%%%%%%%%%%%%%%%%%%%%%%%%%%%%%%%%%%%%%%%%%%%%%%%%%%%%%%%%%%%%%%%%%%%%%%%%%%%%%%%%%%%%%%%%%%%%%%%%%%%%%%%%%%%%%%%%%%%%%%%%%%%%%%%%%%%%%%%%%%%%%%%%%%%%%%%%%%%%%%%%%%%%%%%%%%%%%%%%%%%%%%%%%%%%%%%%%%%%%%%%%%%%%%%%%%%%%%%%%%%%

\usepackage{eurosym}
\usepackage{vmargin}
\usepackage{amsmath}
\usepackage{graphics}
\usepackage{epsfig}
\usepackage{enumerate}
\usepackage{multicol}
\usepackage{subfigure}
\usepackage{fancyhdr}
\usepackage{listings}
\usepackage{framed}
\usepackage{graphicx}
\usepackage{amsmath}
\usepackage{chng%% --- Page}

%\usepackage{bigints}
\usepackage{vmargin}

% left top textwidth textheight headheight

% headsep footheight footskip

\setmargins{2.0cm}{2.5cm}{16 cm}{22cm}{0.5cm}{0cm}{1cm}{1cm}

\renewcommand{\baselinestretch}{1.3}

\setcounter{MaxMatrixCols}{10}

\begin{document}

The manager of a life insurance company wishes to revise the premiums for term
assurance policies. He has asked a trainee to compare the latest mortality estimates
from the Continuous Mortality Investigation (CMI) for ages 40–64 years inclusive
with the estimates the company has been using in its premium calculations, using a
95% significance level.
The trainee says: “I've done the Signs Test and we just pass – one more positive sign
and we would have failed!”.
(i)
Calculate the number of ages for which the company's mortality estimate was
higher than that produced by the CMI.
[3]
Ten minutes later the trainee says: “ I tried the Grouping of Signs test and we just fail.
We needed one more positive run!”.
(ii)
CT4 A2016–2
Determine the number of runs of positive signs in the company’s data.
[3]
[Total 6]

%%%%%%%%%%%%%%%%%%%%%%%%%%%%%%%%%%%%%%%%%%%%%%%%%%%%%%%%%%%%%%%%%%%%%%%%%%%%%%%%%
\newpage

Q4
(i)
The null hypothesis is that the company’s schedule and that of the Continuous
Mortality Investigation (CMI) are the same.

THEN ALTERNATIVE 1
With 25 ages we can use the Normal approximation.
If p is the number of positive signs, a z-score can be computed as
z =
p − 12.5
.
6.25
Hence
p = 12.5 + ( 6.25) z .

We reject the null hypothesis if the number of positive signs is too few OR too
many such that z > 1.96 .

This is true if p > 17.4 or if p < 7.6.
%% --- Page 6
%%%%%%%%%%%%%%%%%%%%%%%%%%%%%%%%%%%%%%%%%%%5(Models Core Technical) – April 2016 – Examiners’ Report
Since we are told by the trainee that one more positive sign would lead to
“failure” (i.e. rejection of the null hypothesis)

we must have 17 positive signs.

OR ALTERNATIVE 2
Define k* as the smallest value of k such that
k
 25  25
 0.5 ≥ 0.025 .
j 
j = 0
  
We have
7
 25  25
 0.5 = 0.000000 + 0.000001 + 0.000009 + 0.000069
j

j = 0
  
+ 0.000377 + 0.001583 + 0.005278 + 0.014326 = 0.021643
and
8
 25  25 7  25  25
   j   0.5 =    j   0.5 + 0.032233 = 0.053876 .
j = 0
j = 0

We reject the null hypothesis if the number of positive signs is too few or too
many.

k* = 8, so we reject the null hypothesis if we have 7 or fewer positive signs, or
if we have 18 or more positive signs.

Since we are told by the trainee that one more positive sign would lead to
“failure” (i.e. rejection of the null hypothesis),

we must have 17 positive signs.

[Max 3]
(ii)
With 17 positive signs we have 8 negative signs.

The null hypothesis is that the company’s schedule and that of the CMI are the
same

THEN ALTERNATIVE 1
Using the table on p. 189 of the Golden Book,
the critical value is 3.


The critical value gives the highest number of runs which is incompatible with
the null hypothesis at the 95% level.

%% --- Page 7%%%%%%%%%%%%%%%%%%%%%%%%%%%%%%%%%%%%%%%%%%%5(Models Core Technical) – April 2016 – Examiners’ Report
We reject the null hypothesis with 3 or fewer runs of positive signs.

Since the trainee considered that with one more positive run we would not
have rejected the null hypothesis, there must be 3 runs of positive signs. 
OR ALTERNATIVE 2
Using the formula we have
 16   9 
  
2 3
(120)(84)
for 3 positive runs,     =
= 0.0093 << 0.05,
1, 081,575
 25 
 
 17  
 16   9 
  
3 4
(560)(126)
and for 4 positive runs     =
= 0.065 > 0.05.
1, 081,575
 25 
 
 17  
At the 95% level we reject the null hypothesis with 3 or fewer runs of positive
signs but do not reject it with 4 or more runs.

Since the trainee considered that with one more positive run we would not
have rejected the null hypothesis, there must be 3 runs of positive signs. 
OR ALTERNATIVE 3
Using the Normal approximation we have
 17(9) [(17)(8)] 2 
,
Number of positive runs ~Normal  
 , which is Normal(6.12,
25 3  
 25
1.18).

z-scores for smaller numbers of positive runs are:
5 positive runs: 5 − 6.12
= − 1.031 ,
1.18
4 positive runs: 4 − 6.12
= − 1.952 , and
1.18
3 positive runs: 3 − 6.12
= − 2.872 .
1.18
Using a one-tailed test, we are looking for z = − 1.645 at the 95% level.
%% --- Page 8

%%%%%%%%%%%%%%%%%%%%%%%%%%%%%%%%%%%%%%%%%%%5(Models Core Technical) – April 2016 – Examiners’ Report
We reject the null hypothesis with 4 or fewer runs of positive signs.

Since the trainee considered that with one more positive run we would not
have rejected the null hypothesis, there must be 4 runs of positive signs. 
[Max 3]
[TOTAL 6]
This question tested a standard application using an unfamiliar context. In
part (i) candidates divided into those who used their conceptual
understanding to work out what was needed, and who scored close to full
marks; and candidates who were unable to make much of an attempt. A
common error was to use the “wrong end” of the distribution to produce the
answer 7 positive runs. Most candidates who obtained a numerical answer to
part (i) were able to use this to answer part (ii). Where candidates had an
incorrect answer to part (i) full credit could still be gained for part (ii) if the
answer to part (i) was followed through correctly.
