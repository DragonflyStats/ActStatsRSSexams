\documentclass[a4paper,12pt]{article}

%%%%%%%%%%%%%%%%%%%%%%%%%%%%%%%%%%%%%%%%%%%%%%%%%%%%%%%%%%%%%%%%%%%%%%%%%%%%%%%%%%%%%%%%%%%%%%%%%%%%%%%%%%%%%%%%%%%%%%%%%%%%%%%%%%%%%%%%%%%%%%%%%%%%%%%%%%%%%%%%%%%%%%%%%%%%%%%%%%%%%%%%%%%%%%%%%%%%%%%%%%%%%%%%%%%%%%%%%%%%%%%%%%%%%%%%%%%%%%%%%%%%%%%%%%%%

\usepackage{eurosym}
\usepackage{vmargin}
\usepackage{amsmath}
\usepackage{graphics}
\usepackage{epsfig}
\usepackage{enumerate}
\usepackage{multicol}
\usepackage{subfigure}
\usepackage{fancyhdr}
\usepackage{listings}
\usepackage{framed}
\usepackage{graphicx}
\usepackage{amsmath}
\usepackage{chngpage}

%\usepackage{bigints}
\usepackage{vmargin}

% left top textwidth textheight headheight

% headsep footheight footskip

\setmargins{2.0cm}{2.5cm}{16 cm}{22cm}{0.5cm}{0cm}{1cm}{1cm}

\renewcommand{\baselinestretch}{1.3}

\setcounter{MaxMatrixCols}{10}

\begin{document}

11
An energy provider is worried about the number of its customers who transfer to other
companies within the first two years of their contract and is trying to direct its
advertising towards the most loyal section of the population.
The company has looked at its records over recent years and has fitted a Cox
proportional hazards model to those who have transferred within the first two years
using the factors which appear to have the most impact on early transfer rates.
The following figures have been derived from the data:
Factor
Gender
Volume of energy
consumed
Area of Residence
Male
Female
High
Low
City Centre
City (not centre)
Rural
Parameter
Estimate
0.25
0
0.32
0
0.19
0
0.35
Variance
0.015
0
0.008
0
0.012
0
0.005
(i) Give the hazard function for this Cox proportional hazard model defining all
the terms and covariates.

(ii) State the features of the person to whom the baseline hazard applies.
(iii) Calculate symmetric 95% confidence intervals for the parameters based on the
standard errors.

(iv) Test the suggestion that women change energy providers more frequently than
men.


There is a 70% probability that a male customer who is a low consumer of energy and
lives in a rural area has transferred providers before the end of two years.
(v) Calculate the probability that a male customer who is a high consumer of
energy and lives in a city centre remains with the company for at least two
years.

(vi) Set out how you would determine whether the effect of any of the factors
depends upon any of the other factors.

[Total 17]
END OF PAPER
%%  ---  CT4 A2016–9



Q11 (i)
h ( t , z i ) = h 0 ( t ) exp ( \beta_ 1 z 1 + \beta_ 2 z 2 + \beta_ 3 z 3 + \beta_ 4 z 4 ) , where 
h(t, z i ) is the hazard at time t; 
h 0 (t) is the baseline hazard; 
\beta_ 1 ... \beta_ 4 are regression parameters; 
z 1 is a covariate which takes the value 1 if the client is Male, 0 otherwise;
z 2 is a covariate which takes the value 1 if energy consumption is high, 0
otherwise;
z 3 is a covariate which takes the value 1 if the area of residence is City Centre,
0 otherwise; and
z 4 is a covariate which takes the value 1 if the area of residence is Rural, 0
otherwise.

[Max 3]
(ii) The baseline hazard refers to a female with low energy consumption who lives
in a city but not in the city centre.
[Total 1]
(iii) The 95% confidence interval for \beta_ is \beta_ ± 1.96 Var( \beta_ ) so the intervals are:
Male
Female
High consumption
Low consumption
City Centre
City (not centre)
Rural
−0.4900, −0.0100
0
0.1447, 0.4953
0
−0.0247, 0.4047
0
−0.4886, −0.2114
[1⁄2 for each correct interval]
%% --- Page 24
[Max 2]%%%%%%%%%%%%%%%%%%%%%%%%%%%%%%%%%%%%%%%%%%%5(Models Core Technical) – April 2016 – Examiners’ Report
(iv)
The parameter associated with males is −0.25 so for two otherwise identical
clients, the transfer rates for males is exp (−0.25) = 0.7788
OR
The hazard ratio between the transfer rates for women and men is
1/exp(−0.25) =1.284.

Therefore women do seem to transfer more than men (or men less than
women).

The 95% confidence interval for the parameter does not span zero
OR
the z-score for the parameter is 0.25/√0.015 = 2.04, and this is greater than
1.96.

So at the 95% confidence level we can state that women do switch providers
more frequently than men.

[Total 3]
(v)
ALTERNATIVE 1
For the rural male, the probability that he has transferred is 0.7, the sum of the

parameters is −0.25 + 0 − 0.35 = –0.6 and the hazard is h 0 ( t ) exp( − 0.6) .
So the probability that the contract is still in force is
2
  2
 
 
 
0.3 = exp  −  h 0 ( t ) exp( − 0.6) dt  = exp  − 0.5488  h 0 ( t ) dt  .
  0
 
 
0
 
2
So
ln 0.3
 h 0 ( t ) dt = 0.5488


0
For the City Centre male, the sum of the parameters is 0.26. 
  2
 
ln 0.3 

So we want exp  −  h 0 ( t ) exp(0.26) dt  = exp  − 1.2969*

0.5488 

  0
  
= 0.058124. 
OR ALTERNATIVE 2
{ 0.3 }
e 0.6
e 0.26
= 0.058124,
= 0.3 e
0.86
[21⁄2]

%% --- Page 25%%%%%%%%%%%%%%%%%%%%%%%%%%%%%%%%%%%%%%%%%%%5(Models Core Technical) – April 2016 – Examiners’ Report
which is the probability that he is still with the company.
[Max 3]
(vi)
For each pair of covariates z i and z j :

fit a model with the original covariates plus the interaction between the pair as
an extra covariate.
OR
fit a model with the original covariates and a term z i * z j .

If the log-likelihood for each of the models are L original and L with interaction , 
then the test statistic is −2(L original − L with interaction ). 
The null hypothesis is that the parameter for the interaction term is zero. 
The test statistic has a chi-squared distribution with one degree of freedom. 
If the test statistic is greater than 3.84 (at the 5% level of significance)
OR
If the 95% confidence interval around the interaction parameter does not
include zero,

we can reject the null hypothesis and conclude that the interaction term is
needed.

[Max 5]
[TOTAL 17]
Most candidates successfully wrote down the equation in part (i) and defined
the covariates. Most candidates also correctly identified the characteristics of
the person whose hazard was equal to the baseline in part (ii). Common
errors in part (iii) were failure to multiply by 1.96 or to take the square root of
the variance. Parts (iv) and (v) were disappointingly answered. In part (vi)
several candidates knew that a likelihood ratio test was required but were
rather vague about the details. These candidates were given limited credit.
END OF EXAMINERS’ REPORT
%% --- Page 26
