\documentclass[a4paper,12pt]{article}

%%%%%%%%%%%%%%%%%%%%%%%%%%%%%%%%%%%%%%%%%%%%%%%%%%%%%%%%%%%%%%%%%%%%%%%%%%%%%%%%%%%%%%%%%%%%%%%%%%%%%%%%%%%%%%%%%%%%%%%%%%%%%%%%%%%%%%%%%%%%%%%%%%%%%%%%%%%%%%%%%%%%%%%%%%%%%%%%%%%%%%%%%%%%%%%%%%%%%%%%%%%%%%%%%%%%%%%%%%%%%%%%%%%%%%%%%%%%%%%%%%%%%%%%%%%%

\usepackage{eurosym}
\usepackage{vmargin}
\usepackage{amsmath}
\usepackage{graphics}
\usepackage{epsfig}
\usepackage{enumerate}
\usepackage{multicol}
\usepackage{subfigure}
\usepackage{fancyhdr}
\usepackage{listings}
\usepackage{framed}
\usepackage{graphicx}
\usepackage{amsmath}
\usepackage{chngpage}

%\usepackage{bigints}
\usepackage{vmargin}

% left top textwidth textheight headheight

% headsep footheight footskip

\setmargins{2.0cm}{2.5cm}{16 cm}{22cm}{0.5cm}{0cm}{1cm}{1cm}

\renewcommand{\baselinestretch}{1.3}

\setcounter{MaxMatrixCols}{10}

\begin{document}
\begin{enumerate}


\item %% - Question 3

Describe the problems which can arise with subdividing data.


[Total 4]
A graduation of a set of crude mortality rates is tested for goodness-of-fit using a
chi-squared test.
Discuss the factors to be considered in determining the number of degrees of freedom
to use for the test statistic.

\end{enumerate}
\newpage
%%%%%%%%%%%%%%%%%%%%%%%%%%%%%%%%%%%%%%%%%%%%%%%%%%%%

\newpage
3
To work out the number of degrees of freedom (d.f.) we start with the number of age groups.
We reduce the d.f. because of the constraints imposed by the graduation process.
The reduction varies according to the graduation method:
parametric formula – one d.f. lost for each parameter estimated;
standard table – one d.f. lost for each parameter fitted and a further reduction due to the constraints imposed by the choice of standard table;
graphical – two or three d.f. lost for about every 10 ages graduated.
This question was generally well answered. Common errors were to suppose that only one d.f. is lost for the choice of standard table, and that for graphical graduation, two or three
d.f. were lost in total, regardless of the number of ages being graduated.

\end{document}
