\documentclass[a4paper,12pt]{article}

%%%%%%%%%%%%%%%%%%%%%%%%%%%%%%%%%%%%%%%%%%%%%%%%%%%%%%%%%%%%%%%%%%%%%%%%%%%%%%%%%%%%%%%%%%%%%%%%%%%%%%%%%%%%%%%%%%%%%%%%%%%%%%%%%%%%%%%%%%%%%%%%%%%%%%%%%%%%%%%%%%%%%%%%%%%%%%%%%%%%%%%%%%%%%%%%%%%%%%%%%%%%%%%%%%%%%%%%%%%%%%%%%%%%%%%%%%%%%%%%%%%%%%%%%%%%

\usepackage{eurosym}
\usepackage{vmargin}
\usepackage{amsmath}
\usepackage{graphics}
\usepackage{epsfig}
\usepackage{enumerate}
\usepackage{multicol}
\usepackage{subfigure}
\usepackage{fancyhdr}
\usepackage{listings}
\usepackage{framed}
\usepackage{graphicx}
\usepackage{amsmath}
\usepackage{chngpage}

%\usepackage{bigints}
\usepackage{vmargin}

% left top textwidth textheight headheight

% headsep footheight footskip

\setmargins{2.0cm}{2.5cm}{16 cm}{22cm}{0.5cm}{0cm}{1cm}{1cm}

\renewcommand{\baselinestretch}{1.3}

\setcounter{MaxMatrixCols}{10}

\begin{document}
\begin{enumerate}

\item %% Question 2
(i) Explain the reasons why data are subdivided when conducting mortality
investigations.
(ii)

\item %% - Question 3

Describe the problems which can arise with subdividing data.



\end{enumerate}
\newpage
%%%%%%%%%%%%%%%%%%%%%%%%%%%%%%%%%%%%%%%%%%%%%%%%%%%%%%%%%%%%%%%%%%%%%%%%%%

2
(i)
Users of data require rates subdivided by age and other criteria.
Models are based on the assumption that we can observe groups of identical lives.
Therefore it is important that we analyse groups of lives which are homogenous (or
have the same mortality).
This can, for example, help avoid anti-selection.
(ii)
Small numbers in some sub-groups leading to scanty data and non-
credible rates or a large variance.
Sometimes relevant factors cannot be used because the relevant information cannot be
collected on the proposal form because questions are unlikely to be answered
honestly,
or because the key questions are intrusive or impractical for marketing or
administrative reasons or make the questionnaire too long, or cannot be asked by law.
Can be difficult to ensure that events data and exposed-to-risk data are subdivided in
the same way, leading to the principle of correspondence being violated.
Answers to this question were disappointing, even though not all the points listed above were required for full credit. In part (ii), many candidates made only the first point, about sparse
data. Some candidates approached this question as practitioners or users of data rather than giving the general principles for which the question was asking. Nevertheless, if good points
were made, this approach could earn full credit.
%%Page 3Subject CT4 (Models) – April 2012 – Examiners’ Report

\end{document}
