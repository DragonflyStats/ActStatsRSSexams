\documentclass[a4paper,12pt]{article}

%%%%%%%%%%%%%%%%%%%%%%%%%%%%%%%%%%%%%%%%%%%%%%%%%%%%%%%%%%%%%%%%%%%%%%%%%%%%%%%%%%%%%%%%%%%%%%%%%%%%%%%%%%%%%%%%%%%%%%%%%%%%%%%%%%%%%%%%%%%%%%%%%%%%%%%%%%%%%%%%%%%%%%%%%%%%%%%%%%%%%%%%%%%%%%%%%%%%%%%%%%%%%%%%%%%%%%%%%%%%%%%%%%%%%%%%%%%%%%%%%%%%%%%%%%%%

\usepackage{eurosym}
\usepackage{vmargin}
\usepackage{amsmath}
\usepackage{graphics}
\usepackage{epsfig}
\usepackage{enumerate}
\usepackage{multicol}
\usepackage{subfigure}
\usepackage{fancyhdr}
\usepackage{listings}
\usepackage{framed}
\usepackage{graphicx}
\usepackage{amsmath}
\usepackage{chngpage}

%\usepackage{bigints}
\usepackage{vmargin}

% left top textwidth textheight headheight

% headsep footheight footskip

\setmargins{2.0cm}{2.5cm}{16 cm}{22cm}{0.5cm}{0cm}{1cm}{1cm}

\renewcommand{\baselinestretch}{1.3}

\setcounter{MaxMatrixCols}{10}

\begin{document}


%%PLEASE TURN OVER10
\large
\noindent An investigation was conducted into the effect marriage has on mortality and a model was constructed with three states: 1 Single, 2 Married and 3 Dead. It is assumed that transition rates between states are constant.
\begin{enumerate}[(a)]
\item (i) Sketch a diagram showing the possible transitions between states.
\item (ii) Write down an expression for the likelihood of the data in terms of transition rates and waiting times, defining all the terms you use.
\medskip 
The following data were collected from information on males and females in their thirties.
Years spent in Married state
Years spent in Single state
Number of transitions from Married to Single
Number of transitions from Single to Dead
Number of transitions from Married to Dead
11
40,062
10,298
1,382
12
9
\item (iii) Derive the maximum likelihood estimator of the transition rate from Single to Dead.
\item (iv) Estimate the constant transition rate from Single to Dead and its variance. 
\end{itemize}

%%%%%%%%%%%%%%%%%%5

\newpage
10
(i)
1. Single
2. Married
3. Dead
%------------------------------%
(ii)
{ (
12
13
L ∝ exp −\mu − \mu
) ν } exp { ( −\mu
1
23
−\mu
21
) ν } ( \mu ) ( \mu ) ( \mu ) ( \mu )
2
12
d 12
21
d 21
13
d 13
23
d 23
where
\begin{itemize}
\item $ { \displaystyle \mu ij }$ is the transition intensity from state i to state j
\item $ { \displaystyle ν i}$ is the total observed waiting time in state i
\item ${ \displaystyle d ij}$ is the number of transitions from state i to state j
\end{itemize}
%%%%%%%%%%%%%%%%%%%%%%%%%%%%%%%%%%5
(iii)
Taking the logarithm of the likelihood we get
log e ( L ) = −\mu 13 ν 1 + d 13 log e ( \mu 13 ) + terms not involving $\mu_{13}$.
13
Differentiate with respect to $\mu$
d ln( L )
d \mu 13
1
= −ν +
d 13
\mu 13
.
Setting this to zero we obtain
\mu ˆ 13 =
d 13
ν 1
.
To check it is a maximum differentiate again giving
d 2 log e ( L )
( d \mu 13 ) 2
(iv)
=−
d 13
( \mu 13 ) 2
which is always negative.
The maximum likelihood estimate of \mu 13 is $ { \displaystyle \frac{12}{10,298} = 0.001165  }$.
The variance is
− 1
d 2 ln( L )
( d \mu 13 ) 2
= 12
10, 298 2
= 1.13 x 10 -7 .
%%%%%%%%%%%%%%%%%%%%%%%%%%%%%%%%%%%%%%%%%%%%%%%%%
\newpage 
This was the best answered question on the paper, with most candidates scoring at least 9 of the 11 marks available. In part (ii) many candidates omitted the constant of proportionality. In part (iv) the question says “estimate”, so we needed some indication of where the answers came from for full marks.
\end{document}
