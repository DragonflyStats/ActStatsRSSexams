5 A no claims discount system operates with three levels of discount, 0%, 15% and
40%. If a policyholder makes no claim during the year he moves up a level of
discount (or remains at the maximum level). If he makes one claim during the year he
moves down one level of discount (or remains at the minimum level) and if he makes
two or more claims he moves down to, or remains at, the minimum level.
The probability for each policyholder of making two or more claims in a year is 25%
of the probability of making only one claim.
The long-term probability of being at the 15% level is the same as the long-term
probability of being at the 40% level.
(i) Derive the probability of a policyholder making only one claim in a given
year. 
(ii) Determine the probability that a policyholder at the 0% level this year will be
at the 40% level after three years. 
(iii) Estimate the probability that a policyholder at the 0% level this year will be at
the 40% level after 20 years, without calculating the associated transition
matrix. [3]
[Total 9]
CT4 S2012-4

%%%%%%%%%%%%%%%%%%%%%%%%%%%%%%%%%%%%%%%%%%%%%%%%%%%%%%%%%%%%%%5
5
(i) Let 5
4
x = c
where c is the probability of exactly one claim in a year and x is the probability of one
or more claims in a year.
The transition matrix is
1 0
0 1
1
4
x x
x x
c c x
??? ???
??? ??? ???
??? ???
??? ??? ???
??? ???
??? ??? ???
??? ???
Using ?? = ??P we get
1 1 2 4 3
?? = x?? + x?? + c ??
??2 = (1??? x)??1 + c??3
??3 = (1??? x)??2 + (1??? x)??3
The equation for ??3 gives
??2 (1??? x) = ??3 {1??? (1??? x)} = ??3x
2 31
x
x
?? = ??
???
Subject CT4 (Models) - September 2012 - Examiners' Report
Page 8
So x = 1 ??? x from which x = 0.5 and c = 0.4
So the probability of exactly one claim in any given year is 0.4.
(ii) EITHER
Using the transition matrix
M =
0.5 0.5 0
0.5 0 0.5
0.1 0.4 0.5
??? ???
??? ???
??? ???
??? ???
??? ???
2
0.5 0.5 0 0.5 0.5 0 0.5 0.25 0.25
= 0.5 0 0.5 0.5 0 0.5 0.3 0.45 0.25
0.1 0.4 0.5 0.1 0.4 0.5 0.3 0.25 0.45
M
??? ?????? ??? ??? ???
??? ?????? ??? = ??? ??? ??? ?????? ??? ??? ???
??? ?????? ??? ??? ???
??? ?????? ??? ??? ???
The required probability is therefore
(0.5 × 0.25) + (0.5 × 0.25) + (0 × 0.45) = 0.25
OR
We require the probability of no claims in either of years 2 and 3 (since only this will
leave the policyholder at the 40% level at the end of year 3).
The probability of one or more claims is 0.5 (from the solution to part (i)).
So the probability of no claims is 0.5, and the probability of no claims in years 2 and 3
is 0.5 × 0.5 = 0.25.
(iii) After 20 years the probabilities of being at any level will be close to the
stationary probability distribution
From part (i) we know that 2 3 ?? =?? .
Using ?? = ??P we get
1 2 3 1 0.5?? + 0.5?? + 0.1?? =?? ,
so 2 1
5
6
?? = ?? .
Since 1 2 3 1 ?? +?? +?? = , +½
we have 1
3
8
?? = , 2 3
5
16
?? =?? = .
Subject CT4 (Models) - September 2012 - Examiners' Report
Page 9
So the probability of being at the 40% level after 20 years is estimated as 0.3125.
This question proved more difficult for candidates that the Examiners had envisaged, and
answers were disappointing. Various alternative specifications of the matrix in part (i) were
acceptable. In all three parts of this question some indication of how each result was
obtained was required. Candidates who just wrote down the numerical answers did not
score full credit. The solution to part (ii) could be found by drawing a diagram and tracing
the possible routes through: this is perfectly valid and is arguably the quickest way to the
correct answer. In part (iii) some indication that the answer is an estimate was required.
This could be provided by saying, for example, that after 20 years the probabilities of being
at any level will be close to the stationary probability distribution.









