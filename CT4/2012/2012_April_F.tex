
[Total 13]
A company operates a sick pay scheme as follows:
• Healthy employees pay a percentage of salary to fund the scheme.
• For the first two consecutive months an employee is sick, the sick pay scheme
pays their full salary.
• For the third and subsequent consecutive months of sickness the sick paY_is
reduced to 50% of full salary.
To simplify administration the scheme operates on whole months only, that is for a
particular month’s payroll an employee is either healthy or sick for the purpose of the
scheme.
The company’s experience is that 10% of healthy employees become sick the
following month, and that sick employees have a 75% chance of being healthy the
next month.
The scheme is to be modelled using a Markov Chain.
(i) Explain what is meant by a Markov Chain. (ii) Identify the minimum number of states under which the payments under the
scheme can be modelled using a time homogeneous Markov Chain, specifying
these states.

(iii) Draw a transition graph for this Markov chain. 
(iv) Derive the stationary distribution for this process. 
(v) Calculate the minimum percentage of salary which healthy employees should
pay for the scheme to cover the sick pay costs.

(vi) Calculate the contributions required if, instead, sick pay continued at 100% of
salarY_indefinitely.

(vii) Comment on the benefit to the scheme of the reduction in sick pay to 50%
from the third month.

[Total 15]
END OF PAPER
CT4 A2012–7

%%%%%%%%%%%%%%%%%%%%%%%%%
\newpage
12
(i)
A process with a discrete state space and discrete time space
where the future development is only dependent on the current state occupied.
OR
P [ X t ∈ A | X s 1 = x 1 , X s 2 = x 2 ,..., X s n = x n ] = P [ X t ∈ A | X s = x ]
for all times s 1 < s 2 < ... < s n < s < t , all states x 1 , x 2 ,..., x n , x in S and all
subsets A of S.
THEN EITHER THE THREE STATE SOLUTION
(ii)
The sick pay depends on the duration of sickness, so to model with a time
homogeneous Markov chain needs as a minimum the states:
Healthy (H)
Sick month 1 (S1)
Sick month 2 or more (S2+)
So the minimum number of states is 3.
Page 16Subject CT4 (Models) – %%%%%%%%%%%%%%%%%%%%%%%%%%%%%%%%%%%%5
(iii)
(iv)
If using H, S1, S2+ then the stationary distribution π is given by:
⎛ 0.9 0.1 0 ⎞
⎜
⎟
π ⎜ 0.75 0 0.25 ⎟ = π
⎜ 0.75 0 0.25 ⎟
⎝
⎠
π H = 0.9 π H + 0.75 π S 1 + 0.75 π S 2 +
π S 1 = 0.1 π H
π S 2 + = 0.25 π S 1 + 0.25 π S 2 +
π H + π S 1 + π S 2 + = 1
3 π S 2 + = π S 1
30 π S 2 + = π H
Implies
π S 2 + =
(v)
1
3
15
, π S 1 = , π H =
34
34
17
Let percentage of salary when healthy be a%
Then in the stationary state looking at payments for the next month we need
Expected income = Expected outgo.
Probability healthy *a% of salary = Probability of 100% sick pay*100% of salary +
Probability on 50% sick pay*50% of salary:
2
1
15
⎛ 15
⎞
⎛ 3
⎞
⎜ *0.9 + *0.75 ⎟ a = ⎜ *0.25 + *0.25*50% + *0.1 ⎟
17
34
17
⎝ 17
⎠
⎝ 34
⎠
Page 17Subject CT4 (Models) – %%%%%%%%%%%%%%%%%%%%%%%%%%%%%%%%%%%%5
0.88235a = 0.113971
a =12.917% of salary.
(vi)
Now just need a two state version {H,S}
π H =
15
2
, π S =
17
17
and need contribution rate > 2/15 = 13.333% of salary.
OR THE FOUR STATE SOLUTION
(ii)
The sick pay depends on the duration of sickness, so to model with a time
homogeneous Markov chain needs as a minimum the states:
Healthy (H)
Sick month 1 (S1)
Sick month 2 (S2)
Sick month 3 or more (S3+)
So the minimum number of states is 4.

(iii)
0.1
0.9
H
0.75
0.25
0.25
S1
S2
S3+
0.75
0.75
(iv)
If using H, S1, S2, S3+ then the stationary distribution π is given by:
0
0 ⎞
⎛ 0.9 0.1
⎜
⎟
0.75 0 0.25
0 ⎟
⎜
π
=π
⎜ 0.75 0
0
0.25 ⎟
⎜
⎟
0
0.25 ⎠
⎝ 0.75 0
π H = 0.9 π H + 0.75 π S 1 + 0.75 π S 2 + 0.75 π S 3 +
π S 1 = 0.1 π H
π S 2 = 0.25 π S 1
π S 3 + = 0.25 π S 2 + 0.25 π S 3 +
π H + π S 1 + π S 2 + π S 3 + = 1
Page 18
0.25Subject CT4 (Models) – %%%%%%%%%%%%%%%%%%%%%%%%%%%%%%%%%%%%5
π H = 10 π S 1
1
π S 2 = π S 1
4
0.75 π S 3 + = 0.25 π S 2
1 1
π S 3 + = . π S 1
3 4
1 1 ⎫
⎧
π S 1 ⎨ 10 + 1 + + ⎬ = 1
4 12 ⎭
⎩
Implies
π S 1 =
(v)
12
3
120 15
3
1
= , π H =
= , π s 2 =
π s 3 + =
136 34
136 17
136
136
Let percentage of salary when healthy be a%
Expected income = Expected outgo
Probability healthy *a% of salary = Probability of 100% sick pay*100% of salary
Probability on 50% sick pay*50% of salary
15
3 ⎞ 1 ⎛ 1 ⎞
⎛ 3
a = ⎜ +
⎟ + ⎜
⎟
17
⎝ 34 136 ⎠ 2 ⎝ 136 ⎠
0.88235a = 0.113971
a =12.917% of salary.
(vi)
Now all those not healthy get 100% so
15
2
a =
17
17
and need contribution rate > 2/15 = 13.333% of salary
(vii)
The reduction in cost is calculated as 3.23%.
This is not particularly significant either relative to the likely uncertaintY_in the
assumptions or because recovery rates are so high.
The reduction in sick paY_is likely to encourage employees to try to get back into
work.
This question was well answered despite its complexity. Most candidates went for the four
state solution, and there were many correct answers to parts (i)-(iv). In parts (v) and (vi) a
Page 19Subject CT4 (Models) – %%%%%%%%%%%%%%%%%%%%%%%%%%%%%%%%%%%%5
common mistake was to fail to divide by the proportion of healthy employees, as only healthy
employees (i.e. those not receiving sick pay) contribute to the scheme. Answers to part (vii)
often included sensible comments that gained credit, even if some candidates answered as if
the scheme had an unlimited supply of funds!
\end{document}
