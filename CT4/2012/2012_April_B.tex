\documentclass[a4paper,12pt]{article}

%%%%%%%%%%%%%%%%%%%%%%%%%%%%%%%%%%%%%%%%%%%%%%%%%%%%%%%%%%%%%%%%%%%%%%%%%%%%%%%%%%%%%%%%%%%%%%%%%%%%%%%%%%%%%%%%%%%%%%%%%%%%%%%%%%%%%%%%%%%%%%%%%%%%%%%%%%%%%%%%%%%%%%%%%%%%%%%%%%%%%%%%%%%%%%%%%%%%%%%%%%%%%%%%%%%%%%%%%%%%%%%%%%%%%%%%%%%%%%%%%%%%%%%%%%%%

\usepackage{eurosym}
\usepackage{vmargin}
\usepackage{amsmath}
\usepackage{graphics}
\usepackage{epsfig}
\usepackage{enumerate}
\usepackage{multicol}
\usepackage{subfigure}
\usepackage{fancyhdr}
\usepackage{listings}
\usepackage{framed}
\usepackage{graphicx}
\usepackage{amsmath}
\usepackage{chngpage}

%\usepackage{bigints}
\usepackage{vmargin}

% left top textwidth textheight headheight

% headsep footheight footskip

\setmargins{2.0cm}{2.5cm}{16 cm}{22cm}{0.5cm}{0cm}{1cm}{1cm}

\renewcommand{\baselinestretch}{1.3}

\setcounter{MaxMatrixCols}{10}

\begin{document}
\begin{enumerate}

[Total 6]5
For a particular investigation the hazard of mortalitY_is assumed to take the form:
h ( t ) = A + Bt
where A and B are constants and t represents time.
For each life i in the investigation (i = 1, ..., n) information was collected on the
length of time the life was observed t i and whether the life exited due to death ( δ i = 1
if the life died, 0 otherwise).
(i)
Show that the likelihood of the data is given by:
n
L = ∏ ( A + Bt i ) δ i exp[ − At i −
i = 1
6
1 2
Bt i ] .
2

(ii) Derive two simultaneous equations from which the maximum likelihood
estimates of the parameters A and B can be calculated.

[Total 6]
(i) List the advantages and disadvantages of using models in actuarial work.

A new town is planned in a currently rural area. A model is to be developed to
recommend the number and size of schools required in the new town. The proposed
modelling approach is as follows:
\item The current age distribution of the population in the area is multiplied by the
planned population of the new town to produce an initial population distribution.
\item Current national fertility and mortality rates by age are used to estimate births and
deaths.
\item The births and deaths are applied to the initial population distribution to generate
a projected distribution of the town’s population by age for each future year, and
hence the number of school age children.
(ii)
CT4 A2012–3
Discuss the appropriateness of the proposed modelling approach.

[Total 9]

%%%%%%%%%%%%%%%%%%%%%%%%%%%
5
(i)
The likelihood of the data is given by:
n
L = ∏ f ( t i ) δ i S ( t i ) 1 −δ i ,
i = 1
where f(t i ) is the probability density function and S(t i ) is the survivor function.
Since f(t i ) is related to the hazard function by
f(t i ) = h(t i ) S(t i )
the likelihood can be rewritten:
n
L = ∏ h ( t i ) δ i S ( t i ).
i = 1
Since
⎡ t i
⎤
1
S ( t i ) = exp ⎢ − ∫ h ( r ) dr ⎥ = exp[ − At i − Bt i 2 ],
2
⎢ r = o
⎥
⎣
⎦
n
1
L = ∏ ( A + Bt i ) δ i exp[ − At i − Bt i 2 ] as required.
2
i = 1
(ii)
The log likelihood is given by:
n
1
⎡
⎤
log L = ∑ ⎢ δ i log( A + Bt i ) − At i − Bt i 2 ⎥ .
2
⎦
i = 1 ⎣
We are trying to maximise likelihood with respect to two parameters,
so need partial differentials with respect to A and B:
n
⎡ δ i
⎤
∂
log L = ∑ ⎢
− t i ⎥ ,
∂ A
⎦
i = 1 ⎣ A + Bt i
Page 5Subject CT4 (Models) – April 2012 – Examiners’ Report
n
⎡ δ t
1 ⎤
∂
log L = ∑ ⎢ i i − t i 2 ⎥ .
∂ B
2 ⎦
i = 1 ⎣ A + Bt i
The simultaneous equations satisfied by the MLEs are obtained by
setting these to zero:
n
⎡
⎤
δ
∑ ⎢ A + i Bt i − t i ⎥ = 0,
i = 1 ⎣
n
⎡ δ t
⎦
1
⎤
∑ ⎢ A + i Bt i i − 2 t i 2 ⎥ = 0.
i = 1 ⎣
⎦
In part (i) many candidates failed to explain where the components of the likelihood came
from by explaining the different contributions of the lives who were observed to die and those
who were not. In part (ii) credit was given for knowing the correct method even if this was
not executed. Credit was also given for differentiating a second time and showing that the
second derivatives were negative (and hence that we do have maxima), even though this was
not required for full marks.
6
(i)
Benefits
Systems with long time frames can be studied in compressed time
Complex systems with stochastic elements can be studied (especially by
simulation modelling).
Different future policies or possible actions can be compared either to see which best
suits the requirements of a user or to examine different scenarios without carrying
them out in practice, or to avoid potential costs associated with trialing in real life.
Models allow control over experimental conditions, so that we can reduce the
variance of the results output without upsetting the mean values.
Parameters can be sensitivity tested using a model.
Limitations
Model development requires a lot of time and expertise, and hence can be /costly.
May need to run model lots of times (essential if it is a stochastic model).
Models more useful for comparing the results of input variations than for
optimising outputs.
Page 6Subject CT4 (Models) – April 2012 – Examiners’ Report
Models can look impressive, but can lull the user into a false sense of security.
Impressive output is not a substitute for validity and close imitation of the real world.
This is more true the further into the future you project
Models rely heavily on the data input. If this is poor or lacking in credibility the
output is likely to be flawed.
Users need to understand the model sufficiently well to be able to know when it is
appropriate to applY_it.
Interpretation of models can be difficult, and often outputs need to be seen in relative
rather than absolute terms.
Models cannot take into account all possible future events (e.g. changes in
legislation).
(ii)
The model should be simple to apply.
The data specified are likely to be available from reliable sources.
Although it is possible that the starting point for the planned population may
be wrong
Unforeseen events may take place such as a national epidemic which change
the rates.
The model is relatively straightforward to explain to the planners/developers.
Should consider whether there are trends in fertility rates, rather than simply using
current rates.
Mortality rates unlikely to be significant relative to the uncertaintY_in the projection,
because rates at ages with non-zero fertility rates should be small and child mortality
rates should be low.
Current age distribution for the area may not be representative of that for the new
town as, for example, rural areas may have different distributions to urban areas
Consider the type of houses being built and how they are marketing e.g. are they
family houses?
May wish to consider experience of similar new towns.
May wish to consider whether national fertility rates are appropriate for a new town,
where many young families may live.
Page 7Subject CT4 (Models) – April 2012 – Examiners’ Report
Migration may affect the profile of the population, for example older families moving
away and younger families buying their houses may mean the age structure remains
relatively constant over time regardless of mortality and fertility rates.
The approach does not take account of non-state schooling or the possibility of
children going to boarding school.
Part (i) of the question was standard bookwork and was well answered. The quality of
answers to part (ii) varied: some candidates wrote lengthy and well-argued discussions;
others made only cursory attempts. In part (ii), not all the points listed above were needed
for full credit, and other sensible comments could also score marks.

