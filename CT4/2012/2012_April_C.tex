PLEASE TURN OVER7
Mr Bunn the baker made 12 pies to sell in his shop. He placed the pies in the shop at
9 a.m. During the rest of the day the following events took place.
Time
10 a.m.
11 a.m.
12 noon
1 p.m.
2 p.m.
3 p.m.
5 p.m.
8
Event
A boy bought two pies
A man bought three pies
Mr Bunn accidentally sat on one pie and squashed it so it could not be
sold
A woman bought two pies
A dog from across the street ran into Mr Bunn’s shop and stole two
pies
A girl on the way home from school bought one pie
Mr Bunn closed for the day and the remaining pie was still in the shop
(i) Estimate the time it takes Mr Bunn to sell 40% of the pies he makes, using the
Nelson-Aalen estimator.
[6]
(ii) Comment on whether you think this estimate would be a good basis for Mr
Bunn to plan his future production of pies.

[Total 9]
The mortality experience of a large company pension scheme is to be tested to see if
the experience of males aged 65–72 years is consistent with a standard table. The
results were collated by the firm conducting the analysis on a computer spreadsheet,
with positive and negative standardised deviations being distinguished only by being
in a different coloured font. Unfortunately the results have been supplied to the
companY_in the form of a printout produced on a black-and-white printer from which
it is not possible to tell the signs of the deviations.
The values of the standardised deviations shown are as follows:
0.052
0.967
2.528
0.328
1.234
0.250
1.023
0.756
(i) Suggest two tests which could be conducted from the information given.
(ii) Carry out the tests you suggested in your answer to part (i).
CT4 A2012–4

[8]

%%%%%%%%%%%%%%%%%%%%%

7
(i)
The sequence of events described may be summarised in the table below
Duration t j Pies in shop
n j Pies bought
d j Pies destroyed or
stolen, c j
1
2
3
4
5
6 12
10
7
6
4
2 2
3
0
2
0
1 0
0
1
0
2
0
The hazard of pies being bought is thus
2/12 at duration
3/10 at duration
2/6 at duration
1/2 at duration
1 hour
2 hours
4 hours
6 hours
The Nelson-Aalen estimate of the survival function, S(t), is then
Duration Nelson-Aalen estimate of S(t)
0 \leq t < 1
1\leq t < 2
2 \leq t < 4
4 \leq t < 6
6 \leq t < 8 1
exp [-2/12] = 0.8465
exp [-(2/12 + 3/10)] = 0.6271
exp [-(2/12 + 3/10 + 2/6)] = 0.4493
exp [-(2/12 + 3/10 + 2/6 + 1/2)] = 0.2725
The Nelson-Aalen estimate is a step function.
We need t for which S(t) = 0.6.
Therefore it will be 4 hours until Mr Bunn has sold 40% of his pies.
Page 8Subject CT4 (Models) – %%%%%%%%%%%%%%%%%%%%%%%%%%%%%%%%%%%%5
(ii)
The estimate would not be a good basis on which to plan future production.
And how long it takes to sell 40% of your goods is not very relevant for future
production.
It is based on only one day’s experience, and a good basis for future production
should be based on several days, probablY_involving different days of the week.
Sales of pies may vary seasonally: data from a winter’s day may tell Mr Bunn little
about the demand for pies in summer.
Mr Bunn might be more careful in future not to sit on his pies, and might take steps to
avoid the dog from across the street stealing pies.
The proportion of pies sold will depend on the number of pies Mr Bunn stocks. He
should not assume if he had twice as many pies he would still sell 40% of them in 4
hours.
Mr Bunn may vary his sales strategy, by, for example, reducing his prices
The method does, however, take account to of censored data.
In part (i) the question said “estimate”, so some indication of how the answer was arrived at
was necessary, although not every detail was required. As a bare minimum full credit could
be obtained for first three hazards at times 1, 2 and 4, some statement of what the Nelson-
Aalen estimate of S(t) is, the fact that we are looking for S(t) = 0.6, and some numbers to
demonstrate that S(t) = 0.6 happens at duration 4 hours. Answers that used the logarithm of
S(t) were acceptable. Answers to part (ii) were encouraging. A substantial proportion of
candidates made sensible points.
8
(i)
Chi-squared test (for overall goodness of fit)
(Modified) individual standardised deviations test (for outliers)
(ii)
Chi-squared test
The null hypothesis is that the mortality among the members of the company’s
pension scheme is represented by the standard table.
The test statistic is
∑ z x 2 , where the z x are the standardised deviations.
x
Under the null hypothesis, this statistic has a chi-squared distribution with 8 degrees
of freedom.
Page 9Subject CT4 (Models) – %%%%%%%%%%%%%%%%%%%%%%%%%%%%%%%%%%%%5
∑ z x 2
= 0.052 2 + 0.967 2 + 2.528 2 + 0.328 2 + 1.234 2 + 0.250 2
x
+ 1.023 2 + 0.756 2 = 10.64.
The critical value of the chi-squared distribution with 8 degrees of freedom at the 5%
significance level is 15.51.
Since 10.64 < 15.51
we do not reject the null hypothesis.
(Modified) individual standardised deviations test
Under the null hypothesis (same as for the chi-squared test)
we would expect individual deviations to be distributed Normal (0,1)
Only 1 in 20 of the z x should lie above 1.96 in absolute value
OR
none should lie above 3 in absolute value
OR
about two thirds of the z x should lie between −1 and +1
OR
Interval
Actual deaths
Expected deaths
(0,1)
5
5.44
(1,2)
2
2.24
(2,∞)
1
0.32
The largest deviation we have here is 2.528 in absolute value,
which is well outside the range −1.96 to +1.96,
therefore we have reason to reject the null hypothesis.
but, since we have 8 ages we cannot say definitively whether the null hypothesis
should be rejected, but the large deviation of 2.528 suggests there may be a problem.
Many candidates scored highly on this question, though the chi-squared test was generally
better done than the individual standardised deviations test. A surprising proportion of
candidates thought that it was possible to perform the serial correlations test with the data
given. The most common errors were to reduce the number of degrees of freedom in the chi-
squared test (incorrect here as we are not testing a graduation) and a failure to spot the
large deviation of 2.528, and state that this is a source of concern.
Page 10Subject CT4 (Models) – %%%%%%%%%%%%%%%%%%%%%%%%%%%%%%%%%%%%5
