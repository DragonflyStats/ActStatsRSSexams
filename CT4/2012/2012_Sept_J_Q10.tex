\documentclass[a4paper,12pt]{article}

%%%%%%%%%%%%%%%%%%%%%%%%%%%%%%%%%%%%%%%%%%%%%%%%%%%%%%%%%%%%%%%%%%%%%%%%%%%%%%%%%%%%%%%%%%%%%%%%%%%%%%%%%%%%%%%%%%%%%%%%%%%%%%%%%%%%%%%%%%%%%%%%%%%%%%%%%%%%%%%%%%%%%%%%%%%%%%%%%%%%%%%%%%%%%%%%%%%%%%%%%%%%%%%%%%%%%%%%%%%%%%%%%%%%%%%%%%%%%%%%%%%%%%%%%%%%

\usepackage{eurosym}
\usepackage{vmargin}
\usepackage{amsmath}
\usepackage{graphics}
\usepackage{epsfig}
\usepackage{enumerate}
\usepackage{multicol}
\usepackage{subfigure}
\usepackage{fancyhdr}
\usepackage{listings}
\usepackage{framed}
\usepackage{graphicx}
\usepackage{amsmath}
\usepackage{chngpage}

%\usepackage{bigints}
\usepackage{vmargin}

% left top textwidth textheight headheight

% headsep footheight footskip

\setmargins{2.0cm}{2.5cm}{16 cm}{22cm}{0.5cm}{0cm}{1cm}{1cm}

\renewcommand{\baselinestretch}{1.3}

\setcounter{MaxMatrixCols}{10}

\begin{document}


10 On a small distant planet lives a race of aliens. The aliens can die in one of two ways,
either through illness, or by being sacrificed according to the ancient custom of the
planet. Aliens who die from either cause may, some time later, become zombies.
(i) Draw a multiple-state diagram with four states illustrating the process by
which aliens die and become zombies, labelling the four states and the
possible transitions between them. 
(ii) Write down the likelihood of the process in terms of the transition intensities,
the numbers of events observed and the waiting times in the relevant states,
clearly defining all the terms you use. 
(iii) Derive the maximum likelihood estimator of the death rate from illness. [3]
The aliens take censuses of their population every ten years (where the year is an
“alien year”, which is the length of time their planet takes to orbit their sun). On
1 January in alien year 46,567, there were 3,189 live aliens in the population. On 1
January in alien year 46,577 there were 2,811 live aliens in the population. During
the intervening ten alien years, a total of 3,690 aliens died from illness and 2,310 were
sacrificed, and the annual death rates from illness and sacrifice were constant and the
same for each alien.
(iv) Estimate the annual death rates from illness and from sacrifice over the ten
alien years between alien years 46,567 and 46,577. 
The rate at which aliens who have died from either cause become zombies is 0.1 per
alien year.
(v) Calculate the probabilities that an alien alive in alien year 46,567 will, ten
alien years later:
(a) still be alive
(b) be dead but not a zombie
\newpage

%%%%%%%%%%%%%%%%%%%%%%%%%%%%%%%%%%%%%%%%%%%%%%%%%%%%%%%%%%%%%%%%%%%%%%%%%%%%%%%%%%%%%%%%%%%%%%%%%5

Subject CT4 (Models) – September 2012 – Examiners’ Report
Page 17
10
(i)
(ii) Let the states be labelled as follows:
Alive, A
Dead from illness, I
Dead from sacrifice, C
Zombie, Z
Let the number of transitions observed between states i and j be dij
and let the transition rate between states i and j be μij .
Let the observed waiting time in state i be vi
The likelihood of the data can be written as follows:
exp[( ) ]exp( ) exp( )( ) ( ) ( ) ( ) L ∝ −μAI − μAC vA −μCZ vC −μIZ vI μAI d AI μAC d AC μCZ dCZ μIZ d IZ
(iii) Taking logarithms of the likelihood we have:
log AI A AI log( AI ) terms not depending on AI
e L = −μ v + d μ + μ .
Differentiating this with respect to μAI gives:
(log ) AI
e A
AI AI
d L v d
d
= − +
μ μ
,
and setting the derivative equal to zero produces the maximum likelihood estimate of
μAI :
ˆ
AI
AI
A
d
v
μ = .
μCZ
μAC μIZ
μAI
Dead from illness
Alive
Zombie
Dead from sacrifice
Subject CT4 (Models) – September 2012 – Examiners’ Report
Page 18
This is a maximum as the second derivative
2
2 2
(log )
( ) ( )
AI
e
AI AI
d L d
d
= −
μ μ
is necessarily negative.
(iv) Using the census formula, we estimate vA as follows
0.5( 0 10 ) 0.5(3,189 2,811) 3,000 vA = P + P = + = .
assuming the population of aliens varies linearly over the ten years between the
censuses.
The estimated annual death rate from illness is therefore
369 0.123
3000
= ,
and the estimated rate of death through sacrifice over the ten years is
231 0.077.
3000
=
(v) (a) The probability that an alien is still alive in ten years’ time
is given by the formula
10
10
0
AA exp ( AI AC ) exp[ (0.077 0.123)10]
px du
⎡ ⎤
= ⎢− μ + μ ⎥ = − +
⎢⎣ ⎥⎦
∫
= exp(−2) = 0.135.
(b) Since we are only interested in whether the alien is dead, not what cause (s)he
died from,
and since the rate at which aliens become zombies does not depend on cause
of death, we can combine the two states “Dead from illness” and “Dead from
sacrifice”, into a single state “Dead”.
For an alien to be Dead in 10 years time (s)he must have survived for u
alien years (0 < u < 10), died at time u, and then survived in the Dead state (i.e.
not become a Zombie) for a duration equal to 10-u alien years.
Subject CT4 (Models) – September 2012 – Examiners’ Report
Page 19
The probability density of this happening for any given value of u is
exp [−0.2u] survival Alive for a period u
×
0.2 du (here we ignore o(du))
×
exp[−0.1(10−u)] survival Dead for a period 10 – u
which is
0.2*exp[−(1+ 0.1u)]du = 0.2*exp(−1) exp(−0.1u)du
= 0.0736exp(−0.1u)du
The required probability is obtained by integrating this expression over
all values of u from 0 to 10.
This is
[ ]
10
10
0
0
0.0736exp( 0.1 ) 0.0736 exp( 0.1 )
0.1
− u du = − u
∫ −
= 0.0736 [1 0.3678)] 0.465
0.1
− =
Parts (i), (ii) and (iii) of this question were well answered by most candidates, but there were
few good attempts at parts (iv), (v) and (vi). A minority of candidates produced an
alternative transition diagram in part (i) as follows:
μDZ
μSH
μ μCZ HD μSD
μHS
Sick
Healthy
Zombie
Dead
Subject CT4 (Models) – September 2012 – Examiners’ Report
Page 20
Full credit was given for this, and for answers to parts (ii) and (iii) which were consistent
with it. In part (iii) some candidates derived the maximum likelihood estimate by applying
the correct method to the wrong transition. In part (v)(b) it was possible to write the integral
as follows:
( ) ( )
10
0
∫exp[−0.2 10 − w ]*0.2*exp −0.1w dw.
The evaluation is:
( ) ( )
10
0
0.2* ∫ exp −2 + 0.2w exp −0.1w dw
= ( )
10
0
0.2∫ exp −2 + 0.1w dw
= ( )
10
0
0.2exp(−2)∫ exp 0.1w dw
= 0.2 exp( 2)[exp(1) exp(0)]
0.1
− −
=2*0.1353*(2.718−1) = 0.465
END OF EXAMINERS’ REPORT
Subject CT4 (Models) – September 2012 – Examiners’ Report
Page 17
10
(i)
(ii) Let the states be labelled as follows:
Alive, A
Dead from illness, I
Dead from sacrifice, C
Zombie, Z
Let the number of transitions observed between states i and j be dij
and let the transition rate between states i and j be μij .
Let the observed waiting time in state i be vi
The likelihood of the data can be written as follows:
exp[( ) ]exp( ) exp( )( ) ( ) ( ) ( ) L ∝ −μAI − μAC vA −μCZ vC −μIZ vI μAI d AI μAC d AC μCZ dCZ μIZ d IZ
(iii) Taking logarithms of the likelihood we have:
log AI A AI log( AI ) terms not depending on AI
e L = −μ v + d μ + μ .
Differentiating this with respect to μAI gives:
(log ) AI
e A
AI AI
d L v d
d
= − +
μ μ
,
and setting the derivative equal to zero produces the maximum likelihood estimate of
μAI :
ˆ
AI
AI
A
d
v
μ = .
μCZ
μAC μIZ
μAI
Dead from illness
Alive
Zombie
Dead from sacrifice
Subject CT4 (Models) – September 2012 – Examiners’ Report
Page 18
This is a maximum as the second derivative
2
2 2
(log )
( ) ( )
AI
e
AI AI
d L d
d
= −
μ μ
is necessarily negative.
(iv) Using the census formula, we estimate vA as follows
0.5( 0 10 ) 0.5(3,189 2,811) 3,000 vA = P + P = + = .
assuming the population of aliens varies linearly over the ten years between the
censuses.
The estimated annual death rate from illness is therefore
369 0.123
3000
= ,
and the estimated rate of death through sacrifice over the ten years is
231 0.077.
3000
=
(v) (a) The probability that an alien is still alive in ten years’ time
is given by the formula
10
10
0
AA exp ( AI AC ) exp[ (0.077 0.123)10]
px du
⎡ ⎤
= ⎢− μ + μ ⎥ = − +
⎢⎣ ⎥⎦
∫
= exp(−2) = 0.135.
(b) Since we are only interested in whether the alien is dead, not what cause (s)he
died from,
and since the rate at which aliens become zombies does not depend on cause
of death, we can combine the two states “Dead from illness” and “Dead from
sacrifice”, into a single state “Dead”.
For an alien to be Dead in 10 years time (s)he must have survived for u
alien years (0 < u < 10), died at time u, and then survived in the Dead state (i.e.
not become a Zombie) for a duration equal to 10-u alien years.
Subject CT4 (Models) – September 2012 – Examiners’ Report
Page 19
The probability density of this happening for any given value of u is
exp [−0.2u] survival Alive for a period u
×
0.2 du (here we ignore o(du))
×
exp[−0.1(10−u)] survival Dead for a period 10 – u
which is
0.2*exp[−(1+ 0.1u)]du = 0.2*exp(−1) exp(−0.1u)du
= 0.0736exp(−0.1u)du
The required probability is obtained by integrating this expression over
all values of u from 0 to 10.
This is
[ ]
10
10
0
0
0.0736exp( 0.1 ) 0.0736 exp( 0.1 )
0.1
− u du = − u
∫ −
= 0.0736 [1 0.3678)] 0.465
0.1
− =
Parts (i), (ii) and (iii) of this question were well answered by most candidates, but there were
few good attempts at parts (iv), (v) and (vi). A minority of candidates produced an
alternative transition diagram in part (i) as follows:
μDZ
μSH
μ μCZ HD μSD
μHS
Sick
Healthy
Zombie
Dead
Subject CT4 (Models) – September 2012 – Examiners’ Report
Page 20
Full credit was given for this, and for answers to parts (ii) and (iii) which were consistent
with it. In part (iii) some candidates derived the maximum likelihood estimate by applying
the correct method to the wrong transition. In part (v)(b) it was possible to write the integral
as follows:
( ) ( )
10
0
∫exp[−0.2 10 − w ]*0.2*exp −0.1w dw.
The evaluation is:
( ) ( )
10
0
0.2* ∫ exp −2 + 0.2w exp −0.1w dw
= ( )
10
0
0.2∫ exp −2 + 0.1w dw
= ( )
10
0
0.2exp(−2)∫ exp 0.1w dw
= 0.2 exp( 2)[exp(1) exp(0)]
0.1
− −
=2*0.1353*(2.718−1) = 0.465
END OF EXAMINERS’ REPORT
