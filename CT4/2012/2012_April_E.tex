[Total 11]
The series Y_i records, for each time period i, whether a car driver is accident free
during that period (Y_i = 0) or has at least one accident (Y_i = 1).
i
Define X i = ∑ Y j with state space {0,1,2,....}.
j = 1
An insurer makes an assumption about the driver’s accident proneness by considering
that the probability of a driver having at least one accident is related to the proportion
of previous time periods in which the driver had at least one accident as follows:
P ( Y n + 1 = 1) =
X
1
(1 + n ),
n
4
for
n ≥ 1
1
2
with P ( Y 1 = 1) =
(i) Demonstrate that the series X i satisfies the Markov property, whilst Y_i does
not.

(ii) Explain whether the series X i is:
(a)
(b)
irreducible
time homogeneous

CT4 A2012–612
(iii) Draw the transition graph for X i covering all transitions which could occur in
the first three time periods, including the transition probabilities.

(iv) Calculate the probability that the driver has accidents during exactly two of the
first three time periods.

(v) Comment on the appropriateness of the insurer’s assumption about accident
proneness.


%%%%%%%%%%%%%%%%%%%%%%%%%%%%%%%%%%%%%%%%%%%%%%%%%%%%%%%%%%%%%%%%%%%%%%%%%%%%%%%%%

11
X n + 1 = X n + Y n + 1 = X n + f ( X n )
(i)
so the series X i depends only on the current state and hence satisfies the
Markov property.
n
⎛
⎜ ∑ Y j
X n ⎞ 1 ⎜
1 ⎛
= 1 + j = 1
Y n + 1 = ⎜ 1 +
⎟
4 ⎝
n ⎠ 4 ⎜
n
⎜
⎝
⎞
⎟
⎟
⎟
⎟
⎠
and hence depends on all the previous values of Y_i .
Page 14Subject CT4 (Models) – %%%%%%%%%%%%%%%%%%%%%%%%%%%%%%%%%%%%5
(ii)
(a)
It is not possible for the cumulative number of accidents to reduce
(OR the cumulative number of accidents is an increasing/
non-decreasing function)
and so the process is not irreducible.
(b)
The probabilities depend on the number of time periods n
so the process is not time homogeneous
(iii)
t=0
t=1
1/2
0
t=2
1/2
1
1/2
t=3
1/2
2
1/2
0
3
1/2
1/4
1
3/8
2
5/8
3/4
0
1/4
1
3/4
0
(iv)
From the diagram above (or otherwise) it can be seen that there are
three paths to the 2 accidents by time 3 box.
Required probability
= Pr(0−0−1−2)+Pr(0−1−1−2)+Pr(0−1−2−2)
1 1 3 1 1 3 1 1 1 3 + 6 + 8 17
=
= . . + . . + . . =
2 4 8 2 2 8 2 2 2
64
64
(v)
It is reasonable to assume that probability of having an accident depends on the
number of previous accidents.
It is also reasonable that the effect of a previous accident should wear off over time.
There are likely to be other factors which have a significant effect on the probability
of an accident,
Page 15Subject CT4 (Models) – %%%%%%%%%%%%%%%%%%%%%%%%%%%%%%%%%%%%5
such as the fact that people who have recently had an accident might drive more
carefully.
May want to give more weighting to recent years.
This was a demanding question, and only a minority of candidates scored highly. In part (i)
few answers were sufficiently rigorous, many either re-stating the question or simply stating
the Markov property. Part (ii)(a) was well answered, but in part (ii)(b) many candidates did
not understand the term “time homogeneous”. Most candidates made only sketchy attempts
at part (iii) and (iv). Credit was given for calculations in part (iv) which demonstrated that
candidates knew the correct method, even though the numbers were incorrect.
