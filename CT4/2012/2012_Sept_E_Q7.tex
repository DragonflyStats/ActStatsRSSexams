7 The volatility of equity prices is classified as being High (H) or Low (L) according to
whether it is above or below a particular level. The volatility status is assumed to
follow a Markov jump process with constant transition rates ϕLH = μ and ϕHL = ρ.
(i) Write down the generator matrix of the Markov jump process. 
(ii) State the distribution of holding times in each state. 
A history of equity price volatility is available over a representative time period.
(iii) Explain how the parameters μ and ρ can be estimated. 
Let ij
t ps be the probability that the process is in state j at time s+t given that it was in
state i at time s (i, j = H, L), where t ≥ 0. Let
__
ii
t ps be the probability that the process
remains in state i from time s to time s+t .
CT4 S2012–5 PLEASE TURN OVER
(iv) Write down Kolmogorov’s forward equations for LL
t Ps
t
∂
∂
, LL
t Ps
t
∂
∂
and
LH
t Ps
t
∂
∂
. 
Equity price volatility is Low at time zero.
(v) Derive an expression for the time after which there is a greater than 50%
chance of having experienced a period of high equity price volatility. 
(vi) Solve the Kolmogorov equation to obtain an expression for 0
LL
t P . 
[Total 12]

%%%%%%%%%%%%%%%%%%%%%%%%%%%%%%%%%%%%%%%%%%%%%%%%%%%%%%%%%%%%%%%%%%55

7
(i) With state space {L, H} we have generator matrix
A
⎛ −μ μ ⎞
= ⎜ ⎟ ⎝ ρ −ρ⎠
(ii) The holding times are exponentially distributed with parameter μ in state L and ρ in
state H.
(iii) EITHER
The time spent in state L before the next visit to H has mean 1/μ.
Therefore a reasonable estimate for μ is the reciprocal of the mean length of each
visit:
= (Number of transitions from L to H) / (Total time spent in state L)
Similarly estimate for ρ is the reciprocal of the mean length of each visit:
= (Number of transitions from H to L) / (Total time spent in state H)
OR
Using the maximum likelihood estimator for μ, we have:
(Number of transitions from L to H)/Total time spent in state L).
Similarly, the MLE of ρ is
(Number of transitions from H to L)/Total time spent in state H).
(iv) LL LL
t Ps t Ps
t
∂
= −μ
∂
LL LL LH
t Ps t Ps t Ps
t
∂
= −μ + ρ
∂
LH LL LH
t Ps t Ps t Ps
t
∂
= μ − ρ
∂
(v) LL LL
t Ps t Ps
t
∂
= −μ
∂
so 0LL exp( )
t P = −μt
Subject CT4 (Models) – September 2012 – Examiners’ Report
Page 12
Looking for time when 0LL 1/ 2
t P =
1/ 2 = exp(−μT )
T = ln(2) / μ
(vi) Observe that 0LL 0LH 1
t P + t P =
so, substituting, we have
0LL 0LL (1 0LL )
t P t P t P
t
∂
= −μ + ρ −
∂
exp(( ) ) 0LL exp(( ) )
t t P t
t
∂ ⎡ ⎤ ⎣ μ + ρ ⎦ = ρ μ + ρ ∂
exp(( ) ) 0LL exp(( ) ) constant
t t P t
ρ
μ + ρ = μ + ρ +
μ + ρ
And in state L at time zero so const
μ
=
μ + ρ
0LL exp( ( ) )
tP t
ρ μ
= + −μ+ρ
μ + ρ μ + ρ
Few candidates scored highly on this question. In particular, very few made a serious
attempt at parts (v) and (vi). In part (iv), there was confusion among some candidates
between 0
LL
t p and 0
LL
t p
−−
and
a common error was to write down LL exp( ).
t Ps t
t
∂
= −μ
∂
Many candidates did not attempt part (v) even though is is relatively straightforward.
In part (vi) working through with tP0
LH then at the end taking one minus the answer is a valid
approach .

