\documentclass[a4paper,12pt]{article}

%%%%%%%%%%%%%%%%%%%%%%%%%%%%%%%%%%%%%%%%%%%%%%%%%%%%%%%%%%%%%%%%%%%%%%%%%%%%%%%%%%%%%%%%%%%%%%%%%%%%%%%%%%%%%%%%%%%%%%%%%%%%%%%%%%%%%%%%%%%%%%%%%%%%%%%%%%%%%%%%%%%%%%%%%%%%%%%%%%%%%%%%%%%%%%%%%%%%%%%%%%%%%%%%%%%%%%%%%%%%%%%%%%%%%%%%%%%%%%%%%%%%%%%%%%%%

\usepackage{eurosym}
\usepackage{vmargin}
\usepackage{amsmath}
\usepackage{graphics}
\usepackage{epsfig}
\usepackage{enumerate}
\usepackage{multicol}
\usepackage{subfigure}
\usepackage{fancyhdr}
\usepackage{listings}
\usepackage{framed}
\usepackage{graphicx}
\usepackage{amsmath}
\usepackage{chngpage}

%\usepackage{bigints}
\usepackage{vmargin}

% left top textwidth textheight headheight

% headsep footheight footskip

\setmargins{2.0cm}{2.5cm}{16 cm}{22cm}{0.5cm}{0cm}{1cm}{1cm}

\renewcommand{\baselinestretch}{1.3}

\setcounter{MaxMatrixCols}{10}

\begin{document}
\begin{enumerate}
\item %% Question 1
(i) Define a general random walk.
(ii) State the conditions under which a general random walk would become a
simple random walk.

[Total 2]
\item %% Question 2
(i) Explain the reasons why data are subdivided when conducting mortality
investigations.
(ii)

\item %% - Question 3

Describe the problems which can arise with subdividing data.


[Total 4]
A graduation of a set of crude mortality rates is tested for goodness-of-fit using a
chi-squared test.
Discuss the factors to be considered in determining the number of degrees of freedom
to use for the test statistic.

4
A new drug treatment for patients suffering from a chronic skin disease with visible
symptoms was tested. The drug was administered through a daily dose for the
duration of the trial. As soon as the drug regime started, the symptoms disappeared in
all patients, but after some time had a tendency to reappear as the agent causing the
disease developed resistance to the drug. The trial lasted for six months.
The data below show the number of patients experiencing a return of their symptoms
in each month after the drug regime started.
Month Number of patient-months
exposed to risk Number of patients experiencing
a return of their symptoms
1
2
3
4
5
6 200
190
175
150
135
125 5
8
15
10
6
3
(i)
Calculate the hazard of symptoms returning in each month.

As part of the investigation, it is desired to assess the impact of certain risk factors on
the hazard of symptoms returning. It is suggested that to achieve this, the hazard
could be modelled using either a Gompertz model or a semi-parametric model.
(ii)
CT4 A2012–2
Comment on the use of each of these models in this situation.
\end{enumerate}
\newpage
%%%%%%%%%%%%%%%%%%%%%%%%%%%%%%%%%%%%%%%%%%%%%%%%%%%%%%%%%%%%%%%%%%%%%%%%%%

1
n
(i)
X n = ∑ Y j
j = 1
where the Y j are i.i.d. random variables and X 0 = 0
(ii)
Is simple random walk when Y j can have values +1 and −1 only
In part (i) few candidates gave the initial condition that X 0 =0. Many candidates were
confused as to the definitions of general, simple, and symmetric random walks (for example
defining a simple random walk in part (i) and then stating for part (ii) that the probabilities
of Y j being +1 and -1 were both equal to 0.5).
2
(i)
Users of data require rates subdivided by age and other criteria.
Models are based on the assumption that we can observe groups of identical lives.
Therefore it is important that we analyse groups of lives which are homogenous (or
have the same mortality).
This can, for example, help avoid anti-selection.
(ii)
Small numbers in some sub-groups leading to scanty data and non-
credible rates or a large variance.
Sometimes relevant factors cannot be used because the relevant information cannot be
collected on the proposal form because questions are unlikely to be answered
honestly,
or because the key questions are intrusive or impractical for marketing or
administrative reasons or make the questionnaire too long, or cannot be asked by law.
Can be difficult to ensure that events data and exposed-to-risk data are subdivided in
the same way, leading to the principle of correspondence being violated.
Answers to this question were disappointing, even though not all the points listed above were required for full credit. In part (ii), many candidates made only the first point, about sparse
data. Some candidates approached this question as practitioners or users of data rather than giving the general principles for which the question was asking. Nevertheless, if good points
were made, this approach could earn full credit.
%%Page 3Subject CT4 (Models) – April 2012 – Examiners’ Report

\newpage
3
To work out the number of degrees of freedom (d.f.) we start with the number of age groups.
We reduce the d.f. because of the constraints imposed by the graduation process.
The reduction varies according to the graduation method:
parametric formula – one d.f. lost for each parameter estimated;
standard table – one d.f. lost for each parameter fitted and a further reduction due to the constraints imposed by the choice of standard table;
graphical – two or three d.f. lost for about every 10 ages graduated.
This question was generally well answered. Common errors were to suppose that only one d.f. is lost for the choice of standard table, and that for graphical graduation, two or three
d.f. were lost in total, regardless of the number of ages being graduated.
4
(i) Month 1
Month 2
Month 3
Month 4
Month 5
Month 6
5/200 = 0.025
8/190 = 0.042
15/175 = 0.086
10/150 = 0.067
6/135 = 0.044
3/125 = 0.024
(ii) To assess the impact of risk factors, a proportional hazards model would be useful
because of its simple interpretation or because it allows the effect of each individual
risk factor to be assessed.
The Gompertz model can be framed as a proportional hazards model, as can a semi-
parametric model (such as the Cox model).
The Gompertz model would not be appropriate here, as it has a monotonically
increasing or decreasing hazard,
whereas it is clear from part (i) that the hazard of symptoms returning first rises and
then falls with duration.
A semi-parametric model allows the shape of the hazard to be determined by the data.
The semi-parametric model would be better than the Gompertz in this case.
In part (i) a minority of candidates subtracted half the deaths from the exposed-to-risk. Partial credit was given for this. Part (ii) was a higher skills question, and was poorly
attempted by many candidates. Only a small proportion related their answers to the data 
given and spotted that the empirical hazard calculated in part (i) was non-monotonic and so
the Gompertz model would be a poor fit. Hardly any candidates pointed out that the
Gompertz model can be framed as a proportional hazards model.
\end{document}
