
\documentclass[a4paper,12pt]{article}

%%%%%%%%%%%%%%%%%%%%%%%%%%%%%%%%%%%%%%%%%%%%%%%%%%%%%%%%%%%%%%%%%%%%%%%%%%%%%%%%%%%%%%%%%%%%%%%%%%%%%%%%%%%%%%%%%%%%%%%%%%%%%%%%%%%%%%%%%%%%%%%%%%%%%%%%%%%%%%%%%%%%%%%%%%%%%%%%%%%%%%%%%%%%%%%%%%%%%%%%%%%%%%%%%%%%%%%%%%%%%%%%%%%%%%%%%%%%%%%%%%%%%%%%%%%%

\usepackage{eurosym}
\usepackage{vmargin}
\usepackage{amsmath}
\usepackage{graphics}
\usepackage{epsfig}
\usepackage{enumerate}
\usepackage{multicol}
\usepackage{subfigure}
\usepackage{fancyhdr}
\usepackage{listings}
\usepackage{framed}
\usepackage{graphicx}
\usepackage{amsmath}
\usepackage{chngpage}

%\usepackage{bigints}
\usepackage{vmargin}

% left top textwidth textheight headheight

% headsep footheight footskip

\setmargins{2.0cm}{2.5cm}{16 cm}{22cm}{0.5cm}{0cm}{1cm}{1cm}

\renewcommand{\baselinestretch}{1.3}

\setcounter{MaxMatrixCols}{10}

\begin{document}


OR
Graduation with reference to a standard table is useful if data are scanty and a
suitable standard table exists (e.g. for female pensioners from a small scheme).
(iii) To test for overall goodness of fit we use the χ2 test.
The null hypothesis is that the graduated rates are the same as the true underlying
rates in the block of business.
The test statistic 2 2
x m
x
Σz ≈ χ where m is the degrees of freedom.
Age Exposed
to risk
Observed
deaths
Graduated
rates ( ˆ ) s q
Expected
deaths
zx zx
2
40 1,284 4 .00240 3.0816 0.5232 0.2737
41 2,038 4 .00266 5.4211 -0.6103 0.3725
42 1,952 12 .00297 5.7974 2.5760 6.6360
43 2,158 7 .00332 7.1646 -0.0615 0.0038
44 2,480 11 .00371 9.2008 0.5932 0.3518
45 1,456 7 .00415 6.0424 0.3896 0.1518
46 2,100 12 .00464 9.7440 0.7227 0.5223
47 1,866 16 .00519 9.6845 2.0294 4.1184
48 1,989 15 .00577 11.4765 1.0401 1.0818
49 1,725 10 .00642 11.0745 -0.3229 0.1043
Total 6.8794 13.6163
The observed test statistic is 13.62
The number of age groups is 10, but we lose an unknown number of degrees for the
graduation, perhaps 2. So m = 8, say.
The critical value of the chi-squared distribution with 8 degrees of
freedom at the 5% level is 15.51.
Since 13.62 < 15.51
we do not reject the null hypothesis.
(iv) It is not necessary to test for smoothness if the graduation was performed using a
parametric formula or a standard table, provided that a small number of parameters
were used in the formula, or in the function linking to the rates in the standard table.
It will be necessary to test for smoothness if the graduation was performed graphically
but this is unlikely to be the case with data from a large insurance company.

%%%%%%%%%%%%%%%%%%%%%%%%%%%%%%%%%%%%%%%%%%%%%%%%%%%%%%%%%%%%%%%%%
%%-- Question 8
(i) When preparing standard tables OR when graduating data from a large industrywide
scheme, or a national population
because there will be lots of data available.
(ii) (a) EITHER Graphical graduation OR Graduation with reference to a standard
table
(b) EITHER
Graphical graduation may be suitable for a analysis of a newly discovered
insect (as data will be scanty and an existing table will not exist)

OR
Graduation with reference to a standard table is useful if data are scanty and a
suitable standard table exists (e.g. for female pensioners from a small scheme).
(iii) To test for overall goodness of fit we use the χ2 test.
The null hypothesis is that the graduated rates are the same as the true underlying
rates in the block of business.
The test statistic 2 2
x m
x
Σz ≈ χ where m is the degrees of freedom.
Age Exposed
to risk
Observed
deaths
Graduated
rates ( ˆ ) s q
Expected
deaths
zx zx
2
40 1,284 4 .00240 3.0816 0.5232 0.2737
41 2,038 4 .00266 5.4211 -0.6103 0.3725
42 1,952 12 .00297 5.7974 2.5760 6.6360
43 2,158 7 .00332 7.1646 -0.0615 0.0038
44 2,480 11 .00371 9.2008 0.5932 0.3518
45 1,456 7 .00415 6.0424 0.3896 0.1518
46 2,100 12 .00464 9.7440 0.7227 0.5223
47 1,866 16 .00519 9.6845 2.0294 4.1184
48 1,989 15 .00577 11.4765 1.0401 1.0818
49 1,725 10 .00642 11.0745 -0.3229 0.1043
Total 6.8794 13.6163
The observed test statistic is 13.62
The number of age groups is 10, but we lose an unknown number of degrees for the
graduation, perhaps 2. So m = 8, say.
The critical value of the chi-squared distribution with 8 degrees of
freedom at the 5% level is 15.51.
Since 13.62 < 15.51
we do not reject the null hypothesis.
(iv) It is not necessary to test for smoothness if the graduation was performed using a
parametric formula or a standard table, provided that a small number of parameters
were used in the formula, or in the function linking to the rates in the standard table.
It will be necessary to test for smoothness if the graduation was performed graphically
but this is unlikely to be the case with data from a large insurance company.

OR
Graduation with reference to a standard table is useful if data are scanty and a
suitable standard table exists (e.g. for female pensioners from a small scheme).
(iii) To test for overall goodness of fit we use the χ2 test.
The null hypothesis is that the graduated rates are the same as the true underlying
rates in the block of business.
The test statistic 2 2
x m
x
Σz ≈ χ where m is the degrees of freedom.
Age Exposed
to risk
Observed
deaths
Graduated
rates ( ˆ ) s q
Expected
deaths
zx zx
2
40 1,284 4 .00240 3.0816 0.5232 0.2737
41 2,038 4 .00266 5.4211 -0.6103 0.3725
42 1,952 12 .00297 5.7974 2.5760 6.6360
43 2,158 7 .00332 7.1646 -0.0615 0.0038
44 2,480 11 .00371 9.2008 0.5932 0.3518
45 1,456 7 .00415 6.0424 0.3896 0.1518
46 2,100 12 .00464 9.7440 0.7227 0.5223
47 1,866 16 .00519 9.6845 2.0294 4.1184
48 1,989 15 .00577 11.4765 1.0401 1.0818
49 1,725 10 .00642 11.0745 -0.3229 0.1043
Total 6.8794 13.6163
The observed test statistic is 13.62
The number of age groups is 10, but we lose an unknown number of degrees for the
graduation, perhaps 2. So m = 8, say.
The critical value of the chi-squared distribution with 8 degrees of
freedom at the 5% level is 15.51.
Since 13.62 < 15.51
we do not reject the null hypothesis.
(iv) It is not necessary to test for smoothness if the graduation was performed using a
parametric formula or a standard table, provided that a small number of parameters
were used in the formula, or in the function linking to the rates in the standard table.
It will be necessary to test for smoothness if the graduation was performed graphically
but this is unlikely to be the case with data from a large insurance company.
