
\documentclass[a4paper,12pt]{article}
%%%%%%%%%%%%%%%%%%%%%%%%%%%%%%%%%%%%%%%%%%%%%%%%%%%%%%%%%%%%%%%%%%%%%%%%%%%%%%%%%%%%%%%%%%%%%%%%%%%%%%%%%%%%%%%%%%%%%%%%%%%%%%%%%%%%%%%%%%%%%%%%%%%%%%%%%%%%%%%%%%%%%%%%%%%%%%%%%%%%%%%%%%%%%%%%%%%%%%%%%%%%%%%%%%%%%%%%%%%%%%%%%%%%%%%%%%%%%%%%%%%%%%%%%%%%
\usepackage{eurosym}
\usepackage{vmargin}
\usepackage{amsmath}

\usepackage{graphics}

\usepackage{epsfig}

\usepackage{enumerate}

\usepackage{multicol}

\usepackage{subfigure}

\usepackage{fancyhdr}

\usepackage{listings}

\usepackage{framed}

\usepackage{graphicx}

\usepackage{amsmath}

\usepackage{chngpage}

%\usepackage{bigints}
\usepackage{vmargin}


% left top textwidth textheight headheight

% headsep footheight footskip

\setmargins{2.0cm}{2.5cm}{16 cm}{22cm}{0.5cm}{0cm}{1cm}{1cm}
\renewcommand{\baselinestretch}{1.3}
\setcounter{MaxMatrixCols}{10}
\begin{document}

\begin{enumerate}
1 Describe two benefits and two limitations of using models in actuarial work. 
2 A large company wishes to construct a model of sickness rates among its employees to use in evaluating the present and future financial health of its sick pay scheme.
Outline factors which the company should take into consideration when developing the model. 

%%%%%%%%%%%%%%%%%%%%%%%%%%%%%%%%%%%%%%%%%%%%%%%%%%%%%%%%%%%%%%%%%%%
1
Subject CT4 (Models) – September 2012 – Examiners’ Report
Page 3
Benefits
Systems with long time frames can be studied in compressed time
Complex systems with stochastic elements can be studied (especially by simulation
modelling).
Different future policies or possible actions can be compared to see which best suits the
requirements of a user.
Models allow control over experimental conditions, so that we can reduce the variance of the results output without upsetting the mean values.
Limitations
Model development requires a lot of time and expertise, and hence can be costly.
Models more useful for comparing the results of input variations than for optimising outputs.
Models can look impressive, but can lull the user into a false sense of security. Impressive output is not a substitute for validity and close imitation of the real world.
Models rely heavily on the data input. If this is poor or lacking in credibility the output is likely to be flawed.
Models rely heavily on the assumptions used, poor assumptions can invalidate the model
output.
Users need to understand the model sufficiently well to be able to know when it is
appropriate to apply it.
Interpretation of models can be difficult.
Models cannot take into account all possible future events, e.g. changes in legislation.
Many candidates scored full marks on this question. The question asked for TWO benefits
and TWO limitations, so credit was given for the most fully described two of each. Extra marks for the benefits could not be transferred to the limitations to make up a shortfall, and
vice versa.
Subject CT4 (Models) – September 2012 – Examiners’ Report
Page 4
2
The nature of the existing sickness data the company possesses. The model can only be as
complex as the data will allow it to be.
Whether the company has made any previous attempts to model sickness rates among its employees, and how successful they were.
The complexity of the model – e.g. whether it should be stochastic or deterministic. More
complex models will be costlier to prepare and run, but eventually there may be diminishing
returns to additional complexity.
General trends in sickness at the national level may need to be built in.
The definition of sickness and level of benefits payable under the scheme.
Does the company plan to change the characteristics of the employees? For example, does it
plan to recruit more mature persons?
The ease of communication of the model.
The budget and resources available for the construction of the model.
Capability of staff. Will outside consultants be required?
By whom will the model be used? Will they be capable of understanding and using it?
Does the model need to interface with models of other aspects of the company’s business
(e.g. taking data from other systems)?
The independence of sickness rates should be taken into account e.g. in the event of an
epidemic claims cannot be considered independent.
Other relevant points were given credit. The Examiners were looking for comments which
made reference to the scenario proposed in the question. Many candidates simply
reproduced one of the lists in the Core Reading (unit 1 page 4 and unit 1 page 6) without
relating to the scenario in the question. Answers along these lines scored limited credit.
Subject CT4 (Models) – September 2012 – Examiners’ Report
