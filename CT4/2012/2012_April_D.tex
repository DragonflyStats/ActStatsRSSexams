\documentclass[a4paper,12pt]{article}

%%%%%%%%%%%%%%%%%%%%%%%%%%%%%%%%%%%%%%%%%%%%%%%%%%%%%%%%%%%%%%%%%%%%%%%%%%%%%%%%%%%%%%%%%%%%%%%%%%%%%%%%%%%%%%%%%%%%%%%%%%%%%%%%%%%%%%%%%%%%%%%%%%%%%%%%%%%%%%%%%%%%%%%%%%%%%%%%%%%%%%%%%%%%%%%%%%%%%%%%%%%%%%%%%%%%%%%%%%%%%%%%%%%%%%%%%%%%%%%%%%%%%%%%%%%%

\usepackage{eurosym}
\usepackage{vmargin}
\usepackage{amsmath}
\usepackage{graphics}
\usepackage{epsfig}
\usepackage{enumerate}
\usepackage{multicol}
\usepackage{subfigure}
\usepackage{fancyhdr}
\usepackage{listings}
\usepackage{framed}
\usepackage{graphicx}
\usepackage{amsmath}
\usepackage{chngpage}

%\usepackage{bigints}
\usepackage{vmargin}

% left top textwidth textheight headheight

% headsep footheight footskip

\setmargins{2.0cm}{2.5cm}{16 cm}{22cm}{0.5cm}{0cm}{1cm}{1cm}

\renewcommand{\baselinestretch}{1.3}

\setcounter{MaxMatrixCols}{10}

\begin{document}
\begin{enumerate}
[Total 10]9
(i)
List four factors other than age and smoker status by which life insurance
mortality statistics are often subdivided.

Two offices in different towns of the same life insurance company write 25-year term
assurance policies. Below are data from these two offices relating to policyholders of
the same age. Both deaths and policies in force are on an age last birthday basis.
Gasperton
Policies in force on 1 January 2009
Policies in force on 1 January 2010
Deaths in calendar year 2009
(ii)
2,000
2,100
25
Great Hawking
1,770
1,674
21
Calculate the central death rate for the calendar year 2009 at this age for the
offices in Gasperton and Great Hawking.

A detailed examination of the records shows that 50\% of the policyholders in
Gasperton at both censuses were smokers, and 20\% of policyholders in Great
Hawking at both censuses were smokers. National death rates at this age for smokers
in 2009 were 40\% higher than those for non-smokers.
(iii)
Estimate the central death rates for smokers and non-smokers in Gasperton
and Great Hawking.

The life insurance company charges policyholders in Gasperton and Great Hawking
the same premiums for the 25-year term assurance policies. It charges smokers in
both towns 40\% more than non-smokers.
(iv)
%%CT4 A2012–5
Comment on the company’s pricing structure in the light of your results from
parts (ii) and (iii) above.

[Total 11]
%%PLEASE TURN OVER10
An investigation was conducted into the effect marriage has on mortality and a model
was constructed with three states: 1 Single, 2 Married and 3 Dead. It is assumed that
transition rates between states are constant.
(i) Sketch a diagram showing the possible transitions between states.

(ii) Write down an expression for the likelihood of the data in terms of transition
rates and waiting times, defining all the terms you use.

The following data were collected from information on males and females in their
thirties.
Years spent in Married state
Years spent in Single state
Number of transitions from Married to Single
Number of transitions from Single to Dead
Number of transitions from Married to Dead
11
40,062
10,298
1,382
12
9
(iii) Derive the maximum likelihood estimator of the transition rate from Single to
Dead.

(iv) Estimate the constant transition rate from Single to Dead and its variance. 


%%%%%%%%%%%%%%%%%%5

9
(i)
Gender
Type of policy
Level of underwriting
Duration in force
Sales channel
Policy size
Occupation
Known impairments
Postcode/geographical area
Education
Socio-economic class / income
Marital status
(ii)
For Gasperton we have, using the census formula central death rate
=
25
1
(2, 000 + 2,100)
2
= 0.0122 .
For Great Hawking we have central death rate
=
(iii)
21
1
(1, 770 + 1, 674)
2
= 0.0122 .
Let the death rate for smokers in Gasperton be γ s , and that for non-smokers be γ n .
We therefore have
0.5 γ s + 0.5 γ n = 0.0122
γ s = 1.4 γ n
Page 11Subject CT4 (Models) – %%%%%%%%%%%%%%%%%%%%%%%%%%%%%%%%%%%%5
and hence
0.5(1.4) γ n + 0.5 γ n = 0.0122
γ n = 0.0122
= 0.0102
1.2
γ s = 0.0122(1.4)
= 0.0142
1.2
Let the death rate for smokers in Great Hawking be \upsilon s , and that for
non-smokers be \upsilon n .
We therefore have
0.2 \upsilon s + 0.8 \upsilon n = 0.0122
\upsilon s = 1.4 \upsilon n
and hence
0.2(1.4) \upsilon n + 0.8 \upsilon n = 0.0122
(iv)
\upsilon n = 0.0122
= 0.0113
1.08
\upsilon s = 0.0122(1.4)
= 0.0158
1.08
The company would do better to vary the premiums on the basis of geographical area,
as it is clear that death rates in Great Hawking for both smokers and non-smokers are higher than those in Gasperton.
If the company does not differentiate its prices on the basis of geographical area, it may lose business in Gasperton to a rival company which does differentiate;
converselY_in Great Hawking it may attract new business from rival companies, but will underprice the product and hence risk its life assurance fund becoming insolvent.
There are relatively little data, so it might be worth adopting a “wait and see”
approach.
Page 12Subject CT4 (Models) – %%%%%%%%%%%%%%%%%%%%%%%%%%%%%%%%%%%%5
1.4 times the death rate will not translate as 1.4 times the premium. The difference
may me relatively small, (although it is a 25 year term assurance so it probablY_is
pretty significant).
Most candidates scored highly on parts (i) and (ii). Part (iii) was very poorly answered. A large number of candidates misinterpreted the question as meaning that the ratio of the
numbers of deaths to smokers and non-smokers was 1.4. This works for Gasperton because there are equal numbers of smokers and non-smokers in the exposed-to-risk, but for Great
Hawking it produces incorrect results. Only a minority of candidates made a serious attempt
at part (iv). Credit was given for any sensible comments in part (iv) which were consistent
with the answers given to parts (ii) and (iii).
\newpage
10
(i)
1. Single
2. Married
3. Dead
(ii)
{ (
12
13
L ∝ exp −μ − μ
) ν } exp { ( −μ
1
23
−μ
21
) ν } ( μ ) ( μ ) ( μ ) ( μ )
2
12
d 12
21
d 21
13
d 13
23
d 23
where
μ ij is the transition intensity from state i to state j
ν i is the total observed waiting time in state i
d ij is the number of transitions from state i to state j
Page 13Subject CT4 (Models) – %%%%%%%%%%%%%%%%%%%%%%%%%%%%%%%%%%%%5
(iii)
Taking the logarithm of the likelihood we get
log e ( L ) = −μ 13 ν 1 + d 13 log e ( μ 13 ) + terms not involving μ 13 .
13
Differentiate with respect to μ
d ln( L )
d μ 13
1
= −ν +
d 13
μ 13
.
Setting this to zero we obtain
μ ˆ 13 =
d 13
ν 1
.
To check it is a maximum differentiate again giving
d 2 log e ( L )
( d μ 13 ) 2
(iv)
=−
d 13
( μ 13 ) 2
which is always negative.
The maximum likelihood estimate of μ 13 is 12/10,298 = 0.001165.
The variance is
− 1
d 2 ln( L )
( d μ 13 ) 2
= 12
10, 298 2
= 1.13 x 10 -7 .
This was the best answered question on the paper, with most candidates scoring at least 9 of
the 11 marks available. In part (ii) many candidates omitted the constant of proportionality.
In part (iv) the question says “estimate”, so we needed some indication of where the answers
came from for full marks.
