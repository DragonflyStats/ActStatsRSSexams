\documentclass[a4paper,12pt]{article}

%%%%%%%%%%%%%%%%%%%%%%%%%%%%%%%%%%%%%%%%%%%%%%%%%%%%%%%%%%%%%%%%%%%%%%%%%%%%%%%%%%%%%%%%%%%%%%%%%%%%%%%%%%%%%%%%%%%%%%%%%%%%%%%%%%%%%%%%%%%%%%%%%%%%%%%%%%%%%%%%%%%%%%%%%%%%%%%%%%%%%%%%%%%%%%%%%%%%%%%%%%%%%%%%%%%%%%%%%%%%%%%%%%%%%%%%%%%%%%%%%%%%%%%%%%%%

\usepackage{eurosym}
\usepackage{vmargin}
\usepackage{amsmath}
\usepackage{graphics}
\usepackage{epsfig}
\usepackage{enumerate}
\usepackage{multicol}
\usepackage{subfigure}
\usepackage{fancyhdr}
\usepackage{listings}
\usepackage{framed}
\usepackage{graphicx}
\usepackage{amsmath}
\usepackage{chngpage}

%\usepackage{bigints}
\usepackage{vmargin}

% left top textwidth textheight headheight

% headsep footheight footskip

\setmargins{2.0cm}{2.5cm}{16 cm}{22cm}{0.5cm}{0cm}{1cm}{1cm}

\renewcommand{\baselinestretch}{1.3}

\setcounter{MaxMatrixCols}{10}

\begin{document}


%%- Question 5
For a particular investigation the hazard of mortalitY_is assumed to take the form:
h ( t ) = A + Bt
where A and B are constants and t represents time.
For each life i in the investigation (i = 1, ..., n) information was collected on the
length of time the life was observed t i and whether the life exited due to death ( \delta  i = 1
if the life died, 0 otherwise).
\begin{enumerate}
\item (i)
Show that the likelihood of the data is given by:
n
L = \product^{n}_{i = 1} ( A + Bt i ) \delta  i exp[ − At i −
i = 1
6
1 2
Bt i ] .
2
\item 
(ii) Derive two simultaneous equations from which the maximum likelihood
estimates of the parameters A and B can be calculated.
\end{enumerate}
%%%%%%%%%%%%%%%%%%%%%%%%%%%%%%%%%%%%%%%%%%%%%%%%%%%%%%%%%%%%%%%%%%%%%%%%%%%%%%%%%%%%%%%%%%
\newpage
5
(i)
The likelihood of the data is given by:
n
L = \product^{n}_{i = 1} f ( t i ) \delta  i S ( t i ) 1 −\delta  i ,
i = 1
where f(t i ) is the probability density function and S(t i ) is the survivor function.
Since f(t i ) is related to the hazard function by
f(t i ) = h(t i ) S(t i )
the likelihood can be rewritten:

\[L = \product^{n}_{i = 1} h ( t i ) \delta  i S ( t i ).\]
%%%%%%%%%%%%5
Since
⎡ t i
⎤
1
S ( t i ) = exp ⎢ − ∫ h ( r ) dr ⎥ = exp[ − At i − Bt i 2 ],
2
⎢ r = o
⎥
⎣
⎦
n
1
L = \product^{n}_{i = 1} ( A + Bt i ) \delta  i exp[ − At i − Bt i 2 ] as required.
2
i = 1
(ii)
The log likelihood is given by:
n
1
⎡
⎤
log L = ∑ ⎢ \delta  i log( A + Bt i ) − At i − Bt i 2 ⎥ .
2
⎦
i = 1 ⎣
We are trying to maximise likelihood with respect to two parameters,
so need partial differentials with respect to A and B:
n
⎡ \delta  i
⎤
∂
log L = ∑ ⎢
− t i ⎥ ,
∂ A
⎦
i = 1 ⎣ A + Bt i
Page 5Subject CT4 (Models) – April 2012 – Examiners’ Report
n
⎡ \delta  t
1 ⎤
∂
log L = ∑ ⎢ i i − t i 2 ⎥ .
∂ B
2 ⎦
i = 1 ⎣ A + Bt i
The simultaneous equations satisfied by the MLEs are obtained by
setting these to zero:
n
⎡
⎤
\delta 
∑ ⎢ A + i Bt i − t i ⎥ = 0,
i = 1 ⎣
n
⎡ \delta  t
⎦
1
⎤
∑ ⎢ A + i Bt i i − 2 t i 2 ⎥ = 0.
i = 1 ⎣
⎦
In part (i) many candidates failed to explain where the components of the likelihood came
from by explaining the different contributions of the lives who were observed to die and those
who were not. In part (ii) credit was given for knowing the correct method even if this was
not executed. Credit was also given for differentiating a second time and showing that the
second derivatives were negative (and hence that we do have maxima), even though this was
not required for full marks.

\end{document}
