3 (i) State the principle of correspondence as it applies to mortality rates. 
A life insurance company has the following data:
  Number of policies in force on
1 January 1 January 1 July 1 January
Age last birthday 2009 2010 2010 2011
49 2,000 2,100 2,300 2,500
50 2,100 2,200 2,300 2,400
51 2,300 2,400 2,500 2,600
Number of deaths classified by age next birthday and calendar year
Age next birthday 2009 2010
49 175 200
50 200 225
51 225 235
(ii) Estimate, using these data, the force of mortality at age 50 next birthday for
the period 1 January 2009 to 1 January 2011. 
(iii) State the exact age to which your answer to part (ii) relates. 
[Total 7]

%%%%%%%%%%%%%%%%%%%%%%%%%%%%%%%%%%%%%%%%%%%%%%%%%%%%%%%%%%%%%%%%%
Page 5
3
(i) The principle of correspondence states that a life should be included in the
denominator of the rate at time t if and only if, were that life to die at time t, his or her
death would be counted in the numerator.
(ii) In order for the exposed to risk to correspond to the deaths data, it needs to be on an
age next birthday basis.
The exposed to risk at age x next birthday may be approximated using the census
approximation.
2
,
0
c
Ex = ∫ Px tdt
Using the trapezium rule (i.e. assuming the population varies linearly between
                          “census” dates) this may be evaluated as
,1/1/09 ,1/1/10 ,1/1/10 ,1/7/10 ,1/7/10 ,1/1/11
1 ( ) 1 ( ) 1 ( )
2 4 4
c
Ex = Px + Px + Px + Px + Px + Px
,1/1/09 ,1/1/10 ,1/7/10 ,1/1/11
1 3 1 1
2 x 4 x 2 x 4 x = P + P + P + P
where Px,t is the population aged x next birthday at time t.
But, in this case, we have data on an age last birthday basis.
If P*x,t is the population aged x last birthday at time t, then
Px,t = P*x−1,t
and the exposed to risk becomes
1,1/1/09 1,1/1/10 1,1/7/10 1,1/1/11
1 * 3 * 1 * 1 *
  2 4 2 4
c
Ex = P x− + P x− + P x− + P x−
So, using the data given, the exposed to risk we need at age 50 is
1 (2,000) 3 (2,100) 1 (2,300) 1 (2,500) 4,350
2 4 2 4
c
Ex = + + + =
  and the estimated force of mortality at age 50 next birthday is
50
ˆ 200 225 0.0977
4,350
+
  μ = =
  Subject CT4 (Models) – September 2012 – Examiners’ Report
Page 6
(iii) The estimate μˆ 50 applies to the middle of the rate interval,
which is exact age 49.5 years.
In part (ii) the question said “estimate” so some indication of how the answer was arrived at
was required for full credit. The correct numerical answer on its own was insufficient. Some
candidates noted that a correct exposed-to-risk could be calculated without using the July
2010 population figures. This was given full credit, provided a valid explanation of why the
July population figures were not needed was given. In part (iii) the question said “state” so
the full mark was awarded for 49.5. In part (iii) for full credit the answer had to be
consistent with what the candidate had done in part (ii).
%%%%%%%%%%%%%%%%%%%%%%%%%%%%%%%%%%%%%%%%%%%%%%%%%%%%%%%%%%%%%%%%%%%%%%%%%%%5
\newpage
4 (i) State one advantage of a semi-parametric model over a fully parametric one.

(ii) Write down a general expression for the Cox proportional hazards model,
defining all the terms you use. 
A life office is trying to understand the impact of certain factors on the lapse rates of
its policies. It has studied the lapse rates on a block of business subdivided by:
  • sex of policyholder (Male or Female)
• policy type (Term Assurance or Whole Life)
• sales channel (Internet, Direct Sales Force or Independent Financial Adviser)
CT4 S2012–3 PLEASE TURN OVER
The office has fitted a Cox proportional hazards model to the data and has calculated
the following regression parameters:
  Covariate Regression parameter
Female 0.2
Male 0
Term Assurance −0.1
Whole Life 0
Internet 0.4
Independent Financial Adviser −0.2
Direct Sales Force 0
(iii) State the sex/sales channel/policy type combination to which the baseline
hazard relates. 
A Term Assurance is sold to a Female by an Independent Financial Adviser.
(iv) Calculate the probability that this Term Assurance is still in force after five
years given that 60% of Whole Life policies bought on the Internet by Males
have lapsed by the end of year five. 
[Total 8]

%%%%%%%%%%%%%%%%%%%%%%%%%%%%%%%%%%%%%%%%%%%%%%%%%%%%%%%%%%%%%%%%%%
  4
(i) We do not need to know the general shape of the hazard/distribution.
(ii) ( ) ( , ) 0 ( ) exp T
h t zi = h t βzi
h(t, zi) is the hazard at time t (or just h(t) is OK)
h0(t) is the baseline hazard
zi are covariates
β is a vector of regression parameters
(iii) Baseline hazard refers to a male sold a whole life policy by the direct sales force.
(iv) For the male policy
the probability still in force is 0.4.
Sum of parameters for male is 0.4
5
0
0
0.4 exp h (t) exp(0.4)dt
⎧⎪ ⎪⎫ = ⎨− ⎬
⎩⎪ ⎪⎭
∫
5
0
0
exp 1.49 h (t)dt
⎧⎪ ⎪⎫ = ⎨− ⎬
⎩⎪ ⎭⎪
∫
So
5
0
0
( ) ln 0.4
1.49
h t dt =
  − ∫
And for the female policy sum of parameters is -0.1
Subject CT4 (Models) – September 2012 – Examiners’ Report
Page 7
THEN EITHER
We therefore want
5
0
0
exp ( ) exp( 0.1) exp 0.905* ln 0.4
1.49
h t dt
⎧⎪ ⎪⎫ ⎧ ⎫ ⎨− − ⎬ = ⎨− ⎬ ⎩⎪ ⎪⎭ ⎩ − ⎭
∫
= 0.57364
OR
We therefore want{ } 0.1
0.4 0.5 0.4 0.4
e
e e
−
− −
= = 0.57364
Parts (i)-(iii) of this question were well answered. Answers to part (iv) were variable.
Common errors included working with the probability of having lapsed (i.e. 1 minus the
                                                                      probability of still being in force), and omission of the integral.



