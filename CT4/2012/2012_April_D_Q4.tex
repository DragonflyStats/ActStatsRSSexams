\documentclass[a4paper,12pt]{article}

%%%%%%%%%%%%%%%%%%%%%%%%%%%%%%%%%%%%%%%%%%%%%%%%%%%%%%%%%%%%%%%%%%%%%%%%%%%%%%%%%%%%%%%%%%%%%%%%%%%%%%%%%%%%%%%%%%%%%%%%%%%%%%%%%%%%%%%%%%%%%%%%%%%%%%%%%%%%%%%%%%%%%%%%%%%%%%%%%%%%%%%%%%%%%%%%%%%%%%%%%%%%%%%%%%%%%%%%%%%%%%%%%%%%%%%%%%%%%%%%%%%%%%%%%%%%

\usepackage{eurosym}
\usepackage{vmargin}
\usepackage{amsmath}
\usepackage{graphics}
\usepackage{epsfig}
\usepackage{enumerate}
\usepackage{multicol}
\usepackage{subfigure}
\usepackage{fancyhdr}
\usepackage{listings}
\usepackage{framed}
\usepackage{graphicx}
\usepackage{amsmath}
\usepackage{chngpage}

%\usepackage{bigints}
\usepackage{vmargin}

% left top textwidth textheight headheight

% headsep footheight footskip

\setmargins{2.0cm}{2.5cm}{16 cm}{22cm}{0.5cm}{0cm}{1cm}{1cm}

\renewcommand{\baselinestretch}{1.3}

\setcounter{MaxMatrixCols}{10}

\begin{document}



%% 4
\large
\noindent A new drug treatment for patients suffering from a chronic skin disease with visible symptoms was tested. The drug was administered through a daily dose for the duration of the trial. As soon as the drug regime started, the symptoms disappeared in all patients, but after some time had a tendency to reappear as the agent causing the disease developed resistance to the drug. The trial lasted for six months.

The data below show the number of patients experiencing a return of their symptoms
in each month after the drug regime started.

Month Number of patient-months
exposed to risk Number of patients experiencing
a return of their symptoms
1
2
3
4
5
6 200
190
175
150
135
125 5
8
15
10
6
3

\begin{enumerate}
\item (i)
Calculate the hazard of symptoms returning in each month.

As part of the investigation, it is desired to assess the impact of certain risk factors on the hazard of symptoms returning. It is suggested that to achieve this, the hazard could be modelled using either a Gompertz model or a semi-parametric model.
\item (ii)
CT4 A2012–2
Comment on the use of each of these models in this situation.
\end{enumerate}
\newpage
%%%%%%%%%%%%%%%%%%%%%%%%%%%%%%%%%%%%%%%%%%%%%%%%%%%%

\newpage

4
(i) Month 1
Month 2
Month 3
Month 4
Month 5
Month 6
5/200 = 0.025
8/190 = 0.042
15/175 = 0.086
10/150 = 0.067
6/135 = 0.044
3/125 = 0.024
(ii) To assess the impact of risk factors, a proportional hazards model would be useful because of its simple interpretation or because it allows the effect of each individual risk factor to be assessed.
The Gompertz model can be framed as a proportional hazards model, as can a semi-parametric model (such as the Cox model).
The Gompertz model would not be appropriate here, as it has a monotonically increasing or decreasing hazard,
whereas it is clear from part (i) that the hazard of symptoms returning first rises and then falls with duration.
\medskip
A semi-parametric model allows the shape of the hazard to be determined by the data.
The semi-parametric model would be better than the Gompertz in this case.

%%%%%%%%%%%%%%%%%%%%5
\newpage
In part (i) a minority of candidates subtracted half the deaths from the exposed-to-risk. Partial credit was given for this. Part (ii) 
was a higher skills question, and was poorly
attempted by many candidates. Only a small proportion related their answers to the data 
given and spotted that the empirical hazard calculated in part (i) was non-monotonic and so
the Gompertz model would be a poor fit. Hardly any candidates pointed out that the
Gompertz model can be framed as a proportional hazards model.
\end{document}
