\documentclass[a4paper,12pt]{article}

%%%%%%%%%%%%%%%%%%%%%%%%%%%%%%%%%%%%%%%%%%%%%%%%%%%%%%%%%%%%%%%%%%%%%%%%%%%%%%%%%%%%%%%%%%%%%%%%%%%%%%%%%%%%%%%%%%%%%%%%%%%%%%%%%%%%%%%%%%%%%%%%%%%%%%%%%%%%%%%%%%%%%%%%%%%%%%%%%%%%%%%%%%%%%%%%%%%%%%%%%%%%%%%%%%%%%%%%%%%%%%%%%%%%%%%%%%%%%%%%%%%%%%%%%%%%

\usepackage{eurosym}
\usepackage{vmargin}
\usepackage{amsmath}
\usepackage{graphics}
\usepackage{epsfig}
\usepackage{enumerate}
\usepackage{multicol}
\usepackage{subfigure}
\usepackage{fancyhdr}
\usepackage{listings}
\usepackage{framed}
\usepackage{graphicx}
\usepackage{amsmath}
\usepackage{chngpage}

%\usepackage{bigints}
\usepackage{vmargin}

% left top textwidth textheight headheight

% headsep footheight footskip

\setmargins{2.0cm}{2.5cm}{16 cm}{22cm}{0.5cm}{0cm}{1cm}{1cm}

\renewcommand{\baselinestretch}{1.3}

\setcounter{MaxMatrixCols}{10}

\begin{document}
\begin{enumerate}

9
10
A life office compared the mortality of its policyholders in the age range 30 to 60
years inclusive with a set of mortality rates prepared by the Continuous Mortality
Investigation (CMI). The mortality of the life office policyholders was higher than
the CMI rates at ages 30–35, 38–41, 45–50 and 54–59 years inclusive, and lower than
the CMI rates at all other ages in the age range.
(i) Perform two tests of the null hypothesis that the underlying mortality of the
life office policyholders is represented by the CMI rates.
[7]
(ii) Comment on your results from part (i).
(ii) Explain the problem which duplicate policies cause in the context of the CMI
mortality investigations.

%% -- [Total 12]
(i) State the Markov property.


A certain non-fatal medical condition affects adults. Adults with the condition suffer
frequent episodes of blurred vision. A study was carried out among a group of adults
known to have the condition. The study lasted one year, and each participant in the
study was asked to record the duration of each episode of blurred vision. All
participants remained under observation for the entire year.
The data from the study were analysed using a two-state Markov model with states:
1.
2.
not suffering from blurred vision.
suffering from blurred vision.
Let the transition rate from state i to state j at time x+t be \mu ijx + t , and let the probability
that a person in state i at time x will be in state j at time x+t be t p ij x .
(ii)
Derive from first principles the Kolmogorov forward equation for the
transition from state 1 to state 2.

The results of the study were as follows:
Participant-days in state 1
21,650
Participant-days in state 2
5,200
Number of transitions from state 1 to state 2 4,330
Number of transitions from state 2 to state 1 4,160
Assume the transition intensities are constant over time.
(iii) Calculate the maximum likelihood estimates of the transition intensities from
state 1 to state 2 and from state 2 to state 1.

(iv) Estimate the probability that an adult with the condition who is presently not
suffering from blurred vision will be suffering from blurred vision in 3 days’
time.

%% -- [Total 13]

%%%%%%%%%%%%%%%%%%%%%%%%%%%%%%%%%%%%%%%%%%%%%%%%%%%%%%%%%%%%%%%%%%%%%%%%%%%%%%%%%%%%%
\newpage

9
(i)
Signs Test
Under the null hypothesis that the underlying mortality of the life office policyholders
is the same as the CMI mortality,
the number of positive deviations is distributed Binomial( m ,0.5)
THEN EITHER ALTERNATIVE 1 (NORMAL APPROXIMATION)
Here we have m = 31, so as m > 20 we can use the Normal approximation, that the
number of positive deviations is distributed Normal ( m /2, m /4).
⎛ 31 31 ⎞
the number of positive deviations is Normal ⎜ , ⎟ .
⎝ 2 4 ⎠
In this case we have 22 positive deviations.
The z -score corresponding to 22 is
22 − 15.5 6.5
=
= 2.33
2.78
7.75
OR Using a continuity correction
21.5 − 15.5
6
=
= 2.16
2.78
7.75
Using a 2-tailed test, we reject the null hypothesis at the 5% level of
significance if | z |>1.96.
The z -score corresponding to 22 is
Since 2.33 (or 2,16) > 1.96 we reject the null hypothesis.
OR ALTERNATIVE 2 (EXACT CALCULATION)
In this case we have 22 positive deviations.
The probability of observing exactly 22 positive deviations is 0.009388
OR
The probability of observing ≥ 22 positive deviations is 0.014725
Using a 2-tailed test, we reject the null hypothesis at the 5% level of significance if
the probability is <0.025
Since 0.014725 < 0.025 we reject the null hypothesis.
Grouping of Signs Test
Define the test statistic:
G = Number of groups of positive deviations.
Page 13 %%%%%%%%%%%%%%%%%%%%%%%%%%%%%%%%%
THEN EITHER ALTERNATIVE 1 (NORMAL APPROXIMATION)
Since m = 31 (which is ≥ 20), we can use a Normal approximation as follows:
⎛ n 1 ( n 2 + 1) ( n 1 n 2 ) 2 ⎞
,
G ~ Normal ⎜
3 ⎟
⎜ n + n
⎟ .
⎝ 1 2 ( n 1 + n 2 ) ⎠
In this case m = 31, n 1 = 22 and n 2 = 9.
Thus G ~ Normal ( 7.10,1.32 ) .
We have 4 groups of positive signs.
The z -score corresponding to 4 is
4 − 7.10 − 3.10
=
= − 2.70
1.15
1.32
Using a 1-tailed test, we reject the null hypothesis at the 5% level of significance if z
< −1.645.
Since −2.70 < −1.645 we reject the null hypothesis.
OR ALTERNATIVE 2 USING TABLE IN GOLD BOOK
m = total number of deviations = 31
n 1 = number of positive deviations = 22
n 2 = number of negative deviations = 9
We want k * the largest k such that
x
⎛ n − 1 ⎞⎛ n + 1 ⎞ ⎛ n + n ⎞
k = ∑ ⎜ 1 ⎟⎜ 2 ⎟ / ⎜ 1 2 ⎟ < 0.05
t ⎠ ⎝ n 1 ⎠
t = 1 ⎝ t − 1 ⎠⎝
We have 4 groups of positive signs.
The test fails at the 5% level if G ≤ k *
From the table in the Gold Book k * = 4
Since G is not greater than this, we reject the null hypothesis
(ii)
The life office’s rates are, overall, different from the CMI rates (actually they are
higher).
Additional tests are needed to examine the magnitude of the difference between the
two sets of rates.
Page 14 %%%%%%%%%%%%%%%%%%%%%%%%%%%%%%%%%
The shape of the life office’s mortality rates is also rather different from the CMI
schedule, and this might require further investigation,
OR
The Grouping of Signs test suggests clumping of the deviations.
It is possible that the difference between the shape of the two sets of rates is so small
in magnitude as to be negligible.
(iii)
We can no longer be sure that we are observing a collection of independent claims.
It is quite possible that two distinct death claims are the result of the death of the same
life.
The effect of this is to increase the variance of the number of claims,
by a factor which may depend on age.
This may affect tests based on standardised deviations.
Answers to this question were very disappointing, especially part (i) . The two tests were
often performed in a rather cursory way, with important steps being missed out. In part (i)
the null hypothesis only needed stating once. Common errors were to work with only 30 ages
or, more seriously, to use eight ages (i.e. to treat each run of consecutive ages of the same
sign as a single age).

