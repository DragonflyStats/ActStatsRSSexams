CT4 A2013–611
(i)
Explain what is meant by a time inhomogeneous Markov chain and give an
example of one.

A No Claims Discount system is operated by a car insurer. There are four levels of
discount: 0%, 10%, 25% and 40%. After a claim-free year a policy holder moves up
one level (or remains at the 40% level). If a policy holder makes one claim in a year
he or she moves down one level (or remains at the 0% level). A policy holder who
makes more than one claim in a year moves down two levels (or moves to or remains
at the 0% level). Changes in level can only happen at the end of each year.
(ii) Describe, giving an example, the nature of the boundaries of this process. 
(iii) (a)
State how many states are required to model this as a Markov chain.
(b)
Draw the transition graph.

The probability of a claim in any given month is assumed to be constant at 0.04. At
most one claim can be made per month and claims are independent.
(iv) Calculate the proportion of policyholders in the long run who are at the 25%
level.

(v) Discuss the appropriateness of the model.
END OF PAPER
CT4 A2013–7

%% -- [Total 15]

%%%%%%%%%%%%%%%%%%%%%%%%%%%%%%%%%%%%%%%%%%%%%%%%%%%%%%%%%%%%%%%%%%%%%%%%%%
\newpage


11
(i)
A Markov chain is a discrete time, discrete space Markov process
For a time-inhomogeneous Markov chain, the transition probabilities depend on the
absolute values of time, rather than just the time difference.
The value of “time” can be represented by many factors, for example the time of year,
age or duration.
An example might be a No Claims Discount scheme where the probability of a claim
reflects trends in accident frequency over time.
(ii)
Both boundaries are mixed as policyholders can either stay in that state for
consecutive periods or move back to another state.
E.g. When at the maximum 40% level, a policyholder who makes no claim will stay
there the next year, whereas one who makes one claim will drop to the 25% level and
one who makes more than one claim will drop to the 10% level.
%%-- Page 17
%%-- Subject CT4 
%%%%%%%%%%%%%%%%%%%%%%%%%%%%%%%%%%%%%%%a%%%%%%%%%%%%%%%%%%%%%%%%%%%%%%%%%
(iii)
Four states are required: 0%, 10%, 25% and 40%.
0%
(iv)
10%
25%
40%
Prob [no claims in year] = 0.96 12 = 0.6127
Prob [exactly 1 claim in year] = 0.96 11 (0.04) 12 = 0.3064
Prob [more than one claim in a year] = 1 – (0.6127 + 0.3064) = 0.0809
0
0 ⎞
⎛ 0.3873 0.6127
⎜
⎟
0.3873
0
0.6127
0 ⎟
π ⎜
=π
⎜ 0.0809 0.3064
0
0.6127 ⎟
⎜
⎟
0.0809 0.3064 0.6127 ⎠
⎝ 0
π 1 = 0.3873 π 1 + 0.3873 π 2 + 0.0809 π 3 (1)
π 2 = 0.6127 π 1 + 0.3064 π 3 + 0.0809 π 4
π 3 = 0.6127 π 2 + 0.3064 π 4 (2)
(3)
π 4 = 0.6127 π 3 + 0.6127 π 4 (4)
π 1 + π 2 + π 3 + π 4 = 1 . (5)
From (4) π 4 ( 1 − 0.6127 ) = 0.6127 π 3
so π 4 = 1.5820 π 3
From (3) π 2 = π 3 ( 1 − 0.3064 × 1.5820 ) / 0.6127
so π 2 = 0.8411 π 3
and (1) gives π 1 = π 3 (0.0809 + 0.3873 × 0.8411) / (1 − 0.3837)
so
Page 18
π 1 = 0.6637 π 3Subject CT4 %%%%%%%%%%%%%%%%%%%%%%%%%%%%%%%%%
Using (5) we get π 3 ( 0.6637 + 0.8411 + 1 + 1.5820 ) = 1
so π 3 = 0.2447
so π 4 = 1.5820 × 0.2447
π 4 = 0.3871
In the long run 24.47% of policyholders are at the 25% level.
(v)
Equal probability of an accident in every month is pretty unlikely.
Perhaps more accidents in winter when driving conditions are worse, or in summer,
when mileage is higher.
The probability of a second claim may differ from the first and may be dependent
upon the level the person is at (e.g. does it make a difference to the future premium?)
Claim probability may depend upon policyholder age/sex or car size/age, and on
many other factors (occupation, geographical area, marital status, mileage, where car
is stored, etc.)
Claim levels may be affected by the past history of a person's claims (so the process is
no longer Markov).
Unrealistic to assume at most one claim per month.
Parts (i), (iii) and (v) of this question were well answered, though in part (i) it was not often
clear how the examples given operated in discrete time. Part (ii) was very poorly attempted.
In part (iv) a common error was to assume that the 0.04 claim rate is annual. This gave the
answer that just under 4 per cent of policyholders were at the 25 per cent level. Candidates
who made this error were penalised for using incorrect probabilities, but were given full
credit for solving the equations to obtain the steady-state probabilities. In part (v) sensible
suggestions other than those listed were given credit.
END OF EXAMINERS’ REPORT
Page 19
