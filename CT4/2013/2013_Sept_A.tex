© Institute and Faculty of Actuaries1
2
Data are often subdivided when investigating mortality statistics.
(i) Explain why this is done.

(ii) Discuss one potential problem with sub-dividing mortality data.

(iii) List four factors which are commonly used to sub-divide mortality data.

%%--------------6]
The two football teams in a particular city are called United and City and there is
intense rivalry between them. A researcher has collected the following history on the
results of the last 20 matches between the teams from the earliest to the most recent,
where:
U indicates a win for United;
C indicates a win for City;
D indicates a draw.
UCCDDUCDCUUDUDCCUDCC
The researcher has assumed that the probability of each result for the next match
depends only on the most recent result. He therefore decides to fit a Markov chain to
this data.
3
(i) Estimate the transition probabilities for the Markov chain.

(ii) Estimate the probability that United will win at least two of the next three
matches against City.

%%--------------6]
(i) Define a Poisson process.


%%%%%%%%%%%%%%%%%%%%%%%%%%%%%%%%%%%%%
1
(i)
All our models and analyses are based on the assumption that we can observe groups
of identical lives (or at least, lives whose mortality characteristics are the same).
In practice, this is never possible.
However, we can at least subdivide our data according to characteristics known, from
experience, to have a significant effect on mortality.
This ought to reduce the heterogeneity of each class so formed.
(ii)
The number of lives in each subdivision may become small. This will lead to
estimates of mortality that are unreliable, with large standard errors.
OR
Information about the factors which affect mortality may be unavailable because it
was not asked on the insurance proposal form, or population census
OR
Information about the factors which affect mortality may be unreliable because
respondents gave inaccurate or false answers to questions.
(iii)
Sex
Age
Type of policy (which often reflects the reason for insuring)
Smoker/non-smoker status
Level of underwriting
Duration in force
Sales channel
Policy size
Occupation of policyholder
Known impairments
Postcode/geographical location
Marital status
Answers to part (i) of this question were disappointing, with few candidates relating the need
for homogeneity to the models we use. Parts (ii) and (iii) were generally well answered. In
part (ii) the instruction was to describe a single limitation, so no credit was given for second
or subsequent limitations. In part (iii) credit was given in some cases for wording different
from that indicated, such as “state of health”, or for certain other factors which are known to
affect mortality, and about which information is asked, for example, in population censuses.
However, genetic factors were not given credit.
Page 3%%%%%%%%%%%%%%%%%%%%%%%%%%%%%%%%%%%%%%%%%%%%%%%
2
(i)
Need to rearrange data as tally chart of next states:
Previous state Number where next state is:
U
C
D
U
C
D 1
11
11
11
111
111
111
11
1
So the transition probabilities are estimated as:
(ii)
From/To U C D
U
C
D 1/6
2/7
1/3 1/3
3/7
1/2 1/2
2/7
1/6
The possible sequences with at least 2 wins for United are:
UUU, UUC, UUD, DUU, CUU, UDU, UCU
The probabilities if the last match was won by City are:
UUU
UUC
UUD
DUU
CUU
UDU
UCU
=
=
=
=
=
=
=
2/7*1/6*1/6
2/7*1/6*1/3
2/7*1/6*1/2
2/7*1/3*1/6
3/7*2/7*1/6
2/7*1/2*1/3
2/7*1/3*2/7
=
=
=
=
=
=
=
1/126
1/63
1/42
1/63
1/49
1/21
4/147
OR (quicker)
UUX
DUU
CUU
UDU
UCU
=
=
=
=
=
2/7*1/6 = 1/21
2/7*1/3*1/6 = 1/63
3/7*2/7*1/6 = 1/49
2/7*1/2*1/3 = 1/21
2/7*1/3*2/7 = 4/147
where X refers to any result
Total = 140/882 = 10/63 = 0.15873
Answers to this question were generally disappointing. In both parts (i) and (ii) the question
said “estimate” so some explanation of where the answer is coming from was required for
full credit (e.g. in part (i) a statement that n ij /n i is needed, or a suitable diagram were
acceptable). A common error was to use 8 as the denominator for the C row. A more serious
error was to use 19 as the denominator for all the transition probabilities. Many candidates
Page 4%%%%%%%%%%%%%%%%%%%%%%%%%%%%%%%%%%%%%%%%%%%%%%%
did not take account of the fact that City had won the last match in the string given and thus
only used pairs, rather than triplets, of probabilities.
3
(i)
A Poisson process is a counting process in continuous time { N t , t  0} , where N t
records the number of occurrences of a type of event within the time interval from 0
to t.
Events occur singly and may occur at any time;
the probability that an event occurs during the short time interval from time t to time
t+h is approximately equal to \lambdah for small h, where the parameter \lambda is the rate of the
Poisson process.
OR
A Poisson process is an integer valued process in continuous time { N t , t  0} , where
Pr[ N t  h  N t  1| F t ]   h  o ( h )
Pr[ N t  h  N t  0 | F t ]  1   h  o ( h )
Pr[ N t  h  N t  0,1| F t ]  o ( h )
o ( h )
 0 .
h  0 h
and o ( h ) is such that lim
OR
A Poisson process with rate  is a continuous-time integer-valued process N t , t  0 ,
with the following properties:
N 0  0
N t has independent increments
N t has Poisson distributed stationary increments
n  t  s
    t  s    e  
,
P  N t  N s  n  
n !
(ii)
s  t , n  0,1,...
The probability that no bus arrives in the first 60 minutes is e  60/15  0.0183 .
By the memoryless property / by independence of increments / because the holding
times are exponential.
Page 5%%%%%%%%%%%%%%%%%%%%%%%%%%%%%%%%%%%%%%%%%%%%%%%
(iii)
The expected time between buses is 15 minutes.
By independence of increments / memoryless property this is the time Mr Bean can
expect to wait for the second bus.
(iv)
The rate at which buses arrive per 10 minute period is 10/15.
Therefore the probability of no buses arriving between 1.10 and 1.20 p.m. is
e  10/15  0.5134 .
The probability of one bus arriving is e  10/15 (10 /15)  0.3423 .
The probability of two or more buses arriving is therefore
1  0.5134  0.3423  0.1443 .
Answers to this question were disappointing. Most candidates managed to score reasonably
well on part (i). In parts (ii) and (iii) some explanations of the answers were required. In
part (iv) several candidates calculated the probability of exactly two buses arriving.
Candidates who used an incorrect rate in part (ii) could score full credit for part (iv) if they
correctly calculated the probability of two or more buses arriving given the incorrect rate.
