\documentclass[a4paper,12pt]{article}

%%%%%%%%%%%%%%%%%%%%%%%%%%%%%%%%%%%%%%%%%%%%%%%%%%%%%%%%%%%%%%%%%%%%%%%%%%%%%%%%%%%%%%%%%%%%%%%%%%%%%%%%%%%%%%%%%%%%%%%%%%%%%%%%%%%%%%%%%%%%%%%%%%%%%%%%%%%%%%%%%%%%%%%%%%%%%%%%%%%%%%%%%%%%%%%%%%%%%%%%%%%%%%%%%%%%%%%%%%%%%%%%%%%%%%%%%%%%%%%%%%%%%%%%%%%%

\usepackage{eurosym}
\usepackage{vmargin}
\usepackage{amsmath}
\usepackage{graphics}
\usepackage{epsfig}
\usepackage{enumerate}
\usepackage{multicol}
\usepackage{subfigure}
\usepackage{fancyhdr}
\usepackage{listings}
\usepackage{framed}
\usepackage{graphicx}
\usepackage{amsmath}
\usepackage{chngpage}

%\usepackage{bigints}
\usepackage{vmargin}

% left top textwidth textheight headheight

% headsep footheight footskip

\setmargins{2.0cm}{2.5cm}{16 cm}{22cm}{0.5cm}{0cm}{1cm}{1cm}

\renewcommand{\baselinestretch}{1.3}

\setcounter{MaxMatrixCols}{10}

\begin{document}

%%--------------9]
A motor insurer offers a No Claims Discount scheme which operates as follows. The discount levels are $\{0\%,25\%, 50\%, 60\%\}$. Following a claim-free year a policyholder moves up one discount level (or stays at the maximum discount). After a year with one or more claims the policyholder moves down two discount levels (or moves to, or stays in, the 0\% discount level).

The probability of making at least one claim in any year is 0.2.
\begin{enumerate}[(a)]
\item (i) Write down the transition matrix of the Markov chain with state space $\{0\%,25\%, 50\%, 60\%\}$.
\item 
(ii) State, giving reasons, whether the process is:
(a)
(b)
irreducible.
aperiodic.

\item 
(iii)
Calculate the proportion of drivers in each discount level in the stationary distribution.

\item 
The insurer introduces a “protected” No Claims Discount scheme, such that if the 60\% discount is reached the driver remains at that level regardless of how many
claims they subsequently make.

%%(iv)
%%CT4 S2013–3
Explain, without doing any further calculations, how the answers to parts (ii)
and (iii) would change as a result of introducing the “protected” No Claims
Discount scheme.
\end{enumerate}
%%--------------11]
%%%%%%%%%%%%%%%%%%%%
5
(i)
0 
 0.2 0.8 0


0.2 0 0.8 0 
P = 
 0.2 0
0 0.8 


 0 0.2 0 0.8 
where the levels are ordered 0%, 25%, 50%, 60%.
(ii)
(iii)
Page 8
(a) The chain is irreducible as it is clear that any state can eventually be reached from any other state.
(b) The process is aperiodic because, for example, the process can loop round in the 0\% or 60\% states giving no set return period to any state.
%---------------------------%
Stationary distribution  satisfies    P
0.2  0  0.2  25  0.2  50   0
0.8  0  0.2  60   25
0.8 25   50
0.8  50  0.8  60   60 (1)
Also  0   25   50   60  1 (5)
(2)
(3)
(4)%%%%%%%%%%%%%%%%%%%%%%%%%%%%%%%%%%%%%%%%%%%%%%%
Working in terms of  60
 50  0.25  60
 25 
 0 
5
 60
16
9
 60
64
Hence
(64  16  20  9)
 60  1
64
 9 
 
1  20 
So the e stationary distribution is
109  16 
 
 64 
and the proportion of drivers at each level is
0%
25%
50%
60%
%------------------------------------------%
(iv)
9/109 = 0.08257
20/109 = 0.18349
16/109 = 0.14679
64/109 = 0.58716.
The 60\% discount level becomes an absorbing state and so it is no longer irreducible. However it is still aperiodic because you cannot get out of the absorbing state 60\% and the other states still have no period.
The process would now be stationary when all drivers are in the absorbing 60\%
discount level.
OR
The new stationary distribution is [0,0,0,1] because the 60% state is now absorbing.
%%%%%%%%%%%%%%%%%%%%%%%%%%5
\newpage
This question was well answered, with many candidates scoring close to full marks. In part (iii) the correct numerical probabilities scored full marks, provided that it was clear to which level each probability applied. In part (iv) some candidates made vague statements that the
probability of being in the 60% state would increase. While this is true, it was not given full
credit, as the key point is that the stationary distribution has everyone in the 60% state.
%%-- Page 9%%%%%%%%%%%%%%%%%%%%%%%%%%%%%%%%%%%%%%%%%%%%%%%

\end{document}
