\documentclass[a4paper,12pt]{article}

%%%%%%%%%%%%%%%%%%%%%%%%%%%%%%%%%%%%%%%%%%%%%%%%%%%%%%%%%%%%%%%%%%%%%%%%%%%%%%%%%%%%%%%%%%%%%%%%%%%%%%%%%%%%%%%%%%%%%%%%%%%%%%%%%%%%%%%%%%%%%%%%%%%%%%%%%%%%%%%%%%%%%%%%%%%%%%%%%%%%%%%%%%%%%%%%%%%%%%%%%%%%%%%%%%%%%%%%%%%%%%%%%%%%%%%%%%%%%%%%%%%%%%%%%%%%

\usepackage{eurosym}
\usepackage{vmargin}
\usepackage{amsmath}
\usepackage{graphics}
\usepackage{epsfig}
\usepackage{enumerate}
\usepackage{multicol}
\usepackage{subfigure}
\usepackage{fancyhdr}
\usepackage{listings}
\usepackage{framed}
\usepackage{graphicx}
\usepackage{amsmath}
\usepackage{chngpage}

%\usepackage{bigints}
\usepackage{vmargin}

% left top textwidth textheight headheight

% headsep footheight footskip

\setmargins{2.0cm}{2.5cm}{16 cm}{22cm}{0.5cm}{0cm}{1cm}{1cm}

\renewcommand{\baselinestretch}{1.3}

\setcounter{MaxMatrixCols}{10}

\begin{document}
\begin{enumerate}
\item 

The Shining Light company has developed a new type of light bulb which it recently
tested. 1,000 bulbs were switched on and observed until they failed, or until 500
hours had elapsed. For each bulb that failed, the duration in hours until failure was noted. Due to an earth tremor after 200 hours, 200 bulbs shattered and had to be
removed from the test before failure.
The results showed that 10 bulbs failed after 50 hours, 20 bulbs failed after 100 hours,
50 bulbs failed after 250 hours, 300 bulbs failed after 400 hours and 50 bulbs failed
after 450 hours.
\begin{enumerate}
\item (i) Calculate the Kaplan-Meier estimate of the survival function, S(t), for the light
bulbs in the test.
\item
(ii) Sketch the Kaplan-Meier estimate calculated in part (i).
\item 
(iii) Estimate the probability that a bulb will not have failed after each of the
following durations: 300 hours, 400 hours and 600 hours. If it is not possible
to obtain an estimate for any of the durations without additional assumptions,
explain why.
\end{enumerate}

%%%%%%%%%%%%%%%%%%%%%%%%%%%%
7
(i)
t j N j d j c j d j / N j 1 - d j / N j 0
50
100
200
250
400
450 1,000
1,000
990
970
770
720
420 10
20
0
50
300
50 0
0
200
0
0
370 0.0100
0.0202 0.9900
0.9798 or 99/100
or 97/99
0.0649
0.4167
0.1190 0.9351
0.5833
0.8810 or 72/77
or 7/12
or 37/42
The Kaplan-Meier estimate is S ˆ ( t ) = ∏ (1 −
t j ≤ t
t
d j
n j
)
Kaplan-Meier estimate of S(t)
0 ≤ t < 50
50 ≤ t < 100
100 ≤ t < 250
250 ≤ t < 400
400 ≤ t < 450
450 ≤ t < 500
1.0000
0.9900
0.9700
0.9070
0.5291
0.4661
or 1
or 99/100
or 97/100
or 1,746/1,925
or 291/550
or 3,589/7,700
(ii)
1
0.9
0.8
0.7
0.6
S(t) 0.5
0.4
0.3
0.2
0.1
0
0
100
200
300
400
500
Duration t
Page 9 %%%%%%%%%%%%%%%%%%%%%%%%%%%%%%%%%
(iii)
S(300) = 0.9070.
S(400) = 0.5291.
S(600) cannot be estimated without additional assumptions
as it lies outside the range of our data.

%%%%%%%%%%%%%%%%%%%%%%%%%%%%%%%%%%%%%%%%%%%%%%%%%%%%%%%%%%%%%%%%%%%%%%%%%%%5
\newpage

%% -- [Total 11]
This question was very well answered, with many candidates scoring 10 or more marks out of
a possible 11. Some of the sketches in part (ii) were very scrappy and were penalised:
though great accuracy was not required, the sketch did need to be sufficiently clear to
demonstrate that the candidate understood the nature of the function being plotted. In part
(iii) some candidates suggested an assumption which would enable them to give an answer
for S(600). Such candidates were given full credit provided they explained why the
assumption was needed, and provided that the stated assumption was consistent with the
numerical answer offered.
\end{document}
