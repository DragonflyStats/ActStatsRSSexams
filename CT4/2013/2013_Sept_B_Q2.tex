\documentclass[a4paper,12pt]{article}

%%%%%%%%%%%%%%%%%%%%%%%%%%%%%%%%%%%%%%%%%%%%%%%%%%%%%%%%%%%%%%%%%%%%%%%%%%%%%%%%%%%%%%%%%%%%%%%%%%%%%%%%%%%%%%%%%%%%%%%%%%%%%%%%%%%%%%%%%%%%%%%%%%%%%%%%%%%%%%%%%%%%%%%%%%%%%%%%%%%%%%%%%%%%%%%%%%%%%%%%%%%%%%%%%%%%%%%%%%%%%%%%%%%%%%%%%%%%%%%%%%%%%%%%%%%%

\usepackage{eurosym}
\usepackage{vmargin}
\usepackage{amsmath}
\usepackage{graphics}
\usepackage{epsfig}
\usepackage{enumerate}
\usepackage{multicol}
\usepackage{subfigure}
\usepackage{fancyhdr}
\usepackage{listings}
\usepackage{framed}
\usepackage{graphicx}
\usepackage{amsmath}
\usepackage{chngpage}

%\usepackage{bigints}
\usepackage{vmargin}

% left top textwidth textheight headheight

% headsep footheight footskip

\setmargins{2.0cm}{2.5cm}{16 cm}{22cm}{0.5cm}{0cm}{1cm}{1cm}

\renewcommand{\baselinestretch}{1.3}

\setcounter{MaxMatrixCols}{10}

\begin{document}
\begin{enumerate}

2
Data are often subdivided when investigating mortality statistics.
(i) Explain why this is done.
[2]
(ii) Discuss one potential problem with sub-dividing mortality data.
[2]
(iii) List four factors which are commonly used to sub-divide mortality data.
[2]
[Total 6]
The two football teams in a particular city are called United and City and there is
intense rivalry between them. A researcher has collected the following history on the
results of the last 20 matches between the teams from the earliest to the most recent,
where:
U indicates a win for United;
C indicates a win for City;
D indicates a draw.
UCCDDUCDCUUDUDCCUDCC
The researcher has assumed that the probability of each result for the next match
depends only on the most recent result. He therefore decides to fit a Markov chain to
this data.
3
(i) Estimate the transition probabilities for the Markov chain.
[3]
(ii) Estimate the probability that United will win at least two of the next three
matches against City.




%%%%%%%%%%%%%%%%%%%%%%%%%%%%%%%%%%%%%

Page 3%%%%%%%%%%%%%%%%%%%%%%%%%%%%%%%%%%%%%%%%%%%%%%%
2
(i)
Need to rearrange data as tally chart of next states:
Previous state Number where next state is:
U
C
D
U
C
D 1
11
11
11
111
111
111
11
1
So the transition probabilities are estimated as:
(ii)
From/To U C D
U
C
D 1/6
2/7
1/3 1/3
3/7
1/2 1/2
2/7
1/6
The possible sequences with at least 2 wins for United are:
UUU, UUC, UUD, DUU, CUU, UDU, UCU
The probabilities if the last match was won by City are:
UUU
UUC
UUD
DUU
CUU
UDU
UCU
=
=
=
=
=
=
=
2/7*1/6*1/6
2/7*1/6*1/3
2/7*1/6*1/2
2/7*1/3*1/6
3/7*2/7*1/6
2/7*1/2*1/3
2/7*1/3*2/7
=
=
=
=
=
=
=
1/126
1/63
1/42
1/63
1/49
1/21
4/147
OR (quicker)
UUX
DUU
CUU
UDU
UCU
=
=
=
=
=
2/7*1/6 = 1/21
2/7*1/3*1/6 = 1/63
3/7*2/7*1/6 = 1/49
2/7*1/2*1/3 = 1/21
2/7*1/3*2/7 = 4/147
where X refers to any result
Total = 140/882 = 10/63 = 0.15873
Answers to this question were generally disappointing. In both parts (i) and (ii) the question
said “estimate” so some explanation of where the answer is coming from was required for
full credit (e.g. in part (i) a statement that n ij /n i is needed, or a suitable diagram were
acceptable). A common error was to use 8 as the denominator for the C row. A more serious
error was to use 19 as the denominator for all the transition probabilities. Many candidates
Page 4%%%%%%%%%%%%%%%%%%%%%%%%%%%%%%%%%%%%%%%%%%%%%%%
did not take account of the fact that City had won the last match in the string given and thus
only used pairs, rather than triplets, of probabilities.
