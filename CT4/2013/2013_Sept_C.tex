6
\item (i)
Explain what is meant by censoring iN_the context of a mortalitY_investigation.

A trial was conducted oN_the effectiveness of a new cream to treat a skin condition.
100 sufferers applied the cream daily for four weeks or until their symptoms
disappeared if this happened sooner. Some of the sufferers left the trial before their
symptoms disappeared.
\item (ii)
Describe two types of censoring that are present and state to whom they apply.

The following data were collected.
7
Number of
sufferers Day symptoms
disappeared Number of
sufferers Day they left
the trial
2
1
1
2 6
7
10
14 3
1
3 2
10
13
\item (iii) Calculate the Nelson-Aalen estimate of the survival function for this trial. 
(iv) Sketch the survival function, labelling the axes.
(v) Estimate the probability that a person using the cream will still have symptoms
of the skin condition after two weeks.

%%%%%%%%%%%%%%%%%%%%%%%

6
\item (i) Censoring is the mechanism which prevents us from knowing when an individual
entered the investigation or the exact date of death.
\item (ii) Right Censoring. The trial is cut short after four weeks when some patients had still
not recovered.
OR
The trial is cut short when some patients left the trial before their symptoms
disappeared.
Type I Censoring. Censoring times are known in advance for all those patients still
not recovered at the end of the trial.
Random Censoring. The time at which patients left the trial before their symptoms
disappeared is a random variable.
Non-Informative Censoring. There is no reasoN_to believe that those who left the trial
had more or less chance of being cured by the cream thanthose who remained.
\item (iii)
Rearranging the data:
Day
People iN_trial
No of exits
Reason for exit
0
100
0
2
100
3
Left
6
97
2
cured
The Nelson-Aalen estimate is  t 
t j
0
2
6
7
10
13
14
n j
100
100
97
95
94
92
89
d j
0
0
2
1
1
0
2
7
95
1
cured

10
94
1
cured
d j
x j  x
n j
c j
0
3
0
0
1
3
0
d j /n j \lambda t
2/97
1/95
1/94 .020619
.031145
.041783
2/89 .064255
Since S ( t )  exp   t  we have
t
0 \leq t < 6
6 \leq t < 7
7 \leq t < 10
10 \leq t < 14
14 \leq t < 28
Page 10
10
93
1
left
S(t)
1
0.97959
0.96934
0.95908
0.93777
13
92
3
left
14
89
2
cured%%%%%%%%%%%%%%%%%%%%%%%%%%%%%%%%%%%%%%%%%%%%%%%
(iv)
(v)
The survival probability at t = 14 is 0.93777, so there is approximately a 94% chance
of still having symptoms after two weeks.
Many candidates answered this question well. In part (ii) Informative Censoring was
acceptable if a sensible argument was made for it, for example those who left may be allergic
to the cream and therefore less likely to be cured by it thanthose who remain. In part (ii)
some candidates did not answer both parts of the question (that is, both describing the
censoring and stating to whom it applied). In part (iii) and part (iv) it was expected that
candidates would recognise that no information about what happened after 28 days could be
gained from the data. In part (v) the answer given should be consistent with the S(t) estimated
in part (iii).
\end{document}
