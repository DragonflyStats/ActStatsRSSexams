\documentclass[a4paper,12pt]{article}

%%%%%%%%%%%%%%%%%%%%%%%%%%%%%%%%%%%%%%%%%%%%%%%%%%%%%%%%%%%%%%%%%%%%%%%%%%%%%%%%%%%%%%%%%%%%%%%%%%%%%%%%%%%%%%%%%%%%%%%%%%%%%%%%%%%%%%%%%%%%%%%%%%%%%%%%%%%%%%%%%%%%%%%%%%%%%%%%%%%%%%%%%%%%%%%%%%%%%%%%%%%%%%%%%%%%%%%%%%%%%%%%%%%%%%%%%%%%%%%%%%%%%%%%%%%%

\usepackage{eurosym}
\usepackage{vmargin}
\usepackage{amsmath}
\usepackage{graphics}
\usepackage{epsfig}
\usepackage{enumerate}
\usepackage{multicol}
\usepackage{subfigure}
\usepackage{fancyhdr}
\usepackage{listings}
\usepackage{framed}
\usepackage{graphicx}
\usepackage{amsmath}
\usepackage{chngpage}

%\usepackage{bigints}
\usepackage{vmargin}

% left top textwidth textheight headheight

% headsep footheight footskip

\setmargins{2.0cm}{2.5cm}{16 cm}{22cm}{0.5cm}{0cm}{1cm}{1cm}

\renewcommand{\baselinestretch}{1.3}

\setcounter{MaxMatrixCols}{10}

\begin{document}
%% -- [Total 7]
6
(i) State the form of the hazard function for the Cox Regression Model, defining
all the terms used.

(ii) State two advantages of the Cox Regression Model.

Susanna is studying for an on-line test. She has collected data on past attempts at the
test and has fitted a Cox Regression Model to the success rate using three covariates:
Employment Z 1 = 0 if an employee, and 1 if self-employed
Attempt
Z 2 = 0 if first attempt, and 1 if subsequent attempt
Study time
Z 3 = 0 if no study time taken, and 1 if study time taken
Having analysed the data Susanna estimates the parameters as:
Employment 0.4
Attempt
−0.2
Study time
1.15
Bill is an employee. He has taken study time and is attempting the test for the second time. Ben is self-employed and is attempting the test for the first time without taking
study time.
(iii)
Calculate how much more or less likely Ben is to pass, compared with Bill. 
Susanna subsequently discovers that the effect of the number of attempts is different
for employees and the self-employed.
(iv)
7
Explain how the model could be adjusted to take this into account.

%% -- [Total 9]

Page 7Subject CT4 %%%%%%%%%%%%%%%%%%%%%%%%%%%%%%%%%
6
(i)
\lambda(t:Z i ) = \lambda 0 (t) exp (\beta Z iT )
Where:
\lambda(t:Z i) is the hazard at time t
\lambda 0 (t) is the baseline hazard
Z i is a vector of covariates
\beta is a vector of regression parameters
(ii)
It ensures the hazard is always positive.
The log-hazard is linear.
You can ignore the shape of the baseline hazard and calculate the effect of covariates
directly from the data.
It is widely available in standard computer packages OR is a popular, well-established
model.
(iii)
Ben, self-employed, first attempt, no study has hazard \lambda 0 (t) exp(0.4)
Bill, employee, re-sit, study leave has hazard \lambda 0 (t) exp (0.95)
So Ben is only exp(−0.55) = 57.7% as likely to pass as Bill OR 42.3% less likely to
pass than Bill.
OR
Bill is 73% more likely to pass than Ben
(iv)
The model could be adjusted by including a covariate measuring the interaction
between the number of attempts and employment status.
The covariate would be equal to Z 1 Z 2 and would take the value 1 for a self-employed
person on his or her second or subsequent attempt, and 0 otherwise.
The effect of the number of attempts for an employee would be equal to exp(\beta 2 ),
where \beta 2 is the parameter related to Z 2 , For a self-employed person, the effect of the
number of attempts would be equal to exp(\beta 2 + \beta 3 ), where \beta 3 is the parameter related
to the interaction term.
This question was well answered by many candidates. In part (iii) the question asked
candidates to “calculate” so the correct numerical answer scored full credit. However a
common error was to use ambiguous or incorrect wording in the final comparison (e.g. Bill
Page 8Subject CT4 %%%%%%%%%%%%%%%%%%%%%%%%%%%%%%%%%
is 57.7 per cent less likely to pass than Ben). In part (iv) no credit was given for the addition
of covariates with no bearing on the interaction term. However redundant parameters were
not penalised provided the modification to the model allowed the interaction to be quantified.


\newpage

