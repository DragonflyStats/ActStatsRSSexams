\documentclass[a4paper,12pt]{article}

%%%%%%%%%%%%%%%%%%%%%%%%%%%%%%%%%%%%%%%%%%%%%%%%%%%%%%%%%%%%%%%%%%%%%%%%%%%%%%%%%%%%%%%%%%%%%%%%%%%%%%%%%%%%%%%%%%%%%%%%%%%%%%%%%%%%%%%%%%%%%%%%%%%%%%%%%%%%%%%%%%%%%%%%%%%%%%%%%%%%%%%%%%%%%%%%%%%%%%%%%%%%%%%%%%%%%%%%%%%%%%%%%%%%%%%%%%%%%%%%%%%%%%%%%%%%

\usepackage{eurosym}
\usepackage{vmargin}
\usepackage{amsmath}
\usepackage{graphics}
\usepackage{epsfig}
\usepackage{enumerate}
\usepackage{multicol}
\usepackage{subfigure}
\usepackage{fancyhdr}
\usepackage{listings}
\usepackage{framed}
\usepackage{graphicx}
\usepackage{amsmath}
\usepackage{chngpage}

%\usepackage{bigints}
\usepackage{vmargin}

% left top textwidth textheight headheight

% headsep footheight footskip

\setmargins{2.0cm}{2.5cm}{16 cm}{22cm}{0.5cm}{0cm}{1cm}{1cm}

\renewcommand{\baselinestretch}{1.3}

\setcounter{MaxMatrixCols}{10}

\begin{document}
\begin{enumerate}

3
(i) Estimate the transition probabilities for the Markov chain.
[3]
(ii) Estimate the probability that United will win at least two of the next three
matches against City.
[3]
[Total 6]
(i) Define a Poisson process.
[2]
A bus route in a large town has one bus scheduled every 15 minutes. Traffic
conditions in the town are such that the arrival times of buses at a particular bus stop
may be assumed to follow a Poisson process.
Mr Bean arrives at the bus stop at 12 midday to find no bus at the stop. He intends to
get on the first bus to arrive.
(ii)
Determine the probability that the first bus will not have arrived by 1.00 pm
the same day.
[2]
The first bus arrived at 1.10 pm but was full, so Mr Bean was unable to board it.
(iii) Explain how much longer Mr Bean can expect to wait for the second bus to
arrive.
[1]
(iv) Calculate the probability that at least two more buses will arrive between
1.10 pm and 1.20 pm.
[2]
[Total 7]

%%%%%%%%%%%%%%%%%%%%%%%%%%%%%%%%%%%%%%%%%%%%%%%%%%%%%%%%%%%


(i)
A Poisson process is a counting process in continuous time { N t , t  0} , where N t
records the number of occurrences of a type of event within the time interval from 0
to t.
Events occur singly and may occur at any time;
the probability that an event occurs during the short time interval from time t to time
t+h is approximately equal to \lambdah for small h, where the parameter \lambda is the rate of the
Poisson process.
OR
A Poisson process is an integer valued process in continuous time { N t , t  0} , where
Pr[ N t  h  N t  1| F t ]   h  o ( h )
Pr[ N t  h  N t  0 | F t ]  1   h  o ( h )
Pr[ N t  h  N t  0,1| F t ]  o ( h )
o ( h )
 0 .
h  0 h
and o ( h ) is such that lim
OR
A Poisson process with rate  is a continuous-time integer-valued process N t , t  0 ,
with the following properties:
N 0  0
N t has independent increments
N t has Poisson distributed stationary increments
n  t  s
    t  s    e  
,
P  N t  N s  n  
n !
(ii)
s  t , n  0,1,...
The probability that no bus arrives in the first 60 minutes is e  60/15  0.0183 .
By the memoryless property / by independence of increments / because the holding
times are exponential.
Page 5%%%%%%%%%%%%%%%%%%%%%%%%%%%%%%%%%%%%%%%%%%%%%%%
(iii)
The expected time between buses is 15 minutes.
By independence of increments / memoryless property this is the time Mr Bean can
expect to wait for the second bus.
(iv)
The rate at which buses arrive per 10 minute period is 10/15.
Therefore the probability of no buses arriving between 1.10 and 1.20 p.m. is
e  10/15  0.5134 .
The probability of one bus arriving is e  10/15 (10 /15)  0.3423 .
The probability of two or more buses arriving is therefore
1  0.5134  0.3423  0.1443 .
Answers to this question were disappointing. Most candidates managed to score reasonably
well on part (i). In parts (ii) and (iii) some explanations of the answers were required. In
part (iv) several candidates calculated the probability of exactly two buses arriving.
Candidates who used an incorrect rate in part (ii) could score full credit for part (iv) if they
correctly calculated the probability of two or more buses arriving given the incorrect rate.
