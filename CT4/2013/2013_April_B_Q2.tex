\documentclass[a4paper,12pt]{article}

%%%%%%%%%%%%%%%%%%%%%%%%%%%%%%%%%%%%%%%%%%%%%%%%%%%%%%%%%%%%%%%%%%%%%%%%%%%%%%%%%%%%%%%%%%%%%%%%%%%%%%%%%%%%%%%%%%%%%%%%%%%%%%%%%%%%%%%%%%%%%%%%%%%%%%%%%%%%%%%%%%%%%%%%%%%%%%%%%%%%%%%%%%%%%%%%%%%%%%%%%%%%%%%%%%%%%%%%%%%%%%%%%%%%%%%%%%%%%%%%%%%%%%%%%%%%

\usepackage{eurosym}
\usepackage{vmargin}
\usepackage{amsmath}
\usepackage{graphics}
\usepackage{epsfig}
\usepackage{enumerate}
\usepackage{multicol}
\usepackage{subfigure}
\usepackage{fancyhdr}
\usepackage{listings}
\usepackage{framed}
\usepackage{graphicx}
\usepackage{amsmath}
\usepackage{chngpage}

%\usepackage{bigints}
\usepackage{vmargin}

% left top textwidth textheight headheight

% headsep footheight footskip

\setmargins{2.0cm}{2.5cm}{16 cm}{22cm}{0.5cm}{0cm}{1cm}{1cm}

\renewcommand{\baselinestretch}{1.3}

\setcounter{MaxMatrixCols}{10}

\begin{document}

%%%%%%%%%%%%%%%%%%%%%%%%%%%%%%%%%%%%%%%%%%%%
2 In the context of a survival model:
3
(i) Define right censoring, Type I censoring and Type II censoring.
(ii) Give an example of a practical situation in which censoring would be
informative.


%%%%%%%%%%%%%%%%%%%%%%%%%%%%%%%%%%%%%%%%%%%%%%%%%%%%%%%%%%%%%%%%%%%%%%%%%%%%%%%%%%%%%%%%%

2
(i)
Right censoring. The duration to the event is not known exactly,
but is known to exceed some value.
OR
the censoring mechanism cuts short observations in progress.
Type I censoring. The durations at which observations will be censored are specified
in advance.
Type II censoring. Observation continues until a pre-determined number/proportion
of individuals have experienced the event of interest.
(ii)
An investigation of mortality based on life office data in which
individuals are censored who discontinue paying their premiums.
Page 3Subject CT4 %%%%%%%%%%%%%%%%%%%%%%%%%%%%%%%%%
Those whose premiums lapse tend, on average, to be in better health
than do those who carry on paying their premiums.
In part (ii) any suitable example was given credit. However, for full credit it was necessary
to describe a comparison between the risk of the event happening in the censored and
uncensored observations (e.g. “in better health than” or “less likely to die than”). Most
candidates made a good attempt at this question.
