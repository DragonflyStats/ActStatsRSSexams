%% -- [Total 7]5
Population censuses in a certain country are taken each year on the President’s
birthday, provided that the President’s astrological advisor deems the taking of a
census favourable. Censuses record the age of every inhabitant in completed years
(that is, curtate age). Deaths in this country are registered as they happen, and
classified according to age nearest birthday at the time of death.
Below are some data from the three most recent censuses.
Age in
completed
years Population
2006
(thousands) Population
2009
(thousands) Population
2010
(thousands)
64
65
66 300
290
280 320
310
300 350
330
320
Between the censuses of 2006 and 2009 there were a total of 3,000 deaths to
inhabitants aged 65 nearest birthday, and between the censuses of 2009 and 2010
there were a total of 1,000 deaths to inhabitants aged 65 nearest birthday.
(i)
Estimate, stating any assumptions you make, the death rate at age 65 years for
each of the following periods:
•
•
(ii)
CT4 A2013–3
the period between the 2006 and 2009 censuses
the period between the 2009 and 2010 censuses
Explain the exact age to which your estimates apply.


%% -- [Total 7]
6
(i) State the form of the hazard function for the Cox Regression Model, defining
all the terms used.

(ii) State two advantages of the Cox Regression Model.

Susanna is studying for an on-line test. She has collected data on past attempts at the
test and has fitted a Cox Regression Model to the success rate using three covariates:
Employment Z 1 = 0 if an employee, and 1 if self-employed
Attempt
Z 2 = 0 if first attempt, and 1 if subsequent attempt
Study time
Z 3 = 0 if no study time taken, and 1 if study time taken
Having analysed the data Susanna estimates the parameters as:
Employment 0.4
Attempt
−0.2
Study time
1.15
Bill is an employee. He has taken study time and is attempting the test for the second time. Ben is self-employed and is attempting the test for the first time without taking
study time.
(iii)
Calculate how much more or less likely Ben is to pass, compared with Bill. 
Susanna subsequently discovers that the effect of the number of attempts is different
for employees and the self-employed.
(iv)
7
Explain how the model could be adjusted to take this into account.

%% -- [Total 9]

%%%%%%%%%%%%%%%%%%%%
5
(i)
We adjust the exposed to risk so that the age definition corresponds with that of the
deaths data.
Let the population at age 65 nearest birthday be P 65 and let the central exposed to risk
c
at age 65 nearest birthday be E 65
.
In 2006 P 65 = 0.5(300,000 + 290,000) = 295,000
In 2009 P 65 = 0.5(320,000 + 310,000) = 315,000
Page 6Subject CT4 %%%%%%%%%%%%%%%%%%%%%%%%%%%%%%%%%
In 2010 P 65 = 0.5(350,000 + 330,000) = 340,000,
assuming that birthdays are uniformly distributed across calendar time.
Using the census approximation (trapezium method) for the period 2006-2009 then
assuming that the population varies linearly between census dates,
c
= 1.5(295,000 + 315,000) = 915,000
E 65
and for the period 2009–2010
c
= 0.5(315,000 + 340,000) = 327,500.
E 65
Assuming that the force of mortality is constant within each year of age
\mu 65 = 3,000
= 0.003279 for the period 2006–2009, and
915,000
\mu 65 = 1,000
= 0.003053 for the period 2009–2010.
327,500
We also assuming that the President doesn't change (so the birthday is on the same
day each year), or if the President does change the new President’s birthday is the
same as the birthday of the old President.
(ii)
The rate interval is the life year, starting at age x – 0.5.
The age in the middle of the rate interval is thus x, so the estimate relates
to exact age 65 years.
A common error in part (i) was to use equal time periods, whereas the period 2006-2009 is three years and 2009-2010 only one year. For full credit, the assumptions had to appear in
the script close to the relevant bit of calculation. Candidates who listed many assumptions, both necessary and unnecessary, in a block at the end of the answer were penalised. In part
(i), some candidates calculated q x rather than \mu x . Full credit was given for this provided that the initial exposed-to-risk was used as the denominator. In part (ii) the age to which q x
applies is 64.5 years (i.e. the age at the start of the rate interval), and for full credit the answers to parts (i) and (ii) had to be consistent.
Page 7Subject CT4 %%%%%%%%%%%%%%%%%%%%%%%%%%%%%%%%%
6
(i)
\lambda(t:Z i ) = \lambda 0 (t) exp (\beta Z iT )
Where:
\lambda(t:Z i) is the hazard at time t
\lambda 0 (t) is the baseline hazard
Z i is a vector of covariates
\beta is a vector of regression parameters
(ii)
It ensures the hazard is always positive.
The log-hazard is linear.
You can ignore the shape of the baseline hazard and calculate the effect of covariates
directly from the data.
It is widely available in standard computer packages OR is a popular, well-established
model.
(iii)
Ben, self-employed, first attempt, no study has hazard \lambda 0 (t) exp(0.4)
Bill, employee, re-sit, study leave has hazard \lambda 0 (t) exp (0.95)
So Ben is only exp(−0.55) = 57.7% as likely to pass as Bill OR 42.3% less likely to
pass than Bill.
OR
Bill is 73% more likely to pass than Ben
(iv)
The model could be adjusted by including a covariate measuring the interaction
between the number of attempts and employment status.
The covariate would be equal to Z 1 Z 2 and would take the value 1 for a self-employed
person on his or her second or subsequent attempt, and 0 otherwise.
The effect of the number of attempts for an employee would be equal to exp(\beta 2 ),
where \beta 2 is the parameter related to Z 2 , For a self-employed person, the effect of the
number of attempts would be equal to exp(\beta 2 + \beta 3 ), where \beta 3 is the parameter related
to the interaction term.
This question was well answered by many candidates. In part (iii) the question asked
candidates to “calculate” so the correct numerical answer scored full credit. However a
common error was to use ambiguous or incorrect wording in the final comparison (e.g. Bill
Page 8Subject CT4 %%%%%%%%%%%%%%%%%%%%%%%%%%%%%%%%%
is 57.7 per cent less likely to pass than Ben). In part (iv) no credit was given for the addition
of covariates with no bearing on the interaction term. However redundant parameters were
not penalised provided the modification to the model allowed the interaction to be quantified.
