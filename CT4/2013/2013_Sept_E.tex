Outside an apartment block there is a small car park with three parking spaces. A
prospective purchaser of an apartment in the block is concerned about how often he
would return in his car to find that there was no empty parking space available. He
decides to model the number of parking spaces free at any time using a time
homogeneous Markov Jump Process where:
• The probability that a car will arrive seeking a parking space in a short interval
dt is A.dt + o(dt).
• For each car which is currently parked, the probability that its owner drives the
car away in a short interval dt is B.dt + o(dt).
where A, B > 0.
(i) Specify the state space for the above process. 
(ii) Draw a transition graph of the process. 
(iii) Write down the generator matrix for the process. 
(iv) Derive the probability that, given all the parking spaces are full, they will
remain full for at least the next two hours. 
(v) Explain what is meant by a jump chain. 
(vi) Specify the transition matrix for the jump chain associated with this process.

Suppose there are currently two empty parking spaces.
(vii)
Determine the probability that all the spaces become full before any cars are
driven away.

(viii) Derive the probability that the car park becomes full before the car park
becomes empty.
(ix)
CT4 S2013–5

Comment on the prospective purchaser’s assumptions regarding the arrival
and departure of cars.


%%%%%%%%%%%%%%%%%%%%%%%%%%%%%
8
(i)
The state space is {0,1,2,3} where the number indicates the number of available
spaces.
(ii)
0
1
2
3 B
0
0 
  3 B


2 B
0 
 A  A  2 B
 0
A
 A  B B 


A
0
 A 
 0
(iii)
where the order of the rows/columns is {0, 1, 2, 3}.
(iv)
d
P 00 ( t )   3 BP 00 ( t ) (as probability of returning to state 0 not of interest)
dt
OR
 t

P 00 ( t )  exp    3 Bdt 
  0
 
P 00 ( t )  exp(  3 Bt )
P 00 (2)  exp(  6 B )
Page 14
3%%%%%%%%%%%%%%%%%%%%%%%%%%%%%%%%%%%%%%%%%%%%%%%
(v)
(vi)
If a Markov jump process X t is examined only at the times of transition, the resulting
process is called the jump chain associated with X t.
OR
A jump chain is each distinct state visited in the order visited where the time set is the
times when states are moved between.
0


 A A  2 B

0

 
0

1
0
A
0
2 B
A  2 B
A  B
0
0
1


0 

B
A  B 

0  
0
where the order of the rows/columns is {0, 1, 2, 3}.
(vii)
This is
A
A
.
A  B A  2 B
(viii) Consider the paths by which the car park can become full before it becomes
empty
Required probability = P 21 P 10  P 21 P 12 P 21 P 10  P 21 P 12 P 21 P 12 P 21 P 10  ......
= A
A 
A
2 B
A
2 B
A
2 B

.
1 
.
.
.

 .... 

A  B A  2 B  A  B A  2 B A  B A  2 B A  B A  2 B

= A
A
A
2 B 

.
/  1 
.
A  B A  2 B  A  B A  2 B  
OR
This can be done by defining a function of the probability full before empty from the
current state, say D x
Then D 0 = 1and D 3 =0
and D 1  D 0 .
D 2  D 1 .
A
2 B
 D 2 .
A  2 B
A  2 B
A
B
 D 3 .
A  B
A  B
Solving these gives
D 2 
A
A
A
2 B 

.
/  1 
.
A  B A  2 B  A  B A  2 B  
Page 15%%%%%%%%%%%%%%%%%%%%%%%%%%%%%%%%%%%%%%%%%%%%%%%
(ix)
A time inhomogeneous model may be more appropriate.
Residents may come and go at particular times, for example if they drive to work.
They are unlikely to be moving their car as regularly in the middle of the night
Independent arrivals questionable because a family might have two cars
arriving/leaving at the same time OR people might arrive and wait until a space
becomes available thus leading to a queue
The Markov assumption may not be valid because neighbours may know from at
experience when cars are moved and time their arrival accordingly.
The model assumes those parking cars are competent drivers, and do not park so as to
take up 2 spaces.
The problem can be worked in terms of the number of occupied spaces. This was not given
full credit for part (i) as the question said “model the number of spaces free, but could gain
full credit for the other parts. In part (ii) it was not necessary to mark the probabilities on the
diagram. A common error was to omit the 2s and 3s in the matrix in part (iii). Part (v) was
not as well answered as might have been expected, with many candidates writing vague
descriptions which did not make it clear that they understood what a jump chain is.
Overall, this question was poorly answered by many candidates. A large proportion of
candidates did not attempt parts (iii)–(viii).

\end{document}
