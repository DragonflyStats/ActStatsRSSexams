\documentclass[a4paper,12pt]{article}

%%%%%%%%%%%%%%%%%%%%%%%%%%%%%%%%%%%%%%%%%%%%%%%%%%%%%%%%%%%%%%%%%%%%%%%%%%%%%%%%%%%%%%%%%%%%%%%%%%%%%%%%%%%%%%%%%%%%%%%%%%%%%%%%%%%%%%%%%%%%%%%%%%%%%%%%%%%%%%%%%%%%%%%%%%%%%%%%%%%%%%%%%%%%%%%%%%%%%%%%%%%%%%%%%%%%%%%%%%%%%%%%%%%%%%%%%%%%%%%%%%%%%%%%%%%%

\usepackage{eurosym}
\usepackage{vmargin}
\usepackage{amsmath}
\usepackage{graphics}
\usepackage{epsfig}
\usepackage{enumerate}
\usepackage{multicol}
\usepackage{subfigure}
\usepackage{fancyhdr}
\usepackage{listings}
\usepackage{framed}
\usepackage{graphicx}
\usepackage{amsmath}
\usepackage{chngpage}

%\usepackage{bigints}
\usepackage{vmargin}

% left top textwidth textheight headheight

% headsep footheight footskip

\setmargins{2.0cm}{2.5cm}{16 cm}{22cm}{0.5cm}{0cm}{1cm}{1cm}

\renewcommand{\baselinestretch}{1.3}

\setcounter{MaxMatrixCols}{10}

\begin{document}
\begin{enumerate}


%%%%%%%%%%%%%%%%%%%%%%%%%%%%%%%%%%%%%%%%
10
(i)
The future development of the process depends only on the state currently occupied
and not on any previous history.
OR
P [ X t \in A ⏐ X s 1 = x 1 , X s 2 = x 2 , ..., X s n = x n , X s = x ] = P [ X t \in A ⏐ X s = x ]
for all times s 1 < s 2 < ... < s n < s < t , all states x 1 , x 2 , ..., x n , x in S and all subsets A of
S.
(ii)
Condition on the state occupied at x + t , using the Markov assumption:
t + dt
11
12
12
22
p 12
x = t p x dt p x + t + t p x dt p x + t
But by Law of Total Probability dt p x 22 + t = 1 − dt p x 21 + t , so
t + dt
11
12
12
21
p 12
x = t p x dt p x + t + t p x (1 − dt p x + t ) .
%%-- Page 15 %%%%%%%%%%%%%%%%%%%%%%%%%%%%%%%%%
Now, since by assumption, for small dt ,
o ( dt )
where lim
= 0 ,
dt → 0 dt
dt
p ij x + t = \mu ij x + t dt + o ( dt ) ,
we can substitute to give
t + dt
11 12
12
21
p 12
x = t p x \mu x + t dt + t p x (1 − \mu x + t dt ) + o ( dt )
12
12
12 21
= t p 11
x \mu x + t dt + t p x − t p x \mu x + t dt + o ( dt )
so that
t + dt
12
11 12
12 21
p 12
x − t p x = t p x \mu x + t dt − t p x \mu x + t dt + o ( dt ) .
Dividing by dt and taking limits gives
12
12 21
⎡
⎡ t p 11
⎤
p 12 − p 12 ⎤ d 12
x \mu x + t dt
t p x \mu x + t dt o ( dt )
lim ⎢ t + dt x t x ⎥ =
=
−
+
lim
p
⎢
⎥
t x
dt → 0 ⎢
dt → 0 ⎢
dt
dt
dt
dt ⎦ ⎥
⎣
⎣
⎦ ⎥ dt
12
12 21
= t p 11
x \mu x + t − t p x \mu x + t .
(iii)
State 1 to state 2: \mu 12 = 4,330/21,650 = 0.2.
State 2 to state 1: \mu 21 = 4,160/5,200 = 0.8.
d 12
11 12
12 21
12
12
12 21
t p x = t p x \mu − t p x \mu = \mu (1 − t p x ) − t p x \mu
dt
(iv)
and substituting the values from the answer to part (iii) gives
d 12
12
12
12
t p x = 0.2(1 − t p x ) − 0.8 t p x = 0.2 − t p x .
dt
d
12
12
t p x + t p x = 0.2
dt
d
⎡ d 12 ⎤ t
12 t
12 t
t
⎢ ⎣ dt t p x ⎥ ⎦ e + t p x e = dt ( t p x e ) = 0.2 e
t
t
t
p 12
x e = 0.2 e + C
Since 0 p 12
x = 0
%%-- Page 16 %%%%%%%%%%%%%%%%%%%%%%%%%%%%%%%%%
C = − 0.2
− t
and t p 12
x = 0.2 − 0.2 e .
− 3
For t = 3 days, t p 12
= 0.1900 .
x = 0.2 − 0.2 e
In part (ii) minor variations on the exact derivation given above were permitted, but all the
steps were required for full credit. In part (iii) some candidates attempted a solution using
integral equations. A relatively common example argued that the required probability could
be obtained from:
3 3 3
0 0 0
12
− 0.2 w
.0.2. e − 0.8(3 − w ) dw = 0.2 ∫ e − 2.4 e 0.6 w dw = 0.2 e − 2.4 ∫ e 0.6 w dw .
3 p 0 = ∫ e
Evaluating this integral produces 3 p 0 12 =
0.2 − 2.4 0.6 w 3 0.2 − 0.6 − 2.4
e ⎡ ⎣ e ⎤ ⎦ =
( e − e ) = 0.1527 .
0
0.6
0.6
%%%%%%%%%%%%%%%%%%%%%%%%%%%%%%%%%%%%%5
This is incorrect as it ignores the possibility that lives might oscillate between states 1and 2
between t = 0 and t = 3. It only considers those lives who move between states 1 and 2 with
exactly one transition and do not return to state 1. However, this alternative shows
considerable understanding of the process, and was given some credit.
\end{document}
