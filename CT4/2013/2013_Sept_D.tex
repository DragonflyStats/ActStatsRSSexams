%%--------------11]
(i) Explain why the Gompertz model is commonly used in investigations of
human mortality.


The following model of mortality was used in an investigation of the effects of where
someone lives and income on the risk of death.
log e \mu x = \alpha + \beta 0 x + \beta 1 U + \beta 2 I ,
where \mu x is the force of mortality at age x, U takes the value 1 if the person lives in an
urban area and 0 if the person lives in a rural area, I is annual income in US dollars,
and \alpha, \beta 0 , \beta 1 and \beta 2 are parameters.
(ii)
Show that the model is both a Gompertz model and a proportional hazards
model.

The estimates of the parameters were \alpha = −9.0 \beta 0 = 0.09, \beta 1 = 0.3 and \beta 2 = −0.0001.
(iii) Calculate the predicted force of mortality for an urban resident aged 40 years
with an annual income of $20,000.

(iv) Calculate the additional income that an urban resident must have in order to
have the same force of mortality as a rural resident of the same age.

CT4 S2013–48
(v) Calculate the 10-year survival probability for an urban resident aged 40 years
whose annual income is $20,000.

(vi) Determine the age of a rural resident with the same income as an urban
resident aged 40 years, who has the same chance of surviving for the next 10
years.

%%--------------14]

%%%%%%%%%%%%%%%%%%%%%%%%%

7
(i)
The Gompertz model is simple to understand and to apply, having only two
parameters.
It also fits human mortality at older ages well (e.g. 30–85 years).
(i) The Gompertz model is simple to understand and to apply, having only two
parameters.
It also fits human mortality at older ages well (e.g. 30–85 years).
(ii) loge \mu_{x}\;=\; \alpha \; + \;\beta_{0}x \; + \;\beta_1U \; + \;\beta_2I
So \mu_{x}\;=\; \operatorname{exp}(\alpha \; + \;\beta_{0}x \; + \;\beta_1U \; + \;\beta_2I) \;=\; \operatorname{exp}(\beta_{0}x)\operatorname{exp}(\alpha \; + \;\beta_1U \; + \;\beta_2I)
This is equal to $Bc^x$ where B \;=\; \operatorname{exp}(\alpha \; + \;\beta_1U \; + \;\beta_2I) and c = exp ß0, hence Gompertz.
\mu_{x}\;=\; \operatorname{exp}(\alpha \; + \;\beta_{0}x \; + \;\beta_1U \; + \;\beta_2I) \;=\; \operatorname{exp}(\alpha \; + \;\beta_{0}x)\operatorname{exp}(\beta_1U \; + \;\beta_2I)


EITHER
Hence the force of mortality factorises into a term \operatorname{exp}(a+\beta_0x) depending
on age x but not the covariates, and a term \operatorname{exp}(\beta_1U + \beta_2I) depending on
the covariates but not x, SO proportional hazards.
OR
Consider any two individuals, i and j, with values of the covariates U_{i} and Ii,
and U_{j} and Ij respectively. Then the hazards for individuals i and j at age x are
\mu_{x,i} \;=\; \operatorname{exp}(\alpha\;+\;\beta_0x)\operatorname{exp}(\beta_1U_{i} + \beta_2Ii )
and
\mu_{x, j} \;=\; \operatorname{exp}(\alpha\;+\;\beta_0x)\operatorname{exp}(\beta_1U_{j} + \beta_2I j )
The ratio between the hazards is thus

\begin{eqnarray*}
\frac{\mu_{x,i} }{  \mu_{x, j} } &=& 
\frac{\operatorname{exp}(\alpha\;+\;\beta_0x)\operatorname{exp}(\beta_1U_{i} + \beta_2Ii )
}{\operatorname{exp}(\alpha\;+\;\beta_0x)\operatorname{exp}(\beta_1U_{j} + \beta_2I j )} \\
 &=& \frac{\operatorname{exp}(\beta_1U_{i} + \beta_2Ii )  }{\operatorname{exp}(\beta_1U_{j} + \beta_2I j )}} \\
\end{eqnarray*}


which does not depend on $x$, hence proportional hazards.
(iii) \[\log_e \mu_40 \;=\; -9+ 0.09(40) + 0.3- 0.0001(20,000) \;=\; -7.1\]
so $\mu_{40} = 0.000825$.
(iv) ${ \displaystyle \mu_x \;=\; \operatorname{exp}(\alpha\;+\;\beta_0x + \beta_1U + \beta_2I)}$

Let the income of the urban resident be IU and that of the rural resident be IR.

\[\operatorname{exp}( \alpha) \operatorname{exp}(\beta_0x ) \operatorname{exp}( \beta_1U + \beta_2I) =  \operatorname{exp}(\alpha ) \operatorname{exp}(\beta_0x ) \operatorname{exp}(\beta_2_R )\]


\[\operatorname{exp}( \beta_1U + \beta_2I) =   \operatorname{exp}(\beta_2_R )\]


\[\operatorname{exp}(0.3 -  0.0001 I_U ) = \operatorname{exp}( -0.0001 I_R )\]
\[0.3 - 0.0001I_U = -0.0001 I_R\]

\[3000 = I_U -I_R\]
So the difference is \$3,000.

%%%%%%%%%%%%%%%%%%%%%%%%%%%%%%%%%%%%%%%%%%%%%%%%%%%

Survival probability is
\begin{eqnarray*}
exp \left( - \left[ \frac{e^{0.09s}}{0.09} \right]^{50}_{40} e^{-9} e^{0.3} e{-0.0001(20,000)}\\ 
&=& exp \left ( \frac{0.00002254(90.017 \;-\; 36.598)}{0.09} \right)\\ 
&=& exp(-0.01338) \\ 
&=& 0.9867.\\
\end{eqnarray*}

(vi) Since 

\[ S_x(t) =  exp \left[ -\int^{x+t}_{x} \mu_s ds \right]  ,\]
then if the rural resident is $a$ years older than the urban resident we have


\[  exp \left[ -\int^{x+t}_{x} e^{0.09s}e^{\alpha}e^{\beta_1}e^{\beta_2I} ds \right]  ,\]
 = 
exp \left[ -\int^{x+a+t}_{x+a} e^{0.09s}e^{\alpha}e^{\beta_1}e^{\beta_2I} ds \right]  ,\]
\]
Therefore

\[ e^{\alpha}e^{\beta_2I} \int^{x+a+t}_{x+a} e^{0.09s}ds = 
e^{\alpha}e^{\beta_2I} \int^{x+t}_{x} e^{\beta_1} e^{0.09s}ds 
\]
\newpage
(ii)
log e  x     0 x   1 U  2 I
So  x  exp(    0 x  1 U   2 I )  exp(  0 x )exp(   1 U   2 I )
This is equal to Bc x where B  exp(    1 U  2 I ) and c = exp \beta 0 , hence Gompertz.
 x  exp(    0 x  1 U   2 I )  exp(   0 x )exp(  1 U   2 I )
Page 11%%%%%%%%%%%%%%%%%%%%%%%%%%%%%%%%%%%%%%%%%%%%%%%
EITHER
Hence the force of mortality factorises into a term exp(\alpha+  0 x) depending
on age x but not the covariates, and a term exp(  1 U   2 I ) depending on
the covariates but not x, SO proportional hazards.
OR
Consider any two individuals, i and j, with values of the covariates U i and I i ,
and U j and I j respectively. Then the hazards for individuals i and j at age x are
 x , i  exp(    0 x )exp(  1 U i   2 I i )
and
 x , j  exp(    0 x ) exp(  1 U j   2 I j )
The ratio between the hazards is thus
 x , i
 x , j

exp(    0 x ) exp(  1 U i   2 I i )
exp(  1 U i   2 I i )

,
exp(    0 x ) exp(  1 U j   2 I j ) exp(  1 U j   2 I j )
which does not depend on x, hence proportional hazards.
(iii)
log e  40   9  0.09(40)  0.3  0.0001(20,000)   7.1
so \mu 40 = 0.000825.
(iv)
 x  exp(   0 x  1 U  2 I )
Let the income of the urban resident be I U and that of the rural resident be I R .
exp(  ) exp(  0 x ) exp(  1   2 I U )  exp(  ) exp(  0 x ) exp(  2 I R )
exp(  1   2 I U )  exp(  2 I R )
exp(0.3  0.0001 I U )  exp(  0.0001 I R )
0.3  0.0001 I U   0.0001 I R
3, 000  I U  I R
So the difference is $3,000.
Page 12%%%%%%%%%%%%%%%%%%%%%%%%%%%%%%%%%%%%%%%%%%%%%%%
Survival probability is
  0.09 s  50

 e
 0.00002254(90.017  36.598) 
 9 0.3  0.0001(20,000) 
exp   
e
e
e

  exp  
 
0.09

    0.09   40
 
 exp(  0.01338)  0.9867.
(vi)
 x  t

Since S x ( t )  exp     s ds  ,
  x
 
then if the rural resident is a years older than the urban resident we have
 x  t

 x  a  t

0.09 s   1  2 I
exp    e
e e e ds   exp    e 0.09 s e  e  2 I ds 
  x
 
  x  a
 
Therefore
  2 I
e e
x  a  t

e
0.09 s
  2 I
x  t
ds  e e
x  a
x  a  t

e
0.09 s
 1 0.09 s
 e
e
ds
x
x  t
ds 
x  a
 1 0.09 s
 e
e
ds
x
x  a  t
 e 0.09 s 


  0.09   x  a
x  t
 e  1 e 0.09 s 


  0.09   x
e 0.09( x  a  t )  e 0.09( x  a )  e  1 ( e 0.09( x  t )  e 0.09 x )
e 0.09 x e 0.09 a ( e 0.09 t  1)  e  1 e 0.09 x ( e 0.09 t  1)
e 0.09a  e  1
a   1 / 0.09  0.3 / 0.09  3.33
So the rural dweller is aged 40 + 3.33 = 43.33 years.
In part (i) very few candidates made the point that the Gompertz model is simple and
convenient to use. Part (ii) was very poorly answered. When demonstrating that the model
was a proportional hazards (PH) model, many candidates simply factorised the expression as
\mu x,i = exp(\alpha)exp(\beta 0 x + \beta 1 U i + \beta 2 I i ) and said that therefore exp(\alpha) was the baseline hazard.
This is incorrect because the second term includes both duration and the covariates. It was
Page 13%%%%%%%%%%%%%%%%%%%%%%%%%%%%%%%%%%%%%%%%%%%%%%%
acceptable in part (ii) only to break up the equation once as long as the argument was
developed further for both the Gompertz and the PH cases. A common error in part (v) was
to assume that the hazard was constant at its value at age 40 years. This produced a survival
probability of 0.99178. In part (vi) the derivation shown in above was not required for full
credit. Candidates who spotted that, if 40+a is the age of the rural dweller in years,
then e 0.09a = e \beta 1 , scored full credit.
Since the question was missing a comma after "were \alpha =  9.0"a small number of candidates
interpreted the parameters differently i.e. \alpha = 0.09, \beta 0 =  0.01, \beta 1 =0.3 and \beta 3 =  0.0001.
This interpretation was given full credit, if followed through correctly.

  B  
In part (vi) the approach using t p x   exp 
 
 log c  

c x ( c t  1)
was acceptable.
