\documentclass[a4paper,12pt]{article}

%%%%%%%%%%%%%%%%%%%%%%%%%%%%%%%%%%%%%%%%%%%%%%%%%%%%%%%%%%%%%%%%%%%%%%%%%%%%%%%%%%%%%%%%%%%%%%%%%%%%%%%%%%%%%%%%%%%%%%%%%%%%%%%%%%%%%%%%%%%%%%%%%%%%%%%%%%%%%%%%%%%%%%%%%%%%%%%%%%%%%%%%%%%%%%%%%%%%%%%%%%%%%%%%%%%%%%%%%%%%%%%%%%%%%%%%%%%%%%%%%%%%%%%%%%%%

\usepackage{eurosym}
\usepackage{vmargin}
\usepackage{amsmath}
\usepackage{graphics}
\usepackage{epsfig}
\usepackage{enumerate}
\usepackage{multicol}
\usepackage{subfigure}
\usepackage{fancyhdr}
\usepackage{listings}
\usepackage{framed}
\usepackage{graphicx}
\usepackage{amsmath}
\usepackage{chngpage}

%\usepackage{bigints}
\usepackage{vmargin}

% left top textwidth textheight headheight

% headsep footheight footskip

\setmargins{2.0cm}{2.5cm}{16 cm}{22cm}{0.5cm}{0cm}{1cm}{1cm}

\renewcommand{\baselinestretch}{1.3}

\setcounter{MaxMatrixCols}{10}

\begin{document}
\begin{enumerate}

A bus route in a large town has one bus scheduled every 15 minutes. Traffic
conditions in the town are such that the arrival times of buses at a particular bus stop
may be assumed to follow a Poisson process.
Mr Bean arrives at the bus stop at 12 midday to find no bus at the stop. He intends to
get on the first bus to arrive.
(ii)
Determine the probability that the first bus will not have arrived by 1.00 pm
the same day.

The first bus arrived at 1.10 pm but was full, so Mr Bean was unable to board it.
(iii) Explain how much longer Mr Bean can expect to wait for the second bus to
arrive.

(iv) Calculate the probability that at least two more buses will arrive between
1.10 pm and 1.20 pm.

%%--------------7]
CT4 S2013–24
(i)
State what needs to be assessed when considering the suitability of a model for
a particular purpose.

A city has care homes for the elderly. When an elderly person in the city is no longer
able to cope alone at home they can move into a care home where they will be looked
after until they die.
The city council runs some of the care homes and is reviewing the number of rooms it needs in all the council-run care homes. The following model has been proposed.
The age distribution of the city population over the age of 60 is taken from the latest census. Statistics from the national health service on the average age of entry into a
care home and the proportion of elderly who go into council-run care homes are applied to the current population to give the rate at which people enter care homes. A
recent national mortality table is then applied to give the rate at which care home residents die.
(ii)
5
Discuss the suitability of this model.

%%--------------9]
A motor insurer offers a No Claims Discount scheme which operates as follows. The discount levels are $\{0\%,25\%, 50\%, 60\%\}$. Following a claim-free year a
policyholder moves up one discount level (or stays at the maximum discount). After a year with one or more claims the policyholder moves down two discount levels (or
moves to, or stays in, the 0\% discount level).

The probability of making at least one claim in any year is 0.2.
(i) Write down the transition matrix of the Markov chain with state space $\{0\%,25\%, 50\%, 60\%\}$.

(ii) State, giving reasons, whether the process is:
(a)
(b)
irreducible.
aperiodic.

(iii)
Calculate the proportion of drivers in each discount level in the stationary distribution.

The insurer introduces a “protected” No Claims Discount scheme, such that if the 60\% discount is reached the driver remains at that level regardless of how many
claims they subsequently make.

(iv)
CT4 S2013–3
Explain, without doing any further calculations, how the answers to parts (ii)
and (iii) would change as a result of introducing the “protected” No Claims
Discount scheme.

%%--------------11]
%%%%%%%%%%%%%%%%%%%%
5
(i)
0 
 0.2 0.8 0


0.2 0 0.8 0 
P = 
 0.2 0
0 0.8 


 0 0.2 0 0.8 
where the levels are ordered 0%, 25%, 50%, 60%.
(ii)
(iii)
Page 8
(a) The chain is irreducible as it is clear that any state can eventually be reached
from any other state.
(b) The process is aperiodic because, for example, the process can loop round in
the 0% or 60% states giving no set return period to any state.
Stationary distribution  satisfies    P
0.2  0  0.2  25  0.2  50   0
0.8  0  0.2  60   25
0.8 25   50
0.8  50  0.8  60   60 (1)
Also  0   25   50   60  1 (5)
(2)
(3)
(4)%%%%%%%%%%%%%%%%%%%%%%%%%%%%%%%%%%%%%%%%%%%%%%%
Working in terms of  60
 50  0.25  60
 25 
 0 
5
 60
16
9
 60
64
Hence
(64  16  20  9)
 60  1
64
 9 
 
1  20 
So the e stationary distribution is
109  16 
 
 64 
and the proportion of drivers at each level is
0%
25%
50%
60%
(iv)
9/109 = 0.08257
20/109 = 0.18349
16/109 = 0.14679
64/109 = 0.58716.
The 60\% discount level becomes an absorbing state and so it is no longer irreducible.
However it is still aperiodic because you cannot get out of the absorbing state 60%
and the other states still have no period.
The process would now be stationary when all drivers are in the absorbing 60%
discount level.
OR
The new stationary distribution is [0,0,0,1] because the 60% state is now absorbing.
This question was well answered, with many candidates scoring close to full marks. In part
(iii) the correct numerical probabilities scored full marks, provided that it was clear to which
level each probability applied. In part (iv) some candidates made vague statements that the
probability of being in the 60% state would increase. While this is true, it was not given full
credit, as the key point is that the stationary distribution has everyone in the 60% state.
Page 9%%%%%%%%%%%%%%%%%%%%%%%%%%%%%%%%%%%%%%%%%%%%%%%


%%%%%%%%%%%%%%%%%%%%%%%%%%%%
4
(i)
The objectives of the modelling exercise.
The validity of the model for the purpose to which it is to be put.
The validity of the data to be used.
The validity of assumptions used.
The possible errors associated with the model or parameters used not being a perfect
representation of the real world situation being modelled.
The impact of correlations between the random variables that “drive” the model.
The extent of correlations between the various results produced from the model.
The current relevance of models written and used in the past.
The credibility of the data input.
The credibility of the results output.
The dangers of spurious accuracy.
The cost of buying or constructing, and of running the model.
Page 6%%%%%%%%%%%%%%%%%%%%%%%%%%%%%%%%%%%%%%%%%%%%%%%
The ease of use and availability of suitable staff to use it.
Risk of model being used incorrectly or with wrong inputs.
The ease with which the model and its results can be communicated.
Compliance with the relevant regulations.
(ii)
The objectives are to determine the number of rooms the council needs.
But we have no information about “down time” between occupants, or
requirement for seasonal variation or any built-in surplus, and so on.
So with these data alone the requirements cannot be fulfilled.
Local mortality may be different from national experience.
Care home residents may experience significantly different mortality to the
national population (the council may have data on deaths to care home residents
which could be used to adjust the national mortality table).
The data are readily available and should be reliable.
However they may not be suitable for projecting more than a couple of years into the
future because for example:






the distribution of the local population may be skewed, e.g. there may be a huge
number of 55 to 59 year olds compared to 60 to 65 year olds;
age of entry into care homes is likely to change as medical advances help people
stay healthier longer;
the proportion of people going into council homes versus private homes may
change as economic conditions change;
the national mortality table may have no mortality improvements built in.
social habits may change e.g. families may start living more as a unit so adult
children/grandchildren may be available to care for the elderly at home for longer,
especially if the ethnic mix of the city changes;
the size/age mix of the city may change.
The average age at entry into a care home needs to be converted to a distribution by
age. This may be subjective.
There might be different types of room for different levels of care needed. In this
case facilities may be inadequate to meet needs even if there are sufficient rooms in
total.
The data used for the model are taken from different sources so should not be unduly
correlated.
Page 7%%%%%%%%%%%%%%%%%%%%%%%%%%%%%%%%%%%%%%%%%%%%%%%
We are not told of any models written in the past. If these existed, it would be useful
to compare its past projections with what has happened in reality.
The resultant “number of rooms” occupied at any one time will need to be adjusted
for example more elderly may decide to enter homes at the start of winter when it
becomes harder/more expensive to stay warm at home, or when epidemics of
influenza happen.
The model is relatively simple to explain.
The local mix of public and private care homes available locally may affect the
proportion of elderly who go into council homes.
Most candidates made a good attempt at part (i). Answers to part (ii) were more variable,
and tended to focus more on the problems with the mortality data rather than issues
connected with the supply and provision of care homes. In both parts of this question, not all
the points listed were required for full marks.
\end{document}
