© Institute and Faculty of Actuaries1 Describe the differences between deterministic and stochastic models.
2 In the context of a survival model:
3
(i) Define right censoring, Type I censoring and Type II censoring.
(ii) Give an example of a practical situation in which censoring would be
informative.



%% -- [Total 5]
For both of the following sets of four stochastic processes, place each process in a
separate cell of the following table, so that each cell correctly describes the state space
and the time space of the process placed in it. Within each set, all four processes
should be placed in the table.
Time Space
Discrete
Continuous
Discrete
Continuous
(a) General Random Walk, Compound Poisson Process, Counting Process,
Poisson Process
(b) Simple Random Walk, Compound Poisson Process, Counting Process,
White Noise

4
The mortality of a certain species of furry animal has been studied. It is known that at
ages over five years the force of mortality, \mu, is constant, but the variation in mortality
with age below five years of age is not understood. Let the proportion of furry
animals that survive to exact age five years be 5 p 0 .
(i) Show that, for furry animals that die at ages over five years, the average age at
5 \mu + 1
.
death in years is

\mu
(ii) Obtain an expression, in terms of \mu and 5 p 0 , for the proportion of all furry
animals that die between exact ages 10 and 15 years.

A new investigation of this species of furry animal revealed that 30 per cent of those
born survived to exact age 10 years and 20 per cent of those born survived to exact
age 15 years.
(iii)
CT4 A2013–2
Calculate \mu and
5
p 0 .

%%%%%%%%%%%%%%%%%%%%%%%%%
1
A stochastic model is one that recognises the random nature of the input components.
A model that does not contain any random component is deterministic in nature.
In a deterministic model, the output is determined once the set of fixed inputs and the
relationships between them have been defined.
By contrast, in a stochastic model the output is random in nature. The output is only a
snapshot or an estimate of the characteristics of the model for a given set of inputs.
A deterministic model is really just a special (simplified) case of a stochastic model.
A deterministic model will give one set of results of the relevant calculations for a single
scenario; a stochastic model will be run many times with the same input and gives
distributions of the relevant results for a distribution of scenarios
The results for a deterministic model can often be obtained by direct calculation.
The results of stochastic models often require Monte Carlo simulation, although some
stochastic models can have an analytical solution.
Correlations can be important in stochastic models as they indicate when the behaviour of
one variable is associated with that of another.
Stochastic models are more complex and more difficult to interpret than deterministic models
and so require more expertise, expense and computer power.
Not all the points listed above were required for full marks. Credit was also given for
sensible points not included in the above list.
2
(i)
Right censoring. The duration to the event is not known exactly,
but is known to exceed some value.
OR
the censoring mechanism cuts short observations in progress.
Type I censoring. The durations at which observations will be censored are specified
in advance.
Type II censoring. Observation continues until a pre-determined number/proportion
of individuals have experienced the event of interest.
(ii)
An investigation of mortality based on life office data in which
individuals are censored who discontinue paying their premiums.
Page 3Subject CT4 %%%%%%%%%%%%%%%%%%%%%%%%%%%%%%%%%
Those whose premiums lapse tend, on average, to be in better health
than do those who carry on paying their premiums.
In part (ii) any suitable example was given credit. However, for full credit it was necessary
to describe a comparison between the risk of the event happening in the censored and
uncensored observations (e.g. “in better health than” or “less likely to die than”). Most
candidates made a good attempt at this question.
3
(a)
Time Space
Discrete
Continuous
Discrete
Continuous
Counting
process Poisson
process
General
random walk Compound
Poisson process
(b)
Time Space
Discrete
Continuous
Discrete Simple
random walk Counting
process
Continuous White noise Compound
Poisson process
This question was answered well, with many candidates scoring full marks. Some candidates
lost marks by failing to follow the instructions in the question precisely. To obtain full credit,
candidates were required to place the processes in grids like those shown above with ONE
process in each of the four cells. What is shown above is the only solution which fulfils this
criterion for groups (a) and (b). In some cases, processes could correctly be placed in cells
other than those shown in the grids above, and credit was given for each process thus
classified correctly.
Page 4Subject CT4 %%%%%%%%%%%%%%%%%%%%%%%%%%%%%%%%%
4
(i)
If the force of mortality, \mu, is constant, then the expected waiting time
1
is .
\mu
Hence expected age at death is 5 +
1 5 \mu + 1
=
.
\mu
\mu

(ii)
EITHER
We need
Since
10 p 0
x p 0
=
− 15 p 0 .
x − 5 p 5 . 5 p 0
and for x > 5,
x
p 5 = e −\mu x ,
then
10
p 0 − 15 p 0 = 5 p 5 . 5 p 0 − 10 p 5 . 5 p 0 = 5 p 0 e − 5 \mu − 5 p 0 e − 10 \mu = 5 p 0 ( e − 5 \mu − e − 10 \mu ) .
OR
We need
=
10
10
p 0 . 5 q 10
p 0 (1 − 5 p 10 )
Since for x > 5,
10
x
p 5 = e −\mu x ,
p 0 (1 − 5 p 10 ) = 5 p 0 ( 5 p 5 − 10 p 5 ) = 5 p 0 ( e − 5 \mu − e − 10 \mu )

(iii)
EITHER
5
p 0 e − 5 \mu = 0.3 and 5 p 0 e − 10 \mu = 0.2 .
So
− 5 \mu
5 p 0 e
− 10 \mu
5 p 0 e
=
0.3
0.2
and e − 5 \mu = 1.5 e − 10 \mu
so that − 5 \mu = log e 1.5 − 10 \mu
Page 5Subject CT4 %%%%%%%%%%%%%%%%%%%%%%%%%%%%%%%%%
5 \mu = 0.4055
\mu = 0.0811.
Therefore 5 p 0 e − 5(0.0811) = 0.3
and
5 p 0
=
0.3
e
− 5(0.0811)
= 0.4500.
OR
10
p 0 = 5 p 0 . 5 p 5 = 0.3
With a constant force after age 5 years, 5 p 5 = 5 p 10 ,
so 15 p 0 = 5 p 0 . 10 p 5 = 5 p 0 . 5 p 5 . 5 p 10 = 5 p 0 ( 5 p 5 ) 2 = 0.2 .
Hence 5 p 5 =
and 5 p 0 =
0.2
0.3
0.3 (0.3) 2
=
= 0.45.
0.2
5 p 5
Then \mu = −
log e 5 p 5 0.4055
=
= 0.0811.
5
5
Answers to this question were extremely disappointing. Few candidates could even attempt
part (i) correctly, and there were similarly few correct attempts at parts (ii) and (iii). In part
(iii) the question asked “calculate” so candidates giving both correct numerical answers
scored full credit. If one of either \mu or 5 p 0 was correct, a minimum of +2 was scored. Where
candidates made the same theoretical error in parts (ii) and (iii), the error was only
penalised once.
