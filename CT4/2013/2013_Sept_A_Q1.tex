\documentclass[a4paper,12pt]{article}

%%%%%%%%%%%%%%%%%%%%%%%%%%%%%%%%%%%%%%%%%%%%%%%%%%%%%%%%%%%%%%%%%%%%%%%%%%%%%%%%%%%%%%%%%%%%%%%%%%%%%%%%%%%%%%%%%%%%%%%%%%%%%%%%%%%%%%%%%%%%%%%%%%%%%%%%%%%%%%%%%%%%%%%%%%%%%%%%%%%%%%%%%%%%%%%%%%%%%%%%%%%%%%%%%%%%%%%%%%%%%%%%%%%%%%%%%%%%%%%%%%%%%%%%%%%%

\usepackage{eurosym}
\usepackage{vmargin}
\usepackage{amsmath}
\usepackage{graphics}
\usepackage{epsfig}
\usepackage{enumerate}
\usepackage{multicol}
\usepackage{subfigure}
\usepackage{fancyhdr}
\usepackage{listings}
\usepackage{framed}
\usepackage{graphicx}
\usepackage{amsmath}
\usepackage{chngpage}

%\usepackage{bigints}
\usepackage{vmargin}

% left top textwidth textheight headheight

% headsep footheight footskip

\setmargins{2.0cm}{2.5cm}{16 cm}{22cm}{0.5cm}{0cm}{1cm}{1cm}

\renewcommand{\baselinestretch}{1.3}

\setcounter{MaxMatrixCols}{10}

\begin{document}
\begin{enumerate}


%%© Institute and Faculty of Actuaries1

1
2
Data are often subdivided when investigating mortality statistics.
(i) Explain why this is done.
[2]
(ii) Discuss one potential problem with sub-dividing mortality data.
[2]
(iii) List four factors which are commonly used to sub-divide mortality data.
[2]
[Tota

%%%%%%%%%%%%%%%%%%%%%%%%%%%%%%%%%%%%%
1
(i)
All our models and analyses are based on the assumption that we can observe groups
of identical lives (or at least, lives whose mortality characteristics are the same).
In practice, this is never possible.
However, we can at least subdivide our data according to characteristics known, from
experience, to have a significant effect on mortality.
This ought to reduce the heterogeneity of each class so formed.
(ii)
The number of lives in each subdivision may become small. This will lead to
estimates of mortality that are unreliable, with large standard errors.
OR
Information about the factors which affect mortality may be unavailable because it
was not asked on the insurance proposal form, or population census
OR
Information about the factors which affect mortality may be unreliable because
respondents gave inaccurate or false answers to questions.
(iii)
Sex
Age
Type of policy (which often reflects the reason for insuring)
Smoker/non-smoker status
Level of underwriting
Duration in force
Sales channel
Policy size
Occupation of policyholder
Known impairments
Postcode/geographical location
Marital status
Answers to part (i) of this question were disappointing, with few candidates relating the need
for homogeneity to the models we use. Parts (ii) and (iii) were generally well answered. In
part (ii) the instruction was to describe a single limitation, so no credit was given for second
or subsequent limitations. In part (iii) credit was given in some cases for wording different
from that indicated, such as “state of health”, or for certain other factors which are known to
affect mortality, and about which information is asked, for example, in population censuses.
However, genetic factors were not given credit.
