\documentclass[a4paper,12pt]{article}

%%%%%%%%%%%%%%%%%%%%%%%%%%%%%%%%%%%%%%%%%%%%%%%%%%%%%%%%%%%%%%%%%%%%%%%%%%%%%%%%%%%%%%%%%%%%%%%%%%%%%%%%%%%%%%%%%%%%%%%%%%%%%%%%%%%%%%%%%%%%%%%%%%%%%%%%%%%%%%%%%%%%%%%%%%%%%%%%%%%%%%%%%%%%%%%%%%%%%%%%%%%%%%%%%%%%%%%%%%%%%%%%%%%%%%%%%%%%%%%%%%%%%%%%%%%%

\usepackage{eurosym}
\usepackage{vmargin}
\usepackage{amsmath}
\usepackage{graphics}
\usepackage{epsfig}
\usepackage{enumerate}
\usepackage{multicol}
\usepackage{subfigure}
\usepackage{fancyhdr}
\usepackage{listings}
\usepackage{framed}
\usepackage{graphicx}
\usepackage{amsmath}
\usepackage{chngpage}

%\usepackage{bigints}
\usepackage{vmargin}

% left top textwidth textheight headheight

% headsep footheight footskip

\setmargins{2.0cm}{2.5cm}{16 cm}{22cm}{0.5cm}{0cm}{1cm}{1cm}

\renewcommand{\baselinestretch}{1.3}

\setcounter{MaxMatrixCols}{10}

\begin{document}
\begin{enumerate}
\item 

The Shining Light company has developed a new type of light bulb which it recently
tested. 1,000 bulbs were switched on and observed until they failed, or until 500
hours had elapsed. For each bulb that failed, the duration in hours until failure was
noted. Due to an earth tremor after 200 hours, 200 bulbs shattered and had to be
removed from the test before failure.
The results showed that 10 bulbs failed after 50 hours, 20 bulbs failed after 100 hours,
50 bulbs failed after 250 hours, 300 bulbs failed after 400 hours and 50 bulbs failed
after 450 hours.
(i) Calculate the Kaplan-Meier estimate of the survival function, S(t), for the light
bulbs in the test.

(ii) Sketch the Kaplan-Meier estimate calculated in part (i).
(iii) Estimate the probability that a bulb will not have failed after each of the
following durations: 300 hours, 400 hours and 600 hours. If it is not possible
to obtain an estimate for any of the durations without additional assumptions,
explain why.

%% -- [Total 11]
CT4 A2013–4
8
During a football match, the referee can caution players if they commit an offence by
showing them a yellow card. If a player commits a second offence which the referee
deems worthy of a caution, they are shown a red card, and are sent off the pitch and
take no further part in the match. If the referee considers a particularly serious
offence has been committed, he can show a red card to a player who has not
previously been cautioned, and send the player off immediately.
The football team manager can also decide to substitute one player for another at any
point in the match so that the substituted player takes no further part in the match.
Due to the risk of a player being sent off, the manager is more likely to substitute a
player who has been shown a yellow card. Experience shows that players who have
been shown a yellow card play more carefully to try to avoid a second offence.
The rate at which uncautioned players are shown a yellow card is 1/10 per hour.
The rate at which those players who have already been shown a yellow card are
shown a red card is 1/15 per hour.
The rate at which uncautioned players are shown a red card is 1/40 per hour.
The rate at which players are substituted is 1/10 per hour if they have not been shown
a yellow card, and 1/5 if they have been shown a yellow card.
(i) Sketch a transition graph showing the possible transitions between states for a
given player.

(ii) Write down the compact form of the Kolmogorov forward equations,
specifying the generator matrix.

A football match lasts 1.5 hours.
(iii) Solve the Kolmogorov equation for the probability that a player who starts the
match remains in the game for the whole match without being shown a yellow
card or a red card.

(iv) Calculate the probability that a player who starts the match is sent off during
the match without previously having been cautioned.

Consider a match that continued indefinitely rather than ending after 1.5 hours.
(v)
CT4 A2013–5
(a) Derive the probability that in this instance a player is sent off without
previously having been cautioned.
(b) Explain your result.

%% -- [Total 12]
%%%%%%%%%%%%%%%%%%%%%%%%%%%%
7
(i)
t j N j d j c j d j / N j 1 - d j / N j 0
50
100
200
250
400
450 1,000
1,000
990
970
770
720
420 10
20
0
50
300
50 0
0
200
0
0
370 0.0100
0.0202 0.9900
0.9798 or 99/100
or 97/99
0.0649
0.4167
0.1190 0.9351
0.5833
0.8810 or 72/77
or 7/12
or 37/42
The Kaplan-Meier estimate is S ˆ ( t ) = ∏ (1 −
t j ≤ t
t
d j
n j
)
Kaplan-Meier estimate of S(t)
0 ≤ t < 50
50 ≤ t < 100
100 ≤ t < 250
250 ≤ t < 400
400 ≤ t < 450
450 ≤ t < 500
1.0000
0.9900
0.9700
0.9070
0.5291
0.4661
or 1
or 99/100
or 97/100
or 1,746/1,925
or 291/550
or 3,589/7,700
(ii)
1
0.9
0.8
0.7
0.6
S(t) 0.5
0.4
0.3
0.2
0.1
0
0
100
200
300
400
500
Duration t
Page 9 %%%%%%%%%%%%%%%%%%%%%%%%%%%%%%%%%
(iii)
S(300) = 0.9070.
S(400) = 0.5291.
S(600) cannot be estimated without additional assumptions
as it lies outside the range of our data.

%% -- [Total 11]
This question was very well answered, with many candidates scoring 10 or more marks out of
a possible 11. Some of the sketches in part (ii) were very scrappy and were penalised:
though great accuracy was not required, the sketch did need to be sufficiently clear to
demonstrate that the candidate understood the nature of the function being plotted. In part
(iii) some candidates suggested an assumption which would enable them to give an answer
for S(600). Such candidates were given full credit provided they explained why the
assumption was needed, and provided that the stated assumption was consistent with the
numerical answer offered.
8
(i)
1/40
Uncautioned
1/10
1/15
1/5
1/10
Substituted
Page 10
Booked
/Yellow
Sent Off
/Red %%%%%%%%%%%%%%%%%%%%%%%%%%%%%%%%%
(ii)
d
P ( t ) = P ( t ) A
dt
where generator matrix
⎛ − 9 / 40 1/10 1/ 40 1/10 ⎞
⎜
⎟
0
− 4 / 15 1/15 1/ 5 ⎟
A = ⎜
⎜ 0
0
0
0 ⎟
⎜
⎟
0
0
0 ⎠
⎝ 0
In order of states {U, Y, R, S}
(iii)
d
9
P UU ( t ) = − P UU ( t )
dt
40
⎛ 9 ⎞
⎛ 9 ⎞
P UU ( t ) = exp ⎜ − t ⎟ + const = exp ⎜ − t ⎟ as looking for probability on
⎝ 40 ⎠
⎝ 40 ⎠
pitch throughout match.
At end t = 3/2 so require exp(−27/80) = 71.36%.
(iv)
Prob[sent off without being booked] =
= ∫
3/2
s = 0
exp( −
3/2
1
∫ s = 0 P UU ( s ). 40 ds
9
1
s ). ds
40 40
3/2
1 ⎛
⎡ 1
⎛ 27 ⎞ ⎞
⎛ 9 ⎞ ⎤
= ⎢ − exp ⎜ − s ⎟ ⎥ = ⎜ 1 − exp ⎜ − ⎟ ⎟ =0.03183
9 ⎝
⎝ 80 ⎠ ⎠
⎝ 40 ⎠ ⎦ 0
⎣ 9
(v)
(a)
EITHER
The upper limit of the integral tends to infinity
so result becomes 1/9.
OR
We need
Pr[sent off directly]/Pr[leaves state U]
=
(b)
1/ 40 1
=
9 / 40 9
This is the ratio of the transition rate to “straight to sent off”
Page 11 %%%%%%%%%%%%%%%%%%%%%%%%%%%%%%%%%
to the total transition rate out of state U.
This was one of the more demanding questions on the paper, and a high proportion of
candidates struggled to get past part (ii). In part (i) the transition rates were not required. In
part (iii) the question asked candidates to solve an equation, so for full credit the equation
had to be written down, and the method of solution described. In part (v) candidates who
used the rationale in (b) to do the calculation in (a) scored full credit.
Candidates who interpreted the question in a manner not intended, but instead combined the
categories “sent off” and “substituted” were not penalised. This interpretation leads to a
three state solution for parts (i) and (ii) as follows.
(i)
Uncautioned
1/10
Booked
/Yellow
1/15
1/5 +
1/15
1/10
+
1/40
Not playing
d
P ( t ) = P ( t ) A
dt
(ii)
where generator matrix
⎛ − 9
⎜ 40
⎜
A = ⎜ 0
⎜
⎜
⎜ ⎜ 0
⎝
1
1 ⎞
10 8 ⎟
⎟
− 4 4 ⎟
15 15 ⎟
⎟
0
0 ⎟
⎟
⎠
In order of states {U, Y, N}

This was given full credit in parts (i), (ii) and (iii). The answer to part (iii) is the same as for
the four-state solution. Credit was given in parts (iv) and (v) for following this alternative
through correctly.
Page 12 %%%%%%%%%%%%%%%%%%%%%%%%%%%%%%%%%
