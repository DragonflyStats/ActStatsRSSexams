\documentclass[a4paper,12pt]{article}

%%%%%%%%%%%%%%%%%%%%%%%%%%%%%%%%%%%%%%%%%%%%%%%%%%%%%%%%%%%%%%%%%%%%%%%%%%%%%%%%%%%%%%%%%%%%%%%%%%%%%%%%%%%%%%%%%%%%%%%%%%%%%%%%%%%%%%%%%%%%%%%%%%%%%%%%%%%%%%%%%%%%%%%%%%%%%%%%%%%%%%%%%%%%%%%%%%%%%%%%%%%%%%%%%%%%%%%%%%%%%%%%%%%%%%%%%%%%%%%%%%%%%%%%%%%%

\usepackage{eurosym}
\usepackage{vmargin}
\usepackage{amsmath}
\usepackage{graphics}
\usepackage{epsfig}
\usepackage{enumerate}
\usepackage{multicol}
\usepackage{subfigure}
\usepackage{fancyhdr}
\usepackage{listings}
\usepackage{framed}
\usepackage{graphicx}
\usepackage{amsmath}
\usepackage{chngpage}

%\usepackage{bigints}
\usepackage{vmargin}

% left top textwidth textheight headheight

% headsep footheight footskip

\setmargins{2.0cm}{2.5cm}{16 cm}{22cm}{0.5cm}{0cm}{1cm}{1cm}

\renewcommand{\baselinestretch}{1.3}

\setcounter{MaxMatrixCols}{10}

\begin{document}
4
The mortality of a certain species of furry animal has been studied. It is known that at ages over five years the force of mortality, \mu, is constant, but the variation in mortality with age below five years of age is not understood. Let the proportion of furry
animals that survive to exact age five years be 5 p 0 .
\begin{enumerate}[(a)]
\item (i) Show that, for furry animals that die at ages over five years, the average age at
5 \mu + 1
.
death in years is

\mu
\item (ii) Obtain an expression, in terms of \mu and 5 p 0 , for the proportion of all furry
animals that die between exact ages 10 and 15 years.

\item A new investigation of this species of furry animal revealed that 30 per cent of those
born survived to exact age 10 years and 20 per cent of those born survived to exact
age 15 years.
(iii)
CT4 A2013–2
Calculate $\mu$ and
5
p 0 .

\end{enumerate}
%%%%%%%%%%%%%%%%%%%%%%%%%%%%%%%%%%%%%%%%%%%%%


4
(i)
If the force of mortality, \mu, is constant, then the expected waiting time
1
is .
\mu
Hence expected age at death is 5 +
1 5 \mu + 1
=
.
\mu
\mu
%--------------------------%
(ii)
EITHER
We need
Since
10 p 0
x p 0
=
− 15 p 0 .
x − 5 p 5 . 5 p 0
and for x > 5,
x
p 5 = e −\mu x ,
then
10
p 0 − 15 p 0 = 5 p 5 . 5 p 0 − 10 p 5 . 5 p 0 = 5 p 0 e − 5 \mu − 5 p 0 e − 10 \mu = 5 p 0 ( e − 5 \mu − e − 10 \mu ) .
OR
We need
=
10
10
p 0 . 5 q 10
p 0 (1 − 5 p 10 )
Since for x > 5,
10
x
p 5 = e −\mu x ,
p 0 (1 − 5 p 10 ) = 5 p 0 ( 5 p 5 − 10 p 5 ) = 5 p 0 ( e − 5 \mu − e − 10 \mu )

(iii)
EITHER
5
p 0 e − 5 \mu = 0.3 and 5 p 0 e − 10 \mu = 0.2 .
So
− 5 \mu
5 p 0 e
− 10 \mu
5 p 0 e
=
0.3
0.2
and e − 5 \mu = 1.5 e − 10 \mu
so that − 5 \mu = log e 1.5 − 10 \mu
Page 5Subject CT4 %%%%%%%%%%%%%%%%%%%%%%%%%%%%%%%%%
5 \mu = 0.4055
\mu = 0.0811.
Therefore 5 p 0 e − 5(0.0811) = 0.3
and
5 p 0
=
0.3
e
− 5(0.0811)
= 0.4500.
OR
10
p 0 = 5 p 0 . 5 p 5 = 0.3
With a constant force after age 5 years, 5 p 5 = 5 p 10 ,
so 15 p 0 = 5 p 0 . 10 p 5 = 5 p 0 . 5 p 5 . 5 p 10 = 5 p 0 ( 5 p 5 ) 2 = 0.2 .
Hence 5 p 5 =
and 5 p 0 =
0.2
0.3
0.3 (0.3) 2
=
= 0.45.
0.2
5 p 5
Then \mu = −
log e 5 p 5 0.4055
=
= 0.0811.
5
5
Answers to this question were extremely disappointing. Few candidates could even attempt
part (i) correctly, and there were similarly few correct attempts at parts (ii) and (iii). In part
(iii) the question asked “calculate” so candidates giving both correct numerical answers
scored full credit. If one of either \mu or 5 p 0 was correct, a minimum of +2 was scored. Where
candidates made the same theoretical error in parts (ii) and (iii), the error was only
penalised once.
