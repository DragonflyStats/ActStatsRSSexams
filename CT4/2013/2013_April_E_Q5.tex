\documentclass[a4paper,12pt]{article}

%%%%%%%%%%%%%%%%%%%%%%%%%%%%%%%%%%%%%%%%%%%%%%%%%%%%%%%%%%%%%%%%%%%%%%%%%%%%%%%%%%%%%%%%%%%%%%%%%%%%%%%%%%%%%%%%%%%%%%%%%%%%%%%%%%%%%%%%%%%%%%%%%%%%%%%%%%%%%%%%%%%%%%%%%%%%%%%%%%%%%%%%%%%%%%%%%%%%%%%%%%%%%%%%%%%%%%%%%%%%%%%%%%%%%%%%%%%%%%%%%%%%%%%%%%%%

\usepackage{eurosym}
\usepackage{vmargin}
\usepackage{amsmath}
\usepackage{graphics}
\usepackage{epsfig}
\usepackage{enumerate}
\usepackage{multicol}
\usepackage{subfigure}
\usepackage{fancyhdr}
\usepackage{listings}
\usepackage{framed}
\usepackage{graphicx}
\usepackage{amsmath}
\usepackage{chngpage}

%\usepackage{bigints}
\usepackage{vmargin}

% left top textwidth textheight headheight

% headsep footheight footskip

\setmargins{2.0cm}{2.5cm}{16 cm}{22cm}{0.5cm}{0cm}{1cm}{1cm}

\renewcommand{\baselinestretch}{1.3}

\setcounter{MaxMatrixCols}{10}

\begin{document}

%% -- [Total 7]5
Population censuses in a certain country are taken each year on the President’s birthday, provided that the President’s astrological advisor deems the taking of a census favourable. Censuses record the age of every inhabitant in completed years (that is, curtate age). Deaths in this country are registered as they happen, and classified according to age nearest birthday at the time of death.
Below are some data from the three most recent censuses.

Age in
completed
years Population
2006
(thousands) Population
2009
(thousands) Population
2010
(thousands)
64
65
66 300
290
280 320
310
300 350
330
320
Between the censuses of 2006 and 2009 there were a total of 3,000 deaths to
inhabitants aged 65 nearest birthday, and between the censuses of 2009 and 2010
there were a total of 1,000 deaths to inhabitants aged 65 nearest birthday.
\begin{enumerate}
\item (i)
Estimate, stating any assumptions you make, the death rate at age 65 years for
each of the following periods:
•
•
\item (ii)
CT4 A2013–3
the period between the 2006 and 2009 censuses
the period between the 2009 and 2010 censuses
Explain the exact age to which your estimates apply.
\end{enumerate}



%%%%%%%%%%%%%%%%%%%%
5
(i)
We adjust the exposed to risk so that the age definition corresponds with that of the
deaths data.
Let the population at age 65 nearest birthday be P 65 and let the central exposed to risk
c
at age 65 nearest birthday be E 65
.


\begin{itemize}
\item In 2006 P 65 = 0.5(300,000 + 290,000) = 295,000
\item In 2009 P 65 = 0.5(320,000 + 310,000) = 315,000
\item In 2010 P 65 = 0.5(350,000 + 330,000) = 340,000,
\end{itemize}
assuming that birthdays are uniformly distributed across calendar time.
Using the census approximation (trapezium method) for the period 2006-2009 then assuming that the population varies linearly between census dates,
c
= 1.5(295,000 + 315,000) = 915,000
E 65
and for the period 2009–2010
c
= 0.5(315,000 + 340,000) = 327,500.
E 65
Assuming that the force of mortality is constant within each year of age
\mu 65 = 3,000
= 0.003279 for the period 2006–2009, and
915,000
\mu 65 = 1,000
= 0.003053 for the period 2009–2010.
327,500
We also assuming that the President doesn't change (so the birthday is on the same
day each year), or if the President does change the new President’s birthday is the
same as the birthday of the old President.
(ii)
The rate interval is the life year, starting at age x – 0.5.
The age in the middle of the rate interval is thus x, so the estimate relates
to exact age 65 years.

%%%%%%%%%%%%%%%%%%%%%%%%%%%%%%%%%%%%%%%%%%%55
\newpage
A common error in part (i) was to use equal time periods, whereas the period 2006-2009 is three years and 2009-2010 only one year. For full credit, the assumptions had to appear in
the script close to the relevant bit of calculation. Candidates who listed many assumptions, both necessary and unnecessary, in a block at the end of the answer were penalised. In part
(i), some candidates calculated q x rather than \mu x . Full credit was given for this provided that the initial exposed-to-risk was used as the denominator. In part (ii) the age to which q x
applies is 64.5 years (i.e. the age at the start of the rate interval), and for full credit the answers to parts (i) and (ii) had to be consistent.
\end{document}
