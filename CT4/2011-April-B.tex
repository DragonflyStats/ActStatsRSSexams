\documentclass[a4paper,12pt]{article}

%%%%%%%%%%%%%%%%%%%%%%%%%%%%%%%%%%%%%%%%%%%%%%%%%%%%%%%%%%%%%%%%%%%%%%%%%%%%%%%%%%%%%%%%%%%%%%%%%%%%%%%%%%%%%%%%%%%%%%%%%%%%%%%%%%%%%%%%%%%%%%%%%%%%%%%%%%%%%%%%%%%%%%%%%%%%%%%%%%%%%%%%%%%%%%%%%%%%%%%%%%%%%%%%%%%%%%%%%%%%%%%%%%%%%%%%%%%%%%%%%%%%%%%%%%%%

\usepackage{eurosym}
\usepackage{vmargin}
\usepackage{amsmath}
\usepackage{graphics}
\usepackage{epsfig}
\usepackage{enumerate}
\usepackage{multicol}
\usepackage{subfigure}
\usepackage{fancyhdr}
\usepackage{listings}
\usepackage{framed}
\usepackage{graphicx}
\usepackage{amsmath}
\usepackage{chngpage}

%\usepackage{bigints}
\usepackage{vmargin}

% left top textwidth textheight headheight

% headsep footheight footskip

\setmargins{2.0cm}{2.5cm}{16 cm}{22cm}{0.5cm}{0cm}{1cm}{1cm}

\renewcommand{\baselinestretch}{1.3}

\setcounter{MaxMatrixCols}{10}

\begin{document}
\begin{enumerate}

Question 4
(i)
(ii)
⎛ 0.7
⎜
⎜ 0.3
⎜ 0.1
⎜
⎝ 0
0.2
0.4
0.2
0.1
0.1 0 ⎞
⎟
0.2 0.1 ⎟
0.4 0.3 ⎟
⎟
0.2 0.7 ⎠
If the probability distribution in the first week is Π , and the transition matrix is M,
then the probability distribution at the end of the third week is
⎛ 0.7
⎜
0.3
Π M 2 = ( 0 1 0 0 ) ⎜
⎜ 0.1
⎜
⎝ 0
0.2
0.4
0.2
0.1
⎛ 0.56
⎜
0.35
= ( 0 1 0 0 ) ⎜
⎜ 0.17
⎜
⎝ 0.05
0.1 0 ⎞ ⎛ 0.7
⎟⎜
0.2 0.1 ⎟ ⎜ 0.3
0.4 0.3 ⎟ ⎜ 0.1
⎟⎜
0.2 0.7 ⎠ ⎝ 0
0.24
0.27
0.21
0.15
0.15
0.21
0.27
0.24
0.2
0.4
0.2
0.1
0.1 0 ⎞
⎟
0.2 0.1 ⎟
0.4 0.3 ⎟
⎟
0.2 0.7 ⎠
0.05 ⎞
⎟
0.17 ⎟
0.35 ⎟
⎟
0.56 ⎠
so that there is a probability of
35% that a child will be graded Poor’,
27% that a child will be graded Satisfactory,
21% that a child will be graded Good and
17% that a child will be graded Excellent..
There were two common errors on this question. The first was to assume that if a child could
not move up or down two levels, he or she would not move at all. The phrase in the question
“[s]ubject to a maximum level of Excellent and a minimum level of Poor” was intended to
indicate that children could not move beyond these limits in either direction, but would move
as far as they could. Thus a child at level “Good”, who had a 20% chance of moving up one
level and a 10% chance of moving up two levels, would have a 30% chance of moving to
level Excellent, as the 10% who would have moved up two levels will only be able to move up
one level. The second error was to use Π M 3 in part (ii). Candidates who made the first
error were penalised in part (i) but could gain full credit for part (ii) if they followed through
correctly.
Page 4Subject CT4 (Models Core Technical) — Examiners’ Report, April 2011
Question 5
(i)
We believe that mortality varies smoothly with age (and evidence from large
experiences supports this belief).
Therefore the crude estimate of mortality at any age carries information about
mortality at adjacent ages.
By smoothing the experience, we can make use of data at adjacent ages to improve
the estimates at each age.
This reduces sampling (or random) errors.
The mortality experience may be used in financial calculations.
Irregularities, jumps and anomalies in financial quantities (such as premiums for life
insurance contracts) are hard to justify to customers.
(ii)
(a)
Female members of a medium-sized pension scheme.
With reference to a standard table, because there are many extant tables
dealing with female pensioners.
(b)
Male population of a large industrial country.
By parametric formula, because the experience is large.
OR
because the graduated rates may form a new standard table for the country.
(c)
Population of a particular species of reptile in the zoological collections of the
southern hemisphere.
Graphical, because no suitable standard table is likely to exist and
the experience is small.
This question was well answered. In part (i)(c) BOTH elements of the reason were needed
for credit (i.e. that no suitable table is likely to exist AND the experience is small).
Page 5Subject CT4 (Models Core Technical) — Examiners’ Report, April 2011
