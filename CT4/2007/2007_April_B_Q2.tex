\documentclass[a4paper,12pt]{article}

%%%%%%%%%%%%%%%%%%%%%%%%%%%%%%%%%%%%%%%%%%%%%%%%%%%%%%%%%%%%%%%%%%%%%%%%%%%%%%%%%%%%%%%%%%%%%%%%%%%%%%%%%%%%%%%%%%%%%%%%%%%%%%%%%%%%%%%%%%%%%%%%%%%%%%%%%%%%%%%%%%%%%%%%%%%%%%%%%%%%%%%%%%%%%%%%%%%%%%%%%%%%%%%%%%%%%%%%%%%%%%%%%%%%%%%%%%%%%%%%%%%%%%%%%%%%

\usepackage{eurosym}
\usepackage{vmargin}
\usepackage{amsmath}
\usepackage{graphics}
\usepackage{epsfig}
\usepackage{enumerate}
\usepackage{multicol}
\usepackage{subfigure}
\usepackage{fancyhdr}
\usepackage{listings}
\usepackage{framed}
\usepackage{graphicx}
\usepackage{amsmath}
\usepackage{chngpage}

%\usepackage{bigints}
\usepackage{vmargin}

% left top textwidth textheight headheight

% headsep footheight footskip

\setmargins{2.0cm}{2.5cm}{16 cm}{22cm}{0.5cm}{0cm}{1cm}{1cm}

\renewcommand{\baselinestretch}{1.3}

\setcounter{MaxMatrixCols}{10}

\begin{document}

2
An insurance company is investigating the mortality of its annuity policyholders. It is
proposed that the crude mortality rates be graduated for use in future premium
calculations.

\begin{enumerate}[(a)]
\item (i)
(a) Suggest, with reasons, a suitable method of graduation in this case.
(b) Describe how you would graduate the crude rates.
\item 
(ii)
Comment on any further considerations that the company should take into
account before using the graduated rates for premium calculations.
\end{enumerate}

%%%%%%%%%%%%%%%%%%%%%%%%%%%%%%%%%%%%%%%%%%%%%%%%%%%%%%%%%%%%%%%%%%%%%%%%
\newpage
2
(b) Number of claims reported to an insurer by time t.
(i) (a)
Graduation by reference to a standard table would be appropriate.
There are likely to be existing standard tables which are suitable and this method is suitable for relatively small data sets.
Alternatively, graduation by parametric formula would be suitable if the volume of data was large enough. But that is unlikely to be the
case here.

Graphical graduation would not be appropriate for rates for premium calculations.
(b)
(ii)
Assuming graduation by reference to a standard table:
\begin{itemize}
\item Select a suitable table, based on a similar group of lives.
\item Plot the crude rates against q x s from the standard table to identify a simple relationship.
\item Find the best-fit parameters, using maximum likelihood or least squares estimates.
\item Test the graduation for goodness of fit. If the fit is not adequate, the process should be repeated.
\end{itemize}
Considerations include:
\item
As the premiums are for annuity policies, it is important not to
overestimate the mortality rates, as the premiums would be too low.
3
%%----Subject CT4 — %%%%%%%%%%%%%%%%%%%%%%%%%%%%%%%%%%%%5 — April 
\end{document}
