\documentclass[a4paper,12pt]{article}

%%%%%%%%%%%%%%%%%%%%%%%%%%%%%%%%%%%%%%%%%%%%%%%%%%%%%%%%%%%%%%%%%%%%%%%%%%%%%%%%%%%%%%%%%%%%%%%%%%%%%%%%%%%%%%%%%%%%%%%%%%%%%%%%%%%%%%%%%%%%%%%%%%%%%%%%%%%%%%%%%%%%%%%%%%%%%%%%%%%%%%%%%%%%%%%%%%%%%%%%%%%%%%%%%%%%%%%%%%%%%%%%%%%%%%%%%%%%%%%%%%%%%%%%%%%%

\usepackage{eurosym}
\usepackage{vmargin}
\usepackage{amsmath}
\usepackage{graphics}
\usepackage{epsfig}
\usepackage{enumerate}
\usepackage{multicol}
\usepackage{subfigure}
\usepackage{fancyhdr}
\usepackage{listings}
\usepackage{framed}
\usepackage{graphicx}
\usepackage{amsmath}
\usepackage{chngpage}

%\usepackage{bigints}
\usepackage{vmargin}

% left top textwidth textheight headheight

% headsep footheight footskip

\setmargins{2.0cm}{2.5cm}{16 cm}{22cm}{0.5cm}{0cm}{1cm}{1cm}

\renewcommand{\baselinestretch}{1.3}

\setcounter{MaxMatrixCols}{10}

\begin{document}
\begin{enumerate}
PLEASE TURN OVER8
(i)
Describe the difference between the central exposed to risk and the initial
exposed to risk.

The following data come from an investigation of the mortality of participants in a
dangerous sport during the calendar year 2005.
Age x
22
23
(ii)
(a)
(b)
Number of lives aged x last
birthday on:
1 January 2005 1 January 2006 Number of deaths
during 2005 to
persons aged x last
birthday at death
150
160 160
155 20
25
Estimate the initial exposed to risk at ages 22 and 23.
Hence estimate q 22 and q 23 .

Suppose that in this investigation, instead of aggregate data we had individual-level
data on each person’s date of birth, date of death, and date of exit from observation (if
exit was for reasons other than death).
(iii)
9
Explain how you would calculate the initial exposed-to-risk for lives aged 22
years last birthday.

[Total 10]
In a game of tennis, when the score is at “Deuce” the player winning the next point
holds “Advantage”. If a player holding “Advantage” wins the following point that
player wins the game, but if that point is won by the other player the score returns to
“Deuce”.
When Andrew plays tennis against Ben, the probability of Andrew winning any point
is 0.6. Consider a particular game when the score is at “Deuce”.
(i)
Show that the subsequent score in the game can be modelled as a Markov
Chain, specifying both:
(a)
(b)
(ii)
the state space; and
the transition matrix

State, with reasons, whether the chain is:
(a)
(b)
irreducible; and
aperiodic

(iii) Calculate the number of points which must be played before there is more than
a 90% chance of the game having been completed.

(iv) (a)
(b)
CT4 S2007—6
Calculate the probability that Andrew wins the game.
Comment on your answer.

[Total 12]

%%%%%%%%%%%%%%%%%%%%%%%%%%%%%%%%%%%%%%%%%%%%%%%%%%%%%%%%%%%%%%%%%%%%%%%%%%%%%%%
8
(i)
The central exposed to risk at age x , E x c , is the observed waiting time in a multiple-state or a Poisson model. It is the sum of the times spent under
observation by each life at age x .
In aggregate data, the central exposed to risk is an estimate of the number of lives exposed to risk at the mid-point of the rate interval.
The initial exposed to risk requires adjustments for those lives who die, whom we continue observing until the end of the rate interval.
It may be approximated as E x c + 0.5 d x , where d x is the number of deaths to persons aged x .
(ii)
The age definition used for both deaths and exposed to risk is the same, so no adjustment is necessary.
Using the census formula, and assuming that the population aged 22 and 23 years changes linearly over the year, we have, for the central exposed to risk:
1
E x c = \int P x , t dt ,
0
so that
E x c =
1
( P x ,0 + P x ,1 ) .
2
Page 10%%%%%%%%%%%%%%%%%%%%%%%%%%%%%%%%5 — September 2007 — Examiners’ Report
The initial exposed to risk, E x , is then obtained using the approximation
E x c + 0.5 d x .
This assumes that deaths are uniformly distributed across each year of age.
Therefore, at age 22 we have
E 22 =
1
20
= 165 ,
(150 + 160) +
2
2
and
E 23 =
1
25
= 170 .
(160 + 155) +
2
2
Hence q 22 =
20
25
= 0.1212 and q 23 =
= 0.1471 .
165
170
[The complete derivation was not required for full marks.]
(iii)
ALTERNATIVE 1
The central exposed to risk is calculated as
∑ ( b i − a i ) , for all lives i for
i
whom b i − a i > 0 ,
where a i and b i are measured in years since the person’s 22nd birthday, and
where b i is the earliest of
the date of person i ’s death
the date of person i ’s 23rd birthday
the end of the calendar year 2005
the date of person i ’s exit from observation for reasons
other than death
and a i is the latest of
the date of person i ’s 22nd birthday
the start of the calendar year 2005
the date of person i ’s entry into observation.
The initial exposed to risk is then calculated by adding on to the central exposed to risk a quantity equal to 1 − b i for all lives who died aged 22 last
birthday during the calendar year 2005.

Page 11%%%%%%%%%%%%%%%%%%%%%%%%%%%%%%%%5 — September 2007 — Examiners’ Report
ALTERNATIVE 2
The initial exposed to risk is calculated as
∑ ( b i − a i ) ,
i
where a i and b i are measured in years since the
and
person’s 22nd birthday,
where b i is the earliest of
the date of person i ’s 23rd birthday
the date of person i ’s exit from observation for reasons other than death
and a i is the latest of
the date of person i ’s 22nd birthday
the start of the calendar year 2005
the date of person i ’s entry into observation.
for all lives i for whom b i − a i > 0 .
9
(i)
State space:
{Deuce, Advantage A(ndrew), Advantage B(en),
Game A(ndrew), Game B(en)}.
Transition matrix:
Deuce
Adv A
Adv B
Game A
Game B
Deuce Adv A Adv B
0
0.4
0.6
0
0 0.6
0
0
0
0 0.4
0
0
0
0
Game
A
0
0.6
0
1
0
Game
B
0
0
0.4
0
1
The chain is Markov because the probability of moving to the next state does not depend on history prior to entering that state (because the probability of
each player winning a point is constant)
(ii)
The chain is reducible because it has two absorbing states Game A and
Game B.
States Game A and Game B are absorbing so have no period. The other three
states each have a period of 2 so the chain is not aperiodic.
Page 12%%%%%%%%%%%%%%%%%%%%%%%%%%%%%%%%5 — September 2007 — Examiners’ Report
(iii)
The game either ends after 2 points or it returns to Deuce.
The probability of it returning to Deuce after two points is:
Prob A wins 1 st point \times  Prob B wins 2 nd point
+ Prob B wins 1 st point \times  Prob A wins 2 nd point
= 0.6 \times  0.4 + 0.4 \times  0.6 = 0.48.
[This can also be obtained by calculating the square of the transition matrix.]
Need to find number of such cycles N such that:
0.48 N < 1 − 0.9 ,
so that
N >
ln 0.1
> 3.14 .
ln(0.48)
But the game can only finish every two points so we require 4 cycles, that is 8
points.
(iv)
(a)
Define A X to be the probability that A ultimately wins the game when
the current state is X .
We require A Deuce .
By definition A Game A = 1 and A Game B = 0.
Conditioning on the first move out of state Adv A:
A Adv A = 0.6 \times  A Game A + 0.4 \times  A Deuce = 0.6 + 0.4 \times  A Deuce .
Similarly:
A Adv B = 0.6 \times  A Deuce ,
and
A Deuce = 0.6 \times  A Adv A + 0.4 \times  A Adv B = 0.6 \times  A Adv A + 0.24 \times  A Deuce .
Page 13%%%%%%%%%%%%%%%%%%%%%%%%%%%%%%%%5 — September 2007 — Examiners’ Report
So,
A Deuce =
0.6
A Adv A ,
0.76
A Adv A = 0.6 + 0.4 \times 
0.6
A Adv A ,
0.76
and
A Adv A = 0.8769 ,
and
A Deuce = 0.6923 .
ALTERNATIVELY
Probability A wins after 2 points = 0.6*0.6 =0.36
Probability that A wins from Deuce
=
∞
∑
Probability A wins after i points have been played
i =1
= Probability A wins after 2 points
+ Probability A wins after 4 points +.....
(as period 2)
= 0.36 + 0.48 * 0.36 + 0.48 2 * 0.36 +.......
= 0.36/(1-0.48) as a geometric progression
= 0.6923
(b)
\end{document}
