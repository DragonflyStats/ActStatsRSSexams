\documentclass[a4paper,12pt]{article}

%%%%%%%%%%%%%%%%%%%%%%%%%%%%%%%%%%%%%%%%%%%%%%%%%%%%%%%%%%%%%%%%%%%%%%%%%%%%%%%%%%%%%%%%%%%%%%%%%%%%%%%%%%%%%%%%%%%%%%%%%%%%%%%%%%%%%%%%%%%%%%%%%%%%%%%%%%%%%%%%%%%%%%%%%%%%%%%%%%%%%%%%%%%%%%%%%%%%%%%%%%%%%%%%%%%%%%%%%%%%%%%%%%%%%%%%%%%%%%%%%%%%%%%%%%%%

\usepackage{eurosym}
\usepackage{vmargin}
\usepackage{amsmath}
\usepackage{graphics}
\usepackage{epsfig}
\usepackage{enumerate}
\usepackage{multicol}
\usepackage{subfigure}
\usepackage{fancyhdr}
\usepackage{listings}
\usepackage{framed}
\usepackage{graphicx}
\usepackage{amsmath}
\usepackage{chngpage}

%\usepackage{bigints}
\usepackage{vmargin}

% left top textwidth textheight headheight

% headsep footheight footskip

\setmargins{2.0cm}{2.5cm}{16 cm}{22cm}{0.5cm}{0cm}{1cm}{1cm}

\renewcommand{\baselinestretch}{1.3}

\setcounter{MaxMatrixCols}{10}

\begin{document}

%%%%%%%%%%%%%%%%%%%%%%%%%%%%%%%%%%%%%%%%%%%%%%%%%%%%%%%%%%%%%%%%%%%%%%%%%%%%%%%%%%%%%
4

The actuary to a large pension scheme carried out an investigation of the mortality of
the scheme’s pensioners over the two years from 1 January 2005 to 1 January 2007.

(i)
List the data required by the actuary for an exact calculation of the central
exposed to risk for lives aged x.

The following is an extract from the data collected by the actuary.
Age x
nearest
birthday
63
64
65
66
67
%%-- CT4 A2007—2
Number of pensioners at:
Deaths during:
1 January
2005 1 January
2006 1 January
2007 2005 2006
1,248
1,465
1,678
1,719
1,686 1,312
1,386
1,720
1,642
1,695 1,290
1,405
1,622
1,667
1,601 10
13
16
22
19 6
15
23
19
25(ii)

%%%%%%%%%%%%%%%%%%%%%%%%%%%%%%%%%%%%%%%%%%%%%%%%%%%%%%%%%%%%%%%%%%%%%%%%%%%%
\newpage
4
(i)
For each pensioner in the investigation, the actuary would need:
Date of entry into the investigation
(the latest of date of retirement, date of xth birthday and 1 January 2005)
Date of exit from the investigation
(the earliest of date of death, date of (x+1)th birthday and 1 January 2007)
(ii)
(a)
The central exposed to risk of pensioners aged x nearest birthday is
given by
2
E x c = \int P x , t
0
1
≈ ∑ 1 2 ( P x , t + P x , t + 1 ) =
0
1 P
2 x ,0
+ P x ,1 + 1 2 P x ,2
Where P x , t is the number of pensioners aged x nearest birthday at time
t , measured from 1 January 2005.
This assumes that P x , t is linear over the calendar year.
(b)
This is a life year rate interval, from age x - 1⁄2 to x +1⁄2. The age in the
middle of the rate interval is x , so \mû estimates \mu x , assuming a constant
force of mortality over the life year.
The estimate of \mu x is therefore given by:
\mu ˆ 65 =
=
d 65,2005 + d 65,2006
c
E 65
(
16 + 23
1 \times1678 + 1720 + 1 \times1622
2
2
)
=
39
3370
= 0.01157
