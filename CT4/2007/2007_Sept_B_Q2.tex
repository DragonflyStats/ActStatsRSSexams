\documentclass[a4paper,12pt]{article}

%%%%%%%%%%%%%%%%%%%%%%%%%%%%%%%%%%%%%%%%%%%%%%%%%%%%%%%%%%%%%%%%%%%%%%%%%%%%%%%%%%%%%%%%%%%%%%%%%%%%%%%%%%%%%%%%%%%%%%%%%%%%%%%%%%%%%%%%%%%%%%%%%%%%%%%%%%%%%%%%%%%%%%%%%%%%%%%%%%%%%%%%%%%%%%%%%%%%%%%%%%%%%%%%%%%%%%%%%%%%%%%%%%%%%%%%%%%%%%%%%%%%%%%%%%%%

\usepackage{eurosym}
\usepackage{vmargin}
\usepackage{amsmath}
\usepackage{graphics}
\usepackage{epsfig}
\usepackage{enumerate}
\usepackage{multicol}
\usepackage{subfigure}
\usepackage{fancyhdr}
\usepackage{listings}
\usepackage{framed}
\usepackage{graphicx}
\usepackage{amsmath}
\usepackage{chngpage}

%\usepackage{bigints}
\usepackage{vmargin}

% left top textwidth textheight headheight

% headsep footheight footskip

\setmargins{2.0cm}{2.5cm}{16 cm}{22cm}{0.5cm}{0cm}{1cm}{1cm}

\renewcommand{\baselinestretch}{1.3}

\setcounter{MaxMatrixCols}{10}

\begin{document}
\begin{enumerate}


A particular baker’s shop in a small town sells only one product: currant buns. These
currant buns are delicious and customers travel many miles to buy them.
Unfortunately, the buns do not keep fresh and cannot be stored overnight.
The baker’s practice is to bake a certain number of buns, K, before the shop opens
each morning, and then during the day to continue baking c buns per hour. He is
concerned that:
\item
\item
3
he does not run out of buns during the day; and
the number of buns left over at the end of each day is as few as possible
\begin{enumerate}[(i)]
\item (i) Describe a model which would allow you to estimate the probability that the baker will run out of buns. State any assumptions you make.
\item 
(ii) Determine the relevant expression for the probability that the baker will run out of buns, in terms of K, c, and B j , the number of buns bought by the day’s jth customer.
\end{enumerate}


%%%%%%%%%%%%%%%%%%%%%%%%%%%%%%%%%%%%%%%%%%%%%
2
(a)
Assume that, during each day, customers arrive at the shop according to a Poisson process.
Assume that the numbers of buns bought by each customer, the B j , are independent and identically distributed random variables.
Then if $X_t$ is the total number of buns sold between the beginning of the day and time $t$, (where $t$ is measured in hours since the shop opens), $X_t$ is a
compound Poisson process defined by
N t
X t = ∑ B j ,
j = 1
where the number of customers arriving between the shop opening and time t
is N t .
(b)
The probability that the baker will run out of buns is
N t
Pr[ K + ct − ∑ B j < 0]
j = 1
for some t.
%% 3
%%  — %%%%%%%%%%%%%%%%%%%%%%%%%%%%%%%%%%%%5 — September 2007 — Examiners’ Report
%%-- 4 — %%%%%%%%%%%%%%%%%%%%%%%%%%%%%%%%%%%%5 — September 2007 — Examiners’ Report
