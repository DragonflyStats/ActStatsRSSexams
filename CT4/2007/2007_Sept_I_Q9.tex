\documentclass[a4paper,12pt]{article}

%%%%%%%%%%%%%%%%%%%%%%%%%%%%%%%%%%%%%%%%%%%%%%%%%%%%%%%%%%%%%%%%%%%%%%%%%%%%%%%%%%%%%%%%%%%%%%%%%%%%%%%%%%%%%%%%%%%%%%%%%%%%%%%%%%%%%%%%%%%%%%%%%%%%%%%%%%%%%%%%%%%%%%%%%%%%%%%%%%%%%%%%%%%%%%%%%%%%%%%%%%%%%%%%%%%%%%%%%%%%%%%%%%%%%%%%%%%%%%%%%%%%%%%%%%%%

\usepackage{eurosym}
\usepackage{vmargin}
\usepackage{amsmath}
\usepackage{graphics}
\usepackage{epsfig}
\usepackage{enumerate}
\usepackage{multicol}
\usepackage{subfigure}
\usepackage{fancyhdr}
\usepackage{listings}
\usepackage{framed}
\usepackage{graphicx}
\usepackage{amsmath}
\usepackage{chngpage}

%\usepackage{bigints}
\usepackage{vmargin}

% left top textwidth textheight headheight

% headsep footheight footskip

\setmargins{2.0cm}{2.5cm}{16 cm}{22cm}{0.5cm}{0cm}{1cm}{1cm}

\renewcommand{\baselinestretch}{1.3}

\setcounter{MaxMatrixCols}{10}

\begin{document}
%%%%%%%%%%%%%%%%%%%%%%%%%%%%%%%%%%%%%%%%%%%%%%%%%%%%%%

In a game of tennis, when the score is at “Deuce” the player winning the next point
holds “Advantage”. If a player holding “Advantage” wins the following point that
player wins the game, but if that point is won by the other player the score returns to
“Deuce”.
When Andrew plays tennis against Ben, the probability of Andrew winning any point
is 0.6. Consider a particular game when the score is at “Deuce”.
\begin{enumerate}[(a)]
\item (i)
Show that the subsequent score in the game can be modelled as a Markov
Chain, specifying both:
(a)
(b)
\item (ii)
the state space; and
the transition matrix

State, with reasons, whether the chain is:
(a)
(b)
irreducible; and
aperiodic

\item (iii) Calculate the number of points which must be played before there is more than
a 90\% chance of the game having been completed.

\item (iv) (a)
(b)
CT4 S2007—6
Calculate the probability that Andrew wins the game.
Comment on your answer.

\end{enumerate}

\newpage
%%%%%%%%%%%%%%%%%%%%%%%%%%%%%%%%%%%%%%%%%%%%%%%%%%%%
9
(i)
State space:
{Deuce, Advantage A(ndrew), Advantage B(en),
Game A(ndrew), Game B(en)}.
Transition matrix:
Deuce
Adv A
Adv B
Game A
Game B
Deuce Adv A Adv B
0
0.4
0.6
0
0 0.6
0
0
0
0 0.4
0
0
0
0
Game
A
0
0.6
0
1
0
Game
B
0
0
0.4
0
1
The chain is Markov because the probability of moving to the next state does not depend on history prior to entering that state (because the probability of
each player winning a point is constant)
(ii)
The chain is reducible because it has two absorbing states Game A and
Game B.
States Game A and Game B are absorbing so have no period. The other three
states each have a period of 2 so the chain is not aperiodic.
Page 12%%%%%%%%%%%%%%%%%%%%%%%%%%%%%%%%5 — September 2007 — Examiners’ Report
(iii)
The game either ends after 2 points or it returns to Deuce.
The probability of it returning to Deuce after two points is:
Prob A wins 1 st point \times  Prob B wins 2 nd point
+ Prob B wins 1 st point \times  Prob A wins 2 nd point
= 0.6 \times  0.4 + 0.4 \times  0.6 = 0.48.
[This can also be obtained by calculating the square of the transition matrix.]
Need to find number of such cycles N such that:
0.48 N < 1 − 0.9 ,
so that
N >
ln 0.1
> 3.14 .
ln(0.48)
But the game can only finish every two points so we require 4 cycles, that is 8
points.
(iv)
(a)
Define A X to be the probability that A ultimately wins the game when
the current state is X .
We require A Deuce .
By definition A Game A = 1 and A Game B = 0.
Conditioning on the first move out of state Adv A:
A Adv A = 0.6 \times  A Game A + 0.4 \times  A Deuce = 0.6 + 0.4 \times  A Deuce .
Similarly:
A Adv B = 0.6 \times  A Deuce ,
and
A Deuce = 0.6 \times  A Adv A + 0.4 \times  A Adv B = 0.6 \times  A Adv A + 0.24 \times  A Deuce .
Page 13%%%%%%%%%%%%%%%%%%%%%%%%%%%%%%%%5 — September 2007 — Examiners’ Report
So,
A Deuce =
0.6
A Adv A ,
0.76
A Adv A = 0.6 + 0.4 \times 
0.6
A Adv A ,
0.76
and
A Adv A = 0.8769 ,
and
A Deuce = 0.6923 .
ALTERNATIVELY
Probability A wins after 2 points = 0.6*0.6 =0.36
Probability that A wins from Deuce
=
∞
∑
Probability A wins after i points have been played
i =1
= Probability A wins after 2 points
+ Probability A wins after 4 points +.....
(as period 2)
= 0.36 + 0.48 * 0.36 + 0.48 2 * 0.36 +.......
= 0.36/(1-0.48) as a geometric progression
= 0.6923
(b)
