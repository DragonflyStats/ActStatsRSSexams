\documentclass[a4paper,12pt]{article}

%%%%%%%%%%%%%%%%%%%%%%%%%%%%%%%%%%%%%%%%%%%%%%%%%%%%%%%%%%%%%%%%%%%%%%%%%%%%%%%%%%%%%%%%%%%%%%%%%%%%%%%%%%%%%%%%%%%%%%%%%%%%%%%%%%%%%%%%%%%%%%%%%%%%%%%%%%%%%%%%%%%%%%%%%%%%%%%%%%%%%%%%%%%%%%%%%%%%%%%%%%%%%%%%%%%%%%%%%%%%%%%%%%%%%%%%%%%%%%%%%%%%%%%%%%%%

\usepackage{eurosym}
\usepackage{vmargin}
\usepackage{amsmath}
\usepackage{graphics}
\usepackage{epsfig}
\usepackage{enumerate}
\usepackage{multicol}
\usepackage{subfigure}
\usepackage{fancyhdr}
\usepackage{listings}
\usepackage{framed}
\usepackage{graphicx}
\usepackage{amsmath}
\usepackage{chngpage}

%\usepackage{bigints}
\usepackage{vmargin}

% left top textwidth textheight headheight

% headsep footheight footskip

\setmargins{2.0cm}{2.5cm}{16 cm}{22cm}{0.5cm}{0cm}{1cm}{1cm}

\renewcommand{\baselinestretch}{1.3}

\setcounter{MaxMatrixCols}{10}

\begin{document}
\begin{enumerate}[(a)]
% [Total 7]
% PLEASE TURN OVER5
\item
Explain why crude mortality rates are graduated before being used for
financial calculations.
\item 
List two methods of graduating a set of crude mortality rates and state, for
each method:
(a)
(b)
under what circumstances it should be used; and
how smoothness is ensured
\end{enumerate}

%%%%%%%%%%%%%%%%%%%%%%%%%%%%%%%%%%%%%%%%%%%%%%%%%%%%%%%
\newpage

5
(i)
We assume that mortality rates progress smoothly with age.
Therefore a crude estimate at age x carries information about the rates at
adjacent ages, and graduation allows us to use this fact to “improve” the
estimate at age x by smoothing.
This reduces the sampling errors at each age.
It is desirable that financial quantities progress smoothly with age,
as irregularities are hard to justify to clients.
(ii)
Any two of the following three methods are acceptable:
By parametric formula:
Should be used for large experiences, especially if the aim is to produce a
standard table;
Depends on a suitable formula being found which fits the data well.
Provided the number of parameters is small, the resulting curve should be
smooth.
With reference to a standard table
Should be used if a standard table for a class of lives similar to the experience
is available, and the experience we are interested in does not provide much
data.
The standard table will be smooth,
and provided the function linking the graduated rates to the rates in the
standard table is simple, this smoothness will be “transferred to the graduated
rates”.
Graphical
if a quick check is needed, or data are very scanty.
The graduation should be tested for smoothness using the third differences of
the graduated rates, which should be small in magnitude and progress
regularly with age.
If the smoothness is unsatisfactory, the curve can be adjusted (“hand-
polishing”) and the smoothness tested again.
%%%%------Page 6%%%%%%%%%%%%%%%%%%%%%%%%%%%%%%%%5 — September 2007 — Examiners’ Report
\newpage


6
Below is an extract from English Life Table 15 (Males)
(i)
Age x l x
58
62 88,792
84,173
Estimate l 60 under each of the following assumptions:
(a) a uniform distribution of deaths between exact ages 58 and 62 years;
and
(b) a constant force of mortality between exact ages 58 and 62 years

(ii)
Find the actual value of l 60 in the tables and hence comment on the relative validity of the two assumptions you used in part (i).

[Total 8]

\end{document}
