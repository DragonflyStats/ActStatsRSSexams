\documentclass[a4paper,12pt]{article}

%%%%%%%%%%%%%%%%%%%%%%%%%%%%%%%%%%%%%%%%%%%%%%%%%%%%%%%%%%%%%%%%%%%%%%%%%%%%%%%%%%%%%%%%%%%%%%%%%%%%%%%%%%%%%%%%%%%%%%%%%%%%%%%%%%%%%%%%%%%%%%%%%%%%%%%%%%%%%%%%%%%%%%%%%%%%%%%%%%%%%%%%%%%%%%%%%%%%%%%%%%%%%%%%%%%%%%%%%%%%%%%%%%%%%%%%%%%%%%%%%%%%%%%%%%%%

\usepackage{eurosym}
\usepackage{vmargin}
\usepackage{amsmath}
\usepackage{graphics}
\usepackage{epsfig}
\usepackage{enumerate}
\usepackage{multicol}
\usepackage{subfigure}
\usepackage{fancyhdr}
\usepackage{listings}
\usepackage{framed}
\usepackage{graphicx}
\usepackage{amsmath}
\usepackage{chngpage}

%\usepackage{bigints}
\usepackage{vmargin}

% left top textwidth textheight headheight

% headsep footheight footskip

\setmargins{2.0cm}{2.5cm}{16 cm}{22cm}{0.5cm}{0cm}{1cm}{1cm}

\renewcommand{\baselinestretch}{1.3}

\setcounter{MaxMatrixCols}{10}

\begin{document}
\begin{enumerate}
10
(i)
Compare the advantages and disadvantages of fully parametric models and the
Cox regression model for assessing the impact of covariates on survival. 
You have been asked to investigate the impact of a set of covariates, including age, sex, smoking, region of residence, educational attainment and amount of exercise undertaken, on the risk of heart attack. Data are available from a prospective study
which followed a set of several thousand persons from an initial interview until their first heart attack, or until their death from a cause other than a heart attack, or until 10 years had elapsed since the initial interview (whichever of these occurred first).
(ii) State the types of censoring present in this study, and explain how each arises.

(iii) Describe a criterion which would allow you to select those covariates which
have a statistically significant effect on the risk of heart attack, when
controlling the other covariates of the model.

Suppose your final model is a Cox model which has three covariates: age (measured
in age last birthday minus 50 at the initial interview), sex (male = 0, female = 1) and
smoking (non-smoker = 0, smoker = 1), and that the estimated parameters are:
Age
Sex
Smoking
Sex x smoking
0.01
-0.4
0.5
-0.25
where “sex x smoking” is an additional covariate formed by multiplying the two
covariates “sex” and “smoking”.
(iv)
(v)
Describe the final model’s estimate of the effect of sex and of smoking
behaviour on the risk of heart attack.

Use the results of the model to determine how old a female smoker must be at
the initial interview to have the same risk of heart attack as a male non-smoker
aged 50 years at the initial interview.

[Total 15]
CT4 S2007—7
%%%%%%%%%%%%%%%%%%%%%%%%%%%%%%%%%%%%%%%%%%%%%%%%%%%%%%%%%%%%%%%%%%%%%%%%%%%%%%%
10
(i)
This is higher than 0.6 because Ben has to win at least two points in a
row to win the game.
Fully parametric models are good for comparing homogenous groups, as
confidence intervals for the fitted parameters give a test of difference between the groups which should be better than non-parametric procedures, or semi-parametric procedures such as the Cox model.
But parametric methods need foreknowledge of the form of the hazard function, which might be the object of the study.
The Cox model is semi-parametric so such knowledge is not required. 14Subject CT4 — %%%%%%%%%%%%%%%%%%%%%%%%%%%%%%%%%%%%5 — September 2007 — Examiners’ Report
The Cox model is a standard feature of many statistical packages for
estimating survival model, but many parametric distributions are not, and numerical methods may be required, entailing additional programming.
(ii)
Type I censoring, since the investigation ends after a period which is fixed in advance.
Random censoring, since death from a cause other than a heart attack is a random variable and may occur at any time.
(iii)
The likelihood ratio statistic is a common criterion.
Suppose we fit a model with p covariates and another model with $p + q$
covariates which include all the p covariates of the first model.
Then if the maximised log-likelihoods of the two models are $L_p$ and $L_{p + q}$ , then
the statistic
\[− 2( L_p \;+\;L_{p + q})\]
has a chi-squared distribution with q degrees of freedom, under the hypothesis that the extra q covariates have no effect in the presence of the original $p$ covariates.
This statistic can be used either will full likelihoods or with partial likelihoods in the Cox model
This statistic can be used to test the statistical significance of any set of $q$ covariates in the presence of any other disjoint set of p covariates.
(iv)
Holding other factors constant,
females have a lower risk of heart attack than males,
and smokers have a higher risk than non-smokers,
but the effect of smoking varies for men and women.
The relative risks, compared with the baseline category of male non-smokers
are as follows.
female non-smokers
male smokers
female smokers
= 0.67
exp( - 0.4)
exp(0.5)
= 1.65
exp( - 0.4+0.5 - 0.25) = 0.86
(or any other numerical example to illustrate the previous points)
15Subject CT4 — %%%%%%%%%%%%%%%%%%%%%%%%%%%%%%%%%%%%5 — September 2007 — Examiners’ Report
(v)
Let the required age for the woman smoker be 50+ x .
The hazard for this woman is
\[h ( t , x ) = h_0 ( t ) exp(0.01 x – 0.4 + 0.5 – 0.25),\]
The hazard for a male non-smoker aged 50 at the initial interview is simply
$h_0 ( x )$, since this is the baseline category.
Thus we have
\[h 0 ( t ) exp(0.01 x – 0.4 + 0.5 – 0.25) = h 0 ( t )\]
so that
\[exp(0.01 x – 0.4 + 0.5 – 0.25) = 1\]
or
exp(0.01 x - 0.15) = 1
so that
0.01 x = 0.15
Therefore x = 15, and the woman’s age at interview must be 65 years.
\end{document}
