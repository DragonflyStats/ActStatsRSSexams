\documentclass[a4paper,12pt]{article}

%%%%%%%%%%%%%%%%%%%%%%%%%%%%%%%%%%%%%%%%%%%%%%%%%%%%%%%%%%%%%%%%%%%%%%%%%%%%%%%%%%%%%%%%%%%%%%%%%%%%%%%%%%%%%%%%%%%%%%%%%%%%%%%%%%%%%%%%%%%%%%%%%%%%%%%%%%%%%%%%%%%%%%%%%%%%%%%%%%%%%%%%%%%%%%%%%%%%%%%%%%%%%%%%%%%%%%%%%%%%%%%%%%%%%%%%%%%%%%%%%%%%%%%%%%%%

\usepackage{eurosym}
\usepackage{vmargin}
\usepackage{amsmath}
\usepackage{graphics}
\usepackage{epsfig}
\usepackage{enumerate}
\usepackage{multicol}
\usepackage{subfigure}
\usepackage{fancyhdr}
\usepackage{listings}
\usepackage{framed}
\usepackage{graphicx}
\usepackage{amsmath}
\usepackage{chngpage}

%\usepackage{bigints}
\usepackage{vmargin}

% left top textwidth textheight headheight

% headsep footheight footskip

\setmargins{2.0cm}{2.5cm}{16 cm}{22cm}{0.5cm}{0cm}{1cm}{1cm}

\renewcommand{\baselinestretch}{1.3}

\setcounter{MaxMatrixCols}{10}

\begin{document}
\begin{enumerate}
CT4 S2007—47
In order to boost sales, a national newspaper in a European country wishes to compile
a “fair play league table” for the country’s leading football clubs. On 1 December it
undertakes a survey of all the players who play for these clubs, in which it collects the
following data:
• number of games played by each player since the beginning of the season (the
football season in this country begins in September); and
• for each player who had been dismissed from the field of play between the
beginning of the season and 1 December (inclusive), the number of games he had
played before the game in which he was first dismissed
No games were played on 1 December.
The statistic the newspaper proposes to use in order to construct its “fair play league
table” is the probability that a player will not have been dismissed in any of his first
10 games. It plans to calculate this statistic for each of the 20 leading clubs.
The following table shows the data collected for the players of the club which was top
of the league on 1 December.
(i)
(ii)
Player Total number
of games played Number of times
dismissed
1
2
3
4
5
6
7
8
9
10
11
12
13
14
15 12
12
12
12
12
12
10
9
9
8
6
5
5
4
4 0
0
1
0
1
0
0
1
1
0
2
0
0
1
0
Games
played before
first dismissal
5
7
0
5
2
0
(a) Explain how the Kaplan-Meier estimator can be used to estimate the
newspaper’s statistic from these data.
(b) Comment on the way in which censoring arises and on the type of
censoring produced.
[4]
Calculate the newspaper’s statistic using the data above.
CT4 S2007—5
[4]
[Total 8]
%%%%%%%%%%%%%%%%%%%%%%%%%%%%%%%%%%%%%%%%%%%%%%%%%%%%%%%%%%%%%%%%%%%%%%
7
(i)
(a)
If, for player i , T i is the number of games played before he is
dismissed, and C i is the total number of games played before
1 December, and d i = 1 if the player had been dismissed before
1 December and 0 otherwise.
then
EITHER
from the data given we can create the two variables
min( T i , C i )
and d i ,
e.g. for player 1, min( T i , C i ) = 12 and d i = 0
OR
Page 8Subject CT4 — Models Core Technical — September 2007 — Examiners’ Report
The required data for the Kaplan-Meier estimator are therefore
(b)
Player min(T i ,C i ) d i
1
2
3
4
5
6
7
8
9
10
11
12
13
14
15 12
12
5
12
7
12
10
0
5
8
2
5
5
0
4 0
0
1
0
1
0
0
1
1
0
1
0
0
1
0
Censoring in these data arises because not all players have been
dismissed before 1 December. Those players who have yet to be
dismissed on that data are right-censored.
This censoring is random [NOT Type I], because the metric of
“duration” is the number of games played since the start of the season,
and this may vary from player to player.
(ii)
ALTERNATIVE 1 (where censorings are assumed to occur immediately
before events)
D j
D j
1 −
t j
N j
D j
C j
N j
N j
0
2
5
7
15
13
9
7
2
1
2
1
0
3
0
6
2/15
1/13
2/9
1/7
13/15
12/13
7/9
6/7
Then the Kaplan-Meier estimate of the survival function is
^
t S ( t )
0≤ t < 2
2≤ t < 5
5≤ t < 7
7≤ t < 12 0.8667
0.8000
0.6222
0.5333
^
Therefore the value of the chosen statistic, S (10) is 0.5333.
Page 9Subject CT4 — Models Core Technical — September 2007 — Examiners’ Report
ALTERNATIVE 2 (where censorings are assumed to occur immediately after
events)
D j
D j
t j
N j
D j
C j
1 −
N j
N j
0
2
5
7
15
13
11
7
2
1
2
1
0
1
2
6
2/15
1/13
2/11
1/7
13/15
12/13
9/11
6/7
Then the Kaplan-Meier estimate of the survival function is
^
t
S ( t )
0≤ t < 2
2≤ t < 5
5≤ t < 7
7≤ t < 12
0.8667
0.8000
0.6545
0.5610
^
Therefore the value of the chosen statistic, S (10) is 0.5610.
