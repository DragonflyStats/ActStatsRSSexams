\documentclass[a4paper,12pt]{article}

%%%%%%%%%%%%%%%%%%%%%%%%%%%%%%%%%%%%%%%%%%%%%%%%%%%%%%%%%%%%%%%%%%%%%%%%%%%%%%%%%%%%%%%%%%%%%%%%%%%%%%%%%%%%%%%%%%%%%%%%%%%%%%%%%%%%%%%%%%%%%%%%%%%%%%%%%%%%%%%%%%%%%%%%%%%%%%%%%%%%%%%%%%%%%%%%%%%%%%%%%%%%%%%%%%%%%%%%%%%%%%%%%%%%%%%%%%%%%%%%%%%%%%%%%%%%

\usepackage{eurosym}
\usepackage{vmargin}
\usepackage{amsmath}
\usepackage{graphics}
\usepackage{epsfig}
\usepackage{enumerate}
\usepackage{multicol}
\usepackage{subfigure}
\usepackage{fancyhdr}
\usepackage{listings}
\usepackage{framed}
\usepackage{graphicx}
\usepackage{amsmath}
\usepackage{chngpage}

%\usepackage{bigints}
\usepackage{vmargin}

% left top textwidth textheight headheight

% headsep footheight footskip

\setmargins{2.0cm}{2.5cm}{16 cm}{22cm}{0.5cm}{0cm}{1cm}{1cm}

\renewcommand{\baselinestretch}{1.3}

\setcounter{MaxMatrixCols}{10}

\begin{document}
\begin{enumerate}
1
(a)
Define, in the context of stochastic processes, a:
1. mixed process
2. counting process
(b)
Give an example application of each type of process.
%%%%%%%%%%%%%%%%%%%%%%%%%%%%%%%%%%%%%%%%%%%%%%%%%%%%%%%%%%%%%%%%%%%%%%%%%%%%%%%%%%%%%
2
An insurance company is investigating the mortality of its annuity policyholders. It is
proposed that the crude mortality rates be graduated for use in future premium
calculations.
(i)
(a) Suggest, with reasons, a suitable method of graduation in this case.
(b) Describe how you would graduate the crude rates.

(ii)
Comment on any further considerations that the company should take into
account before using the graduated rates for premium calculations.

[Total 5]
%%%%%%%%%%%%%%%%%%%%%%%%%%%%%%%%%%%%%%%%%%%%%%%%%%%%%%%%%%%%%%%%%%%%%%%%%%%%%%%%%%%%%
3

The government of a small country has asked you to construct a model for forecasting
future mortality.
Outline the stages you would go through in identifying an appropriate model.
%%%%%%%%%%%%%%%%%%%%%%%%%%%%%%%%%%%%%%%%%%%%%%%%%%%%%%%%%%%%%%%%%%%%%%%%%%%%%%%%%%%%%
4

The actuary to a large pension scheme carried out an investigation of the mortality of
the scheme’s pensioners over the two years from 1 January 2005 to 1 January 2007.
(i)
List the data required by the actuary for an exact calculation of the central
exposed to risk for lives aged x.

The following is an extract from the data collected by the actuary.
Age x
nearest
birthday
63
64
65
66
67
CT4 A2007—2
Number of pensioners at:
Deaths during:
1 January
2005 1 January
2006 1 January
2007 2005 2006
1,248
1,465
1,678
1,719
1,686 1,312
1,386
1,720
1,642
1,695 1,290
1,405
1,622
1,667
1,601 10
13
16
22
19 6
15
23
19
25(ii)

%%%%%%%%%%%%%%%%%%%%%%%%%%%%%%%%%%%%%%%%%%%%%%%%%%%%%%%%%%%%%%%%%%%%%%%%%%%%
1
Mixed process
(a) Is a stochastic process that operates in continuous time, which can also change
value at predetermined discrete instants.
(b) The number of contributors to a pension scheme can be modelled as a mixed
process with state space S = { 1, 2,3,... } and time interval J = [ 0, ∞ ] .
Counting process
(a)
Is a process, X, in discrete or continuous time, whose state space is the natural
numbers {0, 1, 2, ...}.
X(t) is a non-decreasing function of t.
2
(b) Number of claims reported to an insurer by time t.
(i) (a)
Graduation by reference to a standard table would be appropriate.
There are likely to be existing standard tables which are suitable and
this method is suitable for relatively small data sets.
Alternatively, graduation by parametric formula would be suitable if
the volume of data was large enough. But that is unlikely to be the
case here.
Graphical graduation would not be appropriate for rates for premium
calculations.
(b)
(ii)
Assuming graduation by reference to a standard table:
\item Select a suitable table, based on a similar group of lives.
\item Plot the crude rates against q x s from the standard table to identify a
simple relationship.
\item Find the best-fit parameters, using maximum likelihood or least
squares estimates.
\item Test the graduation for goodness of fit. If the fit is not adequate,
the process should be repeated.
Considerations include:
\item
As the premiums are for annuity policies, it is important not to
overestimate the mortality rates, as the premiums would be too low.
3
%%----Subject CT4 — %%%%%%%%%%%%%%%%%%%%%%%%%%%%%%%%%%%%5 — April 2007 — Examiners’ Report
3
\item The rates will be based on current mortality; the company should also take
into account expected future changes, especially any reductions in
mortality rates.
\item Premiums charged by other insurer: if rates are too high the company will
fail to attract business; if too low, it may attract too much, unprofitable
business.
\begin{itemize}
\item Clarify the purpose of the exercise. Why does the government want forecasts of
mortality? What is the period for which the forecast is wanted? Is it short (e.g. 5–10
years) or long (e.g. 50–70 years).
\item Consult the existing literature on models for forecasting mortality, and speak to
experts in this field of application. Consider using or adapting existing models which
are employed in other countries.
\item Establish what data are available (e.g. on past mortality trends in the country,
preferably with deaths classified by age and cause of death).
On the basis of what data are available, define the model you propose to use. If the
data are simple and not detailed, then a complex model is not justified. Will a
deterministic or a stochastic model be appropriate in this case?
\item Identify suitable computer software to implement the model, or, if none exists, write a
bespoke program.
\item Debug the program or, if existing software is used, check that it performs the
operations you intend it to do.
\item Run the model and test the reasonableness of the output. Consider, for example, the
forecast values of quantities such as the expectation of life at birth.
\item Test the sensitivity of the results to changes in the input parameters.
Analyse the output.
\item Write a report documenting the results and the model and communicate the results
and the output to the government of the small country.
\end{itemize}

Subject CT4 — %%%%%%%%%%%%%%%%%%%%%%%%%%%%%%%%%%%%5 — April 2007 — Examiners’ Report
4
(i)
For each pensioner in the investigation, the actuary would need:
Date of entry into the investigation
(the latest of date of retirement, date of xth birthday and 1 January 2005)
Date of exit from the investigation
(the earliest of date of death, date of (x+1)th birthday and 1 January 2007)
(ii)
(a)
The central exposed to risk of pensioners aged x nearest birthday is
given by
2
E x c = \int P x , t
0
1
≈ ∑ 1 2 ( P x , t + P x , t + 1 ) =
0
1 P
2 x ,0
+ P x ,1 + 1 2 P x ,2
Where P x , t is the number of pensioners aged x nearest birthday at time
t , measured from 1 January 2005.
This assumes that P x , t is linear over the calendar year.
(b)
This is a life year rate interval, from age x - 1⁄2 to x +1⁄2. The age in the
middle of the rate interval is x , so \mû estimates \mu x , assuming a constant
force of mortality over the life year.
The estimate of \mu x is therefore given by:
\mu ˆ 65 =
=
d 65,2005 + d 65,2006
c
E 65
(
16 + 23
1 \times1678 + 1720 + 1 \times1622
2
2
)
=
39
3370
= 0.01157
