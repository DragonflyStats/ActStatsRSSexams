\documentclass[a4paper,12pt]{article}

%%%%%%%%%%%%%%%%%%%%%%%%%%%%%%%%%%%%%%%%%%%%%%%%%%%%%%%%%%%%%%%%%%%%%%%%%%%%%%%%%%%%%%%%%%%%%%%%%%%%%%%%%%%%%%%%%%%%%%%%%%%%%%%%%%%%%%%%%%%%%%%%%%%%%%%%%%%%%%%%%%%%%%%%%%%%%%%%%%%%%%%%%%%%%%%%%%%%%%%%%%%%%%%%%%%%%%%%%%%%%%%%%%%%%%%%%%%%%%%%%%%%%%%%%%%%

\usepackage{eurosym}
\usepackage{vmargin}
\usepackage{amsmath}
\usepackage{graphics}
\usepackage{epsfig}
\usepackage{enumerate}
\usepackage{multicol}
\usepackage{subfigure}
\usepackage{fancyhdr}
\usepackage{listings}
\usepackage{framed}
\usepackage{graphicx}
\usepackage{amsmath}
\usepackage{chngpage}

%\usepackage{bigints}
\usepackage{vmargin}

% left top textwidth textheight headheight

% headsep footheight footskip

\setmargins{2.0cm}{2.5cm}{16 cm}{22cm}{0.5cm}{0cm}{1cm}{1cm}

\renewcommand{\baselinestretch}{1.3}

\setcounter{MaxMatrixCols}{10}

\begin{document}

3

The government of a small country has asked you to construct a model for forecasting
future mortality.
Outline the stages you would go through in identifying an appropriate model.
\newpage
3
\item The rates will be based on current mortality; the company should also take into account expected future changes, especially any reductions in
mortality rates.
\item Premiums charged by other insurer: if rates are too high the company will
fail to attract business; if too low, it may attract too much, unprofitable
business.
\begin{itemize}
\item Clarify the purpose of the exercise. Why does the government want forecasts of
mortality? What is the period for which the forecast is wanted? Is it short (e.g. 5–10
years) or long (e.g. 50–70 years).
\item Consult the existing literature on models for forecasting mortality, and speak to
experts in this field of application. Consider using or adapting existing models which are employed in other countries.
\item Establish what data are available (e.g. on past mortality trends in the country,
preferably with deaths classified by age and cause of death).
On the basis of what data are available, define the model you propose to use. If the
data are simple and not detailed, then a complex model is not justified. Will a
deterministic or a stochastic model be appropriate in this case?
\item Identify suitable computer software to implement the model, or, if none exists, write a
bespoke program.
\item Debug the program or, if existing software is used, check that it performs the
operations you intend it to do.
\item Run the model and test the reasonableness of the output. Consider, for example, the
forecast values of quantities such as the expectation of life at birth.
\item Test the sensitivity of the results to changes in the input parameters.
Analyse the output.
\item Write a report documenting the results and the model and communicate the results
and the output to the government of the small country.
\end{itemize}
\end{document}
