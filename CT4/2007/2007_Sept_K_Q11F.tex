\documentclass[a4paper,12pt]{article}

%%%%%%%%%%%%%%%%%%%%%%%%%%%%%%%%%%%%%%%%%%%%%%%%%%%%%%%%%%%%%%%%%%%%%%%%%%%%%%%%%%%%%%%%%%%%%%%%%%%%%%%%%%%%%%%%%%%%%%%%%%%%%%%%%%%%%%%%%%%%%%%%%%%%%%%%%%%%%%%%%%%%%%%%%%%%%%%%%%%%%%%%%%%%%%%%%%%%%%%%%%%%%%%%%%%%%%%%%%%%%%%%%%%%%%%%%%%%%%%%%%%%%%%%%%%%

\usepackage{eurosym}
\usepackage{vmargin}
\usepackage{amsmath}
\usepackage{graphics}
\usepackage{epsfig}
\usepackage{enumerate}
\usepackage{multicol}
\usepackage{subfigure}
\usepackage{fancyhdr}
\usepackage{listings}
\usepackage{framed}
\usepackage{graphicx}
\usepackage{amsmath}
\usepackage{chngpage}

%\usepackage{bigints}
\usepackage{vmargin}

% left top textwidth textheight headheight

% headsep footheight footskip

\setmargins{2.0cm}{2.5cm}{16 cm}{22cm}{0.5cm}{0cm}{1cm}{1cm}

\renewcommand{\baselinestretch}{1.3}

\setcounter{MaxMatrixCols}{10}

\begin{document}
\begin{enumerate}
PLEASE TURN OVER11
The following data have been collected from observation of a three-state process in
continuous time:
State
occupied Total time
spent in state
(hours)
Total transitions to:
State A State B State C
A
B
C 50
25
90 Not applicable
80
120 110
Not applicable
15 90
45
Not applicable
It is proposed to fit a Markov jump model to this data set.

\begin{enumerate}[(a)]
\item 
\begin{enumerate}[(i)]
\item List all the parameters of the model.
\item Describe the assumptions underlying the model. 
\end{enumerate}
\item
\begin{enumerate}[(i)]
\item Estimate the parameters of the model.
\item Give the estimated generator matrix. 
\end{enumerate}
\end{enumerate}
%%%%%%%%%%%%%%%%%%%%%%%%%%%%%%%%%%%%%%%%%%%%%%
The following additional data in respect of secondary transitions were collected from
observation of the same process.
Triplet of
successive
transitions Observed
number of
triplets
n ijk Triplet of
successive
transitions Observed
number of
triplets
n ijk
ABC
ABA
ACA
ACB
BAB
BAC 42
68
85
4
50
30 BCA
BCB
CAB
CAC
CBA
CBC 38
7
64
56
8
7
(iii) State the distribution of the number of transitions from state i to state j , given
the number of transitions out of state i .

(iv) Test the goodness-of-fit of the model by considering whether triplets of
successive transitions adhere to the distribution given in (iii).
[Hint: Use the test statistic χ = ∑∑∑
2
i
j
( n ijk − E ) 2
k
E

where E is the expected
number of triplets under the distribution in (iii)]
(v) Identify two other aspects of the appropriateness of the fitted model that could
be tested, stating suitable tests in each case.

(vi) Outline two methods for simulating the Markov jump process, without
performing any calculations.

%%%%%%%%%%%%%%%%%%%%%%%%%%%%%%%%%%%%%%%%%%%%%%%%%%%%%%%%%%%%%%%%%%%%%%%%%%%%%%%










11
(i)
(a)
The parameters are:
•
•
\begin{itemize}
\item the rate of leaving state i , \lambda i , for each i ,
\item the jump-chain transition probabilities, r ij , for $j \neq i$ , where r ij is the
conditional probability that the next transition is to state j given the
current state is i .
\end{itemize}
[Alternatively the parameters may be expressed as \sigma  ij , where \sigma  ii = - \lambda i
and (for j ≠ i), \sigma  ij = \lambda i r ij .]
(b)
The assumptions are as follows.
\begin{itemize}
\item The holding time in each state is exponentially distributed. The parameter of this distribution varies only by state i . The distribution is independent of anything that happened prior to the current arrival in state i .
\item The destination of the jump on leaving state i is independent of holding time, and of anything that happened prior to the current arrival in state i .
\end{itemize}
%%%%%%%%%%%%%%%%%%%%%%%%%%%%%%%%%Page 16%%%%%%%%%%%%%%%%%%%%%%%%%%%%%%%%5 — September 2007 — Examiners’ Report
ALTERNATIVELY
The holding time in each state is exponentially distributed and the
destination of the jump on leaving state i is independent of holding
time
Both holding time distribution and destination of jump on leaving state
i are independent of anything that happened prior to arrival in state i
(ii)
(a)
The estimator [it is the MLE but this need not
be stated] of \lambda i , \lambdâ , is the inverse of the average duration of each visit
to state i .
so $\hat{\lambda} A = 4 per hour$, $\hat{\lambda} B = 5 per hour$, $\hat{\lambda} C = 1.5 per hour$
The estimator [it is the MLE but this need not be stated] of r ij , r ˆ ij , is
the proportion of observed jumps out of state i to state j .
r ˆ AB = 11/20
r ˆ AC = 9/20
r ˆ BA = 80/125 =16/25
r ˆ BC = 9/25
r ˆ CA = 24/27 =8/9
r ˆ CB = 1/9
(b)
The estimated generator matrix (in hr - 1 ) is:
9 ⎞
⎛ − 4 11
5
5 ⎟
⎜
⎜ 16
9 ⎟
− 5
5 ⎟
⎜ 5
⎜ ⎜ 4
⎟
1
− 3 ⎟
6
2 ⎠
⎝ 3
(iii)
Distribution is binomial with mean n . r ij and variance n.r ij
(1 - r ij ), where n is the given number of transitions.
Page 17%%%%%%%%%%%%%%%%%%%%%%%%%%%%%%%%5 — September 2007 — Examiners’ Report
(iv)
Null hypothesis is that the Markov property applies to successive transitions, or that the observed triplets are from a Binomial distribution with the estimated parameters (given the number of transitions to the middle state).
Using test statistic given in the hint, we can draw up the table below.
Triplet n ijk
ABC
ABA
ACA
ACB
BAB
BAC
BCA
BCB
CAB
CAC
CBA
CBC 42
68
85
4
50
30
38
7
64
56
8
7
E=n i j r ˆ jk
Test statistic
39.6
70.4
80
10
44
36
40
5
66
54
9.6
5.4
( n ijk − E ) 2
E
0.1455
0.08182
0.3125
3.6
0.8182
1
0.1
0.8
0.0606
0.07407
0.2667
0.4741
7.7335
Under the null hypothesis, the test statistic follows a $\chi^2$ distribution with the
following number of degrees of freedom:
Number of triplets
Minus Number of pairs
Plus Number of states
Minus One
12
6
3
1
8 degrees of freedom
The critical value of $\chi^{2}_{8}$ at the 5\% significance level is 15.51
As $7.7335 < 15.51$ there is no evidence to reject the null hypothesis.

%% [Alternative approaches could be taken which resulted in a slightly different result for the test statistic. These were given %% full credit where appropriate.]

%% Page 18%%%%%%%%%%%%%%%%%%%%%%%%%%%%%%%%5 — September 2007 — Examiners’ Report
(v)

[Refer back to part (i) — the test in (iv) has only tested that there is no evidence that the destination that the next jump depends on the previous state occupied. Need to test the other assumptions].
Holding times — are these exponentially distributed?
A chi-squared goodness of fit test would be appropriate Is destination of jump independent of the holding time?
There is no obvious test statistic for doing this. A suitable test would be to
classify jumps as being from short, medium and long holding times and investigating these graphically.
(vi)
APPROXIMATE METHOD
Divide time into very short intervals, h , such that \sigma  ij h is much less than 1.
Simulate a discrete-time Markov chain ${ Y n : n \geq  0 }$ , with transition
probabilities p ij * ( h ) =  \delta_{ij} + h \sigma  ij .
The jump process, X t is given by X t = Y [ t h ] .
EXACT METHOD
Simulate the jump chain as a Markov chain, with transition probabilities
p ij = \sigma  ij \lambda i .
{
}
Once the path X ˆ n : n = 0,1,... has been generated, the holding times
{ T n : n = 0,1,... } are a sequence of independent exponential random variables,
having parameter \lambda X ˆ .
n
\end{document}
