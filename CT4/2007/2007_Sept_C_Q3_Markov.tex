\documentclass[a4paper,12pt]{article}

%%%%%%%%%%%%%%%%%%%%%%%%%%%%%%%%%%%%%%%%%%%%%%%%%%%%%%%%%%%%%%%%%%%%%%%%%%%%%%%%%%%%%%%%%%%%%%%%%%%%%%%%%%%%%%%%%%%%%%%%%%%%%%%%%%%%%%%%%%%%%%%%%%%%%%%%%%%%%%%%%%%%%%%%%%%%%%%%%%%%%%%%%%%%%%%%%%%%%%%%%%%%%%%%%%%%%%%%%%%%%%%%%%%%%%%%%%%%%%%%%%%%%%%%%%%%

\usepackage{eurosym}
\usepackage{vmargin}
\usepackage{amsmath}
\usepackage{graphics}
\usepackage{epsfig}
\usepackage{enumerate}
\usepackage{multicol}
\usepackage{subfigure}
\usepackage{fancyhdr}
\usepackage{listings}
\usepackage{framed}
\usepackage{graphicx}
\usepackage{amsmath}
\usepackage{chngpage}

%\usepackage{bigints}
\usepackage{vmargin}

% left top textwidth textheight headheight

% headsep footheight footskip

\setmargins{2.0cm}{2.5cm}{16 cm}{22cm}{0.5cm}{0cm}{1cm}{1cm}

\renewcommand{\baselinestretch}{1.3}

\setcounter{MaxMatrixCols}{10}

\begin{document}
\begin{enumerate}
1
2
List the factors you would consider when assessing the suitability of an actuarial
model for its purpose.

A particular baker’s shop in a small town sells only one product: currant buns. These
currant buns are delicious and customers travel many miles to buy them.
Unfortunately, the buns do not keep fresh and cannot be stored overnight.
The baker’s practice is to bake a certain number of buns, K, before the shop opens
each morning, and then during the day to continue baking c buns per hour. He is
concerned that:
\item
\item
3
he does not run out of buns during the day; and
the number of buns left over at the end of each day is as few as possible
\begin{enumerate}[(i)]
\item (i) Describe a model which would allow you to estimate the probability that the baker will run out of buns. State any assumptions you make.
\item 
(ii) Determine the relevant expression for the probability that the baker will run out of buns, in terms of K, c, and B j , the number of buns bought by the day’s jth customer.
\end{enumerate}

A no-claims discount system has 3 levels of discount: 0\%, 25\% and 50\%. The rules for moving between discount levels are:
\item After a claim-free year, move up to the next higher level or remain at the 50\% discount level.
\item After a year with one or more claims, move down to the next lower level or remain at the 0\% discount level.
The long-run probability that a policyholder is in the maximum discount level is 0.75. Calculate the probability that a given policyholder has a claim-free year, assuming that this probability is constant.
%%-- 


%%%%%%%%%%%%%%%%%%%%%%%%%%%%%%%%%%%%%%%%%%%%%%%%%%%%%%%%%%%%%%%%%%%%%%%%%%%%%%%%%%%

%%  — %%%%%%%%%%%%%%%%%%%%%%%%%%%%%%%%%%%%5 — September 2007 — Examiners’ Report
3
The transition matrix for the chain is:
\alpha
⎛ 1 − \alpha
⎞
⎜
⎟
\alpha ⎟ .
⎜ 1 − \alpha
⎜
1 − \alpha \alpha ⎟ ⎠
⎝
To determine the long-run probability, we need to solve the equation \pi P = \pi , which
reads:
(I) \pi_1 = ( 1 − \alpha ) \pi_1 + ( 1 − \alpha ) \pi_2
(II) \pi_2 =
(III) \pi_3 =
+ ( 1 − \alpha ) \pi_3
\alpha\pi_1
\alpha\pi_2
+
\alpha\pi_3 .
The probabilities must also satisfy:
(IV)
\pi_1 + \pi_2 + \pi_3 = 1 .
⎛ 1− \alpha ⎞
(III) gives \pi_2 = ⎜
⎟ \pi_3 .
⎝ \alpha ⎠
2
⎛ 1− \alpha ⎞
Substituting in (I) gives \pi_1 = ⎜
⎟ \pi_3 ,
⎝ \alpha ⎠
⎛ ⎛ 1 − \alpha ⎞ 2 ⎛ 1 − \alpha ⎞ ⎞
and so (IV) leads to ⎜ ⎜
+
+ 1 ⎟ \pi = 1 .
⎜ ⎝ \alpha ⎟ ⎠ ⎜ ⎝ \alpha ⎟ ⎠ ⎟ 3
⎝
⎠
We know that \pi_3 = 0.75 , which leads to:
⎛ ( 1 − \alpha ) 2 + \alpha ( 1 − \alpha ) + \alpha 2 ⎞
⎜
⎟ \times0.75 = 1 ,
2
⎜
⎟
\alpha
⎝
⎠
( (
) (
)
)
⇒ 0.75 1 − 2 \alpha + \alpha 2 + \alpha + \alpha 2 + \alpha 2 = \alpha 2 ,
⇒ 0.25 \alpha 2 + 0.75 \alpha − 0.75 = 0 .
Using the quadratic equation formula, this leads to
\alpha=
− 0.75 \pm 0.75 2 + 4 \times0.25 \times0.75
.
2 \times0.25
As \alpha > 0 , we must have \alpha = 0.7913 .
%%-- 4 — %%%%%%%%%%%%%%%%%%%%%%%%%%%%%%%%%%%%5 — September 2007 — Examiners’ Report
%%%%%%%%%%%%%%%%%%%%%%%%%%%%%%%%%%%%%%%%%%%%%%%%%%%%%%%%%%%%%%%55

\end{document}
%%-- 5 — %%%%%%%%%%%%%%%%%%%%%%%%%%%%%%%%%%%%5 — September 2007 — Examiners’ Report
