\documentclass[a4paper,12pt]{article}

%%%%%%%%%%%%%%%%%%%%%%%%%%%%%%%%%%%%%%%%%%%%%%%%%%%%%%%%%%%%%%%%%%%%%%%%%%%%%%%%%%%%%%%%%%%%%%%%%%%%%%%%%%%%%%%%%%%%%%%%%%%%%%%%%%%%%%%%%%%%%%%%%%%%%%%%%%%%%%%%%%%%%%%%%%%%%%%%%%%%%%%%%%%%%%%%%%%%%%%%%%%%%%%%%%%%%%%%%%%%%%%%%%%%%%%%%%%%%%%%%%%%%%%%%%%%

\usepackage{eurosym}
\usepackage{vmargin}
\usepackage{amsmath}
\usepackage{graphics}
\usepackage{epsfig}
\usepackage{enumerate}
\usepackage{multicol}
\usepackage{subfigure}
\usepackage{fancyhdr}
\usepackage{listings}
\usepackage{framed}
\usepackage{graphicx}
\usepackage{amsmath}
\usepackage{chngpage}

%\usepackage{bigints}
\usepackage{vmargin}

% left top textwidth textheight headheight

% headsep footheight footskip

\setmargins{2.0cm}{2.5cm}{16 cm}{22cm}{0.5cm}{0cm}{1cm}{1cm}

\renewcommand{\baselinestretch}{1.3}

\setcounter{MaxMatrixCols}{10}

\begin{document}
\begin{enumerate}

%%%%%%%%%%%%%%%%%%%%%%%%%%%%%%%%%%%%%%%%%%%%%%%%%%%%%%%%%%%%%%%%%%%%%%%%%%%%%%%%%%%%%
8
A medical study was carried out between 1 January 2001 and 1 January 2006, to
assess the survival rates of cancer patients. The patients all underwent surgery during
2001 and then attended 3-monthly check-ups throughout the study.
The following data were collected:
For those patients who died during the study exact dates of death were recorded as
follows:
Patient Date of surgery Date of death
A
B
C
D
E 1 April 2001
1 April 2001
1 May 2001
1 September 2001
1 October 2001 1 August 2005
1 October 2001
1 March 2002
1 August 2003
1 August 2002
For those patients who survived to the end of the study:
Patient Date of surgery
F
G
H
I
J
K
L 1 February 2001
1 March 2001
1 April 2001
1 June 2001
1 September 2001
1 September 2001
1 November 2001
For those patients with whom the hospital lost contact before the end of the
investigation:
Patient Date of surgery Date of last check-up
M
N
O 1 February 2001
1 June 2001
1 September 2001 1 August 2003
1 March 2002
1 September 2005
(i)
Explain whether and where each of the following types of censoring is present
in this investigation:
(a)
(b)
(c)
(ii)
(iii)
type I censoring
interval censoring; and
informative censoring
Calculate the Kaplan-Meier estimate of the survival function for these
patients. State any assumptions that you make.

[7]
Hence estimate the probability that a patient will die within 4 years of surgery.
%%%%%%%%%%%%%%%%%%%%%%%%%%%%%%%%%%%%%%%%%%%%%%%%%%%%%%%%%%%%%%%%%%%%%%%%%%%%%%%%%%%%%

%%%%%%%%%%%%%%%%%%%%%%%%%%%%%%%%%%%%%%%%%%%%%%%%%%%%%%%%%%%%%%%%%%%%%%%%%%%%%%%%%%%%%

8
(i)
(a) Type I censoring is present for those lives still under observation at 31
December 2005 as the censoring times are known in advance.
(b) Interval censoring would be present if we only knew death occurred
between check-ups. However, actual dates of death are known, so
interval censoring is not present.
Right censoring can be seen as a special case of interval censoring (for
those censored before death, we know death occurs in the interval (c i ,
∞ ) where c i is the censoring time for person i).
(c)
Page 10
Informative censoring is not likely to be present. The censoring of
lives gives us no information about future lifetimes.Subject CT4 — Models Core Technical — April 2007 — Examiners’ Report
(ii)
The durations at which lives died or were censored are shown below. Duration
is measured in years and months from the date of surgery.
Patient
A
B
C
D
E
F
G
H
I
J
K
L
M
N
O
Death or censored
death
death
death
death
death
censored
censored
censored
censored
censored
censored
censored
censored
censored
censored
Duration
4 years 4 months
6 months
10 months
1 year 11 months
10 months
4 years 11 months
4 years 10 months
4 years 9 months
4 years 7 months
4 years 4 months
4 years 4 months
4 years 2 months
2 years 6 months
9 months
4 years
The calculation of the survival function is shown in the table below. We
assume that at duration 4 years 4 months, the death occurred before lives were
censored.
t j n j d j c j \hat{\lambda} j = d j / n j
0
0.5
0.833
1.917
4.333 15
15
13
11
7 0
1
2
1
1 0
1
0
3
6 0
1/15
2/13
1/11
1/7
(
)
The estimated survival function is given by, S ˆ ( t ) = ∏ 1 − \lambda j . So,
t j ≤ t
t Ŝ ( t )
0.000 ≤ t < 0.500
0.500 ≤ t < 0.833
0.833 ≤ t < 1.917
1.917 ≤ t < 4.333
4.333 ≤ t < 5.0 1.0000
0.9333
0.7897
0.7179
0.6154
Solutions using different assumptions (for example assuming the death at 4
years 4 months occurred after lives were censored, or assuming lives M, N
and O were censored sometime within 3 months of their last check-up) were
acceptable and received credit.
Page 11Subject CT4 — Models Core Technical — April 2007 — Examiners’ Report
(iii)
The probability that a patient will die within 4 years of surgery is estimated
by:
1 − S ˆ ( 4 ) = 1 – 0.7179
= 0.2821
\end{document}
