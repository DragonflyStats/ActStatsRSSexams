\documentclass[a4paper,12pt]{article}

%%%%%%%%%%%%%%%%%%%%%%%%%%%%%%%%%%%%%%%%%%%%%%%%%%%%%%%%%%%%%%%%%%%%%%%%%%%%%%%%%%%%%%%%%%%%%%%%%%%%%%%%%%%%%%%%%%%%%%%%%%%%%%%%%%%%%%%%%%%%%%%%%%%%%%%%%%%%%%%%%%%%%%%%%%%%%%%%%%%%%%%%%%%%%%%%%%%%%%%%%%%%%%%%%%%%%%%%%%%%%%%%%%%%%%%%%%%%%%%%%%%%%%%%%%%%

\usepackage{eurosym}
\usepackage{vmargin}
\usepackage{amsmath}
\usepackage{graphics}
\usepackage{epsfig}
\usepackage{enumerate}
\usepackage{multicol}
\usepackage{subfigure}
\usepackage{fancyhdr}
\usepackage{listings}
\usepackage{framed}
\usepackage{graphicx}
\usepackage{amsmath}
\usepackage{chngpage}

%\usepackage{bigints}
\usepackage{vmargin}

% left top textwidth textheight headheight

% headsep footheight footskip

\setmargins{2.0cm}{2.5cm}{16 cm}{22cm}{0.5cm}{0cm}{1cm}{1cm}

\renewcommand{\baselinestretch}{1.3}

\setcounter{MaxMatrixCols}{10}

\begin{document}
\begin{enumerate}%%%%%%%%%%%%%%%%%%%%%%%%%%%%%%%%%%%%%%%%%%%%%%%%%%%%%%%%%%%%%%%%%%%%%%%%%%%%%%%%%%%%%
5
(a) Derive an expression that could be used to estimate the central exposed
to risk using the available data. State any assumptions you make.
(b) Use the data to estimate μ 65 . State any further assumptions that you
make.
[4]
[Total 6]
(i) Define the hazard rate, h(t), of a random variable T denoting lifetime.
[1]
(ii) An investigation is undertaken into the mortality of men aged between exact
ages 50 and 55 years. A sample of n men is followed from their 50th
birthdays until either they die or they reach their 55th birthdays.
The hazard of death (or force of mortality) between these ages, h(t), is
assumed to have the following form:
h ( t ) = α + β t
where α and β are parameters to be estimated and t is measured in years since
the 50th birthday.
(a) Derive an expression for the survival function between ages 50 and 55
years.
(b) Sketch this on a graph.
(c) Comment on the appropriateness of the assumed form of the hazard for
modelling mortality over this age range.
[6]
%%%%%%%%%%%%%%%%%%%%%%%%%%%%%%%%%%%%%%%%%%%%%%%%%%%%%%%%%%%%%%%%%%%%%%%%%%%%%%%%%%%%%
6
A three state process with state space {A, B, C} is believed to follow a Markov chain
with the following possible transitions:
A
B
C
An instrument was used to monitor this process, but it was set up incorrectly and only
recorded the state occupied after every two time periods. From these observations the
following two-step transition probabilities have been estimated:
2
P AA
= 0.5625
2
P AB
= 0.125
2
P BA
= 0.475
2
P CC
= 0.4
Calculate the one-step transition matrix consistent with these estimates.
%%%%%%%%%%%%%%%%%%%%%%%%%%%%%%%%%%%%%%%%%%%%%%%%%%%%%%%%%%%%%%%%%%%%%%%%%%%%%%%%%%%%%
Page 5Subject CT4 — Models Core Technical — April 2007 — Examiners’ Report
5
(i)
The hazard function is defined as
h ( t ) = lim
dt → 0 +
(ii)
(a)
1
( Pr[ T ≤ t + dt | T > t ] ) .
dt
Since the survival function S(t) is given by
⎛ t
⎞
S ( t ) = exp ⎜ − ∫ h ( s ) ds ⎟ ,
⎜
⎟
⎝ 0
⎠
then
t
⎛ t
⎞
⎡
⎡
β s 2 ⎤
β t 2 ⎤
S ( t ) = exp ⎜ − ∫ ( α + β s ) ds ⎟ = exp ⎢ −α s −
⎥ = exp ⎢ −α t −
⎥
⎜
⎟
2 ⎦ ⎥
2 ⎦ ⎥
⎢
⎢
⎣
⎣
⎝ 0
⎠
0
where 0 ≤ t ≤ 5 .
(b)
A suitable plot is shown below.
1
0.9
0.8
0.7
0.6
0.5
0.4
0.3
0.2
0.1
0
0
1
2
3
4
5
Duration since age 50 years
Both concave and convex plots were acceptable as this depends on
parameters, α and β.
(c)
If both α and β are positive, then the formula implies a force of
mortality which increases with age, which is sensible for this age
range.
The parameter α measures the ‘level’ of mortality and the parameter β
measures the rate of increase with age. Varying these permits quite a
wide range of forms for S(t).
So the formula seems appropriate.
Page 6Subject CT4 — Models Core Technical — April 2007 — Examiners’ Report
6
Based on the given transition diagram, the one-step transition matrix must be of the
form:
⎛ a
⎜
⎜ d
⎜ 0
⎝
0
e
h
c ⎞
⎟
f ⎟
i ⎟ ⎠
The two-step transition matrix is given by:
⎛ a
⎜
⎜ d
⎜ 0
⎝
0
e
h
c ⎞ ⎛ a
⎟ ⎜
f ⎟ * ⎜ d
i ⎟ ⎠ ⎜ ⎝ 0
0
e
h
2
ch
c ( a + i ) ⎞
c ⎞ ⎜ ⎛ a
⎟
⎟
f ⎟ = ⎜ d ( a + e ) e 2 + fh cd + ef + fi ⎟
⎜
⎟
i ⎟ ⎠ ⎜ dh
h ( e + i )
fh + i 2 ⎟
⎝
⎠
2
P AA
= 0.5625 ⇒ a 2 = 0.5625 ⇒ a = 0.75
Rows of transition matrix must sum to 1.
So,
and
a + c = 1
c = 0.25
2
P AB
= 0.125 ⇒ ch = 0.125 ⇒ h = 0.5
so
h + i = 1
i=0.5
2
P CC
= 0.4 ⇒ f × 0.5 + 0.5 2 = 0.4 ⇒ f = 0.3
2
P BA
= 0.475 ⇒ d (0.75 + e ) = 0.475
Rows sum to 1 so, d + e =0.7
Substitute for e:
d (1.45 − d ) = 0.475 ⇒ d 2 − 1.45 d + 0.475 = 0
Solving using standard quadratic formula:
d =
1.45 ± 1.45 2 − 4 × 0.475 1.45 ± 0.45
=
= 0.95 or 0.5
2
2
0.95 is not possible because e would need to be negative
So
d = 0.5 and e = 0.2
Page 7Subject CT4 — Models Core Technical — April 2007 — Examiners’ Report
Transition matrix is:
⎛ 0.75 0 0.25 ⎞
⎜ 0.5 0.2 0.3 ⎟
⎜
⎟
⎜ 0
0.5 0.5 ⎟ ⎠
⎝

