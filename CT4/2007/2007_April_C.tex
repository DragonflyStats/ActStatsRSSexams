\documentclass[a4paper,12pt]{article}

%%%%%%%%%%%%%%%%%%%%%%%%%%%%%%%%%%%%%%%%%%%%%%%%%%%%%%%%%%%%%%%%%%%%%%%%%%%%%%%%%%%%%%%%%%%%%%%%%%%%%%%%%%%%%%%%%%%%%%%%%%%%%%%%%%%%%%%%%%%%%%%%%%%%%%%%%%%%%%%%%%%%%%%%%%%%%%%%%%%%%%%%%%%%%%%%%%%%%%%%%%%%%%%%%%%%%%%%%%%%%%%%%%%%%%%%%%%%%%%%%%%%%%%%%%%%

\usepackage{eurosym}
\usepackage{vmargin}
\usepackage{amsmath}
\usepackage{graphics}
\usepackage{epsfig}
\usepackage{enumerate}
\usepackage{multicol}
\usepackage{subfigure}
\usepackage{fancyhdr}
\usepackage{listings}
\usepackage{framed}
\usepackage{graphicx}
\usepackage{amsmath}
\usepackage{chngpage}

%\usepackage{bigints}
\usepackage{vmargin}

% left top textwidth textheight headheight

% headsep footheight footskip

\setmargins{2.0cm}{2.5cm}{16 cm}{22cm}{0.5cm}{0cm}{1cm}{1cm}

\renewcommand{\baselinestretch}{1.3}

\setcounter{MaxMatrixCols}{10}

\begin{document}

%% Question 7
Every person has two chromosomes, each being a copy of one of the chromosomes
from one of their parents. There are two types of chromosomes labelled X and Y. A
child born with an X and a Y chromosome is male and a child with two X
chromosomes is female.
The blood-clotting disorder haemophilia is caused by a defective X chromosome
(X*). A female with the defective chromosome (X*X) will not usually exhibit
symptoms of the disease but may pass the defective gene to her children and so is
known as a carrier. A male with the defective chromosome (X*Y) suffers from the
disease and is known as a haemophiliac.
A medical researcher wishes to study the progress of the disease through the first born
child in each generation, starting with a female carrier.
You may assume:
\begin{itemize}
\item every parent has a equal chance of passing either of their chromosomes to their
children
\item the partner of each person in the study does not carry a defective X chromosome;
and
\item no new genetic defects occur
\end{itemize}

\begin{enumerate}[(a)]
\item (i)
Show that the expected progress of the disease through the generations may be
modelled as a Markov chain and specify carefully:
(a)
(b)
the state space; and
the transition diagram

\item (ii)
State, with reasons, whether the chain is:
\begin{enumerate}[(i)]
\item 
irreducible; and
\item aperiodic
\end{enumerate}

\item  (iii)
Calculate the stationary distribution of the Markov chain.
\end{enumerate}


%%%%%%%%%%%%%%%%%%%%%%%%%%%%%%%%%%%%%%%%%%%%%%%%%%%%%%%%%%%%%%%%%%%%%%%%%%%%%%%%%
\newpage
7
(i)
Consider the sequence of the status of the first born child in each generation.
The state space consists of the four possible combinations of chromosomes:
Female non-carrier (FN)
Female carrier
(FC)
Male non-sufferer (MN)
Male haemophiliac (MH)
or XX
or X*X
or XY
or X*Y
Using the assumption that there is an equal chance of either chromosome
being inherited:
\begin{itemize}
\item A female non-carrier will lead to a female non-carrier or male non-carrier.
\item A female carrier may produce:
X*X, XX, X*Y, XY all with equal probability.
\item A male non-sufferer will lead to female non-carrier or male non-carrier.
\item A male haemophiliac may produce:
X*X or XY (because his partner must provide an X) with equal
probability.
\end{itemize}
 %%%%%%%%%%%%%%%%%%%%%%%%%%%%%%%%%%%%5 — April 2007 — Examiners’ Report
The transition diagram is therefore:
0.5
0.25
0.25
FN
0.5
FC
0.5
0.25
0.25
0.5
MN
MH
0.5
0.5



Each of the transition probabilities depends only on state currently occupied,
so the process possesses the Markov property.
(ii)
(iii)
(a) The chain is reducible because once it enters states FN or MN it cannot
access FC or MH.
(b) The chain is aperiodic.
As it is reducible we need to consider each group of states. FN/MN
clearly have no period, and MH/FC do not either because a loop is
possible in state FC.
The transition matrix is
FN FC MN MH
0
0.5
0
FN (0) 0.5
A = FC (1) 0.25 0.25 0.25 0.25
0
0.5
0
MN (2) 0.5
0
0.5 0.5
0
MH (3)
$$\bordermatrix{ &c_1&c_2&\ldots &c_n\cr
                r_1&a_{11} &  0  & \ldots & a_{1n}\cr
                r_2& 0  &  a_{22} & \ldots & a_{2n}\cr
                r_4& 0  &   0       &\ldots & a_{nn}}$$
9 — %%%%%%%%%%%%%%%%%%%%%%%%%%%%%%%%%%%%5 — April 2007 — Examiners’ Report
The stationary distribution $\pi$ must satisfy:
\begin{itemize}
    \item ${\displaystyle \pi 0 = 0.5 \pi 0 + 0.25 \pi_1 + 0.5 \pi_2 }$
    \item ${\displaystyle \pi_1 = 0.25 \pi_1 + 0.5 \pi_3}$
    \item ${\displaystyle \pi_2 = 0.5 \pi 0 + 0.25 \pi_1 + 0.5 \pi_2 + 0.5 \pi_3}$
    \item ${\displaystyle \pi_3 = 0.25 \pi_1}$
\end{itemize}

So,
${\displaystyle \pi_1 = 0.25 \pi_1 + 0.5 \times0.25 \pi_1}$
⇒ 
${\displaystyle \pi_1 = \pi_3 = 0}$
⇒ 
${\displaystyle \pi 0 = \pi_2 = 0.5}$
An alternative solution combines the states FN and MN to give a 3-state model. This
was given credit.
\end{document}
