\documentclass[a4paper,12pt]{article}

%%%%%%%%%%%%%%%%%%%%%%%%%%%%%%%%%%%%%%%%%%%%%%%%%%%%%%%%%%%%%%%%%%%%%%%%%%%%%%%%%%%%%%%%%%%%%%%%%%%%%%%%%%%%%%%%%%%%%%%%%%%%%%%%%%%%%%%%%%%%%%%%%%%%%%%%%%%%%%%%%%%%%%%%%%%%%%%%%%%%%%%%%%%%%%%%%%%%%%%%%%%%%%%%%%%%%%%%%%%%%%%%%%%%%%%%%%%%%%%%%%%%%%%%%%%%

\usepackage{eurosym}
\usepackage{vmargin}
\usepackage{amsmath}
\usepackage{graphics}
\usepackage{epsfig}
\usepackage{enumerate}
\usepackage{multicol}
\usepackage{subfigure}
\usepackage{fancyhdr}
\usepackage{listings}
\usepackage{framed}
\usepackage{graphicx}
\usepackage{amsmath}
\usepackage{chngpage}

%\usepackage{bigints}
\usepackage{vmargin}

% left top textwidth textheight headheight

% headsep footheight footskip

\setmargins{2.0cm}{2.5cm}{16 cm}{22cm}{0.5cm}{0cm}{1cm}{1cm}

\renewcommand{\baselinestretch}{1.3}

\setcounter{MaxMatrixCols}{10}

\begin{document}
\begin{enumerate}
1
(a)
Define, in the context of stochastic processes, a:
1. mixed process
2. counting process
(b)
Give an example application of each type of process.
%%%%%%%%%%%%%%%%%%%%%%%%%%%%%%%%%%%%%%%%%%%%%%%%%%%%%%%%%%%%%%%%%%%%%%%%%%%%%%%%%%%%%

%%%%%%%%%%%%%%%%%%%%%%%%%%%%%%%%%%%%%%%%%%%%%%%%%%%%%%%%%%%%%%%%%%%%%%%%%%%%
1
Mixed process
(a) Is a stochastic process that operates in continuous time, which can also change
value at predetermined discrete instants.
(b) The number of contributors to a pension scheme can be modelled as a mixed
process with state space $S = \{ 1, 2,3,... \}$ and time interval $J = [ 0, \infty ]$ .
Counting process
(a)
Is a process, X, in discrete or continuous time, whose state space is the natural
numbers {0, 1, 2, ...}.
X(t) is a non-decreasing function of t.
2
(b) Number of claims reported to an insurer by time t.
(i) (a)
Graduation by reference to a standard table would be appropriate.
There are likely to be existing standard tables which are suitable and this method is suitable for relatively small data sets.
Alternatively, graduation by parametric formula would be suitable if the volume of data was large enough. But that is unlikely to be the
case here.
\end{document}
