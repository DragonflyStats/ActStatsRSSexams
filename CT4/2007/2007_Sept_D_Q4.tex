%%-- CT4 S2007—2
4
A national mortality investigation was carried out. It was suggested that the mortality of the male population could be represented by the following graduated rates:
o
\mu x + 1 =\mu s x + 2 1
2
2
where \mu sx is from the standard tables, ELT15(males).
The table below shows the graduated rates for part of the age range, together with the exposed to risk, expected and actual deaths at each age. The squared standardised
deviations that were calculated are also shown.
⎛
⎞
c o
⎜ θ x − E x ⋅\mu x + 12 ⎟
⎠
The standardised deviations were calculated as z x = ⎝
o
E x c ⋅\mu x + 1
2
Age Graduated
rates
Exposed
to risk Expected
deaths Deaths Squared
standardised
deviations
x \mu x+ 1 E x c E x c ⋅\mu x + 1 o θ x z x 2
0.00549
0.00610
0.00679
0.00757
0.00845
0.00945
0.01057
0.01182
0.01323
0.01483 10,850
9,812
10,054
9,650
8,563
10,656
9,667
9,560
8,968
8,455 59.57
59.85
68.27
73.05
72.36
100.70
102.18
113.00
118.65
125.39 52
54
60
65
64
87
88
97
103
105 0.9611
0.5724
1.0010
0.8872
0.9653
1.8637
1.9679
2.2653
2.0634
3.3150
o
2
50
51
52
53
54
55
56
57
58
59
2
(i) Test this graduation for overall goodness-of-fit.
(ii) Comment on your findings in (i).
CT4 S2007—3


%%%%%%%%%%%%%%%%%%%%%%%%%%%%%%%%%%%%%%%%%%%%%%%%%%%%%%%%%%%%%%%55
4
(i)
The null hypothesis is that graduated rates are the same as the true underlying
rates in the population.
To test overall goodness-of-fit we use the chi-squared test.
∑ z x 2 ∼ χ 2 m , where m is the number of degrees of freedom.
x
In this case, we have 10 ages.
The graduation was carried out by reference to a standard table, so we lose a number of degrees of freedom because of the choice of standard table.
So, m < 10, and let us say m = 8.
The observed value of the test statistic is
∑ z x 2 = 15.8623
x
The critical value of the chi-squared distribution with 8 degrees of freedom at the 5 per cent level is 15.51.
Since 15.8623 > 15.51,
we reject the null hypothesis and conclude that the graduated rates do not
adhere to the data.
[Credit was given for using other values of m, say m = 7 or m = 9, provided candidates recognized that some degrees of freedom should be lost for the choice of standard table. Note that if m = 9, the null hypothesis will not be rejected.]
(ii)
From the data we can see that the actual deaths are lower than those expected at all ages.
The graduated rates are too high; the graduation should be revisited.
At these ages the force of mortality increases with age, so a suitable adjustment may be to reduce the age shift relative to the
standard table from 2 years.
The standardised deviations also appear to show a systematic increase with age, showing that departure of the graduated rates from the actual rates increases with age.
There appear to be no outliers (all the z x s have absolute values below 1.96).
\end{document}
%%-- 5 — %%%%%%%%%%%%%%%%%%%%%%%%%%%%%%%%%%%%5 — September 2007 — Examiners’ Report
