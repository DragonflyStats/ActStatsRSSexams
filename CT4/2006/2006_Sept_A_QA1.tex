\documentclass[a4paper,12pt]{article}

%%%%%%%%%%%%%%%%%%%%%%%%%%%%%%%%%%%%%%%%%%%%%%%%%%%%%%%%%%%%%%%%%%%%%%%%%%%%%%%%%%%%%%%%%%%%%%%%%%%%%%%%%%%%%%%%%%%%%%%%%%%%%%%%%%%%%%%%%%%%%%%%%%%%%%%%%%%%%%%%%%%%%%%%%%%%%%%%%%%%%%%%%%%%%%%%%%%%%%%%%%%%%%%%%%%%%%%%%%%%%%%%%%%%%%%%%%%%%%%%%%%%%%%%%%%%

%%- A1
\usepackage{eurosym}
\usepackage{vmargin}
\usepackage{amsmath}
\usepackage{graphics}
\usepackage{epsfig}
\usepackage{enumerate}
\usepackage{multicol}
\usepackage{subfigure}
\usepackage{fancyhdr}
\usepackage{listings}
\usepackage{framed}
\usepackage{graphicx}
\usepackage{amsmath}
\usepackage{chngpage}

%\usepackage{bigints}
\usepackage{vmargin}

% left top textwidth textheight headheight

% headsep footheight footskip

\setmargins{2.0cm}{2.5cm}{16 cm}{22cm}{0.5cm}{0cm}{1cm}{1cm}

\renewcommand{\baselinestretch}{1.3}

\setcounter{MaxMatrixCols}{10}

\begin{document}

%%%%%%%%%%%%%%%%%%%%%%%%%%%%%%%%%%%%%%%%%%%%%%%%%%%%%%%%%%%%%%%%%%%
A1
A manufacturer uses a test rig to estimate the failure rate in a batch of electronic
components. The rig holds 100 components and is designed to detect when a
component fails, at which point it immediately replaces the component with another
from the same batch. The following are recorded for each of the n components used
in the test (i = 1,2, ,n):
s i = time at which component i placed on the rig
t i = time at which component i removed from rig
f i
1 Component removed due to failure
0 Component working at end of test period
The test rig was fully loaded and was run for two years continuously.
You should assume that the force of failure, , of a component is constant and
component failures are independent.
(i)
Show that the contribution to the likelihood from component i is:
exp
(ii)
A2
t i s i
f i
Derive the maximum likelihood estimator for .




%%%%%%%%%%%%%%%%%%%%%%%%%%%%%%%%%%%%%
A1
(i)
If the ith component is still working at the end of the test period its
contribution to the likelihood is:
t_{i} \;-\; s_{i} 
p s i = exp( -\mu  ( t_{i} \;-\; s_{i} ))
under the assumption of a constant force of failure.
If the ith component fails at time t i its contribution to the likelihood is:
t_{i} \;-\; s_{i} 
p s i . \mu  t i = exp( -\mu  ( t_{i} \;-\; s_{i} )). \mu 
under the assumption of a constant force of failure.
In both cases the contribution equals:
exp( -\mu  ( t_{i} \;-\; s_{i} )). \mu  f i
(ii)
Denote the total number of components used in the test by n. The likelihood
for n independent components is:
n
L = \product exp( -\mu  ( t_{i} \;-\; s_{i} )). \mu  f i
i = 1
n
n
f i
L = exp( -\mu  \sum  ( t_{i} \;-\; s_{i}  )). \mu  i \sum 
= 1
i = 1
Now the rig contains 100 components at all times because it is fully loaded
n
and failed components are immediately replaced, so
\sum  ( t_{i} \;-\; s_{i}  ) = 200(years) .
i = 1
n
So
L = exp ( - 200 \mu  ) ⋅\mu 
\sum  f i
i = 1
n
ln L = - 200 \mu  + ln \mu  . \sum  f i
i = 1
n
\sum  f i
\frac{\partial}{\partial} ln L
= - 200 + i = 1
\frac{\partial}{\partial}\mu 
\mu 

Setting this to zero the MLE is:
n
\mu  ˆ =
\sum  f i
i = 1
200
To verify this is a maximum we see that:
n
\frac{\partial}{\partial} 2 ln L
\frac{\partial}{\partial}\mu 


%%%%%%%%%%%%%%%%%%%%%%%%%%%%%%%%%%%
\end{document}



\end{document}
