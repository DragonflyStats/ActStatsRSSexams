\documentclass[a4paper,12pt]{article}

%%%%%%%%%%%%%%%%%%%%%%%%%%%%%%%%%%%%%%%%%%%%%%%%%%%%%%%%%%%%%%%%%%%%%%%%%%%%%%%%%%%%%%%%%%%%%%%%%%%%%%%%%%%%%%%%%%%%%%%%%%%%%%%%%%%%%%%%%%%%%%%%%%%%%%%%%%%%%%%%%%%%%%%%%%%%%%%%%%%%%%%%%%%%%%%%%%%%%%%%%%%%%%%%%%%%%%%%%%%%%%%%%%%%%%%%%%%%%%%%%%%%%%%%%%%%

\usepackage{eurosym}
\usepackage{vmargin}
\usepackage{amsmath}
\usepackage{graphics}
\usepackage{epsfig}
\usepackage{enumerate}
\usepackage{multicol}
\usepackage{subfigure}
\usepackage{fancyhdr}
\usepackage{listings}
\usepackage{framed}
\usepackage{graphicx}
\usepackage{amsmath}
\usepackage{chngpage}

%\usepackage{bigints}
\usepackage{vmargin}

% left top textwidth textheight headheight

% headsep footheight footskip

\setmargins{2.0cm}{2.5cm}{16 cm}{22cm}{0.5cm}{0cm}{1cm}{1cm}

\renewcommand{\baselinestretch}{1.3}

\setcounter{MaxMatrixCols}{10}

\begin{document}
\begin{enumerate}
A4
(i) Explain why the system with state space {0%, 25%, 40%, 50%} does not form
a Markov chain.

(ii) (a)
Show how a Markov chain can be constructed by the introduction of
additional states.
(b)
Write down the transition matrix for this expanded system, or draw its
transition diagram.
[4]
(iii) Comment on the appropriateness of the current No Claims Discount system.

[Total 8]
(i) List the benefits of modelling in actuarial work. 
(ii) Describe the difference between a stochastic and a deterministic model. 
(iii) Outline the factors you would consider in deciding whether to use a stochastic
or deterministic model to study a problem.

(iv) Explain how a deterministic model might be used to validate model outcomes
where a stochastic approach has been selected.

[Total 9]
%%%%%%%%%%%%%%%%%%%%%%%%%%%%%%%%%%%%%%%%%%%

 3
PLEASE TURN OVERA5
Employees of a company are given a performance appraisal each year. The appraisal
results in each employee s performance being rated as High (H), Medium (M) or Low
(L). From evidence using previous data it is believed that the performance rating of an
employee evolves as a Markov chain with transition matrix:
H
M
2
H 1
P
M
L
L
2
1 2
2
for some parameter
1
2
.
(i) Draw the transition graph of the chain. 
(ii) Determine the range of values for
transition matrix. 
for which the matrix P is a valid
(iii) Explain whether the chain is irreducible and/or aperiodic.
(iv) For = 0.2, calculate the proportion of employees who, in the long run, are in
state L.

(v) Given that = 0.2, calculate the probability that an employee s rating in the
third year, X 3 , is L:
(a)
(b)
(c)

in the case that the employee s rating in the first year, X 1 , is H
in the case X 1 = M
in the case X 1 = L

[Total 11]
%%%%%%%%%%%%%%%%%%%%%%%%%%%%%%%%%
A4
(i)
Models Core Technical
April 2006
Examiners Report
Systems with long time frames such as the operation of a pension fund can be
studied in compressed time.
Different future policies or possible actions can be compared to see which best
suits the requirements or constraints of a user.
Complex situations can be studied.
Modelling may be the only practicable approach for certain actuarial
problems.
(ii)
A model is described as stochastic if it allows for the random variation in at
least one input variable.
Often the output from a stochastic model is in the form of many simulated
possible outcomes of a process, so distributions can be studied.
A deterministic model can be thought of as a special case of a stochastic
model where only a single outcome from the underlying random processes is
considered.
Sometimes stochastic models have analytical/closed form solutions, such that
simulation is not required, but they are still stochastic as they allow for factors
to be random variables.
(iii)
If the distribution of possible outcomes is required then stochastic
modelling would be needed, or if only interested in a single scenario then
deterministic.
Budget and time available
stochastic modelling can be considerably
more expensive and time consuming.
Nature of existing models.
Audience for the results and the way they will be communicated.
The following factors may favour a stochastic approach:
The regulator may require a stochastic approach.
Extent of non-linear variation
for example existence of options or
guarantees.
Skewness of distribution of underlying variables, such as cost of storm
claims.
Interaction between variables, such as lapse rates with investment
performance.
Page 6Subject CT4
Models Core Technical
April 2006
Examiners Report
The following may favour a deterministic approach:
Lack of credible historic data on which to fit distribution of a variable.
If accuracy of result is not paramount, for example if a simple model with
deliberately cautious assumptions is chosen so as not to underestimate
costs.
(iv)
A deterministic result on best estimate assumptions could be compared with
the mean and median outcomes from a stochastic approach.
A deterministic model may also be used to calculate the expected or median
outcome, with a stochastic approach being used to estimate the volatility
around the central outcome.
A5
(i)
Transition graph given below.
2
1
1 2
State H
State M
2
1
2
2
State L
(ii)
Transition probabilities must lie in [0,1]. Thus we need
and 1
(iii)
2
0, 1 - 2
0
0.
The solution of the quadratic is the interval 1
2
5 1
,
2
2
conditions are satisfied simultaneously for 1
[0, ].
2
5
, so all
2
The chain is both irreducible, as every state can be reached from every other
state, and aperiodic, as the chain may remain at its current state for all H, M,
L.
Page 7Subject CT4
(iv)
Models Core Technical
April 2006
Examiners Report
From the result in (iii), a stationary probability distribution exists and it is
unique. Let = ( H , M , L ) denote the stationary distribution. Then, can be
determined by solving P = .
For
= 0.2, the transition matrix becomes
P
0.76 0.2 0.04
0.2
0.04 0.6
0.2 0.2
0.76
So that the system P =
0.76
0.2
0.04
H
H
H
+ 0.2
+ 0.6
+ 0.2
reads
M
M
M
+0.04
+0.2
+0.76
L
L
L
=
=
=
(1)
H
M
(2)
L
Discard the second of these equations and use also that the stationary
probabilities must also satisfy
H
+
M
+
L
= 1
(3)
Subtracting (2) from (1) gives
H
=
L .
Substituting into (1) we obtain H = M , thus (3) gives that H = M =
The proportion of employees who are in state L in the long run is 1/3.
(v)
L
The second order transition matrix is
P
2
0.76 0.2 0.04 0.76 0.2 0.04
0.2
0.04 0.6
0.2 0.2
0.76 0.2
0.04 0.6
0.2 0.2
0.76
0.6192 0.28
0.28
0.44
0.1008 0.28
0.1008
0.28
0.6192
The relevant entries are those in the last column, so that the answers are:
(a)
(b)
(c)
Page 8
=1/3.
0.1008
0.28
0.6192.Subject CT4
