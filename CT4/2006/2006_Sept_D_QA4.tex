\documentclass[a4paper,12pt]{article}

%%%%%%%%%%%%%%%%%%%%%%%%%%%%%%%%%%%%%%%%%%%%%%%%%%%%%%%%%%%%%%%%%%%%%%%%%%%%%%%%%%%%%%%%%%%%%%%%%%%%%%%%%%%%%%%%%%%%%%%%%%%%%%%%%%%%%%%%%%%%%%%%%%%%%%%%%%%%%%%%%%%%%%%%%%%%%%%%%%%%%%%%%%%%%%%%%%%%%%%%%%%%%%%%%%%%%%%%%%%%%%%%%%%%%%%%%%%%%%%%%%%%%%%%%%%%

%%- A4
\usepackage{eurosym}
\usepackage{vmargin}
\usepackage{amsmath}
\usepackage{graphics}
\usepackage{epsfig}
\usepackage{enumerate}
\usepackage{multicol}
\usepackage{subfigure}
\usepackage{fancyhdr}
\usepackage{listings}
\usepackage{framed}
\usepackage{graphicx}
\usepackage{amsmath}
\usepackage{chngpage}

%\usepackage{bigints}
\usepackage{vmargin}

% left top textwidth textheight headheight

% headsep footheight footskip

\setmargins{2.0cm}{2.5cm}{16 cm}{22cm}{0.5cm}{0cm}{1cm}{1cm}

\renewcommand{\baselinestretch}{1.3}

\setcounter{MaxMatrixCols}{10}

\begin{document}

%%%%%%%%%%%%%%%%%%%%%%%%%%%%%%%%%%%%%%%%%%%%%%%%%%%%%%%%%%%%%%%%%%%
The credit-worthiness of debt issued by companies is assessed at the end of each year
by a credit rating agency. The ratings are A (the most credit-worthy), B and D (debt
defaulted). Historic evidence supports the view that the credit rating of a debt can be
modelled as a Markov chain with one-year transition matrix
0.92 0.05 0.03
0.05 0.85
0
0
0.1
1
\begn{enumerate}
\item (i) Determine the probability that a company rated A will never be rated B in the
future.
\item 
(ii) (a) Calculate the second order transition probabilities of the Markov chain.
(b) Hence calculate the expected number of defaults within the next two
years from a group of 100 companies, all initially rated A.

The manager of a portfolio investing in company debt follows a downgrade trigger
strategy. Under this strategy, any debt in a company whose rating has fallen to B at
the end of a year is sold and replaced with debt in an A-rated company.
\item (iii) Calculate the expected number of defaults for this investment manager over
the next two years, given that the portfolio initially consists of 100 A-rated
bonds.
\item 
(iv) Comment on the suggestion that the downgrade trigger strategy will improve
the return on the portfolio.
\end{enumerate}

%%%%%%%%%%%%%%%%%%%%%%%%%%%%%%%%%%%%%%%%%%%%%%%%%%%%%%%%%%%%%%%%%%%%%%%%%%%%%%%%%%%%


A4
(i)
Probability that a company is never in state B is:
Pr( A \rightarrow D ) + Pr( A \rightarrow A \rightarrow D ) + Pr( A \rightarrow A \rightarrow A \rightarrow D ) + ......
= 0.03+ 0.92 \times 0.03+ 0.92 2 \times 0.03+......
∞
= 0.03 \times \sum  0.92 i =
i = 0
(ii)
(a)
0.03
= 0.375
1 - 0.92
⎛ 0.92 0.05 0.03 ⎞ ⎛ 0.92 0.05 0.03 ⎞
⎜
⎟⎜
⎟
A = ⎜ 0.05 0.85 0.1 ⎟ ⎜ 0.05 0.85 0.1 ⎟
⎜ 0
0
1 ⎟ ⎠ ⎜ ⎝ 0
0
1 ⎟ ⎠
⎝
2
⎛ 0.8489 0.0885 0.0626 ⎞
⎜
⎟
= ⎜ 0.0885 0.725 0.1865 ⎟
⎜ 0
0
1 ⎟ ⎠
⎝
(b)
Probability of default within 2 years for an A rated company 6.26%, so
6.26 defaults expected.
Page 7
(iii)
Either
Calculate revised transition probabilities based on the rating of bonds held by
the investment manager after rebalancing:
⎛ 0.97 0 0.03 ⎞
⎜
⎟
A ′ = ⎜ 0
0
0 ⎟
⎜ 0
0
1 ⎟ ⎠
⎝
(state B is unnecessary so this can be shown as 2 \times 2 or 3 \times 3)
⎛ 0.9409 0 0.0591 ⎞
⎜
⎟
A ′ = ⎜ 0
0
0 ⎟
⎜ 0
0
1 ⎟ ⎠
⎝
2
So the expected number of defaults is 0.0591 \times 100 = 5.91.
Or
Required probability is
Pr( A \rightarrow D ) + Pr( A \rightarrow A ) \times Pr( A \rightarrow D ) + Pr( A \rightarrow B ) \times Pr( A \rightarrow D )
= 0.03 + 0.92 \times 0.03 + 0.05 \times 0.03 = 0.0591
So expected defaults 5.91.
(iv)
The expected number of defaults has been reduced by this strategy. (The
variance of the number of defaults would also reduce.)
However it is not possible to tell whether the overall return is improved as this
depends on the price at which bonds were bought and sold at the end of year 1.
The price of the debt sold may have been depressed by the companies having
been downgraded to rating B, and the manager loses out on any increase in
price if they recover.
The “downgrade trigger” strategy will incur dealing costs, which should be
considered when comparing the returns.


\end{document}
