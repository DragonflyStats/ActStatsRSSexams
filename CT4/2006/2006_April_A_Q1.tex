\documentclass[a4paper,12pt]{article}

%%%%%%%%%%%%%%%%%%%%%%%%%%%%%%%%%%%%%%%%%%%%%%%%%%%%%%%%%%%%%%%%%%%%%%%%%%%%%%%%%%%%%%%%%%%%%%%%%%%%%%%%%%%%%%%%%%%%%%%%%%%%%%%%%%%%%%%%%%%%%%%%%%%%%%%%%%%%%%%%%%%%%%%%%%%%%%%%%%%%%%%%%%%%%%%%%%%%%%%%%%%%%%%%%%%%%%%%%%%%%%%%%%%%%%%%%%%%%%%%%%%%%%%%%%%%

\usepackage{eurosym}
\usepackage{vmargin}
\usepackage{amsmath}
\usepackage{graphics}
\usepackage{epsfig}
\usepackage{enumerate}
\usepackage{multicol}
\usepackage{subfigure}
\usepackage{fancyhdr}
\usepackage{listings}
\usepackage{framed}
\usepackage{graphicx}
\usepackage{amsmath}
\usepackage{chngpage}

%\usepackage{bigints}
\usepackage{vmargin}

% left top textwidth textheight headheight

% headsep footheight footskip

\setmargins{2.0cm}{2.5cm}{16 cm}{22cm}{0.5cm}{0cm}{1cm}{1cm}

\renewcommand{\baselinestretch}{1.3}

\setcounter{MaxMatrixCols}{10}

\begin{document}
\begin{enumerate}

Institute of Actuaries103 Questions
A1
In the context of a stochastic process {X t : t
following conditions:
(a)
(b)
J}, explain the meaning of the
strict stationarity
weak stationarity

A2
A savings provider offers a regular premium pension contract, under which the customer is able to cease paying in premiums and restart them at a later date. In order
to profit test the product, the provider set up the four-state Markov model shown in the following diagram:
Policy
matured
(D)
AD
t
Premium
paying
(A)
BD
t
Premiums
ceased/paid up
(B)
AB
t
BA
t
BC
t
AC
t
Policy
surrendered
(C)
Show, from first principles, that under this model:
t
%%%%%%%%%%%%%%%%%%%%%%%%%%%%%%%%%%%%%%%%%%%

 2
t
p 0 AB
t
p 0 AA .
AB
t
t
p 0 AB .(
BA
t
BC
t
BD
t )
A3
A motor insurer s No Claims Discount system uses the following levels of discount
{0%, 25%, 40%, 50%}. Following a claim free year a policyholder moves up one
discount level (or remains on 50% discount). If the policyholder makes one (or more)
claims in a year they move down one level (or remain at 0% discount).
The insurer estimates that the probability of making at least one claim in a year is 0.1
if the policyholder made no claims the previous year, and 0.25 if they made a claim
the previous year.
New policyholders should be ignored.

%%%%%%%%%%%%%%%%%%%%%%%%%%%%%%%%%%%%%%%%%%%%%%%%%%%%%%%%%%%%%%%%%%%%%%

103 Solutions
A1
(a)
For a process to be strictly stationary, the joint distribution of X t 1 , X t 2 ,..., X t n
and X t
t 1 , X t t 2 ,..., X t t n
are identical for all t , t 1 , t 2 ,..., t n in J and all integers
n.
This means that the statistical properties of the process remain unchanged over
time.
(b)
Because strict stationarity is difficult to test fully in real life, we also use the
less stringent condition of weak stationarity.
Weak stationarity requires that the mean of the process, E[X t ] = m(t), is
constant and the covariance, E[(X s - m(s)) (X t - m(t))], depends only on the
time difference t s.
A2
Condition on the state occupied at time t to consider the survival probability
t dt
p 0 AB (this requires the Markov property):
t dt
p 0 AB
Observe that
dt
t
p 0 AA . dt p t AB
p t CB
DB
dt p t
t
p 0 AB . dt p t BB
t
p 0 AC . dt p t CB
t
p 0 AD . dt p t DB
0
From the law of total probability:
dt
p t BB
Substituting for
t dt
p 0 AB
1
dt
p t BA
BC
dt p t
dt
p t BD
BB
dt p t
t
p 0 AA . dt p t AB
t
p 0 AB .(1
BA
dt p t
BC
dt p t
BD
dt p t )
For small dt:
dt p t BA
dt p t BC
dt p t BD
dt p t AB
BA
t . dt
BC
t . dt
BD
t . dt
AB
t . dt
o ( dt )
o ( dt )
o ( dt )
o ( dt )
Page 3Subject CT4
Models Core Technical
April 2006
Examiners Report
Where o(dt) covers the possibility of more than one transition in time dt and
lim o ( dt )
dt 0 dt
0
Substituting in:
AB
t dt p 0
t
A3
(i)
AB
t p 0
AA AB
t p 0 . t . dt
lim
dt
0
AB
t p 0 (1
AB
t dt p 0
AB
t p 0
dt
BA
t . dt
BC
t . dt
AA AB
t p 0 . t
BD
t . dt )
AB BA
t p 0 ( t
o ( dt )
BC
t
BD
t )
This is not a Markov chain because it does not possess the Markov property,
that is transition probabilities do not depend only on the current state.
Specifically, if you are in the 25% discount level, the transition probability to
state 0% is 0.25 if a claim was made last year and 0.1 if the previous year was
claim free.
(ii)
(a)
Split the 25% and 40% discount states to include whether the previous
year was claim free.
New state space:
0% discount
25%NC (no claim last year)
25%C (at least one claim last year)
40%NC (no claim last year)
40%C (at least one claim last year)
50%
Page 4Subject CT4
Models Core Technical
April 2006
Examiners Report
(b)
0.9
0.75
0%
25% 40%
NC NC
0.1
0.25
50%
0.75
0.75
0.1
0.25
0.9
25% 40%
C C
0.9
0.1
0.25
New state
(iii)
0%
25%C
25%NC
40%C
40%NC
50%
0%
0.25
0.25
0.1
0
0
0
25%C
0
0
0
0.25
0.1
0
25%NC
0.75
0
0
0
0
0
40%C
0
0
0
0
0
0.1
40%NC
0
0.75
0.9
0
0
0
50%
0
0
0
0.75
0.9
0.9
In theory, the insurer should just use 2 NCD states according to whether the
policyholder made a claim in the previous year. This is because the company
believes the claims frequency is the same for drivers who have not made a
claim for 1, 2, 3 years (i.e. it remains at 0.1 whether the driver has been
claims-free for 1 or 10 years).
However there may be other reasons for adopting this scale:
Marketing or competitive pressures.
It may discourage the policyholder from making small claims, or
encourage careful driving, to preserve their discount.
General comments:
The following, more general comments about the appropriateness of an NCD
model also received credit:
It is appropriate to award a no-claims discount because there is empirical
evidence that drivers who have made a recent claim are more likely to
make a further claim.
More factors should be taken into account (with a suitable example such
as how long the policyholder has been driving).
Page 5Subject CT4
