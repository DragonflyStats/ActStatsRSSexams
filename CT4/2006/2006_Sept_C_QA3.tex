\documentclass[a4paper,12pt]{article}

%%%%%%%%%%%%%%%%%%%%%%%%%%%%%%%%%%%%%%%%%%%%%%%%%%%%%%%%%%%%%%%%%%%%%%%%%%%%%%%%%%%%%%%%%%%%%%%%%%%%%%%%%%%%%%%%%%%%%%%%%%%%%%%%%%%%%%%%%%%%%%%%%%%%%%%%%%%%%%%%%%%%%%%%%%%%%%%%%%%%%%%%%%%%%%%%%%%%%%%%%%%%%%%%%%%%%%%%%%%%%%%%%%%%%%%%%%%%%%%%%%%%%%%%%%%%

%%- A3
\usepackage{eurosym}
\usepackage{vmargin}
\usepackage{amsmath}
\usepackage{graphics}
\usepackage{epsfig}
\usepackage{enumerate}
\usepackage{multicol}
\usepackage{subfigure}
\usepackage{fancyhdr}
\usepackage{listings}
\usepackage{framed}
\usepackage{graphicx}
\usepackage{amsmath}
\usepackage{chngpage}

%\usepackage{bigints}
\usepackage{vmargin}

% left top textwidth textheight headheight

% headsep footheight footskip

\setmargins{2.0cm}{2.5cm}{16 cm}{22cm}{0.5cm}{0cm}{1cm}{1cm}

\renewcommand{\baselinestretch}{1.3}

\setcounter{MaxMatrixCols}{10}

\begin{document}
CT4 S2006
2A3
(i)
Define the following types of a stochastic process:
(a)
(b)
(c)
a Poisson process
a compound Poisson process; and
a general random walk

(ii)
(iii)
A4
For each of the processes in (i), state whether it operates in continuous or
discrete time and whether it has a continuous or discrete state space.

For each of the processes in (i), describe one practical situation in which an
actuary could use such a process to model a real world phenomenon.

[Total 8]

%%%%%%%%%%%%%%%%%%%%%%%%%%%%%%%%%%%%%%%%%%%%%%%%%%%%%%%%%%%%%%%%%%%

A3
(i)
(a)
A Poisson process with rate \lambda  is an integer-valued process N t , t
\geq  0 with the following properties:
N 0 = 0;
N t has independent increments;
N t has stationary increments, each having a Poisson distribution, i.e.
P [ N t - N s = n ]
(b)
n
\lambda  ( t - s ) ] e -\lambda  ( t - s )
[
=
,
n !
s < t , n = 0,1, 2,...
Let N t be a Poisson process, t \geq  0 and let Y 1 , Y 2 , ..., Y j , ..., be a
sequence of i.i.d. random variables. Then a compound Poisson process
is defined by
N t
X t = \sum  Y j ,
t \geq  0.
j = 1
(c)
Let Y 1 , Y 2 , ..., Y j , ..., be a sequence of independent and identically
distributed random variables and define
n
X n = \sum  Y j
j = 1
∞
with initial condition X 0 = 0. Then { X n } n = 0 constitutes a general
random walk.
(ii)
(a) A Poisson process operates in continuous time and has a discrete state space, the set of nonnegative integers.
(b) A compound Poisson process operates in continuous time.
It has a discrete or continuous state space depending on whether the variables Y j are discrete or continuous respectively.
(iii)
(c) A general random walk operates in discrete time. Again, this has a discrete or continuous state space according to whether the variables $Y_j$ have a discrete or continuous distribution.
(a) Examples of a Poisson process:
\item 
\item 
\item 
Page 6
claims arriving to an insurance company through time car accidents reported over time arrival of customers at a service point over time
(b) A standard example of a compound Poisson process used by actuaries is for modelling the total amount of claims to an insurance company over time.
(c) Examples of a general random walk:
\item 
\item 
modelling share prices daily
inflation index, measured on say a monthly basis
Other reasonable examples received credit.
\end{document}
