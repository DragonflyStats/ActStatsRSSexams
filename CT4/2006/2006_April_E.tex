\documentclass[a4paper,12pt]{article}

%%%%%%%%%%%%%%%%%%%%%%%%%%%%%%%%%%%%%%%%%%%%%%%%%%%%%%%%%%%%%%%%%%%%%%%%%%%%%%%%%%%%%%%%%%%%%%%%%%%%%%%%%%%%%%%%%%%%%%%%%%%%%%%%%%%%%%%%%%%%%%%%%%%%%%%%%%%%%%%%%%%%%%%%%%%%%%%%%%%%%%%%%%%%%%%%%%%%%%%%%%%%%%%%%%%%%%%%%%%%%%%%%%%%%%%%%%%%%%%%%%%%%%%%%%%%

\usepackage{eurosym}
\usepackage{vmargin}
\usepackage{amsmath}
\usepackage{graphics}
\usepackage{epsfig}
\usepackage{enumerate}
\usepackage{multicol}
\usepackage{subfigure}
\usepackage{fancyhdr}
\usepackage{listings}
\usepackage{framed}
\usepackage{graphicx}
\usepackage{amsmath}
\usepackage{chngpage}

%\usepackage{bigints}
\usepackage{vmargin}

% left top textwidth textheight headheight

% headsep footheight footskip

\setmargins{2.0cm}{2.5cm}{16 cm}{22cm}{0.5cm}{0cm}{1cm}{1cm}

\renewcommand{\baselinestretch}{1.3}

\setcounter{MaxMatrixCols}{10}

\begin{document}
\begin{enumerate}





PLEASE TURN OVERB4
A company is interested in estimating policy lapse rates by age. It conducts an
investigation into this, which lasts for the whole of the calendar year 2003. The
investigation collects the following data for a sample of policies which are funded by
annual premiums:
the age last birthday of the policyholder when the policy was taken out;
the number of premiums the policyholder paid before the policy lapsed.
In addition, the number of policies in-force on 1 January each year is available,
classified by age x last birthday and years t elapsed since 1 January 2003, ( P x , t * ) .
(i) State the rate interval in this investigation.
(ii) Derive an expression for the exposed-to-risk in terms of P x , t * , stating any
assumptions you make.
(iii)


Comment on the reasonableness or otherwise of the assumptions you made in
your answer to part (ii).

[Total 10]
%%%%%%%%%%%%%%%%%%%%%%%%%%%%%%%%%%%%%%%%%%%

 8B5
A life assurance company carried out an investigation of the mortality of male life
assurance policyholders. The investigation followed a group of 100 policyholders
from their 60 th birthday until their 65 th birthday, or until they died or cancelled their
policy (whichever event occurred first).
The ages at which policyholders died or cancelled their policies were as follows:
Died Cancelled policy
Age in
years and months Age in
years and months
60y 5m
61y 1m
62y 6m
63y 0m
63y 0m
63y 8m
64y 3m 60y 2m
60y 3m
60y 8m
61y 0m
61y 0m
61y 0m
61y 5m
62y 2m
62y 9m
63y 9m
64y 5m
(i) Explain which types of censoring are present in the investigation. 
(ii) Calculate the Nelson-Aalen estimate of the integrated hazard for these
policyholders. 
(iii) Sketch the estimated integrated hazard function. 
(iv) Estimate the probability that a policyholder will survive to age 65.
%%%%%%%%%%%%%%%%%%%%%%%%%%%%%%%%%%%%%%%%%%%

 9



%%%%%%%%%%%%%%%%%%%%%%%%%%%%%%%%%%%%%%%%%%%%%%%%%%%%%%%%%%%%%%%%%%%%%%%%%%%%%%%%%%%%%%%%%%%%%%%
\n




B4
(i) We have a policy-year rate interval.
(ii) The age classification of the lapsing data is age last birthday on the policy
anniversary prior to lapsing .
This can be calculated by adding the policyholder s age last birthday when the
policy was taken to out to the number of annual premiums paid minus 1
(assuming that the first premium was paid at policy inception).
Define P x , t as the number of policies in force aged x last birthday at the
preceding policy anniversary at time t. This corresponds with the lapsing
data.
Page 13Subject CT4
Models Core Technical
April 2006
Examiners Report
Then, if t is measured in years since 1 January 2003, a consistent exposed-to-
risk would be
1
E x c
P x , t dt ,
0
which, assuming that policy anniversaries are uniformly distributed across the
calendar year,
may be approximated as
E x c
1
[ P x ,0
2
P x ,1 ] .
But we do not observe P x,t directly. Instead we observe P x , t * the number of
policies in force at time t, classified by age last birthday at time t.
But the range of exact ages that could apply to a life aged x last birthday
the policy anniversary prior to lapsing is (x, x + 2).
on
Assuming that birthdays are uniformly distributed across the policy year, half
of these lives will be aged x last birthday and half will be aged x+ 1 last
birthday.
Hence,
P x , t
1
[ P x , t * P x
2
*
1, t
] .
Therefore, by substituting this into the approximation above, the appropriate
exposed-to-risk is
E x c
(iii)
1 1
[ P x ,0 * P x
2 2
*
1,0 ]
1
[ P x ,1 * P x
2
*
1,1 ]
.
Both assumptions might be unreasonable because:
policies might be taken out in large numbers just before the end of the tax
year,
policies might tend to be taken out just before birthdays,
under group schemes, many policy anniversaries might be identical.
Page 14Subject CT4
(i)
April 2006
Examiners Report
The following types of censoring will be present:
Right censoring because some policyholders cancel their policy before
the end of the period.
Type I censoring
because the investigation stops at a fixed time.
Random censoring
unknown time.
because some lives cancel their policy at an
Informative censoring
in better health.
(ii)
(a)
because those who cancel their policy tend to be
The calculations are as follows:
d j
t j
(years)
d j
n j d j c j n j 100 0 2 0 0
98 1 4 1/98 0.0102
93 1 2 1/93 0.0210
t 3 90 1 1 1/90 0.0321
3 t 3 812 88 2 0 2/88 0.0548
3 8 12 t 4 3 12 86 1 1 1/86 0.0664
4 312 t 84 1 1 1/84 0.0783
5
0 t
5
12
1 112
2 612
t
t
12
1 1 12
2 6 12
j
n j
(b)
B5
Models Core Technical
0.09
0.08
0.07
0.06
0.05
0.04
0.03
0.02
0.01
0
0
1
2
3
4
5
Duration since 60th birthday
Page 15Subject CT4
(iii)
Models Core Technical
April 2006
Examiners Report
Either
Using the results of the calculation in (ii), the survival function can be
estimated by S t exp
t .
And so, for t
S t
4 3 / 12 , we have
exp
0.0783
0.925
which is the probability of survival to 65.
Or
Using the Kaplan-Meier estimate of S t
1
t j t
we get, for t
S t
d j
n j
,
4 3 / 12 :
1
1
98
1
1
93
1
1
90
1
2
88
1
1
86
1
1
84
= 0.9243
