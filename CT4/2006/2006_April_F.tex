\documentclass[a4paper,12pt]{article}

%%%%%%%%%%%%%%%%%%%%%%%%%%%%%%%%%%%%%%%%%%%%%%%%%%%%%%%%%%%%%%%%%%%%%%%%%%%%%%%%%%%%%%%%%%%%%%%%%%%%%%%%%%%%%%%%%%%%%%%%%%%%%%%%%%%%%%%%%%%%%%%%%%%%%%%%%%%%%%%%%%%%%%%%%%%%%%%%%%%%%%%%%%%%%%%%%%%%%%%%%%%%%%%%%%%%%%%%%%%%%%%%%%%%%%%%%%%%%%%%%%%%%%%%%%%%

\usepackage{eurosym}
\usepackage{vmargin}
\usepackage{amsmath}
\usepackage{graphics}
\usepackage{epsfig}
\usepackage{enumerate}
\usepackage{multicol}
\usepackage{subfigure}
\usepackage{fancyhdr}
\usepackage{listings}
\usepackage{framed}
\usepackage{graphicx}
\usepackage{amsmath}
\usepackage{chngpage}

%\usepackage{bigints}
\usepackage{vmargin}

% left top textwidth textheight headheight

% headsep footheight footskip

\setmargins{2.0cm}{2.5cm}{16 cm}{22cm}{0.5cm}{0cm}{1cm}{1cm}

\renewcommand{\baselinestretch}{1.3}

\setcounter{MaxMatrixCols}{10}

\begin{document}
\begin{enumerate}

[Total 11]
PLEASE TURN OVERB6
An investigation was undertaken into the mortality of male term assurance
policyholders for a large life insurance company. The crude mortality rates were
graduated using a formula of the form:
q x
e
x
An extract of the results is shown below.
Age Exposure
(years)
x E x
Crude
mortality rate
Graduated
mortality rate
Standardised
deviation
E x q x
q x
q x
q x
z x
E x q x 1 q x
40
41
42
43
44
45
46
47
48
49
50
51
52
53
54
11,037
12,010
11,654
9,658
8,457
10,541
7,410
12,042
14,038
11,479
12,480
10,567
9,187
14,027
11,581
0.0029
0.00333
0.003
0.003
0.00319
0.00427
0.00472
0.00399
0.00406
0.00375
0.00409
0.00407
0.00512
0.00456
0.00466
0.00348
0.00358
0.00368
0.00379
0.00391
0.00402
0.00415
0.00428
0.00441
0.00455
0.00469
0.00485
0.00500
0.00517
0.00534
-1.035
-0.459
-1.212
-1.264
-1.061
0.406
0.763
-0.487
-0.626
-1.274
-0.981
-1.154
0.163
-1.007
-1.004
(i) Test the graduation for goodness of fit using the chi-squared test.
(ii) (a) By inspection of the data, suggest one aspect of the graduated rates
where adherence to data seems inadequate.
(b) Explain why this may not be detected by the chi-squared test.
(c) Carry out one other test that may detect this deficiency.


(iii)
Suggest how the graduation could be adjusted to correct the deficiency
identified.

[Total 12]
END OF PAPER
%%%%%%%%%%%%%%%%%%%%%%%%%%%%%%%%%%%%%%%%%%%

 10

%%%%%%%%%%%%%%%%%%%%%%%%%%%%%%%%%%%%%%%%%%%%%%%%%%%%%%%%%%%%%%%%%%%%%%%%%%%%%%%%%%%%%%5
B6
(i)
The null hypothesis is that the crude rates come from a population in which
true underlying rates are the graduated rates.
z x 2
The test statistic is X
x
Under the null hypothesis X has a 2 distribution with m degrees of freedom,
where m is the number of age groups less one for each parameter fitted. So in
2
this case m = 15 3 = 12, ie X
12
The observed value of X is 12.816.
The critical value of the
2
12
distribution at the 5% level is 21.03
This is greater than the observed value of X
and so we have insufficient evidence to reject the null hypothesis.
Page 16Subject CT4
(ii)
Models Core Technical
April 2006
Examiners Report
(a) The obvious problem with the graduation is one of overall bias. The
graduated rates are consistently too high, resulting in too many
negative deviations.
(b) This is not detected by the 2 test because the test statistic is the sum
of the squared deviations and so information on the sign and some
information on the size of the individual deviations is lost. The 2 test
would detect large bias, but in this case the graduated and crude rates
are close enough that the statistic is below the critical value.
(c) Signs test
Let P be the number of positive deviations.
Under the null hypothesis, P
Binomial 15, 0.5 .
We have 3 positive deviations. The probability of getting 3 or fewer
positive signs (if the null hypothesis is true) is:
1
2
15
1
2
15
0
15
1
15
2
15
3
15
1 15 105 455
= 0.0176
This is less than 0.025 (this is a two-tailed test)
and so we reject the null hypothesis.
Cumulative deviations test
E x q x
Our test statistic is
E x q x
x
E x q x 1 q x
x
Under the null hypothesis, this has Normal(0, 1) distribution.
Page 17Subject CT4
Models Core Technical
April 2006
Examiners Report
Using the data in the question, we have
Age
x
40
41
42
43
44
45
46
47
48
49
50
51
52
53
54
Total
E x q x q x E x q x 1 q x
-6.40146
-3.0025
-7.92472
-7.62982
-6.08904
2.63525
4.2237
-3.49218
-4.9133
-9.1832
-7.488
-8.24226
1.10244
-8.55647
-7.87508
-72.837 38.2751
42.84188
42.7289
36.46509
32.93758
42.20447
30.62388
51.31917
61.63457
51.99181
58.25669
51.00139
45.70533
72.14466
61.5123
719.643
E x q x
E x q x
x
E x q x 1 q x
72.837
719.643
2.715
x
This is a two-tailed test.
Since
2.715 1.96 , we reject the null hypothesis.
Comments:
Candidates also received credit for using the standardised deviations
test to show that there were too many deviations in the (-2, -1) range.
(iii)
The problem is that the graduated rates are too high. There doesn t appear to
be a problem with the overall shape.
So we should be able to adjust the parameters rather than change the
underlying equation.
Page 18Subject CT4
Models Core Technical
April 2006
Examiners Report
The problem persists across the whole age range, so the first adjustment to try
would be to decrease the value of .
END OF EXAMINERS REPORT
Page 19
