\documentclass[a4paper,12pt]{article}

%%%%%%%%%%%%%%%%%%%%%%%%%%%%%%%%%%%%%%%%%%%%%%%%%%%%%%%%%%%%%%%%%%%%%%%%%%%%%%%%%%%%%%%%%%%%%%%%%%%%%%%%%%%%%%%%%%%%%%%%%%%%%%%%%%%%%%%%%%%%%%%%%%%%%%%%%%%%%%%%%%%%%%%%%%%%%%%%%%%%%%%%%%%%%%%%%%%%%%%%%%%%%%%%%%%%%%%%%%%%%%%%%%%%%%%%%%%%%%%%%%%%%%%%%%%%

%%- A2
\usepackage{eurosym}
\usepackage{vmargin}
\usepackage{amsmath}
\usepackage{graphics}
\usepackage{epsfig}
\usepackage{enumerate}
\usepackage{multicol}
\usepackage{subfigure}
\usepackage{fancyhdr}
\usepackage{listings}
\usepackage{framed}
\usepackage{graphicx}
\usepackage{amsmath}
\usepackage{chngpage}

%\usepackage{bigints}
\usepackage{vmargin}

% left top textwidth textheight headheight

% headsep footheight footskip

\setmargins{2.0cm}{2.5cm}{16 cm}{22cm}{0.5cm}{0cm}{1cm}{1cm}

\renewcommand{\baselinestretch}{1.3}

\setcounter{MaxMatrixCols}{10}

\begin{document}
[Total 6]
The price of a stock can either take a value above a certain point (state A), or take a
value below that point (state B). Assume that the evolution of the stock price in time
can be modelled by a two-state Markov jump process with homogeneous transition
rates AB
, BA
.
The process starts in state A at t = 0 and time is measured in weeks.
(i) Write down the generator matrix of the Markov jump process.

(ii) State the distribution of the holding time in each of states A and B.

(iii) If
3, find the value of t such that the probability that no transition to state
B has occurred until time t is 0.2.

(iv) Assuming all the information about the price of the stock is available for a
time interval [0,T], explain how the model parameters and can be
estimated from the available data.

(v) State what you would test to determine whether the data support the
assumption of a two-state Markov jump process model for the stock price. 
[Total 7]
%%%%%%%%%%%%%%%%%%%%%%%%%%%%%%%%%%%%%%%%%%%%%%%%%%%%%%%%%%%%%%%%%%%
A2
(i)
2
=-
\sum  f i
i = 1
\mu  2
< 0
The generator matrix is
⎛ -\sigma 
A =⎜
⎝ \rho
\sigma  ⎞
- \rho ⎟ ⎠
(ii) The distribution is exponential in both cases; with parameter \sigma  in state A, \rho in
state B.
(iii) The probability that the process stays in A throughout [0, t] is
∞
\int  \sigma  e
-\sigma  s
ds = e -\sigma  t .
t
For \sigma  = 3, we get e - 3 t = 0.2
which gives t = - ln (0.2)/3 = 0.54 weeks.
(iv)
The time spent in state A before the next visit to B has mean 1/\sigma .
Therefore a reasonable estimate for \sigma  is the reciprocal of the mean length of
each visit:
\sigma ̂ = (Number of transitions from A to B) / (Total time spent in state A up until
the last transition from A to B).
[An alternative is to use the maximum likelihood estimator for \sigma , which is
(Number of transitions from A to B)/Total time spent in state A).]
Similarly we can estimate \hat{\rho}.
(v)
Testing whether the successive holding times are exponential variables and
independent would be best. Any procedure which does this test is acceptable.
Page 5
\end{document}
