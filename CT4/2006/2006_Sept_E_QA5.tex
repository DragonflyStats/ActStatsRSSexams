\documentclass[a4paper,12pt]{article}

%%%%%%%%%%%%%%%%%%%%%%%%%%%%%%%%%%%%%%%%%%%%%%%%%%%%%%%%%%%%%%%%%%%%%%%%%%%%%%%%%%%%%%%%%%%%%%%%%%%%%%%%%%%%%%%%%%%%%%%%%%%%%%%%%%%%%%%%%%%%%%%%%%%%%%%%%%%%%%%%%%%%%%%%%%%%%%%%%%%%%%%%%%%%%%%%%%%%%%%%%%%%%%%%%%%%%%%%%%%%%%%%%%%%%%%%%%%%%%%%%%%%%%%%%%%%

%% A-5

\usepackage{eurosym}
\usepackage{vmargin}
\usepackage{amsmath}
\usepackage{graphics}
\usepackage{epsfig}
\usepackage{enumerate}
\usepackage{multicol}
\usepackage{subfigure}
\usepackage{fancyhdr}
\usepackage{listings}
\usepackage{framed}
\usepackage{graphicx}
\usepackage{amsmath}
\usepackage{chngpage}

%\usepackage{bigints}
\usepackage{vmargin}

% left top textwidth textheight headheight

% headsep footheight footskip

\setmargins{2.0cm}{2.5cm}{16 cm}{22cm}{0.5cm}{0cm}{1cm}{1cm}

\renewcommand{\baselinestretch}{1.3}

\setcounter{MaxMatrixCols}{10}

\begin{document}
PLEASE TURN OVERA5
A motor insurance company wishes to estimate the proportion of policyholders who
make at least one claim within a year. From historical data, the company believes that
the probability a policyholder makes a claim in any given year depends on the number
of claims the policyholder made in the previous two years. In particular:
the probability that a policyholder who had claims in both previous years will
make a claim in the current year is 0.25
the probability that a policyholder who had claims in one of the previous two
years will make a claim in the current year is 0.15; and
the probability that a policyholder who had no claims in the previous two years
will make a claim in the current year is 0.1


\begin{enumerate}
(i) Construct this as a Markov chain model, identifying clearly the states of the
chain.

(ii) Write down the transition matrix of the chain.
(iii) Explain why this Markov chain will converge to a stationary distribution. 
(iv) Calculate the proportion of policyholders who, in the long run, make at least
one claim at a given year.
\end{enumerate}

%%%%%%%%%%%%%%%%%%%%%%%%%%%%%%%%%%%%%%%%%%%%%%%%%%%%%%%%%%%%%%%%%%%%%%%%%%%%%%%%%%%%%%%%%%%%%%%%%%%%


%%%%%%%%%%%%%%%%%%%%%%%%%%%%%%%%%%%%%%%%%%%%%%%%%%%%%%%%%%%%%%%%%%%
Page 8
A5
(i)
Consider the following four states that the policyholder might be at the end of
a year:
\item  the policyholder has made at least one claim both in the year just ended
and the previous one (state A)
\item  the policyholder has made no claims in the year just ended but s/he made
at least one claim during the previous year (state B)
\item  the policyholder has made at least one claim in the year just ended but not
in the previous one (state C)
\item  the policyholder has made no claim during either the year ended or the
previous one (state D)
If the year ended is year n, and X n denotes the current state of the policyholder,
then X n constitutes a Markov chain.
(ii)
The transition matrix is
⎛ 0.25
⎜
0
P = ⎜
⎜ 0.15
⎜
⎝ 0
(iii)
0.75
0
0.85
0
0
0.15
0
0.10
0 ⎞
⎟
0.85 ⎟
0 ⎟
⎟
0.90 ⎠
The chain has a finite number of states (A,B,C,D). In order to show that it has
a stationary distribution, it suffices to show that it is irreducible and aperiodic.
It is apparent from the transition matrix above that any state can be reached
from any other; hence the chain is irreducible.
The chain is also aperiodic since for states A, D the state can remain at the
same state after one step, while for states B, C the state may return to its
current state after 2 or 3 steps.
Hence the chain has a stationary distribution (which is unique).
Page 9
(iv)
The set of equations is given (in matrix from) by \pi P=\pi ,
where \pi  = (\pi  A , \pi  B , \pi  C , \pi  D ) denotes the stationary distribution.
Using the transition matrix from (ii) above we obtain the equations
0.25 \pi  A +
0.75 \pi  A +
0.15 \pi  B
0.85 \pi  B
+0.15 \pi  C
+0.85 \pi  C
+0.10 \pi  D
+0.90 \pi  D
= \pi  A
= \pi  B
= \pi  C
= \pi  D
(1)
(2)
(3)
Discard the last of these equations and use also that the stationary probabilities
must also satisfy
\pi  A + \pi  B + \pi  C + \pi  D = 1
(4)
Equation (1) gives
0.75 \pi  A = 0.15 \pi  C
Or
(5)
5 \pi  A = \pi  C
Substituting (5) into (2) yields immediately
\pi  B = \pi  C
and inserting this into (3) we get
\pi  D =
17
\pi  B .
2
In view of the above, we obtain now from (4) that
17 ⎞
10
⎛ 1
.
\pi  B ⎜ + 1 + 1 + ⎟ = 1 ⇒ \pi  B =
2 ⎠
107
⎝ 5
Hence the other probabilities are
\pi  A =
2
10
85
, \pi  C =
, \pi  D =
.
107
107
107
The proportion of policyholders who, in the long run, make at least one claim
in a given year is
\pi  A + \pi  B =
Page 10
12
.


\end{document}
