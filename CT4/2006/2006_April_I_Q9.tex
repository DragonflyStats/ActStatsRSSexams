\documentclass[a4paper,12pt]{article}

%%%%%%%%%%%%%%%%%%%%%%%%%%%%%%%%%%%%%%%%%%%%%%%%%%%%%%%%%%%%%%%%%%%%%%%%%%%%%%%%%%%%%%%%%%%%%%%%%%%%%%%%%%%%%%%%%%%%%%%%%%%%%%%%%%%%%%%%%%%%%%%%%%%%%%%%%%%%%%%%%%%%%%%%%%%%%%%%%%%%%%%%%%%%%%%%%%%%%%%%%%%%%%%%%%%%%%%%%%%%%%%%%%%%%%%%%%%%%%%%%%%%%%%%%%%%

\usepackage{eurosym}
\usepackage{vmargin}
\usepackage{amsmath}
\usepackage{graphics}
\usepackage{epsfig}
\usepackage{enumerate}
\usepackage{multicol}
\usepackage{subfigure}
\usepackage{fancyhdr}
\usepackage{listings}
\usepackage{framed}
\usepackage{graphicx}
\usepackage{amsmath}
\usepackage{chngpage}

%\usepackage{bigints}
\usepackage{vmargin}

% left top textwidth textheight headheight

% headsep footheight footskip

\setmargins{2.0cm}{2.5cm}{16 cm}{22cm}{0.5cm}{0cm}{1cm}{1cm}

\renewcommand{\baselinestretch}{1.3}

\setcounter{MaxMatrixCols}{10}

\begin{document}
\begin{enumerate}


%%%%%%%%%%%%%%%%%%%%%%%%%%%%%%%%%%%%%%%%%%%

 6104 Questions
B1
A Cox proportional hazards model was estimated to assess the effect on survival of a
person s sex and his or her self-esteem (measured on a three-point scale as low ,
medium or high ). The baseline category was males with low self-esteem.
Write down the equation of the model, using algebraic symbols to represent variables
and parameters and defining all the symbols that you use.
[4]
B2
(i)
(a) Explain why it is important to sub-divide data when carrying out
mortality investigations.
(b) Describe the problems that can arise with sub-dividing data.
[4]
B3
(ii) List four factors which are often used to sub-divide life assurance data.


%%%%%%%%%%%%%%%%%%%%%%%%%%%%%%%%%%%%%%%%%%%%%%%%%%%%%%%%%%%%%%%%%%%%%%%%%%%%

(i) Assume that the force of mortality between consecutive integer ages, y and
y + 1, is constant and takes the value μ y .
Let T x be the future lifetime after age x ( x
function of T x .
y ) and S x (t) be the survival
Show that:
y
(ii)
log[ S x ( y x )] log[ S x ( y 1 x )] .
[4]
An investigation was carried out into the mortality of male life office
policyholders. Each life was observed from his 50th birthday until the first of
three possible events occurred: his 55th birthday, his death, or the lapsing of
his policy. For those policyholders who died or allowed their policies to lapse,
the exact age at exit was recorded.
Using the result from part (i) or otherwise, describe how the data arising from
this investigation could be used to estimate:
(a) 50
(b) 5 q 50

[Total 7]
%%%%%%%%%%%%%%%%%%%%%%%%%%%%%%%%%%%%%%%%%%%

 7








%%%%%%%%%%%%%%%%%%%%%%%%%%%%%%%%%%%%%%%%%%%%%%%%%%%%%%%%%%%%%%%%%%%%%%%%%%%%%%%%%

B1
h ( t )
h 0 ( t ) exp[ 1 F
2 M
3 H ]
where
h ( t ) is the estimated hazard,
h 0 ( t ) is the baseline hazard,
F is a variable taking the value 1 if the life is female, and 0 otherwise,
M is a variable taking the value 1 if the life has medium self-
esteem and 0 otherwise,
H is a variable taking the value 1 if the life has high self-esteem and 0 otherwise,
and
1 , 2
B2
(i)
and
(a)
3 are
parameters to be estimated.
The models of mortality we use assume that we can observe a group of
lives with the same mortality characteristics. This is not possible in
practice.
However, data can be sub-divided according to certain characteristics
that we know to have a significant effect on mortality.
This will reduce the heterogeneity of each group, so that we can at
least observe groups with similar, but not the same, characteristics.
(b)
Sub-dividing data using many factors can result in the numbers in each
class being too low.
It is necessary to strike a balance between homogeneity of the group
and retaining a large enough group to make statistical analysis
possible.
Sufficient data may not be collected to allow sub-division.
This may be because marketing pressures mean proposal forms are
kept to a minimum.
Page 11Subject CT4
(ii)
Models Core Technical
April 2006
Examiners Report
The following are factors often used:
Sex
Age
Type of policy
Smoker/Non-smoker status
Level of underwriting
Duration in force
Sales channel
Policy size
Occupation (or social class) of policyholder
Known impairments
Geographical region
B3
(i)
Consider the year of age between y and y + 1. We know that
t
t p y
exp
y s ds
.
0
If t=1 and
y (a
y s
p y
exp
constant), evaluating the integral produces
.
y
Now, conditioning on survival to age x, survival to age y + 1 implies survival
from age x to age y and then survival for a further year:
y 1 x
p x p y . y
y 1 x p x
x
p x .
Thus
p y
y x
p x
,
which, since, in general t p x
p y
Page 12
S x ( y 1 x )
.
S x ( y x )
S x ( t ) , may be writtenSubject CT4

%%%%%%%%%%%%%%%%%%%%%%%%%%%%%%%%%%%%%%%%%%%%%%%%%%%%%%%%%%%%%%%%%%%%%%%%%%%%
Examiners Report
Therefore
exp(
S x ( y 1 x )
,
S x ( y x )
y )
so that
y
(ii)
(a)
S x ( y x )
S x ( y 1 x )
log
log[ S x ( y x )] log[ S x ( y 1 x )] .
Using the result from part (i) and putting x = 50, y = 50 gives
50
log
S 50 (0)
S 50 (1)
log[ S 50 (1)]
Since we have censored data, because of the possibility of policy lapse,
we should estimate S 50 (1) using the Kaplan-Meier or Nelson-Aalen
estimator and hence obtain an estimate of
(b)
5 q 50
50 .
= 1 - 5 p 50 ,
and, since
5 p 50
5 q 50
S 50 (5) ,
can be estimated directly as 1
S 50 (5),
where S 50 (5) is the Kaplan-Meier or Nelson-Aalen estimator of the
probability of a life aged 50 years surviving for a further 5 years.


\end{document}
