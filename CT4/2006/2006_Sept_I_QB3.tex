\documentclass[a4paper,12pt]{article}

%%%%%%%%%%%%%%%%%%%%%%%%%%%%%%%%%%%%%%%%%%%%%%%%%%%%%%%%%%%%%%%%%%%%%%%%%%%%%%%%%%%%%%%%%%%%%%%%%%%%%%%%%%%%%%%%%%%%%%%%%%%%%%%%%%%%%%%%%%%%%%%%%%%%%%%%%%%%%%%%%%%%%%%%%%%%%%%%%%%%%%%%%%%%%%%%%%%%%%%%%%%%%%%%%%%%%%%%%%%%%%%%%%%%%%%%%%%%%%%%%%%%%%%%%%%%

%%- B3


\usepackage{eurosym}
\usepackage{vmargin}
\usepackage{amsmath}
\usepackage{graphics}
\usepackage{epsfig}
\usepackage{enumerate}
\usepackage{multicol}
\usepackage{subfigure}
\usepackage{fancyhdr}
\usepackage{listings}
\usepackage{framed}
\usepackage{graphicx}
\usepackage{amsmath}
\usepackage{chngpage}

%\usepackage{bigints}
\usepackage{vmargin}

% left top textwidth textheight headheight

% headsep footheight footskip

\setmargins{2.0cm}{2.5cm}{16 cm}{22cm}{0.5cm}{0cm}{1cm}{1cm}

\renewcommand{\baselinestretch}{1.3}

\setcounter{MaxMatrixCols}{10}

\begin{document}
B3
An investigation was undertaken into the effect of a new treatment on the survival
times of cancer patients. Two groups of patients were identified. One group was
given the new treatment and the other an existing treatment.
The following model was considered:
h i t
where: h i t
h 0 t
h 0 t exp
T
z
is the hazard at time t, where t is the time since the start of treatment
is the baseline hazard at time t
z
is a vector of covariates such that:
z 1 = sex (a categorical variable with 0 = female, 1 = male)
z 2 = treatment (a categorical variable with 0 = existing treatment,
1 = new treatment)
and
CT4 S2006
6
is a vector of parameters,
1 , 2
.The results of the investigation showed that, if the model is correct:
B4
A the risk of death for a male patient is 1.02 times that of a female
patient; and
B the risk of death for a patient given the existing treatment is 1.05 times
that for a patient given the new treatment
(i) Estimate the value of the parameters
(ii) Estimate the ratio by which the risk of death for a male patient who has been
given the new treatment is greater or less than that for a female patient given
the existing treatment.

(iii) Determine, in terms of the baseline hazard only, the probability that a male
patient will die within 3 years of receiving the new treatment.

[Total 7]
1
and
2 .


%%%%%%%%%%%%%%%%%%%%%%%%%%%%%%%%%%%%%%%%%%%%%%%%%%%%%%%%%%%%%%%%%%%
Page 15
B3
(i)
The hazard for a female patient is:
h f ( t ) = h 0 ( t ) \times exp ( 0 + \beta 2 z 2 )
and the hazard for a male patient is:
h m ( t ) = h 0 ( t ) \times exp ( \beta 1 \times 1 + \beta 2 z 2 )
Using \hat{\beta} i to denote our estimate of \beta i , we know from A that, if the model is
correct,
h m ( t ) = 1.02 \times h f ( t ) , so that:
(
)
(
h 0 ( t ) \times exp \hat{\beta}_{1} + \hat{\beta}_{2} z 2 = 1.02 \times h 0 ( t ) \times exp \hat{\beta}_{2} z 2
)
⇒ exp( \hat{\beta}_{1} ) = 1.02
⇒\hat{\beta}_{1} = ln ( 1.02 ) = 0.0198
And similarly, from B, we know that:
(
)
(
h 0 ( t ) \times exp \hat{\beta}_{1} z 1 + 0 = 1.05 \times h 0 ( t ) \times exp \hat{\beta}_{1} z 1 + \hat{\beta}_{2} z 2
)
( )
⇒ 1 = 1.05 \times exp \hat{\beta}_{2}
(
)
⇒ \hat{\beta}_{2} = ln 1
= - 0.0488
1.05
(ii)
The hazard for a male patient who has been given the new treatment is:
\begin{eqnarray*}
h m , n ( t ) 
&=& h 0 ( t ) \times exp ( \beta 1 \times 1 + \beta 2 \times 1 )\\
&=& h 0 ( t ) \times exp ( 0.0198 - 0.0488 )\\
&=& h 0 ( t ) \times exp ( - 0.029 )\\
&=& 0.9714 \times h 0 ( t )\\
\end{eqnarray*}
The hazard for a female patient given the existing treatment is the baseline
hazard.
Page 16
Hence, the ratio of the hazard for a male patient who has been given the new
treatment to that for a female patient given the existing treatment is:
h m , n ( t )
h 0 ( t )
= 0.9714
ALTERNATIVELY
Candidates may recognise that the proportions given in A and B can be
combined to give:
h m , n ( t )
⎡ h ( t ) ⎤ ⎡ h ( t ) ⎤
1
= ⎢ m , x ⎥ \times ⎢ x , n ⎥ = 1.02 \times
= 0.9714
1.05
h f , e ( t ) ⎣ ⎢ h f , x ( t ) ⎦ ⎥ ⎣ ⎢ h x , e ( t ) ⎦ ⎥
(iii)
The probability of death is given by:
{
{ \int 
}
3
1 - S m , n ( 3 ) = 1 - exp - \int  h m , n ( s ) ds
0
= 1 - exp -
3
0
}
0.9714 \times h 0 ( s ) ds
3
⎧
⎫
= 1 - exp ⎨ 0.9714 \times ⎛ ⎜ - \int  h 0 ( s ) ds ⎞ ⎟ ⎬
⎝ 0
⎠ ⎭
⎩
⎛ - \int  0 3 h 0 ( s ) ds ⎞
= 1 - ⎜ e
⎟
⎝
⎠

\end{document}
