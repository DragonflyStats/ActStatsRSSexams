\documentclass[a4paper,12pt]{article}

%%%%%%%%%%%%%%%%%%%%%%%%%%%%%%%%%%%%%%%%%%%%%%%%%%%%%%%%%%%%%%%%%%%%%%%%%%%%%%%%%%%%%%%%%%%%%%%%%%%%%%%%%%%%%%%%%%%%%%%%%%%%%%%%%%%%%%%%%%%%%%%%%%%%%%%%%%%%%%%%%%%%%%%%%%%%%%%%%%%%%%%%%%%%%%%%%%%%%%%%%%%%%%%%%%%%%%%%%%%%%%%%%%%%%%%%%%%%%%%%%%%%%%%%%%%%

\usepackage{eurosym}
\usepackage{vmargin}
\usepackage{amsmath}
\usepackage{graphics}
\usepackage{epsfig}
\usepackage{enumerate}
\usepackage{multicol}
\usepackage{subfigure}
\usepackage{fancyhdr}
\usepackage{listings}
\usepackage{framed}
\usepackage{graphicx}
\usepackage{amsmath}
\usepackage{chngpage}

%\usepackage{bigints}
\usepackage{vmargin}

% left top textwidth textheight headheight

% headsep footheight footskip

\setmargins{2.0cm}{2.5cm}{16 cm}{22cm}{0.5cm}{0cm}{1cm}{1cm}

\renewcommand{\baselinestretch}{1.3}

\setcounter{MaxMatrixCols}{10}

\begin{document}
\begin{enumerate}
A4
(i) Explain why the system with state space {0%, 25%, 40%, 50%} does not form
a Markov chain.

(ii) (a)
Show how a Markov chain can be constructed by the introduction of
additional states.
(b)
Write down the transition matrix for this expanded system, or draw its
transition diagram.
[4]
(iii) Comment on the appropriateness of the current No Claims Discount system.

[Total 8]
(i) List the benefits of modelling in actuarial work. 
(ii) Describe the difference between a stochastic and a deterministic model. 
(iii) Outline the factors you would consider in deciding whether to use a stochastic
or deterministic model to study a problem.

(iv) Explain how a deterministic model might be used to validate model outcomes
where a stochastic approach has been selected.

[Total 9]
%%%%%%%%%%%%%%%%%%%%%%%%%%%%%%%%%%%%%%%%%%%

%%%%%%%%%%%%%%%%%%%%%%%%%%%%%%%%%
A4
(i)
Models Core Technical
April 2006
Examiners Report
Systems with long time frames such as the operation of a pension fund can be
studied in compressed time.
Different future policies or possible actions can be compared to see which best
suits the requirements or constraints of a user.
Complex situations can be studied.
Modelling may be the only practicable approach for certain actuarial
problems.
(ii)
A model is described as stochastic if it allows for the random variation in at
least one input variable.
Often the output from a stochastic model is in the form of many simulated
possible outcomes of a process, so distributions can be studied.
A deterministic model can be thought of as a special case of a stochastic model where only a single outcome from the underlying random processes is considered.
Sometimes stochastic models have analytical/closed form solutions, such that simulation is not required, but they are still stochastic as they allow for factors
to be random variables.
(iii)
If the distribution of possible outcomes is required then stochastic modelling would be needed, or if only interested in a single scenario then
deterministic.
Budget and time available
stochastic modelling can be considerably more expensive and time consuming.
Nature of existing models.
Audience for the results and the way they will be communicated.
The following factors may favour a stochastic approach:
The regulator may require a stochastic approach.
Extent of non-linear variation
for example existence of options or
guarantees.
Skewness of distribution of underlying variables, such as cost of storm claims.
Interaction between variables, such as lapse rates with investment performance.
Page 6Subject CT4
Models Core Technical
April 2006
Examiners Report
The following may favour a deterministic approach:
Lack of credible historic data on which to fit distribution of a variable.
If accuracy of result is not paramount, for example if a simple model with deliberately cautious assumptions is chosen so as not to underestimate costs.
(iv)
A deterministic result on best estimate assumptions could be compared with the mean and median outcomes from a stochastic approach.

A deterministic model may also be used to calculate the expected or median outcome, with a stochastic approach being used to estimate the volatility around the central outcome.
\end{document}
