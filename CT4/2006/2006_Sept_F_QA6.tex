\documentclass[a4paper,12pt]{article}

%%%%%%%%%%%%%%%%%%%%%%%%%%%%%%%%%%%%%%%%%%%%%%%%%%%%%%%%%%%%%%%%%%%%%%%%%%%%%%%%%%%%%%%%%%%%%%%%%%%%%%%%%%%%%%%%%%%%%%%%%%%%%%%%%%%%%%%%%%%%%%%%%%%%%%%%%%%%%%%%%%%%%%%%%%%%%%%%%%%%%%%%%%%%%%%%%%%%%%%%%%%%%%%%%%%%%%%%%%%%%%%%%%%%%%%%%%%%%%%%%%%%%%%%%%%%

%% A-6

\usepackage{eurosym}
\usepackage{vmargin}
\usepackage{amsmath}
\usepackage{graphics}
\usepackage{epsfig}
\usepackage{enumerate}
\usepackage{multicol}
\usepackage{subfigure}
\usepackage{fancyhdr}
\usepackage{listings}
\usepackage{framed}
\usepackage{graphicx}
\usepackage{amsmath}
\usepackage{chngpage}

%\usepackage{bigints}
\usepackage{vmargin}

% left top textwidth textheight headheight

% headsep footheight footskip

\setmargins{2.0cm}{2.5cm}{16 cm}{22cm}{0.5cm}{0cm}{1cm}{1cm}

\renewcommand{\baselinestretch}{1.3}

\setcounter{MaxMatrixCols}{10}

\begin{document}
A6
(i) Construct this as a Markov chain model, identifying clearly the states of the
chain.

(ii) Write down the transition matrix of the chain.
(iii) Explain why this Markov chain will converge to a stationary distribution. 
(iv) Calculate the proportion of policyholders who, in the long run, make at least
one claim at a given year.

[Total 9]
(i) Explain the difference between a time-homogeneous and a time-
inhomogeneous Poisson process.


An insurance company assumes that the arrival of motor insurance claims follows an
inhomogeneous Poisson process.
Data on claim arrival times are available for several consecutive years.
(ii)
(a) Describe the main steps in the verification of the company s
assumption.
(b) State one statistical test that can be used to test the validity of the
assumption.

(iii)
The company concludes that an inhomogeneous Poisson process with rate
t 3 cos 2 t is a suitable fit to the claim data (where t is measured in
years).
CT4 S2006
(a) Comment on the suitability of this transition rate for motor insurance
claims.
(b) Write down the Kolmogorov forward equations for P 0 j ( s , t ) .
4(c)
Verify that these equations are satisfied by:
P 0 j ( s , t )
( f ( s , t )) j .exp( f ( s , t ))
j !
for some f(s,t) which you should identify.
[Note that cos x dx
(d)
sin x .]
Comment on the form of the solution compared with the case where
is constant.
[8]
[Total 12]
CT4 S2006
5

%%%%%%%%%%%%%%%%%%%%%%%%%%%%%%%%%%%%%%%%%%%%%%%%%%%%%%%%%%%%%%%%%%%

\end{document}





107
A6
(i) The probability that an event occurs during the short time interval between t
and t + h is approximately equal to \lambda ( t ) h for small h where \lambda ( t ) is called the
rate of the process. For a time-inhomogeneous process, \lambda ( t ) depends on the
current time t ; for a time-homogeneous process it is independent of time.
(ii) (a)
Divide the time period into intervals of a suitable size, say one month.
Estimate the arrival rate separately for each time period.
See if the observed data match the pattern which would be expected if
the model were accurate and if the parameters had their values given
by their estimates.
If not, the model should be revised.
(b)
A goodness of fit test, such as the chi-squared test, should be carried
out for each time period chosen.
Tests for serial correlation [e.g. portmanteau test] should use the whole
data set at once.
(iii)
(a)
This implies that claims are seasonal with period 12 months, and that
claims in the peak (presumably winter) are double those at the low
point of the year.
This would be reasonable if in a climate where driving conditions are
worse in winter.
(b)
Kolmogorov forward equations:
\frac{\partial}{\partial}
P ( s , t ) = P ( s , t ). A ( t )
\frac{\partial}{\partial} t
t \geq  s
Where:
⎛ -\lambda  ( t ) \lambda  ( t )
⎜
-\lambda  ( t ) \lambda  ( t )
A ( t ) = ⎜
⎜
-\lambda  ( t )
⎜
⎝
(c)
⎞
⎟
⎟
⎟
⎟
⎠
Consider the case j > 0,
\frac{\partial}{\partial}
P 0 j ( s , t ) = \lambda  ( t ). P 0, j - 1 ( s , t ) - \lambda  ( t ). P 0 j ( s , t )
\frac{\partial}{\partial} t
(I)
with P 0 j ( s , s ) = 0
Page 11
If solution is of the form
P 0 j ( s , t ) =
( f ( s , t )) j .exp( - f ( s , t ))
j !
LHS of I
( j .( f ( s , t )) j - 1 - f ( s , t ) j ).
exp( - f ( s , t )) d
. f ( s , t )
j !
dt
RHS of I
\lambda  ( t ).
f ( s , t ) j - 1
f ( s , t ) j .exp( - f ( s , t ))
.exp( - f ( s , t )) - \lambda  ( t ).
j !
( j - 1)!
These are equal if
\frac{\partial}{\partial}
f ( s , t ) = \lambda  ( t )
\frac{\partial}{\partial} t
Now
t t
s s
\int  \lambda  ( v ) dv = \int  (3 + cos(2 \pi  v )) dv
t
1
⎡
⎤
= ⎢ 3 v +
sin(2 \pi  v ) ⎥
2 \pi 
⎣
⎦ s
= 3( t - s ) +
1
[ sin(2 \pi  t ) - sin(2 \pi  s ) ] ≡ f ( s , t )
2 \pi 
this satisfies the boundary condition.
Consider the case j = 0
\frac{\partial}{\partial}
P 00 ( s , t ) = -\lambda  ( t ). P 00 ( s , t )
\frac{\partial}{\partial} t
with boundary condition P 00 ( s , s ) = 1
Need to verify that P 00 ( s , t ) = exp( - f ( s , t )) satisfies II
Page 12
(II)
LHS of II
- exp( - f ( s , t )).
\frac{\partial}{\partial}
( f ( s , t )) = - P 00 ( s , t ). \lambda  ( t )
\frac{\partial}{\partial} t
and P 00 ( s , s ) = 1
(d)
Solution is of the same form, except that for the homogeneous case
f ( s,t ) = \lambda ( t - s ).
\end{document}
