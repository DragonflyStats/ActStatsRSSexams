\documentclass[a4paper,12pt]{article}

%%%%%%%%%%%%%%%%%%%%%%%%%%%%%%%%%%%%%%%%%%%%%%%%%%%%%%%%%%%%%%%%%%%%%%%%%%%%%%%%%%%%%%%%%%%%%%%%%%%%%%%%%%%%%%%%%%%%%%%%%%%%%%%%%%%%%%%%%%%%%%%%%%%%%%%%%%%%%%%%%%%%%%%%%%%%%%%%%%%%%%%%%%%%%%%%%%%%%%%%%%%%%%%%%%%%%%%%%%%%%%%%%%%%%%%%%%%%%%%%%%%%%%%%%%%%

\usepackage{eurosym}
\usepackage{vmargin}
\usepackage{amsmath}
\usepackage{graphics}
\usepackage{epsfig}
\usepackage{enumerate}
\usepackage{multicol}
\usepackage{subfigure}
\usepackage{fancyhdr}
\usepackage{listings}
\usepackage{framed}
\usepackage{graphicx}
\usepackage{amsmath}
\usepackage{chngpage}

%\usepackage{bigints}
\usepackage{vmargin}

% left top textwidth textheight headheight

% headsep footheight footskip

\setmargins{2.0cm}{2.5cm}{16 cm}{22cm}{0.5cm}{0cm}{1cm}{1cm}

\renewcommand{\baselinestretch}{1.3}

\setcounter{MaxMatrixCols}{10}

\begin{document}
\begin{enumerate}
%%--- Question 4

\item In a large portfolio 65\% of the policies have been in force for more than five years.
An investigation considers a random sample of 500 policies from the portfolio.
Calculate an approximate value for the probability that fewer than 300 of the policies
iN_the sample have been in force for more than five years.


%%% Question 5
\item In a random sample of 200 policies from a company s private motor business, there
are 68 female policyholders and 132 male policyholders.
Let the proportion of policyholders who are female iN_the corresponding population of
all policyholders be denoted .
Test the hypotheses
H 0 :
0.4 v H 1 :
< 0.4
stating clearly the approximate probability value of the observed statistic and your conclusion.
%
%CT3 S2006
%%-- Question 6
%%%%%%%%%%%%%%%%%%%%%%%%
\item It is assumed that claims arising on an industrial policy can be modelled as a Poisson process at a rate of 0.5 per year.

\begin{enumerate}[\item (i)] 
\item Determine the probability that no claims arise in a single year.
%
\item (ii) Determine the probability that, in tthree consecutive years, there is one or more claims in one of the years and no claims in each of the other two years.

\item (iii) Suppose a claim has just occurred. Determine the probability that more thantwo years will elapse before the next claim occurs.
\end{enumerate}
\end{enumerate}
%%%%%%%%%%%%%%%%%%%%%%%%%%%%%%%%%%%%%%%%%%%%%%%%%%%%%%%%%%%%%%%%%%%%%%%%%%%%%%%%%%%%%%%
\newpage

4
Let X be the number in force for more than five years
then X ~ binomial(500,0.65)
Using a normal approximation, $X ≈ N(325, 10.665^2 )$
P(X < 300) becomes P(X < 299.5) using continuity correction
 P ( Z <
299.5 − 325
) where Z ~ N(0,1)
10.665
= P ( Z < − 2.39) = 1 − 0.99158 = 0.0084
%Page 4Subject CT3 %%%%%%%%%%%%%%%%%%%%%%%%%%%%%%%%%%5 — September 2006 — Examiners’ Report

%%%%%%%%%%%%%%%%%%%%%%%%%%%%%%%%%%%%%%%%%%%%%%%%%%%%%%%%%%%%%%%%%%%%%%%%%%%%%%%%%%%%%%
5
Under $H_0$ : sample proportion P is approximately normally distributed with mean 0.4
and standard error (0.4×0.6/200) 1/2 = 0.03464
∴ P-value of observed proportion (68/200 = 0.34)
0.34 − 0.4 ⎞
⎛
= P ⎜ Z <
⎟ = P ( Z < − 1.732 ) = 0.042
0.03464 ⎠
⎝
We reject $H_0$ at the 5\% level of testing and conclude that the proportion of policyholders who are female is less than 0.4.
[OR This is actually better - working with the number of female policyholders (observed = 68), the P-value is
⎛
⎞
68.5 − 80
P ⎜ Z <
= − 1.660 ⎟ = 0.048
⎜
⎟
200(0.4)(0.6)
⎝
⎠
]
Note: We can word the conclusion: we reject $H_0$ at levels of testing dowN_to 4.2% (or
4.8%) and conclude ...
%%%%%%%%%%%%%%%%%%%%%%%%%%%%%%%%%%%%%%%%%%%%%%%%%%%%%%%%%%%%%%%%%
6
\begin{itemize}
\item (i) P(no claims) = P(X = 0) where X ~ Poisson(0.5)
= 0.60653 from tables [or evaluation]
\item (ii) Let Y = number of years with a claim
then Y ~ binomial(3,0.3935) [or just directly as below]
\[P(Y = 1) = 3(0.3935)(0.6065)^2 = 0.434\]
\item (iii)
Let T = time until next claim
thent ~ exp(0.5)
P(T > 2) = e –0.5(2) [or bY_integration]
= e –1 = 0.368
[OR: answer = {P(no claim)} 2 = 0.60653 2 = 0.368]
[OR: claim rate for period of 2 years = 1, so P(no claim in 2 years)
= e -1 = 0.368]
\end{itemize}
%%%%%%%%%%%%%%%%%%%%%%%%%%%%%%%%%%%%%%%%%%%%%%%%%%%%%%%%%%%%%%%%%%%%%%%%%%%%%%%%
\end{document}
