\documentclass[a4paper,12pt]{article}

%%%%%%%%%%%%%%%%%%%%%%%%%%%%%%%%%%%%%%%%%%%%%%%%%%%%%%%%%%%%%%%%%%%%%%%%%%%%%%%%%%%%%%%%%%%%%%%%%%%%%%%%%%%%%%%%%%%%%%%%%%%%%%%%%%%%%%%%%%%%%%%%%%%%%%%%%%%%%%%%%%%%%%%%%%%%%%%%%%%%%%%%%%%%%%%%%%%%%%%%%%%%%%%%%%%%%%%%%%%%%%%%%%%%%%%%%%%%%%%%%%%%%%%%%%%%

\usepackage{eurosym}
\usepackage{vmargin}
\usepackage{amsmath}
\usepackage{graphics}
\usepackage{epsfig}
\usepackage{enumerate}
\usepackage{multicol}
\usepackage{subfigure}
\usepackage{fancyhdr}
\usepackage{listings}
\usepackage{framed}
\usepackage{graphicx}
\usepackage{amsmath}
\usepackage{chngpage}

%\usepackage{bigints}
\usepackage{vmargin}

% left top textwidth textheight headheight

% headsep footheight footskip

\setmargins{2.0cm}{2.5cm}{16 cm}{22cm}{0.5cm}{0cm}{1cm}{1cm}

\renewcommand{\baselinestretch}{1.3}

\setcounter{MaxMatrixCols}{10}

\begin{document}


%%%%%%%%%%%%%%%%%%%%%%%%%%%%%%%%%%%%%%%%%%%

 3
PLEASE TURN OVERA5
Employees of a company are given a performance appraisal each year. The appraisal
results in each employee s performance being rated as High (H), Medium (M) or Low
(L). From evidence using previous data it is believed that the performance rating of an
employee evolves as a Markov chain with transition matrix:
H
M
2
H 1
P
M
L
L
2
1 2
2
for some parameter
1
2
.
\begin{enumerate}[(a)]
\item Draw the transition graph of the chain. 
\item Determine the range of values for
transition matrix. 
for which the matrix P is a valid
\item Explain whether the chain is irreducible and/or aperiodic.
\item For = 0.2, calculate the proportion of employees who, in the long run, are in
state L.

\item (v) Given that = 0.2, calculate the probability that an employee s rating in the
third year, X 3 , is L:
(a)
(b)
(c)

in the case that the employee s rating in the first year, X 1 , is H
in the case X 1 = M
in the case X 1 = L
\end{enumerate}
%%%%%%%%%%%%%%%%%%%%%%%%%%%%%%%%%

A5
(i)
Transition graph given below.
2
1
1 2
State H
State M
2
1
2
2
State L
(ii)
Transition probabilities must lie in [0,1]. Thus we need
and 1
(iii)
2
0, 1 - 2
0
0.
The solution of the quadratic is the interval 1
2
5 1
,
2
2
conditions are satisfied simultaneously for 1
[0, ].
2
5
, so all
2
The chain is both irreducible, as every state can be reached from every other
state, and aperiodic, as the chain may remain at its current state for all H, M,
L.
Page 7Subject CT4
(iv)
Models Core Technical
April 2006
Examiners Report
From the result in (iii), a stationary probability distribution exists and it is
unique. Let = ( H , M , L ) denote the stationary distribution. Then, can be
determined by solving P = .
For
= 0.2, the transition matrix becomes
P
0.76 0.2 0.04
0.2
0.04 0.6
0.2 0.2
0.76
So that the system P =
0.76
0.2
0.04
H
H
H
+ 0.2
+ 0.6
+ 0.2
reads
M
M
M
+0.04
+0.2
+0.76
L
L
L
=
=
=
(1)
H
M
(2)
L
Discard the second of these equations and use also that the stationary
probabilities must also satisfy
H
+
M
+
L
= 1
(3)
Subtracting (2) from (1) gives
H
=
L .
Substituting into (1) we obtain H = M , thus (3) gives that H = M =
The proportion of employees who are in state L in the long run is 1/3.
%%%%%%%%%%%%%%%%%%%%%%%%%%%%%%%%%%%%%%%%%%%%%%%%%%%%%%%%%%%%%%%%%%%%%%%%%%%%%%%%%%
(v)
L
The second order transition matrix is
P
2
0.76 0.2 0.04 0.76 0.2 0.04
0.2
0.04 0.6
0.2 0.2
0.76 0.2
0.04 0.6
0.2 0.2
0.76
0.6192 0.28
0.28
0.44
0.1008 0.28
0.1008
0.28
0.6192
The relevant entries are those in the last column, so that the answers are:
(a)
(b)
(c)
Page 8
=1/3.
0.1008
0.28
0.6192.Subject CT4
\end{document}
