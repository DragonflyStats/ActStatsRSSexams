\documentclass[a4paper,12pt]{article}

%%%%%%%%%%%%%%%%%%%%%%%%%%%%%%%%%%%%%%%%%%%%%%%%%%%%%%%%%%%%%%%%%%%%%%%%%%%%%%%%%%%%%%%%%%%%%%%%%%%%%%%%%%%%%%%%%%%%%%%%%%%%%%%%%%%%%%%%%%%%%%%%%%%%%%%%%%%%%%%%%%%%%%%%%%%%%%%%%%%%%%%%%%%%%%%%%%%%%%%%%%%%%%%%%%%%%%%%%%%%%%%%%%%%%%%%%%%%%%%%%%%%%%%%%%%%

%%- B2
\usepackage{eurosym}
\usepackage{vmargin}
\usepackage{amsmath}
\usepackage{graphics}
\usepackage{epsfig}
\usepackage{enumerate}
\usepackage{multicol}
\usepackage{subfigure}
\usepackage{fancyhdr}
\usepackage{listings}
\usepackage{framed}
\usepackage{graphicx}
\usepackage{amsmath}
\usepackage{chngpage}

%\usepackage{bigints}
\usepackage{vmargin}

% left top textwidth textheight headheight

% headsep footheight footskip

\setmargins{2.0cm}{2.5cm}{16 cm}{22cm}{0.5cm}{0cm}{1cm}{1cm}

\renewcommand{\baselinestretch}{1.3}

\setcounter{MaxMatrixCols}{10}

\begin{document}

%%%%%%%%%%%%%%%%%%%%%%%%%%%%%%%%%%%%%%%%%%%%%%%%%%%%%%%%%%%%%%%%%%%

A national mortality investigation is carried out over the calendar years 2002, 2003
and 2004. Data are collected from a number of insurance companies.
Deaths during the period of the investigation,
x ,
are classified by age nearest at death.
Each insurance company provides details of the number of in-force policies on
1 January 2002, 2003, 2004 and 2005, where policyholders are classified by age
nearest birthday, P x (t).
(i)
(a)
(b)
State the rate year implied by the classification of deaths.
State the ages of the lives at the start of the rate interval.

(ii) Derive an expression for the exposed to risk, in terms of P x (t), which may be
used to estimate the force of mortality in year t at each age. State any
assumptions you make.

(iii) Describe how your answer to (ii) would change if the census information
provided by some companies was P x * t , the number of in-force policies on
1 January each year, where policyholders are classified by age last birthday.

[Total 7]
\end{document}
