\documentclass[a4paper,12pt]{article}

%%%%%%%%%%%%%%%%%%%%%%%%%%%%%%%%%%%%%%%%%%%%%%%%%%%%%%%%%%%%%%%%%%%%%%%%%%%%%%%%%%%%%%%%%%%%%%%%%%%%%%%%%%%%%%%%%%%%%%%%%%%%%%%%%%%%%%%%%%%%%%%%%%%%%%%%%%%%%%%%%%%%%%%%%%%%%%%%%%%%%%%%%%%%%%%%%%%%%%%%%%%%%%%%%%%%%%%%%%%%%%%%%%%%%%%%%%%%%%%%%%%%%%%%%%%%

\usepackage{eurosym}
\usepackage{vmargin}
\usepackage{amsmath}
\usepackage{graphics}
\usepackage{epsfig}
\usepackage{enumerate}
\usepackage{multicol}
\usepackage{subfigure}
\usepackage{fancyhdr}
\usepackage{listings}
\usepackage{framed}
\usepackage{graphicx}
\usepackage{amsmath}
\usepackage{chngpage}

%\usepackage{bigints}
\usepackage{vmargin}

% left top textwidth textheight headheight

% headsep footheight footskip

\setmargins{2.0cm}{2.5cm}{16 cm}{22cm}{0.5cm}{0cm}{1cm}{1cm}

\renewcommand{\baselinestretch}{1.3}

\setcounter{MaxMatrixCols}{10}

\begin{document}
PLEASE TURN OVER104 Questions
B1
B2
Calculate 0.25 p 80 and 0.25 p 80.5 , using the ELT15 (Females) mortality table and
assuming a uniform distribution of deaths.



%%%%%%%%%%%%%%%%%%%%%%%%%%%%%%%%%%%%%%%%%%%%%%%%%%%%%%%%%%%%%%%%%%%
Page 13
104 Solutions
B1
= 1 - 0.25 q 80 = 1 - 0.25 \times q 80
under the assumption of a uniform distribution of deaths (UDD)
between ages 80 and 81.
0.25 p 80
From ELT 15, q 80 = 0.05961, so
= 1 - 0.25 \times 0.05961 = 0.98510
0.25 p 80
ALTERNATIVE 1
Under UDD we have, for 0 - s < t - 1,
t - s q x + s
=
( t - s ) q x
.
1 - sq x
Putting t = 0.75, s = 0.5 and x = 80, therefore,
0.75 - 0.5 q 80 + 0.5
0.25 p 80.5
= 1 -
=
0.25 q 80
, and so
1 - 0.5 q 80
0.25 q 80
.
1 - 0.5 q 80
Using ELT15, this is evaluated as
1 -
0.25 ( 0.05961 )
1 - 0.5 ( 0.05961 )
= 1 -
0.01490
= 1 - 0.01536 = 0.98464
0.97020
ALTERNATIVE 2
Using
t
p x = s p x ⋅ t - s p x + s ,
0.75 p 80 = 0.5 p 80 ⋅ 0.25 p 80.5
Using an assumption of UDD between ages 80 and 81, we have
0.5 p 80
= 1 – 0.5 \times 0.05961 = 0.97020
0.75 p 80 =
Page 14
1 – 0.75 \times 0.05961 = 0.95529
B2
0.75 p 80
=
0.95529
= 0.98463
0.97020
So, 0.25 p 80.5 =
(i) (a) The age definition changes 6 months before/after each birthday, so this
is a life year rate interval.
(b) Lives are aged x - 1⁄2 at the start of the rate interval.
(ii)
0.5 p 80
Under the principle of correspondence the age definition of deaths and census
should correspond, which they do here. So we do not need to adjust the
census information.
3
The exposed to risk is given by
E x c
= \int  P x ( t ) dt .
0
Assuming P x ( t ) is linear over calendar years, we can approximate this to
2
E x c = \sum 
0
1
( P x ( t ) + P x ( t + 1 ) ) , where t is measured from 1 January 2002
2
1
⎛ 1
⎞
= ⎜ P x ( 0 ) + P x ( 1 ) + P x ( 2 ) + P x ( 3 ) ⎟
2
⎝ 2
⎠
(iii)
The age definitions for deaths and census no longer correspond. So, we need
to adjust the census information for those companies who supply details of
P x * ( t ) .
Assuming birthdays are uniformly distributed over the calendar year,
1
we can approximate P x ( t ) ≈ P x * - 1 ( t ) + P x * ( t ) .
2
(
)
And the exposed to risk is then:
2
E x c = \sum 
0
2
1
( P x ( t ) + P x ( t + 1 ) )
2
(
) (
)
1 ⎛ 1
1
⎞
= \sum  ⎜ P x * - 1 ( t ) + P x * ( t ) + P x * - 1 ( t + 1 ) + P x * ( t + 1 ) ⎟
2
⎠
0 2 ⎝ 2
1
1
1
= P x * - 1 ( 0 ) + P x * ( 0 ) + P x * - 1 ( 1 ) + P x * ( 1 ) + P x * - 1 ( 2 ) + P x * ( 2 ) + P x * - 1 ( 3 ) + P x * ( 3 )
4
2
4
(
) (
) (
)

\end{document}
