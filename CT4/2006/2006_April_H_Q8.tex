\documentclass[a4paper,12pt]{article}

%%%%%%%%%%%%%%%%%%%%%%%%%%%%%%%%%%%%%%%%%%%%%%%%%%%%%%%%%%%%%%%%%%%%%%%%%%%%%%%%%%%%%%%%%%%%%%%%%%%%%%%%%%%%%%%%%%%%%%%%%%%%%%%%%%%%%%%%%%%%%%%%%%%%%%%%%%%%%%%%%%%%%%%%%%%%%%%%%%%%%%%%%%%%%%%%%%%%%%%%%%%%%%%%%%%%%%%%%%%%%%%%%%%%%%%%%%%%%%%%%%%%%%%%%%%%

\usepackage{eurosym}
\usepackage{vmargin}
\usepackage{amsmath}
\usepackage{graphics}
\usepackage{epsfig}
\usepackage{enumerate}
\usepackage{multicol}
\usepackage{subfigure}
\usepackage{fancyhdr}
\usepackage{listings}
\usepackage{framed}
\usepackage{graphicx}
\usepackage{amsmath}
\usepackage{chngpage}

%\usepackage{bigints}
\usepackage{vmargin}

% left top textwidth textheight headheight

% headsep footheight footskip

\setmargins{2.0cm}{2.5cm}{16 cm}{22cm}{0.5cm}{0cm}{1cm}{1cm}

\renewcommand{\baselinestretch}{1.3}

\setcounter{MaxMatrixCols}{10}

\begin{document}
\begin{enumerate}


%%%%%%%%%%%%%%%%%%%%%%%%%%%%%%%%%%%%%%%%%%%

B2
(i)
(a) Explain why it is important to sub-divide data when carrying out
mortality investigations.
(b) Describe the problems that can arise with sub-dividing data.
[4]


[Total 7]
%%%%%%%%%%%%%%%%%%%%%%%%%%%%%%%%%%%%%%%%%%%

 7








%%%%%%%%%%%%%%%%%%%%%%%%%%%%%%%%%%%%%%%%%%%%%%%%%%%%%%%%%%%%%%%%%%%%%%%%%%%%%%%%%

B2
(i)
and
(a)
3 are
parameters to be estimated.
The models of mortality we use assume that we can observe a group of lives with the same mortality characteristics. This is not possible in
practice.
However, data can be sub-divided according to certain characteristics that we know to have a significant effect on mortality.
This will reduce the heterogeneity of each group, so that we can at least observe groups with similar, but not the same, characteristics.
(b)
Sub-dividing data using many factors can result in the numbers in each class being too low.
It is necessary to strike a balance between homogeneity of the group and retaining a large enough group to make statistical analysis
possible.
Sufficient data may not be collected to allow sub-division.
This may be because marketing pressures mean proposal forms are kept to a minimum.

%%%%%%%%%%%%%%%%%%%%%%%%%%%%%%%%%%%%%%%%%%%%%%%%%%%%%%%%%%%%%%%%%%%%%%%%%%%%
(ii)
Models Core Technical
April 2006
Examiners Report
The following are factors often used:
Sex
Age
Type of policy
Smoker/Non-smoker status
Level of underwriting
Duration in force
Sales channel
Policy size
Occupation (or social class) of policyholder
Known impairments
Geographical region
%%%%%%%%%%%%%%%%%%%%%%%%%%%%%%%%%%%%%%%%%%%%%%%%%%%%%%%%%%%%%%%%%%%%%%%%%%%%
\end{document}
