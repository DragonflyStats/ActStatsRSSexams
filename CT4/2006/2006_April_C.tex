\documentclass[a4paper,12pt]{article}

%%%%%%%%%%%%%%%%%%%%%%%%%%%%%%%%%%%%%%%%%%%%%%%%%%%%%%%%%%%%%%%%%%%%%%%%%%%%%%%%%%%%%%%%%%%%%%%%%%%%%%%%%%%%%%%%%%%%%%%%%%%%%%%%%%%%%%%%%%%%%%%%%%%%%%%%%%%%%%%%%%%%%%%%%%%%%%%%%%%%%%%%%%%%%%%%%%%%%%%%%%%%%%%%%%%%%%%%%%%%%%%%%%%%%%%%%%%%%%%%%%%%%%%%%%%%

\usepackage{eurosym}
\usepackage{vmargin}
\usepackage{amsmath}
\usepackage{graphics}
\usepackage{epsfig}
\usepackage{enumerate}
\usepackage{multicol}
\usepackage{subfigure}
\usepackage{fancyhdr}
\usepackage{listings}
\usepackage{framed}
\usepackage{graphicx}
\usepackage{amsmath}
\usepackage{chngpage}

%\usepackage{bigints}
\usepackage{vmargin}

% left top textwidth textheight headheight

% headsep footheight footskip

\setmargins{2.0cm}{2.5cm}{16 cm}{22cm}{0.5cm}{0cm}{1cm}{1cm}

\renewcommand{\baselinestretch}{1.3}

\setcounter{MaxMatrixCols}{10}

\begin{document}
\begin{enumerate}
%%%%%%%%%%%%%%%%%%%%%%%%%%%%%%%%%%%%%%%%%%%

 4A6
(i)
(a) Explain what is meant by a Markov jump process.
(b) Explain the condition needed for such a process to be time-
homogeneous.

(ii)
Outline the principal difficulties in fitting a Markov jump process model with
time-inhomogeneous rates.

A company provides sick pay for a maximum period of six months to its employees
who are unable to work. The following three-state, time-inhomogeneous Markov
jump process has been chosen to model future sick pay costs for an individual:
(t)
Healthy
(H)
(t)
(t)
Sick
(S)
(t)
Dead
(D)
Where Sick means unable to work and Healthy means fit to work.
The time dependence of the transition rates is to reflect increased mortality and morbidity rates as an employee gets older. Time is expressed in years.
(iii) Write down Kolmorgorov s forward equations for this process, specifying theappropriate transition matrix.

(iv) (a)
Given an employee is sick at time w < T, write down an expression for the probability that he or she is sick throughout the period w < t < T.
(b)
Given that a transition out of state H occurred at time w, state the probability that the transition was into state S.
(c)
For an employee who is healthy at time , give an approximateexpression for the probability that there is a transition out of state H in
a small time interval [w, w + dw], where w > . Your expression should be in terms of the transition rates and P HH ( , w ) only.

(v)
Using the results of part (iv) or otherwise, derive an expression for the probability that an employee is sick at time T and has been sick for less than 6 months, given that they were healthy at time < T - 1⁄2. Your expression should be in terms of the transition rates and P HH ( , w ) only.

%%%%%%%%%%%%%%%%%%%%%%%%%%%%%%%%%%%%%%%%%%%

 5
PLEASE TURN OVER(vi)
Comment on the suggestions that:
(a)
(b)
(t) should also depend on the holding time in state S, and
mortality rates can be ignored.

[Total 14]
%%%%%%%%%%%%%%%%%%%%%%%%%%%%%%%%%%%%%%%%%%%%%%%%%
A6
(i)
(ii)
%%-- Models Core Technical
%%-- April 2006
%%-- Examiners Report
(a) A continuous-time Markov process X t ,t 0 with a discrete state space S is called a Markov jump process.
(b) In the case where the probabilities P X t j | X s i for i, j in S and 0 s t depend only on the length of time interval t s , the process
is called time-homogeneous.
A model with time-inhomogeneous rates has more parameters, and there may not be sufficient data available to estimate these parameters.
Also, the solution to Kolmogorov s equations may not be easy (or even possible) to find analytically.
(iii)
P ( t )
P ( t ). A ( t )
where
( t )
A ( t )
( t )
( t )
( t )
0
( t )
0
( t )
( t )
( t )
0
T
(iv)
(a)
Pr(Waiting time T
w X w
S ) exp
( ( t )
( t )) dt
w
(b)
Given there is a transition from state H at time w, the probabilities that this is into state S or D are given by the relative transition rates at time $w$.
So Probability into state S =
(c)
( w )
( w )
( w )
This is the probability that the individual is in state H at time w, multiplied by the sum of transition rates out of state H at time w, that is:
P HH ( , w ).( ( w )
( w )) dw
%%-- Page 9Subject CT4
(v)
%%-- Models Core Technical
%%-- April 2006
%%-- Examiners Report
Expressing time in years,
Pr( X T
S , Waiting time 1/ 2 X
H )
T
Pr(Transition from state H at w ) Pr(Transition toS) Pr(stays in S to time T) dW
T 1/ 2
T
=
P HH ( , w ).( ( w )
( w )).
T 1/ 2
T
=
(vi)
(a)
( w )
( w )
.exp
( ( t )
( t )) dt . dw
w
T
P HH ( , w ). ( w ).exp
T 1/ 2
T
( w )
( ( t )
( t )) dt . dw
w
This is likely to improve the predictive power of the model because:
There is empirical evidence that recovery rates depend on the duration of the sickness.
The limit of 6 months on sick pay may cause some durational effects around this point.
However this would make the model more complicated to analyse, and increase the volume of data required to fit parameters reliably.
(b)
%%-- Page 10
For individuals in employment mortality rates are likely to be low, and
may be ignorable. It is less likely that mortality out of state S could be
excluded.Subject CT4
Models Core Technical
April 2006
Examiners Report
104 Solutions
