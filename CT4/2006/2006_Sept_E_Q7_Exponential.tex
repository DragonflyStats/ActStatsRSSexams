\documentclass[a4paper,12pt]{article}

%%%%%%%%%%%%%%%%%%%%%%%%%%%%%%%%%%%%%%%%%%%%%%%%%%%%%%%%%%%%%%%%%%%%%%%%%%%%%%%%%%%%%%%%%%%%%%%%%%%%%%%%%%%%%%%%%%%%%%%%%%%%%%%%%%%%%%%%%%%%%%%%%%%%%%%%%%%%%%%%%%%%%%%%%%%%%%%%%%%%%%%%%%%%%%%%%%%%%%%%%%%%%%%%%%%%%%%%%%%%%%%%%%%%%%%%%%%%%%%%%%%%%%%%%%%%

\usepackage{eurosym}
\usepackage{vmargin}
\usepackage{amsmath}
\usepackage{graphics}
\usepackage{epsfig}
\usepackage{enumerate}
\usepackage{multicol}
\usepackage{subfigure}
\usepackage{fancyhdr}
\usepackage{listings}
\usepackage{framed}
\usepackage{graphicx}
\usepackage{amsmath}
\usepackage{chngpage}

%\usepackage{bigints}
\usepackage{vmargin}

% left top textwidth textheight headheight

% headsep footheight footskip

\setmargins{2.0cm}{2.5cm}{16 cm}{22cm}{0.5cm}{0cm}{1cm}{1cm}

\renewcommand{\baselinestretch}{1.3}

\setcounter{MaxMatrixCols}{10}

\begin{document}
\begin{enumerate}
A commuter catches a bus each morning for 100 days. The buses arrive at the stop according to a Poisson process, at an average rate of one per 15 minutes, so if X_i is the waiting time on day i, then X_i has an exponential distribution with parameter
1
15
so
E[X_i ] = 15, Var[X_i ] = 15 2 = 225.
\begin{enumerate}
\item (i) Calculate (approximately) the probability that the total time the commuter
spends waiting for buses over the 100 days exceeds 27 hours.

\item (ii) At the end of the 100 days the bus frequency is increased, so that buses arrive at one per 10 minutes on average (still behaving as a Poisson process). The commuter then catches a bus each day for a further 99 days. Calculate (approximately) the probability that the total time spent waiting over the whole 199 days exceeds 40 hours.
\end{enumerate}
%%%%%%%%%%%%%%%%%%%%%%%%%%%%%%%%%%%%
\newpage


7
\begin{itemize}
\item (i)
As stated in the question, if $X_i$ is the waiting time on day i, then X_i has an
exponential distribution with parameter
1
15
so E(X_i ) = 15, Var(X_i ) = 15 2 = 225.
\item If X is the total waiting time over the 100 days, X = ∑ i = 1 X_i ,
100
Page 5Subject CT3 (Probability and Mathematical Statistics Core Technical) — September 2006 — Examiners’ Report
so $E [ X ] = 1500$ and $Var [ X ] = 22500$ and by the CLT
X has approximately an N (1500, 22500) distribution,
⎛ 1620 − 1500 ⎞
so P ( X > 1620) ≈ 1 − \Phi ⎜
⎟ = 1 − \Phi(0.8) = 0.2119.
150
⎝
⎠
\item (ii)
If Y j is the waiting time on day j of the extra 99 days, then E ( Y j ) = 10 and
Var ( Y j ) = 100 so that if Y =
∑ j = 1 Y j
99
is the total waiting time over the 99 days, then Y is approximately N (990,9900) by CLT.
\item If Z = X + Y (so that Z is the total waiting time over the whole 199 days), then since X and Y are independent, Z is approximately N (1500+990, 22500+9900),
i.e. N (2490, 32400).
⎛ 2400 − 2490 ⎞
Hence P ( Z > 2400) ≈ 1 − \Phi ⎜
⎟ = 1 − \Phi(−0.5) = \Phi(0.5) = 0.6915.
180
⎝
⎠
\end{itemize}
\end{document}
