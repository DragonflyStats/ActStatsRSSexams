\documentclass[a4paper,12pt]{article}

%% B5 
%%%%%%%%%%%%%%%%%%%%%%%%%%%%%%%%%%%%%%%%%%%%%%%%%%%%%%%%%%%%%%%%%%%%%%%%%%%%%%%%%%%%%%%%%%%%%%%%%%%%%%%%%%%%%%%%%%%%%%%%%%%%%%%%%%%%%%%%%%%%%%%%%%%%%%%%%%%%%%%%%%%%%%%%%%%%%%%%%%%%%%%%%%%%%%%%%%%%%%%%%%%%%%%%%%%%%%%%%%%%%%%%%%%%%%%%%%%%%%%%%%%%%%%%%%%%

\usepackage{eurosym}
\usepackage{vmargin}
\usepackage{amsmath}
\usepackage{graphics}
\usepackage{epsfig}
\usepackage{enumerate}
\usepackage{multicol}
\usepackage{subfigure}
\usepackage{fancyhdr}
\usepackage{listings}
\usepackage{framed}
\usepackage{graphicx}
\usepackage{amsmath}
\usepackage{chngpage}

%\usepackage{bigints}
\usepackage{vmargin}

% left top textwidth textheight headheight

% headsep footheight footskip

\setmargins{2.0cm}{2.5cm}{16 cm}{22cm}{0.5cm}{0cm}{1cm}{1cm}

\renewcommand{\baselinestretch}{1.3}

\setcounter{MaxMatrixCols}{10}

\begin{document}
PLEASE TURN OVERB5
A life insurance company has carried out a mortality investigation. It followed a
sample of independent policyholders aged between 50 and 55 years. Policyholders
were followed from their 50th birthday until they died, they withdrew from the
investigation while still alive, or they celebrated their 55th birthday (whichever of
these events occurred first).
(i)
Describe the censoring that is present in this investigation.

An extract from the data for 12 policyholders is shown in the table below.
B6
Policyholder Last age at which
policyholder was observed
(years and months) Outcome
1
2
3
4
5
6
7
8
9
10
11
12 50 years 3 months
50 years 6 months
51 years 0 months
51 years 0 months
52 years 3 months
52 years 9 months
53 years 0 months
53 years 6 months
54 years 3 months
54 years 3 months
55 years 0 months
55 years 0 months Died
Withdrew
Died
Withdrew
Withdrew
Died
Withdrew
Withdrew
Withdrew
Died
Still alive
Still alive
(ii) Calculate the Nelson-Aalen estimate of the survival function.
(iii) Sketch on a suitably labelled graph the Nelson-Aalen estimate of the survival
function.

[Total 9]


%%%%%%%%%%%%%%%%%%%%%%%%%%%%%%%%%%%%%%%%%%%%%%%%%%%%%%%%%%%%%%%%%%%

B5
(i)
\item  need to compute an initial exposed-to-risk is a pointless complication if
census-type data are available
\item  not so easily generalised as two-state or Poisson models to processes with
more than one decrement, and not so easily generalised as two-state model
to increments
\item  estimate of q x has a higher variance than that of the two-state Poisson
models (though the difference is very small unless mortality is very high)
There will be Type I censoring of lives that survive to age 55 years.
There will be random censoring of lives that withdraw before age 55 years.
(ii)
The calculations are shown in the table below, where durations are measured
in years since the 50th birthday.
Using the convention that, when deaths and withdrawals are observed at the
same duration, deaths occur first:
t j
0
0.25
1.00
2.75
4.25
N j
12
12
10
7
4
d j
1
1
1
1
c j
1
2
2
3
d j / N j ˆ =
Λ
\sum  ( d j / N j )
t
0.0833
0.1000
0.1429
0.25 0.0833
0.1833
0.3262
0.5762
t j ≤ t
ˆ )
Since S ˆ ( t ) = exp( -Λ
t
the estimated survival function is
t S ˆ ( t )
0 ≤ t < 0.25
0.25 ≤ t < 1.00
1.00 ≤ t < 2.75
2.75 ≤ t < 4.25
4.25 ≤ t < 5.00 1.0000
0.9201
0.8325
0.7217
0.5620
Page 19
(iii)
1 1
0.9
0.9
0.8
0.8
0.7
0.7
0.6
0.6
0.5
0.5
0.4
0.4
0.3
0.3
0.2
0.2
0.1
0.1
0 0
0
0.5
1
1.5
2
2.5
3 3
3.5
4 4
4.5
Duration
Duration since 50th birthday

\end{document}
