\documentclass[a4paper,12pt]{article}

%%%%%%%%%%%%%%%%%%%%%%%%%%%%%%%%%%%%%%%%%%%%%%%%%%%%%%%%%%%%%%%%%%%%%%%%%%%%%%%%%%%%%%%%%%%%%%%%%%%%%%%%%%%%%%%%%%%%%%%%%%%%%%%%%%%%%%%%%%%%%%%%%%%%%%%%%%%%%%%%%%%%%%%%%%%%%%%%%%%%%%%%%%%%%%%%%%%%%%%%%%%%%%%%%%%%%%%%%%%%%%%%%%%%%%%%%%%%%%%%%%%%%%%%%%%%

\usepackage{eurosym}
\usepackage{vmargin}
\usepackage{amsmath}
\usepackage{graphics}
\usepackage{epsfig}
\usepackage{enumerate}
\usepackage{multicol}
\usepackage{subfigure}
\usepackage{fancyhdr}
\usepackage{listings}
\usepackage{framed}
\usepackage{graphicx}
\usepackage{amsmath}
\usepackage{chngpage}

%\usepackage{bigints}
\usepackage{vmargin}

% left top textwidth textheight headheight

% headsep footheight footskip

\setmargins{2.0cm}{2.5cm}{16 cm}{22cm}{0.5cm}{0cm}{1cm}{1cm}

\renewcommand{\baselinestretch}{1.3}

\setcounter{MaxMatrixCols}{10}

\begin{document}
An investigation took place into the mortality of persons between exact ages 60 and
61 years. The table below gives an extract from the results. For each person it gives
the age at which they were first observed, the age at which they ceased to be observed
and the reason for their departure from observation.
Person Age at entry
years months Age at exit
years months Reason for exit
1
2
3
4
5
6
7
8
9
10 60
60
60
60
60
60
60
60
60
60 60
61
60
61
60
61
60
61
60
61 withdrew
survived to 61
died
survived to 61
died
survived to 61
died
survived to 61
died
survived to 61
0
1
1
2
3
4
5
7
8
9
6
0
3
0
9
0
11
0
10
0
(i) Estimate q 60 using the Binomial model.
(ii) List the strengths and weaknesses of the Binomial model for the estimation of
empirical mortality rates, compared with the Poisson and two-state models.

[Total 8]
CT4 S2006
7




%%%%%%%%%%%%%%%%%%%%%%%%%%%%%%%%%%%%%%%%%%%%%%%%%%%%%%%%%%%%%%%%%%%
B4
(i)
0.9714
Let the age individual i enters observation be a i and the age that individual i
leaves observation be b i . Define an indicator variable d i such that d i = 0 if
individual i is not observed to die and d i = 1 if individual i dies.
Measure all ages in years since exact age 60.
The estimate of q 60 using the Binomial model is:
10
q ˆ 60 =
\sum  d i
i = 1
10
\sum  ( 1 - a i - ⎡ ⎣ ( 1 - d i )( 1 - b i ) ⎤ ⎦ )
.
i = 1
Page 17
The denominator in this formula shows that for persons who do not die
(d i = 0) the exposed to risk is b i – a i and for persons who die (d i = 1) the
exposed to risk is 1 – a i .
Thus the relevant calculations are shown in the table below (all durations are
in years).
Person a i b i d i 1 - a i 1 – b i 1 - a i - (1 - d i )( - b i )
1
2
3
4
5
6
7
8
9
10 0
1/12
1/12
2/12
3/12
4/12
5/12
7/12
8/12
9/12 6/12
1
3/12
1
9/12
1
11/12
1
10/12
1 0
0
1
0
1
0
1
0
1
0 1
11/12
11/12
10/12
9/12
8/12
7/12
5/12
4/12
3/12 6/12
0
9/12
0
3/12
0
1/12
0
2/12
0 6/12
11/12
11/12
10/12
9/12
8/12
7/12
5/12
4/12
3/12
Totals
4
Therefore q ˆ 60 =
74/12
4
= 0.6486 .
74 /12
ALTERNATIVELY
10
Take the central exposed to risk,
\sum  ( b i - a i ) (in years) and add
1
1⁄2d 60 to give the initial exposed to risk.
This involves estimating q 60 using the formula
q ˆ 60 =
d 60
c
E 60
+ 0.5 d 60
=
4
4
=
= 0.5783.
(59 /12) + 2 83 /12
[This approach is inferior to the first, as it does not use all the information
available in the data, and involves the assumption that the deaths take place,
on average, half way through the year.]
(ii)
Page 18
Strengths of Binomial model
\item  avoids numerical solution of equations
\item  can be generalised to give the Kaplan-Meier estimate
Weaknesses of Binomial model
%%%%%%%%%%%%%%%%%%%%%%%%%%%%%%%%%%%%%%%%%%%%%%%%%%%%%%%%%%%%%%%%%%%%%%%%

\end{document}
