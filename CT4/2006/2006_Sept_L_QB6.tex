\documentclass[a4paper,12pt]{article}

%%- B6
%%%%%%%%%%%%%%%%%%%%%%%%%%%%%%%%%%%%%%%%%%%%%%%%%%%%%%%%%%%%%%%%%%%%%%%%%%%%%%%%%%%%%%%%%%%%%%%%%%%%%%%%%%%%%%%%%%%%%%%%%%%%%%%%%%%%%%%%%%%%%%%%%%%%%%%%%%%%%%%%%%%%%%%%%%%%%%%%%%%%%%%%%%%%%%%%%%%%%%%%%%%%%%%%%%%%%%%%%%%%%%%%%%%%%%%%%%%%%%%%%%%%%%%%%%%%

\usepackage{eurosym}
\usepackage{vmargin}
\usepackage{amsmath}
\usepackage{graphics}
\usepackage{epsfig}
\usepackage{enumerate}
\usepackage{multicol}
\usepackage{subfigure}
\usepackage{fancyhdr}
\usepackage{listings}
\usepackage{framed}
\usepackage{graphicx}
\usepackage{amsmath}
\usepackage{chngpage}

%\usepackage{bigints}
\usepackage{vmargin}

% left top textwidth textheight headheight

% headsep footheight footskip

\setmargins{2.0cm}{2.5cm}{16 cm}{22cm}{0.5cm}{0cm}{1cm}{1cm}

\renewcommand{\baselinestretch}{1.3}

\setcounter{MaxMatrixCols}{10}

\begin{document}
(i) (a)
Describe the general form of the polynomial formula used to graduate
the most recent standard tables produced for use by UK life insurance
companies.
(b)
Show how the Gompertz and Makeham formulae arise as special cases
of this formula.

CT4 S2006
8
(ii)
An investigation was undertaken of the mortality of persons aged between 40
and 75 years who are known to be suffering from a degenerative disease. It is
suggested that the crude estimates be graduated using the formula:
o
(iii)
1
x
2
exp b 0
1
b 1 x
2
b 2
1
x
2
2
.
(a) Explain why this might be a sensible formula to choose for this class of
lives.
(b) Suggest two techniques which can be used to perform the graduation.

The table below shows the crude and graduated mortality rates for part of the
relevant age range, together with the exposed to risk at each age and the
standardised deviation at each age.
Age last
birthday
x
Graduated
force of
mortality
Crude
force of
mortality
Exposed
to risk
o
x 1/ 2
x 1/ 2
E x c
Standardised deviation
E x c
o
x 12
z x
E x c
50
51
52
53
54
55
56
57
0.08127
0.08770
0.09439
0.10133
0.10853
0.11600
0.12373
0.13175
0.07941
0.08438
0.09000
0.10345
0.09200
0.10000
0.11176
0.12222
340
320
300
290
250
200
170
180
x 12
o
x 12
-0.12031
-0.20055
-0.24749
0.11341
-0.79336
-0.66436
-0.44369
-0.35225
Test this graduation for:
(a) overall goodness-of-fit
(b) bias; and
(c) the existence of individual ages at which the graduated rates depart to a
substantial degree from the observed rates
[9]
[Total 15]
END OF PAPER
CT4 S2006
9

%%%%%%%%%%%%%%%%%%%%%%%%%%%%%%%%%%%%%%%%%%%%%%%%%%%%%%%%%%%%%%%%%%%
B6
(i)
(a)
The general form is
\mu  x = (polynomial(1)) + exp(polynomial(2)) ,
where polynomial (1) takes the form
\alpha  0 + \alpha  1 x + \alpha  2 x 2 + ...
and polynomial (2) takes the form
\beta 0 + \beta 1 x + \beta 2 x 2 + ....
(b)
In the case of the Gompertz formula \mu  x = Bc x , then putting
B = exp( \beta 0 ) and c = exp( \beta 1 ) ,
we can re-write the formula as
\mu  x = exp( \beta 0 ) exp( \beta 1 x ) = exp( \beta 0 + \beta 1 x ) ,
which is of the required form if
\alpha  i = 0 for all i
and
\beta i = 0 for i = 2, 3, ....
Page 20
5 5
Similarly the Makeham formula \mu  x = A + Bc x
can be expressed in the required form by putting
A = \alpha  0 , B = exp( \beta 0 ) and c = exp( \beta 1 ) .
(ii)
(a)
The Gompertz formula written
\[\mu_{x} = exp( \beta 0 + \beta 1 x )]\
is an exponential function which implies that the rate of increase of mortality with age is constant.
This is often a reasonable assumption for ordinary lives at middle ages and older ages.
In the special case of the impaired lives known to be suffering from a degenerative disease, it is plausible to suppose that the rate of increase
of mortality might increase with age.
2
1 ⎞
⎛
The term b 2 ⎜ x + ⎟ in the formula can allow for this possibility.
2 ⎠
⎝
(b)
The graduation can be achieved by
maximum likelihood estimation of the parameters
or by ordinary least squares regression
⎡
⎤
1
of log ⎢ \mu  ˆ 1 ⎥ on x + and
2
⎢ ⎣ x+ 2 ⎥ ⎦
(iii)
(a)
2
1 ⎞
⎛
⎜ x + ⎟ .
2 ⎠
⎝
The null hypothesis is that there is no difference between the graduated rates and the underlying rates in the population from which the crude
rates are derived.
To test overall goodness-of-fit we use the chi-squared test.
\sum  z x 2 ∼ \chi^{2} m ,
x
where m is the number of degrees of freedom.
In this case, we have 8 ages, but 3 parameters were estimated when
performing the graduation, so m = 5.
Page 21
The calculations are shown in the table below.
Age x
last
birthday
50
51
52
53
54
55
56
57
Sum
z x
- 0.12031
- 0.20055
- 0.24749
0.11341
- 0.79336
- 0.66436
- 0.44369
- 0.35225
z x 2
0.01447
0.04022
0.06125
0.01286
0.62942
0.44137
0.19686
0.12408
1.52053
The critical value of the chi-squared distribution with 5 degrees of
freedom at the 5 per cent level is 11.07.
Since 1.52052 11.07, we do not reject the null hypothesis and
conclude that the graduation adheres satisfactorily to the data.
(b)
To test for bias we use EITHER the Signs Test or the Cumulative
Deviations test.
Signs Test
The test statistic, P, is the number of signs that is positive.
Under the null hypothesis, P ~ Binomial(8,0.5)
In this case P = 1, and Prob[ P ≤ 1 ] = 0.0352.
Since this probability > 0.025 (two-tailed test) we do not reject the null
hypothesis.
We conclude that the graduated rates are not biased above or below the
crude rates.
Page 22
Cumulative deviations test
The test statistic
o
\sum  ( \mu  ˆ x + 1 E x - \mu  x + 1 2 E x )
x
2
~ Normal(0,1) .
o
\sum  \mu  x + 12 E x
x
The calculations are shown in the table below.
Age x
\mu  ˆ
1 E x
x +
2
o
- \mu  x + 1 E x
2
o
\mu  x + 1 E x
2
last
birthday
50
51
52
53
54
55
56
57
Sum
- 0.63
- 1.06
- 1.32
0.61
- 4.13
- 3.20
- 2.03
- 1.72 27.63
28.06
28.32
29.39
27.13
23.20
21.03
23.72
- 13.48 208.48
The value of the test statistic is therefore
( - 13.48/ √ 208.48) = - 0.9335.
using a two-tailed test, the absolute value of the test statistics is less
than 1.96, so we do not reject the null hypothesis.
We conclude that the graduated rates are not biased above or below the
crude rates.
(c)
To test for the existence of individual ages at which the graduated rates
depart greatly from the observed rates we can use the Individual
Standardised Deviations Test.
There are no ages at which the absolute value of z x exceeds 1.96.
Therefore we do not reject the null hypothesis and conclude that there
are no outliers.
END OF EXAMINERS’ REPORT
Page 23
\end{document}
