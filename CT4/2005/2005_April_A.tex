\documentclass[a4paper,12pt]{article}

%%%%%%%%%%%%%%%%%%%%%%%%%%%%%%%%%%%%%%%%%%%%%%%%%%%%%%%%%%%%%%%%%%%%%%%%%%%%%%%%%%%%%%%%%%%%%%%%%%%%%%%%%%%%%%%%%%%%%%%%%%%%%%%%%%%%%%%%%%%%%%%%%%%%%%%%%%%%%%%%%%%%%%%%%%%%%%%%%%%%%%%%%%%%%%%%%%%%%%%%%%%%%%%%%%%%%%%%%%%%%%%%%%%%%%%%%%%%%%%%%%%%%%%%%%%%

\usepackage{eurosym}
\usepackage{vmargin}
\usepackage{amsmath}
\usepackage{graphics}
\usepackage{epsfig}
\usepackage{enumerate}
\usepackage{multicol}
\usepackage{subfigure}
\usepackage{fancyhdr}
\usepackage{listings}
\usepackage{framed}
\usepackage{graphicx}
\usepackage{amsmath}
\usepackage{chngpage}

%\usepackage{bigints}
\usepackage{vmargin}

% left top textwidth textheight headheight

% headsep footheight footskip

\setmargins{2.0cm}{2.5cm}{16 cm}{22cm}{0.5cm}{0cm}{1cm}{1cm}

\renewcommand{\baselinestretch}{1.3}

\setcounter{MaxMatrixCols}{10}

\begin{document}
\begin{enumerate}
%% Question A1
\item 
(i)
Define each of the following examples of a stochastic process

\begin{enumerate}[(a)]
\item a symmetric simple random walk
\item a compound Poisson process
\end{enumerate}
(ii)
\item %% A2
For each of the processes in (i), classify it as a stochastic process according to its state space and the time that it operates on.
\item 
You have been commissioned to develop a model to project the assets and liabilities of an insurer after one year. 
\begin{itemize}
    \item This has been requested following a change in the regulatory capital requirement. 
    \item Sufficient capital must now be held such that there is
less than a 0.5\% chance of liabilities exceeding assets after one year.
\item The company does not have any existing stochastic models, but estimates have been made in the planning process of worst case scenarios.
\end{itemize}
Set out the steps you would take in the development of the model.

%%%%%%%%%%%%%%%%%%%%%%%%%%%%%%%%%%%%%%%%%%%%%%%%%%%%%%%%%%%%%%%%%%%%%%%%%%%%%%%%%%%%%%%
A1
(i)
(a)
\begin{itemize}
    \item Let $\{Y_1 , Y_2 , \ldots , Y_j\}$ , , be a sequence of independent and identically
distributed random variables with
P Y j
1
P Y j
1
1
2
and define
n
X n
Y j
j 1
\item Then X n
(b)
n 1
constitutes a symmetric simple random walk.
\item Let N t be a Poisson process, t 0 and let $\{Y_1 , Y_2 , \ldots , Y_j\}$ , , be a sequence of i.i.d. random variables. 
\item Then a compound Poisson process is defined by
\[N t
X t
Y j ,
t
0.
j 1\]
\end{itemize}

%%%%%%%%%%%%%%%%%%%%%%%%%%%%%%%%%%%%%%%%%%%%%%%%%%%%%%%%%%%%%%%%%%%%%%%%%%%%%%%%%%%%%%%%%
(ii)
(a) A simple random walk operates on discrete time and has a discrete state space (the set of all integers, Z).
(b) A compound Poisson process operates on continuous time.
It has a discrete or continuous state space depending on whether the variables Y j are discrete or continuous respectively.

%%%%%%%%%%%%%%%%%%%%%%%%%%%%%%%%%%%%%%%%%%%%%%%%%%%%%%%%%%%%%%%%%%%%%%%%%%%%%%%%%%%%%%%%%
\newpage

A2
\begin{itemize}
\item Review the regulatory guidance.
\item Define the scope of the model, for example which factors need to be modelled stochastically.
\item Plan the development of the model, including how the model will be tested and validated.
\item Consider alternative forms of model, and decide and document the chosen approach. Where appropriate, this may involve discussion with experts on the underlying stochastic processes.
\end{itemize}
%Page 4
%Models
%%%%%%%%%%%%%%%%%%%%%%%%%%%%%%%%%%%%%%%%%%%%%%%%%%%%%%%%%%%%%%%%%%%%%%%%%%%%%%%%%%%%%%%%%%%%
\begin{itemize}
    \item Collect any data required, for example historic losses or policy data.
Choose parameters. For economic factors should be able to calibrate to market data. For other factors e.g. expenses, claim distributions need to discuss with staff.
\item Existing worst case scenarios. Discuss with staff who made the estimates, especially to gauge views on the probability of events occurring.
Decide on the software to be used for the model.
\item Write the computer programs.
Debug the program, for example by checking the model behaves as expected for
simple, defined scenarios.
Review the reasonableness of the output. May include:
median outcomes (how do these compare with business plans)
what probability is assigned to worst case scenarios
Test the sensitivity of the model to small changes in parameters.
\item Calculate the capital requirement.
Communicate findings to management. Document.
Other suitable points were given credit, including:
Validate data.
\item Run model on historic data to compare model s predictions with previous
observations.
Review parameters that have greatest effect on outputs.
Present range of capital requirements for differing parameter inputs.
\end{itemize}

\end{document}
