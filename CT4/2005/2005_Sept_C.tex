\documentclass[a4paper,12pt]{article}

%%%%%%%%%%%%%%%%%%%%%%%%%%%%%%%%%%%%%%%%%%%%%%%%%%%%%%%%%%%%%%%%%%%%%%%%%%%%%%%%%%%%%%%%%%%%%%%%%%%%%%%%%%%%%%%%%%%%%%%%%%%%%%%%%%%%%%%%%%%%%%%%%%%%%%%%%%%%%%%%%%%%%%%%%%%%%%%%%%%%%%%%%%%%%%%%%%%%%%%%%%%%%%%%%%%%%%%%%%%%%%%%%%%%%%%%%%%%%%%%%%%%%%%%%%%%

\usepackage{eurosym}
\usepackage{vmargin}
\usepackage{amsmath}
\usepackage{graphics}
\usepackage{epsfig}
\usepackage{enumerate}
\usepackage{multicol}
\usepackage{subfigure}
\usepackage{fancyhdr}
\usepackage{listings}
\usepackage{framed}
\usepackage{graphicx}
\usepackage{amsmath}
\usepackage{chngpage}

%\usepackage{bigints}
\usepackage{vmargin}

% left top textwidth textheight headheight

% headsep footheight footskip

\setmargins{2.0cm}{2.5cm}{16 cm}{22cm}{0.5cm}{0cm}{1cm}{1cm}

\renewcommand{\baselinestretch}{1.3}

\setcounter{MaxMatrixCols}{10}

\begin{document}
\begin{enumerate}

%%%%%%%%%%%%%%%%%%%%%%%%%%%55A6
A Markov jump process X t with state space S = {0, 1, 2,
transition rates:
ii
=
i,i+1
ij
=
= 0
for 0 i N 1
for 0 i N 1
, N} has the following
otherwise
(i) Write down the generator matrix and the Kolmogorov forward equations (in
component form) associated with this process.

(ii) Verify that for 0
p ij ( t ) = e
i
t
N
1 and for all j
i, the function
( t ) j i
( j i )!
is a solution to the forward equations in (i).
(iii)
A7

Identify the distribution of the holding times associated with the jump process.

[Total 7]
A time-inhomogeneous Markov jump process has state space {A, B} and the
transition rate for switching between states equals 2t, regardless of the state currently
occupied, where t is time.
The process starts in state A at t = 0.
(i)
(ii)
Calculate the probability that the process remains in state A until at least
time s.
Show that the probability that the process is in state B at time T, and that it is
in the first visit to state B, is given by T 2 exp
(iii)

T 2
.

(a) Sketch the probability function given in (ii).
(b) Give an explanation of the shape of the probability function.
(c) Calculate the time at which it is most likely that the process is in its
first visit to state B.

[Total 11]
CT4 S2005
4104 Questions


%%%%%%%%%%%%%%%%%%%%%%%%%%%%%%5

A6 The generator matrix is
(i)
September 2005
Examiners Report
0
.
.
.
A
.
.
,
.
0
0
all other entries being zero
The Kolmogorov equations are P ( t )
P ( t ) A .
In a component form the forward equations read
(ii)
p ii ( t ) p ii ( t )
p ij ( t ) p ij ( t )
p iN ( t ) p i , N 1 ( t ).
for 0 i
N 1
for i < j < N
p i , j 1 ( t )
Differentiating the function given in the question, we get first for i = j,
p ii ( t )
e t ,
while for i < j
p ij ( t )
N,
e
t
( t ) j i
( j i )!
e
t
( t ) j i 1
( j i 1)!
We can then check that the above satisfy the forward equations.
(iii)
For i = j(<N), the solution in (ii) implies that p ii ( t ) e
distribution of the holding times T 0 , T 1 ,..., T N
.
1
t
, so that the
is exponential with parameter
For i = N, this is obviously not true; once the chain reaches state N, it stays
there forever.
Page 10Subject CT4
A7
(i)
Models
September 2005
d
P ( t )
dt AA
2 t P AA ( t )
d
ln P AA ( t )
dt
2 t
s 2
ln P AA ( s )
constant
We know P AA (0) 1 , hence constant
Hence, P AA ( s ) exp
(ii)
Examiners Report
0
s 2
P(in first visit to B at time T in state A at t = 0)
T
0
P (remains in A to time s )
P(transition to B in time s, s + ds)
P(remains in B to time T) ds
T
P AA ( s ) 2 s P BB ( s , T ) ds
s 0
Using the result from part (i) and the similar result for P BB with boundary
condition P BB (s, s) = 1, this gives us:
T
e
s 2
2 s e
T 2 s 2
ds
s 0
T
2 s e
T 2
ds
s 0
e
T 2
T 2
Page 11Subject CT4
(a)
September 2005
Examiners Report
The sketch should be shaped like:
(iii)
Models
Time
(b)
Commentary:
Initially probability increases from 0 at T = 0, and
accelerates as the transition rate from A to B increases.
However, as transitions increase, it becomes more likely that the
process has already visited state B and jumped back to A.
Therefore the probability of being in the first visit to B tends
(exponentially) to zero.
(c)
Differentiate to find turning point:
d
e
dt
t 2
t 2
2 t e
t 2
2 t 3 e
t 2
set derivative equal to zero
e
t 2
2 t (1 t 2 ) 0
implies t = 1 for a positive solution
and, from above analysis, this is clearly a maximum.
Page 12Subject CT4
Models
