\documentclass[a4paper,12pt]{article}

%%%%%%%%%%%%%%%%%%%%%%%%%%%%%%%%%%%%%%%%%%%%%%%%%%%%%%%%%%%%%%%%%%%%%%%%%%%%%%%%%%%%%%%%%%%%%%%%%%%%%%%%%%%%%%%%%%%%%%%%%%%%%%%%%%%%%%%%%%%%%%%%%%%%%%%%%%%%%%%%%%%%%%%%%%%%%%%%%%%%%%%%%%%%%%%%%%%%%%%%%%%%%%%%%%%%%%%%%%%%%%%%%%%%%%%%%%%%%%%%%%%%%%%%%%%%

\usepackage{eurosym}
\usepackage{vmargin}
\usepackage{amsmath}
\usepackage{graphics}
\usepackage{epsfig}
\usepackage{enumerate}
\usepackage{multicol}
\usepackage{subfigure}
\usepackage{fancyhdr}
\usepackage{listings}
\usepackage{framed}
\usepackage{graphicx}
\usepackage{amsmath}
\usepackage{chngpage}

%\usepackage{bigints}
\usepackage{vmargin}

% left top textwidth textheight headheight

% headsep footheight footskip

\setmargins{2.0cm}{2.5cm}{16 cm}{22cm}{0.5cm}{0cm}{1cm}{1cm}

\renewcommand{\baselinestretch}{1.3}

\setcounter{MaxMatrixCols}{10}

\begin{document}
\begin{enumerate}
B6
t (weeks) S(t)
0
1
3
6 1.0000
0.9167
0.7130
0.4278
t < 1
t < 3
t < 6
t
(i) Calculate the number of insects dying at durations 3 and 6 weeks.
(ii) Calculate the number of insects whose history was censored.
[6]
[1]
[Total 7]
An investigation into mortality collects the following data:
x
= total number of policies under which death claims are made when the
policyholder is aged x last birthday in each calendar year
P x (t) = number of in-force policies where the policyholder was aged x nearest
birthday on 1 January in year t
(i) State the principle of correspondence.
(ii) Obtain an expression, in terms of the P x (t), for the central exposed to risk, E x c ,
which corresponds to the claims data and which may be used to estimate the
force of mortality in year t at each age x, x . State any assumptions you
make.
[4]
(iii) Comment on the effect on the estimation of the fact that the
rather than deaths, and the P x ( t ) relate to policies, not lives.
CT4 A2005
7
[1]
x
relate to claims,
[4]
[Total 9]
PLEASE TURN OVERB7
An investigation took place into the mortality of pensioners. The investigation began
on 1 January 2003 and ended on 1 January 2004. The table below gives the data
collected in this investigation for 8 lives.
Date of birth Date of entry
into observation Date of exit from
observation Whether
or not exit was
due to death (1)
or other
reason (0)
1 April 1932
1 October 1932
1 November 1932
1 January 1933
1 January 1933
1 March 1933
1 June 1933
1 October 1933 1 January 2003
1 January 2003
1 March 2003
1 March 2003
1 June 2003
1 September 2003
1 January 2003
1 June 2003 1 January 2004
1 January 2004
1 September 2003
1 June 2003
1 September 2003
1 January 2004
1 January 2004
1 January 2004 0
0
1
1
0
0
0
0
The force of mortality,
(i)
70 ,
between exact ages 70 and 71 is assumed to be constant.
(a) Estimate the constant force of mortality,
and the data for the 8 lives in the table.
(b) Hence or otherwise estimate q 70 .
70 ,
using a two-state model
[7]
(ii) Show that the maximum likelihood estimate of the constant force, 70 , using a
Poisson model of mortality is the same as the estimate using the two-state
model.
[5]
(iii) Outline the differences between the two-state model and the Poisson model
when used to estimate transition rates.
[3]
[Total 15]
END OF PAPER
CT4 A2005
8

%%%%%%%%%%%%%%%%%%%%%%%%%%%%%%%%%%%%%%%%%%%%%%%%%%%%%%%%%%%%%%%%%%%%%%%%%%%%%%%%%%%
Examiners report
B6 (i) The principle of correspondence states that a life alive at time t should be
included in the exposure at age x at time t if and only if were that life to die
immediately, he or she would be counted in the deaths data x at age x.
(ii) P x (t) is the number of policies under observation aged x nearest birthday on
1 January in year t.
To correspond with the claims data, we wish to have policies classified by age
last birthday.
Let the number of policies aged x last birthday on 1 January in year t be P x ( t ) .
Then, assuming that birthdays are evenly distributed,
1
P x ( t ) P x 1 ( t ) .
2
P x ( t )
The central exposed to risk is then given by
1
E x c
P x ( t ) dt .
0
Using the trapezium approximation this is
E x c
1
P x ( t ) P x ( t 1) ,
2
and, substituting for the P x ( t ) in terms of P x (t) from the equation above
produces
E x c
(iii)
1 1
P x ( t ) P x 1 ( t )
2 2
1
P x ( t 1) P x 1 ( t 1) .
2
The principle of correspondence still holds, because we are dealing with
claims and policies: one policy can only lead to one claim.
However, because one life may have more than one policy it is possible that
two distinct death claims are the result of the death of the same life.
Therefore claims are not independent, whereas deaths are.
Page 17Subject CT4
Models
April 2005
Examiners report
The effect of this is to increase the variance of the number of claims
(compared to the situation in which each life has one and only one policy) by
the ratio
i 2
i
i
i
,
i
i
where i is the proportion of the lives in the investigation owning i policies (i
= 1, 2, 3, ...).
Typically the ratio will vary for each age x.
B7
d 70
, where v 70 is the total time the members
v 70
of the sample are under observation between exact ages 70 and 71 years.
(i)(a) The two-state estimate of
70
is
v 70,i ,
v 70
i
where v 70,i is the duration that sample member i is under observation between
exact ages 70 and 71 years.
For each sample member, v 70,i = ENDDATE
STARTDATE
where ENDDATE is the earliest of the date at which the observation of that
member ceases and the date of the member s 71st birthday,
and STARTDATE is the latest of the date at which observation of that
member begins and the date of the member s 70th birthday.
The table below shows the computation of v 70 .
i Date
obs.
begins Date of
70th
birthday Date
obs.
ends Date of
71 st
birthday v 70,i
(years)
1
2
3
4
5
6
7
8 1/1/2003
1/1/2003
1/3/2003
1/3/2003
1/6/2003
1/9/2003
1/1/2003
1/6/2003 1/4/2002
1/10/2002
1/11/2002
1/1/2003
1/1/2003
1/3/2003
1/6/2003
1/10/2003 1/1/2004
1/1/2004
1/9/2003
1/6/2003
1/9/2003
1/1/2004
1/1/2004
1/1/2004 1/4/2003
1/10/2003
1/11/2003
1/1/2004
1/1/2004
1/3/2004
1/6/2004
1/10/2004 0.25
0.75
0.5
0.25
0.25
0.3333
0.5833
0.25
Therefore v 70
v 70,i = 3.167.
i
Page 18Subject CT4
Models
April 2005
Examiners report
We observed two deaths (members 3 and 4), so
2
0.6316 .
70
3.167
(b)
q 70 1 exp(
70 )
1 exp( 0.6316) 1 0.5318 0.4682.
(ii)
The contributions to the Poisson likelihood made by each member are
proportional to the following
Member
1 exp(-0.25 70 )
2 exp(-0.75 70 )
3 70 exp(-0.5 70 )
4 70 exp(-0.25 70 )
5 exp(-0.25
70 )
6 exp(-0.3333 70 )
7 exp(-0.5833 70 )
8 exp(-0.25
70 )
The total likelihood, L, is proportional to the product
L
[exp( 3.167
70 )]( 70 )
2
.
Then
log L
3.167
70
2 log
70
so that
d log L
d 70
3.167
2
.
70
Setting this equal to zero and solving for
likelihood estimate,
which is 2/3.167 = 0.6316
Since
d 2 log L
d 70
maximum.
2
2
2
70 produces
the maximum
, which is always negative, we definitely have a
70
This is the same as the estimate from the two-state model.
Page 19Subject CT4
(iii)
Models
April 2005
Examiners report
The Poisson model is not an exact model, since it allows for a non-zero
probability of more than n deaths in a sample of size n.
The variance of the maximum likelihood estimator for the two-state model is
only available asymptotically, whereas that for the Poisson model is available
exactly in terms of the true .
The two-state model extends to processes with increments, whereas the
Poisson model does not.
The Poisson model is a less satisfactory approximation to the multiple state
model when transition rates are high.
Page 20
