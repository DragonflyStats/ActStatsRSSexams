\documentclass[a4paper,12pt]{article}

%%%%%%%%%%%%%%%%%%%%%%%%%%%%%%%%%%%%%%%%%%%%%%%%%%%%%%%%%%%%%%%%%%%%%%%%%%%%%%%%%%%%%%%%%%%%%%%%%%%%%%%%%%%%%%%%%%%%%%%%%%%%%%%%%%%%%%%%%%%%%%%%%%%%%%%%%%%%%%%%%%%%%%%%%%%%%%%%%%%%%%%%%%%%%%%%%%%%%%%%%%%%%%%%%%%%%%%%%%%%%%%%%%%%%%%%%%%%%%%%%%%%%%%%%%%%

\usepackage{eurosym}
\usepackage{vmargin}
\usepackage{amsmath}
\usepackage{graphics}
\usepackage{epsfig}
\usepackage{enumerate}
\usepackage{multicol}
\usepackage{subfigure}
\usepackage{fancyhdr}
\usepackage{listings}
\usepackage{framed}
\usepackage{graphicx}
\usepackage{amsmath}
\usepackage{chngpage}

%\usepackage{bigints}
\usepackage{vmargin}

% left top textwidth textheight headheight

% headsep footheight footskip

\setmargins{2.0cm}{2.5cm}{16 cm}{22cm}{0.5cm}{0cm}{1cm}{1cm}

\renewcommand{\baselinestretch}{1.3}

\setcounter{MaxMatrixCols}{10}

\begin{document}
\begin{enumerate}

Institute of Actuaries103 Questions
A1
An insurance company has a block of in-force business under which policyholders
have been given options and investment-related guarantees. A stochastic model has
been developed which projects option and guarantee costs. You have used the model
to estimate, for the Company Board, the probability of the insurance company having
insufficient assets to honour the payouts under the policies. A Board member has
asked whether there are any factors which could cause this probability to be
inaccurate.
Outline the items you would mention in your response.
A2
(i)
In the context of a stochastic process denoted by {X t : t
(a)
(b)
(c)

J}, define:
state space
time set
sample path

(ii)
Stochastic process models can be placed in one of four categories according to
whether the state space is continuous or discrete, and whether the time set is
continuous or discrete. For each of the four categories:
(a) State a stochastic process model of that type.
(b) Give an example of a problem an actuary may wish to study using a
model from that category.

[Total 6]
A3
A die is rolled repeatedly. Consider the following two sequences:
I
II
B n is the largest number rolled in the first n outcomes.
C n is the number of sixes rolled in the first n outcomes.
For each of these two sequences:
(a)
(b)
(c)
(d)
(e)
CT4 S2005
Explain why it is a Markov chain.
Determine the state space of the chain.
Derive the transition probabilities.
Explain whether the chain is irreducible and/or aperiodic.
Describe the equilibrium distribution of the chain.
2


%%%%%%%%%%%%%%%%%%%%%%%%%%%%%%%%%%%%%%%%%%%%%%%%%%%%%%%%%%%%%%%%%%%%%%%%%%%%%%%%%%%%%%%%%%%%%
103 Part
A1
Items to be mentioned include:
Models will be chosen which it is felt give a reasonable reflection of the underlying
real world processes, but this may not turn out to be the case. (Model error.)
The model may be very sensitive to parameters chosen, and the parameters are
estimates because the true underlying parameters cannot be observed. (Parameter
error.)
Sampling error may result from running insufficient simulations. (It should be
possible to give a confidence interval for the error that could result from this source.)
The management actions assumed may not match what would happen in extreme
circumstances.
Policyholder behaviour, such as take-up rates for options, may differ in practice.
There may be future events, such as legislative changes which affect the
interpretation of the policy conditions, which have not been anticipated in the
modelling.
There may be errors in the coding of the model. The model is likely to be complex
and difficult to verify completely.
The model relies on input data, which may be grouped rather than being able to run
every policy. Any errors in the data could cause the output to be inaccurate.
Page 4Subject CT4
Models
September 2005
Examiners Report
A2
(i)
(ii)
(a) The state space is the set of values which it is possible for each random
variable X t to take.
(b) The time set is the set J, the times at which the process contains a random
variable X t .
(c) A sample path is a joint realisation of the variables X t for all t in J, that is a set
of values for X t (at each time in the time set) calculated using the previous
values for X t in the sample path.
Discrete State Space, Discrete Time
(a) Simple random walk, Markov chain, or any other suitable example
(b) Any reasonable example. For example: No Claims Discount systems, Credit
Rating at end of each year
Discrete State Space, Continuous Time
(a) Poisson process, Markov jump process, for example
(b) Any reasonable example. For example: Claims received by an insurer, Status
of pension scheme member
Continuous State Space, Discrete Time
(a) General random walk, time series, for example
(b) Any reasonable example. For example: Share prices at end of each trading
day, Inflation index
Continuous State Space, Continuous Time
(a) Brownian motion, diffusion or Itô process, for example.
Compound Poisson process if the defined state space is continuous.
(b) Any reasonable example. For example: Share prices during trading period,
Value of claims received by insurer
Page 5Subject CT4 Models
A3 (a) Given the current state (the largest outcome or the number of sixes) up to the
nth roll, no additional information is required to predict the status of the chain
after the next roll. Therefore both B n and C n have the Markov property.
(b) B n has state space {1, 2, 3, 4, 5, 6},
the state space for C n is the set of non-negative integers.
(c) For B n , and 1
and
September 2005
i, j
6,
P B n 1 j | B n i i
6 for j = i,
P B n 1 j | B n i 1
6 for each j >i
P B n 1 j | B n i 0 for i > j
For C n , and for k = 0,1,2,
and
(d)
Examiners Report
,
1
,
6
P C n 1 k 1| C n
k
P C n 1 k | C n k 5
,
6
P C n 1 j | C n k 0 for all other j k , k 1
The chain B n is clearly aperiodic; if currently at state i, it can remain there if
the next outcome is at most i.
It is not irreducible, as it cannot be reached from j for i < j.
C n is again aperiodic; if currently at state i, it can remain there if the next
outcome is not a 6.
It is not irreducible; state k cannot be reached from m if k < m.
(e)
In the long run, B n will reach state 6 and will remain there; hence in
equilibrium P(B n = 6) = 1 for sufficiently large n.
C n cannot decrease and has an infinite state space; therefore, it is certain that it
will escape to infinity with probability one.
Page 6Subject CT4 Models

