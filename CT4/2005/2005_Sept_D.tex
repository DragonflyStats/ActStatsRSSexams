\documentclass[a4paper,12pt]{article}

%%%%%%%%%%%%%%%%%%%%%%%%%%%%%%%%%%%%%%%%%%%%%%%%%%%%%%%%%%%%%%%%%%%%%%%%%%%%%%%%%%%%%%%%%%%%%%%%%%%%%%%%%%%%%%%%%%%%%%%%%%%%%%%%%%%%%%%%%%%%%%%%%%%%%%%%%%%%%%%%%%%%%%%%%%%%%%%%%%%%%%%%%%%%%%%%%%%%%%%%%%%%%%%%%%%%%%%%%%%%%%%%%%%%%%%%%%%%%%%%%%%%%%%%%%%%

\usepackage{eurosym}
\usepackage{vmargin}
\usepackage{amsmath}
\usepackage{graphics}
\usepackage{epsfig}
\usepackage{enumerate}
\usepackage{multicol}
\usepackage{subfigure}
\usepackage{fancyhdr}
\usepackage{listings}
\usepackage{framed}
\usepackage{graphicx}
\usepackage{amsmath}
\usepackage{chngpage}

%\usepackage{bigints}
\usepackage{vmargin}

% left top textwidth textheight headheight

% headsep footheight footskip

\setmargins{2.0cm}{2.5cm}{16 cm}{22cm}{0.5cm}{0cm}{1cm}{1cm}

\renewcommand{\baselinestretch}{1.3}

\setcounter{MaxMatrixCols}{10}

\begin{document}
\begin{enumerate}

B1 Describe the advantages and disadvantages of graduating a set of observed mortality
rates using a parametric formula.

B2 A lecturer at a university gives a course on Survival Models consisting of 8 lectures.
50 students initially register for the course and all attend the first lecture, but as the
course proceeds the numbers attending lectures gradually fall.
Some students switch to another course. Others intend to sit the Survival Models
examination but simply stop attending lectures because they are so boring. In this
university, students who decide not to attend a lecture are not permitted to attend any
subsequent lectures.
The table below gives the number of students switching courses and stopping
attending lectures after each of the first 7 lectures of the course.
Lecture
number Number of students
switching courses Number of students ceasing to
attend lectures but remaining
registered for Survival Models
1
2
3
4
5
6
7 5
3
2
0
0
0
0 1
0
3
1
2
1
0
The university s Teaching Quality Monitoring Service has devised an Index of
Lecture Boringness. This index is defined as the Kaplan-Meier estimate of the
proportion of students remaining registered for the course who attend the final lecture.
In calculating the Index, students who switch courses are to be treated as censored
after the last lecture they attend.
(i) Calculate the Index of Lecture Boringness for the Survival Models course. 
(ii) Explain whether the censoring in this example is likely to be non-informative.

[Total 6]
CT4 S2005
5
%%%%%%%%%%%%%%%%%%%%%%%%%%%55B3
A mortality investigation has been carried out over the three calendar years, 2002,
2003 and 2004.
The deaths during the period of investigation,
date of death, where
x = calendar year of death
x ,
have been classified by age x at the
calendar year of birth.
Censuses of the numbers alive on 1 January in each of the years 2002, 2003, 2004 and
2005 have been tabulated and denoted by
P x (2002), P x (2003), P x (2004) and P x (2005)
respectively, where x is the age last birthday at the date of each census.
(i) State the rate year implied by the classification of deaths, and give the ages of
the lives at the beginning of the rate year.

(ii) Derive an expression for the exposed to risk in terms of the P x (t) (t = 2002,
2003, 2004, 2005) which corresponds to the deaths data and which may be
used to estimate the force of mortality, x+f at age x + f.

(iii) Determine the value of f, stating any assumptions you make.
CT4 S2005
6

[Total 9]

%%%%%%%%%%%%%%%%%%%%%%%%%%%%%%%%%%%%%%%%%%%%%%%%%%%%%%%%%%%%%%%%%%%%%%%%%%%%%%%%%%%%%%%%%%

September 2005
Examiners Report
104 Part
B1
Advantages:
The graduated rates will progress smoothly provided the number of parameters is
small.
Good for producing standard tables.
Can easily be extended to more complex formulae, provided optimisation can be
achieved.
Can fit the same formula to different experiences and compare parameter values to
highlight differences between them.
Disadvantages:
It can be hard to find a formula to fit well at all ages without having lots of
parameters.
Care is required when extrapolating: the fit is bound to be best at ages where we have
lots of data, and can often be poor at extreme ages.
Page 13Subject CT4 Models
B2 The table below gives the relevant calculations.
(i)
September 2005
Lecture n j
d j c j j
1
0
3
1
2
1
0 5
3
2
0
0
0
0 1/50
0
3/41
1/36
2/35
1/33
0
Examiners Report
1
j
S(j)
j
1
2
3
4
5
6
7
8
50
44
41
36
35
33
32
32
49/50
1
38/41
35/36
33/35
32/33
1
0.980
0.980
0.908
0.883
0.833
0.807
0.807
The Index of Lecture Boringness is therefore equal to 0.807.
(ii)
Censoring in this case is unlikely to be non-informative.
This is because the students who switched courses were probably less
interested in the subject matter of Survival Models than those who remained
registered.
Therefore they would have been more likely, had they not switched courses,
to cease attending lectures than those who did not switch.
Page 14Subject CT4 Models
B3 The classification of deaths implies a calendar year rate interval.
(i)
September 2005
Examiners Report
A person who dies will be aged x on the birthday in the calendar year of death,
which implies that he or she will be aged x next birthday on 1 January in the
calendar year of death.
Since 1 January is the start of the rate interval, the age range at the start is x
1 to x.
(ii)
A census of those aged x next birthday on 1 January in each year would
correspond to the classification of deaths.
But we have lives classified by age x last birthday.
However, the number alive aged x next birthday on any date is equal to the
number alive aged x 1 last birthday.
The number alive aged x
P x 1 (t).
1 last birthday on 1 January in year t is given by
At the end of year t this cohort will be aged x last birthday.
Thus, using the trapezium rule, the correct exposed to risk at age x in year t is
given by
1
P x 1 ( t ) P x ( t 1) .
2
Over the three calendar years 2002, 2003 and 2004, we have, therefore,
exposed to risk =
1
P x 1 (2002) P x (2003)
2
1
P x 1 (2003) P x (2004)
2
1
P x 1 (2004) P x (2005) .
2
(iii)
Assuming birthdays are uniformly distributed over the calendar year, the
average age at the start of the rate interval will be x 1⁄2.
Therefore the average age in the middle of the rate interval is x.
Assuming a constant force of mortality between x
f = 0.
1⁄2 and x + 1⁄2, therefore,
Page 15Subject CT4 Models

