\documentclass[a4paper,12pt]{article}

%%%%%%%%%%%%%%%%%%%%%%%%%%%%%%%%%%%%%%%%%%%%%%%%%%%%%%%%%%%%%%%%%%%%%%%%%%%%%%%%%%%%%%%%%%%%%%%%%%%%%%%%%%%%%%%%%%%%%%%%%%%%%%%%%%%%%%%%%%%%%%%%%%%%%%%%%%%%%%%%%%%%%%%%%%%%%%%%%%%%%%%%%%%%%%%%%%%%%%%%%%%%%%%%%%%%%%%%%%%%%%%%%%%%%%%%%%%%%%%%%%%%%%%%%%%%

\usepackage{eurosym}
\usepackage{vmargin}
\usepackage{amsmath}
\usepackage{graphics}
\usepackage{epsfig}
\usepackage{enumerate}
\usepackage{multicol}
\usepackage{subfigure}
\usepackage{fancyhdr}
\usepackage{listings}
\usepackage{framed}
\usepackage{graphicx}
\usepackage{amsmath}
\usepackage{chngpage}

%\usepackage{bigints}
\usepackage{vmargin}

% left top textwidth textheight headheight

% headsep footheight footskip

\setmargins{2.0cm}{2.5cm}{16 cm}{22cm}{0.5cm}{0cm}{1cm}{1cm}

\renewcommand{\baselinestretch}{1.3}

\setcounter{MaxMatrixCols}{10}

\begin{document}
\begin{enumerate}
B4
A life insurance company has investigated the recent mortality experience of its male
term assurance policy holders by estimating the mortality rate at each age, q x . It is
proposed that the crude rates might be graduated by reference to a standard mortality
table for male permanent assurance policy holders with forces of mortality s 1 , so
x
that the forces of mortality
x
1
2
2
implied by the graduated rates q x are given by the
function:
x
1
2
=
s
x
1
2
k ,
where k is a constant.
(i)
CT4 A2005
Describe how the suitability of the above function for graduating the crude
rates could be investigated.

6(ii)
(a) Explain how the constant k can be estimated by weighted least squares.
(b) Suggest suitable weights.

(iii)
B5
Explain how the smoothness of the graduated rates is achieved.
[1]
[Total 7]
A study of the mortality of 12 laboratory-bred insects was undertaken. The insects
were observed from birth until either they died or the period of study ended, at which
point those insects still alive were treated as censored.
The following table shows the Kaplan-Meier estimate of the survival function, based
on data from the 12 insects.
%%%%%%%%%%%%%%%%%%%%%%%%%%%%%%%%%%%%%%%%%%%%%%%%%%%%%%%%%%%%%%%%%%%%%%%%%%%%%%%%%%%%%%%%%%%%%%%%%%%%%%%%%%%%
B4
(i)
The suitability of a linear relationship between
investigated by plotting
x
1
2
against
s
x
1
2
log(1 q x ) against
s
x
1
2
and
x
1
2
could be
log(1 q x s ) or by plotting
and
looking for a linear relationship.
An approximately linear relationship will suffice.
If data are scarce, too close a fit is not to be expected, especially at extreme
ages.
14Subject CT4
(ii)
Models
(a)

Examiners report
We can work with either q x s or
s
x
1 .
2
The value of k which minimises either
w x ( q x q x ) 2
x
or
2
w x
x
x
1
2
x
1
2
should be found (note that the summations are over all relevant ages x)
At each age there will be a different sample size or exposed to risk, E x .
This will usually be largest at ages where many term assurances are
sold (e.g. ages 25 to 50 years) and smaller at other ages.
(b)
The estimation procedure should pay more attention to ages where
there are lots of data. These ages should have a greater influence on
the choice of k than other ages.
This implies weights w x E x .
A suitable choice would be
1
or w x
var q x
w x
(iii)
1
var
or w x = E x
1
x
2
The graduated forces of mortality are a linear function of the forces in the
standard table.
Since the forces in the standard table should already be smooth, a linear
function of them will also be smooth.
B5
(i)
Consider the durations t j at which events take place.
Let the number of deaths at duration t j be d j and the number of insects still at
risk of death at duration t j be n j .
At t j = 1, S(t) falls from 1.0000 to 0.9167.
Since the Kaplan-Meier estimate of S(t) is
S ( t )
(1
( t j )) ,
t j t
15Subject CT4
Models

we must have 0.9167 1
Examiners report
(1) ,
so that (1) 0.0833.
Since (1)
d 1
d
, then we have 1
n 1
n 1
0.0833 ,
and, since all 12 insects are at risk of dying at t j = 1, we must therefore have
d 1 = 1 and n 1 = 12.
Similarly, at t j = 3, we must have 0.7130 0.9167(1
so that (3)
0.9167 0.7130
0.9167
0.222
(3))
d 3
.
n 3
Since we can have at most 11 insects in the risk set at t j = 3, we must have
d 3 = 2 and n 3 = 9.
Similarly, at t j = 6, we must have 0.4278 0.7130(1 (6)) ,
so that (6)
0.7130 0.4278
0.7130
0.400
d 6
.
n 6
Since we can have at most 7 insects in the risk set at t j = 6, we must have
d 6 = 2 and n 6 = 5.
Therefore 2 insects died at duration 3 weeks and 2 insects died at duration 6
weeks.
Alternatively
Some candidates worked back to produce a table in the usual format, as
follows; this received full credit.
t
0
1
3
6
(ii)
S(t) = (1- t )
1.0000
0.9167
0.7130
0.4278
t
0
0.0833
0.22
0.4
n t
12
12
9
5
d t
0
1
2
2
5
c t
2
2
3
7
Summing up the number of deaths we have
total deaths = d 1 d 3 d 6 1 2 2 5 .
Since we started with 12 insects, the remaining 7 insects histories were right-
censored.
16Subject CT4 Models

\end{document}
