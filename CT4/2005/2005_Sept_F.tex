\documentclass[a4paper,12pt]{article}

%%%%%%%%%%%%%%%%%%%%%%%%%%%%%%%%%%%%%%%%%%%%%%%%%%%%%%%%%%%%%%%%%%%%%%%%%%%%%%%%%%%%%%%%%%%%%%%%%%%%%%%%%%%%%%%%%%%%%%%%%%%%%%%%%%%%%%%%%%%%%%%%%%%%%%%%%%%%%%%%%%%%%%%%%%%%%%%%%%%%%%%%%%%%%%%%%%%%%%%%%%%%%%%%%%%%%%%%%%%%%%%%%%%%%%%%%%%%%%%%%%%%%%%%%%%%

\usepackage{eurosym}
\usepackage{vmargin}
\usepackage{amsmath}
\usepackage{graphics}
\usepackage{epsfig}
\usepackage{enumerate}
\usepackage{multicol}
\usepackage{subfigure}
\usepackage{fancyhdr}
\usepackage{listings}
\usepackage{framed}
\usepackage{graphicx}
\usepackage{amsmath}
\usepackage{chngpage}

%\usepackage{bigints}
\usepackage{vmargin}

% left top textwidth textheight headheight

% headsep footheight footskip

\setmargins{2.0cm}{2.5cm}{16 cm}{22cm}{0.5cm}{0cm}{1cm}{1cm}

\renewcommand{\baselinestretch}{1.3}

\setcounter{MaxMatrixCols}{10}

\begin{document}
\begin{enumerate}

B6
Studies of the lifetimes of a certain type of electric light bulb have shown that the
probability of failure, q 0 , during the first day of use is 0.05 and after the first day of
use the force of failure , x , is constant at 0.01.
(i) Calculate the probability that a light bulb will fail within the first 20 days. 
(ii) Calculate the complete expectation of life (in days) of:
(a)
(b)
a one-day old light bulb
a new light bulb

(iii)
Comment on the difference between the complete expectations of life
calculated in (ii) (a) and (b).

[Total 11]
END OF PAPER
CT4 S2005
9

%%%%%%%%%%%%%%%%%%%%%%%%%%%%%%%%%%%%%%%%%%%%%%%%%%%%

\newpage

B6 Let the probability of failure within the first 20 days be 20 q 0 .
(i)
September 2005
Examiners Report
We have:
20 q 0
1
20 p 0
1 (1
1
1 p 0 . 19 p 1
1 q 0 ) exp(
19 )
1 0.95exp( 19 0.01)
1 0.95exp( 0.19)
1 0.95(0.82696)
which is 0.21439.
(ii)
(a)
The complete expectation of life of a one-day old light bulb, e 1 is
given by
e 1
t p 1 dt
0
e
0.01 t
dt
0
Integrating, this gives
1
e
0.01
e 1
1
0 1
0.01
0.01 t
0
= 100 days.
(b)
The complete expectation of life of a new light bulb, e 0 is given by
1
e 0
t p 0 dt
0
t p 0 dt
0
t p 0 dt .
(*)
1
Alternative 1
Assume a uniform distribution of failure times between exact ages 0
and 1,
the first term in (*) is equal to
Page 19Subject CT4
Models
September 2005
1
1 1 p 0
2
1
1 (1
2
Examiners Report
1 q 0 )
1
(1 0.95)
2
0.975
The second term is equal to
1 p 0
t p 1 dt
0.95(100)
0
(using the result from part (i) above).
Therefore:
e 0 0.975 100 0.95 95.975 days.
Alternative 2
Assume a constant force of failure between exact ages 0 and 1
Let this constant force be .
Then
1
1 p 0
exp
ds
exp(
)
0
1
1 q 0
0.95.
So that
exp(
) 0.95
and
log(0.95) 0.0513.
Thus the first term on the right-hand side of (*) is
Page 20Subject CT4
Models
September 2005
1
Examiners Report
1
t p 0 dt
0
exp( 0.0513 t ) dt
0
1
1
exp( 0.0513 t ) 0
0.0513
1
exp( 0.0513) 1
0.0513
0.97478,
and the second term is equal to
1 p 0
t p 1 dt
0.95(100)
0
(using the result from part (i) above).
So that
e 0
(iii)
0.97478 100 0.95 95.97478 days.
The complete expectation of life of a light bulb at any age is an average of the
future lifetimes of all bulbs which have not failed before that age.
The value of e 0 is lower than e 1 because the average e 0 includes the very
short lifetimes of the relatively large proportion of bulbs which fail in the first
day, which deflate the average, whereas e 1 excludes these.
END OF EXAMINERS REPORT
Page 21
