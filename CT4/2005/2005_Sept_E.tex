\documentclass[a4paper,12pt]{article}

%%%%%%%%%%%%%%%%%%%%%%%%%%%%%%%%%%%%%%%%%%%%%%%%%%%%%%%%%%%%%%%%%%%%%%%%%%%%%%%%%%%%%%%%%%%%%%%%%%%%%%%%%%%%%%%%%%%%%%%%%%%%%%%%%%%%%%%%%%%%%%%%%%%%%%%%%%%%%%%%%%%%%%%%%%%%%%%%%%%%%%%%%%%%%%%%%%%%%%%%%%%%%%%%%%%%%%%%%%%%%%%%%%%%%%%%%%%%%%%%%%%%%%%%%%%%

\usepackage{eurosym}
\usepackage{vmargin}
\usepackage{amsmath}
\usepackage{graphics}
\usepackage{epsfig}
\usepackage{enumerate}
\usepackage{multicol}
\usepackage{subfigure}
\usepackage{fancyhdr}
\usepackage{listings}
\usepackage{framed}
\usepackage{graphicx}
\usepackage{amsmath}
\usepackage{chngpage}

%\usepackage{bigints}
\usepackage{vmargin}

% left top textwidth textheight headheight

% headsep footheight footskip

\setmargins{2.0cm}{2.5cm}{16 cm}{22cm}{0.5cm}{0cm}{1cm}{1cm}

\renewcommand{\baselinestretch}{1.3}

\setcounter{MaxMatrixCols}{10}

\begin{document}
\begin{enumerate}

B4
An investigation was carried out into the mortality of male undergraduate students at
a large university. The resulting crude rates were graduated graphically. The
following table shows the observed numbers of deaths at each age x, d x , and the q x s
obtained from the graduation, together with the number of lives exposed to risk at
each age.
(i)
Age x d x q x Exposed-to-risk
18
19
20
21
22
23
24 6
8
12
8
9
6
8 0.0012
0.0013
0.0015
0.0017
0.0019
0.0020
0.0021 5,200
5,000
4,800
5,000
3,800
3,600
3,200
Test whether the overall fit of the graduated rates to the crude data is
satisfactory using a chi-squared test. 
(ii) Comment on your results in (i). [1]
(iii) (a) Describe three possible shortcomings in a graduation which the chi-
squared test cannot detect, and (b) State a test which can be used to detect each one.
CT4 S2005
7

[Total 9]
%%%%%%%%%%%%%%%%%%%%%%%%%%%55B5
An investigation was carried out into the effects of lifestyle factors on the mortality of
people aged between 50 and 65 years. The investigation took the form of a
prospective study following a sample of several hundred individuals from their 50th
birthdays until their 65th birthdays and collecting data on the following covariates for
each person:
X 1 Sex (a categorical variable with 0 = female, 1 = male)
X 2 Cigarette smoking (a categorical variable with 0 = non-smoker, 1 = smoker)
X 3 Alcohol consumption (a categorical variable with 0 = consumes fewer than
21 units of alcohol per week, 1 = consumes 21 or more units of alcohol
per week)
In addition, data were collected on the age at death for persons who died during the
period of investigation.
In order to analyse the data, it was decided to use a Gompertz hazard,
x is the duration since the start of the observation.
(i)
(ii)
x
= Bc x , where
Explain why the Gompertz hazard might be appropriate for analysing the
mortality of persons aged between 50 and 65 years.

Show that the substitution:
B = exp(
0
+
1
X 1 +
2
X 2 +
3
X 3 ),
in the Gompertz model (where 0 ... 3 are parameters to be estimated), leads
to a proportional hazards model for this particular analysis.

(iii)
Using the Gompertz hazard, the parameter estimates in the proportional
hazards model were as follows:
Covariate
Sex
Cigarette smoking
Alcohol consumption
Parameter estimate
1
2
3
0
c
CT4 S2005
Parameter
+0.40
+0.75
0.20
5.00
+1.10
(a) Describe the characteristics of the person to whom the baseline hazard
applies in this model.
(b) Calculate the estimated hazard for a female cigarette smoker aged 55
years who does not consume alcohol.
(c) Show that, according to this model, a cigarette smoker at any age has a
risk of death roughly equal to that of a non-smoker aged eight years
older.

[Total 11]
8

%%%%%%%%%%%%%%%%%%%%%%%%%%%%%%%%%%%%%%%%%%%%%%%%%%%%%%%%%%%%%%%%%%%%%%%%%%%%%%%%%
B4 The null hypothesis is that the observed data come from a population in which
the graduated rates are the true rates.
(i)
September 2005
Examiners Report
The chi-squared statistic is given by the formula:
( d x
x
E x q x ) 2
.
E x q x
The calculations are shown in the table below.
Age E x q x
18
19
20
21
22
23
24
6
8
12
8
9
6
8
E x q x ( E x q x
6.24
6.50
7.20
8.50
7.22
7.20
6.72
E x q x ) 2
0.0576
2.2500
23.0400
0.2500
3.1684
1.4400
1.6384
( E x q x E x q x ) 2
E x q x
0.0092
0.3461
3.2000
0.0294
0.4388
0.2000
0.2438
Therefore the calculated chi-squared value is
0.0092 + 0.3461 + 3.2000 + 0.0294 + 0.4388 + 0.2000 + 0.2438 = 4.4673
Since we have 7 ages, we compare this with the tabulated value at the 5%
level at, say, 4 degrees of freedom (since we lose 2 3 degrees for every
10 ages graduated graphically).
The tabulated value with 4 degrees of freedom is 9.488.
Since 4.4673 < 9.488 we have no evidence to reject the null hypothesis.
(ii)
On the basis of the chi-squared test, the graphical graduation adheres to the
data satisfactorily.
However, there is a large deviation at age 20 which requires further
investigation.
(iii)
Possible shortcomings, and the relevant tests are:
There may be long runs of deviations of the same sign caused by
undergraduation.
These can be detected by the grouping of signs test or the serial correlations
test.
Page 16Subject CT4
Models
September 2005
Examiners Report
There may be one or two large deviations at particular ages, balanced by lots
of small deviations (as in the example in part (i))
These can be detected by the individual standardised deviations test.
The graduated rates may be too high or too low over the whole of the age
range, but by an amount too small for the chi-squared test to detect.
The signs test or the cumulative deviations test will detect this.
The results of the graduation may not be smooth.
This can be detected by looking at the third order differences of the graduated
rates q x . If the rates are smooth, these should be small in magnitude
compared with the quantities themselves and should progress regularly.
B5
(i)
Taking logarithms of the Gompertz hazard produces
log
x
= log B + x log c
which indicates that the rate of increase of the hazard with age is constant.
Empirically, this is often a reasonable assumption for middle ages and older
ages, which include the age range 50 65 years.
(ii)
Putting B = exp(
produces 0 + 1 X 1 + 2 X 2 + 3 X 3 ) into the Gompertz model
= exp( 0 + 1 X 1 + 2 X 2 + 3 X 3 ) . c x ,
x
defining x as duration since 50th birthday.
The hazard can therefore be factorised into two parts:
exp( 0 + 1 X 1 + 2 X 2 +
the covariates, and
3
X 3 ), which depends only on the values of
c x , which depends only on duration.
Therefore the ratio between the hazards for any two persons with different
characteristics does not depend on duration, and so the model is a proportional
hazards model.
(iii)
(a)
The baseline hazard in this model relates to
a female,
non-smoker,
who drinks less than 21 units of alcohol per week.
Page 17Subject CT4
Models
(b)
September 2005
Examiners Report
For a female cigarette smoker who does not consume alcohol we have
X 1 = 0, X 2 = 1, X 3 = 0 and x = 5.
Therefore the hazard is given by
5
(c)
= exp( 0 + 1 .0 + 2 .1 + 3 .0) . c 5
= exp( 5 + 0.75) 1.10 5
= 0.0230.
The hazard for a non-smoker at duration u is given by the formula
u
= exp(
0
+
1
X 1 +
3
X 3 ) . c u ,
The hazard for a smoker at duration v is given by the formula
* v = exp(
0
+
1
X 1 + 0.75 +
3
X 3 ) . c v .
If the smoker s and non-smoker s hazards are the same, then
u = * v ,
which implies that
exp( 0 + 1 X 1 +
= exp( 0 +
X 3 ).c u
1 X 1 + 0.75 +
3
3
X 3 ) . c v .
which simplifies to
c u = exp(0.75) . c v ,
so that
c u /c v = c u
v
= exp(0.75) = 2.117.
Since c = 1.1, we have
1.1 u v = 2.117.
Therefore
u
v = log(2.117)/log(1.1)
= 0.75/0.0953 = 7.87.
So when the two hazards are equal, the non-smoker is approximately
eight years older than the smoker.
Alternatively this could be demonstrated by calculating
and showing that they are approximately the same.
Page 18
u
and * u-8Subject CT4 Models
