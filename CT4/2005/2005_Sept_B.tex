\documentclass[a4paper,12pt]{article}

%%%%%%%%%%%%%%%%%%%%%%%%%%%%%%%%%%%%%%%%%%%%%%%%%%%%%%%%%%%%%%%%%%%%%%%%%%%%%%%%%%%%%%%%%%%%%%%%%%%%%%%%%%%%%%%%%%%%%%%%%%%%%%%%%%%%%%%%%%%%%%%%%%%%%%%%%%%%%%%%%%%%%%%%%%%%%%%%%%%%%%%%%%%%%%%%%%%%%%%%%%%%%%%%%%%%%%%%%%%%%%%%%%%%%%%%%%%%%%%%%%%%%%%%%%%%

\usepackage{eurosym}
\usepackage{vmargin}
\usepackage{amsmath}
\usepackage{graphics}
\usepackage{epsfig}
\usepackage{enumerate}
\usepackage{multicol}
\usepackage{subfigure}
\usepackage{fancyhdr}
\usepackage{listings}
\usepackage{framed}
\usepackage{graphicx}
\usepackage{amsmath}
\usepackage{chngpage}

%\usepackage{bigints}
\usepackage{vmargin}

% left top textwidth textheight headheight

% headsep footheight footskip

\setmargins{2.0cm}{2.5cm}{16 cm}{22cm}{0.5cm}{0cm}{1cm}{1cm}

\renewcommand{\baselinestretch}{1.3}

\setcounter{MaxMatrixCols}{10}

\begin{document}
\begin{enumerate}

%%-- -A4
A life insurance company prices its long-term sickness policies using a three-state
Markov model in continuous time. The states are healthy (H), ill (I) and dead (D). The
forces of transition in the model are HI = , IH = , HD = , ID = v and they are
assumed to be constant over time.
For a group of policyholders observed over a 1-year period, there are:
\begin{itemize}
\item 23 transitions from State to State ;
\item 15 transitions from State to State ;
\item 3 deaths from State ;
\item 5 deaths from State .
\end{itemize}
%%%%%%%%%%%%%%%%%%%%%%%%%%%%%%%%%%%%%%%%%%%
The total time spent in State H is 652 years and the total time spent in State I is 44
years.
A5
\begin{enumerate}[(i)]
\item (i) Write down the likelihood function for these data. 
\item (ii) Derive the maximum likelihood estimate of . 
\item (iii) Estimate the standard deviation of
, the maximum likelihood estimator of .
\end{enumerate}

%%--- [Total 7]
Claims arrive at an insurance company according to a Poisson process with rate
week.
per
Assume time is expressed in weeks.
\begin{enumerate}[(i)]
\item (i) Show that, given that there is exactly one claim in the time interval $[t, t + s]$,
the time of the claim arrival is uniformly distributed on $[t, t + s]$.

\item (ii) State the joint density of the holding times T 0 , T 1 ,
claims.
\item (iii) Show that, given that there are n claims in the time interval $[0, t]$, the number
of claims in the interval [0, s] for s < t is binomial with parameters n and s/t.
\end{enumerate}
%%[Total 7]
%%CT4 S2005
3
, T n between successive
[1]


%%%%%%%%%%%%%%%%%%%%%%%%%%%%%%%%%%%%%%%%%%%%%%%%%%%%%%%%%%%%%%%%%%

A4 The likelihood is
(i)
%%September 2005
L K exp( 652(
(ii)
)) exp( 44(
l = ln L = 652 +23 ln
))
23 15 3 5
+ constant with respect to
Differentiating with respect to
l
%%Examiners Report
gives
23
652
and setting equal to zero gives
0
23
652
23
652
0.0353 p.a.
Differentiating again gives
2
therefore
l 23
2 2
0
is the maximum likelihood estimate
2
(iii)
The variance of
is
1
l
2
2
23
,
2
which we can estimate by
23
.
Therefore the estimated standard deviation of
is
23
0.00736.
%%Page 7
%%Subject CT4 Models
\newpage
A5 Let $N_t$ denote the number of claims up to time t. Since the Poisson process has
stationary increments, we may take t = 0, so that the required conditional
distribution is
(i)
%September 2005
P T 0
y | N s
P T 0
1
%Examiners Report
y , N s
P N s
P
N y
1
1
1, N s
P N s
N y
0
1
But N s N y is independent of N y
and has the same distribution as N s y .
Thus the right hand side above equals
( ye
y
( s y )
) e
y
,
s
s
se
which is the cdf of the uniform distribution on [0, s].
(ii)
Since holding times are independent, each having an exponential distribution,
their joint density is
n
(iii)
e
t 1 t 2 ... t n
1 t , t
1 2 ,..., t n
0.
We have, as in part (i),
P N s
k | N t
n
P N s
k , N t
P N t
P
N s
n
n
k , N t N s
P N t n
n k
Using again that the Poisson process has stationary and independent
increments, and that the number of claims in an interval [0, t] is Poisson ( t),
we derive from above that

%%-- Page 8
%%-- Subject CT4
%%-- Models
%%-- September 2005
s
( s ) k e
k !
e
P N s
k | N t
n

%%-- Examiners Report
( t s ) n k
e
( t s ) n
( n k )!
t
t n k
s ( t s ) n
k !( n k )!
e
( t ) n
n !
k
n !
k
s
t
k
s
1
t
t n n
e
n !
s k ( t s ) n
k !( n k )!
t k t n k
n
k
t
k
n k
which is binomial with parameters n and s/t.
%%Page 9
%%Subject CT4 Models
\end{document}
