\documentclass[a4paper,12pt]{article}

%%%%%%%%%%%%%%%%%%%%%%%%%%%%%%%%%%%%%%%%%%%%%%%%%%%%%%%%%%%%%%%%%%%%%%%%%%%%%%%%%%%%%%%%%%%%%%%%%%%%%%%%%%%%%%%%%%%%%%%%%%%%%%%%%%%%%%%%%%%%%%%%%%%%%%%%%%%%%%%%%%%%%%%%%%%%%%%%%%%%%%%%%%%%%%%%%%%%%%%%%%%%%%%%%%%%%%%%%%%%%%%%%%%%%%%%%%%%%%%%%%%%%%%%%%%%

\usepackage{eurosym}
\usepackage{vmargin}
\usepackage{amsmath}
\usepackage{graphics}
\usepackage{epsfig}
\usepackage{enumerate}
\usepackage{multicol}
\usepackage{subfigure}
\usepackage{fancyhdr}
\usepackage{listings}
\usepackage{framed}
\usepackage{graphicx}
\usepackage{amsmath}
\usepackage{chngpage}

%\usepackage{bigints}
\usepackage{vmargin}

% left top textwidth textheight headheight

% headsep footheight footskip

\setmargins{2.0cm}{2.5cm}{16 cm}{22cm}{0.5cm}{0cm}{1cm}{1cm}

\renewcommand{\baselinestretch}{1.3}

\setcounter{MaxMatrixCols}{10}

\begin{document}
\begin{enumerate}
A3
Let Y 1 , Y 3 , Y 5 ,
variables with
P Y 2 k
, be a sequence of independent and identically distributed random
1
= 1 = P Y 2 k
and define Y 2 k = Y 2 k
(i)
[6]
1 / Y 2 k 1
1
= 1 =
1
,
2
for k = 1, 2,
k = 0, 1, 2,...
.
Show that Y k : k = 1, 2,... is a sequence of independent and identically
distributed random variables.
Hint: You may use the fact that, if X, Y are two variables that take only two
values and E XY
E X E ( Y ), then X, Y are independent.
[4]
(ii) Explain whether or not Y k : k = 1, 2,... constitutes a Markov chain.
(iii) (a)
State the transition probabilities p ij ( n ) = P Y m
n
[1]
= j | Y m = i of the
sequence Y k : k = 1, 2,... .
(b)
CT4 A2005
2
Hence show that these probabilities do not depend on the current state
and that they satisfy the Chapman-Kolmogorov equations.
[3]
[Total 8]A4
Marital status is considered using the following time-homogeneous, continuous time
Markov jump process:
the transition rate from unmarried to married is 0.1 per annum
the divorce rate is equivalent to a transition rate of 0.05 per annum
the mortality rate for any individual is equivalent to a transition rate of 0.025 per
annum, independent of marital status
The state space of the process consists of five states: Never Married (NM),
Married (M), Widowed (W), Divorced (DIV) and Dead (D).
P x is the probability that a person currently in state x, and who has never previously
been widowed, will die without ever being widowed.
(i) Construct a transition diagram between the five states. [2]
(ii) Show, by general reasoning or otherwise, that P NM equals P DIV . [1]
(iii) Demonstrate that:
P NM
P M
1
5
1
4
4
P M
5
1
P DIV
2
[2]
(iv) Calculate the probability of never being widowed if currently in state NM. [2]
(v) Suggest two ways in which the model could be made more realistic.
CT4 A2005
3
[1]
[Total 8]
%%%%%%%%%%%%%%%%%%%%%%%%%%%%%%%%%%%%%%%%%%%%%%%%%%%%%%%%%%%%%%%%%%%%%%%%%%%%%%%%%%%%%
A3
(i)
It is clear that Y 2k can only take two values, ±1, with probabilities
1
P Y 2 k 1 P Y 2 k 1 Y 2 k 1
1 P Y 2 k 1 Y 2 k 1
1
2
and
P Y 2 k
P Y 2 k
1
1
1, Y 2 k
1
1
P Y 2 k
1
1, Y 2 k
1
1
1
2
so that they have the same distribution as Y 2k+1 .
To show that Y 2 k , Y 2 k
1
are independent, we observe first that
Page 5Subject CT4
Models
April 2005
E Y 2 k
E Y 2 k
Examiners report
0 .
1
Next,
E Y 2 k Y 2 k
1
1
E Y 2 k Y 2 k
2
1 | Y 2 k 1
1
E Y 2 k Y 2 k
2
1
1 | Y 2 k 1
1
But
E Y 2 k Y 2 k
1 | Y 2 k 1
1
1 1 0 ( 1) 1,
and similarly E Y 2 k Y 2 k
E Y 2 k Y 2 k
1
1 | Y 2 k 1
1
1
1
2
2
1
1
1, which yields that
0.
Since
E Y 2 k
E Y 2 k
1
E ( Y 2 k Y 2 k 1 )
it now follows from the hint that Y 2 k , Y 2 k
1
are independent.
For the proof to be complete, we need to show that Y 2 k , Y 2 m are also
independent for all k, m. This is obvious from the statement for all k, m
except when m = k + 1 or m = k - 1. For this case, we could either argue as
above or simply state that it is obvious by symmetry.
(ii)
(iii)
Page 6
The sequence Y k : k
P Y 2 k 1 1| Y 2 k
but
P Y 2 k 1 1| Y 2 k
(a)
1 , 2 ,... is not Markov; for instance
1
1, Y 2 k
1
2
1
1
0.
Since the Y k are pairwise independent, we see that for all i, j, m, n,
1
p ij ( n ) P Y m n j | Y m i
.
2Subject CT4
Models
(b)
April 2005
Examiners report
The probabilities do not depend on the current state as they are all 1⁄2
Using the result in (a) we therefore see that
1 1 1 1 1
p ik ( n ) p kj ( r )
2 2 2 2 2
k
1,1
p ij ( n
r ).
which shows that the Chapman
although Y k : k
A4
Kolmogorov equations are satisfied
1 , 2 ,... is not Markov.
(i)
0.025
M
W
0.1
0.1
NM
0.05
0.025
0.1
DIV
0.025
0.025
0.025
D
(ii)
The transitions out of the divorced state are to the same states, and with the
same transition probabilities, as the transitions out of state NM.
Therefore the probability of ever reaching state W is the same from both
states.
Alternatively, this could be shown by producing the equation conditioning on
the first move out of DIV, as in part (iii), and showing this is identical to that
for P NM .
(iii)
Conditioning on the first move out of each state:
P NM
P M
As P D
0.025
P D
0.125
0.025
P D
0.1
1 and P W
0.1
P M
0.125
0.05
P DIV
0.1
0.025
P W
0.1
0 , these give
Page 7Subject CT4
Models
April 2005
0.025
0.125
P NM
0.025
0.1
P M
0.1
0.125
0.05
0.1
P M
P DIV
%%%%%%%%%%%%%%%%%%%%%%%%%%%%%%%%%%%%%%%%%%%%%%%%%%%%%%%%%%%%%%%%%%%%%%%%%%%%
1
5 4
5 P M
1
4 1
2 P DIV
as required.
(iv)
Using P NM
P DIV in the above equations gives:
P NM 1
5 4
5
1 2
5 P NM
P NM
1
4
1
P NM
2
2
5
2
3
(v)
Make mortality and marriage rates age dependent.
Divorce rate dependent on duration of marriage.
Divorce rate dependent on whether previously divorced.
Make mortality rate marital status-dependent.
Other sensible suggestions received credit.
