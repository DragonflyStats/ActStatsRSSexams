\documentclass[a4paper,12pt]{article}

%%%%%%%%%%%%%%%%%%%%%%%%%%%%%%%%%%%%%%%%%%%%%%%%%%%%%%%%%%%%%%%%%%%%%%%%%%%%%%%%%%%%%%%%%%%%%%%%%%%%%%%%%%%%%%%%%%%%%%%%%%%%%%%%%%%%%%%%%%%%%%%%%%%%%%%%%%%%%%%%%%%%%%%%%%%%%%%%%%%%%%%%%%%%%%%%%%%%%%%%%%%%%%%%%%%%%%%%%%%%%%%%%%%%%%%%%%%%%%%%%%%%%%%%%%%%

\usepackage{eurosym}
\usepackage{vmargin}
\usepackage{amsmath}
\usepackage{graphics}
\usepackage{epsfig}
\usepackage{enumerate}
\usepackage{multicol}
\usepackage{subfigure}
\usepackage{fancyhdr}
\usepackage{listings}
\usepackage{framed}
\usepackage{graphicx}
\usepackage{amsmath}
\usepackage{chngpage}

%\usepackage{bigints}
\usepackage{vmargin}

% left top textwidth textheight headheight

% headsep footheight footskip

\setmargins{2.0cm}{2.5cm}{16 cm}{22cm}{0.5cm}{0cm}{1cm}{1cm}

\renewcommand{\baselinestretch}{1.3}

\setcounter{MaxMatrixCols}{10}

\begin{document}
\begin{enumerate}
(i)
Draw a transition diagram for the process defined by the number of breakdowns occurring up to time t.
(ii) Write down the Kolmogorov equations obeyed by P 0 ( t ), P 1 ( t ) and P 2 ( t ) .
(iii) (a)
(b)
Derive an expression for P 0 ( t ) and
demonstrate that P 1 ( t ) = e
t
10
e
t
5 .
[3]
(iv) Derive an expression for P 2 ( t ) .
(v) Calculate the expected number of claims under the policy.
CT4 A2005
5
%=======================================================%
PLEASE TURN OVER104 Questions
B1
B2
(i) Write down the equation of the Cox proportional hazards model in which the hazard function depends on duration t and a vector of covariates z. You should define all the other terms that you use.
[2]
(ii) Explain why the Cox model is sometimes described as semi-parametric . 

Show that if the force of mortality
x t
=
x t
(0
t
1) is given by
q x
,
1 tq x
this implies that deaths between exact ages x and x + 1 are uniformly distributed.
B3
[4]
An investigation of mortality over the whole age range produced crude estimates of q x for exact ages x from 2 years to 93 years inclusive. The actual deaths at each age were compared with the number of deaths which would have been expected had the
mortality of the lives in the investigation been the same as English Life Table 15 (ELT15). 53 of the deviations were positive and 39 were negative.
Test whether the underlying mortality of the lives in the investigation is represented
by ELT15.
[5]

%%%%%%%%%%%%%%%%%%%%%%%%%%%%%%%%%%%%%%%%%%%%%%%%%%%%%%%%%%%%%%%%%%%%%%%%%%%%%%%%%%%%%%%%%%%%%

104 Part
B1
(i)
If the hazard for life i is ( t ; z i ) , then
l 0 ( t ) exp( b z i T ) ,
l ( t ; z i )
where
(ii)
0 ( t ) is
the baseline hazard,
and is a vector of regression parameters.
The model is semi-parametric because is possible to estimate
from the data without estimating the baseline hazard.
Therefore the baseline hazard can have any shape determined by
the data.
B2
Since
t
t
p x
exp
x s ds
,
0
t
t q x
1
t
p x
1 exp
x s ds
.
0
Substituting for
x s
produces
t
t q x
1 exp
0
q x ds
1 sq x
Performing the integration we have
t q x
t
0
1 exp log(1 sq x )
1 exp log(1 tq x ) log1
1 exp log(1 tq x )
1 exp log(1 tq x )
1 (1 tq x )
tq x .
This is the assumption of a uniform distribution of deaths and implies that deaths
between exact ages x and x + 1 are uniformly distributed.
Page 13Subject CT4
B3
Models
%========================================%
The null hypothesis is that the observed rates are a sample from a population in which English Life Table 15 represents the true rates.
If the null hypothesis is true, then the observed number of positive deviations, P, will be such that P ~ Binomial (92, 1⁄2).
We use the normal approximation to the Binomial distribution because we have > 20 ages
This means that, approximately, P ~ Normal (46, 23).
The z-score associated with the probability of getting 53 positive deviations if the null hypothesis is true is, therefore
53 46
23
7
1.46 .
4.79
We use a two-tailed test, since both an excess of positive and an excess of negative
deviations are of interest.
Using a 5 % significance level, we have -1.96 < 1.46 < +1.96.
(Alternatively, the p-value of the test statistic could be calculated.)
This means we have insufficient evidence to reject the null hypothesis.
\end{document}
