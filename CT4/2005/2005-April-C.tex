\documentclass[a4paper,12pt]{article}

%%%%%%%%%%%%%%%%%%%%%%%%%%%%%%%%%%%%%%%%%%%%%%%%%%%%%%%%%%%%%%%%%%%%%%%%%%%%%%%%%%%%%%%%%%%%%%%%%%%%%%%%%%%%%%%%%%%%%%%%%%%%%%%%%%%%%%%%%%%%%%%%%%%%%%%%%%%%%%%%%%%%%%%%%%%%%%%%%%%%%%%%%%%%%%%%%%%%%%%%%%%%%%%%%%%%%%%%%%%%%%%%%%%%%%%%%%%%%%%%%%%%%%%%%%%%

\usepackage{eurosym}
\usepackage{vmargin}
\usepackage{amsmath}
\usepackage{graphics}
\usepackage{epsfig}
\usepackage{enumerate}
\usepackage{multicol}
\usepackage{subfigure}
\usepackage{fancyhdr}
\usepackage{listings}
\usepackage{framed}
\usepackage{graphicx}
\usepackage{amsmath}
\usepackage{chngpage}

%\usepackage{bigints}
\usepackage{vmargin}

% left top textwidth textheight headheight

% headsep footheight footskip

\setmargins{2.0cm}{2.5cm}{16 cm}{22cm}{0.5cm}{0cm}{1cm}{1cm}

\renewcommand{\baselinestretch}{1.3}

\setcounter{MaxMatrixCols}{10}

\begin{document}
\begin{enumerate}
PLEASE TURN OVERA5
A No-Claims Discount system operated by a motor insurer has the following four
levels:
Level 1:
Level 2:
Level 3:
Level 4:
0% discount
25% discount
40% discount
60% discount
The rules for moving between these levels are as follows:
Following a year with no claims, move to the next higher level, or remain at
level 4.
Following a year with one claim, move to the next lower level, or remain at
level 1.
Following a year with two or more claims, move back two levels, or move to
level 1 (from level 2) or remain at level 1.
For a given policyholder the probability of no claims in a given year is 0.85 and the
probability of making one claim is 0.12.
X(t) denotes the level of the policyholder in year t.
(i)
(a)
(b)
Explain why X(t) is a Markov chain.
Write down the transition matrix of this chain.
[2]
(ii)
Calculate the probability that a policyholder who is currently at level 2 will be
at level 2 after:
(a)
(b)
(c)
one year
two years
three years
[3]
(iii) Explain whether the chain is irreducible and/or aperiodic.
[2]
(iv) Calculate the long-run probability that a policyholder is in discount level 2.
[5]
[Total 12]
CT4 A2005
4A6
An insurance policy covers the repair of a washing machine, and is subject to a
maximum of 3 claims over the year of coverage.
The probability of the machine breaking down has been estimated to follow an
exponential distribution with the following annualised frequencies, :
=
1/10
} 1/5
1/4
If the machine has not suffered any previous breakdown.
If the machine has broken down once previously.
If the machine has broken down on two or more occasions.
As soon as a breakdown occurs an engineer is despatched. It can be assumed that the
repair is made immediately, and that it is always possible to repair the machine.
The washing machine has never broken down at the start of the year (time t = 0).
P i (t) is the probability that the machine has suffered i breakdowns by time t.
%%%%%%%%%%%%%%%%%%%%%%%%%%%%%%%%%%%%%%%%%%%%%%%%%%%%%%%%%%%%%%%%%%%%%%%%%%%%%%%%%%%%%%%%
A5
(i)(a) It is clear that X(t) is a Markov chain; knowing the present state, any
additional information about the past is irrelevant for predicting the next
transition.
(b)
The transition matrix of the process is
0.15 0.85 0
P =
0.15 0
0.85 0
0.03 0.12 0
0
0
0.85
0.03 0.12 0.85
(ii)(a) For the one year transition, p 22 0 ,
as can be seen from above (or is obvious from the statement).
(b)
The possible transitions, and relevant probabilities are:
2
2
Page 8
1
3
2:
2:
0.15 0.85
0.85 0.12
0.1275
0.102Subject CT4
Models
April 2005
Examiners report
The required probability is 0.1275 + 0.102 = 0.2295
Alternatively
The second order transition matrix is
P 2 =
0.15 2 0.85 0.15 0.85 0.15
0.85 2
0.15 2 0.85 0.03 0.85 0.15 0.85 0.12 0
0
0.85 2
0.85 2
0.03 0.15 0.12 0.15 0.85 0.03 2 0.85 0.12 2
0.03 0.15 0.12 0.03 0.12 2 0.85 0.03 0.85 0.03 0.85 0.12 0.12 0.85 0.85 2
=
0.15 0.1275 0.7225 0
0.048 0.2295 0
0.0225 0.051
0.7225
0.204
0.7225
0.0081 0.0399 0.1275 0.8245
Hence the required probability is 0.2295.
(c)
The possible transitions, and relevant probabilities are:
2
2
2
1
3
3
1
1
4
2:
2:
2:
0.15 0.15 0.85
0.85 0.03 0.85
0.85 0.85 0.03
0.019125
0.021675
0.021675
The required probability is
0.019125 + 0.021675 + 0.021675 = 0.062475
Alternatively
The relevant entry from the third-order transition matrix equals
0.15 0.1275 0.85 0.051 0.062475.
(iii)
The chain is irreducible as
any state is reachable from any other.
It is also aperiodic;
If currently at either state 1 or 4, it can remain there. This is not true for states
2 and 3, however these are also aperiodic states since the chain may return e.g.
to state 2 after 2 or 3 transitions.
Page 9Subject CT4
(iv)
Models
April 2005
Examiners report
In matrix form, the equation we need to solve is P = ,
where is the vector of equilibrium probabilities.
This reads
0.15 1
0.85 1
0.15
2
0.03
3
0.12
0.85
3
0.03 4 2 (2)
0.12 4 3 (3)
4 (4)
2
0.85
(1)
1
3
0.85
4
4
i 1 i
Discard the first of these equations and use also that
obtain first from (4) that 0.85
Substituting in (3) this gives
17
0.85 2 0.12
3
3
3
(2) now yields that
0.85 p 1 p 2 0.12 p 3
3
4
or, that
2.65625
4
17
3 /3
2
0.03 p 4
1
p 3
2.65625
so that finally we get
0.15
3
1 . Then, we
0.12 p 3
1
0.17 p 3
0.0865 p 3 ,
0.10173 3 .
Using now that the probabilities must add up to one, we obtain
1
2
3
4 (0.10173 0.3765 1 5.666) 3 1,
or that
3
0.13996.
Solving back for the other variables we get that
1 0.01424,
2 0.05269,
4 0.79311
The long-run probability that the motorist is in discount level 2 is therefore
0.05269.
Page 10Subject CT4
A6
Models
April 2005
Examiners report
(i)
1/10
No
Breakdowns
(ii)
1/5
1/4
One
Breakdown
P 0 ( t )
Two
Breakdowns
Three
Breakdowns
1
P 0 ( t )
10
P 1 ( t ) 1
P 0 ( t )
10
P 2 ( t ) 1
P 1 ( t )
5
1
P 1 ( t )
5
1
P 2 ( t )
4
(iii)(a) Dividing the first equation by P 0 ( t ) :
d
1
ln P 0 ( t )
dt
10
Hence, using the boundary condition P 0 (0) 1
P 0 ( t )
(b)
t
e 10
Substitute into the second equation above to obtain
P 1 ( t )
1
e
10
t
10
1
* P 1 ( t )
5
t
Using an integrating factor e 5 , we get
t
e 5
1
P 1 ( t )
5
P 1 ' t
t
d 5
e
dt
t
5
e
P 1 ( t )
1
e
10
t t
10 5
t
1
e 10
10
P 1 ( t )
t
10
e
P 1 ( t )
e
t
10
const
const e
t
5
Page 11Subject CT4
Models
April 2005
P 1 ( t )
t
10
exp
exp
Examiners report
t
5
using boundary condition P 1 (0)
0
Alternatively
Differentiate the suggested solution and verify it obeys the second equation.
And that the boundary condition is satisfied.
(iv)
Proceeding in a similar way with the equation for P 2 ( t )
P 2 ( t )
1
exp
5
t
10
t
d
exp 4 P 2 ( t )
dt
t
4
exp
P 2 ( t )
(v)
1
exp
5
t
5
1
* P 2 ( t )
4
3
1
t
t
1
(exp 20 exp 20 )
5
3
P 2 ( t )
4
[exp
3
Expected Claims
1
t
t 8
4
exp 20 4 exp 20
3
3
t
10
3 exp
1 P 1 (1)
t
5
t
4 ]
2 exp
2 P 2 (1)
3
P i (1)
i 3
P 1 (1)
P 0 (1) exp 1/10
P 1 (1) exp 1/10
P 2 (1)
4
[exp
3
1
10
2 P 2 (1)
1 P 0 (1)
P 1 (1)
P 2 (1)
0.905
exp
1/ 5
3 exp
Substituting these values gives:
Expected Claims = 0.1049
Page 12
3
0.0861
1
5
2 exp
1
4 ]
0.00832896Subject CT4
Models
