\documentclass[a4paper,12pt]{article}

%%%%%%%%%%%%%%%%%%%%%%%%%%%%%%%%%%%%%%%%%%%%%%%%%%%%%%%%%%%%%%%%%%%%%%%%%%%%%%%%%%%%%%%%%%%%%%%%%%%%%%%%%%%%%%%%%%%%%%%%%%%%%%%%%%%%%%%%%%%%%%%%%%%%%%%%%%%%%%%%%%%%%%%%%%%%%%%%%%%%%%%%%%%%%%%%%%%%%%%%%%%%%%%%%%%%%%%%%%%%%%%%%%%%%%%%%%%%%%%%%%%%%%%%%%%%

\usepackage{eurosym}
\usepackage{vmargin}
\usepackage{amsmath}
\usepackage{graphics}
\usepackage{epsfig}
\usepackage{enumerate}
\usepackage{multicol}
\usepackage{subfigure}
\usepackage{fancyhdr}
\usepackage{listings}
\usepackage{framed}
\usepackage{graphicx}
\usepackage{amsmath}
\usepackage{chngpage}

%\usepackage{bigints}
\usepackage{vmargin}

% left top textwidth textheight headheight

% headsep footheight footskip

\setmargins{2.0cm}{2.5cm}{16 cm}{22cm}{0.5cm}{0cm}{1cm}{1cm}

\renewcommand{\baselinestretch}{1.3}

\setcounter{MaxMatrixCols}{10}

\begin{document}
\begin{enumerate}
1
2
List the factors you would consider when assessing the suitability of an actuarial
model for its purpose.
[4]
A particular baker’s shop in a small town sells only one product: currant buns. These
currant buns are delicious and customers travel many miles to buy them.
Unfortunately, the buns do not keep fresh and cannot be stored overnight.
The baker’s practice is to bake a certain number of buns, K, before the shop opens
each morning, and then during the day to continue baking c buns per hour. He is
concerned that:
•
•
3
he does not run out of buns during the day; and
the number of buns left over at the end of each day is as few as possible
(i) Describe a model which would allow you to estimate the probability that the
baker will run out of buns. State any assumptions you make.
[3]
(ii) Determine the relevant expression for the probability that the baker will run
out of buns, in terms of K, c, and B j , the number of buns bought by the day’s
jth customer.
[1]
[Total 4]
A no-claims discount system has 3 levels of discount: 0%, 25% and 50%. The rules
for moving between discount levels are:
• After a claim-free year, move up to the next higher level or remain at the 50%
discount level.
• After a year with one or more claims, move down to the next lower level or
remain at the 0% discount level.
The long-run probability that a policyholder is in the maximum discount level is 0.75.
Calculate the probability that a given policyholder has a claim-free year, assuming
that this probability is constant.
[5]
CT4 S2007—24
A national mortality investigation was carried out. It was suggested that the mortality
of the male population could be represented by the following graduated rates:
o
μ x + 1 =μ s x + 2 1
2
2
where μ sx is from the standard tables, ELT15(males).
The table below shows the graduated rates for part of the age range, together with the
exposed to risk, expected and actual deaths at each age. The squared standardised
deviations that were calculated are also shown.
⎛
⎞
c o
⎜ θ x − E x ⋅μ x + 12 ⎟
⎠
The standardised deviations were calculated as z x = ⎝
o
E x c ⋅μ x + 1
2
Age Graduated
rates
Exposed
to risk Expected
deaths Deaths Squared
standardised
deviations
x μ x+ 1 E x c E x c ⋅μ x + 1 o θ x z x 2
0.00549
0.00610
0.00679
0.00757
0.00845
0.00945
0.01057
0.01182
0.01323
0.01483 10,850
9,812
10,054
9,650
8,563
10,656
9,667
9,560
8,968
8,455 59.57
59.85
68.27
73.05
72.36
100.70
102.18
113.00
118.65
125.39 52
54
60
65
64
87
88
97
103
105 0.9611
0.5724
1.0010
0.8872
0.9653
1.8637
1.9679
2.2653
2.0634
3.3150
o
2
50
51
52
53
54
55
56
57
58
59
2
(i) Test this graduation for overall goodness-of-fit.
(ii) Comment on your findings in (i).
CT4 S2007—3
[5]
[2]
%%%%%%%%%%%%%%%%%%%%%%%%%%%%%%%%%%%%%%%%%%%%%%%%%%%%%%%%%%%%%%%%%%%%%%%%%%%%%%%%%%%
1
Factors to be considered include:
•
•
•
•
•
•
•
•
•
•
•
•
the objectives of the modelling exercise,
the validity of the model for the purpose to which it is to be put,
the validity of the data to be used,
the possible errors associated with the model or parameters used not being a
perfect fit,
representation of the real world situation being modelled,
the impact of correlations between the random variables that drive the model,
the extent of correlations between the various results produced from the model,
the current relevance of models written and used in the past,
the credibility of the data input,
the credibility of the results output,
the dangers of spurious accuracy,
the ease with which the model and its results can be communicated.
Not all these factors needed to be mentioned for full marks to be awarded.
2
(a)
Assume that, during each day, customers arrive at the shop according to a
Poisson process.
Assume that the numbers of buns bought by each customer, the B j , are
independent and identically distributed random variables.
Then if X t is the total number of buns sold between the beginning of the day
and time t, (where t is measured in hours since the shop opens), X t is a
compound Poisson process defined by
N t
X t = ∑ B j ,
j = 1
where the number of customers arriving between the shop opening and time t
is N t .
(b)
The probability that the baker will run out of buns is
N t
Pr[ K + ct − ∑ B j < 0]
j = 1
for some t.
Page 3Subject CT4 — Models Core Technical — September 2007 — Examiners’ Report
3
The transition matrix for the chain is:
α
⎛ 1 − α
⎞
⎜
⎟
α ⎟ .
⎜ 1 − α
⎜
1 − α α ⎟ ⎠
⎝
To determine the long-run probability, we need to solve the equation π P = π , which
reads:
(I) π 1 = ( 1 − α ) π 1 + ( 1 − α ) π 2
(II) π 2 =
(III) π 3 =
+ ( 1 − α ) π 3
απ 1
απ 2
+
απ 3 .
The probabilities must also satisfy:
(IV)
π 1 + π 2 + π 3 = 1 .
⎛ 1− α ⎞
(III) gives π 2 = ⎜
⎟ π 3 .
⎝ α ⎠
2
⎛ 1− α ⎞
Substituting in (I) gives π 1 = ⎜
⎟ π 3 ,
⎝ α ⎠
⎛ ⎛ 1 − α ⎞ 2 ⎛ 1 − α ⎞ ⎞
and so (IV) leads to ⎜ ⎜
+
+ 1 ⎟ π = 1 .
⎜ ⎝ α ⎟ ⎠ ⎜ ⎝ α ⎟ ⎠ ⎟ 3
⎝
⎠
We know that π 3 = 0.75 , which leads to:
⎛ ( 1 − α ) 2 + α ( 1 − α ) + α 2 ⎞
⎜
⎟ × 0.75 = 1 ,
2
⎜
⎟
α
⎝
⎠
( (
) (
)
)
⇒ 0.75 1 − 2 α + α 2 + α + α 2 + α 2 = α 2 ,
⇒ 0.25 α 2 + 0.75 α − 0.75 = 0 .
Using the quadratic equation formula, this leads to
α=
− 0.75 ± 0.75 2 + 4 × 0.25 × 0.75
.
2 × 0.25
As α > 0 , we must have α = 0.7913 .
Page 4Subject CT4 — Models Core Technical — September 2007 — Examiners’ Report
4
(i)
The null hypothesis is that graduated rates are the same as the true underlying
rates in the population.
To test overall goodness-of-fit we use the chi-squared test.
∑ z x 2 ∼ χ 2 m , where m is the number of degrees of freedom.
x
In this case, we have 10 ages.
The graduation was carried out by reference to a standard table, so we
lose a number of degrees of freedom because of the choice of standard
table.
So, m < 10, and let us say m = 8.
The observed value of the test statistic is
∑ z x 2 = 15.8623
x
The critical value of the chi-squared distribution with 8 degrees of freedom at
the 5 per cent level is 15.51.
Since 15.8623 > 15.51,
we reject the null hypothesis and conclude that the graduated rates do not
adhere to the data.
[Credit was given for using other values of m, say m = 7 or m = 9, provided
candidates recognized that some degrees of freedom should be lost for the choice of
standard table. Note that if m = 9, the null hypothesis will not be rejected.]
(ii)
From the data we can see that the actual deaths are lower than those
expected at all ages.
The graduated rates are too high; the graduation should be revisited.
At these ages the force of mortality increases with age,
so a suitable adjustment may be to reduce the age shift relative to the
standard table from 2 years.
The standardised deviations also appear to show a systematic increase
with age, showing that departure of the graduated rates from the actual
rates increases with age.
There appear to be no outliers (all the z x s have absolute values below
1.96).
Page 5Subject CT4 — Models Core Technical — September 2007 — Examiners’ Report
