\documentclass[a4paper,12pt]{article}

%%%%%%%%%%%%%%%%%%%%%%%%%%%%%%%%%%%%%%%%%%%%%%%%%%%%%%%%%%%%%%%%%%%%%%%%%%%%%%%%%%%%%%%%%%%%%%%%%%%%%%%%%%%%%%%%%%%%%%%%%%%%%%%%%%%%%%%%%%%%%%%%%%%%%%%%%%%%%%%%%%%%%%%%%%%%%%%%%%%%%%%%%%%%%%%%%%%%%%%%%%%%%%%%%%%%%%%%%%%%%%%%%%%%%%%%%%%%%%%%%%%%%%%%%%%%

\usepackage{eurosym}
\usepackage{vmargin}
\usepackage{amsmath}
\usepackage{graphics}
\usepackage{epsfig}
\usepackage{enumerate}
\usepackage{multicol}
\usepackage{subfigure}
\usepackage{fancyhdr}
\usepackage{listings}
\usepackage{framed}
\usepackage{graphicx}
\usepackage{amsmath}
\usepackage{chngpage}

%\usepackage{bigints}
\usepackage{vmargin}

% left top textwidth textheight headheight

% headsep footheight footskip

\setmargins{2.0cm}{2.5cm}{16 cm}{22cm}{0.5cm}{0cm}{1cm}{1cm}

\renewcommand{\baselinestretch}{1.3}

\setcounter{MaxMatrixCols}{10}

\begin{document}



11
Consider the random variable defined by $$ X_n =
\sum_{i=1}Y_i $ with each $Y_i$ mutually

independent with probability:
\[P[Y_i = 1] = p, P[Y_i = -1] = 1- p\]
0 < p < 1
\begin{enumerate}[(a)]
\item (i) Write down the state space and transition graph of the sequence X_n .
\item (ii) State, with reasons, whether the process:
\begin{enumerate}[(a)]
\item is aperiodic.
\item is reducible.
\item admits a stationary distribution.
\end{enumerate}

Consider $j > i > 0$.
\item (iii) Derive an expression for the number of upward movements in the sequence X_n
between t and (t + m) if X t = i and X t+m = j.

\item (iv) Derive expressions for the m-step transition probabilities $p_{ij} ^{( m )}$ .
\item (v) Show how the one-step transition probabilities would alter if X_n was restricted
to non-negative numbers by introducing:
\begin{enumerate}[(a)]
\item a reflecting boundary at zero.
\item an absorbing boundary at zero.
\end{enumerate}
\item (vi)
For each of the examples in part (v), explain whether the transition
probabilities $p_{ij} ^{( m )}$ would increase, decrease or stay the same.
(Calculation of the transition probabilities is not required.)
\end{enumerate}
%%%%%%%%%%%%%%%%%%%%%%%%%%%%%%%%%%%%%%%%%%%%%%%%%%%%%%%%%%%%%%%%%%%%%%%%%%%5555
%% Page 14 — Models Core Technical — September 2008 — Examiners’ Report
\newpage 
%%--- Question 11
\begin{itemize}
\item (i)
State space is the set of integers Ζ .
Transition graph:
p
-2
-1
1- p
\item (ii)
(a)
p
p
0
1- p
p
1
1- p
2
1- p
\item The process is not aperiodic
because it has period 2:
for example, starting from an even number the process is only even after an even number of steps
(b)
\item The process is irreducible
as the probabilities of X_n increasing and decreasing by 1 are both non-zero so any state can be reached.
(c)
\item (iii)
No stationary distribution will exist because the state space is infinite.
%%- Part iii

\begin{itemize}
\item Suppose there are u upward movements.
\item Then there must be $m − u$ downward movements,
and $u – ( m – u ) = j – i$
\item So 
\[ u = \frac{m + j − i}{2}. \]
\end{itemize}

%%- part iv

\begin{itemize}
\item The maximum number of upward steps is m so the
transition probability is zero if $j – i > m$ .
\item 
As the chain is periodic with period 2, it can only occupy
state j after m steps if $m + j − i$ is even.
\item If $m + j − i$ is even and $j – i \leq m$ then there must be $u$
upward jumps and $( m − u )$ downward jumps.

\item These can be ordered in ${ \displaystyle {m \choose u} }$ ways.
\item So the transition probabilities are:
\[
p_{ij}^{( m )}  = \begin{cases}
{m \choose u} p^{u}(1-p)^{m-u}\\ if j − i \leq m \mbox{ and }m + j − i \mbox{ even }
0 & otherwise\\
\end{cases}
\]
\end{itemize}

EITHER
\item In both cases the transition probabilities
are unaltered unless X_i = 0.
\item (a) Reflecting boundarY_implies
\[P[X_i+1 = 1│X_i = 0] = 1 (or p 01(1) = 1)\]
\item (b) Absorbing boundarY_implies
P[X_i+1 = 0│X_i = 0] = 1 (or p 00(1) = 1)
OR
A matrix solution for the transition probabilities is acceptable
Reflecting:
1
0
0
⎛ 0
⎜
p
0
0
⎜ 1 − p
⎜ 0
p
1 − p
0
⎜
0
1 − p
0
⎜ 0
⎜ 0
0
0
1 − p
⎜ ⎜
:
:
:
⎝ :
0
0
0
p
0
: ... ⎞
⎟
... ⎟
... ⎟
⎟
... ⎟
... ⎟
⎟ ⎟
⎠
0
0
0
p
0
: ... ⎞
⎟
... ⎟
... ⎟
⎟
... ⎟
... ⎟
⎟ ⎟
⎠
Absorbing:
0
0
0
⎛ 1
⎜
p
0
0
⎜ 1 − p
⎜ 0
p
1 − p
0
⎜
0
1 − p
0
⎜ 0
⎜ 0
0
0
1 − p
⎜ ⎜
:
:
:
⎝ :
OR
\item A diagrammatic solution is also acceptable:
%%--- Page 16 — Models Core Technical — September 2008 — Examiners’ Report
Reflecting
1
0
p
1
1-p
2
1-p
Absorbing:
p
1
0
1
1-p
(vi)
2
1-p
\item In both cases the zero transition probabilities remain
zero as the period is still 2 where relevant.
\item If i is sufficiently above 0 then conditions at zero
will not be relevant and all the m-step transition
probabilities will remain the same. (This applies if m < i.)
Otherwise
\item In (a) some sample paths which would have
taken X below zero will be reflected, increasing the
probability of reaching j at step m.
So the m-step transition probabilities would increase.
\item In (b) any sample path which reaches zero would
no longer be able to access state j
so the transition probabilities would decrease.
%%---- Page 17 — Models Core Technical — September 2008 — Examiners’ Report
\end{itemize}
\end{document}
