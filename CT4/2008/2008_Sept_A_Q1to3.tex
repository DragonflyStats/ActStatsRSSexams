\documentclass[a4paper,12pt]{article}

%%%%%%%%%%%%%%%%%%%%%%%%%%%%%%%%%%%%%%%%%%%%%%%%%%%%%%%%%%%%%%%%%%%%%%%%%%%%%%%%%%%%%%%%%%%%%%%%%%%%%%%%%%%%%%%%%%%%%%%%%%%%%%%%%%%%%%%%%%%%%%%%%%%%%%%%%%%%%%%%%%%%%%%%%%%%%%%%%%%%%%%%%%%%%%%%%%%%%%%%%%%%%%%%%%%%%%%%%%%%%%%%%%%%%%%%%%%%%%%%%%%%%%%%%%%%

\usepackage{eurosym}
\usepackage{vmargin}
\usepackage{amsmath}
\usepackage{graphics}
\usepackage{epsfig}
\usepackage{enumerate}
\usepackage{multicol}
\usepackage{subfigure}
\usepackage{fancyhdr}
\usepackage{listings}
\usepackage{framed}
\usepackage{graphicx}
\usepackage{amsmath}
\usepackage{chngpage}

%\usepackage{bigints}
\usepackage{vmargin}

% left top textwidth textheight headheight

% headsep footheight footskip

\setmargins{2.0cm}{2.5cm}{16 cm}{22cm}{0.5cm}{0cm}{1cm}{1cm}

\renewcommand{\baselinestretch}{1.3}

\setcounter{MaxMatrixCols}{10}

\begin{document}
\begin{enumerate}
%%-- Question 1
You work for a consultancy which has created an actuarial model and is now
preparing documentation for the client.
List the key items you would include in the documentation on the model.


%%-- Question 2

The classification of stochastic models according to:
\begin{itemize}
\item
discrete or continuous time variable
\item discrete or continuous state space
\end{itemize}
gives rise to a four-way classification.
Give four examples, one of each type, of stochastic models which may be used to
model observed processes, and suggest a practical problem to which each model may
be applied.
%%%
%%- Quesiton 3
Compare the advantages and disadvantages of the Binomial and the multiple-state
models in the following situations:
(a)
(b)
analysing human mortality without distinguishing between causes of death
analysing human mortality when distinguishing between causes of death

%%%%%%%%%%%%%%%%%%%%%%%%%%%%%%%%%%%%%%%%%%%%%%%%%%%%%%%%%%%%%%%%%%%%%%%%%%%%%
\newpage
Solution to Question 1
Instructions on how to run the model
Tests performed to validate the output of the model.
Definition of input data.
Any limitations of the model identified (e.g. potential unreliability).
Basis on which the form of the model chosen (e.g. deterministic or stochastic)
References to any research papers or discussions with appropriate experts.
Summary of model results.
Name and professional qualification.
Purpose or objectives of the model.
Assumptions underlying the model.
How the model might be adapted or extended.
%%%%%%%%%%%%%%%%%%%%%%%%%%%%%%%%%%%%%%%%%%%%
\newpage
Solution to Question 2
Discrete time, discrete state space
Counting process, random walk, Markov chain
No claims bonus in motor insurance.
Continuous time, discrete state space
Counting process, Poisson process, Markov jump process
Healthy-sick-dead model in sickness insurance
Discrete time, continuous state space
General random walk, ARIMA time series model, moving average model
Share price at end of each day
Continuous time, continuous state space
Compound Poisson process, Brownian motion, Ito process, white noise
Value of claims reaching an insurance company monitored
continuously
4Subject CT4 — %%%%%%%%%%%%%%%%%%%%%%%%%%%%%%%%%%%%5 — September 2008 — Examiners’ Report
\newpage
Solution to Question 3
(a)
Both models produce consistent and unbiased estimators.
The estimate of q x made using the Binomial model
will have a higher variance than that made using the
multiple-state model, though the difference is tiny if the forces of mortality are small.
If data on exact ages at entry into and exit from observation are available, the multiple state model is
simpler to apply. The Binomial model requires further
assumptions (e.g. uniform distribution of deaths). The Binomial model also does not use all the information
available if exact ages at entry into and exit from observation are available.
However, if the forces of mortality are small, both models will give very similar results.
(b)
The multiple state model can simply be extended
The estimators have the same form and the same statistical properties as in the classic life table.
The Binomial model is hard to extend to several causes of death. Although the life table as a computational tool can be
extended, the calculations are more complex and awkward than those in the multiple-state model.
\end{document}
