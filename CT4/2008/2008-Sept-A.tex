\documentclass[a4paper,12pt]{article}

%%%%%%%%%%%%%%%%%%%%%%%%%%%%%%%%%%%%%%%%%%%%%%%%%%%%%%%%%%%%%%%%%%%%%%%%%%%%%%%%%%%%%%%%%%%%%%%%%%%%%%%%%%%%%%%%%%%%%%%%%%%%%%%%%%%%%%%%%%%%%%%%%%%%%%%%%%%%%%%%%%%%%%%%%%%%%%%%%%%%%%%%%%%%%%%%%%%%%%%%%%%%%%%%%%%%%%%%%%%%%%%%%%%%%%%%%%%%%%%%%%%%%%%%%%%%

\usepackage{eurosym}
\usepackage{vmargin}
\usepackage{amsmath}
\usepackage{graphics}
\usepackage{epsfig}
\usepackage{enumerate}
\usepackage{multicol}
\usepackage{subfigure}
\usepackage{fancyhdr}
\usepackage{listings}
\usepackage{framed}
\usepackage{graphicx}
\usepackage{amsmath}
\usepackage{chngpage}

%\usepackage{bigints}
\usepackage{vmargin}

% left top textwidth textheight headheight

% headsep footheight footskip

\setmargins{2.0cm}{2.5cm}{16 cm}{22cm}{0.5cm}{0cm}{1cm}{1cm}

\renewcommand{\baselinestretch}{1.3}

\setcounter{MaxMatrixCols}{10}

\begin{document}
\begin{enumerate}
© Institute of Actuaries1
You work for a consultancy which has created an actuarial model and is now
preparing documentation for the client.
List the key items you would include in the documentation on the model.
2
[4]
The classification of stochastic models according to:
•
•
discrete or continuous time variable
discrete or continuous state space
gives rise to a four-way classification.
Give four examples, one of each type, of stochastic models which may be used to
model observed processes, and suggest a practical problem to which each model may
be applied.
[4]
3
Compare the advantages and disadvantages of the Binomial and the multiple-state
models in the following situations:
(a)
(b)
analysing human mortality without distinguishing between causes of death
analysing human mortality when distinguishing between causes of death
[5]
4
In the village of Selborne in southern England in the year 1637 the number of babies
born each month was as follows
January
February
March
April
May
June
2
1
1
2
1
2
July
August
September
October
November
December
5
1
0
2
0
3
Data show that over the 20 years before 1637 there was an average of 1.5 births per
month. You may assume that births in the village historically follow a Poisson
process.
An historian has suggested that the large number of births in July 1637 is unusual.
(i) Carry out a test of the historian’s suggestion, stating your conclusion.
(ii) Comment on the assumption that births follow a Poisson process.
CT4 S2008—2
[4]
[1]
%%%%%%%%%%%%%%%%%%%%%%%%%%%%%%%%%%%%%%%%%%%%%%%%%%%%%%%%%%%%%%%%%%%%%%%%%%%%%
1
Instructions on how to run the model
Tests performed to validate the output of the model.
Definition of input data.
Any limitations of the model identified (e.g. potential unreliability).
Basis on which the form of the model chosen (e.g. deterministic or stochastic)
References to any research papers or discussions with appropriate experts.
Summary of model results.
Name and professional qualification.
Purpose or objectives of the model.
Assumptions underlying the model.
How the model might be adapted or extended.
2
Discrete time, discrete state space
Counting process, random walk, Markov chain
No claims bonus in motor insurance.
Continuous time, discrete state space
Counting process, Poisson process, Markov jump process
Healthy-sick-dead model in sickness insurance
Discrete time, continuous state space
General random walk, ARIMA time series model, moving average model
Share price at end of each day
Continuous time, continuous state space
Compound Poisson process, Brownian motion, Ito process, white noise
Value of claims reaching an insurance company monitored
continuously
Page 4Subject CT4 — Models Core Technical — September 2008 — Examiners’ Report
3
(a)
Both models produce consistent and unbiased estimators.
The estimate of q x made using the Binomial model
will have a higher variance than that made using the
multiple-state model, though the difference is tiny
if the forces of mortality are small.
If data on exact ages at entry into and exit from
observation are available, the multiple state model is
simpler to apply. The Binomial model requires further
assumptions (e.g. uniform distribution of deaths).
The Binomial model also does not use all the information
available if exact ages at entry into and exit from
observation are available.
However, if the forces of mortality are small, both
models will give very similar results.
(b)
The multiple state model can simply be extended
The estimators have the same form and the same statistical
properties as in the classic life table.
The Binomial model is hard to extend to several causes of
death. Although the life table as a computational tool can be
extended, the calculations are more complex and awkward than
those in the multiple-state model.
4
(i)
Suppose that the number of births each month, B , is the outcome of a Poisson
process with a rate λ = 1.5.
The probability of obtaining b births per month
exp( − 1.5).1.5 b
is given by the formula Pr[ B = b ] =
b !
Therefore we have
b Pr[ B = b ]
0
1
2
3
4
5
6+ 0.223
0.335
0.251
0.126
0.047
0.014
0.004
Page 5Subject CT4 — Models Core Technical — September 2008 — Examiners’ Report
Therefore, if the number of births per month is the
outcome of a Poisson process with a rate of 1.5 per
month the probability of obtaining 5 or more births in
a single month is 0.014 + 0.004 = 0.018.
EITHER This is very small OR this is < 0.05
which suggests that the historian may be correct
to suspect something unusual about July 1637.
But only July has a number of births more than 5, and at the 5% level of
statistical significance we expect 1 month in 20 to have such a large
number, then unless we have a prior expectation that July is unusual, we
should be cautious before accepting the historian’s suggestion.
(ii)
The assumption that births follow a Poisson process is
unlikely to be entirely realistic
EITHER because of the occurrence of multiple births
(twins and triplets)
OR because births tend to occur seasonally
OR because the process might be time inhomogeneous.
