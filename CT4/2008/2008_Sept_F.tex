\documentclass[a4paper,12pt]{article}

%%%%%%%%%%%%%%%%%%%%%%%%%%%%%%%%%%%%%%%%%%%%%%%%%%%%%%%%%%%%%%%%%%%%%%%%%%%%%%%%%%%%%%%%%%%%%%%%%%%%%%%%%%%%%%%%%%%%%%%%%%%%%%%%%%%%%%%%%%%%%%%%%%%%%%%%%%%%%%%%%%%%%%%%%%%%%%%%%%%%%%%%%%%%%%%%%%%%%%%%%%%%%%%%%%%%%%%%%%%%%%%%%%%%%%%%%%%%%%%%%%%%%%%%%%%%

\usepackage{eurosym}
\usepackage{vmargin}
\usepackage{amsmath}
\usepackage{graphics}
\usepackage{epsfig}
\usepackage{enumerate}
\usepackage{multicol}
\usepackage{subfigure}
\usepackage{fancyhdr}
\usepackage{listings}
\usepackage{framed}
\usepackage{graphicx}
\usepackage{amsmath}
\usepackage{chngpage}

%\usepackage{bigints}
\usepackage{vmargin}

% left top textwidth textheight headheight

% headsep footheight footskip

\setmargins{2.0cm}{2.5cm}{16 cm}{22cm}{0.5cm}{0cm}{1cm}{1cm}

\renewcommand{\baselinestretch}{1.3}

\setcounter{MaxMatrixCols}{10}

\begin{document}
\begin{enumerate}
12
(i) Explain the meaning of the rates of mortality usually denoted q x and m x , and the relationship between them.
[3]
(ii) Write down a formula for t q x , $0 \leq t \leq 1$ , under each of the following assumptions about the distribution of deaths in the age range $[x, x+1]$:
(a)
(b)
(c)
uniform distribution of deaths
constant force of mortality
the Balducci assumption
[2]
%%%%%%%%%%%%%%
A group of animals experiences a mortality rate q x = 0.1.
(iii) Calculate m x under each of the assumptions (a) to (c) above.
(iv) Comment on your results in part (iii).
END OF PAPER
CT4 S2008—9
[8]
[3]
\newpage
%%%%%%%%%%%%%%%%%%%%%%%%%%%%%%%%%%%%%%%%%%%%%%%%%%%%%%%%%%%%%%%%%%%%%%%%%
12
(i)
q x is the probability that a life aged exactly x will die before reaching exact age x+1, and is called the initial rate of mortality.
m x is called the central rate of mortality and represents the probability that a life alive between the ages of $x$ and $x+1$ dies
They are related by:
m x =
q x
1
∫ t p x dt
0
(ii)
(a)
Uniform distribution of deaths (UDD)
t q x
(b)
Constant force of mortality (CFM)
t q x
(c)
= t * q x
= 1 − e −μ * t
Balducci assumption
1 − t q x + t
(iii)
(a)
= (1 − t ) * q x
UDD
1
⎡ t 2 ⎤
⎢ ⎥ = 0.95
(1
0.1
)
1
0.1
p
dt
=
−
t
dt
=
−
∫ t x ∫
⎢ ⎣ 2 ⎥ ⎦
0
0
0
1
1
(or other reasoning why exposure is 0.95
under UDD)
m x = 0.1/0.95 = 0.105263
(b)
CFM
μ given by:
1 − e −μ = 0.1
μ = − ln 0.9 = 0.1053605
%%------- Page 18Subject CT4 — Models Core Technical — September 2008 — Examiners’ Report
EITHER
If force of mortality constant over [x, x+1] then
central rate must be equal to the force μ
so m x = 0.1053605
OR
1 1
0 0
∫ t p x dt = ∫ (1 − (1 − e
−μ t
1
1
1
−μ t
⎤
)) dt = − ⎡ e
= (1 − e −μ ) = 0.949122
μ ⎣
⎦ 0 μ
m x = 0.1/0.949122=0.1053605
(c)
Balducci
For consistency, observe that 1 p x = t p x . 1 − t p x + t
So
p x =
t
1 p x
1 − t
p x + t
1 1
0 0
=
0.9
0.9
=
1 − 1 − t q x + t 0.9 + 0.1 t
0.9
0.9
∫ t p x dt = ∫ 0.9 + 0.1 t dt = 0.1 [ ln(0.9 + 0.1 t ) ] 0 = − 9 ln 0.9 = 0.9482446
1
So m x = 0.1/0.9482446=0.1054580
(iv)
The Balducci assumption implies a decreasing mortality rate over [x, x+1] and UDD
an increasing mortality rate. CFM is obviously constant For a given number of deaths over the period,
the estimated exposure would be highest if we assumed an increasing mortality rate.
We would expect the central rate to be highest for that with the lowest estimate exposure, hence
Balducci > CFM > UDD is the expected order.
\end{document}
