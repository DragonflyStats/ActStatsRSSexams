\documentclass[a4paper,12pt]{article}

%%%%%%%%%%%%%%%%%%%%%%%%%%%%%%%%%%%%%%%%%%%%%%%%%%%%%%%%%%%%%%%%%%%%%%%%%%%%%%%%%%%%%%%%%%%%%%%%%%%%%%%%%%%%%%%%%%%%%%%%%%%%%%%%%%%%%%%%%%%%%%%%%%%%%%%%%%%%%%%%%%%%%%%%%%%%%%%%%%%%%%%%%%%%%%%%%%%%%%%%%%%%%%%%%%%%%%%%%%%%%%%%%%%%%%%%%%%%%%%%%%%%%%%%%%%%

\usepackage{eurosym}
\usepackage{vmargin}
\usepackage{amsmath}
\usepackage{graphics}
\usepackage{epsfig}
\usepackage{enumerate}
\usepackage{multicol}
\usepackage{subfigure}
\usepackage{fancyhdr}
\usepackage{listings}
\usepackage{framed}
\usepackage{graphicx}
\usepackage{amsmath}
\usepackage{chngpage}

%\usepackage{bigints}
\usepackage{vmargin}

% left top textwidth textheight headheight

% headsep footheight footskip

\setmargins{2.0cm}{2.5cm}{16 cm}{22cm}{0.5cm}{0cm}{1cm}{1cm}

\renewcommand{\baselinestretch}{1.3}

\setcounter{MaxMatrixCols}{10}

\begin{document}
\begin{enumerate}
CT4 S2008—610
In an investigation of reconviction rates among those who have served prison
sentences, let X be a random variable which measures the duration from the date of
release from prison until the ex-prisoner is convicted of a subsequent offence. The
investigation monitored a sample of 100 ex-prisoners (who were all released on the
same date) at one-monthly intervals from their date of release for a period of 6
months. Those who could not be traced in any month were removed from the sample
at that point and not traced in subsequent months. Reconviction was assumed to take
place at the duration that a prisoner was first known to have been reconvicted.
(i)
Express the hazard rate at duration x months in terms of probabilities.
[1]
The investigation produced the following data for a sample of 100 ex-prisoners.
Months since release
1
2
3
4
5
6
(ii)
Number of prisoners
contacted
Number who had
been reconvicted
since last contact
100
97
95
90
85
80
0
0
4
3
5
0
Calculate the Nelson-Aalen estimate of the survival function.
[5]
A previous investigation found that the probability that a prisoner would be
reconvicted within 6 months of release was 0.2.
(iii)
Estimate confidence intervals around the integrated hazard using the results
from part (ii) to test the hypothesis that the rate of reconviction has declined
since the previous investigation.
[6]
[Total 12]
CT4 S2008—7
PLEASE TURN OVERn
%%%%%%%%%%%%%%%%%%%%%%%%%%%%%%%%%%%%%%%%%%%%%%%%%%%%%%%%%%%%%%%%%%%%%%%%%%%%%%%
10
(i)
∂ t t 35
13
p 13
x = −μ x + t . t p x
∂ t t 45
14
p 14
x = −μ x + t . t p x
∂ t t 35
13
45
14
p 15
x = μ x + t . t p x + μ x + t . t p x
EITHER
The hazard rate at duration x is given by
Pr[ X ≤ x + dt | X > x ]
.
dt → 0
dt
lim
OR
In discrete time, the hazard rate at duration x is given by, Pr[ X = x | X ≥ x ] .
OR
The hazard rate at duration x is given by h ( x ) = −
1 d
[ S ( x )] ,
S ( x ) dx
where S(x) is the survival function defined as Pr[X > x].
(ii)
The integrated hazard, Λ x , is estimated as follows:
x j
n j
d j
c j
0 100 0 0
1
2
3
4
5
6 100
97
95
90
85
80 0
0
4
3
5
0 3
2
1
2
0
80
d j
n j
Λ x =
d j
∑ n j
x j ≤ x
0 0
0
0
4/95 = 0.0421
3/90 = 0.0333
5/85 = 0.0588
0 0
0
0.0421
0.0754
0.1343
0.1343
Page 12Subject CT4 — Models Core Technical — September 2008 — Examiners’ Report
The survival function S ( x ) is given by exp( − Λ x ), so that we have
(iii)
x S ( x )
0 ≤ x < 3
3 ≤ x < 4
4 ≤ x < 5
5 ≤ x 1.0000
0.9588
0.9274
0.8744
Confidence intervals around the integrated hazard may
be estimated using the formula
~
Var [ Λ x ] =
∑
d j ( n j − d j )
n 3 j
x j ≤ x
Applying this to the data gives
x j
0
1
2
3
4
5
6
n j
100
100
97
95
90
85
80
d j
0
0
0
4
3
5
0
d j ( n j − d j )
∑
n 3 j
0
0
0
0.000425
0.000358
0.000651
0
d j ( n j − d j )
x j ≤ x
n 3 j
0
0
0
0.000425
0.000783
0.001434
0.001434
95 per cent confidence intervals around the integrated
hazard at duration 6 can therefore be computed as
^
^
Λ 6 ± 1.96 var Λ 6
= 0.1343 ± 1.96 0.001434
= (0.1343 – 0.0742, 0.1343 + 0.0742)
= (0.0601, 0.2085).
Page 13Subject CT4 — Models Core Technical — September 2008 — Examiners’ Report
THEN EITHER
^
The estimated survival function, S ( x ) is given
^
by exp( − Λ x ) ,
^
so that the 95 per cent confidence interval for S ( x ) is
[exp( −0.0601), exp(−0.2085)]
which is (0.9417, 0.8118).
In the previous investigation the probability that a
prisoner would not be reconvicted within 6 months
of release was 1 – 0.2 = 0.8.
^
Since the 95 per cent confidence interval around S ( x ) in the current
^
investigation does not include the value 0.8, and our estimate of S ( x ) > 0.8
we conclude that the rate of reconviction has declined since the previous
investigation.
OR
In the previous investigation the probability that a
prisoner would not be reconvicted within 6 months
of release was 1 – 0.2 = 0.8 – i.e. S (6) = 0.8
Since S ( x ) = exp(−Λ x ), the value of Λ 6 corresponding to S (6) = 0.8 is
Λ 6 = −log e (0.8) = 0.2231.
Since this is higher than the upper limit in the range (0.0601, 0.2085) we
conclude that the rate of reconviction has declined since the previous
investigation.
