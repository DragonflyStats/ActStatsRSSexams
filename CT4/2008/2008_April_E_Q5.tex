\documentclass[a4paper,12pt]{article}

%%%%%%%%%%%%%%%%%%%%%%%%%%%%%%%%%%%%%%%%%%%%%%%%%%%%%%%%%%%%%%%%%%%%%%%%%%%%%%%%%%%%%%%%%%%%%%%%%%%%%%%%%%%%%%%%%%%%%%%%%%%%%%%%%%%%%%%%%%%%%%%%%%%%%%%%%%%%%%%%%%%%%%%%%%%%%%%%%%%%%%%%%%%%%%%%%%%%%%%%%%%%%%%%%%%%%%%%%%%%%%%%%%%%%%%%%%%%%%%%%%%%%%%%%%%%

\usepackage{eurosym}
\usepackage{vmargin}
\usepackage{amsmath}
\usepackage{graphics}
\usepackage{epsfig}
\usepackage{enumerate}
\usepackage{multicol}
\usepackage{subfigure}
\usepackage{fancyhdr}
\usepackage{listings}
\usepackage{framed}
\usepackage{graphicx}
\usepackage{amsmath}
\usepackage{chngpage}

%\usepackage{bigints}
\usepackage{vmargin}

% left top textwidth textheight headheight

% headsep footheight footskip

\setmargins{2.0cm}{2.5cm}{16 cm}{22cm}{0.5cm}{0cm}{1cm}{1cm}

\renewcommand{\baselinestretch}{1.3}

\setcounter{MaxMatrixCols}{10}

\begin{document}

A survey of first marriage patterns among women in a remote population in central
Asia collected the following data for a sample of women:
•
•
[6]
calendar year of birth
calendar year of first marriage
Data are also available about the population of never-married women on 1 January
each year, classified by age last birthday.
You have been asked to estimate the intensity, λ x , of first marriage for women
aged x.
(i) State the rate interval implied by the first marriages data. [1]
(ii) Derive an appropriate exposed to risk which corresponds to the first
marriages data. State any assumptions that you make. [4]
(iii)
Explain to what age x your estimate of λ x applies. State any assumptions
that you make.


5
(i) Calendar year rate interval starting on 1 January each
year.
(ii) The first marriages data may be described as
m x = number of first marriages, age x on the birthday in the
calendar year of marriage, during a defined period of investigation of
length N years
A definition of the population data which is compatible with these data on first
marriages is
P x,t = number of lives under observation at time t since the start of the
investigation who were aged x next birthday on the 1 January
immediately preceding t
Since we follow each cohort of lives through each calendar year, this exposed
to risk is
N
E x c
=
∫ P x , t dt
0
which may be approximated as
E x c =
N − 1
∑ 2 ( P x , t + P x + 1, t + 1 )
1
0
(where the summation considers just integer values of t).
This assumes that the population varies linearly across the
calendar year.
However, we have data classified by age last birthday
so we need to make a further adjustment.
If the number of lives aged x last birthday on 1 January
in year t is P x,t * then
P x,t = P x-1,t *
and an appropriate exposed to risk in terms of the data we
have is
E x c =
K + N
∑
t = K
1
( P x − 1, t * + P x , t + 1 * ) .
2
Page 7Subject CT4 — Models Core Technical — April 2008 — Examiners’ Report
(iii)
The age range at the start of the rate interval is (x–1, x)
exact.
So, assuming that birthdays are uniformly distributed
across the calendar year the average age at the start of the rate interval is
x–1⁄2 and the average age in the middle of the rate interval
is x.
Therefore the estimate of λ x applies to age x.
