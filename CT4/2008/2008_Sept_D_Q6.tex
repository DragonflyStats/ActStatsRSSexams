\documentclass[a4paper,12pt]{article}

%%%%%%%%%%%%%%%%%%%%%%%%%%%%%%%%%%%%%%%%%%%%%%%%%%%%%%%%%%%%%%%%%%%%%%%%%%%%%%%%%%%%%%%%%%%%%%%%%%%%%%%%%%%%%%%%%%%%%%%%%%%%%%%%%%%%%%%%%%%%%%%%%%%%%%%%%%%%%%%%%%%%%%%%%%%%%%%%%%%%%%%%%%%%%%%%%%%%%%%%%%%%%%%%%%%%%%%%%%%%%%%%%%%%%%%%%%%%%%%%%%%%%%%%%%%%

\usepackage{eurosym}
\usepackage{vmargin}
\usepackage{amsmath}
\usepackage{graphics}
\usepackage{epsfig}
\usepackage{enumerate}
\usepackage{multicol}
\usepackage{subfigure}
\usepackage{fancyhdr}
\usepackage{listings}
\usepackage{framed}
\usepackage{graphicx}
\usepackage{amsmath}
\usepackage{chngpage}

%\usepackage{bigints}
\usepackage{vmargin}

% left top textwidth textheight headheight

% headsep footheight footskip

\setmargins{2.0cm}{2.5cm}{16 cm}{22cm}{0.5cm}{0cm}{1cm}{1cm}

\renewcommand{\baselinestretch}{1.3}

\setcounter{MaxMatrixCols}{10}

\begin{document}

%% [5]
%% PLEASE TURN OVER
%% - Question  6
A portfolio of term assurance policies was transferred from insurer A to insurer B on 1 January 2001. Each policy in the portfolio was written with premiums payable annually in advance. Insurer B wishes to investigate the mortality experience of its
ac\hat{q}uired portfolio and has collected the following data over the period 1 January 2001
to 1 January 2005:
d x numbers of deaths aged x
P x,t number of policies in force aged x at time t (t = 0, 1, 2, 3, 4 years measured
from 1 January 2001)
Where x is defined as:
age last birthday at the most recent policy anniversary prior to the portfolio
transfer + number of premiums received by insurer B.
(i)
(ii)
(a)
(b)
State the rate interval implied by the above data.
Write down the range of ages at the start of the rate interval.

Give an expression which can be used to estimate the initial exposed to risk at age x, E x , stating any assumptions made.

The following is an extract from the data collected in the investigation:
x d x \sum P x , t \sum P x , t + 1
39
40
41 28
36
33 10,536
10,965
10,421 11,005
10,745
10,577
where the summations are from t = 0 to t = 3.
7
(iii) Estimate $\hat{q}_{40}$ , stating any further assumptions made.

%%%%%%%%%%%%%%%%%%%%%%%%%%%%%%%%%%%%%%%%%%%%%%%%%%%%%%%%%%%%%%%%%%%%%%%%%%%%%%
\newpage
\begin{itemize}
\item 
Age label changes on the receipt of the
premium on the policy anniversary so this is a
policy year rate interval.
Policyholders’ ages range from x to x +1
at start of the rate interval.
\item (ii)
Central exposed to risk
E x c
4
1 3
= \int P x , t dt \approx \sum ( P x , t + P x , t + 1 )
2 t = 0
t = 0
Approximation assumes population changes linearly over each year
during the period of investigation.
1 3
1
( P x , t + P x , t + 1 ) + d x ,
\sum
2 t = 0
2
assuming deaths are uniform over the rate interval OR deaths occur on
average half way through the rate interval.
Initial exposed to risk E x \approx
(but NOT deaths are uniform over the “year”, or occur on average half
way through the “year”)
\item (iii)
d x
estimates \hat{q} x for the average age
E x
at the start of the rate interval.
\hat{q} x =
Assuming birthdays are uniformly distributed
across policy years,
the average age at the start of the rate interval
is x+1⁄2, so we re\hat{q}uire \hat{q}̂_{39} 1 to estimate \hat{q}_40}$ .
2
1
\item Assuming $\hat{q}_{39} 1 = [ \hat{q}_{39} + \hat{q}_{40} ]$ we have
2
2
\hat{q}_{39} =
\hat{q}_{40} =
28
= 0.002596
1
1
(10536 + 11005) + * 28
2
2
36
1
1
(10965 + 10745) + *36
2
2
= 0.003311
and hence our estimate of \hat{q}_{40} is 0.5[0.002596 + 0.003311) = 0.002954.
%%--Page 7Subject CT4 — Models Core Technical — September 2008 — Examiners’ Report
\end{itemize}
\end{enumerate}
