\documentclass[a4paper,12pt]{article}

%%%%%%%%%%%%%%%%%%%%%%%%%%%%%%%%%%%%%%%%%%%%%%%%%%%%%%%%%%%%%%%%%%%%%%%%%%%%%%%%%%%%%%%%%%%%%%%%%%%%%%%%%%%%%%%%%%%%%%%%%%%%%%%%%%%%%%%%%%%%%%%%%%%%%%%%%%%%%%%%%%%%%%%%%%%%%%%%%%%%%%%%%%%%%%%%%%%%%%%%%%%%%%%%%%%%%%%%%%%%%%%%%%%%%%%%%%%%%%%%%%%%%%%%%%%%

\usepackage{eurosym}
\usepackage{vmargin}
\usepackage{amsmath}
\usepackage{graphics}
\usepackage{epsfig}
\usepackage{enumerate}
\usepackage{multicol}
\usepackage{subfigure}
\usepackage{fancyhdr}
\usepackage{listings}
\usepackage{framed}
\usepackage{graphicx}
\usepackage{amsmath}
\usepackage{chngpage}

%\usepackage{bigints}
\usepackage{vmargin}

% left top textwidth textheight headheight

% headsep footheight footskip

\setmargins{2.0cm}{2.5cm}{16 cm}{22cm}{0.5cm}{0cm}{1cm}{1cm}

\renewcommand{\baselinestretch}{1.3}

\setcounter{MaxMatrixCols}{10}

\begin{document}
\begin{enumerate}
[Total 5]5
An investigation into the mortality experience of a sample of the male student population of a large university has been carried out. The university authorities wish to know whether the mortality of male students at the university is the same as that of
males in the country as a whole. They have drawn up the following table.
Age x Number of deaths Expected number
of deaths assuming
national mortality
18
19
20
21
22
23 13
15
14
20
12
8 10
12
14
12
8
5
Carry out an overall test of the university authorities’ hypothesis, stating your
conclusion.
CT4 S2008—3
[5]
PLEASE TURN OVER6
A portfolio of term assurance policies was transferred from insurer A to insurer B on 1 January 2001. Each policy in the portfolio was written with premiums payable annually in advance. Insurer B wishes to investigate the mortality experience of its
acquired portfolio and has collected the following data over the period 1 January 2001
to 1 January 2005:
d x numbers of deaths aged x
P x,t number of policies in force aged x at time t (t = 0, 1, 2, 3, 4 years measured
from 1 January 2001)
Where x is defined as:
age last birthday at the most recent policy anniversary prior to the portfolio
transfer + number of premiums received by insurer B.
(i)
(ii)
(a)
(b)
State the rate interval implied by the above data.
Write down the range of ages at the start of the rate interval.

Give an expression which can be used to estimate the initial exposed to risk at age x, E x , stating any assumptions made.

The following is an extract from the data collected in the investigation:
x d x ∑ P x , t ∑ P x , t + 1
39
40
41 28
36
33 10,536
10,965
10,421 11,005
10,745
10,577
where the summations are from t = 0 to t = 3.
7
(iii) Estimate q 40 , stating any further assumptions made.

(i) Explain why, under Continuous Mortality Investigation investigations, the data analysed are usually based upon the number of policies in force and number of policies giving rise to claims, rather than the number of lives
exposed and number of lives who die during the period of study.
[2]
Suppose N identical and independent lives are observed from age x exact for one year
or until death if earlier.

Define:
\begin{itemize}
\item $pi_i$ to be the proportion of the N lives exposed who hold i policies (i = 1,2,3,....);
\item D i to be a random variable denoting the number of deaths amongst lives with i
policies
\item C i to be a random variable denoting the number of claims arising from lives with i
policies.
\end{itemize}
%%%%%%%%%%%%%%%%%%%%%%%%%%%%%%%%%%%%%%%%%%%%%%%%%%%%%%%%%%%%%%%%%%%%5
Using the chi-squared test (a suitable overall test).
actual deaths - expected deaths
, then the test statistic is
expected deaths
where m is the number of ages, which in this case is 6.
If z x =
∑ z x 2 ∼ χ 2 m ,
x
The calculations are shown below.
Age x z x z x 2
18
19
20
21
22
23 0.9487
0.8660
0
2.3094
1.4142
1.3416 0.9
0.75
0
5.3333
2
1.8
Therefore the value of the test statistic is 10.783.
The critical value of the chi-squared distribution
at the 5% level of significance with 6 degrees of
freedom is 12.59.
Since 10.783 < 12.59 there is insufficient evidence to reject
the hypothesis that the mortality rate of men in the University is the same as that of
the national population.
Page 6Subject CT4 — Models Core Technical — September 2008 — Examiners’ Report
6
(i)
Age label changes on the receipt of the
premium on the policy anniversary so this is a
policy year rate interval.
Policyholders’ ages range from x to x +1
at start of the rate interval.
(ii)
Central exposed to risk
E x c
4
1 3
= ∫ P x , t dt ≈ ∑ ( P x , t + P x , t + 1 )
2 t = 0
t = 0
Approximation assumes population changes linearly over each year
during the period of investigation.
1 3
1
( P x , t + P x , t + 1 ) + d x ,
∑
2 t = 0
2
assuming deaths are uniform over the rate interval OR deaths occur on
average half way through the rate interval.
Initial exposed to risk E x ≈
(but NOT deaths are uniform over the “year”, or occur on average half
way through the “year”)
(iii)
d x
estimates q x for the average age
E x
at the start of the rate interval.
q ˆ x =
Assuming birthdays are uniformly distributed
across policy years,
the average age at the start of the rate interval
is x+1⁄2, so we require q̂ 39 1 to estimate q 40 .
2
1
Assuming q̂ 39 1 = [ q ˆ 39 + q ˆ 40 ] we have
2
2
q ˆ 39 =
q ˆ 40 =
28
= 0.002596
1
1
(10536 + 11005) + * 28
2
2
36
1
1
(10965 + 10745) + *36
2
2
= 0.003311
and hence our estimate of q 40 is 0.5[0.002596 + 0.003311) = 0.002954.
Page 7Subject CT4 — Models Core Technical — September 2008 — Examiners’ Report
7
(i)
Individual life offices are likely to have their systems set up to provide
information on a “by policy” basis.
When data from different offices is pooled, it would not be practicable to
establish whether an individual held policies with other companies.
(ii)
If the mortality rate is q x then since the lives are independent the number of
deaths D i will be distributed Binomial ( q x , π i N )
So
∑ C i = ∑ i D i .
i
i
⎡
⎤
⎡
⎤
Hence Var[ C ] = Var ⎢ ∑ C i ⎥ = Var ⎢ ∑ i D i ⎥ = ∑ i 2 Var [ D i ]
⎣ ⎢ i
⎦ ⎥ i
⎣ ⎢ i
⎦ ⎥
by independence of deaths
= ∑ i 2 π i Nq x (1 − q x )
i
If instead there were
∑ i π i N independent
i
policies/lives the variance would be additive so:
Var [ C ′ ] = ∑ i π i Nq x (1 − q x )
i
∑ i 2 π i
So the variance is increased by the ratio i
∑ i π i
i
(iii)
If the proportions of lives holding i policies were known, the variance ratio
could be allowed for in statistical tests
by using the ratio to adjust the variance upwards.
However, the variance ratio is unlikely to be known exactly.
Special investigations may be performed from time to time to estimate the
variance ratios by matching up policyholders, which could then be applied to
subsequent mortality investigations.
Page 8Subject CT4 — Models Core Technical — September 2008 — Examiners’ Report

