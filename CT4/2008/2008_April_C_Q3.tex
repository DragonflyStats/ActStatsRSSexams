
\documentclass[a4paper,12pt]{article}

%%%%%%%%%%%%%%%%%%%%%%%%%%%%%%%%%%%%%%%%%%%%%%%%%%%%%%%%%%%%%%%%%%%%%%%%%%%%%%%%%%%%%%%%%%%%%%%%%%%%%%%%%%%%%%%%%%%%%%%%%%%%%%%%%%%%%%%%%%%%%%%%%%%%%%%%%%%%%%%%%%%%%%%%%%%%%%%%%%%%%%%%%%%%%%%%%%%%%%%%%%%%%%%%%%%%%%%%%%%%%%%%%%%%%%%%%%%%%%%%%%%%%%%%%%%%

\usepackage{eurosym}
\usepackage{vmargin}
\usepackage{amsmath}
\usepackage{graphics}
\usepackage{epsfig}
\usepackage{enumerate}
\usepackage{multicol}
\usepackage{subfigure}
\usepackage{fancyhdr}
\usepackage{listings}
\usepackage{framed}
\usepackage{graphicx}
\usepackage{amsmath}
\usepackage{chngpage}

%\usepackage{bigints}
\usepackage{vmargin}

% left top textwidth textheight headheight

% headsep footheight footskip

\setmargins{2.0cm}{2.5cm}{16 cm}{22cm}{0.5cm}{0cm}{1cm}{1cm}

\renewcommand{\baselinestretch}{1.3}

\setcounter{MaxMatrixCols}{10}

\begin{document}
\begin{enumerate}


3
\item (i)
Define the following stochastic processes:
(a)
(b)
Poisson process
compound Poisson process

\item (ii)
Identify the circumstances in which a compound Poisson process is also a Poisson process.

%%%%%%%%%%%%%%%%%%%%%%%%%%%%%%%%%%%%%%%%%%%%%%%%%%%%%%%%%%%%%%%%%%%%%%%
3
\item (i)
(a)
EITHER
A Poisson process with rate \lambda is a continuous-time
integer-valued process N_t ,
t \geq 0), with the following properties:
N 0 = 0
N_t has independent increments
N_t has stationary increments
P [ N_t − N s = n ] =
[ \lambda ( t − s )] n e −\lambda ( t − s )
s < t, n = 0, 1, 2.....
n !
OR
A Poisson process with rate \lambda is a continuous-time integer-valued process N_t ,
t \geq 0), with the following properties:
N 0 = 0
P [ N_t + h − N_t = 1] = \lambda h + o ( h )
P [ N_t + h − N_t = 0] = 1 − \lambda h + o ( h )
P [ N_t + h − N_t ≠ 0,1] = o ( h )
(b)
If N_t is a Poisson process oN_t \geq 0 and Y_i is a sequence of
independent and identically distributed random variables then a
compound Poisson process is defined by:
N_t
X t = ∑ Y_i
i = 1
\item (ii)
A compound Poisson process meets the conditions for being
a Poisson process if Y_i is an indicator function OR if each Y_i is identically
1 (which is a special case of the indicator function)
\end{document}

\end{document}
