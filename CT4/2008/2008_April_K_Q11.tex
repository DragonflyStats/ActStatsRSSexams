\documentclass[a4paper,12pt]{article}

%%%%%%%%%%%%%%%%%%%%%%%%%%%%%%%%%%%%%%%%%%%%%%%%%%%%%%%%%%%%%%%%%%%%%%%%%%%%%%%%%%%%%%%%%%%%%%%%%%%%%%%%%%%%%%%%%%%%%%%%%%%%%%%%%%%%%%%%%%%%%%%%%%%%%%%%%%%%%%%%%%%%%%%%%%%%%%%%%%%%%%%%%%%%%%%%%%%%%%%%%%%%%%%%%%%%%%%%%%%%%%%%%%%%%%%%%%%%%%%%%%%%%%%%%%%%

\usepackage{eurosym}
\usepackage{vmargin}
\usepackage{amsmath}
\usepackage{graphics}
\usepackage{epsfig}
\usepackage{enumerate}
\usepackage{multicol}
\usepackage{subfigure}
\usepackage{fancyhdr}
\usepackage{listings}
\usepackage{framed}
\usepackage{graphicx}
\usepackage{amsmath}
\usepackage{chngpage}

%\usepackage{bigints}
\usepackage{vmargin}

% left top textwidth textheight headheight

% headsep footheight footskip

\setmargins{2.0cm}{2.5cm}{16 cm}{22cm}{0.5cm}{0cm}{1cm}{1cm}

\renewcommand{\baselinestretch}{1.3}

\setcounter{MaxMatrixCols}{10}

\begin{document}
\begin{enumerate}
11
An investigation was carried out into the relationship between sickness and mortality
in an historical population of working class men. The investigation used a three-state
model with the states:
1
2
3
Healthy
Sick
Dead
Let the probability that a person in state i at time x will be in state j at time x+t be
ij
ij
t p x . Let the transition intensity at time x+t between any two states i and j be \mu x + t .
\begin{enumerate}
\item (i) Draw a diagram showing the three states and the possible transitions between
them.

\item (ii) Show from first principles that
∂
23
21 13
22 23
t p x = t p x \mu x + t + t p x \mu x + t .
∂ t
\item (iii)

Write down the likelihood of the data iN_the investigation iN_terms of the
transition rates and the waiting times iN_the Healthy and Sick states, under the
assumptioN_that the transition rates are constant.

The investigation collected the following data:
\begin{itemize}
\item man-years in Healthy state : 265
\item man-years in Sick state : 140
\item number of transitions from Healthy to Sick : 20
\item number of transitions from Sick to Dead : 40
\end{itemize}
%%%%%%%%%%%%%%%%%%%%%%%%%%%%%%%%%%%%%%%%%%%%%%%%%%%%%


\item (iv) Derive the maximum likelihood estimator of the transition rate from Sick to
Dead.

\item (v) Hence estimate:
(a)
(b)
the value of the constant transition rate from Sick to Dead
95 per cent confidence intervals around this transition rate
\end{enumerate}
% [Total 17]
% END OF PAPER
% CT4 A2008—7

%%%%%%%%%%%%%%%%%%%%%%%%%%%%%%%%%%%

11
(i)
1 Healthy
2 Sick
3 Dead
(ii)
By the Markov assumption OR conditioning on the
state occupied at time x+t
t + dt 22
23
23
33
p x 23 = t p x 21 dt p 13
x + t + t p x dt p x + t + t p x dt p x + t .
But dt
t + dt 22
23
23
p x 23 = t p x 21 dt p 13
x + t + t p x dt p x + t + t p x .
p 33
x + t = 1, so
We now assume that
dt
p x 23 + t = μ 23
x + t dt + o ( dt ) and
dt
=
p 13
= μ 13
x + t dt + o ( dt )
x + t
o ( dt )
= 0 .
dt → 0 dt
where o ( dt ) is defined such that lim
Substituting for
t + dt
dt
p x 23 + t and
dt
p 13
x + t produces
21 13
23
p x 23 = t p x 22 [ μ 23
x + t dt + o ( dt )] + t p x [ μ x + t dt + o ( dt )] + t p x ,
and, subtracting t p x 23 from both sides and taking limits
gives
d
23
22 23
t p x − t + dt p x
= t p x 21 μ 13
t p x = lim
x + t + t p x μ x + t
→
0
dt
dt
dt
Page 18Subject CT4 — Models Core Technical — April 2008 — Examiners’ Report
(iii)
The likelihood, L, is proportional to
12
21
13
exp[( −μ 12 − μ 13 ) v 1 ]exp[( −μ 23 − μ 21 ) v 2 ]( μ 12 ) d ( μ 21 ) d ( μ 13 ) d ( μ 23 ) d
23
where v i is the total observed waiting time in state i,
and d ij is the number of transitions observed from
state i to state j.
(iv)
Taking the logarithm of the likelihood in the
answer to part (iii) gives
log L = −μ 23 v 2 + d 23 log μ 23 + terms not involving μ 23
Differentiating this with respect to μ 23 we obtain
d log L
d μ 23
= − v 2 +
d 23
μ 23
.
Setting this to 0 we obtain the maximum likelihood
estimator of μ 23
^
μ 23 =
d 23
v 2
.
This is a maximum because
d 2 (log L )
( d μ 23 ) 2
=−
d 23
( μ 23 ) 2
which is always negative.
(v)
(a) Therefore, if there are 40 transitions from
the Sick state to the Dead state and 140 man-years
observed in the sick state, the maximum
40
= 0.2857 .
likelihood estimate of μ 23 is
140
(b) The maximum likelihood estimator of μ 23 has a
μ 23
, μ 23 is the true
E [ V ]
transition rate in the population and E [ V ] is the
expected waiting time in the Sick state.
variance equal to
Page 19Subject CT4 — Models Core Technical — April 2008 — Examiners’ Report
^
Approximating μ 23 by μ 23 and E [ V ] by v 2 we
0.2857
estimate for the variance as
= 0.00204 .
140
A 95 per cent confidence interval around our
estimate of μ 23 is therefore 0.2857 ± 1.96 0.00204
which is 0.2857 ± 0.0885
or (0.1972, 0.3742).
END OF EXAMINERS’ REPORT
Page 20
\end{document}
