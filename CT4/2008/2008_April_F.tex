\documentclass[a4paper,12pt]{article}

%%%%%%%%%%%%%%%%%%%%%%%%%%%%%%%%%%%%%%%%%%%%%%%%%%%%%%%%%%%%%%%%%%%%%%%%%%%%%%%%%%%%%%%%%%%%%%%%%%%%%%%%%%%%%%%%%%%%%%%%%%%%%%%%%%%%%%%%%%%%%%%%%%%%%%%%%%%%%%%%%%%%%%%%%%%%%%%%%%%%%%%%%%%%%%%%%%%%%%%%%%%%%%%%%%%%%%%%%%%%%%%%%%%%%%%%%%%%%%%%%%%%%%%%%%%%

\usepackage{eurosym}
\usepackage{vmargin}
\usepackage{amsmath}
\usepackage{graphics}
\usepackage{epsfig}
\usepackage{enumerate}
\usepackage{multicol}
\usepackage{subfigure}
\usepackage{fancyhdr}
\usepackage{listings}
\usepackage{framed}
\usepackage{graphicx}
\usepackage{amsmath}
\usepackage{chngpage}

%\usepackage{bigints}
\usepackage{vmargin}

% left top textwidth textheight headheight

% headsep footheight footskip

\setmargins{2.0cm}{2.5cm}{16 cm}{22cm}{0.5cm}{0cm}{1cm}{1cm}

\renewcommand{\baselinestretch}{1.3}

\setcounter{MaxMatrixCols}{10}

\begin{document}
\begin{enumerate}
11
An investigation was carried out into the relationship between sickness and mortality
in an historical population of working class men. The investigation used a three-state
model with the states:
1
2
3
Healthy
Sick
Dead
Let the probability that a person in state i at time x will be in state j at time x+t be
ij
ij
t p x . Let the transition intensity at time x+t between any two states i and j be \mu x + t .
\begin{enumerate}
\item (i) Draw a diagram showing the three states and the possible transitions between
them.

\item (ii) Show from first principles that
∂
23
21 13
22 23
t p x = t p x \mu x + t + t p x \mu x + t .
∂ t
\item (iii)

Write down the likelihood of the data iN_the investigation iN_terms of the
transition rates and the waiting times iN_the Healthy and Sick states, under the
assumptioN_that the transition rates are constant.

The investigation collected the following data:
\begin{itemize}
\item man-years in Healthy state : 265
\item man-years in Sick state : 140
\item number of transitions from Healthy to Sick : 20
\item number of transitions from Sick to Dead : 40
\end{itemize}
%%%%%%%%%%%%%%%%%%%%%%%%%%%%%%%%%%%%%%%%%%%%%%%%%%%%%


\item (iv) Derive the maximum likelihood estimator of the transition rate from Sick to
Dead.

\item (v) Hence estimate:
(a)
(b)
the value of the constant transition rate from Sick to Dead
95 per cent confidence intervals around this transition rate
\end{enumerate}
% [Total 17]
% END OF PAPER
% CT4 A2008—7
\end{document}
