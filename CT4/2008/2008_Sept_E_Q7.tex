\documentclass[a4paper,12pt]{article}

%%%%%%%%%%%%%%%%%%%%%%%%%%%%%%%%%%%%%%%%%%%%%%%%%%%%%%%%%%%%%%%%%%%%%%%%%%%%%%%%%%%%%%%%%%%%%%%%%%%%%%%%%%%%%%%%%%%%%%%%%%%%%%%%%%%%%%%%%%%%%%%%%%%%%%%%%%%%%%%%%%%%%%%%%%%%%%%%%%%%%%%%%%%%%%%%%%%%%%%%%%%%%%%%%%%%%%%%%%%%%%%%%%%%%%%%%%%%%%%%%%%%%%%%%%%%

\usepackage{eurosym}
\usepackage{vmargin}
\usepackage{amsmath}
\usepackage{graphics}
\usepackage{epsfig}
\usepackage{enumerate}
\usepackage{multicol}
\usepackage{subfigure}
\usepackage{fancyhdr}
\usepackage{listings}
\usepackage{framed}
\usepackage{graphicx}
\usepackage{amsmath}
\usepackage{chngpage}

%\usepackage{bigints}
\usepackage{vmargin}

% left top textwidth textheight headheight

% headsep footheight footskip

\setmargins{2.0cm}{2.5cm}{16 cm}{22cm}{0.5cm}{0cm}{1cm}{1cm}

\renewcommand{\baselinestretch}{1.3}

\setcounter{MaxMatrixCols}{10}

\begin{document}
\begin{enumerate}[(a)]
(i) Explain why, under Continuous Mortality Investigation investigations, the data analysed are usually based upon the number of policies in force and number of policies giving rise to claims, rather than the number of lives
exposed and number of lives who die during the period of study.
[2]
Suppose N identical and independent lives are observed from age x exact for one year
or until death if earlier.
\medskip
Define:
\begin{itemize}
\item $pi_i$ to be the proportion of the N lives exposed who hold i policies (i = 1,2,3,....);
\item D i to be a random variable denoting the number of deaths amongst lives with i
policies
\item C i to be a random variable denoting the number of claims arising from lives with i
policies.
\end{itemize}


\item (ii) Derive an expression for the ratio of the variance of the number of claims
arising compared with that if each policy covered an independent life.

\item (iii) Explain how the expression derived in \item (ii) could be used in practice.
\end{enumerate}


%%%%%%%%%%%%%%%%%%%%%%%%%%%%%%5
\newpage
7
\begin{itemize}
\item (i)
Individual life offices are likely to have their systems set up to provide
information on a “by policy” basis.
When data from different offices is pooled, it would not be practicable to
establish whether an individual held policies with other companies.
\item (ii)
If the mortality rate is q x then since the lives are independent the number of
deaths D i will be distributed Binomial $( q x , \pi i N )$
So
$\sum C i = \sum i D i .$
i
i
⎡
⎤
⎡
⎤
Hence Var[ C ] = Var ⎢ \sum C i ⎥ = Var ⎢ \sum i D i ⎥ = \sum i 2 Var [ D i ]
⎣ ⎢ i
⎦ ⎥ i
⎣ ⎢ i
⎦ ⎥
by independence of deaths
= \sum i 2 \pi i Nq x (1 − q x )
i
If instead there were
\sum i \pi i N independent
i
policies/lives the variance would be additive so:
Var [ C ′ ] = \sum i \pi i Nq x (1 − q x )
i
\sum i 2 \pi i
So the variance is increased by the ratio i
\sum i \pi i
i
\item (iii)
If the proportions of lives holding i policies were known, the variance ratio
could be allowed for in statistical tests
by using the ratio to adjust the variance upwards.
\item However, the variance ratio is unlikely to be known exactly.
Special investigations may be performed from time to time to estimate the
variance ratios by matching up policyholders, which could then be applied to
subsequent mortality investigations.
%%-- Page 8Subject CT4 — Models Core Technical — September 2008 — Examiners’ Report
\end{itemize}
\end{document}
