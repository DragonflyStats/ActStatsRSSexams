\documentclass[a4paper,12pt]{article}

%%%%%%%%%%%%%%%%%%%%%%%%%%%%%%%%%%%%%%%%%%%%%%%%%%%%%%%%%%%%%%%%%%%%%%%%%%%%%%%%%%%%%%%%%%%%%%%%%%%%%%%%%%%%%%%%%%%%%%%%%%%%%%%%%%%%%%%%%%%%%%%%%%%%%%%%%%%%%%%%%%%%%%%%%%%%%%%%%%%%%%%%%%%%%%%%%%%%%%%%%%%%%%%%%%%%%%%%%%%%%%%%%%%%%%%%%%%%%%%%%%%%%%%%%%%%

\usepackage{eurosym}
\usepackage{vmargin}
\usepackage{amsmath}
\usepackage{graphics}
\usepackage{epsfig}
\usepackage{enumerate}
\usepackage{multicol}
\usepackage{subfigure}
\usepackage{fancyhdr}
\usepackage{listings}
\usepackage{framed}
\usepackage{graphicx}
\usepackage{amsmath}
\usepackage{chngpage}

%\usepackage{bigints}
\usepackage{vmargin}

% left top textwidth textheight headheight

% headsep footheight footskip

\setmargins{2.0cm}{2.5cm}{16 cm}{22cm}{0.5cm}{0cm}{1cm}{1cm}

\renewcommand{\baselinestretch}{1.3}

\setcounter{MaxMatrixCols}{10}

\begin{document}

8

[Total 8]
A No-Claims Discount system operated by a motor insurer has the following four
levels:

\begin{itemize}
\item Level 1: 0\% discount
\item Level 2: 25\% discount
\item Level 3: 40\% discount
\item Level 4: 60\% discount
\end{itemize}





The rules for moving between these levels are as follows:
\begin{enumerate}
\item Following a year with no claims, move to the next higher level, or remain at
level 4.
\item  Following a year with one claim, move to the next lower level, or remain at
level 1.
\item  Following a year with two or more claims, move down two levels, or move to
level 1 (from level 2) or remain at level 1.
\end{enumerate}
For a given policyholder in a given year the probability of no claims is 0.85 and the
probability of making one claim is 0.12.
\begin{enumerate}
\item (i) Write down the transition matrix of this No-Claims Discount process.
\item (ii) Calculate the probability that a policyholder who is currently at level 2 will be
at level 2 after:
(a)one year.

(b) two years.
\item (iii)



Calculate the long-run probability that a policyholder is in discount level 2.
\end{enumerate}
%%-- [Total 9]
CT4 S2008—5
%%%%%%%%%%%%%%%%%%%%%%%%%%%%%%%%%%%%%%%%%%%%%%%%%%%%%%%%%%%%%%

\newpage



8
\begin{itemize}
\item (i)
The transition matrix of the process is
The transition matrix of the process is

\[ \begin{bmatrix}
0.15 & 0.85 & 0 & 0 \\
0.15 & 0 & 0.85 & 0 \\
0.03 & 0.12 & 0 & 0.85 \\
0 & 0.03 & 0.12 & 0.85 
\end{bmatrix}  \]

(a) For the one year transition, p 22 = 0, as can be seen
from above (or is obvious from the statement).
(b) The second order transition matrix is
\[P^2 =  \begin{bmatrix}
0.15 & 0.85 & 0 & 0 \\
0.15 & 0 & 0.85 & 0 \\
0.03 & 0.12 & 0 & 0.85 \\
0 & 0.03 & 0.12 & 0.85 
\end{bmatrix}  \times \begin{bmatrix}
0.15 & 0.85 & 0 & 0 \\
0.15 & 0 & 0.85 & 0 \\
0.03 & 0.12 & 0 & 0.85 \\
0 & 0.03 & 0.12 & 0.85 
\end{bmatrix}  \]
%%%%%%%%%%%%%%%%%%%%%%%%%%%%%%%%%%%%%%%%%%%%%%%%%%%
(a) For the one year transition, p 22 = 0, as can be seen
from above (or is obvious from the statement).
(b) The second order transition matrix is
% \[P^2 =  \begin{bmatrix}
% (0.15 \times 0.15) + (0.85 \times 0.15) & 
%     (0.15 \times 0.85) + (0) + (0) + (0) & 
%       (0.15 \times 0) + (0.85 \times 0.85) + (0) + (0)& 
%       (0) + (0) + (0) + (0) \\
%--------------------%
%(0.15 \times 0.15) + (0) +  (0.85 \times 0.03) + (0) & 
%   (0.15 \times 0.15) + (0) + (0.85 \times 0.15) + (0)  & 
%      (0.15 \times 0.15) + (0) + (0.85 \times 0.15) + (0)  & 
%         (0.15 \times 0.15) + (0)  + (0.85 \times 0.15) + (0)  \\
%--------------------%
%(0.03 \times 0.15) + (0.12 \times 0.15) + (0)+  (0.85 \times 0.15) & 
%   (0.03 \times 0.15) + (0.12 \times 0.15) + (0)+ (0.85 \times 0.15) & 
%      (0.03 \times 0.15) + (0.12 \times 0.15)+ (0) +  (0.85 \times 0.15) & 
%         (0.03 \times 0.15) + (0.12 \times 0.15)+ (0) +  (0.85 \times 0.15) \\
%--------------------%
% (0) + (0.03 \times 0.15) + (0.12 \times 0.03) +  (0.85 \times 0)  &  
%     (0) +  (0.03 \times 0) + (0.12 \times 0.15) +  (0.85 \times 0.15) & 
%        (0) + (0.03 \times 0.15) + (0.12 \times 0.12) +  (0.85 \times 0.15) & 
%           (0) + (0.03 \times 0.15) + (0.12 \times 0.15) +  (0.85 \times 0.15)  \\
%\end{bmatrix}\]

\[ P^2 =\begin{bmatrix}
0.15 & 0.1275 & 0.7225 & 0 \\
0.048 & 0.2995 & 0 & 0.7225 \\
0.0225 & 0.051 & 0.204 & 0.7225 \\
0 & 0.0399 & 0.1275 & 0.8245 
\end{bmatrix}  \]


hence the required probability is $0.229$5.
\item (iii)
In matrix form, the equation we need to solve is $\pi P = \pi$,
where $\pi$ is the vector of equilibrium probabilities.

\[ \begin{bmatrix}
\pi_1& \pi_2 & \pi_3 & \pi_4 
\end{bmatrix}\begin{bmatrix}
0.15 & 0.85 & 0 & 0 \\
0.15 & 0 & 0.85 & 0 \\
0.03 & 0.12 & 0 & 0.85 \\
0 & 0.03 & 0.12 & 0.85 
\end{bmatrix} = \begin{bmatrix}
\pi_1& \pi_2 & \pi_3 & \pi_4 
\end{bmatrix}  \]

This reads
\begin{enumerate}[(1)]
\item ${ \displaystyle 0.15 \pi_1 + 0.15 \pi_2 + 0.03 \pi_3 = \pi_1 }$
\item ${ \displaystyle 0.85 \pi_1 + 0.12 \pi_3 + 0.03 \pi_4 = \pi_2 }$
\item ${ \displaystyle 0.85 \pi_2 + 0.12 \pi_4 = \pi_3 }$
\item ${ \displaystyle  0.85 \pi_3 + 0.85 \pi_4 = \pi_4 }$
\end{enumerate}
%%%%%%%%%%%%%%%%%%%%%%%%%%%%%%%%%%%%%%
\item Discard the first of these equations and use also the fact that
\[\sum_{i= 1} \pi_i = 1 .\]
4

\item Then, we obtain first from (4) that $0.85 \pi_3 = 0.15 \pi_4$
or, that $\pi_4 = 17 \pi_3 / 3$
%- Page 9 — Models Core Technical — September 2008 — Examiners’ 

\item Substituting in (3) this gives
\[0.85 \pi_2 + 0.12 ×
17
\pi_3 = \pi_3 ⇒ \pi_3 = 2.65625 \pi_2
3\]
(2) now yields that
\[0.85 \pi_1 = \pi_2 − 0.12 \pi_3 − 0.03 \pi_4 =
1\]
\[\pi_3 − 0.12 \pi_3 − 0.17 \pi_3 = 0.0865 \pi_3 ,
2.65625\]
so that finally we get $\pi_1 = 0.10173 \pi_3$ .
Using now that the probabilities must add up to one, we obtain
\[ \pi_1 + \pi_2 + \pi_3 + \pi_4 = (0.10173 + 0.3765 + 1 + 5.666) \pi_3 = 1, \]
or that $\pi_3 = 0.13996$.
Solving back for the other variables we get that
\begin{itemize}
    \item $ { \displaystyle \pi_1 = 0.01424}$, 
    \item $ { \displaystyle \pi_2 = 0.05269}$, 
    \item $ { \displaystyle \pi_4 = 0.79311}$
\end{itemize}

\item The long-run probability that the motorist is in discount level 2 is therefore
0.05269.

\end{itemize}
\end{document}
