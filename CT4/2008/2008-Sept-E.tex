\documentclass[a4paper,12pt]{article}

%%%%%%%%%%%%%%%%%%%%%%%%%%%%%%%%%%%%%%%%%%%%%%%%%%%%%%%%%%%%%%%%%%%%%%%%%%%%%%%%%%%%%%%%%%%%%%%%%%%%%%%%%%%%%%%%%%%%%%%%%%%%%%%%%%%%%%%%%%%%%%%%%%%%%%%%%%%%%%%%%%%%%%%%%%%%%%%%%%%%%%%%%%%%%%%%%%%%%%%%%%%%%%%%%%%%%%%%%%%%%%%%%%%%%%%%%%%%%%%%%%%%%%%%%%%%

\usepackage{eurosym}
\usepackage{vmargin}
\usepackage{amsmath}
\usepackage{graphics}
\usepackage{epsfig}
\usepackage{enumerate}
\usepackage{multicol}
\usepackage{subfigure}
\usepackage{fancyhdr}
\usepackage{listings}
\usepackage{framed}
\usepackage{graphicx}
\usepackage{amsmath}
\usepackage{chngpage}

%\usepackage{bigints}
\usepackage{vmargin}

% left top textwidth textheight headheight

% headsep footheight footskip

\setmargins{2.0cm}{2.5cm}{16 cm}{22cm}{0.5cm}{0cm}{1cm}{1cm}

\renewcommand{\baselinestretch}{1.3}

\setcounter{MaxMatrixCols}{10}

\begin{document}
\begin{enumerate}


11
Consider the random variable defined by X n =
∑ Y i with each Y i mutually
i = 1
independent with probability:
P[Y i = 1] = p, P[Y i = -1] = 1- p
0 < p < 1
(i) Write down the state space and transition graph of the sequence X n .
(ii) State, with reasons, whether the process:
(a)
(b)
(c)
is aperiodic.
is reducible.
admits a stationary distribution.
[2]
[3]
Consider j > i > 0.
(iii) Derive an expression for the number of upward movements in the sequence X n
between t and (t + m) if X t = i and X t+m = j.
[2]
(iv) Derive expressions for the m-step transition probabilities p ij ( m ) .
(v) Show how the one-step transition probabilities would alter if X n was restricted
to non-negative numbers by introducing:
(a)
(b)
[3]
a reflecting boundary at zero.
an absorbing boundary at zero.
[2]
(vi)
For each of the examples in part (v), explain whether the transition
probabilities p ij ( m ) would increase, decrease or stay the same.
(Calculation of the transition probabilities is not required.)
CT4 S2008—8
[3]
%%%%%%%%%%%%%%%%%%%%%%%%%%%%%%%%%%%%%%%%%%%%%%%%%%%%%%%%%%%%%%%%%%%%%%%%%%%5555
Page 14Subject CT4 — Models Core Technical — September 2008 — Examiners’ Report
11
(i)
State space is the set of integers Ζ .
Transition graph:
p
-2
-1
1- p
(ii)
(a)
p
p
0
1- p
p
1
1- p
2
1- p
The process is not aperiodic
because it has period 2:
for example, starting from an even number the
process is only even after an even number of steps
(b)
The process is irreducible
as the probabilities of X n increasing and decreasing by 1 are both
non-zero so any state can be reached.
(c)
(iii)
No stationary distribution will exist because the state space is infinite.
Suppose there are u upward movements.
Then there must be m − u downward movements,
and u – ( m – u ) = j – i
So u =
(iv)
m + j − i
.
2
The maximum number of upward steps is m so the
transition probability is zero if j – i > m .
As the chain is periodic with period 2, it can only occupy
state j after m steps if m + j − i is even.
If m + j − i is even and j – i ≤ m then there must be u
upward jumps and ( m − u ) downward jumps.
⎛ m ⎞
These can be ordered in ⎜ ⎟ ways.
⎝ u ⎠
Page 15Subject CT4 — Models Core Technical — September 2008 — Examiners’ Report
So the transition probabilities are:
p ij ( m )
(v)
⎧⎛ m ⎞ u
p (1 − p ) m − u
⎪
= ⎨ ⎜ ⎝ u ⎟ ⎠
⎪
0
⎩
if j − i ≤ m and m + j − i even
otherwise
EITHER
In both cases the transition probabilities
are unaltered unless X i = 0.
(a) Reflecting boundary implies
P[X i+1 = 1│X i = 0] = 1 (or p 01(1) = 1)
(b) Absorbing boundary implies
P[X i+1 = 0│X i = 0] = 1 (or p 00(1) = 1)
OR
A matrix solution for the transition probabilities is acceptable
Reflecting:
1
0
0
⎛ 0
⎜
p
0
0
⎜ 1 − p
⎜ 0
p
1 − p
0
⎜
0
1 − p
0
⎜ 0
⎜ 0
0
0
1 − p
⎜ ⎜
:
:
:
⎝ :
0
0
0
p
0
: ... ⎞
⎟
... ⎟
... ⎟
⎟
... ⎟
... ⎟
⎟ ⎟
⎠
0
0
0
p
0
: ... ⎞
⎟
... ⎟
... ⎟
⎟
... ⎟
... ⎟
⎟ ⎟
⎠
Absorbing:
0
0
0
⎛ 1
⎜
p
0
0
⎜ 1 − p
⎜ 0
p
1 − p
0
⎜
0
1 − p
0
⎜ 0
⎜ 0
0
0
1 − p
⎜ ⎜
:
:
:
⎝ :
OR
A diagrammatic solution is also acceptable:
Page 16Subject CT4 — Models Core Technical — September 2008 — Examiners’ Report
Reflecting
1
0
p
1
1-p
2
1-p
Absorbing:
p
1
0
1
1-p
(vi)
2
1-p
In both cases the zero transition probabilities remain
zero as the period is still 2 where relevant.
If i is sufficiently above 0 then conditions at zero
will not be relevant and all the m-step transition
probabilities will remain the same. (This applies if m < i.)
Otherwise
In (a) some sample paths which would have
taken X below zero will be reflected, increasing the
probability of reaching j at step m.
So the m-step transition probabilities would increase.
In (b) any sample path which reaches zero would
no longer be able to access state j
so the transition probabilities would decrease.
Page 17Subject CT4 — Models Core Technical — September 2008 — Examiners’ Report
