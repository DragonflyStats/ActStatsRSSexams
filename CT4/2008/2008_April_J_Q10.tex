\documentclass[a4paper,12pt]{article}

%%%%%%%%%%%%%%%%%%%%%%%%%%%%%%%%%%%%%%%%%%%%%%%%%%%%%%%%%%%%%%%%%%%%%%%%%%%%%%%%%%%%%%%%%%%%%%%%%%%%%%%%%%%%%%%%%%%%%%%%%%%%%%%%%%%%%%%%%%%%%%%%%%%%%%%%%%%%%%%%%%%%%%%%%%%%%%%%%%%%%%%%%%%%%%%%%%%%%%%%%%%%%%%%%%%%%%%%%%%%%%%%%%%%%%%%%%%%%%%%%%%%%%%%%%%%

\usepackage{eurosym}
\usepackage{vmargin}
\usepackage{amsmath}
\usepackage{graphics}
\usepackage{epsfig}
\usepackage{enumerate}
\usepackage{multicol}
\usepackage{subfigure}
\usepackage{fancyhdr}
\usepackage{listings}
\usepackage{framed}
\usepackage{graphicx}
\usepackage{amsmath}
\usepackage{chngpage}

%\usepackage{bigints}
\usepackage{vmargin}

% left top textwidth textheight headheight

% headsep footheight footskip

\setmargins{2.0cm}{2.5cm}{16 cm}{22cm}{0.5cm}{0cm}{1cm}{1cm}

\renewcommand{\baselinestretch}{1.3}

\setcounter{MaxMatrixCols}{10}

\begin{document}
\begin{enumerate}
\item An investigation into the mortality of patients following a specific type of major operation was undertaken. A sample of 10 patients was followed from the date of the operation until either they died, or they left the hospital where the operation was
carried out, or a period of 30 days had elapsed (whichever of these events occurred first). The data on the 10 patients are given in the table below.
(i)
Patient number Duration of
observation
(days) Reason for
observation
ceasing
1
2
3
4
5
6
7
8
9
10 2
6
12
20
24
27
30
30
30
30 Died
Died
Died
Left hospital
Left hospital
Died
Study ended
Study ended
Study ended
Study ended
State whether the following types of censoring are present in this
investigation. In each case give a reason for your answer.
(a)
(b)
(c)
Type I
Type II
Random

\begin{enumerate}
\item (ii) State, with a reason, whether the censoring in this investigation is likely to be informative.
\item (iii) Calculate the value of the Kaplan-Meier estimate of the survival function at duration 28 days.
\item (iv) Write down the Kaplan-Meier estimate of the hazard of death at duration 8 days.
\item (v) Sketch the Kaplan-Meier estimate of the survival function.
\end{enumerate}

\newpage
%%%%%%%%%%%%%%%%%%%%%%%%%%%%%%%%%%%%%%%%%%%%%%%%%%%%%%%%%%%%%%%%%%%%%%%%%%%%%%
%%- Question  10
An internet service provider (ISP) is modelling the capacity requirements for its network. It assumes that if a customer is not currently connected to the internet (“offline”) the probability of connecting in the short time interval $[t,dt]$ is
$0.2dt + o(dt)$. If the customer is connected to the internet (“online”) then it assumes the probability of disconnecting in the time interval is given by $0.8dt + o(dt)$.
The probabilities that the customer is online and offline at time t are P ON (t) and
P OFF (t) respectively.
\begin{enumerate}
\item (i)
Explain why the status of an individual customer can be considered as a Markov Jump Process. 
\item (ii) ′ ( t ) .
Write down Kolmogorov’s forward equation for P OFF
\item (iii) Solve the equation in part (ii) to obtain a formula for the probability that a customer is offline at time t, given that they were offline at time 0.
\item (iv) Calculate the expected proportion of time spent online over the period [0,t].
[HINT: Consider the expected value of an indicator function which takes the
value 1 if offline and 0 otherwise.]
\item (v) (a)
(b)
%%%%%%%%%%%%%%%%%%%%%%%%%%%%%%%%%%%%%%%%%%%%%%%%%%%55
Sketch a graph of your answer to (iv) above.
Explain its shape.
\end{enumerate}
\end{document}
