\documentclass[a4paper,12pt]{article}

%%%%%%%%%%%%%%%%%%%%%%%%%%%%%%%%%%%%%%%%%%%%%%%%%%%%%%%%%%%%%%%%%%%%%%%%%%%%%%%%%%%%%%%%%%%%%%%%%%%%%%%%%%%%%%%%%%%%%%%%%%%%%%%%%%%%%%%%%%%%%%%%%%%%%%%%%%%%%%%%%%%%%%%%%%%%%%%%%%%%%%%%%%%%%%%%%%%%%%%%%%%%%%%%%%%%%%%%%%%%%%%%%%%%%%%%%%%%%%%%%%%%%%%%%%%%

\usepackage{eurosym}
\usepackage{vmargin}
\usepackage{amsmath}
\usepackage{graphics}
\usepackage{epsfig}
\usepackage{enumerate}
\usepackage{multicol}
\usepackage{subfigure}
\usepackage{fancyhdr}
\usepackage{listings}
\usepackage{framed}
\usepackage{graphicx}
\usepackage{amsmath}
\usepackage{chngpage}

%\usepackage{bigints}
\usepackage{vmargin}

% left top textwidth textheight headheight

% headsep footheight footskip

\setmargins{2.0cm}{2.5cm}{16 cm}{22cm}{0.5cm}{0cm}{1cm}{1cm}

\renewcommand{\baselinestretch}{1.3}

\setcounter{MaxMatrixCols}{10}

\begin{document}
4
In the village of Selborne in southern England in the year 1637 the number of babies
born each month was as follows
January
February
March
April
May
June
2
1
1
2
1
2
July
August
September
October
November
December
5
1
0
2
0
3
Data show that over the 20 years before 1637 there was an average of 1.5 births per
month. You may assume that births in the village historically follow a Poisson
process.
An historian has suggested that the large number of births in July 1637 is unusual.
\begin{enumerate}[(a)]
\item (i) Carry out a test of the historian’s suggestion, stating your conclusion.
\item (ii) Comment on the assumption that births follow a Poisson process.
\end{enumerate}


%%%%%%%%%%%%%%%%%%%%%%%%%%%%%%%%%%%%%%%%%%%%%%%%%%%%%%%%%%%%%%%%%%%%%%%%%%%%%%%%%%%%%%%4
\newpage
\begin{itemize}
    \item
Suppose that the number of births each month, B , is the outcome of a Poisson
process with a rate $\lambda = 1.5$.
\item The probability of obtaining b births per month
exp( − 1.5).1.5 b
is given by the formula Pr[ B = b ] =
b !
Therefore we have
b Pr[ B = b ]
0
1
2
3
4
5
6+ 0.223
0.335
0.251
0.126
0.047
0.014
0.004
%%-- 5Subject CT4 — %%%%%%%%%%%%%%%%%%%%%%%%%%%%%%%%%%%%5 — September 2008 — Examiners’ Report
\item Therefore, if the number of births per month is the
outcome of a Poisson process with a rate of 1.5 per
month the probability of obtaining 5 or more births in
a single month is 0.014 + 0.004 = 0.018.
\item EITHER This is very small OR this is < 0.05
which suggests that the historian may be correct
to suspect something unusual about July 1637.
But only July has a number of births more than 5, and at the 5% level of
statistical significance we expect 1 month in 20 to have such a large
number, then unless we have a prior expectation that July is unusual, we
should be cautious before accepting the historian’s suggestion.
\item (ii)
The assumption that births follow a Poisson process is
unlikely to be entirely realistic
\item EITHER because of the occurrence of multiple births
(twins and triplets)
OR because births tend to occur seasonally
OR because the process might be time inhomogeneous.
\end{itemize}
\end{document}
