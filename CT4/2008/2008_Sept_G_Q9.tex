\documentclass[a4paper,12pt]{article}

%%%%%%%%%%%%%%%%%%%%%%%%%%%%%%%%%%%%%%%%%%%%%%%%%%%%%%%%%%%%%%%%%%%%%%%%%%%%%%%%%%%%%%%%%%%%%%%%%%%%%%%%%%%%%%%%%%%%%%%%%%%%%%%%%%%%%%%%%%%%%%%%%%%%%%%%%%%%%%%%%%%%%%%%%%%%%%%%%%%%%%%%%%%%%%%%%%%%%%%%%%%%%%%%%%%%%%%%%%%%%%%%%%%%%%%%%%%%%%%%%%%%%%%%%%%%

\usepackage{eurosym}
\usepackage{vmargin}
\usepackage{amsmath}
\usepackage{graphics}
\usepackage{epsfig}
\usepackage{enumerate}
\usepackage{multicol}
\usepackage{subfigure}
\usepackage{fancyhdr}
\usepackage{listings}
\usepackage{framed}
\usepackage{graphicx}
\usepackage{amsmath}
\usepackage{chngpage}

%\usepackage{bigints}
\usepackage{vmargin}

% left top textwidth textheight headheight

% headsep footheight footskip

\setmargins{2.0cm}{2.5cm}{16 cm}{22cm}{0.5cm}{0cm}{1cm}{1cm}

\renewcommand{\baselinestretch}{1.3}

\setcounter{MaxMatrixCols}{10}

\begin{document}


%%- Question 9
A company pension scheme, with a compulsory scheme retirement age of 65, is
modelled using a multiple state model with the following categories:


\begin{enumerate}
    \item currently employed by the company
    \item no longer employed by the company, but not yet receiving a pension
    \item pension in payment, pension commenced early due to ill health retirement
    \item pension in payment, pension commenced at scheme retirement age
    \item dead
\end{enumerate} 

\begin{enumerate}
\item (i) Describe the nature of the state space and time space for this process.
\item (ii) Draw and label a transition diagram indicating appropriate transitions between
the states.


\medskip 

For i,j in {1,2,3,4,5}, let:
t
p 1i x
the probability that a life is in state i at age x+t, given they are in state 1 at age
x
$\mu ijx + t$ the transition intensity from state i to state j at age $x+t$
\item (iii) Write down equations which could be used to determine the evolution of t p 1i x
(for each i) appropriate for:
(a)
(b)
(c)
x + t < 65.
x + t = 65.
x + t > 65.
\end{enumerate}

%%%%%%%%%%%%%%%%%%%%%%%%%%%%%%%%%%%%%%%%%%%%%%%%%%%%%%%%%%%%%%%%%%%%%%%%%%%%%%%%%%%%%%%%%%%%
\newpage


9
\begin{itemize}
\item (i)
The state space is discrete with states as given in the question.
The process operates in continuous time.
However, at the compulsory scheme retirement
age of 65 there is a discrete step change.
This is sometimes described as a mixed process.

%%%%%%%%%%%%%%%%%%%%%%%%%%%%%%%%%%%%%%%%%%%%%%%%%

\item (ii)
2
$\mu12$
x + t
No longer
employed
$\mu 24$
x + t
$\mu 14$
x + t
1
Currently
employed
4
$\mu 13$
x + t
3
Ill health
% \mu 45
x + t
% \mu 35
x + t
% \mu 15
x + t
Pensioner
% \mu 25
x + t
5
Dead
\item (iii)
(a)
For x + t < 65
% \frac{\partial}{\partial t} 
% \frac{\partial}{\partial t} 
% \frac{\partial}{\partial t} 
% \frac{\partial}{\partial t} 
% \frac{\partial}{\partial t}  t t 12
13
15
11
p 11

\[\frac{\partial}{\partial t}  t t 12 x = − ( \mu x + t + \mu x + t + \mu x + t ) t p x \]
11
25
12
p 12
\[
\frac{\partial}{\partial t}  t t 13 x = \mu x + t . t p x − \mu x + t . t p x\]
11
35
13
p 13

\[\frac{\partial}{\partial t} x = \mu x + t . t p x − \mu x + t . t p x t t 15\]
11
25
12
35
13
p 15
\[x = \mu x + t . t p x + \mu x + t . t p x + \mu x + t . t p x\]
and t p 14
X is zero.

\item (b)
For x + t = 65
t
12
p 11
% x and t p x become 0 at $x + t = 65+ \delta$
% t +\delta
% p 14
% x =
% t −\delta
% 12
% p 11
% x + t −\delta p x

(c)
For x + t >65
t
12
p 11
x = t p x = 0

\[\frac{\partial}{\partial t} \]
\[\frac{\partial}{\partial t} \]
\[\frac{\partial}{\partial t} \]
\end{itemize}
\end{document}
