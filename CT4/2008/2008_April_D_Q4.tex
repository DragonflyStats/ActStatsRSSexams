
\documentclass[a4paper,12pt]{article}

%%%%%%%%%%%%%%%%%%%%%%%%%%%%%%%%%%%%%%%%%%%%%%%%%%%%%%%%%%%%%%%%%%%%%%%%%%%%%%%%%%%%%%%%%%%%%%%%%%%%%%%%%%%%%%%%%%%%%%%%%%%%%%%%%%%%%%%%%%%%%%%%%%%%%%%%%%%%%%%%%%%%%%%%%%%%%%%%%%%%%%%%%%%%%%%%%%%%%%%%%%%%%%%%%%%%%%%%%%%%%%%%%%%%%%%%%%%%%%%%%%%%%%%%%%%%

\usepackage{eurosym}
\usepackage{vmargin}
\usepackage{amsmath}
\usepackage{graphics}
\usepackage{epsfig}
\usepackage{enumerate}
\usepackage{multicol}
\usepackage{subfigure}
\usepackage{fancyhdr}
\usepackage{listings}
\usepackage{framed}
\usepackage{graphicx}
\usepackage{amsmath}
\usepackage{chngpage}

%\usepackage{bigints}
\usepackage{vmargin}

% left top textwidth textheight headheight

% headsep footheight footskip

\setmargins{2.0cm}{2.5cm}{16 cm}{22cm}{0.5cm}{0cm}{1cm}{1cm}

\renewcommand{\baselinestretch}{1.3}

\setcounter{MaxMatrixCols}{10}

\begin{document}

%%%%%%%%%%%%%%%%%%%%%%%%%%%%%%%%%%%%%%%%%%%%%%%%%%%%%%%%%%%%%%%%%%%%%%%%%%%%%%%%%
4 Describe the benefits and limitations of modelling in actuarial work.

% Solution -Q4
\noindent \textbf{Benefits}
Systems with long time frames can be studied in compressed time,
for example the operation of a pension fund (or other suitable example).
Complex systems with stochastic elements can be studied Different future policies or possible actions can be compared.
In a model of a complex system we can usually get much better control over the experimental conditions so that we can reduce the variance of the results output from the model without upsetting their mean values
Avoids costs and risks of making changes iN_the real world, so we can studY_impact of changing inputs before making decisions.

\noindent \textbf{Limitations}
Model development requires a considerable investment of time and expertise.
In a stochastic model, for any given set of inputs each run gives only estimates of a model’s outputs. So to study the outputs for any given set of inputs, several independent runs of the model are needed.

Models can look impressive when run on a computer so that there is a danger that one gets lulled into a false sense of confidence.
If a model has not passed the tests of validity and verification its impressive output is a poor substitute for its ability to imitate its corresponding real world system.
Models rely heavily oN_the data input. If the data qualitY_is poor or lacks credibility theN_the output from the model is likely to be flawed.
It is important that the users of the model understand the model and the uses to which it can be safely put. There is a danger of using a model as a black box from which it is assumed that all results are valid without considering the appropriateness of using that
model for the particular data input and the output expected.
It is not possible to include all future events in a model. For example a change in legislation could invalidate the results of a model, but may be impossible to predict wheN_the model is constructed. It may be difficult to interpret some of the outputs of the model. They may only be valid in relative, rather than absolute, terms. For example comparing the level of risk
of the outputs associated with different inputs.

%%%%%%%%%%%%%%%%%%%%%%%%%%%%%%%%%%%%%%%%%%%%%%%%%%%%%%%%%%%%%%%%%%%%%%%%%%%%%%%%%%%%%%%%%%%%%%%%%%%%%%%%

\end{document}
