\documentclass[a4paper,12pt]{article}

%%%%%%%%%%%%%%%%%%%%%%%%%%%%%%%%%%%%%%%%%%%%%%%%%%%%%%%%%%%%%%%%%%%%%%%%%%%%%%%%%%%%%%%%%%%%%%%%%%%%%%%%%%%%%%%%%%%%%%%%%%%%%%%%%%%%%%%%%%%%%%%%%%%%%%%%%%%%%%%%%%%%%%%%%%%%%%%%%%%%%%%%%%%%%%%%%%%%%%%%%%%%%%%%%%%%%%%%%%%%%%%%%%%%%%%%%%%%%%%%%%%%%%%%%%%%

\usepackage{eurosym}
\usepackage{vmargin}
\usepackage{amsmath}
\usepackage{graphics}
\usepackage{epsfig}
\usepackage{enumerate}
\usepackage{multicol}
\usepackage{subfigure}
\usepackage{fancyhdr}
\usepackage{listings}
\usepackage{framed}
\usepackage{graphicx}
\usepackage{amsmath}
\usepackage{chngpage}

%\usepackage{bigints}
\usepackage{vmargin}

% left top textwidth textheight headheight

% headsep footheight footskip

\setmargins{2.0cm}{2.5cm}{16 cm}{22cm}{0.5cm}{0cm}{1cm}{1cm}

\renewcommand{\baselinestretch}{1.3}

\setcounter{MaxMatrixCols}{10}

\begin{document}
\begin{enumerate}
%%CT4 A2008—26
\item An investigation was carried out into mortality rates among a certain class of female
pensioners. Crude mortality rates were estimated by single years of age from ages
65–89 years last birthday inclusive. The investigators decided to ask an actuary to
compare the crude rates with a standard table. They calculated the relevant
standardised deviations, printed them out and sent them to the actuary.
\begin{itemize}
    \item Unfortunately, because of a printing error, the right-hand edge of the document
containing the standardised deviations failed to print properly.
\item The actuary was
unable to read the magnitude of the standardised deviations. However, the sign of
each deviation was clear.
\item This revealed that the crude mortality rates were higher
than the standard table rates at ages 65–72 years and 75–84 years inclusive, but that
the crude mortality rates were lower than the standard table rates at ages 73–74 years
and 85–89 years inclusive.
\item The null hypothesis to be tested is that the crude mortality rates come from a
population with underlying mortality consistent with that in the standard table.
\end{itemize}

%%%%%%%%%%%%%%%%
\begin{enumerate}[(a)]
    \item (i) List two statistical tests of the null hypothesis which the actuary could carry
out on the basis of the information received.

    \item (ii) Carry out both tests. For each test, state what feature of the experience it is
specifically testing, and give your conclusion.
\end{enumerate}

\newpage
\item 
In a certain small country all listed companies are required to have their accounts
audited on an annual basis by one of the three authorised audit firms (A, B and C).
The terms of engagement of each of the audit firms require that a minimum of two
annual audits must be conducted by the newly appointed firm. Whenever a company
is able to choose to change auditors, the likelihood that it will retain its auditors for a
further year is (80\%, 70\%, 90\%) where the current auditor is (A,B,C) respectively. If
changing auditors a company is equally likely to choose either of the alternative firms.
(i)
(ii)
A company has just changed auditors to firm A. Calculate the expected
number of audits which will be undertaken before the company changes
auditors again.

Formulate a Markov chain which can be used to model the audit firm used by
a company, specifying:
(a)
(b)
\begin{enumerate}[(i)]
\item the state space
\item the transition matrix
\end{enumerate}
(iii)
Calculate the expected proportion of companies using each audit firm in the
long term.
\end{enumerate}


%%-- [Total 11]
%%%%%%%%%%%%%%%%%%%%%%%%%%%%%%%%%%%%%%%%%%%%%%%%%%%%%%%%%%%%%%
\newpage
7
(i)
Required number
∞
= ∑ probability ith audit takes place prior to changing auditors
i = 1
= 1 + 1 + 0.8 + 0.8 2 +0.8 3 +........
= 1 + 1/(1 - 0.8) = 6
(ii)
The transition probabilities depend on
whether it is the first year with the
current auditors, so need additional states to cover this.
State space = ${A L , A, B L , B, C L , C}$ where subscript L
indicates locked in to the current auditor.
Transition matrix A is
A L
A
B L
B A L
A
0
1
0
0.8
0
0
0.15 0 B L
0
0.1
0
0
C L
C 0
0.05 0
0.05
0
0
B C L
0
0
0
0.1
1
0
0.7 0.15
0
0
0
0
C
0
0
0
0
1
0.9
$$\bordermatrix{ &c_1&c_2&\ldots &c_n\cr
                r_1&a_{11} &  0  & \ldots & a_{1n}\cr
                r_2& 0  &  a_{22} & \ldots & a_{2n}\cr
                r_4& 0  &   0       &\ldots & a_{nn}}$$
                
This is a Markov chain because the probability
of future transitions is independent of history
prior to arrival in current state (Markov property).
(iii)
Need to find stationary distribution
\pi which by definition satisfies:
\[\pi = \pi_A\]

\begin{itemize}
    \item[(1)] ${ \displaystyle0.15 \pi_B + 0.05 \pi_C = \pi_A L }$
    \item[(2)] ${ \displaystyle \pi_A L + 0.8 \pi_A = \pi_A }$
    \item[(3)] ${ \displaystyle 0.1 \pi_A + 0.05 \pi_C = \pi_B L }$
    \item[(4)] ${ \displaystyle \pi_B L + 0.7 \pi_B = \pi_B }$
    \item[(5)] ${ \displaystyle 0.1 \pi_A + 0.15 \pi_B = \pi_C L }$
    \item[(6)] ${ \displaystyle \pi_C L + 0.9 \pi_C = \pi_C }$
\end{itemize}

Page 11%%%%%%%%%%%%%%%%%%%%%%%%%%%%%%%%5 — April 2008 — Examiners’ Report
Combining (1) and (2), (3) and (4), and (5) and (6)
\begin{itemize}
    \item
${ \displaystyle 0.15 \pi_B + 0.05 \pi_C = 0.2 \pi_A }$ (1A)
\item ${ \displaystyle 0.1 \pi_A + 0.05 \pi_C = 0.3 \pi_B}$ (3A)
\item ${ \displaystyle 0.1 \pi_A + 0.15 \pi_B = 0.1 \pi_C }$(5A)
\end{itemize}

(1A) – (3A) gives
$\pi_A = 1.5 \pi_B$
(3A) – (5A) produces
$\pi_C = 3 \pi_B$
∑
\pi i = 1 implies
i
(1.5 + 0.3 + 1 + 0.3 + 3 + 0.3) \pi_B = 1
⎛ \pi_A L ⎞ ⎛ 0.046875 ⎞
⎜
⎟
⎜ \pi_A ⎟ ⎜ ⎜ 0.234375 ⎟ ⎟
⎜
⎟ ⎜
\pi
0.046875 ⎟
So ⎜ B L ⎟ = ⎜
⎜ \pi ⎟ ⎜ 0.15625 ⎟ ⎟
⎜ B ⎟ ⎜
⎜ \pi_C L ⎟ ⎜ 0.046875 ⎟ ⎟
⎜ ⎜
⎟ ⎟ ⎜ 0.46875 ⎟ ⎠
⎝ \pi_C ⎠ ⎝
And proportions using (A,B,C) are
\[(0.28125, 0.203125, 0.515625).\]
Page 12%%%%%%%%%%%%%%%%%%%%%%%%%%%%%%%%5 — April 2008 — Examiners’ Report
\end{document}
