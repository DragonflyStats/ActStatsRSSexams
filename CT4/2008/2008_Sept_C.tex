\documentclass[a4paper,12pt]{article}

%%%%%%%%%%%%%%%%%%%%%%%%%%%%%%%%%%%%%%%%%%%%%%%%%%%%%%%%%%%%%%%%%%%%%%%%%%%%%%%%%%%%%%%%%%%%%%%%%%%%%%%%%%%%%%%%%%%%%%%%%%%%%%%%%%%%%%%%%%%%%%%%%%%%%%%%%%%%%%%%%%%%%%%%%%%%%%%%%%%%%%%%%%%%%%%%%%%%%%%%%%%%%%%%%%%%%%%%%%%%%%%%%%%%%%%%%%%%%%%%%%%%%%%%%%%%

\usepackage{eurosym}
\usepackage{vmargin}
\usepackage{amsmath}
\usepackage{graphics}
\usepackage{epsfig}
\usepackage{enumerate}
\usepackage{multicol}
\usepackage{subfigure}
\usepackage{fancyhdr}
\usepackage{listings}
\usepackage{framed}
\usepackage{graphicx}
\usepackage{amsmath}
\usepackage{chngpage}

%\usepackage{bigints}
\usepackage{vmargin}

% left top textwidth textheight headheight

% headsep footheight footskip

\setmargins{2.0cm}{2.5cm}{16 cm}{22cm}{0.5cm}{0cm}{1cm}{1cm}

\renewcommand{\baselinestretch}{1.3}

\setcounter{MaxMatrixCols}{10}

\begin{document}
\begin{enumerate}
8
\item (ii) Derive an expression for the ratio of the variance of the number of claims
arising compared with that if each policy covered an independent life.

\item (iii) Explain how the expression derived in \item (ii) could be used in practice.

[Total 8]
A No-Claims Discount system operated by a motor insurer has the following four
levels:
Level 1:
Level 2:
Level 3:
Level 4:
0% discount
25% discount
40% discount
60% discount
The rules for moving betweeN_these levels are as follows:
• Following a year with no claims, move to the next higher level, or remain at
level 4.
• Following a year with one claim, move to the next lower level, or remain at
level 1.
• Following a year with two or more claims, move dowN_two levels, or move to
level 1 (from level 2) or remain at level 1.
For a given policyholder in a given year the probability of no claims is 0.85 and the
probability of making one claim is 0.12.
\item (i) Write dowN_the transition matrix of this No-Claims Discount process.
\item (ii) Calculate the probability that a policyholder who is currently at level 2 will be
at level 2 after:
(a)
(b)
\item (iii)

one year.
two years.

Calculate the long-run probability that a policyholder is in discount level 2.

[Total 9]
CT4 S2008—5
%%%%%%%%%%%%%%%%%%%%%%%%%%%%%%%%%%%%%%%%%%%%%%%%%%%%%%%%%%%%%%%%%%%%%%%%%%55
9
A company pension scheme, with a compulsory scheme retirement age of 65, is
modelled using a multiple state model with the following categories:
1
2
3
4
5 currently employed by the company
no longer employed by the company, but not yet receiving a pension
pension in payment, pension commenced early due to ill health retirement
pension in payment, pension commenced at scheme retirement age
dead
\item (i) Describe the nature of the state space and time space for this process.
\item (ii) Draw and label a transition diagram indicating appropriate transitions between
the states.


For i,j in {1,2,3,4,5}, let:
t
p 1i x
the probability that a life is in state i at age x+t, giveN_they are in state 1 at age
x
\mu ijx + t the transition intensity from state i to state j at age x+t
\item (iii) Write down equations which could be used to determine the evolution of t p 1i x
(for each i) appropriate for:
(a)
(b)
(c)
x + t < 65.
x + t = 65.
x + t > 65.
[6]
[Total 10]

8
\item (i)
The transition matrix of the process is
⎛ 0.15
⎜
0.15
P = ⎜
⎜ 0.03
⎜
⎝ 0
\item (ii)
0.85
0
0.12
0.03
0
0.85
0
0.12
0 ⎞
⎟
0 ⎟
0.85 ⎟
⎟
0.85 ⎠
(a) For the one year transition, p 22 = 0, as can be seen
from above (or is obvious from the statement).
(b) The second order transition matriX_is
⎛ 0.15 2 + 0.85 × 0.15
⎜
⎜ 0.15 2 + 0.85 × 0.03
⎜
⎜ 0.03 × 0.15 + 0.12 × 0.15
⎜
⎜ 0.03 × 0.15 + 0.12 × 0.03
⎝
⎛ 0.15
⎜
0.048
= ⎜
⎜ 0.0225
⎜
⎝ 0.0081
⎞
⎟
⎟
0.85 × 0.15 + 0.85 × 0.12 0
0.85 2
⎟
⎟
0.85 × 0.03 × 2
0.85 × 0.12 × 2
0.85 2
⎟
0.12 2 + 0.85 × 0.03
0.85 × 0.03 + 0.85 × 0.12 0.12 × 0.85 + 0.85 2 ⎟ ⎠
0.85 2
0.85 × 0.15
0.1275
0.2295
0.051
0.0399
0.7225
0
0.204
0.1275
0
0
⎞
⎟
0.7225 ⎟
0.7225 ⎟
⎟
0.8245 ⎠
hence the required probabilitY_is 0.2295.
\item (iii)
In matrix form, the equation we need to solve is \piP = \pi,
where \pi is the vector of equilibrium probabilities.
This reads
0.15 \pi_1 + 0.15 \pi_2 + 0.03 \pi_3
0.85 \pi_1 +
+ 0.85 \pi_2
= \pi_1
+ 0.12 \pi_3 + 0.03 \pi_4 = \pi_2
+ 0.12 \pi_4 = \pi_3
0.85 \pi_3 + 0.85 \pi_4 = \pi_4
(1)
(2)
(3)
(4)
Discard the first of these equations and use also the fact that
∑ i= 1 \pi i = 1 .
4
Then, we obtain first from (4) that 0.85 \pi_3 = 0.15 \pi_4
or, that \pi_4 = 17 \pi_3 / 3
Page 9 — Models Core Technical — September 2008 — Examiners’ Report
Substituting in (3) this gives
0.85 \pi_2 + 0.12 ×
17
\pi_3 = \pi_3 ⇒ \pi_3 = 2.65625 \pi_2
3
(2) now yields that
0.85 \pi_1 = \pi_2 − 0.12 \pi_3 − 0.03 \pi_4 =
1
\pi_3 − 0.12 \pi_3 − 0.17 \pi_3 = 0.0865 \pi_3 ,
2.65625
so that finally we get \pi_1 = 0.10173 \pi_3 .
Using now that the probabilities must add up to one, we obtain
\pi_1 + \pi_2 + \pi_3 + \pi_4 = (0.10173 + 0.3765 + 1 + 5.666) \pi_3 = 1,
or that \pi_3 = 0.13996.
Solving back for the other variables we get that
\pi_1 = 0.01424, \pi_2 = 0.05269, \pi_4 = 0.79311
The long-run probability that the motorist is in discount level 2 is therefore
0.05269.
9
\item (i)
The state space is discrete with states as given iN_the question.
The process operates in continuous time.
However, at the compulsory scheme retirement
age of 65 there is a discrete step change.
This is sometimes described as a mixed process.
Page 10 — Models Core Technical — September 2008 — Examiners’ Report
\item (ii)
2
\mu 12
x + t
No longer
employed
\mu 24
x + t
\mu 14
x + t
1
Currently
employed
4
\mu 13
x + t
3
Ill health
\mu 45
x + t
\mu 35
x + t
\mu 15
x + t
Pensioner
\mu 25
x + t
5
Dead
\item (iii)
(a)
For x + t < 65
∂
∂
∂
∂
∂ t t 12
13
15
11
p 11
x = − ( \mu x + t + \mu x + t + \mu x + t ) t p x
∂ t t 12
11
25
12
p 12
x = \mu x + t . t p x − \mu x + t . t p x
∂ t t 13
11
35
13
p 13
x = \mu x + t . t p x − \mu x + t . t p x
∂ t t 15
11
25
12
35
13
p 15
x = \mu x + t . t p x + \mu x + t . t p x + \mu x + t . t p x
and t p 14
X_is zero.
(b)
For x + t = 65
t
12
p 11
x and t p x become 0 at x + t = 65+ δ
t +δ
p 14
x =
t −δ
12
p 11
x + t −δ p x
Page 11 — Models Core Technical — September 2008 — Examiners’ Report
(c)
For x + t >65
t
12
p 11
x = t p x = 0
∂
∂
∂

