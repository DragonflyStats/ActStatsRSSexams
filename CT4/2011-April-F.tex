\documentclass[a4paper,12pt]{article}

%%%%%%%%%%%%%%%%%%%%%%%%%%%%%%%%%%%%%%%%%%%%%%%%%%%%%%%%%%%%%%%%%%%%%%%%%%%%%%%%%%%%%%%%%%%%%%%%%%%%%%%%%%%%%%%%%%%%%%%%%%%%%%%%%%%%%%%%%%%%%%%%%%%%%%%%%%%%%%%%%%%%%%%%%%%%%%%%%%%%%%%%%%%%%%%%%%%%%%%%%%%%%%%%%%%%%%%%%%%%%%%%%%%%%%%%%%%%%%%%%%%%%%%%%%%%

\usepackage{eurosym}
\usepackage{vmargin}
\usepackage{amsmath}
\usepackage{graphics}
\usepackage{epsfig}
\usepackage{enumerate}
\usepackage{multicol}
\usepackage{subfigure}
\usepackage{fancyhdr}
\usepackage{listings}
\usepackage{framed}
\usepackage{graphicx}
\usepackage{amsmath}
\usepackage{chngpage}

%\usepackage{bigints}
\usepackage{vmargin}

% left top textwidth textheight headheight

% headsep footheight footskip

\setmargins{2.0cm}{2.5cm}{16 cm}{22cm}{0.5cm}{0cm}{1cm}{1cm}

\renewcommand{\baselinestretch}{1.3}

\setcounter{MaxMatrixCols}{10}

\begin{document}
\begin{enumerate}[1]
[Total 14]
Farmer Giles makes hay each year and he makes far more than he could possibly store
and use himself, but he does not always sell it all. He has decided to offer incentives
for people to buy large quantities so it does not sit in his field deteriorating. He has
devised the following “discount” scheme.
He has a Base price, B of £8 per bale. Then he has three levels of discount: Good
price, G , is a 10% discount, Loyalty price, L is a 20% discount and Super price, S , is a
25% discount on the Base price.
• Customers who increase their order compared with last year move to one higher
discount level, or remain at level S .
• Customers who maintain their order from last year stay at the same discount level.
• Customers who reduce their order from last year drop one level of discount or
remain at level B provided that they maintained or increased their order the
previous year.
CT4 A2011—6•
Customers who reduce their order from last year drop two levels of discount if
they also reduced their order last year, subject to remaining at the lowest level B .
(i) Explain why a process with the state space of { B , G , L , S } does not display the
Markov property.
[2]
(ii) (a)
Define any additional state(s) required to model the system
with the Markov property.
(b)
Construct a transition graph of this Markov process clearly labelling
all the states.
[3]
Farmer Giles thinks that each year customers have a 60% likelihood of increasing
their order and a 30% likelihood of reducing it, irrespective of the discount level they
are currently in.
(iii)
(a) Write down the transition matrix for the Markov process.
(b) Calculate the stationary distribution.
(c) Hence calculate the long run average price he will get for each bale of
hay.
[8]
(iv) Calculate the probability that a customer who is currently paying the Loyalty
price, L , will be paying L in two years’ time.
[3]
(v) Suggest reasons why the assumptions Farmer Giles has made about his
customers’ behaviour may not be valid.
[3]
[Total 19]
END OF PAPER
CT4 A2011—7

%%%%%%%%%%%%%%%%%%%%%%%%%%%%%%%%%%%%%%%%%%%%%%%%%%%%%%%%%%%%%%%%%%%%%%%%%%%%%%%%%%%%%%%%%%%%
Question 12
(i)
Past history is needed to decide where to go in the chain.
If a customer is at L and reduces his or her order, you need to know what level of
discount he was at the previous year to determine whether he or she drops one or two
levels of discount.
(ii)
The L level needs to be split into two.
L + is Loyalty Price with no reduction in demand last year
L – is Loyalty Price with reduction in demand last year
0.4
0.1
0.6
B
G
0.6
L +
0.3
0.3
0.7
0.1
0.6
S
0.3
0.1
L –
0.3
The probabilities were not required for full credit for this diagram.
(iii)
(a)
B
G
L +
L –
π 1 π 2 π 3 π 4
0
⎛ 0.4 0.6 0
⎜
⎜ 0.3 0.1 0.6 0
⎜ 0 0.3 0.1 0
⎜
⎜ 0.3 0 0.1 0
⎜ 0
0
0 0.3
⎝
Page 22
S
π 5
0 ⎞
⎟
0 ⎟
0.6 ⎟
⎟
0.6 ⎟
0.7 ⎟ ⎠
0.6Subject CT4 (Models Core Technical) — Examiners’ Report, April 2011
(b)
π = π P
π 1
π 2
π 3
π 4
π 5
π 1
= 0.4 π 1 + 0.3 π 2 + 0.3 π 4
= 0.6 π 1 + 0.1 π 2 + 0.3 π 3
= 0.6 π 2 + 0.1 π 3 + 0.1 π 4
= 0.3 π 5
= 0.6 π 3 + 0.6 π 4 + 0.7 π 5
+ π 2 + π 3 + π 4 + π 5 = 1
(1)
(2)
(3)
(4)
(5)
= 0.3 π 5
(4) gives π 4
(5) gives 0.6π 3 = 0.3 π 5 – 0.6(0.3π 5 )
= 0.12 π 5
= 0.2 π 5
π 3
(3) gives 0.6 π 2 =
=
=
=
π 2
(2) gives 0.6 π 1 = 0.9 π 2 – 0.3 π 3
= 0.9(0.25) π 5 – 0.3(0.2) π 5
= 0.225 π 5 – 0.06 π 5
= 0.165 π 5
= 0.275 π 5
π 1
0.9 π 3 – 0.1 π 4
0.18 π 5 – 0.03 π 5
0.15 π 5
0.25 π 5
π 5 (0.275 + 0.25 + 0.2 + 0.3 + 1) = 1
π 5 = 1 / 2.025
= 0.49382716
π 1 = 0.13580
π 2 = 0.12346
π 3 = 0.09877
π 4 = 0.14815
π 5 = 0.49383
(c)
)
) 0.24692
OR
OR
OR
OR
OR
11/81
10/81
8/81
12/81
40/81
Average price for a bale of hay is
£8 × (1 × 0.1358 + 0.9 × 0.12346 + 0.8 × (0.09877 + .14815) + 0.75 ×.49383 )
= £6.5181
Page 23Subject CT4 (Models Core Technical) — Examiners’ Report, April 2011
(iv)
0
0 ⎞
0
0 ⎞ ⎛ 0.4 0.6 0
⎛ 0.4 0.6 0
⎟
⎜
⎟ ⎜
0 ⎟
0 ⎟ ⎜ 0.3 0.1 0.6 0
⎜ 0.3 0.1 0.6 0
⎜ 0 0.3 0.1 0 0.6 ⎟ ⎜ 0 0.3 0.1 0 0.6 ⎟
⎟
⎜
⎟ ⎜
⎜ 0.3 0 0.1 0 0.6 ⎟ ⎜ 0.3 0 0.1 0 0.6 ⎟
⎜ 0
0
0 0.3 0.7 ⎟ ⎠
0
0 0.3 0.7 ⎟ ⎠ ⎜ ⎝ 0
⎝
−
− ⎞ ⎛ .34
.24 + .06
.36
⎛ .16 + .18
⎜
⎟ ⎜
.36 ⎟ ⎜ .15
⎜ .12 + .03 .18 + .01 + .18 .06 + .06 −
⎜
.09 .03 + .03
.18 + .01 .18
.18 ⎟ ⎜ .09
=⎜
⎟ =⎜
.12 .18 + .03
.01
.18 .06 + .42 ⎟ ⎜ .12
⎜
⎜
.09 −
.03 .21
.18 + .42 ⎟ ⎜ .09
⎜ ⎜
⎟ ⎟ ⎜ ⎜
⎝
⎠ ⎝
− ⎞
.3 .36 −
⎟
.37 .12 − .36 ⎟
.06 .19 .18 .48 ⎟
⎟
.21 .01 .18 .48 ⎟
− .03 .21 .67 ⎟
⎟ ⎟
⎠
THEN ALTERNATIVE 1
Using the long-run probabilities of being in L + and L ― , therefore
the chance of being at L in two years’ time is
(0.19 + 0.18)*0.4 + (0.18 + 0.01)*0.6 = 0.262.
OR ALTERNATIVE 2
Assuming there is an equal probability of being L + and L ― ,
the chance of being at L in two years’ time is
(0.19 + 0.18)*0.5 + (0.18 + 0.01)*0.5 = 0.28.
OR ALTERNATIVE 3
We do not know the relative proportions in L + and L ― ,
but for those in L + the chance of being in L in two years’ time is 0.19 + 0.18 = 0.37,
and for those in L + the chance of being in L in two years’ time is 0.18 + 0.01 = 0.19.
OR ALTERNATIVE 4
We do not know the relative proportions in L + and L ― ,
and so it is not possible to evaluate the overall probability that a customer in L will be
in L in two years’ time.
(v)
A constant figure takes no account of the amount of hay which Farmer Giles has to
sell: for example a drought year could produce very little which one large customer
may buy in its entirety.
Page 24Subject CT4 (Models Core Technical) — Examiners’ Report, April 2011
The amount of hay in the local market is important.
Another supplier may try a heavy discounted year to get into the market.
Customers’ behaviour may depend on the discount level they are at.
There may be national trends in the demand for hay e.g. a sudden trend towards
vegetarianism.
A 60% chance of increasing may be implausible, as field space is likely to be limited,
so a constant increase in numbers unlikely.
Customers’ behaviour may depend on the amount of hay they typically purchase.
A common error in part (ii) was to split state G into two states as well as splitting state L.
This is not required to model the system with the Markov property and so was penalised.
However, candidates who split state G and then followed through with a correct matrix in
part (iii)(a) and correct solutions in part (iii)(b) were not penalised again. Note that splitting
state G should produce the same answer to part (iii)(b), though more work will be needed!
In part (iv) candidates who adopted ALTERNATIVE 4, in which they declined to give an
overall answer on the grounds that they do not know the proprotions in states L + and L ― ,
were only given credit if they presented a reasoned argument with evidence.
In part (v) credit was given for other sensible suggestions.
END OF EXAMINERS’ REPORT
Page 25
