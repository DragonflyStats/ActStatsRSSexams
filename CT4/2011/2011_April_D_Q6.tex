\documentclass[a4paper,12pt]{article}

%%%%%%%%%%%%%%%%%%%%%%%%%%%%%%%%%%%%%%%%%%%%%%%%%%%%%%%%%%%%%%%%%%%%%%%%%%%%%%%%%%%%%%%%%%%%%%%%%%%%%%%%%%%%%%%%%%%%%%%%%%%%%%%%%%%%%%%%%%%%%%%%%%%%%%%%%%%%%%%%%%%%%%%%%%%%%%%%%%%%%%%%%%%%%%%%%%%%%%%%%%%%%%%%%%%%%%%%%%%%%%%%%%%%%%%%%%%%%%%%%%%%%%%%%%%%

\usepackage{eurosym}
\usepackage{vmargin}
\usepackage{amsmath}
\usepackage{graphics}
\usepackage{epsfig}
\usepackage{enumerate}
\usepackage{multicol}
\usepackage{subfigure}
\usepackage{fancyhdr}
\usepackage{listings}
\usepackage{framed}
\usepackage{graphicx}
\usepackage{amsmath}
\usepackage{chngpage}

%\usepackage{bigints}
\usepackage{vmargin}

% left top textwidth textheight headheight

% headsep footheight footskip

\setmargins{2.0cm}{2.5cm}{16 cm}{22cm}{0.5cm}{0cm}{1cm}{1cm}

\renewcommand{\baselinestretch}{1.3}

\setcounter{MaxMatrixCols}{10}

\begin{document}
\begin{enumerate}
6
State an appropriate method of graduation for each of the three classes of lives
and, for each class, briefly explain your choice.

%%-- [Total 6]
A study of the mortality of a certain species of insect reveals that for the first 30 days
of life, the insects are subject to a constant force of mortality of 0.05. After 30 days,
the force of mortality increases according to the formula:
\[\mu 30 + x = 0.05exp(0.01 x ) ,\]
where x is the number of days after day 30.
7
\begin{enumerate}[(a)]
\item (i) Calculate the probability that a newly born insect will survive for at least 10
days.
\item 
(ii) Calculate the probability that an insect aged 10 days will survive for at least a
further 30 days.
\item  Calculate the age in days by which 90 per cent of insects are expected to have
died.
\end{enumerate}

%%%%%%%%%%%%%%%%%%%%%%%%%%%%%%%%%%%%%%%%%%%%%%%%%%%%%%%%5
\newpage
%%--- Question 6
(i)
The probability that an insect will survive for 10 days, 10 p 0 , is given by the formula

%%-- Question 6

The probability that an insect will survive for 10 days, 10 p 0 , is given by the formula

\[ _{10}p_{0} = \mbox{exp} left[ - \int^{10}_{0} \mu_x dx \right)   \]

Since the force of mortality is constant up to age 30 days at a value of 0.05,

\begin{eqnarray*} _{10}p_{0} &=& \mbox{exp} left[ - \int^{10}_{0} -0.005 dx \right)   \\
&=& \mbox{exp} \left( -\[eflt[ -0.05x \right]^{10}_{0} right) \\
&=& \mbox{exp} \left( -0.5 right) \\
&=& 0.6065 \\
\end{eqnarray*}

(
(ii)
)
The probability that an insect 10 days old will survive for a further 30 days (that is to
exact age 40 days) is given by
\[ _{30}p_{10} = \mbox{exp} left[ - \int^{40}_{10} \mu_x dx \right)   \]


Since
$_{30}p_{10} = _{20}p_{10} \times _{10}p_{30} $, this is equal to


%%- check line 3 below
\begin{eqnarray*} 
\mbox{exp} left[ - \int^{30}_{10} -0.05 dx \right) \times \mbox{exp} left[ - \int^{10}_{0} -0.01 dx \right)    \\
&=& \mbox{exp} \left( -\[left[ -0.05x \right]^{30}_{10} right) \times \mbox{exp} \left( -\left[ -0.05x \right]^{10}_{0} right)\\
&=& e^{-(1.5 - 0.05)} \times e^{5 exp(0.1)- 5(exp(0))} \\
&=& \mbox{exp}(-1) \times \mbox{exp}(-0.5258) \\
&=& 0.3679 \times 0.5911 \\
&=& 0.2174\\
\end{eqnarray*}
(ii)
)
The probability that an insect 10 days old will survive for a further 30 days (that is to exact age 40 days) is given by
⎛ 40
⎞
⎜ − \int \mu x dx ⎟ .
p
=
exp
30 10
⎜
⎟
⎝ 10
⎠
Since
30 p 10
= 20 p 10 . 10 p 30 , this is equal to

(iii)
If the required age is 30+a, then we have
30 + a
p 0 =
30
p 0 . a p 30 = 0.1 .
Now
⎡ 30
⎤
30 p 0 = exp ⎢ − \int 0.05 dx ⎥ = exp( − 1.5) = 0.2231 .
⎣ 0
⎦
So
6
a
p 30 =
0.1
= 0.4483 .
0.2231Subject CT4 %%%%%%%%%%%%%%%%%%%%%%%%%%%%%%%%%%%%%%%%%% — Examiners’ Report, April 2011
Using the result from part (ii), we have
⎛ ⎡ 0.05 0.01 x ⎤ a ⎞
⎛ ⎡ 0.05 0.01 a 0.05 ⎤ ⎞
0.01 a
− ⎢
e ⎥ ⎟ = exp ⎜ − ⎢
e
−
a p 30 = exp ⎜
⎟ = exp(5 − 5 e )
⎥
⎜ ⎣ 0.01
⎟
0.01
0.01
⎦ 0 ⎠
⎦ ⎠
⎝ ⎣
⎝
Therefore
e 5(1 − exp(0.01 a )) = 0.4483 ,
whence
log e 0.4483 = 5(1 − e 0.01 a ) ,
so that
1 − e 0.01 a = − 0.1605
e 0.01 a = 1.1605
0.01 a = 0.1488
a = 14.88
Therefore the required age is 14.88+30 = 44.88 days.

%Most candidates answered part (i) of this question correctly. Part (ii) was less well answered, and only a minority of candidates managed to obtain the correct answer to part (iii). A common error was to use the limits 40 and 30 when integrating 0.05exp(0.01x).

\newpage


\end{document}
