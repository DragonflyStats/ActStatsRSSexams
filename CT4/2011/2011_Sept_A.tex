\documentclass[a4paper,12pt]{article}

%%%%%%%%%%%%%%%%%%%%%%%%%%%%%%%%%%%%%%%%%%%%%%%%%%%%%%%%%%%%%%%%%%%%%%%%%%%%%%%%%%%%%%%%%%%%%%%%%%%%%%%%%%%%%%%%%%%%%%%%%%%%%%%%%%%%%%%%%%%%%%%%%%%%%%%%%%%%%%%%%%%%%%%%%%%%%%%%%%%%%%%%%%%%%%%%%%%%%%%%%%%%%%%%%%%%%%%%%%%%%%%%%%%%%%%%%%%%%%%%%%%%%%%%%%%%

\usepackage{eurosym}
\usepackage{vmargin}
\usepackage{amsmath}
\usepackage{graphics}
\usepackage{epsfig}
\usepackage{enumerate}
\usepackage{multicol}
\usepackage{subfigure}
\usepackage{fancyhdr}
\usepackage{listings}
\usepackage{framed}
\usepackage{graphicx}
\usepackage{amsmath}
\usepackage{chngpage}

%\usepackage{bigints}
\usepackage{vmargin}

% left top textwidth textheight headheight

% headsep footheight footskip

\setmargins{2.0cm}{2.5cm}{16 cm}{22cm}{0.5cm}{0cm}{1cm}{1cm}

\renewcommand{\baselinestretch}{1.3}

\setcounter{MaxMatrixCols}{10}

\begin{document}
\begin{enumerate}
© Institute and Faculty of Actuaries1
The diagrams below show three Markov chains, where arrows indicate a non-zero
transition probability.
State whether each of the chains is:
(a)
(b)
irreducible.
periodic, giving the period where relevant.

A.
State 1
State 2
B.
State 1 State 2
State 1 State 2
State 3 State 4
State 3
C.
2
3
(i) Describe what is represented by each of the central rate of mortality, m x , and
the initial rate of mortality, q x .

(ii) State the circumstance in which m x = μ x .

[Total 3]
Describe how a strictly stationary stochastic process differs from a weakly stationary
stochastic process.

%%CT4 S2011—24
Question 4
A new weedkiller was tested which was designed to kill weeds growing in grass. The
weedkiller was administered via a single application to 20 test areas of grass. Within
hours of applying the weedkiller, the leaves of all the weeds went black and died, but
after a time some of the weeds re-grew as the weedkiller did not always kill the roots.
The test lasted for 12 months, but after six months five of the test areas were
accidentally ploughed up and so the trial on these areas had to be discontinued. None
of these five areas had shown any weed re-growth at the time they were ploughed up.
5
\begin{itemize}
\item Ten of the remaining 15 areas experienced a re-growth of weeds at the following
durations (in months): \[\{1, 2, 2, 2, 5, 5, 8, 8, 8, 8.\}\]
\item  Five areas still had no weed re-growth when the trial ended after 12 months.
\end{itemize}
\begin{enumerate}[(i)]
\item Describe, giving reasons, the types of censoring present in the data.
\item (ii) Estimate the probability that there is no re-growth of weeds nine months after application of the weedkiller using either the Kaplan-Meier or the Nelson-
Aalen estimator.
\end{enumerate}
%%%%%%%%%%%%%%%%%%%%%%%%%%%%%%%%%%%%%%5
%% [Total 6]
\begin{enumerate}[(i)]
\item (i) List the factors which should be considered in assessing the suitability of a
model for a particular exercise.

\item (ii) Assess the suitability of a multiple state model with three states: Healthy, Sick
and Dead, for estimating the transition intensities in an analysis of claims for
sickness benefit, in the light of your answer to (i).

%% [Total 7]
%% CT4 S2011—3
\end{enumerate}
%%%%%%%%%%%%%%%%%%%%%%%%%%%%%%%%%%%%%%%%%%%%%%%%%%%%%%%%%%%%%%%%%%%%%%%%%
\newpage
Question 1
(a) A
B
C Yes, irreducible.
No, not irreducible.
Yes, irreducible.
(b) A
B
C Yes, period is 2
No, not periodic.
No, not periodic.

% This question was well answered, although many candidates failed to identify that C was aperiodic.
%%%%%%%%%%%%%%%%%%%%%%%%%%%%%%%%%%%%%%%%%%%%%%%%%%%%
Question 2
(i)
m x is the probability that a life alive between exact ages x and x dies
OR
m x is the probability of dying between exact ages x and x per person-year lived
between exact ages x and x
q x is the probability that a life alive at exact age x dies before exact age x
(ii)

$m_x$ and $μ_x$ are equal when the force of mortality μ x+t is constant for 0 ≤ t < 1.
Answers to this question were disappointing. In part (i) some candidates defined m x as
q x
For full credit, candidates who did this were required to explain what this
1
∫ t p x dt
0
expression means (e.g. by stating that t p x is the expected amount of time spent alive between
x and x+1 by a life alive at age x).
\newpage

%%%%%%%%%%%%%%%%%%%%%%%%
Question 3
\begin{itemize}
\item A stochastic process is said to be strictly stationary if the joint distributions of
X t 1 , X t 2 ,..., X t n and X t + t 1 , X t + t 2 ,..., X t + t n are identical for all t , $t 1 , t 2 ,..., t n$ in the time set J and
for all integers n .
\item This means that the statistical properties of the process remain unchanged as time elapses.
Weak stationarity requires that the mean of the process, $m ( t ) = E ( X t )$ , is constant, and

%%Page 3%%%%%%%%%%%%%%%%%%%%%%%%%%%%%%%%%%%%%%%%%%%%%%%%%%%5 — Examiners’ Report, September 2011
\item EITHER that the covariance of the process $E [( X s − m ( s ))( X t − m ( t ))]$ depends only on the
time difference $t – s$ .
\end{itemize}
OR \[Cov( X ( t 1 ), X ( t 2 )) = Cov( X ( t 1 + h ), X ( t 2 + h )\] for all t 1 , t 2 and h > 0.
Strict stationarity is a stringent condition which is hard to test, weak stationarity is a less
stringent condition but easier to test in practice.
This question was well answered. The last sentence was not required for full credit.

%%%%%%%%%%%%%%%%%%%%%%%%%%%%%%%%%%%%%%%%%%%
\newpage
Question 4
(i)
\begin{itemize}
\item Right censoring: some areas never developed new weeds.
Type I censoring as the study lasts for a pre-determined time.
\item Random censoring as the accidental ploughing happened at a time which was not pre-
determined.
\item Interval censoring as we do not know exactly when in each month the weed re-growth
happened.
\item Non-informative censoring as the fact that an area was ploughed up tells us nothing
about the duration to weed re-growth in any of the remaining areas.
\end{itemize}
(ii)
EITHER
Kaplan-Meier estimator
t j N j D j
0
1
2
5
8 20
20
19
16
9 0
1
3
2
4
C j
0
0
0
5
5
D j
N j
–
1/20
3/19
2/16
4/9
1 −
D j
N j
1
19/20
16/19
14/16
5/9
Kaplan-Meier estimate of the survival function at 9 months is given by product of
D j
1 −
for t j < 9
N j
which is
Page 4
19 16 14 5 7
. . . = = 0.3889 .
20 19 16 9 18%%%%%%%%%%%%%%%%%%%%%%%%%%%%%%%%%%%%%%%%%%%%%%%%%%%5 — Examiners’ Report, September 2011
OR
Nelson-Aalen estimator
t j N j D j C j
0
1
2
5
8 20
20
19
16
9 0
1
3
2
4 0
0
0
5
5
D j
D j
N j ∑ N j
–
1/20
3/19
2/16
4/9 0
0.0500
0.2079
0.3329
0.7773
Nelson-Aalen estimate of the survival function at 9 months is given by
⎛
D j ⎞
exp ⎜ − ∑
⎟ for t j < 9
⎜
⎟
N
j ⎠
⎝
which is exp(−0.7773) = 0.4596.

% Many candidates scored highly on this question. In part (i) the reason was needed for credit.
% Just mentioning the type of censoring without giving a reason was not awarded any marks.
% In part (ii) some indication of how the estimate was arrived at (normally a statement of the formula being applied) was needed for full credit. An impressive proportion of candidates performed the calculations correctly.
\end{document}
