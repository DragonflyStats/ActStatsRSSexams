\documentclass[a4paper,12pt]{article}

%%%%%%%%%%%%%%%%%%%%%%%%%%%%%%%%%%%%%%%%%%%%%%%%%%%%%%%%%%%%%%%%%%%%%%%%%%%%%%%%%%%%%%%%%%%%%%%%%%%%%%%%%%%%%%%%%%%%%%%%%%%%%%%%%%%%%%%%%%%%%%%%%%%%%%%%%%%%%%%%%%%%%%%%%%%%%%%%%%%%%%%%%%%%%%%%%%%%%%%%%%%%%%%%%%%%%%%%%%%%%%%%%%%%%%%%%%%%%%%%%%%%%%%%%%%%

\usepackage{eurosym}
\usepackage{vmargin}
\usepackage{amsmath}
\usepackage{graphics}
\usepackage{epsfig}
\usepackage{enumerate}
\usepackage{multicol}
\usepackage{subfigure}
\usepackage{fancyhdr}
\usepackage{listings}
\usepackage{framed}
\usepackage{graphicx}
\usepackage{amsmath}
\usepackage{chngpage}

%\usepackage{bigints}
\usepackage{vmargin}

% left top textwidth textheight headheight

% headsep footheight footskip

\setmargins{2.0cm}{2.5cm}{16 cm}{22cm}{0.5cm}{0cm}{1cm}{1cm}

\renewcommand{\baselinestretch}{1.3}

\setcounter{MaxMatrixCols}{10}

\begin{document}
\begin{enumerate}
%%--- PLEASE TURN OVER8
A continuous-time Markov process with states {Able to work (A), Temporarily unable to work (T), Permanently unable to work (P), Dead (D)} is used to model the cost of providing an incapacity benefit when a person is permanently unable to work. The generator matrix, with rates expressed per annum, for the process is estimated as:
A
T
P
D
A − 0.15 0.1 0.02 0.03
T 0.45 − 0.6 0.1 0.05
0
0
P
− 0.2 0.2
0
0
0
0
D
\begin{enumerate}[(a)]
\item (i) Draw the transition graph for the process.
\item (ii) Calculate the probability of a person remaining in state A for at least 5 years continuously.

Define F(i) to be the probability that a person, currently in state i, will never be in state P.

\item (iii)
Derive an expression for:
(a)
(b)
9
$F(A)$ by conditioning on the first move out of state $A$.
$F(T)$ by conditioning on the first move out of state $T$.

\item 
(iv) Calculate F(A) and F(T).
\item 
(v) Calculate the expected future duration spent in state P, for a person currently
in state A.
\end{enumerate}
%%%%%%%%%%%%%%%%%%%%%%%%%%%%%%

[Total 11]
(i) State the principle of correspondence as it applies to the estimation of
mortality rates.
(ii)


Explain why it might be difficult to ensure the principle of correspondence is
adhered to, and give a specific example of an investigation where this may be
the case.

An actuary was asked to investigate the mortality of lives in a particular geographical area. Data are available of the population of this area, classified by age last birthday, on 1 January in each year. Data on the number of deaths in this area in each calendar
year, classified by age nearest birthday at death, are also available.
(iii)
Derive a formula which would allow the actuary to estimate the force of mortality at age x + f, μ x + f , in a particular calendar year, in terms of the available data, and derive a value for f.
(iv)

List four factors other than geographical location which a government statistical office might use to subdivide data for national mortality analysis. 
[Total 11]

%%%%%%%%%%%%%%%%%%%%%%%%%%%%%%%%%%%%%%%%%%%%%%%%%%%%
\newpage
%%--- Question 8
(i)
T
0.1
0.1
0.45
A
P
0.02
0.05
0.03
0.2
D
(ii)
The force of leaving state A is 0.15.
d
d
dt ( P AA ( t )) = − 0.15 P AA ( t )
dt (ln( P AA ( t ))) = − 0.15
P AA ( t ) = exp( − 0.15 t )
So the probability of staying in state A for at least 5 years continuously is given by
exp(−.75) = 0.472.
(iii)
(a)
\begin{itemize}
\item Conditioning on the first move out of A:
\item Probability 0.1/0.15 of moving to T, at which point probability becomes F(T).
\item Probability 0.02/0.15 of moving to P, at which point certain to travel through
state P.
\item Probability 0.03/0.15 of moving straight to D, at which point certain never to
reach state P.
\end{itemize}
%%%--Page 10%%%%%%%%%%%%%%%%%%%%%%%%%%%%%%%%%%%%%%%%%%%%%%%%%%%5 — Examiners’ Report, September 2011
So F(A) = 0.1/0.15*F(T)+0.02/0.15*0+0.03/0.15*1 = 2/3*F(T) + 1/5.
(b)
Similarly conditioning on first move out of T
\begin{itemize}
\item Probability 0.45/0.6 to A when probability becomes F(A).
\item Probability 0.1/0.6 to P when probability becomes 0.
\item Probability 0.05/0.6 to D when probability becomes 1.
\end{itemize}

So F(T) = 3/4 * F(A)+ 1/12
(iv)
Substituting for F(T) in first equation:
F(A) = 1/2*F(A) /18 /5
F(A) = 23/45
F(T) = 7/15
(v)
Time spent in state P from point of entry is exponentially distributed
with rate 0.2,
so mean time spent in state P from point of entry is 1/0.2 = 5 years.
So expected time spent in state P for a person currently
able to work is (1 − F(A))*5 = 22/45*5 = 22/9 years.

% Parts (i) and (ii) were well answered by most candidates. However, the majority of candidates struggled with parts (iii)–(v), many not attempting these sections. The rates were not required on the diagram in (i) for full credit. Alternative approaches to parts (iii) onwards are possible (for example involving geometric progressions) and were attempted by a few candidates. These approaches involve more complicated equations than the solution above and were rarely successfully completed.
\newpage
%%%%%%%%%%%%%%%%%%%%%%%%%%%%%%%%%%%%%%%%%%%%%%%%%%%
Question 9%%%%
\begin{itemize}
\item (i) A life alive at time t should be included in the exposure at age x at time t if and only
if, were that life to die immediately, he or she would be counted in the deaths data at age x.
\item (ii) When the deaths data and the exposed to risk data come from different sources. E.g. occupational mortality investigations where deaths data come from death registers and exposed to risk data from census
\item OR
where deaths data come from claims department of an office, whereas exposed to risk data are based on policies in force, which come from a different part of the office.
\end{itemize}
%%--- Page 11%%%%%%%%%%%%%%%%%%%%%%%%%%%%%%%%%%%%%%%%%%%%%%%%%%%5 — Examiners’ Report, September 2011
(iii)
We need to adjust the exposed-to-risk to correspond to the age definition of deaths.
Let the population aged x nearest birthday on 1 January in year t be $P x,t$ .

A central exposed to risk for calendar year t can be approximated by
1
1
E x c , t = ∫ P x , t + s ds ≈ ( P x , t + P x , t + 1 )
2
0
assuming that the population varies linearly over the calendar year.
Let P x , t * be the population aged x last birthday on 1 January in year t .
Then
P x , t =
(
)
1 *
P x , t + P x * − 1, t .
2
This assumes that birthdays are distributed evenly across the calendar year
If the number of deaths in year t aged x nearest birthday on the date of death is θ x , t ,
then the required formula for estimating μ x + f , t is thus
μ x + f , t =
θ x , t
1
( P x , t + P x , t + 1 )
2
=
(
1 ⎡ 1 *
P x , t + P x * − 1, t
2 ⎢ ⎣ 2
θ x , t
.
1 *
⎤
*
+ P x , t + 1 + P x − 1, t + 1 ⎥
2
⎦
) (
)
The age range at the start of the rate interval is $[x − 1, x]$, so the age range at the middle of the rate interval is $[x − 1⁄2, x + 1⁄2]$.
The average age at the middle of the rate interval is therefore x.
So f = 0.
(iv)
Sex
Age
Marital status
Occupation
Socio-economic status
Ethnic origin
Educational attainment
Housing tenure
Disability, chronic health condition, limiting long-term illness
In part (ii), candidates who stated that “different age definitions” are a reason why
correspondence is difficult to achieve were given limited credit. If they went on to suggest
Page 12%%%%%%%%%%%%%%%%%%%%%%%%%%%%%%%%%%%%%%%%%%%%%%%%%%%5 — Examiners’ Report, September 2011
that different age definitions can arise because the deaths data and the exposed-to-risk data come from different sources, and gave a relevant example, full credit was awarded. Many candidates, however, did not describe the different age definitions clearly. Part (iii) was better answered than have been exposed-to-risk questions in recent examination papers. In part (iv) “smoking behaviour” is NOT correct as a factor which a national statistical office might use to classify mortality, neither are factors such as “type of policy”, “policy size” or “sales channel”. Candidates are reminded to read the question carefully!
\end{document}
