\documentclass[a4paper,12pt]{article}

%%%%%%%%%%%%%%%%%%%%%%%%%%%%%%%%%%%%%%%%%%%%%%%%%%%%%%%%%%%%%%%%%%%%%%%%%%%%%%%%%%%%%%%%%%%%%%%%%%%%%%%%%%%%%%%%%%%%%%%%%%%%%%%%%%%%%%%%%%%%%%%%%%%%%%%%%%%%%%%%%%%%%%%%%%%%%%%%%%%%%%%%%%%%%%%%%%%%%%%%%%%%%%%%%%%%%%%%%%%%%%%%%%%%%%%%%%%%%%%%%%%%%%%%%%%%

\usepackage{eurosym}
\usepackage{vmargin}
\usepackage{amsmath}
\usepackage{graphics}
\usepackage{epsfig}
\usepackage{enumerate}
\usepackage{multicol}
\usepackage{subfigure}
\usepackage{fancyhdr}
\usepackage{listings}
\usepackage{framed}
\usepackage{graphicx}
\usepackage{amsmath}
\usepackage{chngpage}

%\usepackage{bigints}
\usepackage{vmargin}

% left top textwidth textheight headheight

% headsep footheight footskip

\setmargins{2.0cm}{2.5cm}{16 cm}{22cm}{0.5cm}{0cm}{1cm}{1cm}

\renewcommand{\baselinestretch}{1.3}

\setcounter{MaxMatrixCols}{10}

\begin{document}
\begin{enumerate}
%%%%%%%%%%%%%%%%%%%%%%%%%%%%%%%%%%%%%%%%%%%%%%%%%%%%%%%%%%
11
An historian has investigated the force of mortality from tuberculosis in a particular town in a developed country in the 1860s using a sample of records from a cemetery. He wishes to test whether the underlying mortality from tuberculosis in the town is
the same as the national force of mortality from this cause of death, as reported in death registration data. The data are shown in the table below.

Deaths in
sample Central exposed to
risk in sample National force
of mortality
5–14
15–24
25–34
35–44
45–54
55–64
65–74
75–84 13
47
52
50
33
23
13
3 3,685
2,540
1,938
1,687
1,386
1,018
663
260 0.0051
0.0199
0.0309
0.0316
0.0286
0.0230
0.0202
0.0070
(i) Carry out an overall test of the null hypothesis that the underlying mortality
from tuberculosis in the town is the same as the national force of mortality,
and state your conclusion.
[6]
(ii) (a)
Identify two differences between the experience of the sample
and the national experience which the test you performed in (i)
might not detect.
(b)
Carry out a test for each of the differences in (ii)(a).
(iii)
12
Age-group
Comment on the results from all the tests carried out in (i) and (ii).
\newpage
Question 11
(i)
The chi-squared test is a suitable overall test.
\begin{itemize}
\item Let μ x be the force of mortality in age-group x in the sample.
\item Let μ x s be the force of mortality in age group x in the national population.
\item Let E x c be the central exposed to risk in the sample.
\item Then if z x =
E x c μ x − E x c μ s x
E x c μ s x
the test statistic is
∑ z x 2 ∼ χ 2 m ,
x
\end{itemize}
%%%%%%%%%%%%%%%%%%%%%%%%%%%%%%%%%%5
THEN EITHER
where m is the number of age groups, which in this case is 8.
The calculations are shown below.




%%%%%%%%%%%%%%%%%%%%%%%%%%%%%%%%%%%%%%%%%%%%%%%%%%%%%%%%%%%%%%%%%5

%%-- Quesion 11 - April 2011
\begin{center}
\begin{tabular}{|c|c|c|c|}
Age-group	&	 Expected deaths	&	 z x 	&	z x 2	\\ \hline
5–14	&	8.7935	&	–1.3364	&	1.786	\\ \hline
15–24	&	50.546	&	–0.4988	&	0.2488	\\ \hline
25–34	&	59.8842	&	–1.0188	&	1.038	\\ \hline
35–44	&	53.3092	&	–0.4532	&	0.2054	\\ \hline
45–54	&	39.6396	&	–1.0546	&	1.1121	\\ \hline
55–64	&	23.414	&	–0.0856	&	0.0073	\\ \hline
65–74	&	13.3926	&	–0.1073	&	0.0115	\\ \hline
75–84 	&	1.82	&	0.8747	&	0.7651	\\ \hline
\end{tabular}
\end{center}

Therefore the value of the test statistic is 5.1742.
The critical value of the chi-squared distribution at the 5% level of significance with 8
degrees of freedom is 15.51.
Since 5.1742 < 15.51 we do not reject the null hypothesis that the mortality rate from tuberculosis in the sample is the same as that in the national population.
%%%%%%%%%%%%%%%%%%%%%%%%%%%%%%%%%%%%%%%%%%%%%%%%%%%%%%%%%%%%%%%%%
OR
where m is the number of age groups, which in this case is 7, because we should combine age groups 65–74 and 75–84 as the expected number of deaths in age group 75–84 years is less than 5
The calculations are shown below.
Age-group Expected deaths z x z x 2
5–14
15–24
25–34
35–44
45–54
55–64
65–84 18.7935
50.5460
59.8842
53.3092
39.6396
23.4140
15.2126 –1.3364
–0.4988
–1.0188
–0.4532
–1.0546
–0.0856
0.2019 1.7860
0.2488
1.0380
0.2054
1.1121
0.0073
0.0408
\begin{itemize}
\item Therefore the value of the test statistic is 4.438.
\item The critical value of the chi-squared distribution at the 5\% level of significance with 7
degrees of freedom is 14.07.
\item Since 4.438 < 14.07 we do not reject the null hypothesis that the mortality rate from tuberculosis in the sample is the same as that in the national population
\end{itemize}
%%%%%%%%%%%%%%%%%%%%
(ii)
(a)
Small bias which is not great enough for the chi-squared test to detect.
EITHER
(b)
Signs test
Under the null hypothesis that the mortality rate from tuberculosis in the sample is the same as that in the national population,
the number of positive signs is distributed Binomial (m, 0.5), where m is the number of ages.
We have 1 positive sign.
The probability of 1 or fewer positive signs is given by
⎛ 8 ⎞ 8 ⎛ 8 ⎞ 8
⎜ ⎟ 0.5 + ⎜ ⎟ 0.5 = 0.0352 .
⎝ 0 ⎠
⎝ 1 ⎠
Page 17Subject CT4 (Models Core Technical) — Examiners’ Report, April 2011
OR (if only 7 age groups are being used)
⎛ 7 ⎞ 7 ⎛ 7 ⎞ 7
⎜ ⎟ 0.5 + ⎜ ⎟ 0.5 = 0.0625 .
⎝ 0 ⎠
⎝ 1 ⎠
We use a two-tailed test (since too few or too many positive signs would be a
problem)
so we reject the null hypothesis if the probability of 1 or fewer positive signs
is less than 0.025.
Since 0.0352 (or 0.0625) > 0.025
we do not reject the null hypothesis.
OR
(b)
Cumulative deviations test
Under the null hypothesis that the mortality rate from tuberculosis in the sample is the same as that in the national population
the test statistic
∑ ( E x c μ x − E x c μ s x )
x
∑ E x c μ s x
∼ Normal(0,1)
x
The calculations are shown in the table below
Age-group
5–14
15–24
25–34
35–44
45–54
55–64
65–74
75–84
Σ
E x c μ x − E x c μ s x E x c μ s x
–5.7935
–3.5460
–7.8842
–3.3092
–6.6396
–0.4140
–0.3926
1.1800 18.7935
50.5460
59.8842
53.3092
39.6396
23.4140
13.3926
1.8200
–26.7991 260.7991
So the value of the test statistic is
Page 18
− 26.7991
= 1.6595 .
260.7991

%%-- Subject CT4 (Models Core Technical) — Examiners’ Report, April 2011
\begin{itemize}
\item Using a 5\% level of significance, we see that $−1.96 < 1.6596 < 1.96$.
\item We do not reject the null hypothesis.
(a) Individual ages at which there are unusually large differences between the sample and the national experience.
(b) Individual standardised deviations
\item Under the null hypothesis that the mortality rate from tuberculosis in the sample is the same as that in the national population
we would expect the individual deviations to be distributed Normal (0,1) and therefore only 1 in 20 z x s should have absolute magnitudes greater than
1.96
OR
none should lie outside the range (–3, +3)
OR
diagram showing split of deviations actual versus expected.
\item Since the largest deviation is less in absolute magnitude than 1.96 we do not reject the null hypothesis.
\end{itemize}
(a)
Sections of the data where there is appreciable bias, revealed by runs or clumps of signs of the same type.
EITHER
(b)
Grouping of signs test
Under the null hypothesis that the mortality rate from tuberculosis in the
sample is the same as that in the national population
\begin{itemize}
\item G = Number of groups of positive zs = 1
\item m = number of deviations = 8 (or 7 if last two age groups combined)
\item n 1 = number of positive deviations = 1
\item n 2 = number of negative deviations = 7 (or 6 if last two age groups combined)
\end{itemize}

THEN EITHER
We want k* the largest k such that
⎛ n 1 − 1 ⎞⎛ n 2 + 1 ⎞
k ⎜
⎟⎜
⎟
⎝ t − 1 ⎠⎝ t ⎠
⎛ m ⎞
t = 1
⎜ ⎟
⎝ n 1 ⎠
∑
< 0.05
The test fails at the 5% level if G ≤ k *.
%%--- Page 19
%%--- Subject CT4 (Models Core Technical) — Examiners’ Report, April 2011
In the table in the Gold Book a value for k * is not given,
OR
The table in the Gold Book shows that k * = 0,
so we are not able to reject the null hypothesis
OR
so there is no evidence of clumping.
OR
For t = 1
⎛ n 1 − 1 ⎞ ⎛ 0 ⎞
⎜
⎟ = ⎜ ⎟
⎝ t − 1 ⎠ ⎝ 0 ⎠
which is 1
So this test is automatically passed
OR
There is no evidence of clumping
OR
We cannot reject the null hypothesis.
OR
(b)
Serial correlations (lag 1)
The calculations are shown in the tables below.
EITHER USING SEPARATE MEANS FOR THE z x AND z x + 1
Age
group z x z x A = z x − z
5–14
15–24
25–34
35–44
45–54
55–64
65–74
75–84 –1.336
–0.499
–1.019
–0.453
–1.055
–0.086
–0.107 –0.499
–1.019
–0.453
–1.055
–0.086
–0.107
0.875 –0.686
0.152
–0.368
0.197
–0.404
0.565
0.543
z –0.651 –0.335
0.595/√(1.446*2.604) = 0.307

\begin{center}
\begin{tabular}{|c|c|c|c|c|c|c|}
Age-group	&	 Expected deaths	&	 z x 	&	z x 2	&	B = z x + 1 − z	&	AB	&	A 2	\\ \hline
5–14	&	–1.336	&	–0.499	&	–0.876	&	–0.039	&	0.034	&	0.767	\\ \hline
15–24	&	–0.499	&	–1.019	&	–0.039	&	–0.559	&	0.022	&	0.002	\\ \hline
25–34	&	–1.019	&	–0.453	&	–0.559	&	0.007	&	–0.004	&	0.312	\\ \hline
35–44	&	–0.453	&	–1.055	&	0.007	&	–0.595	&	–0.004	&	0	\\ \hline
45–54	&	–1.055	&	–0.086	&	–0.595	&	0.374	&	–0.223	&	0.354	\\ \hline
55–64	&	–0.086	&	-0.107	&	0.374	&	0.353	&	0.132	&	0.14	\\ \hline
65–74	&	–0.107	&	0.875	&	0.353	&	1.335	&	0.471	&	0.125	\\ \hline
75–84 	&	0.87	&		&	1.335	&	Sum	&		&	1.782	\\ \hline
	&		&		&	z –0.460	&		&		&	0.428	\\ \hline
	&		&		&		&		&		&	3.481	\\ \hline
\end{tabular}
\end{center}


AB
%%--- Subject CT4 (Models Core Technical) — Examiners’ Report, April 2011
Test 0.307 (√8) = 0.868 against Normal (0,1), and, since
0.868 < 1.645, we do not reject the null hypothesis.
that the mortality rate from tuberculosis in the sample is the same as that in the
national population
OR USING THE FORMULA IN THE GOLD BOOK
Age
group z x z x A = z x − z
5–14
15–24
25–34
35–44
45–54
55–64
65–74
75–84 –1.336
–0.499
–1.019
–0.453
–1.055
–0.086
–0.107
0.875 –0.499
–1.019
–0.453
–1.055
–0.086
-0.107
0.875 –0.876
–0.039
–0.559
0.007
–0.595
0.374
0.353
1.335
z –0.460
B = z x + 1 − z
–0.039
–0.559
0.007
–0.595
0.374
0.353
1.335
Sum
A 2
AB
0.034
0.022
–0.004
–0.004
–0.223
0.132
0.471 0.767
0.002
0.312
0.000
0.354
0.140
0.125
1.782
0.428 3.481
1
(0.428)
7
= 0.141
1
(3.481)
8
Test 0.141 (√8) = 0.397 against Normal (0,1), and, since
0.397 < 1.645, we do not reject the null hypothesis.
that the mortality rate from tuberculosis in the sample is the same as that in the national population
(iii)

\begin{itemize}
\item In none of the tests we have performed do we reject the null hypothesis.
\item Therefore it seems that the mortality from tuberculosis in the town is the same as the national force of mortality.
\item In part (ii) the null hypothesis should be stated somewhere for each test. It could be stated at
the beginning, or in the conclusion. As long as it is correctly stated somewhere, full credit was given. In part (iii), the comment should be consistent with the results of the tests performed in parts (i) and (ii) to gain credit.
\end{itemize}

%% Most candidates made a good attempt at part (i). Attempts at part (ii) were more varied. In particular, most candidates did not point out that the chi-squared test only fails to detect SMALL (but consistent) bias. If the bias is large and consistent, the chi-squared test will detect it.
\end{document}
