\documentclass[a4paper,12pt]{article}

%%%%%%%%%%%%%%%%%%%%%%%%%%%%%%%%%%%%%%%%%%%%%%%%%%%%%%%%%%%%%%%%%%%%%%%%%%%%%%%%%%%%%%%%%%%%%%%%%%%%%%%%%%%%%%%%%%%%%%%%%%%%%%%%%%%%%%%%%%%%%%%%%%%%%%%%%%%%%%%%%%%%%%%%%%%%%%%%%%%%%%%%%%%%%%%%%%%%%%%%%%%%%%%%%%%%%%%%%%%%%%%%%%%%%%%%%%%%%%%%%%%%%%%%%%%%

\usepackage{eurosym}
\usepackage{vmargin}
\usepackage{amsmath}
\usepackage{graphics}
\usepackage{epsfig}
\usepackage{enumerate}
\usepackage{multicol}
\usepackage{subfigure}
\usepackage{fancyhdr}
\usepackage{listings}
\usepackage{framed}
\usepackage{graphicx}
\usepackage{amsmath}
\usepackage{chngpage}

%\usepackage{bigints}
\usepackage{vmargin}

% left top textwidth textheight headheight

% headsep footheight footskip

\setmargins{2.0cm}{2.5cm}{16 cm}{22cm}{0.5cm}{0cm}{1cm}{1cm}

\renewcommand{\baselinestretch}{1.3}

\setcounter{MaxMatrixCols}{10}

\begin{document}
\begin{enumerate}
(i) List the factors which should be considered in assessing the suitability of a
model for a particular exercise.
[3]
(ii) Assess the suitability of a multiple state model with three states: Healthy, Sick
and Dead, for estimating the transition intensities in an analysis of claims for
sickness benefit, in the light of your answer to (i).
[4]
[Total 7]
CT4 S2011—3
[2]













%%%%%%%%%%%%%%%%%%%%%%%%%%%%%%%%%%%%%%%%%%%%%%%%%%%%%%%%%%%%%%%%%%%%%%%%%%%%%%%
\newpage

Question 5
(i) Objectives of the modelling exercise.
Validity of the model for the purpose to which it is to be put.
Validity of the data to be used.
Possible errors associated with the model or parameters used not being a perfect
representation of the real world situation being modelled.
Impact of correlations between the random variables that “drive” the model.
Extent of correlations between the results produced from the model.
Current relevance of models written and used in the past.
Credibility of the data input.
Credibility of the results output.
Dangers of spurious accuracy.
Ease with which the model and its results can be communicated.
The time and cost of constructing and maintaining the model.
(ii) The model is capable of meeting the objective, specifically the estimation of transition
intensities.
The model is valid for this purpose.
Page 5%%%%%%%%%%%%%%%%%%%%%%%%%%%%%%%%%%%%%%%%%%%%%%%%%%%5 — Examiners’ Report, September 2011
The data required are the total waiting times in each of the states Healthy and Sick for
the lives in the investigation during the period of the investigation, together with the
number of transitions from Healthy to Sick, from Sick to Healthy, from Healthy to Dead and from Sick to Dead.
Provided these data are available, the data will be valid for the application of the
model.
The model is as good a representation of the real world process as we can obtain.
The model requires that we estimate constant intensities. The results will be credible provided we estimate the intensities separately for short age intervals, over which the
assumption of constant transition intensities is credible.
The concept of transition intensities is not intuitively easy for non-specialists to understand.
The results can be made easier to understand and the results clearer by converting the
transition intensities to probabilities – e.g. the probability that a Healthy life aged x will make a sickness claim before he or she is aged x+t years.
Some candidates scored well on part (i), which was standard bookwork, but a disappointing
number did not. Answers to part (ii) were variable. To score highly, the points made in part
(ii) should relate to those made in part (i). Within this general criterion, sensible points other
than those listed above were given credit. Full marks could be obtained for less than is given
in the model solution above.
Question 6
(i)
(a)
EITHER
The parameters are the rate of leaving state i , \lambda i , for each i , and the jump-chain
transition probabilities, r ij , for j \neq i , where r ij is the conditional probability that
the next transition is to state j given the current state is i .
OR
If the rate of leaving state i , is \lambda i , and r ij is the conditional probability that the
next transition is to state j given the current state is i .
The parameters are μ ij, where, for i = j , μ ii = -\lambda i and, for i \neq j , μ ij = \lambda i r ij .
Page 6%%%%%%%%%%%%%%%%%%%%%%%%%%%%%%%%%%%%%%%%%%%%%%%%%%%5 — Examiners’ Report, September 2011
OR
The parameters are the six transition rates from state i to state j ( i \neq j ):
μ AB
μ AC
μ BA
μ BC
μ CA
μ CB
(b)
The assumptions are as follows.
EITHER The holding time in each state is exponentially distributed
OR The transition intensities from each state are not time-dependent.
The parameter of this distribution varies only by state i , so that the distribution
is independent of anything that happened prior to the arrival in current state i .
The destination of the jump on leaving state i is independent of holding time,
and of anything that happened prior to the current arrival in state i .
(ii)
(a)
The estimator [it is the maximum likelihood estimator (MLE)
but this need not be stated] of \lambda i , \hat{\lambda} i , is the inverse of the
average duration of each visit to state i .
^
so \hat{\lambda} A = 3 per hour, \hat{\lambda} B = 1/2 per hour, \lambda C = 1/3 per hour
The estimator [it is the MLE but this need not be stated]
of r ij , r ˆ ij , is the proportion of observed jumps out of
state i to state j .
r ˆ AB = 60/105=4/7
r ˆ AC = 45/105=3/7
r ˆ BA =50/75=2/3
r ˆ BC =25/75=1/3
r ˆ CA =55/70=11/14
r ˆ CB =15/70=3/14
Page 7%%%%%%%%%%%%%%%%%%%%%%%%%%%%%%%%%%%%%%%%%%%%%%%%%%%5 — Examiners’ Report, September 2011
(b)
The estimated generator matrix (in hr −1 ) is:
9 ⎞
⎛ − 3 12
7
7 ⎟
⎜
⎜ 1
1 ⎟
− 1
2
6 ⎟
⎜ 3
⎜ ⎜ 11
1
− 1 ⎟ ⎟
14
3 ⎠
⎝ 42
(iii)
EITHER Binomial, with mean n.r ij and variance n.r ij .(1 – r ij ), n being the number of
transitions out of state i .
OR Binomial ( n , r ij ) n being the number of transitions out of state i .
This was a relatively straightforward question, so the Examiners were looking for accurate
and incisive answers. In part (i)(b) many candidates offered vague statements about the process not depending on past history. These candidates scored only limited credit for this
part. In part (ii)(a) candidates who simply wrote down the values of the transition intensities,
viz:
μ AB = 12 / 7
μ AC = 9 / 7
μ BA = 1/ 3
μ BC = 1/ 6
μ CA = 11/ 42
μ CB = 1/14
scored partial credit. Some candidates combined parts (ii)(a) and (b) by simply writing down
the generator matrix. If this was correct, they were awarded most of the marks for this part,
but for full marks some indication of how they arrived at the numbers in the generator matrix
was needed. It was extremely disappointing how few candidates were able to state the
distribution in part (iii): this seems to indicate a gap in knowledge of the subject.
