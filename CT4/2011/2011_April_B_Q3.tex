\documentclass[a4paper,12pt]{article}

%%%%%%%%%%%%%%%%%%%%%%%%%%%%%%%%%%%%%%%%%%%%%%%%%%%%%%%%%%%%%%%%%%%%%%%%%%%%%%%%%%%%%%%%%%%%%%%%%%%%%%%%%%%%%%%%%%%%%%%%%%%%%%%%%%%%%%%%%%%%%%%%%%%%%%%%%%%%%%%%%%%%%%%%%%%%%%%%%%%%%%%%%%%%%%%%%%%%%%%%%%%%%%%%%%%%%%%%%%%%%%%%%%%%%%%%%%%%%%%%%%%%%%%%%%%%

\usepackage{eurosym}
\usepackage{vmargin}
\usepackage{amsmath}
\usepackage{graphics}
\usepackage{epsfig}
\usepackage{enumerate}
\usepackage{multicol}
\usepackage{subfigure}
\usepackage{fancyhdr}
\usepackage{listings}
\usepackage{framed}
\usepackage{graphicx}
\usepackage{amsmath}
\usepackage{chngpage}

%\usepackage{bigints}
\usepackage{vmargin}

% left top textwidth textheight headheight

% headsep footheight footskip

\setmargins{2.0cm}{2.5cm}{16 cm}{22cm}{0.5cm}{0cm}{1cm}{1cm}

\renewcommand{\baselinestretch}{1.3}

\setcounter{MaxMatrixCols}{10}

\begin{document}
\begin{enumerate}

3 Describe the ways in which the design of a model used to project over only a short time frame may differ from one used to project over fifty years.

%%%%%%%%%%%%%%%%%%%%%%%%%%%%%%%%%%%%%%%%%%%%%%%%%%%%%%%%%%%%%%%%%%%%%%%%%%%%%%%%%%%%


\newpage

Question 3
Individual variables may behave differently, for example a model over 50 years may be more
sensitive to differences in the input values of certain variables than one over the short term.
A variable which has an ignorable effect in the short term may have a non-ignorable effect
over 50 years.
Over the short term, it may be reasonable to assume the values of some variables to be
constant or to vary linearly, whereas this would not be reasonable over 50 years. For
example, growth which is exponential may appear linear if studied over a short time frame.
The interaction between variables in the short-term may be different from that over the long-
term.
Higher order relationships between variables may be ignored for simplicity if modelling over
a short time frame.
The time units used in the model might be shorter for a model projecting over a short time
frame, so that the total number of time units used in each model is roughly the same.
Over 50 years, regulatory changes and other “shock” events are more likely to occur, and the
model design may need to consider the circumstances in which the results or conclusions may
be materially impacted (e.g. in the short term the tax basis may be known, but in the long run
it is likely to change).
The marks on this question were the lowest on any question. The question was a “higher
skills” question and so required candidates to think about the context. Little credit was given
to candidates who uncritically reproduced sections of the Core Reading. In particular, the
question is about model DESIGN, so the points made should relate to the design of the
model.
%%--- Page 3Subject CT4 (Models Core Technical) — Examiners’ Report, April 2011

%%%%%%%%%%%%%%%%%%%%%%%%%%%%%%%%%%%%%%%%%%%

\end{document}
