\documentclass[a4paper,12pt]{article}

%%%%%%%%%%%%%%%%%%%%%%%%%%%%%%%%%%%%%%%%%%%%%%%%%%%%%%%%%%%%%%%%%%%%%%%%%%%%%%%%%%%%%%%%%%%%%%%%%%%%%%%%%%%%%%%%%%%%%%%%%%%%%%%%%%%%%%%%%%%%%%%%%%%%%%%%%%%%%%%%%%%%%%%%%%%%%%%%%%%%%%%%%%%%%%%%%%%%%%%%%%%%%%%%%%%%%%%%%%%%%%%%%%%%%%%%%%%%%%%%%%%%%%%%%%%%

\usepackage{eurosym}
\usepackage{vmargin}
\usepackage{amsmath}
\usepackage{graphics}
\usepackage{epsfig}
\usepackage{enumerate}
\usepackage{multicol}
\usepackage{subfigure}
\usepackage{fancyhdr}
\usepackage{listings}
\usepackage{framed}
\usepackage{graphicx}
\usepackage{amsmath}
\usepackage{chngpage}

%\usepackage{bigints}
\usepackage{vmargin}

% left top textwidth textheight headheight

% headsep footheight footskip

\setmargins{2.0cm}{2.5cm}{16 cm}{22cm}{0.5cm}{0cm}{1cm}{1cm}

\renewcommand{\baselinestretch}{1.3}

\setcounter{MaxMatrixCols}{10}

\begin{document}
\begin{enumerate}

PLEASE TURN OVER11
An actuary walks from his house to the office each morning, and walks back again
each evening. He owns two umbrellas. If it is raining at the time he sets off, and one
or both of his umbrellas is available, he takes an umbrella with him. However if it is
not raining at the time he sets off he always forgets to take an umbrella.
Assume that the probability of it raining when he sets off on any particular journey is
a constant p, independent of other journeys.
This situation is examined as a Markov Chain with state space {0,1,2} representing
the number of his umbrellas at the actuary’s current location (office or home) and
each time step representing one journey.
(i)
Explain why the transition graph for this process is given by:
p
1
Zero
Two
1 − p
One
p
1 − p

(ii) Derive the transition matrix for the number of umbrellas at the actuary’s house
before he leaves each morning, based on the number before he leaves the
previous morning.

(iii) Calculate the stationary distribution for the Markov Chain.
(iv) Calculate the long run proportion of journeys (to or from the office) on which the actuary sets out in the rain without an umbrella.


The actuary considers that the weather at the start of a journey, rather than being independent of past history, depends upon the weather at the start of the previous journey. He believes that if it was raining at the start of a journey the probability of it
raining at the start of the next journey is r (0 < r <1), and if it was not raining at the
start of a journey the probability of it raining at the start of the next journey is
s $(0 < s < 1, r \neq s)$.
(v) Write down the transition matrix for the Markov Chain for the weather.
(vi) Explain why the process with three states {0,1,2}, being the number of his umbrellas at the actuary’s current location, would no longer satisfy the
Markov property.

(vii) Describe the additional state(s) needed for the Markov property to be satisfied, and draw a transition diagram for the expanded system.

%%-- [Total 18]
%%-- END OF PAPER
%%-- CT4 S2011—8
\newpage

%%%%%%%%%%%%%%%%%%%%%%%%%%%%%%%%%%%%%%%%%%%%%%%%%%%%%%%%%%%%%%%%%%%%%%%%%%%%%%%%%%%%%%%%%%%%%%%%%%%%%%%%%%%%%%%%5

%% -- Question 11
(i)
\begin{itemize}
\item Transitions from state “Zero”
\item No umbrellas to take so must be two at the other location.
\item Transitions from state “One”
\item If it does not rain, then there remains one at each location, probability 1 − p .
\item If it does rain, both umbrellas end up at the next destination, probability p .
\item Transitions from state “Two”
\item If it does not rain, then forgets to take an umbrella so none is at the next location,
probability 1 − p .
\item If it does rain, takes one of the umbrellas to the other location, probability p .
\end{itemize}
%%%%%%%%%%%%%%%%%%%%%%%%%%%%%%%%%%%%%%%%%%%%%%%%%%%%%%%%%%%%%%
(ii)
One step transition matrix is:
0
⎛ 0
⎜
1 − p
⎜ 0
⎜ 1 − p
p
⎝
1 ⎞
⎟
p ⎟
0 ⎟ ⎠
Seeking the two-step transition matrix as the square of this matrix:
0
⎛ 0
⎜
1 − p
⎜ 0
⎜ 1 − p
p
⎝
(iii)
0
⎛ 0
⎜
1 − p
\pi ⎜ 0
⎜ 1 − p
p
⎝
1 ⎞ ⎛ 0
0
⎟ ⎜
p ⎟ . ⎜ 0
1 − p
⎟
⎜
p
0 ⎠ ⎝ 1 − p
1 ⎞
⎟
p ⎟ = \pi
0 ⎟ ⎠
(1 − p ) \pi 3 = \pi 1
(1 − p ) \pi 2 + p \pi 3 = \pi 2 or \pi 2 = \pi 3
\pi 1 + p \pi 2 = \pi 3
and
\pi 1 + \pi 2 + \pi 3 = 1
((1 − p ) + 1 + 1) \pi 3 = 1
Page 18
p
0
⎞
1 ⎞ ⎛ 1 − p
⎟
⎟ ⎜
p ⎟ = ⎜ p (1 − p ) (1 − p ) 2 + p 2
p (1 − p ) ⎟
⎟
0 ⎟ ⎠ ⎜ 0
p (1 − p )
1 − p + p 2 ⎠
⎝
(I)
(II)
(III)
(IV)%%%%%%%%%%%%%%%%%%%%%%%%%%%%%%%%%%%%%%%%%%%%%%%%%%%5 — Examiners’ Report, September 2011
\pi 2 = \pi 3 =
\pi 1 =
(iv)
1
3 − p
1 − p
3 − p
He gets wet if it rains on a journey when he is state “Zero”.
So the long run probability is p . \pi 1 =
(v)
Denoting R = raining, NR = not raining
From / To R
R
NR
(vi)
p (1 − p )
.
3 − p
r
NR
1 − r
s 1 − s
This would not satisfy the Markov property because (in states “One” and “Two”)
would need to know, in addition, whether it was raining or not on the last journey to
determine the future evolution of the process.
e.g. if in state “Two”, probability of next moving to “Zero” is 1- r if it rained on the
last journey and 1 − s if it did not. As r does not equal s the Markov property is not
satisfied.
(vii)
\begin{itemize}
\item If we expand the states to include information about whether it rained on the last
journey, then the Markov property is satisfied.
\item Five states are needed, as cannot be in position with zero umbrellas when it rained on
last journey,
so the state space is {Zero, One Rained, One Did Not Rain, Two Rained, Two Did
Not Rain}
\end{itemize}

%%-- Page 19%%%%%%%%%%%%%%%%%%%%%%%%%%%%%%%%%%%%%%%%%%%%%%%%%%%5 — Examiners’ Report, September 2011

% Many candidates scored highly on parts (i)–(iii) of this question, but a much smaller proportion made a solid effort at parts (iv)–(vii). In part (vi), candidates who simply said that the process would not satisfy the Markov property because it depended on the “past history” scored only limited credit. For full credit, it was necessary to say that what matters is whether it was raining or not on the last journey, and to give an example of transitions with differing probabilities. In part (vii), some candidates produced four-state solutions, splitting either of states One or Two, but not both. These candidates were given credit for diagrams correct for the solution they were offering.

%% END OF EXAMINERS’ REPORT
%% Page 20
\end{document}
