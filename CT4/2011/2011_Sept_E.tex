\documentclass[a4paper,12pt]{article}

%%%%%%%%%%%%%%%%%%%%%%%%%%%%%%%%%%%%%%%%%%%%%%%%%%%%%%%%%%%%%%%%%%%%%%%%%%%%%%%%%%%%%%%%%%%%%%%%%%%%%%%%%%%%%%%%%%%%%%%%%%%%%%%%%%%%%%%%%%%%%%%%%%%%%%%%%%%%%%%%%%%%%%%%%%%%%%%%%%%%%%%%%%%%%%%%%%%%%%%%%%%%%%%%%%%%%%%%%%%%%%%%%%%%%%%%%%%%%%%%%%%%%%%%%%%%

\usepackage{eurosym}
\usepackage{vmargin}
\usepackage{amsmath}
\usepackage{graphics}
\usepackage{epsfig}
\usepackage{enumerate}
\usepackage{multicol}
\usepackage{subfigure}
\usepackage{fancyhdr}
\usepackage{listings}
\usepackage{framed}
\usepackage{graphicx}
\usepackage{amsmath}
\usepackage{chngpage}

%\usepackage{bigints}
\usepackage{vmargin}

% left top textwidth textheight headheight

% headsep footheight footskip

\setmargins{2.0cm}{2.5cm}{16 cm}{22cm}{0.5cm}{0cm}{1cm}{1cm}

\renewcommand{\baselinestretch}{1.3}

\setcounter{MaxMatrixCols}{10}

\begin{document}
\begin{enumerate}

CT4 S2011—610
(i)
Describe three shortcomings of the χ 2 test for comparing crude estimates
of mortality with a standard table and why they may occur.

The following table gives an extract of data from a mortality investigation conducted
in the rural highlands of a developed country. The raw data have been graduated by
reference to a standard mortality table of assured lives.
(ii)
Expected
deaths Observed
deaths z x z x 2
60
61
62
63
64
65
66
67
68
69
70 36.15
28.92
31.34
38.01
26.88
37.59
33.85
26.66
22.37
18.69
18.24 35
24
27
35
32
36
34
32
26
33
22 –0.191
–0.915
–0.775
–0.488
0.988
–0.259
0.026
1.034
0.767
3.310
0.880 0.037
0.837
0.601
0.238
0.975
0.067
0.001
1.070
0.589
10.956
0.775
For each of the three shortcomings you described in (i):
(a)
(b)
(iii)
Age
x
name a test that would detect that shortcoming.
carry out the test on the data above.
Comment on your results from (ii).
CT4 S2011—7
[12]

[Total 18]

%%%%%%%%%%%%%%%%%%%%%%%%%%%%%%%%%%%%%%%%%%%%%%%%%%%%%%%%%%%%%%%%%%%%%%%%%%%%

Question 10
(i)
Outliers. Since all the information is summarised in one number, a few large
deviations may be offset or hidden by a large number of small deviations.
Small bias. Since the squares of the differences are used, the sign of the differences
are lost, hence small but consistent bias above or below may not be noticed.
Clumps or runs. Again because the squares of the differences are used, the sign of the
differences are lost, so significant groups of (clumps or runs) of bias over ranges of
the data may not be detected.
(ii)
(a)
A few large deviations or outliers – Individual Standardised Deviations Test.
Small but consistent bias – Signs Test OR Cumulative Deviations Test.
Clumps or runs of bias over ranges of the data - Grouping of Signs Test OR
Serial Correlations Test.
(b)
Individual Standardised Deviations Test
Under the null hypothesis that the standard table rates OR graduated
rates are the true rates underlying the observed data
we would expect individual deviations to be distributed Normal (0,1).
EITHER only 1 in 20 z x should lie above 1.96 in absolute value
OR none should lie above 3 in absolute value
OR table (see below) showing split of deviations, actual versus expected.
Expected
Observed
( −∞ , − 2) ( −2, −1) (−1, 0) (0, 1) (1, 2) ( 2, +∞ )
0.22
1.54 3.74 3.74 1.54
0.22
0
0
5
4
1
1
The largest deviation we have here is 3.31.
This is well outside the range −1.96 to 1.96 so we have reason to reject the
null hypothesis.
Page 13%%%%%%%%%%%%%%%%%%%%%%%%%%%%%%%%%%%%%%%%%%%%%%%%%%%5 — Examiners’ Report, September 2011
EITHER Signs Test OR Cumulative Deviations Test
Signs Test
Under the null hypothesis that the standard table rates OR graduated
rates are the true rates underlying the observed data
The number of positive signs amongst the z x is distributed Binomial (11, 1⁄2 )
We observe 6 positive signs.
EITHER the probability of observing 6 or more positive signs in 11
observations is 0.5
OR the probability of observing exactly 6 positive signs is 0.2256.
which implies that Pr[observing 6 or more] > 0.025 (a two-tailed test),
so we have no evidence to reject the null hypothesis.
Cumulative Deviations Test
Under the null hypothesis that the standard table rates OR graduated
rates are the true rates underlying the observed data,
the test statistic
∑ (Observed deaths - Expected deaths)
x
∑ Expected deaths
~ Normal(0,1)
x
The calculations are shown in the table below.
Age x
Page 14
Expected deaths
Observed – expected
deaths
60
61
62
63
64
65
66
67
68
69
70 36.15
28.92
31.34
38.01
26.88
37.59
33.85
26.66
22.37
18.69
18.24 − 1.15
− 4.92
− 4.34
− 3.01
5.12
− 1.59
0.15
5.34
3.63
14.31
3.76
Totals 318.70 17.30%%%%%%%%%%%%%%%%%%%%%%%%%%%%%%%%%%%%%%%%%%%%%%%%%%%5 — Examiners’ Report, September 2011
The value of the test statistic is
17.30
= 0.969
318.70
and, since – 1.96 < test statistic < +1.96 we have insufficient evidence to
reject the null hypothesis.
EITHER Grouping of Signs Test OR Serial Correlations Test
Grouping of Signs Test
Under the null hypothesis that the standard table rates OR the graduated rates
are the true rates underlying the observed data
G = Number of groups of positive deviations = 2
m = number of deviations = 11
n 1 = number of positive deviations = 6
n 2 = number of negative deviations = 5
THEN EITHER
We want k * the largest k such that
⎛ n 1 − 1 ⎞⎛ n 2 + 1 ⎞
k ⎜
⎟⎜
⎟
⎝ t − 1 ⎠⎝ t ⎠
⎛ m ⎞
t = 1
⎜ ⎟
⎝ n 1 ⎠
∑
< 0.05
The test fails at the 5% level if G ≤ k *.
From the Gold Book k * = 1.
So we have insufficient evidence to reject the null hypothesis.
OR
For t = 2
⎛ n 1 − 1 ⎞ ⎛ 5 ⎞
⎛ n 2 + 1 ⎞ ⎛ 6 ⎞
⎜
⎟ = ⎜ ⎟ = 5 and ⎜
⎟ = ⎜ ⎟ = 15
⎝ t − 1 ⎠ ⎝ 1 ⎠
⎝ t ⎠ ⎝ 2 ⎠
⎛ m ⎞ ⎛ 11 ⎞
and ⎜ ⎟ = ⎜ ⎟ = 462
⎝ n 1 ⎠ ⎝ 6 ⎠
So Pr[ t = 2] if the null hypothesis is true is 75/462 = 0.162, which is greater
than 5% so we have insufficient evidence reject the null hypothesis.
Page 15%%%%%%%%%%%%%%%%%%%%%%%%%%%%%%%%%%%%%%%%%%%%%%%%%%%5 — Examiners’ Report, September 2011
Serial Correlations Test (lag 1)
Under the null hypothesis that the standard table rates OR graduated
rates are the true rates underlying the observed data.
The calculations are shown in the tables below.
EITHER USING SEPARATE MEANS FOR THE z x AND z x + 1
Age z x z x A = z x − z
60
61
62
63
64
65
66
67
68
69
70 –0.191
–0.915
–0.775
–0.488
0.988
–0.259
0.026
1.034
0.767
3.310
0.880 –0.915
–0.775
–0.488
0.988
–0.259
0.026
1.034
0.767
3.310
0.880 –0.541
–1.264
–1.125
–0.838
0.638
–0.609
–0.324
0.685
0.418
2.960
0.531
z 0.350 0.457
B = z x + 1 − z
AB
A 2 B 2
–1.372
–1.232
–0.945
0.531
–0.716
–0.431
0.577
0.311
2.853
0.424 0.742
1.558
1.063
–0.445
–0.457
0.262
–0.187
0.213
1.192
1.254 0.293
1.599
1.265
0.702
0.407
0.371
0.105
0.469
0.175
8.764 1.881
1.518
0.893
0.282
0.513
0.186
0.333
0.097
8.141
0.179
Average 0.520 1.415 1.402
0.520/√(1.415*1.402) = 0.369.
Test 0.369 (√11) = 1.223 against Normal (0,1), and, since 1.223 < 1.645, we
do not reject the null hypothesis.
OR USING THE FORMULA IN THE GOLD BOOK
Page 16
Age z x z x A = z x − z
60
61
62
63
64
65
66
67
68
69
70 –0.191
–0.915
–0.775
–0.488
0.988
–0.259
0.026
1.034
0.767
3.310
0.880 –0.915
–0.775
–0.488
0.988
–0.259
0.026
1.034
0.767
3.310
0.880 –0.589
–1.313
–1.173
–0.886
0.590
–0.657
–0.372
0.636
0.370
2.912
0.483
z 0.350 0.457
B = z x + 1 − z
–1.313
–1.173
–0.886
0.590
–0.657
–0.372
0.636
0.370
2.912
0.483
Sum
AB
A 2
0.773
1.540
1.039
–0.523
–0.388
0.245
–0.237
0.235
1.076
1.405 0.347
1.723
1.376
0.785
0.348
0.432
0.138
0.432
0.137
8.481
0.233
0.517 1.310%%%%%%%%%%%%%%%%%%%%%%%%%%%%%%%%%%%%%%%%%%%%%%%%%%%5 — Examiners’ Report, September 2011
1
(5.617)
10
= 0.395 .
1
(14.405)
11
Test 0.395 (√11) = 1.309 against Normal (0,1), and, since 1.309 < 1.645, we
do not reject the null hypothesis.
(iii)
The result of the Individual Standard Deviation test suggests outliers in the data.
The actual and expected deaths are relatively low, suggesting that the population in
the rural area is not very large.
The ages under consideration are also high, exacerbating this scarcity of data.
However there are at least five (actual/expected) deaths in each age group, so the data are adequate.
So this is unlikely to account for the outlier at age 69 years, which should be investigated further.
The period of the observation is not stated and could affect the results, as, for example
if the observation only covered one winter a particularly bad influenza epidemic may
have caused more deaths than usual (although this would likely impact all ages in this
range similarly).
Both the signs and grouping of signs test suggest no bias over the whole or part of the
data.
However there does seem to be a drift towards the number of observed deaths
exceeding the expected at higher ages, and the number observed being smaller than
expected at younger ages.
Perhaps if a larger extract from the investigation were considered or the
table in its entirety, bias may be observed.
Answers to this question were disappointing. Too many answers to part (i) were sketchy and
failed to explain WHY the chi-squared test sometimes fails to detect small bias, outliers or
“runs” of deviations of the same sign. In part (ii) some candidates failed to relate the tests they were performing to the deficiencies of the chi-squared test identified in part (i); other
candidates performed two tests for the same deficiency (only the higher scoring of which
received credit). Many candidates lost marks for vagueness in the execution of the tests.
Although not all the points listed above were required in part (iii) for full credit, the number
of marks available indicated that candidates were expected to go beyond the basic results of
the tests. Disappointingly few did this.
Page 17%%%%%%%%%%%%%%%%%%%%%%%%%%%%%%%%%%%%%%%%%%%%%%%%%%%5 — Examiners’ Report, September 2011
