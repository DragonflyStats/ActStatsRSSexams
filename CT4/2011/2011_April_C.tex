\documentclass[a4paper,12pt]{article}

%%%%%%%%%%%%%%%%%%%%%%%%%%%%%%%%%%%%%%%%%%%%%%%%%%%%%%%%%%%%%%%%%%%%%%%%%%%%%%%%%%%%%%%%%%%%%%%%%%%%%%%%%%%%%%%%%%%%%%%%%%%%%%%%%%%%%%%%%%%%%%%%%%%%%%%%%%%%%%%%%%%%%%%%%%%%%%%%%%%%%%%%%%%%%%%%%%%%%%%%%%%%%%%%%%%%%%%%%%%%%%%%%%%%%%%%%%%%%%%%%%%%%%%%%%%%

\usepackage{eurosym}
\usepackage{vmargin}
\usepackage{amsmath}
\usepackage{graphics}
\usepackage{epsfig}
\usepackage{enumerate}
\usepackage{multicol}
\usepackage{subfigure}
\usepackage{fancyhdr}
\usepackage{listings}
\usepackage{framed}
\usepackage{graphicx}
\usepackage{amsmath}
\usepackage{chngpage}

%\usepackage{bigints}
\usepackage{vmargin}

% left top textwidth textheight headheight

% headsep footheight footskip

\setmargins{2.0cm}{2.5cm}{16 cm}{22cm}{0.5cm}{0cm}{1cm}{1cm}

\renewcommand{\baselinestretch}{1.3}

\setcounter{MaxMatrixCols}{10}

\begin{document}
\begin{enumerate}
6
State an appropriate method of graduation for each of the three classes of lives
and, for each class, briefly explain your choice.
[3]
[Total 6]
A study of the mortality of a certain species of insect reveals that for the first 30 days
of life, the insects are subject to a constant force of mortality of 0.05. After 30 days,
the force of mortality increases according to the formula:
μ 30 + x = 0.05exp(0.01 x ) ,
where x is the number of days after day 30.
7
(i) Calculate the probability that a newly born insect will survive for at least 10
days.
[1]
(ii) Calculate the probability that an insect aged 10 days will survive for at least a
further 30 days.
[3]
(iii) Calculate the age in days by which 90 per cent of insects are expected to have
died.
[4]
[Total 8]
(i) Define a counting process.
[2]
For each of the following processes:
•
•
•
(ii)
simple random walk
compound Poisson
Markov chain
(a) State whether each of the state space and the time set is discrete,
continuous or can be either.
(b) Give an example of an application which may be useful to a
shopkeeper selling dried fruit and nuts loose.
[6]

%%%%%%%%%%%%%%%%%%%%%%%%%%%%%%%%%%%%%%%%%%%%%%%%%%%%%%%%5

Question 6
(i)
The probability that an insect will survive for 10 days, 10 p 0 , is given by the formula
⎛ 10
⎞
⎜
⎟ .
=
−
μ
p
exp
dx
10 0
⎜ ∫ x ⎟
⎝ 0
⎠
Since the force of mortality is constant up to age 30 days at a value of 0.05,
⎛ 10
⎞
10
10 p 0 = exp ⎜ − ∫ 0.05 dx ⎟ = exp − [ 0.05 x ] 0 = exp ( − 0.5 ) = 0.6065 .
⎜
⎟
⎝ 0
⎠
(
(ii)
)
The probability that an insect 10 days old will survive for a further 30 days (that is to
exact age 40 days) is given by
⎛ 40
⎞
⎜ − ∫ μ x dx ⎟ .
p
=
exp
30 10
⎜
⎟
⎝ 10
⎠
Since
30 p 10
= 20 p 10 . 10 p 30 , this is equal to
⎛ 30
⎞
⎛ 10
⎞
exp ⎜ − ∫ 0.05 dx ⎟ exp ⎜ − ∫ 0.05exp(0.01 x ) dx ⎟
⎜
⎟
⎜
⎟
⎝ 10
⎠
⎝ 0
⎠
10
⎛ ⎡ 0.05
30
⎤ ⎞
= exp − [ 0.05 x ] 10 exp ⎜ − ⎢
exp(0.01 x ) ⎥ ⎟
⎜ ⎣ 0.01
⎦ 0 ⎟ ⎠
⎝
)
(
= e − (1.5 − 0.5) e − (5exp(0.1) − 5exp(0))
= e − 1 e − 0.5258 = 0.3679 × 0.5911 = 0.2174.
(iii)
If the required age is 30+a, then we have
30 + a
p 0 =
30
p 0 . a p 30 = 0.1 .
Now
⎡ 30
⎤
30 p 0 = exp ⎢ − ∫ 0.05 dx ⎥ = exp( − 1.5) = 0.2231 .
⎣ 0
⎦
So
Page 6
a
p 30 =
0.1
= 0.4483 .
0.2231Subject CT4 (Models Core Technical) — Examiners’ Report, April 2011
Using the result from part (ii), we have
⎛ ⎡ 0.05 0.01 x ⎤ a ⎞
⎛ ⎡ 0.05 0.01 a 0.05 ⎤ ⎞
0.01 a
− ⎢
e ⎥ ⎟ = exp ⎜ − ⎢
e
−
a p 30 = exp ⎜
⎟ = exp(5 − 5 e )
⎥
⎜ ⎣ 0.01
⎟
0.01
0.01
⎦ 0 ⎠
⎦ ⎠
⎝ ⎣
⎝
Therefore
e 5(1 − exp(0.01 a )) = 0.4483 ,
whence
log e 0.4483 = 5(1 − e 0.01 a ) ,
so that
1 − e 0.01 a = − 0.1605
e 0.01 a = 1.1605
0.01 a = 0.1488
a = 14.88
Therefore the required age is 14.88+30 = 44.88 days.
Most candidates answered part (i) of this question correctly. Part (ii) was less well answered,
and only a minority of candidates managed to obtain the correct answer to part (iii). A
common error was to use the limits 40 and 30 when integrating 0.05exp(0.01x).
Question 7
(i)
It is a stochastic process in discrete or continuous time.
The state space is all the natural numbers {0, 1, 2, ... }
The value of the process X(t) is a non-decreasing (OR an increasing) function of
time t
OR
the value of the process goes up one at a time.
(ii)
(a)
Process State space Time set
Simple random walk
Compound Poisson process
Markov Chain Discrete
Either
Discrete Discrete
Continuous
Discrete
Page 7Subject CT4 (Models Core Technical) — Examiners’ Report, April 2011
(b)
Process Application
Simple random walk The number of customers in the
shop each time the door is opened
Compound Poisson process The weight of almonds remaining in
stock at any time in the day.
OR
value of goods sold at any time during
the day
Markov chain
The number of customers owning
loyalty cards at the end of each week.
In part (i), it was not sufficient just to say “discrete state space”. The fact the state space is
all natural numbers should be indicated for credit. In part (ii)(B), some candidates only gave
one example IN TOTAL, whereas the question asked for an example FOR EACH PROCESS.
In part (ii)(b) other examples were given credit. The criterion used to award credit were
whether the example COULD be modelled using the relevant process and how USEFUL to
the shopkeeper such a model might be!
