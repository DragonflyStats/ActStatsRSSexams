\documentclass[a4paper,12pt]{article}

%%%%%%%%%%%%%%%%%%%%%%%%%%%%%%%%%%%%%%%%%%%%%%%%%%%%%%%%%%%%%%%%%%%%%%%%%%%%%%%%%%%%%%%%%%%%%%%%%%%%%%%%%%%%%%%%%%%%%%%%%%%%%%%%%%%%%%%%%%%%%%%%%%%%%%%%%%%%%%%%%%%%%%%%%%%%%%%%%%%%%%%%%%%%%%%%%%%%%%%%%%%%%%%%%%%%%%%%%%%%%%%%%%%%%%%%%%%%%%%%%%%%%%%%%%%%

\usepackage{eurosym}
\usepackage{vmargin}
\usepackage{amsmath}
\usepackage{graphics}
\usepackage{epsfig}
\usepackage{enumerate}
\usepackage{multicol}
\usepackage{subfigure}
\usepackage{fancyhdr}
\usepackage{listings}
\usepackage{framed}
\usepackage{graphicx}
\usepackage{amsmath}
\usepackage{chngpage}

%\usepackage{bigints}
\usepackage{vmargin}

% left top textwidth textheight headheight

% headsep footheight footskip

\setmargins{2.0cm}{2.5cm}{16 cm}{22cm}{0.5cm}{0cm}{1cm}{1cm}

\renewcommand{\baselinestretch}{1.3}

\setcounter{MaxMatrixCols}{10}

\begin{document}
\begin{enumerate}
%%[Total 8]
%%CT4 A2011—3
Question 8
(i)
Explain the difference between the central and the initial exposed to
risk, in the context of mortality investigations.

An investigation studied the mortality of infants aged under 1 year. The following table gives details of 10 lives involved in the investigation. Infants with no date of
death given were still alive on their first birthday.
Life
1
2
3
4
5
6
7
8
9
10
9
Date of birth
1 August 2008
1 September 2008
1 December 2008
1 January 2009
1 February 2009
1 March 2009
1 June 2009
1 July 2009
1 September 2009
1 November 2009
Date of death
-
-
1 February 2009
-
-
1 December 2009
-
-
-
1 December 2009

\begin{enumerate}[(a)]
\item (ii) Calculate the maximum likelihood estimate of the force of mortality, using a
two-state model and assuming that the force is constant.
\item 
(iii) Hence estimate the infant mortality rate, q 0 .
\item 
(iv) Estimate the infant mortality rate, q 0 , using the initial exposed to risk.
\item 
(v) Explain the difference between the two estimates.
\end{enumerate}
%%%%%%%%%%%%%
\newpage

%%%%%%%%%%%%%%%%%%%%%%%%%%%%%%%%%%%%%%%%%%%%%%%%%%%%%%%%%%%%%%%%%%%%%%%%%%%%%%%%%%%%%%
Question 8
(i)
EITHER
The central exposed to risk at age x, E x c , is the waiting time in a multiple-state or
Poisson model.
The initial exposed to risk is equal to the central exposed to risk plus the time elapsing
between the date of death and the end of the rate interval for those who are observed
to die during the rate interval.
OR
If the age at entry of life i is x + a i , and the age at exit is x+b i for lives which do not
die, and x+t i for lives who die, then the central exposed to risk is equal to
∑ [( x i + b i ) − ( x i − a i )] = ∑ ( b i − a i ) for lives who do not die, and
i
i
∑ [( x + t ) − ( x − a )] = ∑ ( t − a ) for lives who die.
i
i
i
i
i
i
i
i
The initial exposed to risk is given by the central exposed to risk plus a quantity equal
to ∑ (1 − t i ) for the lives who die.
i
%%-- 8
%%-- Subject CT4 
%%%%%%%%%%%%%%%%%%%%%%%%%%%%%%%%%%%%%%%%%% — Examiners’ Report, April 2011
If the rate interval is the year of age between exact ages x and x+1, and if deaths are
approximately uniformly distributed across the year of age, the initial exposed to risk
is approximately equal to E x c + 0.5 d x , where d x is the number of deaths between exact
ages x and x+1.
The central exposed to risk estimates \mu x whereas the initial exposed-to risk estimates
q x .
(ii)
The maximum likelihood estimate of the force of mortality in the two-state model is
deaths divided by the central exposed to risk.
The central exposed to risk is calculated as shown in the table below.
Life Entry into
observation Exit from
observation
1
2
3
4
5
6
7
8
9
10 1 August 2008
1 September 2008
1 December 2008
1 January 2009
1 February 2009
1 March 2009
1 June 2009
1 July 2009
1 September 2009
1 November 2009 1 August 2009
1 September 2009
1 February 2009
1 January 2010
1 February 2010
1 December 2009
1 June 2010
1 July 2010
1 September 2010
1 December 2009
Months
exposed
to risk
12
12
2
12
12
9
12
12
12
1
The total number of months exposed to risk is therefore
12 + 12 + 2 + 12 + 12 + 9 + 12 + 12 + 12 + 1 = 96
which is 8 years
There were 3 deaths.
Therefore the maximum likelihood estimate of the force of mortality is
(iii)
3
= 0.375 .
8
If the force of mortality is \mu 0 , then
q 0 = 1 − exp( −\mu 0 ) = 1 − exp( − 0.375) = 0.3127 .
EITHER ALTERNATIVE 1
(iv)
The initial exposed to risk, E 0 is approximately equal to E 0 c + 0.5 d 0 , where E 0 c is the
central exposed to risk and d 0 is the number of deaths.
%% --- 9 Subject CT4 %%%%%%%%%%%%%%%%%%%%%%%%%%%%%%%%%%%%%%%%%% — Examiners’ Report, April 2011
Therefore we have
q 0 * =
(v)
d 0
E 0 c
+ 0.5 d 0
=
3
3
=
= 0.3158 .
8 + 0.5(3) 9.5
q 0 * is calculated assuming a uniform distribution of deaths over the year of age
between birth and exact age 1 year, whereas q 0 assumes a constant force of mortality
between exact ages 0 and 1.
These assumptions are different, implying a different distribution of deaths over the
first year of life.
OR ALTERNATIVE 2
(iv) As the only way of leaving observation is through death, the initial exposed to risk is 10
3
and q 0 * =
= 0.3 .
10
(v) q 0 * is calculated using the exact initial exposed to risk, making no assumptions about
the shape of the force of mortality during the interval,
OR
\begin{itemize}
\item In the calculation of q 0 * lives could die at any time during the year of age, so they are
treated as being exposed to risk for the entire year, whereas q 0 assumes a constant force
of mortality between exact ages 0 and 1, which implies an assumption about the
distribution of deaths over this interval.
\item In part (i) full credit could be obtained for rather less than is written in the solution above.
\item Credit can be given for any clear algebraic expressions in terms of the entry age x+a i , the
age at death, x+t i and the age at exit if the life did not die, x+b i , which made clear the
difference between the central and initial exposeds to risk.
\item In part (v) the wording did not have to be precise. The Examiners were looking for some
understanding of the idea that different assumptions are made about the shape of the force of
mortality over the rate interval.
\end{itemize}
\end{document}
