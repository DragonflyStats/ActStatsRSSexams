\documentclass[a4paper,12pt]{article}

%%%%%%%%%%%%%%%%%%%%%%%%%%%%%%%%%%%%%%%%%%%%%%%%%%%%%%%%%%%%%%%%%%%%%%%%%%%%%%%%%%%%%%%%%%%%%%%%%%%%%%%%%%%%%%%%%%%%%%%%%%%%%%%%%%%%%%%%%%%%%%%%%%%%%%%%%%%%%%%%%%%%%%%%%%%%%%%%%%%%%%%%%%%%%%%%%%%%%%%%%%%%%%%%%%%%%%%%%%%%%%%%%%%%%%%%%%%%%%%%%%%%%%%%%%%%

\usepackage{eurosym}
\usepackage{vmargin}
\usepackage{amsmath}
\usepackage{graphics}
\usepackage{epsfig}
\usepackage{enumerate}
\usepackage{multicol}
\usepackage{subfigure}
\usepackage{fancyhdr}
\usepackage{listings}
\usepackage{framed}
\usepackage{graphicx}
\usepackage{amsmath}
\usepackage{chngpage}

%\usepackage{bigints}
\usepackage{vmargin}

% left top textwidth textheight headheight

% headsep footheight footskip

\setmargins{2.0cm}{2.5cm}{16 cm}{22cm}{0.5cm}{0cm}{1cm}{1cm}

\renewcommand{\baselinestretch}{1.3}

\setcounter{MaxMatrixCols}{10}

\begin{document}
\begin{enumerate}
PLEASE TURN OVER6
A recording instrument is set up to observe a continuous time process, and stores the
results for the most recent 250 transitions. The data collected are as follows:
State
i Total time
spent in
state i
(hours)
Number of transitions to
State A
State B
State C
A 35 Not
applicable 60 45
B 150 50 Not
applicable 25
C 210 55 15 Not
applicable
It is proposed to fit a Markov jump model using the data.
(i)
(a)
(b)
State all the parameters of the model.
Outline the assumptions underlying the model.

(ii)
(a)
(b)
Estimate the parameters of the model.
Write down the estimated generator matrix of the model.

(iii)
7
Specify the distribution of the number of transitions from state i to state j,
given the number of transitions out of state i.

[Total 9]
A study is made of the impact of regular exercise and gender on the risk of
developing heart disease among 50–70 year olds. A sample of people is followed from
exact age 50 years until either they develop heart disease or they attain the age of 70
years. The study uses a Cox regression model.
(i)
List reasons why the Cox regression model is a suitable model for analyses of
this kind.

The investigator defined two covariates as follows:
•
•
Z 1 = 1 if male, 0 if female.
Z 2 = 1 if takes regular exercise, 0 otherwise.
CT4 S2011—4The investigator then fitted three models, one with just gender as a covariate, a second
with gender and exercise as covariates, and a third with gender, exercise and the interaction between them as covariates. The maximised log-likelihoods of the three
models and the maximum likelihood estimates of the parameters in the third model
were as follows:
null model
gender
gender + exercise
gender + exercise + interaction –1,269
–1,256
–1,250
–1,246
Covariate Parameter
Gender
Exercise
Interaction 0.2
–0.3
–0.35
(ii)
(iii)
Show that the interaction term is required in the model by performing a
suitable statistical test.
Interpret the results of the model.
CT4 S2011—5


[Total 11]



%%%%%%%%%%%%%%%%%%%%%%%%%%%%%%%%%%%%%%%%%%%%%%%%%%%%%%%%%%%%%%%%%%%%%%%%%%%






%%%%%%%%%%%%%%%%%%%%%%%%%%%%%%%%%%%%%%%%%%%%%%%%%%%%%%%%%%%%%%%%%%%%%%%%%%%
Question 7
(i)
Cox’s model ensures that the hazard is always positive.
Standard software packages often include Cox’s model.
Cox’s model allows the general “shape” of the hazard function for all individuals to
be determined by the data, giving a high degree of flexibility,
The data in this investigation are censored, and Cox’s model can handle censored
data.
Page 8%%%%%%%%%%%%%%%%%%%%%%%%%%%%%%%%%%%%%%%%%%%%%%%%%%%5 — Examiners’ Report, September 2011
In Cox’s model the hazards of individuals with different values of the covariates are proportional, meaning that they bear the same ratio to one another at all ages.
If we are not primarily concerned with the precise form of the hazard, we can ignore
the shape of the baseline hazard and estimate the effects of the covariates from
the data directly.
(ii)
A suitable statistical test is that using the likelihood ratio statistic.
We compare the model with gender + exercise with the model with gender + exercise +
the interaction.
If the log-likelihood for these two models are L and L interaction respectively, then the test
statistic is −2( L − L interaction ).
This is equal to −2{−1,250 – (−1,246)} = −2(−4) = 8.
Under the null hypothesis that the parameter on the interaction term is zero, this statistic has a chi-squared distribution with one degree of freedom (since the interaction term
involves one parameter).
Since 8 > 7.879, the critical value of the chi-squared distribution at the 0.5% level (or 8
> 3.84 for the 5% level),
we reject the null hypothesis even at the 99.5% level (or 95% level) and conclude that
the interaction term is required in the model.
(iii)
The baseline category is females who do not take regular exercise.
The hazards of developing heart disease in the other three categories, relative to the
baseline category, are as follows:
Gender Regular exercise
Male
Male
Female No
Yes
Yes
exp(0.2) = 1.22
exp(0.2 – 0.3 – 0.35) = 0.64
exp(−0.3) = 0.74
Males who do not take regular exercise are more likely to develop heart disease than
females.
Regular exercise decreases the risk of heart disease for both males and females.
The effect of regular exercise in reducing the risk of heart disease is greater for males
than for females, so much so that among those who take regular exercise, males have a
lower risk of developing heart disease than females.
There was a wide variation of performance among candidates on this question. Answers to part (i) suffered from wordiness and lack of precision, giving general descriptions of the
model rather than focusing on its attractive qualities. Part (ii) was very well answered by
Page 9%%%%%%%%%%%%%%%%%%%%%%%%%%%%%%%%%%%%%%%%%%%%%%%%%%%5 — Examiners’ Report, September 2011
many candidates. In part (iii) many candidates seemed not to understand the interpretation of
the interaction term. For example, it was common to read that males had a higher risk of
heart disease than females. However, this is only true for persons who do not take regular exercise. Among persons who do take regular exercise, females have a higher risk of heart
disease than males.
