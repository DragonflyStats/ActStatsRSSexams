\documentclass[a4paper,12pt]{article}

%%%%%%%%%%%%%%%%%%%%%%%%%%%%%%%%%%%%%%%%%%%%%%%%%%%%%%%%%%%%%%%%%%%%%%%%%%%%%%%%%%%%%%%%%%%%%%%%%%%%%%%%%%%%%%%%%%%%%%%%%%%%%%%%%%%%%%%%%%%%%%%%%%%%%%%%%%%%%%%%%%%%%%%%%%%%%%%%%%%%%%%%%%%%%%%%%%%%%%%%%%%%%%%%%%%%%%%%%%%%%%%%%%%%%%%%%%%%%%%%%%%%%%%%%%%%

\usepackage{eurosym}
\usepackage{vmargin}
\usepackage{amsmath}
\usepackage{graphics}
\usepackage{epsfig}
\usepackage{enumerate}
\usepackage{multicol}
\usepackage{subfigure}
\usepackage{fancyhdr}
\usepackage{listings}
\usepackage{framed}
\usepackage{graphicx}
\usepackage{amsmath}
\usepackage{chngpage}

%\usepackage{bigints}
\usepackage{vmargin}

% left top textwidth textheight headheight

% headsep footheight footskip

\setmargins{2.0cm}{2.5cm}{16 cm}{22cm}{0.5cm}{0cm}{1cm}{1cm}

\renewcommand{\baselinestretch}{1.3}

\setcounter{MaxMatrixCols}{10}

\begin{document}
\begin{enumerate}

1 Give three advantages of the two-state model over the Binomial model for estimating transition intensities where exact dates of entry into and exit from observation are
known.
[3]
2 Distinguish between the conditions under which a Markov chain:
(a)
(b)
(c)
has at least one stationary distribution.
has a unique stationary distribution.
converges to a unique stationary distribution.
[3]
3 Describe the ways in which the design of a model used to project over only a short time frame may differ from one used to project over fifty years.
[4]
4 Children at a school are given weekly grade sheets, in which their effort is graded in four levels: 1 “Poor”, 2 “Satisfactory”, 3 “Good” and 4 “Excellent”. Subject to a
maximum level of Excellent and a minimum level of Poor, between each week and the next, a child has:
•
•
•
•
a 20 per cent chance of moving up one level.
a 20 per cent chance of moving down one level.
a 10 per cent chance of moving up two levels.
a 10 per cent chance of moving down two levels.
Moving up or down three levels in a single week is not possible.
(i)
Write down the transition matrix of this process.
[2]
Children are graded on Friday afternoon in each week. On Friday of the first week of the school year, as there is little evidence on which to base an assessment, all children
are graded “Satisfactory”.
(ii)
Calculate the probability distribution of the process after the grading on Friday of the third week of the school year.
[3]
[Total 5]
CT4 A2011—25
(i)
Explain why a mortality experience would need to be graduated.
[3]
An actuary has conducted investigations into the mortality of the following classes of
lives:
(a) the female members of a medium-sized pension scheme
(b) the male population of a large industrial country
(c) the population of a particular species of reptile in the zoological
collections of the southern hemisphere
The actuary wishes to graduate the crude rates.
(ii)


%%%%%%%%%%%%%%%%%%%%%%%%%%%%%%%%%%%%%%%%%%%%%%%%%%%%%%%%%%%%%%%%%%%%%%%%%%%%%%%%%%%%

Question 1
We can calculate the maximum likelihood estimate (MLE) of the transition intensities directly using the two-state model, whereas the Binomial model requires additional
assumptions.
The variance of the Binomial estimate is greater than that of the estimate from the two-state model (though the difference is tiny unless the transition intensities are large).
The MLE in the two-state model is consistent and unbiased, whereas the Binomial estimate is only consistent and unbiased if lives are observed for exactly one year, which is rarely the
case.
The two-state model is easily extended to encompass increments and additional decrements, whereas the Binomial model is not.
The two-state model uses the exact times of the transitions, whereas the Binomial model only uses the number of transitions.
This question was poorly answered by many candidates, despite being straightforward bookwork. Many candidates commented that the two-state model and the Binomial model
make different assumptions about the shape of the force of mortality within the year of age.
This was only be given credit if candidates also explained why the multiple state model’s assumption is BETTER than the Binomial model’s assumption (which it might be, for
example, at younger ages).
Full marks could be obtained for giving three reasons. It was not necessary to give all the
points listed above in order to obtain full marks.
Question 2
(a) A Markov chain with a finite state space has at least one stationary probability
distribution.
(b) An irreducible Markov chain with a finite state space has a unique stationary probability distribution.
(c) A Markov chain with a finite state space which is irreducible, and which is also aperiodic converges to a unique stationary probability distribution.
Many candidates scored full marks on this question. The question asked candidates to “distinguish”. Therefore for full credit it is important that candidates did, indeed,
understand and make the relevant distinction.

Page 2Subject CT4 (Models Core Technical) — Examiners’ Report, April 2011

\newpage

Question 3
Individual variables may behave differently, for example a model over 50 years may be more
sensitive to differences in the input values of certain variables than one over the short term.
A variable which has an ignorable effect in the short term may have a non-ignorable effect
over 50 years.
Over the short term, it may be reasonable to assume the values of some variables to be
constant or to vary linearly, whereas this would not be reasonable over 50 years. For
example, growth which is exponential may appear linear if studied over a short time frame.
The interaction between variables in the short-term may be different from that over the long-
term.
Higher order relationships between variables may be ignored for simplicity if modelling over
a short time frame.
The time units used in the model might be shorter for a model projecting over a short time
frame, so that the total number of time units used in each model is roughly the same.
Over 50 years, regulatory changes and other “shock” events are more likely to occur, and the
model design may need to consider the circumstances in which the results or conclusions may
be materially impacted (e.g. in the short term the tax basis may be known, but in the long run
it is likely to change).
The marks on this question were the lowest on any question. The question was a “higher
skills” question and so required candidates to think about the context. Little credit was given
to candidates who uncritically reproduced sections of the Core Reading. In particular, the
question is about model DESIGN, so the points made should relate to the design of the
model.
Page 3Subject CT4 (Models Core Technical) — Examiners’ Report, April 2011
