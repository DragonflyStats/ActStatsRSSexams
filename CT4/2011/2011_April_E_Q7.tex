\documentclass[a4paper,12pt]{article}

%%%%%%%%%%%%%%%%%%%%%%%%%%%%%%%%%%%%%%%%%%%%%%%%%%%%%%%%%%%%%%%%%%%%%%%%%%%%%%%%%%%%%%%%%%%%%%%%%%%%%%%%%%%%%%%%%%%%%%%%%%%%%%%%%%%%%%%%%%%%%%%%%%%%%%%%%%%%%%%%%%%%%%%%%%%%%%%%%%%%%%%%%%%%%%%%%%%%%%%%%%%%%%%%%%%%%%%%%%%%%%%%%%%%%%%%%%%%%%%%%%%%%%%%%%%%

\usepackage{eurosym}
\usepackage{vmargin}
\usepackage{amsmath}
\usepackage{graphics}
\usepackage{epsfig}
\usepackage{enumerate}
\usepackage{multicol}
\usepackage{subfigure}
\usepackage{fancyhdr}
\usepackage{listings}
\usepackage{framed}
\usepackage{graphicx}
\usepackage{amsmath}
\usepackage{chngpage}

%\usepackage{bigints}
\usepackage{vmargin}

% left top textwidth textheight headheight

% headsep footheight footskip

\setmargins{2.0cm}{2.5cm}{16 cm}{22cm}{0.5cm}{0cm}{1cm}{1cm}

\renewcommand{\baselinestretch}{1.3}

\setcounter{MaxMatrixCols}{10}

\begin{document}

%% [Total 8]
%%%%%%%%%%%%%%%%%%%%%%%%%%%%%%%%%%%%%%%%%%%%%%%%%%%%%%%%%%%%%%%%%%%%%%%%%%%%%%%%%%%%%%%%%%%
(i) Define a counting process.

For each of the following processes:
\begin{itemize}
\item simple random walk
\item compound Poisson
\item Markov chain
\end{itemize}
\begin{itemize}
\item (a) State whether each of the state space and the time set is discrete,
continuous or can be either.
\item (b) Give an example of an application which may be useful to a
shopkeeper selling dried fruit and nuts loose.
\end{itemize}

%%%%%%%%%%%%%%%%%%%%%%%%%%%%%%%%%%%%%%%%%%%%%%%%%%%%%%%%5
\newpage

%% Question 7
(i)
\begin{itemize}
\item It is a stochastic process in discrete or continuous time.
\item The state space is all the natural numbers $\{0, 1, 2, ... \}$
\item The value of the process $X(t)$ is a non-decreasing (OR an increasing) function of
time t
OR

\item the value of the process goes up one at a time.
\end{itemize}

(ii)
(a)
%%%%%%%%%%%%%%%%%%%%%%%%%%%%%%%%%%%%%%%%%%%%%%%

%- April 2011
%- Question 7

(a)
\begin{center}
\begin{tabular}{|c|c|c|}
Process  & State space & Time set \\ \hline
Simple random walk  & Discrete & Discrete \\
Compound Poisson process & Either & Continuous \\ \hline
Markov Chain & Discrete & Discrete \\ \hline
\end{tabular}
\end{center}
%%%%%%%%%%%%%%%%%%%%%%%%%%%%%%%%%%%%%%%%%%%%%%%%%%%%%%%
(b)
Process Application
Simple random walk The number of customers in the
shop each time the door is opened
Compound Poisson process The weight of almonds remaining in
stock at any time in the day.
OR
value of goods sold at any time during
the day
Markov chain
The number of customers owning
loyalty cards at the end of each week.


%In part (i), it was not sufficient just to say “discrete state space”. The fact the state space is all natural numbers should be indicated for credit. In part (ii)(B), some candidates only gave one example IN TOTAL, whereas the question asked for an example FOR EACH PROCESS.
% In part (ii)(b) other examples were given credit. The criterion used to award credit were whether the example COULD be modelled using the relevant process and how USEFUL to the shopkeeper such a model might be!
\end{document}
