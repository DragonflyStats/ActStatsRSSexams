\documentclass[a4paper,12pt]{article}

%%%%%%%%%%%%%%%%%%%%%%%%%%%%%%%%%%%%%%%%%%%%%%%%%%%%%%%%%%%%%%%%%%%%%%%%%%%%%%%%%%%%%%%%%%%%%%%%%%%%%%%%%%%%%%%%%%%%%%%%%%%%%%%%%%%%%%%%%%%%%%%%%%%%%%%%%%%%%%%%%%%%%%%%%%%%%%%%%%%%%%%%%%%%%%%%%%%%%%%%%%%%%%%%%%%%%%%%%%%%%%%%%%%%%%%%%%%%%%%%%%%%%%%%%%%%

\usepackage{eurosym}
\usepackage{vmargin}
\usepackage{amsmath}
\usepackage{graphics}
\usepackage{epsfig}
\usepackage{enumerate}
\usepackage{multicol}
\usepackage{subfigure}
\usepackage{fancyhdr}
\usepackage{listings}
\usepackage{framed}
\usepackage{graphicx}
\usepackage{amsmath}
\usepackage{chngpage}

%\usepackage{bigints}
\usepackage{vmargin}

% left top textwidth textheight headheight

% headsep footheight footskip

\setmargins{2.0cm}{2.5cm}{16 cm}{22cm}{0.5cm}{0cm}{1cm}{1cm}

\renewcommand{\baselinestretch}{1.3}

\setcounter{MaxMatrixCols}{10}

\begin{document}
\begin{enumerate}

1 Give three advantages of the two-state model over the Binomial model for estimating transition intensities where exact dates of entry into and exit from observation are
known.

2 Distinguish between the conditions under which a Markov chain:
(a)
(b)
(c)
has at least one stationary distribution.
has a unique stationary distribution.
converges to a unique stationary distribution.

%%%%%%%%%%%%%%%%%%%%%%%%%%%%%%%%%%%%%%%%%%%%%%%%%%%%%%%%%%%%%%%%%%%%%%%%%%%%%%%%%%%%

Question 1
We can calculate the maximum likelihood estimate (MLE) of the transition intensities directly using the two-state model, whereas the Binomial model requires additional
assumptions.
The variance of the Binomial estimate is greater than that of the estimate from the two-state model (though the difference is tiny unless the transition intensities are large).
The MLE in the two-state model is consistent and unbiased, whereas the Binomial estimate is only consistent and unbiased if lives are observed for exactly one year, which is rarely the
case.
The two-state model is easily extended to encompass increments and additional decrements, whereas the Binomial model is not.
The two-state model uses the exact times of the transitions, whereas the Binomial model only uses the number of transitions.
This question was poorly answered by many candidates, despite being straightforward bookwork. Many candidates commented that the two-state model and the Binomial model
make different assumptions about the shape of the force of mortality within the year of age.
This was only be given credit if candidates also explained why the multiple state model’s assumption is BETTER than the Binomial model’s assumption (which it might be, for
example, at younger ages).
Full marks could be obtained for giving three reasons. It was not necessary to give all the
points listed above in order to obtain full marks.
%%%%%%%%%%%%%%%%%%%%%%%%%%%%%%%%%%%%%%%%%%%
\newpage
Question 2
(a) A Markov chain with a finite state space has at least one stationary probability
distribution.
(b) An irreducible Markov chain with a finite state space has a unique stationary probability distribution.
(c) A Markov chain with a finite state space which is irreducible, and which is also aperiodic converges to a unique stationary probability distribution.
Many candidates scored full marks on this question. The question asked candidates to “distinguish”. Therefore for full credit it is important that candidates did, indeed,
understand and make the relevant distinction.

Page 2Subject CT4 (Models Core Technical) — Examiners’ Report, April 2011
\end{document}
