\documentclass[a4paper,12pt]{article}

%%%%%%%%%%%%%%%%%%%%%%%%%%%%%%%%%%%%%%%%%%%%%%%%%%%%%%%%%%%%%%%%%%%%%%%%%%%%%%%%%%%%%%%%%%%%%%%%%%%%%%%%%%%%%%%%%%%%%%%%%%%%%%%%%%%%%%%%%%%%%%%%%%%%%%%%%%%%%%%%%%%%%%%%%%%%%%%%%%%%%%%%%%%%%%%%%%%%%%%%%%%%%%%%%%%%%%%%%%%%%%%%%%%%%%%%%%%%%%%%%%%%%%%%%%%%

\usepackage{eurosym}
\usepackage{vmargin}
\usepackage{amsmath}
\usepackage{graphics}
\usepackage{epsfig}
\usepackage{enumerate}
\usepackage{multicol}
\usepackage{subfigure}
\usepackage{fancyhdr}
\usepackage{listings}
\usepackage{framed}
\usepackage{graphicx}
\usepackage{amsmath}
\usepackage{chngpage}

%\usepackage{bigints}
\usepackage{vmargin}

% left top textwidth textheight headheight

% headsep footheight footskip

\setmargins{2.0cm}{2.5cm}{16 cm}{22cm}{0.5cm}{0cm}{1cm}{1cm}

\renewcommand{\baselinestretch}{1.3}

\setcounter{MaxMatrixCols}{10}

\begin{document}
\begin{enumerate}

3 Describe the ways in which the design of a model used to project over only a short time frame may differ from one used to project over fifty years.

%%%%%%%%%%%%%%%%%%%%%%%%%%%%%%%%%%%%%%%%%%%%%%%%%%%%%%%%%%%%%%%%%%%%%%%%%%%%%%%%%%%%


\newpage

Question 3
Individual variables may behave differently, for example a model over 50 years may be more
sensitive to differences in the input values of certain variables than one over the short term.
A variable which has an ignorable effect in the short term may have a non-ignorable effect
over 50 years.
Over the short term, it may be reasonable to assume the values of some variables to be
constant or to vary linearly, whereas this would not be reasonable over 50 years. For
example, growth which is exponential may appear linear if studied over a short time frame.
The interaction between variables in the short-term may be different from that over the long-
term.
Higher order relationships between variables may be ignored for simplicity if modelling over
a short time frame.
The time units used in the model might be shorter for a model projecting over a short time
frame, so that the total number of time units used in each model is roughly the same.
Over 50 years, regulatory changes and other “shock” events are more likely to occur, and the
model design may need to consider the circumstances in which the results or conclusions may
be materially impacted (e.g. in the short term the tax basis may be known, but in the long run
it is likely to change).
The marks on this question were the lowest on any question. The question was a “higher
skills” question and so required candidates to think about the context. Little credit was given
to candidates who uncritically reproduced sections of the Core Reading. In particular, the
question is about model DESIGN, so the points made should relate to the design of the
model.
%%--- Page 3Subject CT4 (Models Core Technical) — Examiners’ Report, April 2011

%%%%%%%%%%%%%%%%%%%%%%%%%%%%%%%%%%%%%%%%%%%

4 Children at a school are given weekly grade sheets, in which their effort is graded in four levels: 1 “Poor”, 2 “Satisfactory”, 3 “Good” and 4 “Excellent”. Subject to a
maximum level of Excellent and a minimum level of Poor, between each week and the next, a child has:
\item
\item
\item
\item
a 20 per cent chance of moving up one level.
a 20 per cent chance of moving down one level.
a 10 per cent chance of moving up two levels.
a 10 per cent chance of moving down two levels.
Moving up or down three levels in a single week is not possible.
(i)
Write down the transition matrix of this process.

Children are graded on Friday afternoon in each week. On Friday of the first week of the school year, as there is little evidence on which to base an assessment, all children
are graded “Satisfactory”.
(ii)
Calculate the probability distribution of the process after the grading on Friday of the third week of the school year.

[Total 5]
\newpage
Question 4
(i)
(ii)
⎛ 0.7
⎜
⎜ 0.3
⎜ 0.1
⎜
⎝ 0
0.2
0.4
0.2
0.1
0.1 0 ⎞
⎟
0.2 0.1 ⎟
0.4 0.3 ⎟
⎟
0.2 0.7 ⎠
If the probability distribution in the first week is Π , and the transition matrix is M,
then the probability distribution at the end of the third week is
⎛ 0.7
⎜
0.3
Π M 2 = ( 0 1 0 0 ) ⎜
⎜ 0.1
⎜
⎝ 0
0.2
0.4
0.2
0.1
⎛ 0.56
⎜
0.35
= ( 0 1 0 0 ) ⎜
⎜ 0.17
⎜
⎝ 0.05
0.1 0 ⎞ ⎛ 0.7
⎟⎜
0.2 0.1 ⎟ ⎜ 0.3
0.4 0.3 ⎟ ⎜ 0.1
⎟⎜
0.2 0.7 ⎠ ⎝ 0
0.24
0.27
0.21
0.15
0.15
0.21
0.27
0.24
0.2
0.4
0.2
0.1
0.1 0 ⎞
⎟
0.2 0.1 ⎟
0.4 0.3 ⎟
⎟
0.2 0.7 ⎠
0.05 ⎞
⎟
0.17 ⎟
0.35 ⎟
⎟
0.56 ⎠
so that there is a probability of
35% that a child will be graded Poor’,
27% that a child will be graded Satisfactory,
21% that a child will be graded Good and
17% that a child will be graded Excellent..
There were two common errors on this question. The first was to assume that if a child could
not move up or down two levels, he or she would not move at all. The phrase in the question
“[s]ubject to a maximum level of Excellent and a minimum level of Poor” was intended to
indicate that children could not move beyond these limits in either direction, but would move
as far as they could. Thus a child at level “Good”, who had a 20% chance of moving up one
level and a 10% chance of moving up two levels, would have a 30% chance of moving to
level Excellent, as the 10% who would have moved up two levels will only be able to move up
one level. The second error was to use Π M 3 in part (ii). Candidates who made the first
error were penalised in part (i) but could gain full credit for part (ii) if they followed through
correctly.
\end{document}
