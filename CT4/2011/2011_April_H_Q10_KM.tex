\documentclass[a4paper,12pt]{article}

%%%%%%%%%%%%%%%%%%%%%%%%%%%%%%%%%%%%%%%%%%%%%%%%%%%%%%%%%%%%%%%%%%%%%%%%%%%%%%%%%%%%%%%%%%%%%%%%%%%%%%%%%%%%%%%%%%%%%%%%%%%%%%%%%%%%%%%%%%%%%%%%%%%%%%%%%%%%%%%%%%%%%%%%%%%%%%%%%%%%%%%%%%%%%%%%%%%%%%%%%%%%%%%%%%%%%%%%%%%%%%%%%%%%%%%%%%%%%%%%%%%%%%%%%%%%

\usepackage{eurosym}
\usepackage{vmargin}
\usepackage{amsmath}
\usepackage{graphics}
\usepackage{epsfig}
\usepackage{enumerate}
\usepackage{multicol}
\usepackage{subfigure}
\usepackage{fancyhdr}
\usepackage{listings}
\usepackage{framed}
\usepackage{graphicx}
\usepackage{amsmath}
\usepackage{chngpage}

%\usepackage{bigints}
\usepackage{vmargin}

% left top textwidth textheight headheight

% headsep footheight footskip

\setmargins{2.0cm}{2.5cm}{16 cm}{22cm}{0.5cm}{0cm}{1cm}{1cm}

\renewcommand{\baselinestretch}{1.3}

\setcounter{MaxMatrixCols}{10}

\begin{document}
10
At Miracle Cure hospital a pioneering new surgery was tested to replace human lungs with synthetic implants. Operations were carried out throughout June 2010. Patients
who underwent the surgery were monitored daily until the end of August 2010, or until they died or left hospital if sooner. The results are shown below. Where no date
is given, the patient was alive and still in hospital at the end of August.

(i)
Patient Date of surgery Date of leaving
observation Reason for
leaving
observation
A
B
C
D
E
F
G
H
I
J
K
L
M
N June 1
June 3
June 5
June 8
June 9
June 12
June 16
June 17
June 22
June 24
June 25
June 26
June 29
June 30 June 3
July 2 Died
Left Hospital
July 11 Died
June 21
Aug 12 Died
Left Hospital
June 29
Aug 20 Died
Died
Aug 6 Left Hospital
Explain whether each of the following types of censoring is present and for
those present explain where they occur:
•
•
•
right censoring
left censoring
informative censoring

(ii)
Calculate the Kaplan-Meier estimate of the survival function for these
patients, stating all assumptions that you make.
[6]
(iii) Sketch, on a suitably labelled graph, the Kaplan-Meier estimate of the survival
function.

(iv) Estimate the probability that a patient will die within four weeks of surgery.

%%%%%%%%%%%%%%%%%%%%%%%%%%%%%%%%%%%%%%%%%%%%%%%%%%%%%%%%%%%%%%%%%%%%%%%%%%%%%%%%%%%
\newpage
Question 10
(i)
Right censoring is present
for those still alive and in hospital at the end of August
OR
for those who left hospital while still alive
Left censoring is not present
The censoring is likely to be informative, since those leaving hospital are
likely to be in much better health than those who remain. (The idea of going home to
die when you have had a lung transplant is a little tenuous.)
(ii)
The durations and outcomes are shown in the table below.
Patient Died/Censored Duration
A
G
J
B
E
M
H
K
N
L
I
F
D
C Died
Died
Died
Censored
Died
Censored
Censored
Died
Censored
Censored
Censored
Censored
Censored
Censored 2
5
5
29
32
38
56
56
62
66
70
80
84
87
%%-- Page 13
%%-- Subject CT4 (Models Core Technical) — Examiners’ Report, April 2011
EITHER ALTERNATIVE 1
Assuming that at duration 56 the death occurred before the life was censored, the
Kaplan-Meier estimate is as follows:
t j n j d j c j
0
2
5
32
56
+1⁄2 14
14
13
10
8
+1⁄2 0
1
2
1
1
+1⁄2 0
0
1
1
7
+1⁄2
λ j =
d j
n j
0
1/14
2/13
1/10
1/8
The Kaplan-Meier estimate at duration t is given by the product of 1 −
d j
n j
over
durations up to and including t. Thus the Kaplan-Meier estimate of the survival
function is
^
t S ( t )
0≤ t < 2
2≤ t < 5
5≤ t < 32
32≤ t < 56
56≤ t < 92 1.0000
0.9286
0.7857
0.7071
0.6188
OR
OR
OR
OR
13/14
11/14
99/140
99/160
OR ALTERNATIVE 2
Assuming that at duration 56 the death occurred after the life was
censored, the Kaplan-Meier estimate is as follows:
Page 14
t j n j d j c j
0
2
5
32
56 14
14
13
10
7 0
1
2
1
1 0
0
1
2
6
λ j =
d j
n j
0
1/14
2/13
1/10
1/7
%%-- Subject CT4 (Models Core Technical) — Examiners’ Report, April 2011
The Kaplan-Meier estimate at duration t is given by the product of 1 −
d j
n j
over
durations up to and including t. Thus the Kaplan-Meier estimate of the survival
function is
^
t S ( t )
0≤ t < 2
2≤ t < 5
5≤ t < 32
32≤ t < 56
56≤ t < 92 1.0000
0.9286
0.7857
0.7071
0.6061
OR 13/14
OR 11/14
OR 99/140
OR 297/490
(iii)
(iv)
The probability of death within 4 weeks is 1 – S(28) = 0.2143.
In part (i) candidates could receive credit for saying that left censoring was present IF they gave a valid reason (which typically involved the imprecise measurement of the times of
surgery or of events – the left censoring arising as a special case of interval censoring).
In part (ii) each error was only penalised once. Correct calculations which carried forward earlier errors were given full credit. However, candidates who did not list the durations they
were using, but then presented incorrect estimates of the survival function, were more heavily
penalised, as it was not clear how many errors they had made.
In part (ii) candidates who assume that the death at duration 56 takes place after the censoring at the same duration (ALTERNATIVE 2) were required to state this assumption for
full credit. For ALTERNATIVE 1, the assumption that the death at duration 56 takes place before the censoring does not need to be stated for full credit, as it is the convention when
calculating Kaplan-Meier estimates.
Page 15Subject CT4 (Models Core Technical) — Examiners’ Report, April 2011
In part (iii) the plotted function should be consistent with the answer to part (ii). If the
answer to part (ii) was incorrect but the incorrect answer to part (ii) was correctly plotted in
part (iii), full credit could be awarded to part (iii).
\end{document}
