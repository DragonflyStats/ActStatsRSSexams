\documentclass[a4paper,12pt]{article}

%%%%%%%%%%%%%%%%%%%%%%%%%%%%%%%%%%%%%%%%%%%%%%%%%%%%%%%%%%%%%%%%%%%%%%%%%%%%%%%%%%%%%%%%%%%%%%%%%%%%%%%%%%%%%%%%%%%%%%%%%%%%%%%%%%%%%%%%%%%%%%%%%%%%%%%%%%%%%%%%%%%%%%%%%%%%%%%%%%%%%%%%%%%%%%%%%%%%%%%%%%%%%%%%%%%%%%%%%%%%%%%%%%%%%%%%%%%%%%%%%%%%%%%%%%%%

\usepackage{eurosym}
\usepackage{vmargin}
\usepackage{amsmath}
\usepackage{graphics}
\usepackage{epsfig}
\usepackage{enumerate}
\usepackage{multicol}
\usepackage{subfigure}
\usepackage{fancyhdr}
\usepackage{listings}
\usepackage{framed}
\usepackage{graphicx}
\usepackage{amsmath}
\usepackage{chngpage}

%\usepackage{bigints}
\usepackage{vmargin}

% left top textwidth textheight headheight

% headsep footheight footskip

\setmargins{2.0cm}{2.5cm}{16 cm}{22cm}{0.5cm}{0cm}{1cm}{1cm}

\renewcommand{\baselinestretch}{1.3}

\setcounter{MaxMatrixCols}{10}

\begin{document}
\begin{enumerate}
5
(i)
Explain why a mortality experience would need to be graduated.
[3]
An actuary has conducted investigations into the mortality of the following classes of
lives:
(a) the female members of a medium-sized pension scheme
(b) the male population of a large industrial country
(c) the population of a particular species of reptile in the zoological
collections of the southern hemisphere
The actuary wishes to graduate the crude rates.
(ii)
6
State an appropriate method of graduation for each of the three classes of lives
and, for each class, briefly explain your choice.
[3]
[Total 6]
\newpage
Question 5
(i)
\begin{itemize}
\item We believe that mortality varies smoothly with age (and evidence from large
experiences supports this belief).
\item Therefore the crude estimate of mortality at any age carries information about
mortality at adjacent ages.
\item By smoothing the experience, we can make use of data at adjacent ages to improve
the estimates at each age.
\item This reduces sampling (or random) errors.
\item The mortality experience may be used in financial calculations.
\item Irregularities, jumps and anomalies in financial quantities (such as premiums for life
insurance contracts) are hard to justify to customers.
\end{itemize}
%%%%%%%%%%%%%%%
(ii)
(a)
Female members of a medium-sized pension scheme.
With reference to a standard table, because there are many extant tables
dealing with female pensioners.
(b)
Male population of a large industrial country.
By parametric formula, because the experience is large.
OR
because the graduated rates may form a new standard table for the country.
(c)
Population of a particular species of reptile in the zoological collections of the southern hemisphere.
Graphical, because no suitable standard table is likely to exist and the experience is small.
This question was well answered. In part (i)(c) BOTH elements of the reason were needed for credit (i.e. that no suitable table is likely to exist AND the experience is small).
%% 5Subject CT4 %%%%%%%%%%%%%%%%%%%%%%%%%%%%%%%%%%%%%%%%%% — Examiners’ Report, April 2011
\end{document}
