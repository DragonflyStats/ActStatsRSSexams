CT4 A2008—26
An investigation was carried out into mortality rates among a certain class of female
pensioners. Crude mortality rates were estimated by single years of age from ages
65–89 years last birthday inclusive. The investigators decided to ask an actuary to
compare the crude rates with a standard table. They calculated the relevant
standardised deviations, printed them out and sent them to the actuary.
Unfortunately, because of a printing error, the right-hand edge of the document
containing the standardised deviations failed to print properly. The actuary was
unable to read the magnitude of the standardised deviations. However, the sign of
each deviation was clear. This revealed that the crude mortality rates were higher
than the standard table rates at ages 65–72 years and 75–84 years inclusive, but that
the crude mortality rates were lower than the standard table rates at ages 73–74 years
and 85–89 years inclusive.
The null hypothesis to be tested is that the crude mortality rates come from a
population with underlying mortality consistent with that in the standard table.
7
(i) List two statistical tests of the null hypothesis which the actuary could carry
out on the basis of the information received.
[1]
(ii) Carry out both tests. For each test, state what feature of the experience it is
specifically testing, and give your conclusion.
[10]
[Total 11]
In a certain small country all listed companies are required to have their accounts
audited on an annual basis by one of the three authorised audit firms (A, B and C).
The terms of engagement of each of the audit firms require that a minimum of two
annual audits must be conducted by the newly appointed firm. Whenever a company
is able to choose to change auditors, the likelihood that it will retain its auditors for a
further year is (80%, 70%, 90%) where the current auditor is (A,B,C) respectively. If
changing auditors a company is equally likely to choose either of the alternative firms.
(i)
(ii)
A company has just changed auditors to firm A. Calculate the expected
number of audits which will be undertaken before the company changes
auditors again.
[2]
Formulate a Markov chain which can be used to model the audit firm used by
a company, specifying:
(a)
(b)
the state space
the transition matrix
[4]
(iii)
Calculate the expected proportion of companies using each audit firm in the
long term.
[5]
[Total 11]
CT4 A2008—3
PLEASE TURN OVER8
An education authority provides children with musical instrument tuition. The
authority is concerned about the number of children giving up playing their
instrument and is testing a new tuition method with a proportion of the children which
it hopes will improve persistency rates. Data have been collected and a Cox
proportional hazards model has been fitted for the hazard of giving up playing the
instrument. Symmetric 95% confidence intervals (based upon standard errors) for the
regression parameters are shown below.
Covariate
Confidence Interval
Instrument
Piano
Violin
Trumpet 0
[-0.05,0.19]
[0.07,0.21]
Tuition method
Traditional
New 0
[-0.15,0.05]
Sex
Male
Female
[-0.08,0.12]
0
(i) Write down a general expression for the Cox proportional hazards model,
defining all terms that you use.
[3]
(ii) State the regression parameters for the fitted model. [2]
(iii) Describe the class of children to which the baseline hazard applies. [1]
(iv) Discuss the suggestion that the new tuition method has improved the chances
of children continuing to play their instrument.
[3]
(v) Calculate, using the results from the model, the probability that a boy will still
be playing the piano after 4 years if provided with the new tuition method,
given that the probability that a girl will still be playing the trumpet after 4
years following the traditional method is 0.7.
[3]
[Total 12]
