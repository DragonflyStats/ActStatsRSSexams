\documentclass[a4paper,12pt]{article}

%%%%%%%%%%%%%%%%%%%%%%%%%%%%%%%%%%%%%%%%%%%%%%%%%%%%%%%%%%%%%%%%%%%%%%%%%%%%%%%%%%%%%%%%%%%%%%%%%%%%%%%%%%%%%%%%%%%%%%%%%%%%%%%%%%%%%%%%%%%%%%%%%%%%%%%%%%%%%%%%%%%%%%%%%%%%%%%%%%%%%%%%%%%%%%%%%%%%%%%%%%%%%%%%%%%%%%%%%%%%%%%%%%%%%%%%%%%%%%%%%%%%%%%%%%%%

\usepackage{eurosym}
\usepackage{vmargin}
\usepackage{amsmath}
\usepackage{graphics}
\usepackage{epsfig}
\usepackage{enumerate}
\usepackage{multicol}
\usepackage{subfigure}
\usepackage{fancyhdr}
\usepackage{listings}
\usepackage{framed}
\usepackage{graphicx}
\usepackage{amsmath}
\usepackage{chngpage}

%\usepackage{bigints}
\usepackage{vmargin}

% left top textwidth textheight headheight

% headsep footheight footskip

\setmargins{2.0cm}{2.5cm}{16 cm}{22cm}{0.5cm}{0cm}{1cm}{1cm}

\renewcommand{\baselinestretch}{1.3}

\setcounter{MaxMatrixCols}{10}

\begin{document}
\begin{enumerate}

%%%%%%%%%%%%%%%%%%%%%%%%%%%%%%%%%%%%%%%%%%%%%%%%%%%%%%%%%%%%%%%%%%%%%%%%%%%%%%%%%%%%%
8
A medical study was carried out between 1 January 2001 and 1 January 2006, to
assess the survival rates of cancer patients. The patients all underwent surgery during
2001 and then attended 3-monthly check-ups throughout the study.
The following data were collected:
For those patients who died during the study exact dates of death were recorded as
follows:
Patient Date of surgery Date of death
A
B
C
D
E 1 April 2001
1 April 2001
1 May 2001
1 September 2001
1 October 2001 1 August 2005
1 October 2001
1 March 2002
1 August 2003
1 August 2002
For those patients who survived to the end of the study:
Patient Date of surgery
F
G
H
I
J
K
L 1 February 2001
1 March 2001
1 April 2001
1 June 2001
1 September 2001
1 September 2001
1 November 2001
For those patients with whom the hospital lost contact before the end of the
investigation:
Patient Date of surgery Date of last check-up
M
N
O 1 February 2001
1 June 2001
1 September 2001 1 August 2003
1 March 2002
1 September 2005
(i)
Explain whether and where each of the following types of censoring is present
in this investigation:
(a)
(b)
(c)
(ii)
(iii)
type I censoring
interval censoring; and
informative censoring
Calculate the Kaplan-Meier estimate of the survival function for these
patients. State any assumptions that you make.
[3]
[7]
Hence estimate the probability that a patient will die within 4 years of surgery.
%%%%%%%%%%%%%%%%%%%%%%%%%%%%%%%%%%%%%%%%%%%%%%%%%%%%%%%%%%%%%%%%%%%%%%%%%%%%%%%%%%%%%
9
An insurance company is concerned that the ratio between the mortality of its female
and male pensioners is unlike the corresponding ratio among insured pensioners in
general. It conducts an investigation and estimates the mortality of male and female
f
ˆ x+
ˆ m
pensioners, μ ˆ m
x+ 1/ 2 and μ
x+ 1/ 2 to calculate what the expected
1/ 2 . It then uses the μ
mortality of its female pensioners would be if the ratio between male and female
mortality rates reflected the corresponding ratio in the PMA92 and PFA92 tables,
S x + 1/ 2 , using the formula
 x f + 1/ 2 = μ ˆ m
μ
x + 1/ 2 S x + 1/ 2 .
The table below shows, for a range of ages, the numbers of female deaths actually
observed in the investigation and the number which would be expected from the
f
 x+
μ
1/ 2 .
Age
x Actual deaths
E x c μ ˆ x f + 1/ 2 Expected deaths
 x f + 1/ 2
E x c μ
65
66
67
68
69
70
71
72 30
20
25
40
45
50
50
45 28.4
30.1
31.2
33.5
34.1
41.8
46.5
44.5
(i) Describe and carry out an overall test of the hypothesis that the ratios between
male and female death rates among the company’s pensioners are the same as
those of insured pensioners in general. Clearly state your conclusion.
[5]
(ii) Investigate further the possible existence of unusual ratios between male and
female death rates among the company’s pensioners, using two other
appropriate statistical tests.
[6]
[Total 11]
%%%%%%%%%%%%%%%%%%%%%%%%%%%%%%%%%%%%%%%%%%%%%%%%%%%%%%%%%%%%%%%%%%%%%%%%%%%%%%%%%%%%%

8
(i)
(a) Type I censoring is present for those lives still under observation at 31
December 2005 as the censoring times are known in advance.
(b) Interval censoring would be present if we only knew death occurred
between check-ups. However, actual dates of death are known, so
interval censoring is not present.
Right censoring can be seen as a special case of interval censoring (for
those censored before death, we know death occurs in the interval (c i ,
∞ ) where c i is the censoring time for person i).
(c)
Page 10
Informative censoring is not likely to be present. The censoring of
lives gives us no information about future lifetimes.Subject CT4 — Models Core Technical — April 2007 — Examiners’ Report
(ii)
The durations at which lives died or were censored are shown below. Duration
is measured in years and months from the date of surgery.
Patient
A
B
C
D
E
F
G
H
I
J
K
L
M
N
O
Death or censored
death
death
death
death
death
censored
censored
censored
censored
censored
censored
censored
censored
censored
censored
Duration
4 years 4 months
6 months
10 months
1 year 11 months
10 months
4 years 11 months
4 years 10 months
4 years 9 months
4 years 7 months
4 years 4 months
4 years 4 months
4 years 2 months
2 years 6 months
9 months
4 years
The calculation of the survival function is shown in the table below. We
assume that at duration 4 years 4 months, the death occurred before lives were
censored.
t j n j d j c j λ ˆ j = d j / n j
0
0.5
0.833
1.917
4.333 15
15
13
11
7 0
1
2
1
1 0
1
0
3
6 0
1/15
2/13
1/11
1/7
(
)
The estimated survival function is given by, S ˆ ( t ) = ∏ 1 − λ j . So,
t j ≤ t
t Ŝ ( t )
0.000 ≤ t < 0.500
0.500 ≤ t < 0.833
0.833 ≤ t < 1.917
1.917 ≤ t < 4.333
4.333 ≤ t < 5.0 1.0000
0.9333
0.7897
0.7179
0.6154
Solutions using different assumptions (for example assuming the death at 4
years 4 months occurred after lives were censored, or assuming lives M, N
and O were censored sometime within 3 months of their last check-up) were
acceptable and received credit.
Page 11Subject CT4 — Models Core Technical — April 2007 — Examiners’ Report
(iii)
The probability that a patient will die within 4 years of surgery is estimated
by:
1 − S ˆ ( 4 ) = 1 – 0.7179
= 0.2821
9
(i)
The chi-squared test is a suitable overall test.
The test statistic is
∑ z x 2 , where
x
z x =
 x f + 1/ 2
E x c μ ˆ x f + 1/ 2 − E x c μ
 x f + 1/ 2
E x c μ
∑ z x 2
.
has the χ 8 2 distribution.
x
The calculations are shown in the table below
Age
x Actual
deaths
E x c μ ˆ x f + 1/ 2
65
66
67
68
69
70
71
72 30
20
25
40
45
50
50
45
∑ z x 2
Expected
deaths
 x f + 1/ 2
E x c μ
28.4
30.1
31.2
33.5
34.1
41.8
46.5
44.5
z x z x 2
0.3002
- 1.8409
- 1.1100
1.1230
1.8666
1.2683
0.5133
0.0750 0.0901
3.3890
1.2321
1.2612
3.4842
1.6086
0.2634
0.0056
= 11.3343.
x
The critical value of the χ 8 2 distribution at the 5% level of statistical
significance is 15.51.
Since 11.3343 < 15.51, we have no reason to reject the null hypothesis that the
sex ratios of death rates among the company’s pensioners are the same as
those prevailing in the PMA92 and PFA92 tables.
Page 12Subject CT4 — Models Core Technical — April 2007 — Examiners’ Report
(ii)
Standardised deviations test
Using the individual standardised deviations test, we note that none of the z x s
exceeds 1.96 in absolute value, so there is no evidence that the sex ratios
among the company’s pensioners are unusual at any specific ages
Signs test
Under the null hypothesis of no difference between the company’s pensioners
and insured pensioners in general, the number of positive signs should have a
Binomial (8, 0.5) distribution.
There are 2 negative and 6 positive signs.
The probability of obtaining 6 positive signs if the null hypothesis is true is
⎛ 8 ⎞ 8
⎜ ⎟ 0.5 = 0.1094
⎝ 6 ⎠
Since this is greater than 0.025 (two-tailed test), the sex ratios of death rates
among the company’s pensioners are not systematically higher or lower than
those derived from the PMA92 and PFA92 tables.
Cumulative deviations test
The cumulative deviation
∑ ( E x c μ ˆ x f + 1/ 2 − E x c μ  x f + 1/ 2 ) ~ Normal (0, E x c μ  x f + 1/ 2 ) ,
x
so that under the null hypothesis
∑ ( E x c μ ˆ x f + 1/ 2 − E x c μ  x f + 1/ 2 )
x
∑ E x c μ  x f + 1/ 2
~ Normal (0,1).
x
Using the figures in the table above we have
∑ ( E x c μ ˆ x f + 1/ 2 − E x c μ  x f + 1/ 2 )
x
∑ E x c μ  x f + 1/ 2
=
14.9
= 0.875
290
x
and since |0.875| < 1.96 using a two-tailed test, the sex ratios of death rates
among the company’s pensioners are not systematically higher or lower than
those derived from the PMA92 and PFA92 tables.
Page 13Subject CT4 — Models Core Technical — April 2007 — Examiners’ Report
Credit was only given for one of the Signs test and the Cumulative Deviations
test as they both test for bias.
Serial correlations test (lag 1)
The calculations are shown in the tables below
z (1) =
1 7
1 8
(2)
,
and
0.3029
z
=
z
=
∑ x
∑ z x = 0.2707
7 1
7 2
Age x
65
66
67
68
69
70
71
z x − z (1)
z x + 1 − z (2)
- 0.0027
- 2.1439
- 1.4129
0.8201
1.5637
0.9654
0.2103
( z x − z (1) )( z x + 1 − z (2) )
- 2.1117
- 1.3807
0.8523
1.5958
0.9976
0.2425
- 0.1958
0.0057
2.9601
- 1.2042
1.3087
1.5598
0.2341
- 0.0412
Sum
4.8231
2
⎡ z x + 1 − z (2) ⎤
⎣
⎦
Age ⎡ z x − z (1) ⎤
⎣
⎦ 65
66
67
68
69
70
71 0.0000
4.5962
1.9963
0.6726
2.4450
0.9320
0.0442 4.4592
1.9064
0.7264
2.5467
0.9951
0.0588
0.0383
Sum 10.6863 10.7310
2
The correlation coefficient is therefore
r 1 =
4.8231
= 0.4503
(10.6863)(10.7310)
We test r 1 8 = 1.27 against the Normal (0,1) distribution using a one-tailed
test.
Page 14Subject CT4 — Models Core Technical — April 2007 — Examiners’ Report
Since 1.27 < 1.645, we conclude that there is no evidence that the sex ratios of
death rates among the company’s pensioners vary with age in a way different
from the ratios derived from PMA92 and PFA92.
Note that the Grouping of Signs test is not appropriate with 8 ages, 6 positive
and 2 negative signs.
