\documentclass[a4paper,12pt]{article}

%%%%%%%%%%%%%%%%%%%%%%%%%%%%%%%%%%%%%%%%%%%%%%%%%%%%%%%%%%%%%%%%%%%%%%%%%%%%%%%%%%%%%%%%%%%%%%%%%%%%%%%%%%%%%%%%%%%%%%%%%%%%%%%%%%%%%%%%%%%%%%%%%%%%%%%%%%%%%%%%%%%%%%%%%%%%%%%%%%%%%%%%%%%%%%%%%%%%%%%%%%%%%%%%%%%%%%%%%%%%%%%%%%%%%%%%%%%%%%%%%%%%%%%%%%%%

\usepackage{eurosym}
\usepackage{vmargin}
\usepackage{amsmath}
\usepackage{graphics}
\usepackage{epsfig}
\usepackage{enumerate}
\usepackage{multicol}
\usepackage{subfigure}
\usepackage{fancyhdr}
\usepackage{listings}
\usepackage{framed}
\usepackage{graphicx}
\usepackage{amsmath}
\usepackage{chngpage}

%\usepackage{bigints}
\usepackage{vmargin}

% left top textwidth textheight headheight

% headsep footheight footskip

\setmargins{2.0cm}{2.5cm}{16 cm}{22cm}{0.5cm}{0cm}{1cm}{1cm}

\renewcommand{\baselinestretch}{1.3}

\setcounter{MaxMatrixCols}{10}

\begin{document}


[Total 4]
(i) Describe how a classification based on the nature of the state and time spaces
of stochastic processes leads to a four-way categorisation.

(ii) List FOUR stochastic processes, one for each of the four categories in your
answer to part (i).

[Total 4]

%%%%%%%%%%%%%%%%%%%%%%%%%%%%%%%%%%%%%%%%%%%%%%%%%%%%%%%%%%%%%%%%%%%%%%%%%%%%%%%%%%%%%%%%%%%%%%%5
Q4
(i)
The state space may be either discrete or continuous. 
The time set may be either discrete or continuous. 
Hence we have 2  2 = 4 possibilities:
State space Time set
Discrete
Discrete
Continuous
Continuous Discrete
Continuous
Discrete
Continuous
+1

(ii)
State space Time set Examples
Discrete Discrete Simple random walk
Counting process
Markov chain
Markov jump chain
Discrete Continuous Poisson process
Counting process
Markov jump process
Compound Poisson
process
Continuous Discrete White noise
General random walk
Time series
Continuous Continuous Compound Poisson
process
White noise
Brownian motion
Itô process
+2

[Total 4]
This question was well answered, with many candidates scoring full marks.
As indicated in the model solutions above, credit was given in part (ii) for
examples which are not on the CT4 syllabus, but which were correctly
classified (e.g. Itô process).
%% ----- Page 6
