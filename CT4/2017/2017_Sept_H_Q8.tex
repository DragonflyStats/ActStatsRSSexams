\documentclass[a4paper,12pt]{article}

%%%%%%%%%%%%%%%%%%%%%%%%%%%%%%%%%%%%%%%%%%%%%%%%%%%%%%%%%%%%%%%%%%%%%%%%%%%%%%%%%%%%%%%%%%%%%%%%%%%%%%%%%%%%%%%%%%%%%%%%%%%%%%%%%%%%%%%%%%%%%%%%%%%%%%%%%%%%%%%%%%%%%%%%%%%%%%%%%%%%%%%%%%%%%%%%%%%%%%%%%%%%%%%%%%%%%%%%%%%%%%%%%%%%%%%%%%%%%%%%%%%%%%%%%%%%

\usepackage{eurosym}
\usepackage{vmargin}
\usepackage{amsmath}
\usepackage{graphics}
\usepackage{epsfig}
\usepackage{enumerate}
\usepackage{multicol}
\usepackage{subfigure}
\usepackage{fancyhdr}
\usepackage{listings}
\usepackage{framed}
\usepackage{graphicx}
\usepackage{amsmath}
\usepackage{chngpage}

%\usepackage{bigints}
\usepackage{vmargin}

% left top textwidth textheight headheight

% headsep footheight footskip

\setmargins{2.0cm}{2.5cm}{16 cm}{22cm}{0.5cm}{0cm}{1cm}{1cm}

\renewcommand{\baselinestretch}{1.3}

\setcounter{MaxMatrixCols}{10}

\begin{document}

%%- Question 8
A company has for many years offered a car insurance policy with four levels of No
Claims Discount (NCD): 0%, 15%, 30% and 40%. A policyholder who does not
claim in a year moves up one level of discount, or remains at the highest level. A
policyholder who claims one or more times in a year moves down a level of discount
or remains at the lowest level. The company pays a maximum of three claims in any
year on any one policy.
The company has established that:
\item  the arrival of claims follows a Poisson process with a rate of 0.35 per year.
\item  the average cost per claim is £2,500.
\item  the proportion of policyholders at each level of discount is as follows:
(i)
Discount level Proportion of policyholders
0%
15%
30%
40% 4.4%
10.5%
25.1%
60.0%
Calculate the premium paid by a policyholder at the 40% discount level
ignoring expenses and profit.

The company has decided to introduce a protected NCD feature whereby
policyholders can make one claim on their policy in a year and, rather than move
down a level of discount, remain at the level they are at. All other features of the
policy remain the same.
(ii)
Draw the transition graph for this process.

(iii)
Calculate the premium paid, in the long term, by a policyholder at the 40%
discount level of the policy with protected NCD, ignoring expenses and profit.

(iv)

CT4 S2017–7 
Discuss THREE issues with the policy with protected NCD which may each
be either a disadvantage or an advantage to the company.

[Total 15]
%%%%%%%%%%%%%%%%%%%%%%%%%%%%%%%%%%%%%%%%%%%%%%

Q8
(i)
The probability of making n claims in a year is given by
 n exp (  )
n !
where \lambda = 0.35.


Number of claims Probability Cost
0
1
2
3 or more 0.7047
0.2466
0.0432
0.0055 0
616.60
215.81
41.32

Giving an average cost per policy of £873.73

If P is the premium paid by someone at the 0% no claims
discount (NCD) level, then the premiums paying for this cost per policy
are P * (1 – discount) * proportion at that level.
Discount Level Proportion of P Proportion at
level Payment
0
15%
30%
40% 1
0.85
0.7
0.6 0.044
0.105
0.251
0.600 0.0440 P
0.0893 P
0.1757 P
0.3600 P
Total 0.6690 P

%%---
So we have 0.6690 P = £873.73,

giving P = £1,306.12.
So the premium at 40% NCD level is £783.67
Page 14

%%%%%%%% – Examiners’ Report
(ii)
0%
15%
30%
40%

(iii)
The transition matrix for the process is
0  0.2953 0.7047
0
0 


15  0.0487 0.2466 0.7047
0 
.
30  0
0.0487 0.2466 0.7047 


40  0
0
0.0487 0.9513 
%%---
The stationary distribution, \pi , satisfies \pi  = \pi P

\pi_{1} \;=\; 0.2953 \pi_{1} \;+\; 0.0487 \pi_{2} (1)
\pi_{2} \;=\; 0.7047 \pi_{1} \;+\; 0.2466 \pi_{2} \;+\; 0.0487 \pi_{3} (2)
\pi_{3} \;=\; 0.7047 \pi_{2} \;+\; 0.2466 \pi_{3} \;+\; 0.0487 \pi_{4} (3)
\pi_{4} \;=\; 0.7047 \pi_{3} \;+\; 0.9513 \pi_{4} (4)
%%---
Also \pi_{1} \;+\; \pi_{2} \;+\; \pi_{3} \;+\; \pi_{4} \;=\; 1 .
(5)

Working in terms of \pi_{4} :
(4) gives
0.7047 \pi_{3} \;=\; \pi_{4} ( 1  0.9513 )
\pi_{3} \;=\; 0.069068 \pi_{4} ;
(3) gives
0.7047 \pi_{2} \;=\; ( 1  0.2466 ) \pi_{3}  0.0487 \pi_{4}
0.7047 \pi_{2} \;=\; 0.7534 ( 0.069068 ) \pi_{4}  0.0487 \pi_{4}
0.7047 \pi_{2} \;=\; 0.003362 \pi_{4}
\pi_{2} \;=\; 0.00477 \pi_{4} ;
(1) gives \pi_{1} ( 1  0.2953 ) \;=\; 0.0487 ( 0.00477 ) \pi_{4}
\pi_{1} \;=\; 0.000329 \pi_{4} ; and
(5) gives \pi_{4} (0.000329 \;+\; 0.00477 \;+\; 0.069068 \;+\; 1) \;=\; 1

Page 15%%%%%%%% – Examiners’ Report
So \pi_{4} \;=\; 0.930954
\pi_{3} \;=\; 0.064299
\pi_{2} \;=\; 0.004441
\pi_{1} \;=\; 0.000307
%%---
Page 16%%%%%%%% – Examiners’ Report
As before:
Discount Level Proportion of P Proportion at
level Payment
0
15%
30%
40% 1
0.85
0.7
0.6 0.000307
0.004441
0.064299
0.930954 0.00031 P
0.00377 P
0.04501 P
0.55857 P
Total 0.60766 P
%%---
0.60766 P = £873.73, so P = £1,437.85, and those at the 40% NCD level
pay £862.71
(iv)


This may be a common feature in the market. If competitors offer it
and this company does not, it may lose business. %%---
The previous system may have discouraged claims if it meant
that people lost their NCD. Introducing the new system may change the
incidence of claims. Or the average size of claims may change
(smaller ones may have gone unreported previously). %%---
The one-off increase in premium when they introduce the scheme
may prompt otherwise loyal customers to shop around for a better deal. %%---
If the company is the first in the market to launch this option, they
may win lots of new business. %%---
Extra administrative costs may be incurred. 

The protected NCD may appear unfair to policyholders as customers not making a claim can end up with the same discount as those who made a claim. %%---
The new system may embody a moral hazard as it could make customers drive less carefully. 
The new system may induce selection against the office if any new customers are more likely to make claims than those who leave the company and seek a better deal elsewhere.

%%%%%%%%%%%%%%%%%%%%%%%%%%%%%%%%%%%
\newpage

This question proved to be more difficult than anticipated. Few
candidates we able to apply the correct approach to calculating the Page 17%%%%%%%% – Examiners’ Report
premium in part (i). Those who did calculate a premium often used the
incorrect approach of multiplying £2,500 by 0.35 to obtain a cost per
policy of £875. They then simply multiplied this by 0.6 to obtain the premium for someone with a no claims discount of 40%. In part (ii) a
common error was to suppose that no customer could ever move down from 40% to 30%, 30% to 15% or 15% to 0%. Some candidates
attempted to subdivide the 40%, 30% and 15% levels according to whether the customer made no claims or exactly one claim in the
previous year. This produces a seven-state model which is inconsistent
with the scenario in the question. A seven-state model can be constructed by subdividing the 40%, 30% and 15% levels according to
whether the customer made no claims or one or more claims in the
previous year, but it is complex. Credit was given for this if it was correct. In part (iii) it was expected that answers would be consistent
with part (ii). Thus it was expected that candidates who supposed, in part (ii) that no customer could ever move down from 40% to 30%,
30% to 15% or 15% to 0% would write in part (iii) that in the long run
all customers would be at the 40% discount level.

\end{document}
