\documentclass[a4paper,12pt]{article}

%%%%%%%%%%%%%%%%%%%%%%%%%%%%%%%%%%%%%%%%%%%%%%%%%%%%%%%%%%%%%%%%%%%%%%%%%%%%%%%%%%%%%%%%%%%%%%%%%%%%%%%%%%%%%%%%%%%%%%%%%%%%%%%%%%%%%%%%%%%%%%%%%%%%%%%%%%%%%%%%%%%%%%%%%%%%%%%%%%%%%%%%%%%%%%%%%%%%%%%%%%%%%%%%%%%%%%%%%%%%%%%%%%%%%%%%%%%%%%%%%%%%%%%%%%%%

\usepackage{eurosym}
\usepackage{vmargin}
\usepackage{amsmath}
\usepackage{graphics}
\usepackage{epsfig}
\usepackage{enumerate}
\usepackage{multicol}
\usepackage{subfigure}
\usepackage{fancyhdr}
\usepackage{listings}
\usepackage{framed}
\usepackage{graphicx}
\usepackage{amsmath}
\usepackage{chngpage}

%\usepackage{bigints}
\usepackage{vmargin}

% left top textwidth textheight headheight

% headsep footheight footskip

\setmargins{2.0cm}{2.5cm}{16 cm}{22cm}{0.5cm}{0cm}{1cm}{1cm}

\renewcommand{\baselinestretch}{1.3}

\setcounter{MaxMatrixCols}{10}

\begin{document}


4
A study was conducted into the mortality of persons aged between exact ages 85 and
86 years. The study took place from 1 April 2015 to 31 March 2016. The following
table shows information on 10 lives observed in the study.
Life number
1
2
3
4
5
6
7
8
9
10
Date of 85th birthday Date of death
1 August 2014
1 November 2014
1 January 2015
1 February 2015
1 March 2015
1 April 2015
1 June 2015
1 July 2015
1 September 2015
1 January 2016 –
–
1 February 2016
–
–
1 January 2016
1 November 2015
–
1 March 2016
–
(i) Calculate a central exposed to risk for the 10 lives in the sample, working in
months.
(ii) Give the maximum likelihood estimate of the mortality hazard at age 85 last
birthday.
(iii)
 Estimate q 85 .
[Total 5]


%%%%%%%%%%%%%%%%%%%%%%%%%%%%%%%%%%%%%%%%%%%%%%%%%5

Q4
(i)
For the central exposed to risk for each life we need the difference between
STARTDATE and ENDDATE (in months) where:
STARTDATE = latest of 85th birthday and 1 April 2015
ENDDATE = earliest of 86th birthday, date of death, 31 March 2016
(ii)
Life number STARTDATE ENDDATE Contribution to
exposed to risk
(months)
1
2
3
4
5
6
7
8
9
10 1 April 2015
1 April 2015
1 April 2015
1 April 2015
1 April 2015
1 April 2015
1 June 2015
1 July 2015
1 September 2015
1 January 2016 1 August 2015
1 November 2015
1 January 2016
1 February 2016
1 March 2016
1 January 2016
1 November 2015
31 March 2016
1 March 2016
31 March 2016 4
7
9
10
11
9
5
9
6
3
The total exposed to risk is therefore 73 months. %%---

There are 3 deaths at age 85. 
Maximum likelihood estimate is 3/73 = 0.04110 (monthly)
(36/73 = 0.49315 working annually).
(iii)
+2


EITHER (CONSTANT FORCE)
Page 5%%%%%%%% – Examiners’ Report
q 85 = 1 – p 85 = 1 – exp (-12 * 0.04110) 
= 0.3893. 
OR (ACTUARIAL ESTIMATE)
d
3
q 85  c 85
\;=\;
E 85 \;+\; 0.5 d 85 (73 /12) \;+\; 1.5 
= 0.3956. 
OR (EXACT EXPOSURE)
We add to the exposure for the deaths the duration between death and
the time at which the deceased would have attained exact age 86 years.
This is 3 months for life 6, 7 months for life 7 and 6 months for life 9,
a total of 16 months.
q 85 

d 85
3
\;=\;
\;=\; 0.4045.
E \;+\; (16 /12) (73 \;+\; 16) /12

c
85

[Total 5]
Many candidates correctly calculated the exposed to risk in part (i). A
large number of candidates did not realise that one of the deaths took
place after the life’s 86th birthday and so used 4 deaths rather than 3
in part (ii). It was acceptable to work in years rather than months, and
credit could also be obtained for parts (i) and (ii) working in days. In
part (iii), however, q 85 is the probability of death within one year for a
person at exact age 85 years, so it was not acceptable to compute the
probability of death per month.


%%%%%%%%%%%%%%%%%%%%%%%%%%%%%%%%%%%%%%%%%%%%
