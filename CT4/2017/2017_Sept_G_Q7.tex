\documentclass[a4paper,12pt]{article}

%%%%%%%%%%%%%%%%%%%%%%%%%%%%%%%%%%%%%%%%%%%%%%%%%%%%%%%%%%%%%%%%%%%%%%%%%%%%%%%%%%%%%%%%%%%%%%%%%%%%%%%%%%%%%%%%%%%%%%%%%%%%%%%%%%%%%%%%%%%%%%%%%%%%%%%%%%%%%%%%%%%%%%%%%%%%%%%%%%%%%%%%%%%%%%%%%%%%%%%%%%%%%%%%%%%%%%%%%%%%%%%%%%%%%%%%%%%%%%%%%%%%%%%%%%%%

\usepackage{eurosym}
\usepackage{vmargin}
\usepackage{amsmath}
\usepackage{graphics}
\usepackage{epsfig}
\usepackage{enumerate}
\usepackage{multicol}
\usepackage{subfigure}
\usepackage{fancyhdr}
\usepackage{listings}
\usepackage{framed}
\usepackage{graphicx}
\usepackage{amsmath}
\usepackage{chngpage}

%\usepackage{bigints}
\usepackage{vmargin}

% left top textwidth textheight headheight

% headsep footheight footskip

\setmargins{2.0cm}{2.5cm}{16 cm}{22cm}{0.5cm}{0cm}{1cm}{1cm}

\renewcommand{\baselinestretch}{1.3}

\setcounter{MaxMatrixCols}{10}

\begin{document}
%%- Question 7
The following diagram shows the transitions under a Healthy-Sick-Dead multiple
state model under which:
\item  transition rates are dependent on time, t.
\item  transitions out of the Sick state are dependent on the duration, C t , a person has
been in the Sick state as well as on time.
σ(t)
Sick
Healthy
ρ(t, C t )
υ(t, C t )
μ(t)
Dead
(i)
Show from first principles, that if p ij (x,t) is the probability of being in state j at
time t conditional on being in state i at time x, that
∂
p HH ( x , t ) = p HH ( x , t ) ( −σ ( t ) − μ ( t ) ) + p HS ( x , t ) ρ ( t , C t ) 
∂ t
(ii)
Determine the probability that a life is in the Healthy state throughout the
period 0 to t if the life is in the Healthy state at time 0.

(iii) Describe how integrated Kolmogorov equations can be constructed by
conditioning on the first or the last jump, illustrating your answer with a
diagram.
(iv) Explain the difference in approach between deriving forward and backward
integrated Kolmogorov equations.

An actuarial student suggests the following integrated Kolmogorov equation for this
model:
t
Pr[X t = H |X s = S, C s = w] = ∫ e
y
− ∫ ( ρ ( u , w − s + u ) +υ ( u , w − s + u ) ) du
0
0
(v)

CT4 S2017–6
Identify TWO errors in this equation.
υ ( y , w − s + y ) p HH ( y , t ) dy

[Total 13]

%%%%%%%%%%%%%%%%%%%%%

Q7
(i)
Page 10
EITHER
Using the Markov assumption,
OR
The Chapman Kolmogorov equation is
%%%%%%%% – Examiners’ Report
p HH ( x , t \;+\; dt ) \;=\; p HH ( x , t ) p HH ( t , t \;+\; dt )
\;+\; p HS ( x , t ) p SH ( t , t \;+\; dt ) \;+\; p HD ( x , t ) p DH ( t , t \;+\; dt ) 
But p DH ( t , t \;+\; dt ) \;=\; 0 or other explanation why path through D can be
ignored 
So:
p HH ( x , t \;+\; dt ) \;=\; p HH ( x , t ) p HH ( t , t \;+\; dt ) \;+\; p HS ( x , t ) p SH ( t , t \;+\; dt )

Assuming that, for small dt
p ij ( t , t \;+\; dt ) \;=\;  ij ( t ) dt \;+\; o ( dt )
i  j

p ii ( t , t \;+\; dt ) \;=\; 1 \;+\;  ii ( t ) dt \;+\; o ( dt )
OR
p ii ( t , t \;+\; dt ) \;=\; 1    ij ( t ) dt \;+\; o ( dt )

j  i
where the λs are the instantaneous transition rates and
o ( dt )
\;=\; 0 ,
dt  0 dt
lim
then substituting, we have
p HH ( x , t \;+\; dt ) \;=\; p HH ( x , t )(1   ( t ) dt   ( t ) dt ) \;+\; p HS ( x , t )  ( t , C t ) \;+\; o ( dt )
%%---
so that
p HH ( x , t \;+\; dt )  p HH ( x , t ) \;=\; p HH ( x , t )(  ( t )   ( t )) dt
\;+\; p HS ( x , t )  ( t , C t ) dt \;+\; o ( dt )
and hence
p ( x , t \;+\; dt )  p HH ( x , t )
d
p HH ( x , t ) \;=\; lim HH
dt  0
dt
dt
\;=\; p HH ( x , t )(  ( t )   ( t )) \;+\; p HS ( x , t )  ( t , C t )
%%---

(ii)
The equation simplifies when considering p HH ( t ) to
d
p ___ (0, t ) \;=\;  (  ( t ) \;+\;  ( t )) p ___ ( t )
HH
dt HH

Page 11%%%%%%%% – Examiners’ Report
1
d
d
p ___ (0, t ) \;=\;  (  ( t ) \;+\;  ( t )) \;=\; ln p ___ ( t ) .
HH
p ___ (0, t ) dt HH
dt
HH
Integrate both sides:
t
t
  ln p HH (0, t )   \;=\;
0

 (  ( s ) \;+\;  ( s )) ds 
(  ( s ) \;+\;  ( s )) ds ) %%---
s \;=\; 0
as p HH (0) \;=\; 1
t
p HH (0, t ) \;=\; exp  (

s \;=\; 0

(iii)
One method of deriving probabilities for continuous time Markov
processes is by integral equations. 
Using the law of total probability 
we can consider the full set of possibilities for the first
jump from state X or the last jump to state Y. 
For a given time of this first/last jump, the probabilities that the
jump was from each state will be in proportion to transition rates
at that time. %%---
By integrating across all possible times for the first/last jump
we obtain the overall probability. 
Where a probability for being in the same state at start and
end is required, an additional term is needed for the probability
of remaining in the same state throughout the period i.e. no jumps. 
EITHER THE BACKWARD EQUATION
Suppose looking at P XY (s,t) we condition on jumps to all other states Z.
State X
Jumps out of X to Z (say)
Stays in X
Start time
Page 12
State Y
Transition Z to Y in this interval
Time of first jump
End time%%%%%%%% – Examiners’ Report
s
s+w
t

OR THE FORWARD EQUATION
Suppose looking at P XY (s,t) we condition on jumps from all other states Z.
State X
Jumps out of Z (say) to Y
State Y
Transition X to Z in interval
Start time
s
Stays in Y
Time of last jump
t-w
End time
t

[max. 3]
(iv)
(v)
The integrated forward equation is derived by conditioning
on the last jump to state Y and the integrated backward
equation from the first jump from state X.
%%---
For reference, the correct equation is:
y
t
Pr[ X t \;=\; H X s \;=\; S , C s \;=\; w ] \;=\;  e
  (  ( u , w  s \;+\; u ) \;+\; ( u , w  s \;+\; u )) du
s
 ( y , w  s \;+\; y ) P HH ( y , t ) dy
s
1.
2.
3.
The term saying  ( y , w  s \;+\; y ) is wrong: 
it should be  ( y , w  s \;+\; y ) . 
The lower limit on the integral in the exponential term is wrong: 
it should be s. 
The lower limit on the outer integral is wrong: 
it should be s. 
[max. 2]
[Total 13]
This was one of the more difficult questions on the examination paper, and performance was, as expected, variable. A substantial number of
candidates only attempted part (i). In part (ii) several candidates attempted to evaluate the integral by assuming the transition rates were
Page 13%%%%%%%% – Examiners’ Report
constant, which is incorrect as the question stated that the transition rates were dependent on time. Of those candidates who attempted part
(iii), many made a good effort at the diagrams. Few candidates offered correct solutions to part (iv). It should be noted that the correct
equation noted here does not match that shown in the Core Reading, Unit 4, page 17. The lower limit on the outer integral should be s
rather than 0. Candidates could score credit for spotting the error in the Core Reading, but even without this, full credit could be scored, as
there were two further errors to be identified.
