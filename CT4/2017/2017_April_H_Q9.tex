\documentclass[a4paper,12pt]{article}

%%%%%%%%%%%%%%%%%%%%%%%%%%%%%%%%%%%%%%%%%%%%%%%%%%%%%%%%%%%%%%%%%%%%%%%%%%%%%%%%%%%%%%%%%%%%%%%%%%%%%%%%%%%%%%%%%%%%%%%%%%%%%%%%%%%%%%%%%%%%%%%%%%%%%%%%%%%%%%%%%%%%%%%%%%%%%%%%%%%%%%%%%%%%%%%%%%%%%%%%%%%%%%%%%%%%%%%%%%%%%%%%%%%%%%%%%%%%%%%%%%%%%%%%%%%%

\usepackage{eurosym}
\usepackage{vmargin}
\usepackage{amsmath}
\usepackage{graphics}
\usepackage{epsfig}
\usepackage{enumerate}
\usepackage{multicol}
\usepackage{subfigure}
\usepackage{fancyhdr}
\usepackage{listings}
\usepackage{framed}
\usepackage{graphicx}
\usepackage{amsmath}
\usepackage{chngpage}

%\usepackage{bigints}
\usepackage{vmargin}

% left top textwidth textheight headheight

% headsep footheight footskip

\setmargins{2.0cm}{2.5cm}{16 cm}{22cm}{0.5cm}{0cm}{1cm}{1cm}

\renewcommand{\baselinestretch}{1.3}

\setcounter{MaxMatrixCols}{10}

\begin{document}

The government statistical service in a country with a population of 10 million has estimated mortality rates among males in that country aged 20 to 99 years inclusive.
It wishes to create a new standard mortality table.

\begin{enumerate}[(a)]
\item Describe why the crude mortality rates should be graduated during the
production of this standard mortality table.
\item 
(ii) Describe a suitable method of graduation for these mortality rates.
\item 
(iii) Explain the limitations of the method described in your answer to part (ii) in
this situation.
\end{itemize}
The government performs the graduation and compares the crude and graduated rates.
Below are some of the results of the comparison:
Value of individual standardised
deviation at age x, z x
z x   3
 2  z x   3
Number of ages
0
7
 1  z x   2
0  z x   1
1  z x  0
2  z x  1
3  z x  2 16
z x  3 3
26
16
10
2
(iv) Assess the quality of the graduated rates for use as a new standard mortality
table by applying TWO statistical tests to the above information. The two
tests should each examine a different aspect of the graduation.
[6]
(v) Comment on the implications of your results in part (iv) for the government
using the new standard mortality table for economic and financial planning
purposes.



%%%%%%%%%%%%%%%%%%%%%%%%%%%%%%%%%%%%%%%%%%%%%%%%%%%%%%%%%%%%%%%%%%%%%%%%%%%
Q10
(i)
The crude rates are estimated independently at each age. 
Therefore they are subject to sampling error. 
This is a fairly small country, so there may be scanty data at certain ages. 
This means that the crude rates may exhibit “roughness” in the
progression from age to age. 
We believe that underlying mortality actually progresses smoothly
from age to age. 
Graduation “irons out” the roughness in the crude rates which is due to sampling error while preserving the underlying level and shape of the
mortality curve. 
It uses information from adjacent ages to adjust the rate at any particular age. 
We believe that the graduated rates are close to the true mortality
rates underlying the crude rates. 
The government of the country may wish to use the new table for financial calculations and economic planning (e.g. for a state
pension scheme), so it is important that sampling errors are removed.
(ii)

[Max 3]
Graduation using a parametric formula. 
As the results are being used to create a standard table. 
As the data reflect adult ages, a formula from the Gompertz or Makeham family would seem appropriate. 
The parameters can be estimated by maximum likelihood or least squares, using weights which are proportional to the amount of data at each age.

[Max 1]
(iii)
Might not find a suitable formula with a small number of parameters
to fit the data at all ages.
At extreme ages data might be scanty and so less weight should be given to those ages when estimating the parameters of the formula.
(iv)
+1
+1

Under the null hypothesis that the graduated rates are equal to the mortality underlying the crude rates, 
the individual standardised deviations should be distributed N(0,1) 
%% ----- Page 21
There are 80 deviations. 
The number of positive deviations P is distributed Binomial (80, 0.5). 
We have a large number of deviations so
we can use the Normal approximation 
P ~ Normal(40, 20)
We have 31 positive deviations.

EITHER WITHOUT CONTINUITY CORRECTION
We compute the z-score which is
31  40
\;=\;  2.01 .
20

Since –2.01 < 1.96 
we have sufficient evidence to reject the null hypothesis at the
5% level (two-tailed test). 
OR WITH CONTINUITY CORRECTION
We compute the z-score which is
31.5  40
\;=\;  1.9007 .
20
Since –1.9007 > 1.96


we have insufficient evidence to reject the null hypothesis at the 5% level
(two-tailed test).

EITHER
We expect 1 in 20 of the standardised deviations
to be greater than 2 in absolute magnitude. Here we have
12 out of 80 which is 3 in 20, which is more than we should expect.
+1
OR
We also expect fewer than 1 in 100 of the standardised
deviations to be greater than 3 in absolute magnitude, and here
we have 3 out of 80, which is again a larger proportion than we
should expect.
%% ----- Page 22
+1
We can compute a chi-squared test. The calculations are shown in the table
below.
Range Actual number
of z x s Expected number
of z x s (A – E) 2 /E
z x   1 23 12.8 8.2
0  z x   1
1  z x  0 26 27.2 0.1
16 27.2 4.6
z x  1 15 12.8 0.4
+1
The chi-squared statistic is 13.3. 
There are 3 degrees of freedom (the number of groups minus 1) 
because the total number of ages is fixed, so once the z x s have been
allocated to three groups, the number to go in the final group is
determined. 
The critical value at the 5% level is 7.82. 
Since 13.3 > 7.82 we reject the null hypothesis.
(v)

[Max 6]
The tests seem to indicate that the graduated rates are statistically
different from the “true” mortality rates underlying the crude rates. 
The government would be unwise to use these graduated rates for
financial calculations or economic planning. 
They will tend to overestimate mortality and therefore underestimate
the amount of money required to fund future pensions. 
This could lead to fiscal problems for future governments.


[Total 14]
Answers to this question were often relatively weak, especially part (iv). The
data were provided to allow the chi-squared test described on Unit 11, %% ----- Page
10 of the Core Reading, but only a small minority of candidates attempted
such a test. Most were content with a non-rigorous check for outliers.
Several candidates, however, made sensible comments in part (v). Credit
was given in part (v) for any thoughtful comments which were consistent with
candidates’ answer to part (iv). So, if a candidate used the continuity
correction in part (iv) and did not reject the null hypothesis, credit was given in
part (v) for stating that the graduated rates were not biased up or down.
%% ----- Page 23
\end{document}
