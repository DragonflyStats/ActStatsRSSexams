\documentclass[a4paper,12pt]{article}

%%%%%%%%%%%%%%%%%%%%%%%%%%%%%%%%%%%%%%%%%%%%%%%%%%%%%%%%%%%%%%%%%%%%%%%%%%%%%%%%%%%%%%%%%%%%%%%%%%%%%%%%%%%%%%%%%%%%%%%%%%%%%%%%%%%%%%%%%%%%%%%%%%%%%%%%%%%%%%%%%%%%%%%%%%%%%%%%%%%%%%%%%%%%%%%%%%%%%%%%%%%%%%%%%%%%%%%%%%%%%%%%%%%%%%%%%%%%%%%%%%%%%%%%%%%%

\usepackage{eurosym}
\usepackage{vmargin}
\usepackage{amsmath}
\usepackage{graphics}
\usepackage{epsfig}
\usepackage{enumerate}
\usepackage{multicol}
\usepackage{subfigure}
\usepackage{fancyhdr}
\usepackage{listings}
\usepackage{framed}
\usepackage{graphicx}
\usepackage{amsmath}
\usepackage{chngpage}

%\usepackage{bigints}
\usepackage{vmargin}

% left top textwidth textheight headheight

% headsep footheight footskip

\setmargins{2.0cm}{2.5cm}{16 cm}{22cm}{0.5cm}{0cm}{1cm}{1cm}

\renewcommand{\baselinestretch}{1.3}

\setcounter{MaxMatrixCols}{10}

\begin{document}


%%%%%%%%%%%%%%%%%%%%%%%%%%%%%%%%%%%%%%%%%%%
1
A Markov Chain has the following transition graph:
A
B
C
The following is a partially completed transition matrix for this Markov Chain:
⎛ 0.2
⎜ –
⎝ –
–
–
–
–
1.0
0.4
⎛
⎜
⎝
A
B
C
(i) Determine the remaining entries in the transition matrix.
(ii) Explain whether each of the following is a valid sample path for this process.
(a)

Path 1:
State
A
B
C
0
CT4 S2017–2
1
2
3
4
5
6
7
8
9
10
Time(b)
Path 2:
State
A
B
C
0
1
2
3
4
5
6
7
8
9 10 Time


[Total 4]

%%%%%%%%%%%%%%%%%%%%%%%%%%%%%%%%%%%%%%%%%%%%%%%%%%%%%%%%%%%%%%%%%%%%%%%%%5


Solutions
Q1
(i)
Rows must sum to 1.
P AC = 0 and P CB = 0 from transition graph.
So full transition matrix is:
A  0.2 0.8 0 


B  0
0 1.0  .
C   0.6 0 0.4  
+2

(ii)
Path 1 is a valid sample path.

All the movements between states are valid transitions from the transition
graph.

Path 2 is not a valid sample path.

EITHER
There are transitions from C to B, which is not possible according to the
transition graph.
OR
The process cannot stay in B.


[Total 4]
The vast majority of candidates correctly computed the transition
matrix in part (i). In part (ii), some candidates interpreted the
diagrams in the question to depict a continuous time process and
therefore argued neither sample path was valid, as a Markov chain is a
discrete time process. Full credit was given for this.
Page 3%%%%%%%% – Examiners’ Report
