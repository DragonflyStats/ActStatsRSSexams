\documentclass[a4paper,12pt]{article}

%%%%%%%%%%%%%%%%%%%%%%%%%%%%%%%%%%%%%%%%%%%%%%%%%%%%%%%%%%%%%%%%%%%%%%%%%%%%%%%%%%%%%%%%%%%%%%%%%%%%%%%%%%%%%%%%%%%%%%%%%%%%%%%%%%%%%%%%%%%%%%%%%%%%%%%%%%%%%%%%%%%%%%%%%%%%%%%%%%%%%%%%%%%%%%%%%%%%%%%%%%%%%%%%%%%%%%%%%%%%%%%%%%%%%%%%%%%%%%%%%%%%%%%%%%%%

\usepackage{eurosym}
\usepackage{vmargin}
\usepackage{amsmath}
\usepackage{graphics}
\usepackage{epsfig}
\usepackage{enumerate}
\usepackage{multicol}
\usepackage{subfigure}
\usepackage{fancyhdr}
\usepackage{listings}
\usepackage{framed}
\usepackage{graphicx}
\usepackage{amsmath}
\usepackage{chngpage}

%\usepackage{bigints}
\usepackage{vmargin}

% left top textwidth textheight headheight

% headsep footheight footskip

\setmargins{2.0cm}{2.5cm}{16 cm}{22cm}{0.5cm}{0cm}{1cm}{1cm}

\renewcommand{\baselinestretch}{1.3}

\setcounter{MaxMatrixCols}{10}

\begin{document}

\begin{enumerate}[(a)]
\item Define a Markov Chain.
\item Describe the difference between a time-homogeneous and a time-inhomogeneous Markov Chain, giving an example of each.
\end{enumerate}

%%%%%%%%%%%%%%%%%%%%%%%%%%%%%%%%%%%%%%%%%%%%%%%%%%%%%%%%%%%%%%%
\newpage

Q3
(i)
A Markov Chain is a process operating in discrete time with a discrete state
space
+1
EITHER
It obeys the Markov property: 
that the future state of the process can be predicted from its present
state alone, without any reference to its past history 
OR
P[X t \in A | X s_1 = x 1 , X s 2 = x 2 , ..., X s n = x n , X s = x] = P[X t \in A | X s = x]

for all times $s_1 < s_2 < ... < sn < s < t$, all states $x_1 , x_2 , \ldots, x_n , x in S$ and
all subsets A of S.
%%%%%%%%%%%%%%%%%%%%%%%%%%%%%%%%%%%%%%%%%%%%%%%%%%%%%%
(ii)


In a time-homogeneous Markov chain the transition probabilities are time-
independent.

In a time-inhomogeneous Markov chain the transition probabilities depend on
the absolute values of time, not just the time difference.

Examples: time-homogeneous – no claims discount system in which the
probability of a claim in each year is constant.

time-inhomogeneous – no claims discount system in which accident
probabilities reflect changing traffic conditions from one year to the next. 

%%%%%%%%%%%%%%%%%%%%%%%%%%%%%%%%%%%%%%%%%%%%%%%%%%%%%%
% [Total 4]
% Most candidates could define a Markov Chain. There was less sure- footedness about describing the difference between a time-% homogeneous and a time-inhomogeneous Markov chain. Many candidates referred to “transition rates” when they meant one-step transition probabilities. Others made vague statements about the “Chain depending [or not] on time”, for which only limited credit was given. In part (ii) credit was given only for examples which could sensibly be analysed using a Markov Chain.
%% ----- Page 5
\end{document}
