\documentclass[a4paper,12pt]{article}

%%%%%%%%%%%%%%%%%%%%%%%%%%%%%%%%%%%%%%%%%%%%%%%%%%%%%%%%%%%%%%%%%%%%%%%%%%%%%%%%%%%%%%%%%%%%%%%%%%%%%%%%%%%%%%%%%%%%%%%%%%%%%%%%%%%%%%%%%%%%%%%%%%%%%%%%%%%%%%%%%%%%%%%%%%%%%%%%%%%%%%%%%%%%%%%%%%%%%%%%%%%%%%%%%%%%%%%%%%%%%%%%%%%%%%%%%%%%%%%%%%%%%%%%%%%%

\usepackage{eurosym}
\usepackage{vmargin}
\usepackage{amsmath}
\usepackage{graphics}
\usepackage{epsfig}
\usepackage{enumerate}
\usepackage{multicol}
\usepackage{subfigure}
\usepackage{fancyhdr}
\usepackage{listings}
\usepackage{framed}
\usepackage{graphicx}
\usepackage{amsmath}
\usepackage{chngpage}

%\usepackage{bigints}
\usepackage{vmargin}

% left top textwidth textheight headheight

% headsep footheight footskip

\setmargins{2.0cm}{2.5cm}{16 cm}{22cm}{0.5cm}{0cm}{1cm}{1cm}

\renewcommand{\baselinestretch}{1.3}

\setcounter{MaxMatrixCols}{10}

\begin{document}


%% CT4 A2017–3
%% PLEASE TURN OVER6
(i)
Define a Poisson process.

An insurance company observed that 200 claims arrived in the 52 weeks of the
calendar year 2014. The company has for some years modelled the number of claims
per week, D, using a time-homogeneous Poisson process model with parameter λ.
The company proposes to use a value  \;=\; 3.846 for 2014.
(ii)
Explain why this is a sensible value of λ to use.

The company’s records for 2014 show that the number of claims per week was
distributed as follows:
Claims
per week Number of weeks
0
1
2
3
4
5
6
7
8
9 or more 1
5
8
10
12
4
6
4
1
1
(iii) Test the fit of the Poisson distribution with parameter  \;=\; 3.846 to the data.
[7]
(iv) State why the company might perform a serial correlations test on the
experience.

%%% [Total 11]
\newpage
%%%%%%%%%%%%%%%%%%%%%%%%%%%%%%%%%%%%%%%%%%%%%%%%%%%%%%%%%%%%%%%%%%%%%%%%%%%%%5
Q6
(i)
A Poisson process is a counting process in continuous time { N t , t  0} ,
where N t records the number of occurrences of a type of event within
the time interval from 0 to t. +1
Events occur singly and may occur at any time; 
the probability that an event occurs during the short time interval from time t
to time t + h is approximately equal to λh for small h, where the parameter λ is
the rate of the Poisson process.

OR
A Poisson process is an integer valued process in continuous time { N t , t  0} ,
where

N 0 \;=\; 0

Pr[ N t + h  N t \;=\; 1| F t ] \;=\;  h + o ( h )
Pr[ N t + h  N t \;=\; 0 | F t ] \;=\; 1   h + o ( h )
Pr[ N t + h  N t  0,1| F t ] \;=\; o ( h )
+1
%% ----- Page 9
OR
A Poisson process with rate  is a continuous-time integer-valued process
N t , t  0 , with the following properties:

N 0 \;=\; 0 
N t has independent increments 
N t has Poisson distributed stationary increments
n  t  s
   ( t  s )   e ( )
,
P  N t  N s \;=\; n  \;=\;
n !
s  t , n \;=\; 0,1,...


(ii)
Because it is total claims divided by number of weeks (to the nearest whole
week):
3.846 \;=\;
200
,
52
and this can be shown by formal procedures (e.g. maximum likelihood)
to be a good estimate.
(iii)
The null hypothesis is that the number of claims per week follows a
Poisson distribution with parameter 3.846.
Using the Poisson formula, the probability of getting exactly d claims
in a week is given by
e  (200/52) (200 / 52) d
Pr[D = d] =
d !
%% ----- Page 10




This produces an expected distribution of claims per week as follows:
Claims
Probability
0
1
2
3
4
5
6
7
8
9 or more
Expected number of weeks
0.02136
0.08216
0.15800
0.20257
0.19478
0.14983
0.09604
0.05278
0.02537
0.01711
1.11
4.27
8.22
10.53
10.13
7.79
4.99
2.74
1.32
0.89
+2
Combining the categories 0 and 1, and the categories 69 so that
each of the expected values exceeds 5, we can test the fit using a
chi-squared test as follows:
Claims
Actual number
of weeks, A
01
2
3
4
5
69
6
8
10
12
4
12
Expected number
of weeks, E (A – E) 2 /E
5.38
8.22
10.53
10.13
7.79
9.94 0.071
0.006
0.027
0.345
1.844
0.427

+1
The chi-squared statistic is 2.72. 
The number of degrees of freedom is the number of categories less one
because of the constraint imposed by the number of weeks in the year. 
So there are 5 degrees of freedom.. 
The critical value at the 5% level is 11.07. 
Since 2.72 < 11.07 
we do not reject the null hypothesis, and conclude that the
Poisson distribution with parameter 3.846 fits the data well. 
%% ----- Page 11
ACCEPTABLE ALTERNATIVE GROUPING OF CLAIMS
Claims
Actual number
of weeks, A
01
2
3
4
5
6
79
6
8
10
12
4
6
6
Expected number
of weeks, E (A – E) 2 /E
5.38
8.22
10.53
10.13
7.79
4.99
4.95 0.071
0.006
0.027
0.345
1.844
0.203
0.223
+1
The chi-squared statistic is 2.72. 
The number of degrees of freedom is the number of categories less one because of the constraint imposed by the number of weeks in the year. 
So there are 6 degrees of freedom. 
The critical value at the 5\% level is 12.59. 
Since 2.72 < 12.59 
we do not reject the null hypothesis, and conclude that the
Poisson distribution with parameter 3.846 fits the data well.
(iv)
In order to test for independence of the number of claims in
successive weeks.
Parts (i) and (ii) of this question were well answered. In part (ii) many candidates correctly calculated the expected number of weeks having 0, 1, 2, ... claims using a Poisson distribution with parameter 3.846. Fewer knew how to test the hypothesis that this Poisson distribution fit the data. Of those who did a chi-squared test, only a minority grouped the claims categories so
that the expected number of claims in each category was 5 or more. A
significant minority of candidates calculated a table based on the actual and expected number of claims occurring in weeks with 0, 1, 2, ... claims, and tried to do a chi-squared test on this. This test cannot include the weeks in
which no claims take place (when both actual and expected numbers are zero whatever the value of the parameter), so limited credit was given for this approach. A common error was to state that the number of degrees of freedom was the number of groups minus 1 because of the estimation of the
Poisson parameter. This is incorrect; it is the number of groups minus one
because of the constraint that there are 52 weeks in the year. In part (iv) only
%% ----- Page 12

[7]
+1

% [Total 11]
% a small number of candidates correctly identified the reason for performing a
% serial correlations test. Many candidates made vague reference to “clumps”
% or “runs”, suggesting that they were thinking in a general way about statistical
% tests of a graduation, rather than trying to understand the scenario in the
% question.
\end{document}
