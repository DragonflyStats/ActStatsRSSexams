

\documentclass[a4paper,12pt]{article}

%%%%%%%%%%%%%%%%%%%%%%%%%%%%%%%%%%%%%%%%%%%%%%%%%%%%%%%%%%%%%%%%%%%%%%%%%%%%%%%%%%%%%%%%%%%%%%%%%%%%%%%%%%%%%%%%%%%%%%%%%%%%%%%%%%%%%%%%%%%%%%%%%%%%%%%%%%%%%%%%%%%%%%%%%%%%%%%%%%%%%%%%%%%%%%%%%%%%%%%%%%%%%%%%%%%%%%%%%%%%%%%%%%%%%%%%%%%%%%%%%%%%%%%%%%%%

\usepackage{eurosym}
\usepackage{vmargin}
\usepackage{amsmath}
\usepackage{graphics}
\usepackage{epsfig}
\usepackage{enumerate}
\usepackage{multicol}
\usepackage{subfigure}
\usepackage{fancyhdr}
\usepackage{listings}
\usepackage{framed}
\usepackage{graphicx}
\usepackage{amsmath}
\usepackage{chngpage}

%\usepackage{bigints}
\usepackage{vmargin}

% left top textwidth textheight headheight

% headsep footheight footskip

\setmargins{2.0cm}{2.5cm}{16 cm}{22cm}{0.5cm}{0cm}{1cm}{1cm}

\renewcommand{\baselinestretch}{1.3}

\setcounter{MaxMatrixCols}{10}

\begin{document}
2
For each of the following processes:
\item  General Random Walk
\item  Markov Jump Process
\item  Compound Poisson Process
\item  Markov Chain
(a)
State whether the state space is discrete, continuous or can be either.
(b)
State whether the time set is discrete, continuous, or can be either.



%%%%%%%%%%%%%%%%%%%%%%%%%%%%%%

Q2
General random walk
Markov jump process
Compound Poisson process
Markov chain.
State space Time domain
Can be either
Discrete
Can be either
Discrete Discrete
Continuous
Continuous
Discrete
%%---
%%---
%%---
%%---

% This question was generally well answered. 
% The most common errors were in the description of the state space and time domains of the general random walk, and in failing to recognise that the compound Poisson process can occupy either a continuous or a discrete state
% space.

\end{document}
