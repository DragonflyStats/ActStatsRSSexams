\documentclass[a4paper,12pt]{article}

%%%%%%%%%%%%%%%%%%%%%%%%%%%%%%%%%%%%%%%%%%%%%%%%%%%%%%%%%%%%%%%%%%%%%%%%%%%%%%%%%%%%%%%%%%%%%%%%%%%%%%%%%%%%%%%%%%%%%%%%%%%%%%%%%%%%%%%%%%%%%%%%%%%%%%%%%%%%%%%%%%%%%%%%%%%%%%%%%%%%%%%%%%%%%%%%%%%%%%%%%%%%%%%%%%%%%%%%%%%%%%%%%%%%%%%%%%%%%%%%%%%%%%%%%%%%

\usepackage{eurosym}
\usepackage{vmargin}
\usepackage{amsmath}
\usepackage{graphics}
\usepackage{epsfig}
\usepackage{enumerate}
\usepackage{multicol}
\usepackage{subfigure}
\usepackage{fancyhdr}
\usepackage{listings}
\usepackage{framed}
\usepackage{graphicx}
\usepackage{amsmath}
\usepackage{chngpage}

%\usepackage{bigints}
\usepackage{vmargin}

% left top textwidth textheight headheight

% headsep footheight footskip

\setmargins{2.0cm}{2.5cm}{16 cm}{22cm}{0.5cm}{0cm}{1cm}{1cm}

\renewcommand{\baselinestretch}{1.3}

\setcounter{MaxMatrixCols}{10}

\begin{document}


[Total 11]
PLEASE TURN OVER8
A careful shopkeeper takes delivery of a batch of 20 packets of cheese. Every
morning at 8 a.m. precisely she checks to see if any of the cheese has gone mouldy
and throws away any mouldy packets.
As she runs a high quality establishment, she has lots of customers and some of the
cheese is sold. After ten days she decides the cheese will be too old to sell and throws
out the remaining packets.
A curious customer observes that the shopkeeper has created an observational plan for
calculating the hazard of cheese going mouldy.
(i) State, with reasons, THREE types of censoring present in this situation.

(ii) Assess, for EACH type of censoring listed in your answer to part (i), whether a
change to the observational plan could be made which would remove that type
of censoring.

The shopkeeper made notes at 8 a.m. each day as follows:
Day Shopkeeper’s notes
1
2
3
4
5
6
7
8
9
10 Sold three packets already
Sold one more packet
One went mouldy
Two more mouldy ones, I hope my fridge is cold enough
Seems OK, nothing to report
Sold four more – all to one customer!
Nothing to report
Another two mouldy ones this morning
Sold two more
Three more mouldy ones – I’ll throw the rest out
(iii) Calculate the Kaplan-Meier estimate of the survival function for cheese
staying free from mould.
[6]
%%%%%%%%%%%%%%%%%%%%%%%%%%%%%%%%%%%%%%%%%%%%%%%%%%%%%%%%%%%%%%%%%%%%%%%%%%%%%%%%%%%%%%
Q8
(i)
Right censoring 
Of those packets of cheese which are sold or discarded after 10 days.
We do not know when these packets would have gone mouldy, but
we know that it was after they ceased to be observed. 
Random censoring 
The time at which a packet of cheese is sold may be considered as
a random variable. 
Type 1 censoring 
For those packets thrown out at day 10, the time of censoring is known in
advance.

Interval censoring 
Because data were only collected once per day: we only know that
events happened between 8 a.m. one day and 8 a.m. the next day;
we do not know exactly when within this period they happened. 
%% ----- Page 15
EITHER
Informative censoring 
If packets that were sold were those which looked the freshest then
those packets censored for this reason might be less likely to go
mouldy at each duration in excess of the censoring time than those
which remain on the shelves. 
OR
Non-informative censoring
If shoppers select packets of cheese at random there is no reason to
believe that those packets bought are any more or less likely to go
mouldy than those remaining on the shelves.
(ii)


[Max 3]
Right censoring
Hard to change this, as would need to prevent customers buying cheese.
+1
Random censoring
Hard to change as would need to prevent customers buying cheese.
+1
Type 1 censoring
Could be removed by leaving all the cheese in the shelves until the
last packet went mouldy or was sold.
+1
OR
Could be removed by throwing the remaining cheese away once a certain
number had already gone mouldy, in which case we would have
introduced Type II censoring.
+1
Interval censoring
Could be reduced by more frequent checks. Hard to remove,
other than by continuous monitoring of the cheese and the
removal of cheeses at the first sign of mould. Video surveillance
might be a solution.
+1
Informative censoring
Could be removed by giving customers no choice about which
packet of cheese they bought. However, customers
might object to being treated this way.
%% ----- Page 16
+1
Non-informative censoring
This feature is desirable so we would not want to remove it.
(iii)
t j n j
0
2
2
3
4
6
9
10 20
17
16
15
13
9
7
5

+1
d j
c j
+1
[Max 3]
d j /n j 1  d j /n j
1/16
2/15 15/16
13/15
2/9 7/9
3
1
1
2
4
2
3 2
2 3/5 2/5
   
+31⁄2
 d j
The Kaplan-Meier estimate is S(t) =   1 
t j  t 
 n j

 . 


t Kaplan-Meier estimate of S(t)
0 \leq t < 3
3 \leq t < 4
4 \leq t < 8
8 \leq t < 10
t = 10 1
15/16=0.9375
13/16=0.8125
91/144=0.6319
91/360=0.2528
+1 +1
+2
[6]
[Total 12]
The most common error with this question was to assume the wrong
decrement (i.e. that the decrement was cheese being sold, whereas the
question stated that interest is in “the hazard of cheese going mouldy”).
Candidates who assumed the wrong decrement scored little for part (i) but
could gain credit for part (ii) if their suggestions were sensible given what they
had written in part (i), and for part (iii) if they applied the correct method.
%% ----- Page 17
