\documentclass[a4paper,12pt]{article}

%%%%%%%%%%%%%%%%%%%%%%%%%%%%%%%%%%%%%%%%%%%%%%%%%%%%%%%%%%%%%%%%%%%%%%%%%%%%%%%%%%%%%%%%%%%%%%%%%%%%%%%%%%%%%%%%%%%%%%%%%%%%%%%%%%%%%%%%%%%%%%%%%%%%%%%%%%%%%%%%%%%%%%%%%%%%%%%%%%%%%%%%%%%%%%%%%%%%%%%%%%%%%%%%%%%%%%%%%%%%%%%%%%%%%%%%%%%%%%%%%%%%%%%%%%%%

\usepackage{eurosym}
\usepackage{vmargin}
\usepackage{amsmath}
\usepackage{graphics}
\usepackage{epsfig}
\usepackage{enumerate}
\usepackage{multicol}
\usepackage{subfigure}
\usepackage{fancyhdr}
\usepackage{listings}
\usepackage{framed}
\usepackage{graphicx}
\usepackage{amsmath}
\usepackage{chngpage}

%\usepackage{bigints}
\usepackage{vmargin}

% left top textwidth textheight headheight

% headsep footheight footskip

\setmargins{2.0cm}{2.5cm}{16 cm}{22cm}{0.5cm}{0cm}{1cm}{1cm}

\renewcommand{\baselinestretch}{1.3}

\setcounter{MaxMatrixCols}{10}

\begin{document}


%%- Question 6

A pharmaceutical company is undertaking trials on a new drug which, it claims, cures
a particularly uncomfortable but not life threatening condition. It has conducted
extensive testing of the drug on a large group of people suffering from the condition
and has noticed that the drug is much more effective in some groups of patients than
others. It has fitted a Cox regression for the hazard of symptoms disappearing h(t)
with three parameters
h ( t ) = h 0 ( t ) exp ( S \beta S + A \beta A + G \beta G )
where \beta S , \beta A , and \beta G are parameters and
\item  S represents the sex of the patient and takes a value of 1 if the patient is female, 0
if male.
\item  A represents the age, in years minus 20, of the patient when the drug was
administered.
\item  G takes the value 1 if the patient attended a gym, 0 otherwise.
The company has discovered the following, where the age given is the age when the
drug was administered:
\item  a 25 year old female who attended a gym had a hazard of symptoms disappearing
equal to twice that of a male of the same age who did not attend a gym;
\item  a 45 year old male who did not attend a gym had a hazard of symptoms
disappearing half that of a 43 year old male who attended a gym; and
\item  a 32 year old female who attended a gym had a hazard of symptoms disappearing
60% greater than that of a 45 year old female who did not attend a gym.
(i) Calculate the values of the parameters \beta S , \beta A , and \beta G .
(ii) Determine for which group of people the drug is most effective.

The probability that a woman who attended a gym and was aged 38 years when
she was given the drug still had symptoms of the condition after 28 days was found to
be 0.75.
(iii)

CT4 S2017–5 
Calculate the probability of still having symptoms after 28 days for a male
aged 26 years when given the drug who did not attend a gym.

[Total 12]
%%%%%%%%%%%%%%%%%%%%%%%%%%%%%%%

Q6
(i)
The statements in the question give rise to the following equations:
(1)
(2)
(3)
exp (  S \;+\; 5  A \;+\;  G ) \;=\; 2exp ( 5  A )
exp ( 25  A ) \;=\; 0.5exp ( 23  A \;+\;  G )
exp (  S \;+\; 12  A \;+\;  G ) \;=\; 1.6exp (  S \;+\; 25  A )
+2
(2) gives us
ln 0.5 \;=\; 2  A   G
(3) gives us
(4)
ln1.6 \;=\;  G  13  A
Combining these gives
 11  A \;=\; ln 0.5 \;+\; ln1.6
So  A \;=\; 0.02029 %%---
Substituting in (4) gives  G \;=\; 0.73372 %%---
(1) gives us
 S \;=\; ln 2   G
So  S \;=\;  0.04057
Page 8
%%---
%%%%%%%% – Examiners’ Report
(ii)
Here h(t) is the hazard of symptoms disappearing so we wish to find
the group with the maximum value of the hazard, or the minimum
value of
 t

S t \;=\; exp    h 0 ( t ) exp ( S  S \;+\; A  A \;+\; G  G )  ,


 0

the probability of still suffering from the symptoms. %%---
\beta S is negative, so we want S = 0 i.e. male. 
\beta A is positive, so we want A to be as large as possible i.e. the person to
be as old as possible when the drug is administered. %%---
\beta G is positive so we want G = 1 i.e. someone who attends a gym.
(iii)


EITHER
 28

S givenfemale ( 28 ) \;=\; exp    h 0 ( t ) exp (  S \;+\; 18  A \;+\;  G ) dt  \;=\; 0.75
  0
 
28
  h 0 ( t ) \;=\;
0
ln ( 0.75 )
exp (  S \;+\; 18  A \;+\;  G )
.
%%---
%%---
The probability for the required male is
 28

S required ( 28 ) \;=\; exp    h 0 ( t ) exp ( 6  A ) dt  .
  0
 

So
ln ( 0.75 )
 
 
S required ( 28 ) \;=\; exp 
exp ( 6  A )  .
  exp (  S \;+\; 18  A \;+\;  G )
 

Inserting the calculated values for \beta S , \beta A and \beta G gives
S required ( 28 ) \;=\; exp (  0.11276 ) \;=\; 0.89337 ,
which is an 89.3% probability of still having the symptoms for the
Page 9%%%%%%%% – Examiners’ Report
male aged 26 years when given the drug who did not attend as gym.
%%---
OR
 28

S givenfemale ( 28 ) \;=\; exp    h 0 ( t ) exp (  S \;+\; 18  A \;+\;  G ) dt  \;=\; 0.75
  0
  %%---
The hazard for the given female is h 0 ( t ) exp (  S \;+\; 18  A \;+\;  G ) . %%---
The hazard for the required male is h 0 ( t ) exp ( 6  A ) . 
The ratio of the hazards is therefore
e 6  A
e 18  A \;+\; S \;+\; G
\;=\; 0.39195 .

Therefore
S required ( 28 ) \;=\; 0.75 0.39195 \;=\; 0.89337 ,
which is an 89.3% probability of still having the symptoms for the
male aged 26 years when given the drug who did not attend as gym
%%---

[Total 12]
Many candidates were able to formulate the equations in part (i),
though a wide range of numerical errors was made when solving them.
In parts (ii) and (iii), credit was given for answers that were correct
given the solutions offered in part (i) for the values of \beta A , \beta G , and \beta S .
Few candidates spotted that the age related to when the drug was
administered. Some candidates interpreted the question as asking for a
comparison of the effectiveness of the drug among the groups of people
mentioned in the bullet points in the question. Credit was given for this
if the hazards for the six groups were correctly calculated and the
correct group identified (43 year old males who attended the gym).
Part (iii) was less well answered than the other parts. Candidates who
worked only with hazards, rather than survival probabilities, received
little credit.
%%%%%%%%%%%%%%%%%%%%%%%%%%%%%%%%%%%%%%%%%%%%%%%5
