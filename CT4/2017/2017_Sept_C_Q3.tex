\documentclass[a4paper,12pt]{article}

%%%%%%%%%%%%%%%%%%%%%%%%%%%%%%%%%%%%%%%%%%%%%%%%%%%%%%%%%%%%%%%%%%%%%%%%%%%%%%%%%%%%%%%%%%%%%%%%%%%%%%%%%%%%%%%%%%%%%%%%%%%%%%%%%%%%%%%%%%%%%%%%%%%%%%%%%%%%%%%%%%%%%%%%%%%%%%%%%%%%%%%%%%%%%%%%%%%%%%%%%%%%%%%%%%%%%%%%%%%%%%%%%%%%%%%%%%%%%%%%%%%%%%%%%%%%

\usepackage{eurosym}
\usepackage{vmargin}
\usepackage{amsmath}
\usepackage{graphics}
\usepackage{epsfig}
\usepackage{enumerate}
\usepackage{multicol}
\usepackage{subfigure}
\usepackage{fancyhdr}
\usepackage{listings}
\usepackage{framed}
\usepackage{graphicx}
\usepackage{amsmath}
\usepackage{chngpage}

%\usepackage{bigints}
\usepackage{vmargin}

% left top textwidth textheight headheight

% headsep footheight footskip

\setmargins{2.0cm}{2.5cm}{16 cm}{22cm}{0.5cm}{0cm}{1cm}{1cm}

\renewcommand{\baselinestretch}{1.3}

\setcounter{MaxMatrixCols}{10}

\begin{document}

CT4 S2017–3 
%%%%%%%%%%%%%%%%%%%%%%%%%%%%%%%3
Calls arrive on Fred’s desk phone according to a Poisson Process with parameter 3,
with time measured in hours.
(i)
Write down the expected number of phone calls Fred receives each hour.

Fred has not received a phone call for 15 minutes.
(ii)
Give the expected time until Fred next receives a phone call.

Fred goes into a meeting for half an hour.
(iii)
Determine the probability that Fred has NOT missed a call when he returns to
his desk.

The average length of a call to Fred is 7 minutes.
(iv)
Determine the probability that if a caller phones Fred the line will be engaged,
assuming that Fred is at his desk to receive calls.

[Total 5]
%%%%%%%%%%%%%%%%%%%%%%%%%%%%%%%%%%%%%%%%%%%%%%%%%%%%%%%%%%%5


Q3
(i) The mean is equal to the parameter, so there are 3 calls per hour.
(ii) The process is memoryless so the fact that Fred has not had a call for
15 minutes is irrelevant.
%%---

Expected time until next call is 20 minutes.
(iii)
%%---

This is the probability of zero calls in time 0.5 hours.
Using p j ( t ) \;=\; e  t (  t ) j / j !
OR
Since p 0 (0.5) \;=\;
e  1.5 (1.5) 0
,
0!

p 0 (0.5) \;=\; e  1.5 \;=\; 0.2231 .
(iv)
The expected time that Fred is on the phone is the expected number of calls times the expected length of a call.
Per hour this is 3 calls times 7 minutes = 21 minutes.
So the probability that the phone is engaged is 21/60 = 0.35.
Page 4


%%---
%%---

[Total 5]%%%%%%%% – Examiners’ Report
Many answers to this question were poor. Basic errors were made. For example, in part (ii) many candidates argued that the mean waiting
time was 20 minutes, but that as Fred had not received a call for 15 minutes he only had 5 minutes more to wait. This contradicts the
memoryless property of the exponential distribution. In part (iv) a common incorrect alternative was to argue that, since Fred would (on
average) be on the telephone for 3 x 7 = 21 minutes per hour, the probability that Fred would be engaged is equal to the probability that
at least one call would be received in 21 minutes [exp(-21/60)] = 0.2953. This is not correct because it ignores the fact that if there is
more than one call in the 21 minutes, more than one caller will find Fred engaged, and we require the probability to be calculated from the
perspective of the callers, not from Fred’s perspective.
