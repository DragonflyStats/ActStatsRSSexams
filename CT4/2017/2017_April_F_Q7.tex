\documentclass[a4paper,12pt]{article}

%%%%%%%%%%%%%%%%%%%%%%%%%%%%%%%%%%%%%%%%%%%%%%%%%%%%%%%%%%%%%%%%%%%%%%%%%%%%%%%%%%%%%%%%%%%%%%%%%%%%%%%%%%%%%%%%%%%%%%%%%%%%%%%%%%%%%%%%%%%%%%%%%%%%%%%%%%%%%%%%%%%%%%%%%%%%%%%%%%%%%%%%%%%%%%%%%%%%%%%%%%%%%%%%%%%%%%%%%%%%%%%%%%%%%%%%%%%%%%%%%%%%%%%%%%%%

\usepackage{eurosym}
\usepackage{vmargin}
\usepackage{amsmath}
\usepackage{graphics}
\usepackage{epsfig}
\usepackage{enumerate}
\usepackage{multicol}
\usepackage{subfigure}
\usepackage{fancyhdr}
\usepackage{listings}
\usepackage{framed}
\usepackage{graphicx}
\usepackage{amsmath}
\usepackage{chngpage}

%\usepackage{bigints}
\usepackage{vmargin}

% left top textwidth textheight headheight

% headsep footheight footskip

\setmargins{2.0cm}{2.5cm}{16 cm}{22cm}{0.5cm}{0cm}{1cm}{1cm}

\renewcommand{\baselinestretch}{1.3}

\setcounter{MaxMatrixCols}{10}

\begin{document}


%%-- CT4 A2017–47
(i)
Describe the essential feature of a proportional hazards model.

A study was made of the impact of drinking beer on men aged 60 years and over. A sample of men was followed from their 60th birthdays until they died, or left the study for other reasons. The baseline hazard of death, $\mu$, was assumed to be constant, and a proportional hazards model was estimated with a single covariate: the average daily
beer intake in standard-sized glasses consumed, x. The equation of the model is:
\[h ( t ) \;=\; \mu_ exp(  x )\]
where h ( t ) is the hazard of death at age 60 + t.
The estimated value of μ is 0.03, and the estimated value of β is 0.2.
(ii) Explain how $\mu$ and $\beta$ should be interpreted, in the context of this model.

(iii) Calculate the estimated hazard of death of a man aged exactly 62 years who
drinks two glasses of beer a day.

A man is aged exactly 60 years and drinks three glasses of beer a day.
(iv)
(a) Calculate the estimated probability that this man will still be alive in 10 years’ time.
(b) Calculate the expectation of life at age 60 years for this man.

Another man is aged exactly 60 years. He drinks beer only in his local bar. He drinks all the beer he buys and is expected to continue drinking the same amount of beer every day until he dies. The owner of the bar is interested in selling as much beer as
possible.
(v)
%% CT4 A2017–5
Determine the average number of glasses of beer a day the owner must sell the
man in order to maximise the total amount of beer the man buys over his
remaining lifetime.


%%%%%%%%%%%%%%%%%%%%%%%%%%%%%%%%%%%%%%%%%%%%%%%%%%%%%%%%%%%%%%%%%%%%%%%%%%%%%%%%
\newpage

Q7
(i)
EITHER
In a proportional hazards model the hazard factorises  into a component which depends only on duration and a
component which depends only on the covariates. 
OR
In a proportional hazards model the hazard, h(t), may be represented as
\[h(t) = h 0 (t)g(z),\] 
where $h_0 (t)$ depends only on duration and g(z) depends only on the covariates 
As a consequence, the effect of a covariate is to shift the hazard up or down by a the same proportion at all durations OR the ratio between the hazards for two individuals, A and B, with different
covariate vectors, $h_A (t)/h_B (t)$, is constant/does not depend on duration $t$.
%%%%%%%%%%%%%%%%%%%%%%%%%%%%%%%%%%%%%%%%%%%%%%%%%%%%%%%%
(ii)
μ is the hazard for a man who does not drink beer.
+1
β measures the impact on the hazard of a one glass increase in the daily
amount of beer drunk
%%%%%%%%%%%%%%%%%%%%%%%%%%%%%
(iii)
+1

The hazard for a man who drinks two glasses of beer a day is
\[0.03exp(0.2 * 2) \;=\; 0.0448.\]
(iv)
+1

(a)
+1

The probability that a man aged 60 years who drinks three glasses of
beer a day will survive to his 70th birthday is
 10

exp    0.03exp(0.2 *3) dt  \;=\; exp(  0.547) \;=\; 0.579 .


 0

+1
%% ----- Page 13
(b)
Because the hazard of death is constant, the expectation of life at age
60 years is given by
1
1
1
\;=\;
\;=\;
\;=\; 18.3 years.
h ( t ) \mu_ exp(  x ) 0.03exp(0.2*3)
+1

(v)
The owner’s total revenue R is proportional to the average number of glasses
of beer drunk per day multiplied by the man's expectation of life:
R 
x
.
0.03exp(0.2 x )
+1
THEN EITHER
We maximise R with respect to x.
dR 0.03exp(0.2 x )  x (0.03*0.2)exp(0.2 x )
1  0.2 x
\;=\;
\;=\;
.
dx
0.03exp(0.2 x )
[0.03exp(0.2 x )] 2
This is zero when 1 – 0.2x = 0, or when x = 5.
d 2 R  0.2[0.03exp(0.2 x )]  (1  0.2 x )[0.03*0.2exp(0.2 x )]
\;=\;
dx 2
[0.03exp(0.2 x )] 2



\;=\;
 0.2  0.2(1  0.2 x )
[0.03exp(0.2 x )]
\;=\;
0.2(0.2 x  2)
[0.03exp(0.2 x )]
which is negative when x = 5, 
so we have a maximum, 
and the owner should sell the man five glasses of beer per day. 
OR
Taking the logarithm of R we have
log R \;=\; log x  log 0.03  0.2 x
d log R 1
\;=\;  0.2 .
dx
x
This is zero when x = 5.
%% ----- Page 14


Since
d 2 log R
1
\;=\; 2
2
dx
x

which is always negative, 
we have a maximum. 
So the owner should sell the man five glasses of beer per day.


% [Total 11]
% The better-prepared candidates answered part (i) well. Part (ii) was answered
% poorly by many candidates. The question asked for the interpretation of the
% parameters μ and β “in the context of this model”, but most candidates simply
% gave general interpretation which were not given credit. Few candidates
% made serious attempts at part (v). Most seemed to have little idea of how to
% express the total amount of beer the man buys over his remaining lifetime as
% a function of the number of glasses of beer to be sold to him per day. Credit
% was given for an alternative approach in which candidates computed the
% expected revenue for 0, 1, 2, ... glasses per day and showed that this
% reached a maximum at 5 glasses.
\end{document}
