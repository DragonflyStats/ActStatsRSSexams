\documentclass[a4paper,12pt]{article}

%%%%%%%%%%%%%%%%%%%%%%%%%%%%%%%%%%%%%%%%%%%%%%%%%%%%%%%%%%%%%%%%%%%%%%%%%%%%%%%%%%%%%%%%%%%%%%%%%%%%%%%%%%%%%%%%%%%%%%%%%%%%%%%%%%%%%%%%%%%%%%%%%%%%%%%%%%%%%%%%%%%%%%%%%%%%%%%%%%%%%%%%%%%%%%%%%%%%%%%%%%%%%%%%%%%%%%%%%%%%%%%%%%%%%%%%%%%%%%%%%%%%%%%%%%%%

\usepackage{eurosym}
\usepackage{vmargin}
\usepackage{amsmath}
\usepackage{graphics}
\usepackage{epsfig}
\usepackage{enumerate}
\usepackage{multicol}
\usepackage{subfigure}
\usepackage{fancyhdr}
\usepackage{listings}
\usepackage{framed}
\usepackage{graphicx}
\usepackage{amsmath}
\usepackage{chngpage}

%\usepackage{bigints}
\usepackage{vmargin}

% left top textwidth textheight headheight

% headsep footheight footskip

\setmargins{2.0cm}{2.5cm}{16 cm}{22cm}{0.5cm}{0cm}{1cm}{1cm}

\renewcommand{\baselinestretch}{1.3}

\setcounter{MaxMatrixCols}{10}

\begin{document}


[Total 14]
CT4 A2017–811
A large company operates a health benefits scheme which pays a sickness benefit to
any employee who is unable to work through ill-health and a death benefit to any
employee who dies.
\begin{enumerate}
\item (i) Draw a transition diagram with three states which could be used to analyse
data from this scheme.
\item 
(ii) Give the likelihood of the data, defining all the terms you use. 
\item (iii) Derive the maximum likelihood estimator of the rate of falling sick. 
\end{enumerate}
%-------------------------%
The company records data on 1 January each year, classified by age last birthday on:


the total number of employees (including those in receipt of sickness benefit).
the number of employees in receipt of a sickness benefit.
Some recent data are given in the table below:
Age last
birthday
51
52
53
Total number of employees
on 1 January in year
Number of employees in receipt of
sickness benefit on 1 January in year
2014 2015 2016 2014 2015 2016
148
146
140 162
148
144 180
160
146 12
10
8 20
18
20 8
7
6
The company wishes to estimate the rates of falling sick and recovery at age 52 years
nearest birthday over the two-year period consisting of the calendar years 2014 and
2015.
(iv)
Determine suitable exposed-to-risks for calculating the rates of falling sick and
recovery.

%%%%%%%%%%%%%%%%%%%%%%%%%%%%%%%%%%%%%%%%%%%%%%%%%%%%%%%%%%%%%%%%%%%%%%%%%%%%%%%%%%%%5
\newpage
Q11
(i)
Sick (in receipt of
sick pay)
Healthy
Dead
+2

(ii)
Let the number of transitions observed from state i to state j be d ij . 
Let the total waiting time in state i be W i . 
Let the intensity of the transition from state i to state j be μ ij . 
Then the likelihood, L, is
HS
L  exp[  W H ( \mu_ HS + \mu_ HD )]exp[  W S ( \mu_ SH + \mu_ SD )]( \mu_ HS ) d ( \mu_ HD ) d
HD
SH
( \mu_ SH ) d ( \mu_ SD ) d
SD
+21⁄2

%%%%%%%%%%%%%%%%%%%%%%%%%%%%%%%%%%%%%%%%%%%%555
(iii)
Taking logarithms of the likelihood we have
log e L \;=\;  W H ( \mu_ HS ) + d HS log e \mu_ HS + terms not involving \mu_ HS

HS
To maximize this with respect to \mu_ we proceed as follows:
d log e L
d \mu_ HS
\;=\;  W
H

d HS
\mu_ HS
.
Setting this to zero produces the estimate
\mu_
^
HS
\;=\;
d HS



W H
and since
d 2 log e L
( d \mu_ HS ) 2
\;=\;
d HS
( \mu_ HS ) 2
which is negative, we have a maximum.
%% ----- Page 24



%%%%%%%%%%%%%%%%%%%%%%%%%%%%%%%%%%%%%%%%%%%%555
(iv)
Let the number of healthy members aged x last birthday on 1 January in
year t be H x,t , and the number of sick members be S x,t .
We need to adjust the exposed-to-risk to age nearest birthday. 
Assuming birthdays are evenly distributed across the calendar year, 
those aged 52 nearest birthday consist of half of those aged 51 last
birthday and half of those aged 52 last birthday. 
Further, assuming the population varies linearly over each calendar year
we can apply the trapezium rule. 
c
The required central exposed to risk for sick members, E 52, S is given by
c
E 52,
S \;=\;
1  1
1
1
1

S 51,2014 + S 52,2014 ) + ( S 51,2015 + S 52,2015 ) + ( S 51,2015 + S 52,2015 ) + ( S 51,2016 + S 52,2016 ) 
(

2  2
2
2
2

+1
which is
c
E 52,
S \;=\;
1  1
1
1
1
( 12 + 10 ) + ( 20 + 18 ) + ( 20 + 18 ) + ( 8 + 7 )   \;=\; 28.25 .

2  2
2
2
2

%%%%%%%%%%%%%%%%%%%%%%%%%%%%%%%%%%%%%%%%%%%%%%%%%%%%%%%%%%%%%%%%%%%%%%%%%%%%%%%%%%%%%%%%%%%%%%%%%%%%%%%%
Similarly, we have
c
E 52,
H \;=\;
1  1
1
1
1

H 51,2014 + H 52,2014 ) + ( H 51,2015 + H 52,2015 ) + ( H 51,2015 + H 52,2015 ) + ( H 51,2016 + H 52,2016 ) 
(

2  2
2
2
2

+1
and, since H x,t = total members – S x,t , we have
c
E 52,
H \;=\;
+1
1  1
1
1
1
( 136 + 136 ) + ( 142 + 130 ) + ( 142 + 130 ) + ( 172 + 153 )   \;=\; 285.25 

2  2
2
2
2


%%%%%%%%%%%%%%%%%%%%%%%%%%%%%%%%%%%%%%%%%%%%%%%%%%%%%%%%%%%%%%%%%%%%%%%%%%%%%%%%%%%%%%%
\newpage

This question was well answered by most candidates. A very common error in part (iii) was to forget that the exposed to risk for falling sick should not include those persons who are already sick. The total exposed to risk was 313.5, of whom 285.25 were Healthy (and hence at risk of falling sick) and 28.25 were Sick (and hence at risk of recovery).

%% ----- Page 25
\end{document}
