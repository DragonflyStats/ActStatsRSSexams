\documentclass[a4paper,12pt]{article}

%%%%%%%%%%%%%%%%%%%%%%%%%%%%%%%%%%%%%%%%%%%%%%%%%%%%%%%%%%%%%%%%%%%%%%%%%%%%%%%%%%%%%%%%%%%%%%%%%%%%%%%%%%%%%%%%%%%%%%%%%%%%%%%%%%%%%%%%%%%%%%%%%%%%%%%%%%%%%%%%%%%%%%%%%%%%%%%%%%%%%%%%%%%%%%%%%%%%%%%%%%%%%%%%%%%%%%%%%%%%%%%%%%%%%%%%%%%%%%%%%%%%%%%%%%%%

\usepackage{eurosym}
\usepackage{vmargin}
\usepackage{amsmath}
\usepackage{graphics}
\usepackage{epsfig}
\usepackage{enumerate}
\usepackage{multicol}
\usepackage{subfigure}
\usepackage{fancyhdr}
\usepackage{listings}
\usepackage{framed}
\usepackage{graphicx}
\usepackage{amsmath}
\usepackage{chngpage}

%\usepackage{bigints}
\usepackage{vmargin}

% left top textwidth textheight headheight

% headsep footheight footskip

\setmargins{2.0cm}{2.5cm}{16 cm}{22cm}{0.5cm}{0cm}{1cm}{1cm}

\renewcommand{\baselinestretch}{1.3}

\setcounter{MaxMatrixCols}{10}

\begin{document}


%%-- CT4 A2017–25
A city operates a bicycle rental scheme. Bicycles are stored in racks at locations
around the city and may be rented for a fee and ridden from one location and
deposited at another, provided there is space in the rack. The rack outside the
actuarial profession’s headquarters in that city has spaces for four bicycles.
The profession would like the city to increase the size of the rack. The city has said it
will do so if the profession can demonstrate that, in the long run, the rack is full or
empty for more than 35 per cent of the working day. The profession commissions a
study to monitor the rack every 10 minutes during the working day.
The study shows that, on average:
 there is a probability of 0.3 that the number, m, of bicycles in the rack will
increase by 1 over a 10-minute interval (where 0 \leq m < 4).
 there is a probability of 0.2 that the number of bicycles in the rack will decrease
by 1 over a 10-minute interval (where 0 < m \leq 4).
 the probability of more than one increase or decrease per 10-minute interval can
be regarded as 0.
\begin{enumerate}
\item (i) Give the transition matrix for the number of bicycles in the rack. 
\item (ii) Determine whether the city will increase the size of the rack. [6]
\item (iii) Comment on whether an increase in the size of the rack will reduce the
proportion of time for which the rack is empty or full.
\end{enumerate}
%%-- [Total 10]
\newpage
%%%%%%%%%%%%%%%%%%%%%%%%%%%%%%%%%%%%%%%%%%%%%%%%%%%%%%%%%%%%%%%%%%%%%%%%%%%%%%%%%%%%%%%%%%%%%%%%%%%%%%%%%%%%%
Q5
(i)
0  0.7 0.3 0
0
0 


1  0.2 0.5 0.3 0
0 
2  0 0.2 0.5 0.3 0  .


3  0
0 0.2 0.5 0.3 
4   0
0
0 0.2 0.8  
+2

(ii)
For the long run proportion of time that the rack is either full or empty,
we need to find the stationary distribution.
OR
Let the long-run probability of there being i bicycles in the rack be \pi_ i .

Then if P is the transition matrix we need to solve
( \pi_ 0
\pi_ 1 \pi_ 2 \pi_ 3 \pi_ 4 ) \;=\; ( \pi_ 0 \pi_ 1 \pi_ 2 \pi_ 3 \pi_ 4 ) P .

The equations are
\pi_ 0 \;=\; 0.7 \pi_ 0 + 0.2 \pi_ 1
\pi_ 1 \;=\; 0.3 \pi_ 0 + 0.5 \pi_ 1 + 0.2 \pi_ 2
\pi_ 2 \;=\; 0.3 \pi_ 1 + 0.5 \pi_ 2 + 0.2 \pi_ 3
+11⁄2
\pi_ 3 \;=\; 0.3 \pi_ 2 + 0.5 \pi_ 3 + 0.2 \pi_ 4
\pi_ 4 \;=\; 0.3 \pi_ 3 + 0.8 \pi_ 4
Starting with the equation for π 0 we have
0.3 \pi_ 0 \;=\; 0.2 \pi_ 1
\pi_ 1 \;=\; 1.5 \pi_ 0
Then in the equation for \pi_ 1 we have
1.5 \pi_ 0 \;=\; 0.3 \pi_ 0 + 0.5(1.5 \pi_ 0 ) + 0.2 \pi_ 2
0.45 \pi_ 0 \;=\; 0.2 \pi_ 2
\pi_ 2 \;=\; 2.25 \pi_ 0
%% ----- Page 7
and in the equation for \pi_ 2 we have
2.25 \pi_ 0 \;=\; 0.3(1.5 \pi_ 0 ) + 0.5(2.25 \pi_ 0 ) + 0.2 \pi_ 3
0.675 \pi_ 0 \;=\; 0.2 \pi_ 3
\pi_ 3 \;=\; 3.375 \pi_ 0
and, finally, in the equation for \pi_ 4 we have
\pi_ 4 \;=\; 0.3(3.375 \pi_ 0 ) + 0.8 \pi_ 4
0.2 \pi_ 4 \;=\; 1.0125 \pi_ 0
\pi_ 4 \;=\; 5.0625 \pi_ 0
+11⁄2
Since
\pi_ 0 + \pi_ 1 + \pi_ 2 + \pi_ 3 + \pi_ 4 \;=\; 1 ,

we have
EITHER
\pi_ 0 + 1.5 \pi_ 0 + 2.25 \pi_ 0 + 3.375 \pi_ 0 + 5.0625 \pi_ 0 \;=\; 13.1875 \pi_ 0 \;=\; 1 ,
OR
 16 24 36 54 81 
\pi_ 0  + + + +  \;=\; 1
 24 24 24 24 24 
and hence
\pi_ 0 \;=\;
16
81
\;=\; 0.0758 \pi_ 4 \;=\;
\;=\; 0.3839
211
211
Thus the rack is full or empty 45.97 per cent (97/211) of the time,
which is more than 35 per cent,
and the city will increase the size of the rack.
(iii)
The increase in the size of the rack is likely to reduce the proportion
of time for which the rack is empty or full, as the fact that there are
more states will “dilute” the probabilities.



[6]
+1
This assumes that behaviour of the users of the scheme remains the same. 
%% ----- Page 8
However, if the behaviour/characteristics of those using the bicycle
scheme change in response to the increased size of the rack
(for example new people join the scheme)
then the transition matrix may change so that the proportion of time
the rack is empty or full may shift.


[Max 2]
[Total 10]
Many candidates scored highly on parts (i) and (ii) of this question.
Candidates who produced an incorrect matrix in part (i) could potentially score
full credit in part (ii) if they followed through correctly. In part (iii) a substantial
number of candidates spotted that, with more states, the dilution of the
probability should mean that the proportion of time the rack is empty or full will
decrease. Some, however, thought that that chance of it being full would be
less, but that the chance of it being empty would remain the same. Only a
few candidates went further to comment on the possible effect on the
increase in the size of the rack on the behaviour of those using the bicycle
scheme.
