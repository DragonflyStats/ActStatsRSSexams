\documentclass[a4paper,12pt]{article}

%%%%%%%%%%%%%%%%%%%%%%%%%%%%%%%%%%%%%%%%%%%%%%%%%%%%%%%%%%%%%%%%%%%%%%%%%%%%%%%%%%%%%%%%%%%%%%%%%%%%%%%%%%%%%%%%%%%%%%%%%%%%%%%%%%%%%%%%%%%%%%%%%%%%%%%%%%%%%%%%%%%%%%%%%%%%%%%%%%%%%%%%%%%%%%%%%%%%%%%%%%%%%%%%%%%%%%%%%%%%%%%%%%%%%%%%%%%%%%%%%%%%%%%%%%%%

\usepackage{eurosym}
\usepackage{vmargin}
\usepackage{amsmath}
\usepackage{graphics}
\usepackage{epsfig}
\usepackage{enumerate}
\usepackage{multicol}
\usepackage{subfigure}
\usepackage{fancyhdr}
\usepackage{listings}
\usepackage{framed}
\usepackage{graphicx}
\usepackage{amsmath}
\usepackage{chngpage}

%\usepackage{bigints}
\usepackage{vmargin}

% left top textwidth textheight headheight

% headsep footheight footskip

\setmargins{2.0cm}{2.5cm}{16 cm}{22cm}{0.5cm}{0cm}{1cm}{1cm}

\renewcommand{\baselinestretch}{1.3}

\setcounter{MaxMatrixCols}{10}

\begin{document}

%%%%%%%%%%%%%%%%%%%%%%%%%%%%%%%%%%%%%%%%%%%%%%%%%%%%%%
1
In a certain country, the force of mortality, \mu_ x , in the age range 90-105 years exact is
given by:
Age range (years) \mu_ x
90 \leq x < 95
95 \leq x < 100
100 \leq x < 105 0.10
0.15
0.20
The head of state sends a congratulatory card on a citizen’s 100th birthday and again on reaching age 105.
Derive the probability that a person aged exactly 93 WILL receive a congratulatory card for reaching age 100 but NOT receive a second congratulatory card for reaching age 105.

2
(i)
Define an increment of a process.

The rate of mortality in a certain population at ages over exact age 30 years, h (30 + u ) ,
is described by the process:
\[h (30 + u ) \;=\; B (1 + \gamma ) u u \geq 0\]
where B and $\gamma$ are constants.
3
4
(ii) Show that the increments of the process log[ h (30 + u )] are stationary.



%%%%%%%%%%%%%%%%%%%%%%%%%%%%%%%%%%%Q1
We require 5 q 100 . 7 p 93 \;=\; (1  5 p 100 ). 7 p 93
7
7

7
 2

p 93 \;=\; exp    0.10 dt   0.15 dt 
2
 0
 
p 93 \;=\; exp(  0.95) \;=\; 0.386741 
 5

exp
p
\;=\;
   0.20 dt 
5 100
 0

5 p 100
\;=\; exp(  1.00) \;=\; 0.367879
So 5 q 100 . 7 p 93 \;=\; 0.632121*0.386741 \;=\; 0.244467



%%-- [Total 3]
%%-- Full credit could be obtained for the correct numerical answer and some indication of the method used. A common error was to use an incorrect age range when evaluating survival probabilities. Most candidates, however, scored well on this question.
%% ----- Page 3
\newpage
Q2
(i) An increment of a process is the amount by which its value increases over a
period of time, for example $X(u + t) – X(u)$ where $t > 0$.
+1

(ii) EITHER
\[log[ h (30 + u + t )]  log[ h (30 + u )] \;=\; log[ B (1 + \gamma ) u + t ]  log[ B (1 + \gamma ) u ] +1
\;=\; log B + ( u + t )log(1 + \gamma )  (log B + u log(1 + \gamma )). 
\;=\; t log(1 + \gamma ) \]
The increment thus depends on t, the duration of the process but not
on u, hence the process is stationary. +1
OR
log h (30 + u ) \;=\; log B + u log(1 + \gamma ) +1
d
log h (30 + u ) \;=\; log(1 + \gamma )
du +1
which is constant and does not depend on u, so the process is stationary.
+1

[Total 4]
Most candidates (though by no means all) could define an increment. Many candidates did not make a serious attempt at part (ii), which dealt with a topic that had not been examined for several sessions. The better-prepared candidates saw that the increment for each time unit was constant and hence the process was stationary. A few candidates failed to spot this and tried to
demonstrate weak stationarity. Credit was given for this, although such candidates often encountered difficulty in calculating the covariance of the process.
%% ----- Page 4

%%%%%%%%%%%%%%%%%%%%%%%%%%%%%%%%%%%%%%%%%%%%%%%%%%%%%%%%%%%%%%%%%%%%%%%%%%%%%%%%
\end{document}
