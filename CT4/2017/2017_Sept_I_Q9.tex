\documentclass[a4paper,12pt]{article}

%%%%%%%%%%%%%%%%%%%%%%%%%%%%%%%%%%%%%%%%%%%%%%%%%%%%%%%%%%%%%%%%%%%%%%%%%%%%%%%%%%%%%%%%%%%%%%%%%%%%%%%%%%%%%%%%%%%%%%%%%%%%%%%%%%%%%%%%%%%%%%%%%%%%%%%%%%%%%%%%%%%%%%%%%%%%%%%%%%%%%%%%%%%%%%%%%%%%%%%%%%%%%%%%%%%%%%%%%%%%%%%%%%%%%%%%%%%%%%%%%%%%%%%%%%%%

\usepackage{eurosym}
\usepackage{vmargin}
\usepackage{amsmath}
\usepackage{graphics}
\usepackage{epsfig}
\usepackage{enumerate}
\usepackage{multicol}
\usepackage{subfigure}
\usepackage{fancyhdr}
\usepackage{listings}
\usepackage{framed}
\usepackage{graphicx}
\usepackage{amsmath}
\usepackage{chngpage}

%\usepackage{bigints}
\usepackage{vmargin}

% left top textwidth textheight headheight

% headsep footheight footskip

\setmargins{2.0cm}{2.5cm}{16 cm}{22cm}{0.5cm}{0cm}{1cm}{1cm}

\renewcommand{\baselinestretch}{1.3}

\setcounter{MaxMatrixCols}{10}

\begin{document}

%%- Question 9 
%%%%%%%%%%%%%%%%%%%%%%%%%%%%%%%9
(i)
(a) List TWO different methods of graduating crude mortality data.
(b) State, for each method, TWO advantages and ONE disadvantage.

A large pension scheme is examining its most recent experience and has graduated its
data over a range of ages using μ x = 0.0005 + 0.00005(1.1 x ). The table below gives
some of the data.
Age Exposed to Risk Observed Deaths Graduated Rates
60
61
62
63
64
65
66
67
68
69 7,966
7,728
7,870
7,622
7,097
7,208
6,833
6,474
6,208
5,914 127
139
162
167
205
179
185
212
209
195 0.015724
0.017246
0.018921
0.020763
0.022790
0.025019
0.027470
0.030167
0.033134
0.036398
(ii) Perform an overall goodness of fit test on the data.
(iii) (a)

State THREE possible defects of the graduation which the test you
performed in (ii) would fail to detect.
(b)
Suggest, for each defect in part (a), an alternative test which would
detect each defect.

(iv)

CT4 S2017–8
Carry out TWO of the tests you mentioned in part (iii), clearly stating your
conclusions in relation to the relevant defects.

[Total 17]

%%%%%%%%%%%%%%%%%%%%%%%%%%%%%%%%%%%%%%%%%5

Q9
(i)
(a)
(b)
Graduation by parametric formula. 
Graduation by reference to a standard table. 
Graphical graduation. 
{1]
Parametric formula
Advantages:
The resultant graduation will be sufficiently smooth provided
few parameters are used.
It is a suitable method to produce standard tables.
It can be useful to fit the same formula to several
experiences to give insight into the differences between
experiences.
Disadvantages
It may be difficult to find one equation which fits at all ages.




Reference to a standard table
Advantages
It can be used to fit relatively small data sets where a
suitable standard table exists.
The graduated rates should be smooth provided that a simple
function is used.
The standard table can provide information at extreme ages
where data may be scanty.
Page 18


%%%%%%%% – Examiners’ Report
Disadvantages
It may be difficult to find a standard table which
correctly reflects the population under investigation.

Graphical graduation
Advantages
It can be used for scanty data sets.
It enables an experienced analyst to allow for known (or likely)
features of the data.
It can give a quick initial feel for the rates.
Disadvantages
The expertise may not be available.
The resultant figures may not be to sufficient
decimal places for e.g. premium calculations
It is difficult to ensure smoothness of the resultant rates.
(ii)






{2}
[max. 3]
To test for the overall goodness of fit use the χ 2 test.
The null hypothesis is that the graduated rates are the same as the true
underlying rates in the block of business.
The test statistic

 z x 2   m 2 where m is the degrees of freedom.
x
Age Exposed to
Risk Observed
Deaths Graduated
Rates Expected
Deaths
60
61
62
63
64
65
66
67
68
69 7,966
7,728
7,870
7,622
7,097
7,208
6,833
6,474
6,208
5,914 127
139
162
167
205
179
185
212
209
195 0.015724
0.017246
0.018921
0.020763
0.022790
0.025019
0.027470
0.030167
0.033134
0.036398 125.26
133.28
148.91
158.26
161.74
180.33
187.71
195.30
205.70
215.26
Total
zx
0.1556
0.4954
1.0728
0.6949
3.4018
–0.0993
–0.1974
1.1947
0.2303
–1.3806
zx 2
0.0242
0.2454
1.1508
0.4830
11.5720
0.0099
0.0390
1.4273
0.0530
1.9060
16.9106

The observed test statistic is 16.91.

The number of age groups is 10,
Page 19%%%%%%%% – Examiners’ Report
but we lose three degrees of freedom, one for each parameter, 
so m = 7. 
The critical value of the χ 2 distribution with 7 degrees of freedom at the 95%
significance level is 14.07.

(iii)
Since 16.91 > 14.07, 
we have sufficient evidence to reject the null hypothesis. 

May fail to detect small but consistent bias. 
For this use the signs test or cumulative deviations test over the
whole age range. 
May fail to detect a few large deviations offset by a lot of small deviations
OR
may fail to detect outliers.

For this use the standardised deviations test 
The shape of the graduation may be wrong
OR
even if there is not bias over the whole range, there may be areas of the
graduation where there is significant bias.
OR
there may be clumping of the signs. 
For this use the grouping of signs test, the cumulative deviations test over
sections of the age range, or the serial correlations test at lag 1.

(iv)
The graduated rates may not be smooth. 
For this use the third differences test. 

Signs test
The null hypothesis is that the graduated rates are the same as the true
underlying rates in the block of business. 
We have 3 negative signs out of 10 ages. 
EITHER
The probability of getting exactly 3 negative signs is equal to
Page 20%%%%%%%% – Examiners’ Report
 10  10
  0.5 \;=\; 0.1172
 3 
OR
The probability of getting 3 or fewer negative signs is 0.1719, %%---
which is greater than 0.025 (two tailed test) 
Therefore at the 95% significance level we do not reject the null
hypothesis: we can say that there is no bias. 
Cumulative Deviations test
The null hypothesis is that the graduated rates are the same as the true
underlying rates in the block of business.

 (Observed deaths  Expected deaths)
the test statistic
x
 Expected deaths
~ Normal(0,1)

x
So, using the results in the table, the value of the test statistic is
1,780  1,712
\;=\; 1.6499
1,712 %%---
Since –1.96 < test statistic < %%---.96 
Therefore at the 95% significance level we do not reject the null
hypothesis: there does not appear to be a bias. 
Grouping of Signs test
The null hypothesis is that the graduated rates are the same as the true
underlying rates in the block of business. 
We have n 1 = 7 positive signs and n 2 = 3 negative signs. 
There are 2 positive runs. 
EITHER
Using the table on p. 189 of the Golden Book, we reject the null
hypothesis with 1 positive run or fewer.

Since 2 > 1, we do not reject the null hypothesis: there do not seem
to be an unduly large number of runs of consecutive ages with the
Page 21%%%%%%%% – Examiners’ Report
same sign.
%%---
OR
Since
 6  4 
  
0 1
4
1
\;=\;
Pr[1 positive run] =     \;=\;
120 30
 10 
 
 7 
 6  4 
  
1 2
36
3
\;=\;
Pr[2 positive runs] =     \;=\;
120 10
 10 
 
 7 
The calculations show that Pr[1 positive run] < 0.05, but
Pr[2 positive runs] > 0.05. 
Hence we do not reject the null hypothesis: there do not seem
to be an unduly large number of runs of consecutive ages with the
same sign. %%---
Individual Standardised Deviations test
The null hypothesis is that the graduated rates are the same as the true
underlying rates in the block of business. 
Under the null hypothesis we would expect the individual z x s to be
distributed Normal (0,1). 
EITHER
Only 1 in 20 z x s should have absolute magnitude
greater than 1.96, and none should be outside the range 3 to +3,
OR
a table showing split of deviations, actual versus expected as below.
Range
∞,3
Expected 0.0
Actual
0
3,2
0.2
0
2,1
1.4
1
1,0
3.4
2
0,1
3.4
4
1,2
1.4
2
2,3
0.2
0
3, +∞
0.0
1
%%---
Page 22%%%%%%%% – Examiners’ Report
z 64 = 3.40, is a definite outlier. 
Therefore we reject the null hypothesis. 
Test for smoothness
Age
x Graduated rates
μ x First difference
 x \;=\;  x   x  1 Second difference
 2 x \;=\;  x   x 1 Third difference
 3 x \;=\;  2 x   2 x 1
60
61
62
63
64
65
66
67
68
69 0.015724
0.017246
0.018921
0.020763
0.022790
0.025019
0.027470
0.030167
0.033134
0.036398 0.001522
0.001675
0.001842
0.002027
0.002229
0.002451
0.002697
0.002967
0.003264 0.000153
0.000167
0.000185
0.000202
0.000222
0.000246
0.000270
0.000297 0.000014
0.000018
0.000017
0.000020
0.000024
0.000024
0.000027
+2
Note that it does not matter against which ages the third differences appear in the table: they could appear against ages 60-66 or against ages 63-69.
The third differences are small and progress smoothly with age.
Therefore the graduation is acceptably smooth.



[Total 17]
This question was very well answered by many candidates, a high proportion scoring 13 or more. In part (iv), credit was given to candidates who attempted the serial correlations test with lag 1 to test
whether the shape of the graduated rates were consistent with the underlying mortality. However, we do not recommend that candidates attempt this test in the examination if they have a choice. The calculations are very time-consuming and errors are likely; and other, quicker, tests for the same defect are available.


\end{document}
