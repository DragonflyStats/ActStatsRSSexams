\documentclass[a4paper,12pt]{article}

%%%%%%%%%%%%%%%%%%%%%%%%%%%%%%%%%%%%%%%%%%%%%%%%%%%%%%%%%%%%%%%%%%%%%%%%%%%%%%%%%%%%%%%%%%%%%%%%%%%%%%%%%%%%%%%%%%%%%%%%%%%%%%%%%%%%%%%%%%%%%%%%%%%%%%%%%%%%%%%%%%%%%%%%%%%%%%%%%%%%%%%%%%%%%%%%%%%%%%%%%%%%%%%%%%%%%%%%%%%%%%%%%%%%%%%%%%%%%%%%%%%%%%%%%%%%

\usepackage{eurosym}
\usepackage{vmargin}
\usepackage{amsmath}
\usepackage{graphics}
\usepackage{epsfig}
\usepackage{enumerate}
\usepackage{multicol}
\usepackage{subfigure}
\usepackage{fancyhdr}
\usepackage{listings}
\usepackage{framed}
\usepackage{graphicx}
\usepackage{amsmath}
\usepackage{chngpage}

%\usepackage{bigints}
\usepackage{vmargin}

% left top textwidth textheight headheight

% headsep footheight footskip

\setmargins{2.0cm}{2.5cm}{16 cm}{22cm}{0.5cm}{0cm}{1cm}{1cm}

\renewcommand{\baselinestretch}{1.3}

\setcounter{MaxMatrixCols}{10}

\begin{document}

CT4 S2017–45
(i) List the key steps involved in developing an actuarial model.
(ii) Comment on considerations which would apply if you were developing a
model of the spread of a newly discovered disease.

[Total 7]
%%%%%%%%%%%%%%%%%%%%%%%%%%%%



Q5
Page 6
(i)
Define the objectives of the modelling process. 
Plan the modelling process and how it will be validated. 
Collect and validate the data required. 
Define the parameters for the model and consider appropriate parameter
values 
Define the model by capturing the essence of the real world system. 
Involve experts on the real world system/get feedback on validity. %%%%%%%% – Examiners’ Report
Decide on software to be used, choose random number generator, etc. 
Write the computer program. 
Debug the program. 
Analyse the output. 
Test the reasonableness of the output. 
Consider appropriateness of response of the model to small changes in
input parameters. 
Ensure that any relevant professional guidance or standards have been
complied with. 
Document the model and ensure the results are in a format which can
easily be communicated.

.
[max. 4]
(ii)
There will, by definition, be (virtually) no data about this disease yet. 
Consideration may need to be given to using data from other diseases. 
Expert input will be particularly important. 
As some of the parameters may be highly uncertain and depend
on the form of transmission, may need to test the robustness
to a wider change in input parameters than usual. %%---
Typical parameters would be the period for which a person is
contagious, the probability of passing on the disease on contact,
and the number of people each person would be in contact with. %%---
The form of model may not need to differ from one used to study the spread of other diseases. 
Results could be validated against the spread of other emerging diseases. 
Careful reporting of the model findings, emphasising margins of error, may be advisable so as not to cause panic. 
It may be relevant to consider the characteristics of the environments
in which the outbreak will happen, and whether the disease is likely to affect the whole population, or some population sub-groups,
defined on the basis of age, sex, etc. 
The possibility of finding a vaccine may be an important
consideration. 
Page 7%%%%%%%% – Examiners’ Report
The time required to develop the model may be critical as results may be
required quickly if the disease is spreading rapidly.

[max. 3]
[Total 7]
Part (i) of this question was very well answered, with many candidates
scoring full marks. Part (ii) was more demanding, and many
candidates only made a cursory attempt. The Examiners were looking for answers which revealed thought about the specific scenario
described in the question. Credit was given for a wide range of points,
including some not listed above. Little credit, however, was awarded
to answers which were couched in general terms, without reference to
the spread of a newly discovered disease. In both sections of this
question, not all the points listed above were required for full credit.
