\documentclass[a4paper,12pt]{article}

%%%%%%%%%%%%%%%%%%%%%%%%%%%%%%%%%%%%%%%%%%%%%%%%%%%%%%%%%%%%%%%%%%%%%%%%%%%%%%%%%%%%%%%%%%%%%%%%%%%%%%%%%%%%%%%%%%%%%%%%%%%%%%%%%%%%%%%%%%%%%%%%%%%%%%%%%%%%%%%%%%%%%%%%%%%%%%%%%%%%%%%%%%%%%%%%%%%%%%%%%%%%%%%%%%%%%%%%%%%%%%%%%%%%%%%%%%%%%%%%%%%%%%%%%%%%

\usepackage{eurosym}
\usepackage{vmargin}
\usepackage{amsmath}
\usepackage{graphics}
\usepackage{epsfig}
\usepackage{enumerate}
\usepackage{multicol}
\usepackage{subfigure}
\usepackage{fancyhdr}
\usepackage{listings}
\usepackage{framed}
\usepackage{graphicx}
\usepackage{amsmath}
\usepackage{chngpage}

%\usepackage{bigints}
\usepackage{vmargin}

% left top textwidth textheight headheight

% headsep footheight footskip

\setmargins{2.0cm}{2.5cm}{16 cm}{22cm}{0.5cm}{0cm}{1cm}{1cm}

\renewcommand{\baselinestretch}{1.3}

\setcounter{MaxMatrixCols}{10}

\begin{document}



%%- Question 10
(i)
Write down the formulae for the Kaplan-Meier estimator Ŝ(t) and Nelson-Aalen estimator S̃(t) of survival in the presence of a stated hazard,
defining all terms used.

The following graph shows the functions:
y = 1 – x and
y = e –x over the range 0 \leq x \leq 1.
1.00
0.90
0.80
0.70
0.60
y = 1 – x
0.50
y = exp(– x)
0.40
0.30
0.20
0.10
0.00
(ii)
0
0.1 0.2 0.3 0.4 0.5 0.6 0.7 0.8 0.9 1.00
Value of x
Demonstrate that the Nelson-Aalen estimator is never lower than the
Kaplan-Meier estimator.

A trial is conducted amongst 20 patients who have suffered from eczema but are
in remission (that is, they are clear of the condition). The trial is to assess whether
continuing with periodic doses of a certain steroid cream in remission reduces the rate
at which eczema recurs. Patients are invited to tests every 3 months for a period of up
to 5 years from when first declared to be in remission.
(iii)
Describe THREE types of censoring present in the investigation.
/
The data for the trial are subdivided into a group who continued to receive the steroid
cream, and a control group who did not receive the steroid cream. The data for the
patients in the trial showing the quarterly test at which eczema recurred, or censoring
occurred, are as follows (an * indicates a patient who was censored):
For group receiving steroid cream:
For control group:
CT4 S2017–9 
3, 5, 6*, 7*, 10, 10, 12*, 14*, 18, 19*
6, 8, 8, 10*, 11*, 12*, 14, 15*, 18, 18
%%%%%%%%%%%%%%%%%%%%%%%%%%%%%%%(iv)
Calculate the Kaplan-Meier estimates of the survival function for remaining
clear of eczema for:
(a)
the group who continued to receive the steroid cream; and
(b)
the control group.

(v)
(a)
Recommend, without performing any calculations, a method of
establishing whether the hazard of eczema returning is statistically
lower for those continuing to receive the steroid cream.
(b)
Comment on the chance of being able to conclude from the trial
data that continuing to receive the steroid cream reduces the risk of
recurrence of eczema.


%%%%%%%%%%%%%%%%%%%%%%%%%%%%%%%%%%%%%%%%%%%%%%%%%%%%%%%%%%%

Q10
(i)
The Kaplan-Meier estimator is
æ d ö
Ŝ(t) = Õ ç 1- j ÷
n j ø
t j <=t è

and the Nelson-Aalen estimator is
Page 23%%%%%%%% – Examiners’ Report
,

where d j represents the number of occurrences of the event of
interest at duration t j ,

and n j represents the number exposed to the hazard at duration t j .
(ii)


Expanding into the individual terms:
 d j
 d  d 
S ˆ ( t ) \;=\;  1  1  .  1  2  .......  1 
 n j
 n 1   n 2 


  ,


and

d j
S ( t ) \;=\; exp   
 t \;=\; t n j
 j

 d
 \;=\; exp    d 1   .exp    d 2   ........exp   j
 n j

 n 1 
 n 2 



 .



As each d j /n j must be between 0 and 1 the chart shows each term in the
Nelson-Aalen estimator is no lower than the parallel in the Kaplan-Meier. 
Hence the Nelson-Aalen estimator is always no lower than the
Kaplan-Meier estimator.
(iii)


Interval censoring is present because the tests only take place every three
months and recurrence of eczema could occur between tests.
%%---
Type 1 censoring is present because it is specified in advance that the study
will end after 5 years.
%%---
Random censoring is present as for patients who leave the study, the time of
their censoring can be considered a random variable.
%%---
Right censoring is present for patients still free of eczema after 5 years or
patients who left the study, as we do not know when the reoccurrence of
eczema happened, just that it happened after a certain date.
%%---
Non-informative censoring could be said to be present as we have no reason
to believe that those patients who left the study were any more or less likely to
have the eczema recur than those who remained in the study.
%%---
Page 24%%%%%%%% – Examiners’ Report
Informative censoring could be said to be present as we could argue that
those who left the study may have done so because they considered
themselves cured, and were therefore less likely to suffer a recurrence than
those still in the study.
%%---
[max. 3]
(iv)
For the group continuing to receive steroid cream:
t j n j d j c j \lambda j 1 – \lambda j
3
5
10
18 10
9
6
2 1
1
2
1 0
2
2
1 1/10
1/9
1/3
1/2 9/10
8/9
2/3
1/2
+2
The Kaplan-Meier estimate of the survival function, S(t) KM , is
t S(t) KM
0 \leq t < 3
3 \leq t < 5
5 \leq t < 10
10 \leq t < 18
18 \leq t < 20 1
9/10
4/5
8/15
4/15
%%--- %%---
+2
For the control group:
t j n j d j c j \lambda j (1 – \lambda j )
6
8
14
18 10
9
4
2 1
2
1
2 0
3
1
0 1/10
2/9
1/4
2/2 9/10
7/9
3/4
0
+2
The Kaplan-Meier estimate of the survival function, S(t) KM , is:
t S(t) KM
0 \leq t < 6
6 \leq t < 8
8 \leq t < 14
14 \leq t < 18
18 \leq t < 20 1
9/10
7/10
21/40
0
Page 25%%%%%%%% – Examiners’ Report
+2

(v)
(a)
In order to assess whether the risk is statistically lower a simple and quick approach would be to calculate confidence intervals around each survival function. 
If the confidence intervals do not overlap the survival rate is statistically higher or lower at the chosen confidence level. 
For the Kaplan Meier estimate the variance can be estimated using
Greenwood’s formula,

which is:
Var[ S ( t )]  ( S ( t )) 2  t
(b)
d j
j  t
n j ( n j  d j )
.

Methods such as the log-rank test or Wilcoxon’s test could
be used. 
In this case it is unlikely it could be shown that continuing to
receive steroid cream statistically reduces the risk of recurrence, 
as the sample size is small 
and the survival rates do not appear markedly better for the group
receiving steroid cream.

[max. 3]
[Total 18]
In part (i) many candidates were imprecise about the durations over
which the product should be calculated for the Kaplan-Meier estimate,
or over which the hazards should be summed for the Nelson-Aalen method. Vague descriptions, such as ‘sum over j’, lost marks. Many
candidates failed to state that the d j and n j were the deaths and
remaining risk set at duration t j . Part (ii) was very poorly answered,
with many candidates supposing that the graph in the question paper
was of the Kaplan-Meier and Nelson-Aalen estimates and merely describing the graph. Such answers scored little or no credit. In part
(iii) a minority of the candidates worked on the basis that each of the
20 sample members was in both groups. Good answers to part (iv)
were very few. Credit was given for mentioning sensible approaches
other than those listed above.
\end{document}
