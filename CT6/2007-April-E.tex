\documentclass[a4paper,12pt]{article}

%%%%%%%%%%%%%%%%%%%%%%%%%%%%%%%%%%%%%%%%%%%%%%%%%%%%%%%%%%%%%%%%%%%%%%%%%%%%%%%%%%%%%%%%%%%%%%%%%%%%%%%%%%%%%%%%%%%%%%%%%%%%%%%%%%%%%%%%%%%%%%%%%%%%%%%%%%%%%%%%%%%%%%%%%%%%%%%%%%%%%%%%%%%%%%%%%%%%%%%%%%%%%%%%%%%%%%%%%%%%%%%%%%%%%%%%%%%%%%%%%%%%%%%%%%%%

\usepackage{eurosym}
\usepackage{vmargin}
\usepackage{amsmath}
\usepackage{graphics}
\usepackage{epsfig}
\usepackage{enumerate}
\usepackage{multicol}
\usepackage{subfigure}
\usepackage{fancyhdr}
\usepackage{listings}
\usepackage{framed}
\usepackage{graphicx}
\usepackage{amsmath}
\usepackage{chngpage}

%\usepackage{bigints}
\usepackage{vmargin}

% left top textwidth textheight headheight

% headsep footheight footskip

\setmargins{2.0cm}{2.5cm}{16 cm}{22cm}{0.5cm}{0cm}{1cm}{1cm}

\renewcommand{\baselinestretch}{1.3}

\setcounter{MaxMatrixCols}{10}

\begin{document}

\begin{enumerate}
%%%%%%%%%%%%%%%%%%%%%%%%%%%%%%%%%%%%%%%%%%

9
An insurer’s NCD scale for motor policies has 3 levels of discount: 0%, 25% and
40%. The rules for moving between these levels are as follows:
• following a claim-free year, a policyholder moves to the next higher level of
discount, or remains at 40% discount
• following a year of one or more claims, a policyholder at 40% discount moves to
25% discount while a policyholder at 25% or 0% moves to or stays at 0% discount
The full premium for each policyholder is £1,000. Following an accident,
policyholders decide whether or not to claim by considering total outgoing over the next two years, assuming no further claims in this period and ignoring interest.
\begin{enumerate}[(i)]
\item (i) Find the claim threshold for each level of discount.
\item (ii) The probability of no accidents in any year for each policyholder is 0.88 and
individual losses are assumed to have a lognormal distribution with μ = 6.0
and σ = 3.33. Ignoring the possibility of more than 1 accident occurring in a
year, calculate the transition matrix.
\item (iii) Calculate the stationary distribution.
\item (iv) Derive the stationary distribution under the alternative assumption that a
policyholder always claims after a loss (regardless of the size of the claim). [3]
\item (v) Comment on the difference between the results of (iii) and (iv).
\end{enumerate}
%%%%%%%%%%%%%%%%%%%%%%%%%%%%%%%%%%%%%%%%%%%
10
(i)
The Gamma distribution with mean μ and variance μ 2 /α has density function
\[f(y) =
α α
μ α Γ ( α )
y
α− 1
e
−
y α
μ
( y > 0 )\]
(a) Show that this may be written in the form of an exponential family.
(b) Use the properties of exponential families to confirm that the mean and
[9]
variance of the distribution are μ and μ 2 /α.
(ii) Explain the difference between a continuous covariate and a factor.
[3]
(iii) A company is analysing its claims data on a portfolio of motor policies, and uses a gamma distribution to model the claim severities. The company uses
three rating factors:
policyholder age (as a continuous variable);
policyholder gender;
vehicle rating group (as a factor).
(a) Write down the form of the linear predictor when all rating factors are included as main effects.
(b) State how the linear predictor changes if an interaction between
policyholder age and gender is included.
\newpage
%%%%%%%%%%%%%%%%%%%%%%%%%%%%%%%%%%%%%%%%%%%%%%%%%%%%%%%%%%%%%%%%%%%%%%%%%%%%%%%%%%%%%%%%%%%%%%%
9
(i)
£
Starting level (Year 0)
0%
25%
40%
Year 1 Premium if claim in year 0
Premium if no claim in year 0
Saving
Year 2 Premium if claim in year 0
Premium if no claim in year 0
Saving
Claim threshold
(ii)
1,000
750
250
750
600
150
400
1,000
600
400
750
600
150
550
750
600
150
600
600
0
150
P(X > 400) = P(log X > log 400) = P(Z > (log400 - 6)/3.33) = P(Z > - 0.0026)
= 1 - 0.4990 = 0.5010
P(X > 550) = P(log X > log 550) = P(Z > (log550 - 6)/3.33)
= P(Z > 0.0931) = 1 – 0.5371 = 0.4629
P(X > 150) = P(log X > log 150) = P(Z > (log150 - 6)/3.33)
= P(Z > - 0.2971) = 0.6168
0.12 × 0.5010 = 0.0601
0.12 × 0.4629 = 0.0556
0.12 × 0.6168 = 0.0740
Hence the transition matrix is
0 ⎤
⎡ 0.0601 0.9399
⎢ 0.0556
0
0.9444 ⎥ ⎥
⎢
⎢ ⎣ 0
0.0740 0.9260 ⎥ ⎦
(iii)
π = π P
0.0601 π 0 +
π 0 =
π 25 = 0.9399 π 0 +
π 40 =
π 25 = 16.905 π 0
π 40 = 215.740 π 0
π 0 + π 25 + π 40 = 1
0.0556 π 25
0.9444 π 25 +
0.0740 π 40
0.9260 π 40
Page 9Subject CT6 (Statistical Methods Core Technical) — April 2007 — Examiners’ Report
Hence
π = (0.0043, 0.0724, 0.9234)
(iv)
Stationary distribution if policyholder claims after a loss
0 ⎤
⎡ 0.12 0.88
P = ⎢ ⎢ 0.12
0
0.88 ⎥ ⎥
⎢ ⎣ 0
0.12 0.88 ⎥ ⎦
π 0 =
π 25 =
π 40 =
0.12 π 0 +
0.88 π 0 +
0.12 π 25
0.88 π 25 +
0.12 π 40
0.88 π 40
π 25 = 7.333 π 0
π 40 = 7.333 π 25 = 53.778 π 0
π 0 + π 25 + π 40 = 1
Hence:
π = (0.0161, 0.1181, 0.8658)
10
(v) Award 1 mark for any sensible comment on the reduction in the number of
policyholders in the lower discount categories (or increase in the higher discount categories)
(i) (a)
⎡ y α
⎤
f(y) = exp ⎢ −
− α log μ + ( α − 1) log y + α log α − log Γ ( α ) ⎥
⎣ μ
⎦
⎡ ⎛ y
⎤
⎞
= exp ⎢ α ⎜ − − log μ ⎟ + ( α − 1) log y + α log α − log Γ ( α ) ⎥
⎠
⎣ ⎝ μ
⎦
which is in the form of an exponential family.
θ = −
1
μ
b(θ) = log μ
⎛ 1 ⎞
= log ⎜ − ⎟ = - log( - θ)
⎝ θ ⎠
%%--- Page 10Subject CT6 (Statistical Methods Core Technical) — April 2007 — Examiners’ Report
(b)
The mean and variance of the distribution are b ′ ( θ ) and a ( φ ) b ′′ ( θ )
b ′ ( θ ) = −
1
= μ
θ
which confirms that the mean is μ.
b ′′ ( θ ) =
a(φ) =
1
θ
2
= μ 2
1
α
hence the variance is
(ii)
1 2
μ , as required
α
A factor is categorical e.g. male/female.
For a continuous covariate, the value is included. For example, if x is a continuous covariate, a main effect would be $\alpha  + \beta x$.
(iii)
(a)
The linear predictor has the form
\[α i + β j + γx\]
where α i is the factor for policyholder gender (i = 1,2)
β j is the factor for vehicle rating group
x is the policyholder age
(α 1 = 0, β 1 = 0)
(b)
The linear predictor becomes
\[α i + β j + γ i x\]
\end{document}
