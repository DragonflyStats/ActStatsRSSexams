CT6 A2006 610
An insurance company has two portfolios of independent policies, on each of which
claims occur according to a Poisson process. For the first portfolio, all claims are for
a fixed amount of £5,000 and 10 claims are expected per annum. For the second
portfolio, claim amounts are exponentially distributed with mean £4,000 and 30
claims are expected per annum.
Let S denote aggregate annual claims from the two portfolios together.
A check is made for ruin only at the end of the year.
The insurer includes a loading of 10% in the premiums, for all policies.
(i) Calculate the mean and variance of S.
[4]
(ii) Use a normal approximation to the distribution of S to calculate the initial
capital, u, required in order that the probability of ruin at the end of the first
year is 0.01.
[3]
The insurer is considering purchasing proportional reinsurance from a reinsurer that
includes a loading of in its premiums. The proportion of each claim to be retained
by the direct insurer is (0
1).
Let S I denote the aggregate annual claims paid by the direct insurer on the two
portfolios together, net of reinsurance.
(iii)
Use a normal approximation to the distribution of S I to show that the initial
capital, u , required in order that the probability of ruin at the end of the first
year is 0.01 can be written as
u = u + (1
) (
0.1) E[S].
(iv) Show that u > u , as long as < 0.476.
(v) Show that u u decreases as
implications of this result.
[3]
increases, and discuss the practical
END OF PAPER
CT6 A2006 7
[6]
[3]
[Total 19]

%%%%%%%%%%%%%%%%%%%%%%%%%%%%%%%%%%%%%%%%%%%%%%%%%%%%%%%%%%%%%%%%%%%%%%%%%%%%%%%%%%%%%%%%%%%%%%%%%

10
(i)
E(S) = E[S 1 ] + E[S 2 ]
= 10 5,000 + 30
April 2006
Examiners Report
4,000 = 170,000
Var[S] = Var[S 1 ] + Var[S 2 ]
= 10 5,000 2 + 30 (4,000 2 + 4,000 2 )
= 1.21 10 9
(ii)
We require u such that
P(u + c < S) = 0.01
S E ( S )
Var( S )
u c E ( S )
= 0.01
Var( S )
i.e. P
so u c E ( S )
= 2.326
Var( S )
u = 2.326 Var( S )
E ( S ) 1.1 E ( S )
= 2.326 Var( S ) 0.1 E ( S )
= 63,922
(iii)
We require u such that
P ( u
c c R
S I ) = 0.01
where c R = reinsurance premium
u
c c R E [ S I ]
u c E [ S ]
= 2.326 =
Var[ S I ]
Var( S )
E[S I ] = E[S] and Var[S I ] =
c R = (1 + )(1
Hence
u
2 Var[S].
) E[S]
1.1 E [ S ] (1
)(1 ) E [ S ]
Var[ S ]
E [ S ]
=
u 1.1 E [ S ] E [ S ]
Var( S )
u = (u + 0.1E[S]) 1.1E[S] + (1 + )(1
) E[S] + E[S]
= u + (0.1 1.1 + 1
+ (1
) + ) E(S)
= u + (1
) (
0.1) E(S)
Page 12Subject CT6 (Statistical Methods Core Technical)
(iv)
u u = (1
u u
(v)
0
)[u
u
(
(
Examiners Report
0.1) E(S)]
0.1) E(S) > 0
i.e. < u
0.1
E ( S )
i.e. < 63,922
0.1 = 0.476
170, 000
Since (1
April 2006
) E[S] > 0, u
u decreases as increases.
The greater the premium loading required by the reinsurer, the smaller the
reduction in capital required by the insurer, i.e. the less effective the
reinsurance is in reducing P(ruin) and hence replacing the capital.
Comments on question 10: After Q8, candidates found this the most difficult question on the
paper. Parts (i) and (ii) were well answered but the inclusion of premium loadings confused
most candidates and consequently very few scored well on (iii), (iv) and (v).
\end{document}
