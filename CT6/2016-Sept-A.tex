1 An actuary is considering a portfolio of insurance policies. This portfolio only
contains five policies. Each policy has been taken out by a different couple who are
getting married on the same day in the same city. Each policy would pay out in the
event of rain on that day in that city.
(i) Explain why these policies may not meet the ideal criteria for an insurable
risk. [3]
(ii) Suggest two ways in which either the portfolio or the policy terms could be
changed in order to better meet the criteria for insurable risks. [2]
%%%%%%%%%%%%%%%%%%%%%%%%%%%%%%%%%%%%%%%%%%%%%%%%%%%%%%%%%%%%%%%%%%%%%%%%%%%%%%%%%%%%%%%%%%%
2 Andy is playing a game, which involves rolling four-sided fair dice. Each time a dice
is rolled, it is equally likely to show one of the numbers: 1, 2, 3 or 4.
Before each roll, he has three strategies:
  a1: Receive 1.5 times the number showing.
a2: Receive half the number showing if it is odd, and twice the number if it is even.
a3: Receive the number showing if it is even, and twice the number if it is odd.
(i) Construct Andy’s payoff matrix. [2]
(ii) State which, if any, of the decision functions are dominated. [1]
(iii) Determine Andy’s optimal strategy under the Bayes criterion. [3]

%%%%%%%%%%%%%%%%%%%%%%%%%%%%%%%%%%%%%%%%%%%%%%%%%%%%%%%%%%%%%%%%%%%%%%%%%%%%%%%%%%%%%%%%%%%
Solutions
Q1 (i) Not sufficiently diversified / not independent / not large enough sample [1]
Not easily quantifiable [1]
Risk is potentially quite large (not necessarily a remote risk) [1]
Policyholders may not have financial interest in risk [1]
Moral hazard [1]
[Max 3]
(ii) Increase number of policies [1]
ensure definition of bad weather is extreme [1]
Diversify policies between different cities [1]
Diversify policies between different days [1]
Introduce policy excess [1]
[Max 2]
Most candidates were able to relate the theory of insurable risks to this
question. A few candidates failed to sufficiently distinguish between their
points, or did not consider the actual context presented.
Q2 (i) Let θi be the state of nature when the roll of the die = i.
Then the payoff matrix is:
  θ1 θ2 θ3 θ4
a1 1.5 3 4.5 6
a2 0.5 4 1.5 8
a3 2 2 6 4
[1 mark for first correct row, ½ mark thereafter]
(ii) None of the decision functions is dominated. [1]
(iii) Since each number is equally likely, this is equivalent to summing up the
payoffs for each decision function. [1]
This is 15, 14 and 14. [1]
So a1 is the optimal decision under the Bayes criterion. [1]
This straightforward question was very well answered by most candidates,
with many scoring full marks.
Subject CT6 (Statistical Methods Core Technical) – September 2016 – Examiners’ Report
Page 4
