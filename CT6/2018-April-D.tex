7
(i) Set out the four other strategies in addition to A (labelled B to E) which Liam
could adopt.
[4]
(ii) Construct the payoff matrix to Tarik.
(iii)
 Explain whether or not there is a saddle point.
[3]
[2]
[Total 9]
Consider the following time series model:
(1 – αB) 3 X t = ε t
where B is the backshift operator and e t is a standard white noise process with
variance s 2 .
(i)
Determine for which values of a the process X t is stationary.
[2]
Now assume that X t is stationary.
(ii) Calculate the autocorrelation function for the first two lags: r 1 and r 2 , using
the Yule-Walker equations.
[7]
(iii) State the formulae, in terms of r 1 and r 2 , for the first two values of the partial
auto correlation function f 1 and f 2 .[1]
Now assume that a = 1.
(iv)

CT6 A2018–4
Explain how to fit the parameter of this model, given the time series
observations X 1 , X 2 , ..., X T .[2]
[Total 12]8
Claim events on a portfolio of insurance policies follow a Poisson process with
parameter l. Individual claim amounts, X, follow a Normal distribution with
parameters m = 500 and s 2 = 200.
The insurance company calculates premiums using a premium loading factor of 20%.
(i)
Show that the adjustment coefficient, r = 0.000708 to three significant figures.
[4]
The insurance company’s initial surplus is 5,000.
(ii)
Calculate an upper bound on the probability of ruin, using Lundberg’s
inequality.[1]
The insurance company actuary believes that claim amounts are better modelled using
an exponential distribution. You may assume that the mean m is unchanged, and is
now equal to the standard deviation.
(iii) Calculate the new upper bound of the probability of ruin.
(iv) Give a reason why claim amounts on insurance policies are not usually
modelled using a Normal distribution, and suggest an alternative distribution,
other than the exponential.
[2]
[Total 12]

CT6 A2018–5 

Q7
(i) The characteristic polynomial has a triple root of B = 1 / α
and therefore for α < 1 ensures stationarity
(ii) Expanding the cubic sum in the initial equation we have
( 1 − 3 α B + 3 α B
2 2
[1]
[1]
)
− α 3 B 3 X t = ε t
X t − 3 α X t − 1 + 3 α 2 X t − 2 − α 3 X t − 3 = ε t
[1]
We have an AR(3) process with parameters a 1 = 3 α , α 2 =− 3 α 2 and a 3 = α 3
.
[2]
The Yule walker equations for this process are for ρ 1 :
Page 7Subject CT6 (Statistical Methods Core Technical) – April 2018 – Examiners’ Report
ρ 1 = a 1 + a 2 ρ 1 + a 3 ρ 2
ρ 1 ( 1 − a 2 ) =           
a 1 + a 3 ρ 2
( 1 )
[1]
And for ρ 2 :
ρ 2 = a 1 ρ 1 + a 2 + a 3 ρ 1
ρ 2 =
( a 1 + a 3 ) ρ 1 + a 2           ( 2 )
[1]
And therefore rearranging (1) and (2)
a 1 + a 3 a 2
3 α − 3 α 5
=
ρ 1
=
1 − a 2 − a 1 a 3 − a 3 2 1 + 3 α 2 − 3 α 4 − α 6
[1]
And
ρ 2 =
(iii)
( a 1 + a 3 ) ρ 1 + a 2 =
(
3 α + α 3
) 1 + 3 α
3 α − 3 α 5
2
− 3 α − α
4
6
− 3 α 2
Using the definition (Tables have these too)
ρ − ρ 2
φ 1 =ρ 1   
and φ 2 = 2 2 1
1 − ρ 1
(iv)
[1]
[1]
For α = 1 , the process is not stationary but ∇ 3 X t is as it becomes a white
noise.
[1]
In this case we need to fit the white noise to the T − 3 third differenced
observations ∇ 3 X 1 , ∇ 3 X 2 ... ∇ 3 X T − 3 .
[1]
[Total 12]
Many candidates scored well on this relatively straightforward time-
series question, although a good number struggled with the algebra in
part (ii) and only the strongest candidates were able to answer part (iv)
correctly.
Q8
(i)
r is the unique positive root of the equation:
λ + cr = λ M X ( r )
[1⁄2]
c 1.2 λ E ( X ) , so the equation simplifies to
Here =
1 + 1.2 E ( X ) r =
M X ( r )
[1⁄2]
Page 8Subject CT6 (Statistical Methods Core Technical) – April 2018 – Examiners’ Report
M X ( r ) =
1
μ r + σ 2 r 2
e 2
(from tables)
[1⁄2]
E ( X ) = 500 (from question)
So 1 + 600 r
1
500 r + 200 r 2
2
− e
[1⁄2]
= 0
[1]
At r = 0.000 7075, LHS is +0.000 03, and at r = 0.000 708 5, LHS is
–0.000 08
[1]
So the adjustment coefficient must be 0.000 708 to 3sf
− 0.000 708*5000
=
By Lundberg’s inequality, upper bound given
by e − RU e =
0.029
(ii)
(iii)
λ = 0.002
MGF is
(iv)
[1]
[1⁄2]
λ
λ − r
[1⁄2]
0.002
1 + 600 r =
0.002 − r [1]
(1 + 600 r )(0.002 −=
r ) 0.002 ⇒ 1.2 r − r − 600 =
r 2 0 [1]
r (0.2 − 600 r ) = 0 ⇒ r = 1 [1]
3000
e − RU = 0.189 [1]
Normal distributions allow the possibility of negative claim amounts [1]
Normal distributions do not have “fat tails”, commonly observed in insurance
claims
[1]
Normal distributions are not positively skewed, unlike typical claim amounts
[1]
[Max 1]
Any sensible alternative (gamma, Pareto, Weibull etc.)
[1]
[Total 12]
Most candidates are now familiar with the method required in part (i),
and were able to score well throughout this question.
Page 9Subject CT6 (Statistical Methods Core Technical) – April 2018 – Examiners’ Report
