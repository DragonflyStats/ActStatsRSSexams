\documentclass[a4paper,12pt]{article}

%%%%%%%%%%%%%%%%%%%%%%%%%%%%%%%%%%%%%%%%%%%%%%%%%%%%%%%%%%%%%%%%%%%%%%%%%%%%%%%%%%%%%%%%%%%%%%%%%%%%%%%%%%%%%%%%%%%%%%%%%%%%%%%%%%%%%%%%%%%%%%%%%%%%%%%%%%%%%%%%%%%%%%%%%%%%%%%%%%%%%%%%%%%%%%%%%%%%%%%%%%%%%%%%%%%%%%%%%%%%%%%%%%%%%%%%%%%%%%%%%%%%%%%%%%%%

\usepackage{eurosym}
\usepackage{vmargin}
\usepackage{amsmath}
\usepackage{graphics}
\usepackage{epsfig}
\usepackage{enumerate}
\usepackage{multicol}
\usepackage{subfigure}
\usepackage{fancyhdr}
\usepackage{listings}
\usepackage{framed}
\usepackage{graphicx}
\usepackage{amsmath}
\usepackage{chngpage}

%\usepackage{bigints}
\usepackage{vmargin}

% left top textwidth textheight headheight

% headsep footheight footskip

\setmargins{2.0cm}{2.5cm}{16 cm}{22cm}{0.5cm}{0cm}{1cm}{1cm}

\renewcommand{\baselinestretch}{1.3}

\setcounter{MaxMatrixCols}{10}

\begin{document}
\begin{enumerate}

%%--- Question 9
(i)
List the main steps in the Box-Jenkins approach to fitting an ARIMA time
series to observed data.

Observations x 1 , x 2 , ..., x 200 are made from a stationary time series and the following
summary statistics are calculated:
200 200 200
i = 1 i = 1 i = 2
Σ x i = 83.7
Σ ( x i − x ) 2 = 35.4
Σ ( x i − x )( x i − 1 − x ) = 28.4
200
Σ ( x i − x )( x i − 2 − x ) = 17.1
i = 3
(ii) Calculate the values of the sample auto-covariances γ ˆ 0 , γ ˆ 1 and γ ˆ 2 .
(iii) Calculate the first two values of the partial correlation function φ̂ 1 and φ ˆ 2 .


The following model is proposed to fit the observed data:
X t − μ = a 1 (X t−1 − μ) + e t
where e t is a white noise process with variance σ 2 .
(iv)
Estimate the parameters μ, a 1 and σ 2 in the proposed model.

After fitting the model in part (iv) the 200 observed residual values e ˆ t were
calculated. The number of turning points in the residual series was 110.
(v)
Carry out a statistical test at the 95% significance level to test the hypothesis

that e ˆ t is generated from a white noise process.
[Total 18]
END OF PAPER
CT6 S2014–6

%%%%%%%%%%%%%%%%%%%%%%%%%%%%%%%%%%%%%%%%%%%%%%%%%%%%%%%%%%%%%%%%%%%%

9
(i)
The three main steps are:



Model identification
Parameter estimation
Diagnostic checking

Page 13%%%%%%%%%%%%%%%%%%%%%%%%%%%%%%%%%%%%%%%%%%%% – September 2014 – Examiners’ Report
(ii)
̂ 0 = 35.4
= 0.177
200
̂ 1 = 28.4
= 0.142
200
̂ 2 = 17.1
= 0.0855
200

(iii)
̂ 1 = ̂ 1 =
̂ 2 =
 ˆ 1
0.142
= 0.8023
=
0.177
 ˆ 0
 ˆ 2   ˆ 1 2
1   ˆ 1 2
0.0855
 0.8023 2
= 0.4506
= 0.177
1  0.8023 2

(iv)
83.7
= 0.4185.
200
Firstly ̂ = x =
The Yule-Walker equations for this model give
 0 = a 1  1 +  2
 1 = a 1  0
so we have â 1 =
 ˆ 1
= ̂ 1 = 0.8023
 ˆ 0
and ̂ 2 =  ˆ 0  â 1  ˆ 1 = 0.177  0.8023  0.142 = 0.0631
(v)
The number of turning points T is approximately Normally distributed with
E ( T ) =
2
2
( N  2) =  198 = 132
3
3
Var( T ) =
16 N  29 16  200  29
= 35.2333 = 5.936 2
=
90
90
so a 95% confidence interval for T is
[132  1.96  5.936, 132 + 1.96  5.936] = [120.4, 143.6]
Page 14
%%%%%%%%%%%%%%%%%%%%%%%%%%%%%%%%%%%%%%%%%%%% – September 2014 – Examiners’ Report
We are testing
H 0 : observed e ˆ t are from a white noise process
H 1 : observed e ˆ t are not from a white noise process
Our observed value T = 110 does not lie within the 95% confidence interval.
Therefore we have evidence to reject the H 0 and conclude that the observed e ˆ t
to not come from a white noise process.
A different model is required.

[Total 18]
Full credit was given for considering p-values or significant values and also to candidates
who applied a continuity correction.
Unusually for a time series question this was well answered by many candidates.
END OF EXAMNERS’ REPORT
Page 15
