\documentclass[a4paper,12pt]{article}

%%%%%%%%%%%%%%%%%%%%%%%%%%%%%%%%%%%%%%%%%%%%%%%%%%%%%%%%%%%%%%%%%%%%%%%%%%%%%%%%%%%%%%%%%%%%%%%%%%%%%%%%%%%%%%%%%%%%%%%%%%%%%%%%%%%%%%%%%%%%%%%%%%%%%%%%%%%%%%%%%%%%%%%%%%%%%%%%%%%%%%%%%%%%%%%%%%%%%%%%%%%%%%%%%%%%%%%%%%%%%%%%%%%%%%%%%%%%%%%%%%%%%%%%%%%%

\usepackage{eurosym}
\usepackage{vmargin}
\usepackage{amsmath}
\usepackage{graphics}
\usepackage{epsfig}
\usepackage{enumerate}
\usepackage{multicol}
\usepackage{subfigure}
\usepackage{fancyhdr}
\usepackage{listings}
\usepackage{framed}
\usepackage{graphicx}
\usepackage{amsmath}
\usepackage{chngpage}

%\usepackage{bigints}
\usepackage{vmargin}

% left top textwidth textheight headheight

% headsep footheight footskip

\setmargins{2.0cm}{2.5cm}{16 cm}{22cm}{0.5cm}{0cm}{1cm}{1cm}

\renewcommand{\baselinestretch}{1.3}

\setcounter{MaxMatrixCols}{10}

\begin{document}

Sara is a car mechanic for a racing team. She knows that there is a problem with the
car, but is unsure whether the fault is with the gearbox or the engine. Sara is able to
observe one practice race.
If the underlying problem is with the gearbox there is a 40% chance the car will not
complete the practice race. If the underlying problem is with the engine there is a
90% chance the car will not complete the practice race.
At the end of the practice race Sara must decide, on the basis of whether the car
completes the practice race, whether the fault lies with the gearbox or the engine.
\begin{enumerate}
\item (i)
Write down the four decision functions Sara could adopt.
\medskip 
If Sara correctly identifies the fault there is no cost. The cost of incorrectly deciding
the fault is with the gearbox is £1m. The cost of incorrectly deciding the fault is with
the engine is £5m.
\item (ii)
Show that one of the decision functions is dominated.

The probability that the fault lies with the gearbox is p.
\item (iii)

Determine the range of values of p for which Sara will, under the Bayes
criterion, choose a decision function whose outcome is affected by whether or
not the car completes the practice race.
\end{enumerate}
%%%%%%%%%%%%%%%%%%%%%%%%%%%%%%%%%%%%%%%%%%%%%%%%%%%%%%%%%%%%%%%%%%%%%%%%%%%%%%%%%%%%%%%%%%%%%%%%%%%%%%%%%%

\newpage


3
(i)
The four decision functions are:
d 1 – choose the gearbox regardless
d 2 – choose the gearbox if the car stops and the engine otherwise
d 3 – choose the engine if the car stops and the gearbox otherwise
d 4 – choose the engine regardless
(ii)
Let  1 = state of nature where gearbox is at fault
 2 = state of nature where engine is at fault
Let R(d i ,  j ) = E(L(d i ,  j ))
Then the expected loss matrix is:
 1
 2
d 1 d 2 d 3 d 4
0
1 3
2
0.9 0.1 5
0
where
R(d 2 ,  1 ) = 0.6  5 = 3
R(d 2 ,  2 ) = 0.9  1 = 0.9
R(d 3 ,  1 ) = 0.4  5 = 2
R(d 4 ,  2 ) = 0.1  1 = 0.1
Page 4
%%%%%%%%%%%%%%%%%%%%%%%%%%%%%%%%%%%%%%%%%%%% – September 2014 – Examiners’ Report
It is clear that d 2 is dominated by d 3 .
(iii)

Under Bayes criteria, we need to minimise the expected loss.
Expected losses are
\begin{eqnarray*}
E(L(d 1 )) &=& 0.p + 1.(1  p) = 1  p
\\E(L(d 3 )) &=& 2p + 0.1(1  p) = 1.9p + 0.1\\
E(L(d 4 )) &=& 5p + 0.(1  p) = 5p\\
\end{eqnarray*}

We need to choose p so that d 3 has the lowest expected loss, i.e.
1.9p + 0.1 < 1  p i.e. 2.9p < 0.9 i.e. p < 0.3103
1.9p + 0.1 < 5p i.e. 0.1 < 3.1p i.e. p > 0.03226
and
so we need 0.03226 < p < 0.3103
9 
 1
.


p
  31
29  

[Total 9]
This Bayes’ Criterion question was very disappointingly answered, with only the best
candidates managing to calculate the correct answer to part (iii).
\end{document}
