\documentclass[a4paper,12pt]{article}

%%%%%%%%%%%%%%%%%%%%%%%%%%%%%%%%%%%%%%%%%%%%%%%%%%%%%%%%%%%%%%%%%%%%%%%%%%%%%%%%%%%%%%%%%%%%%%%%%%%%%%%%%%%%%%%%%%%%%%%%%%%%%%%%%%%%%%%%%%%%%%%%%%%%%%%%%%%%%%%%%%%%%%%%%%%%%%%%%%%%%%%%%%%%%%%%%%%%%%%%%%%%%%%%%%%%%%%%%%%%%%%%%%%%%%%%%%%%%%%%%%%%%%%%%%%%

\usepackage{eurosym}
\usepackage{vmargin}
\usepackage{amsmath}
\usepackage{graphics}
\usepackage{epsfig}
\usepackage{enumerate}
\usepackage{multicol}
\usepackage{subfigure}
\usepackage{fancyhdr}
\usepackage{listings}
\usepackage{framed}
\usepackage{graphicx}
\usepackage{amsmath}
\usepackage{chngpage}

%\usepackage{bigints}
\usepackage{vmargin}

% left top textwidth textheight headheight

% headsep footheight footskip

\setmargins{2.0cm}{2.5cm}{16 cm}{22cm}{0.5cm}{0cm}{1cm}{1cm}

\renewcommand{\baselinestretch}{1.3}

\setcounter{MaxMatrixCols}{10}

\begin{document}


5

A particular portfolio of insurance policies gives rise to aggregate claims which
follow a Poisson process with parameter $\lambda = 25$. The distribution of individual claim
amounts is as follows:
\begin{center}
\begin{tabular}{cc}
Claim & Probability\\ \hline 
50& 30\% \\  \hline 
100 & 50\% \\  \hline 
200 & 20\% \\  \hline 
\end{tabular}
\end{center}
The insurer initially has a surplus of 240. Premiums are paid annually in advance.
Calculate approximately the smallest premium loading such that the probability of
ruin in the first year is less than 10%.



%%%%%%%%%%%%%%%%%%%%%%%%%%%
5
\begin{itemize}
\item Mean claim is 50 \times  0.3 + 100 \times  0.5 + 200 \times  0.2
= 15 + 50 + 40
= 105

\item Also
\[E(X^2 ) = 50^2 \times  0.3 + 100^2 \times  0.5 + 200^2 \times  0.2
= 13,750\]
so over 1 year the mean aggregate claim amount is
\[25 \times  105 = 2625\]
and the variance of aggregate claims is
\[25 \times  13,750 = 586.30^2\]

\item Using a Normal approximation we need to find $\theta$ such that
\[P(N(2625, 586.3^2 ) > 240 + 25 \times 105 \times  (1 + \theta)) = 0.1\]
i.e.
P(N(2625, 586.3 2 ) > 240 + 2625(1 + \theta)) = 0.1
i.e. 240 + 2625 \theta ⎞
⎛
P ⎜ N (0,1) >
⎟ = 0.1
586.3
⎝
⎠
so \[240 + 2625 \theta
= 1.2816
586.3\]
i.e. \[\theta =
1.2816 \times  586.3 − 240
2625
= 0.1948\]
\item This question was well answered with many candidates scoring well.
\end{itemize}

\end{document}
