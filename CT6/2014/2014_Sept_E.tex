\documentclass[a4paper,12pt]{article}

%%%%%%%%%%%%%%%%%%%%%%%%%%%%%%%%%%%%%%%%%%%%%%%%%%%%%%%%%%%%%%%%%%%%%%%%%%%%%%%%%%%%%%%%%%%%%%%%%%%%%%%%%%%%%%%%%%%%%%%%%%%%%%%%%%%%%%%%%%%%%%%%%%%%%%%%%%%%%%%%%%%%%%%%%%%%%%%%%%%%%%%%%%%%%%%%%%%%%%%%%%%%%%%%%%%%%%%%%%%%%%%%%%%%%%%%%%%%%%%%%%%%%%%%%%%%

\usepackage{eurosym}
\usepackage{vmargin}
\usepackage{amsmath}
\usepackage{graphics}
\usepackage{epsfig}
\usepackage{enumerate}
\usepackage{multicol}
\usepackage{subfigure}
\usepackage{fancyhdr}
\usepackage{listings}
\usepackage{framed}
\usepackage{graphicx}
\usepackage{amsmath}
\usepackage{chngpage}

%\usepackage{bigints}
\usepackage{vmargin}

% left top textwidth textheight headheight

% headsep footheight footskip

\setmargins{2.0cm}{2.5cm}{16 cm}{22cm}{0.5cm}{0cm}{1cm}{1cm}

\renewcommand{\baselinestretch}{1.3}

\setcounter{MaxMatrixCols}{10}

\begin{document}
\begin{enumerate}

[Total 14]8
Claims on a portfolio of insurance policies follow a Poisson process with rate \lambda.
Individual claim amounts follow a distribution X with mean μ and variance σ 2 . The
insurance company charges premiums of c per policy per year.
(i) Write down the equation satisfied by the adjustment coefficient R.
(ii) Show that R can be approximated by

2( c − μ )
R ˆ = 2
σ + μ 2

Now suppose that individual claims follow a distribution given by
Value
10
Probability 0.3
20
0.5
50
0.15
100
0.05
The insurance company uses a premium loading of 30%. It is considering the
following reinsurance arrangements:
(iii)
(iv)
CT6 S2014–5
A No reinsurance.
B Proportional reinsurance where the insurer retains 70% of all claims with
a reinsurer using a 20% premium loading.
C Excess of loss reinsurance with retention 70 with a reinsurer using a 40%
premium loading.
Determine which arrangement gives the insurance company the lowest
probability of ultimate ruin, using the approximation in part (ii)
Comment on your result in part (iii).
[10]

[Total 17]


%%%%%%%%%%%%%%%%%%%%%%%%%%%%%%%%%%%%%%%%%%%%%%%%%%%%%%%%%%%%%%%%
8
Note: the question should have read “... premiums of c per claim per year”, rather
than “per policy”. This would have meant the equation in (i) simplified to
1 + cR = M x (R).
(i)
R is the solution to
 + ncR =  M x ( R ), where n is the number of policies
Note: Full credit also given for  + cR =  M x ( R ) and 1 + cR = M x ( R )
Note: The solution shown in part (ii) is based on the equation 1 + cR = M x ( R )

(ii)
1 + cR = E ( e XR )


R 2 X 2
 ... 
= E  1  RX 


2


R 2
= 1 + RE ( X ) 
E ( X 2 )  ...
2
Now E(X) = 
and E(X 2 ) = Var(X) + E(X) 2 =  2 +  2 so we have
1 + cR = 1 + R +
R 2 2
(    2 )  ...
2
truncating at the term involving R 2 gives
R ˆ 2 2
1  cR ˆ = 1   R ˆ 
(    2 )
2
Page 11%%%%%%%%%%%%%%%%%%%%%%%%%%%%%%%%%%%%%%%%%%%% – September 2014 – Examiners’ Report
i.e. c =  +
R̂ =
R ˆ 2
(    2 )
2
2( c   )
 2   2

Note: if candidates assumed that  + cR =  M x (R), the alternative correct solution receiving
2  c   
.
full credit is R ˆ 
  2   2


If candidates assumed that  + ncR =  M x (R), the alternative correct solution receiving full
2  nc   
.
credit is R ˆ 
  2   2


For part (iii), most candidates used the formula given in the question, although full credit
was given to candidates who used the alternative formulae above and then correctly worked
through the reinsurance outcomes, whether or not they left their answers in terms of  and n,
or set them to be some sensible value.
(iii)
(A)
We have E(X) = 10  0.3 + 20  0.5 + 50  0.15 + 100  0.05
= 25.5
and E(X 2 ) = 10 2  0.3 + 20 2  0.5 + 50 2  0.15 + 100 2  0.05
= 1105
Here c = 25.5  1.3 = 33.15
and so R̂ =
(B)
2(33.15  25.5)
= 0.013846
1105
We now have  = 0.7  25.5 = 17.85
 2 +  2 = E ((0.7X) 2 ) = 0.7 2  1105 = 541.45
and
c = 33.15  0.3  25.5  1.2
= 33.15  9.18
= 23.97
and so R̂ =
Page 12
2(23.97  17.85)
= 0.02261
541.45%%%%%%%%%%%%%%%%%%%%%%%%%%%%%%%%%%%%%%%%%%%% – September 2014 – Examiners’ Report
(C)
We now have  = 10  0.3 + 20  0.5 + 50  0.15 + 70  0.05
= 24
and  2 +  2 = 10 2  0.3 + 20 2  0.5 + 50 2  0.15 + 70 2  0.05
= 850
the reinsurer charges premiums of 30  0.05  1.4 = 2.1
so c = 33.15  2.1 = 31.05
and R̂ =
2(31.05  24)
= 0.01659
850
The higher the adjustment coefficient the lower the probability of ruin, so
approach B gives the lowest probability of ruin.
[10]
(iv)
It is clear that B is better than A since the reinsurer’s premium loading is
lower than the insurer’s. So we have a 30% reduction in claims but a lower
than 30% reduction in premiums.
The excess of loss reinsurance in C does reduce risk relative to A but not as
much as B. This will be a combination of the relatively high retention and the
reinsurer’s premium loading being higher than the insurer’s.

[Total 17]
Full credit was given for alternative comments reflecting the answers derived by candidates
using the alternative formulae.
Despite the issue with the wording in the question, many candidates scored well on this
question.
