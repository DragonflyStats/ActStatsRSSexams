\documentclass[a4paper,12pt]{article}

%%%%%%%%%%%%%%%%%%%%%%%%%%%%%%%%%%%%%%%%%%%%%%%%%%%%%%%%%%%%%%%%%%%%%%%%%%%%%%%%%%%%%%%%%%%%%%%%%%%%%%%%%%%%%%%%%%%%%%%%%%%%%%%%%%%%%%%%%%%%%%%%%%%%%%%%%%%%%%%%%%%%%%%%%%%%%%%%%%%%%%%%%%%%%%%%%%%%%%%%%%%%%%%%%%%%%%%%%%%%%%%%%%%%%%%%%%%%%%%%%%%%%%%%%%%%

\usepackage{eurosym}
\usepackage{vmargin}
\usepackage{amsmath}
\usepackage{graphics}
\usepackage{epsfig}
\usepackage{enumerate}
\usepackage{multicol}
\usepackage{subfigure}
\usepackage{fancyhdr}
\usepackage{listings}
\usepackage{framed}
\usepackage{graphicx}
\usepackage{amsmath}
\usepackage{chngpage}

%\usepackage{bigints}
\usepackage{vmargin}

% left top textwidth textheight headheight

% headsep footheight footskip

\setmargins{2.0cm}{2.5cm}{16 cm}{22cm}{0.5cm}{0cm}{1cm}{1cm}

\renewcommand{\baselinestretch}{1.3}

\setcounter{MaxMatrixCols}{10}

\begin{document}
8

(i) (a) Write down the Box-Muller algorithm for generating samples from a
standard Normal distribution.
(b) Give an advantage and a disadvantage of the Box-Muller algorithm
relative to the Polar method.

[Total 8]
[3]
(ii)
Extend the algorithm in part (i) to generate samples from a Lognormal
distribution with parameters μ and σ 2 .
[1]
A portfolio of insurance policies contains n independent policies. The probability of a
claim on a policy in a given year is p and the probability of more than one claim is
zero. Claim amounts follow a Lognormal distribution with parameters μ and σ 2 . The
insurance company is interested in estimating the probability θ that aggregate claims
exceed a certain fixed level M.
(iii)
Construct an algorithm to simulate aggregate annual claims from this
portfolio.

The insurance company estimates that θ is around 10%.
(iv)
Calculate the smallest number of simulations the insurance company should
undertake to be able to estimate θ to within 1% with 95% confidence.

The insurance company is considering the impact on θ of entering into a reinsurance
arrangement.
(v)
CT6 A2014–4
Explain whether the insurance company should use the same pseudo random
numbers when simulating the impact of reinsurance.
[1]
[Total 9]
%%%%%%%%%%%%%%%%%%%%%%%%%%%%%%%%%
\newpage


8
(i)
(a)
Let u 1 and u 2 be independent samples from a U (0,1) distribution.
Then Z 1 = − 2 log u 1 cos(2 π u 2 )
Z 2 = − 2 log u 1 sin(2 π u 2 )
are independent standard normal variables.
(b)
Advantage –
generates a sample of every pair of u 1 and u 2 – no
possibility of rejection.
Disadvantage – requires calculation of sin and cos functions which is
more computationally intensive.
(ii)
Generate Z as in (i). Then
Y = exp(μ + σ Z )
is a sample from the required Lognormal distribution.
(iii)
Set
X = 0, k = 0
Step 1 generate a sample u from U (0,1), set k = k + 1
Step 2 If u ≤ p then go to step 3 else go to step 4
Step 3 Generate a sample Y from the Lognormal distribution in (ii) and set
X = X + Y
Step 4 If k = n finish else go to step 1
X represents aggregate claims on the portfolio.
Page 8Subject CT6 (Statistical Methods Core Technical) – April 2014 – Examiners’ Report
(iv)
θ ˆ (1 − θ ˆ )
0.09
=
.
n
n
The standard error will be approximately
We want
i.e.
n >
0.09
× 1.96 < 0.01
n
0.09 × 1.96
= 58.8
0.01
i.e. n > 3457.44
so 3,458 simulations are needed.
(v)
The insurer should use the same pseudo-random numbers so that any variation
in simulation results is due to the impact of the reinsurance and not just due to
random variation in the simulation process.
Parts (i) and (v) were well answered. The remaining parts were found by many candidates to
be the hardest questions on the paper. In part (ii) many candidates could not relate the
Lognormal distribution to the Normal distribution from which samples had been generated in
(i). Only the best candidates attempted parts (iii) and (iv).
\end{document}
