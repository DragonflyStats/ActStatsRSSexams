\documentclass[a4paper,12pt]{article}

%%%%%%%%%%%%%%%%%%%%%%%%%%%%%%%%%%%%%%%%%%%%%%%%%%%%%%%%%%%%%%%%%%%%%%%%%%%%%%%%%%%%%%%%%%%%%%%%%%%%%%%%%%%%%%%%%%%%%%%%%%%%%%%%%%%%%%%%%%%%%%%%%%%%%%%%%%%%%%%%%%%%%%%%%%%%%%%%%%%%%%%%%%%%%%%%%%%%%%%%%%%%%%%%%%%%%%%%%%%%%%%%%%%%%%%%%%%%%%%%%%%%%%%%%%%%

\usepackage{eurosym}
\usepackage{vmargin}
\usepackage{amsmath}
\usepackage{graphics}
\usepackage{epsfig}
\usepackage{enumerate}
\usepackage{multicol}
\usepackage{subfigure}
\usepackage{fancyhdr}
\usepackage{listings}
\usepackage{framed}
\usepackage{graphicx}
\usepackage{amsmath}
\usepackage{chngpage}

%\usepackage{bigints}
\usepackage{vmargin}

% left top textwidth textheight headheight

% headsep footheight footskip

\setmargins{2.0cm}{2.5cm}{16 cm}{22cm}{0.5cm}{0cm}{1cm}{1cm}

\renewcommand{\baselinestretch}{1.3}

\setcounter{MaxMatrixCols}{10}

\begin{document}
\begin{enumerate}

[Total 9]4
As part of a simulation study an actuary is asked to design an algorithm for simulating
claims from a particular type of insurance policy. The probability distribution of the
annual number of claims on a policy is given by:
Probability
No claims One claim Two claims
0.4 0.4 0.2
The claim size distributions of the first and second claims are different. The size of
the first claim follows an exponential distribution with mean 10. The size of the
second claim follows a Weibull distribution with parameters c = 1 and γ = 4.
5
(i) Construct an algorithm to simulate the first claim on a given policy. 
(ii) Construct an algorithm to simulate the second claim on a given policy. 
(iii) Construct an algorithm to simulate the total annual claims on a given policy.

[Total 10]


%%%%%%%%%%%%%%%%%%%%%%%%%%%%%%%%%%%%%%%%%%%%%%%%%%%%%%%%%%%%%%%%%%%%%

The table below shows the incremental claims incurred for a certain portfolio of
insurance policies.
Accident year
2011
2012
2013
Development year
0
1
2
2,233
3,380
4,996
1,389
1,808
600
Cumulative numbers of claims are shown in the following table:
Accident year
2011
2012
2013
Development year
0
1
2
140
180
256
203
230
224
(i) Calculate the outstanding claim reserve for this portfolio using the average
cost per claim method with grossing up factors.
[7]
(ii) State the assumptions underlying the calculations in part (i).
CT6 S2014–3

[Total 10]
%%%%%%%%%%%%%%%%%%%%%%%%%%%%%%%%%%%%%%%%%%%%%%%

4
(i)
Using the inversion method, set
u = F(x) = 1  e


x
10
x
10
i.e. 1  u = e
i.e. x
log(1  u) = 10
i.e. x = 10 log(1  u)
so the algorithm is:
Step 1
Step 2
Generate a sample u from a U(0,1) distribution.
Set x = 10 log(1  u).

Page 5%%%%%%%%%%%%%%%%%%%%%%%%%%%%%%%%%%%%%%%%%%%% – September 2014 – Examiners’ Report
(ii)
Again using the inversion method, set
 x
u = F(x) = 1  e
i.e.  x
1  u = e
i.e. x 4 = log(1  u)
i.e. x = [log(1  u)] 1⁄4
4
4
so the algorithm is
Step 1
Step 2
Generate a sample u from a U(0,1) distribution.
Set x = [log(1  u)] 1⁄4 .

(iii)
Our algorithm is as follows:
Step 1 Generate a sample u from a U(0,1) distribution.
Step 2 If 0  u  0.4 then total claim amount X = 0 and stop
Else continue to step 3.
Step 3 If 0.4 < u 0.8 then simulate a claim from Exp(1/10)
distribution using the algorithm in (i) and set X = this value
and stop.
Else go to step 4.
Step 4
Simulate claims using the algorithms in (i) and (ii) and set
X = total of the two simulated claims.

[Total 10]
This question was well answered.
5
(i)
The cumulative cost of claims is given by:
Accident year
2011
2012
2013
Page 6
Development year
0
1
2
2,233
3,380
4,996
3,622 4222
5,188%%%%%%%%%%%%%%%%%%%%%%%%%%%%%%%%%%%%%%%%%%%% – September 2014 – Examiners’ Report
Dividing by cumulative claim numbers:
Accident year
Development year
0
1
2
2011
2012
2013
15.950
18.778
19.516
17.842
22.557
18.848
using grossing up factors to estimate the ultimate average cost per claim for
each accident year:
Accident
year
2011
2012
2013
0
Development year
1
84.623%
15.950
78.805%
18.778
81.714%
19.516
94.663%
17.842
94.663%
22.557
2
100%
18.848
23.828
23.883
Taking the same approach for the claim numbers gives:
Accident
year
2011
2012
2013
Development year
0
1
62.5%
140
70.924%
180
66.712%
256
90.625%
203
90.625%
230
2
100%
224
253.8
383.7
Total outstanding claims are therefore
253.8  23.828 + 383.7  23.883  5188  4996
= 5028.2
(ii)
[7]
Assumptions
 The number of claims relating to each development year is a constant
proportion of the total claim numbers from the relevant accident year.
 Claim amounts for each development year are a constant proportion of the
total claim amount for the relevant accident year.
 Claims are fully run off after development year 2.

[Total 10]
Page 7%%%%%%%%%%%%%%%%%%%%%%%%%%%%%%%%%%%%%%%%%%%% – September 2014 – Examiners’ Report
Alternative valid points received full credit. This question was well answered, although many
candidates scored poorly on part (ii).
