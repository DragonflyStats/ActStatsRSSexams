\documentclass[a4paper,12pt]{article}

%%%%%%%%%%%%%%%%%%%%%%%%%%%%%%%%%%%%%%%%%%%%%%%%%%%%%%%%%%%%%%%%%%%%%%%%%%%%%%%%%%%%%%%%%%%%%%%%%%%%%%%%%%%%%%%%%%%%%%%%%%%%%%%%%%%%%%%%%%%%%%%%%%%%%%%%%%%%%%%%%%%%%%%%%%%%%%%%%%%%%%%%%%%%%%%%%%%%%%%%%%%%%%%%%%%%%%%%%%%%%%%%%%%%%%%%%%%%%%%%%%%%%%%%%%%%

\usepackage{eurosym}
\usepackage{vmargin}
\usepackage{amsmath}
\usepackage{graphics}
\usepackage{epsfig}
\usepackage{enumerate}
\usepackage{multicol}
\usepackage{subfigure}
\usepackage{fancyhdr}
\usepackage{listings}
\usepackage{framed}
\usepackage{graphicx}
\usepackage{amsmath}
\usepackage{chngpage}

%\usepackage{bigints}
\usepackage{vmargin}

% left top textwidth textheight headheight

% headsep footheight footskip

\setmargins{2.0cm}{2.5cm}{16 cm}{22cm}{0.5cm}{0cm}{1cm}{1cm}

\renewcommand{\baselinestretch}{1.3}

\setcounter{MaxMatrixCols}{10}

\begin{document}




%%%%%%%%%%%%%%%%%%%%%%%%%%%%%%%%%%%%%%%%%%%%%%%%%%%%%%%%%%%%%%%%%%%%%

The table below shows the incremental claims incurred for a certain portfolio of
insurance policies.
Accident year
2011
2012
2013
Development year
0
1
2
2,233
3,380
4,996
1,389
1,808
600
Cumulative numbers of claims are shown in the following table:
Accident year
2011
2012
2013
Development year
0
1
2
140
180
256
203
230
224
(i) Calculate the outstanding claim reserve for this portfolio using the average
cost per claim method with grossing up factors.
[7]
(ii) State the assumptions underlying the calculations in part (i).
CT6 S2014–3

[Total 10]
%%%%%%%%%%%%%%%%%%%%%%%%%%%%%%%%%%%%%%%%%%%%%%%
\newpage
5
(i)
The cumulative cost of claims is given by:
Accident year
2011
2012
2013
Page 6
Development year
0
1
2
2,233
3,380
4,996
3,622 4222
5,188%%%%%%%%%%%%%%%%%%%%%%%%%%%%%%%%%%%%%%%%%%%% – September 2014 – Examiners’ Report
Dividing by cumulative claim numbers:
Accident year
Development year
0
1
2
2011
2012
2013
15.950
18.778
19.516
17.842
22.557
18.848
using grossing up factors to estimate the ultimate average cost per claim for
each accident year:
Accident
year
2011
2012
2013
0
Development year
1
84.623%
15.950
78.805%
18.778
81.714%
19.516
94.663%
17.842
94.663%
22.557
2
100%
18.848
23.828
23.883
Taking the same approach for the claim numbers gives:
Accident
year
2011
2012
2013
Development year
0
1
62.5%
140
70.924%
180
66.712%
256
90.625%
203
90.625%
230
2
100%
224
253.8
383.7
Total outstanding claims are therefore
253.8  23.828 + 383.7  23.883  5188  4996
= 5028.2
(ii)
[7]
Assumptions
 The number of claims relating to each development year is a constant
proportion of the total claim numbers from the relevant accident year.
 Claim amounts for each development year are a constant proportion of the
total claim amount for the relevant accident year.
 Claims are fully run off after development year 2.

[Total 10]
Page 7%%%%%%%%%%%%%%%%%%%%%%%%%%%%%%%%%%%%%%%%%%%% – September 2014 – Examiners’ Report
Alternative valid points received full credit. This question was well answered, although many
candidates scored poorly on part (ii).
\end{document}
