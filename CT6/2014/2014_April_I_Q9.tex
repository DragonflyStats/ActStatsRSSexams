\documentclass[a4paper,12pt]{article}

%%%%%%%%%%%%%%%%%%%%%%%%%%%%%%%%%%%%%%%%%%%%%%%%%%%%%%%%%%%%%%%%%%%%%%%%%%%%%%%%%%%%%%%%%%%%%%%%%%%%%%%%%%%%%%%%%%%%%%%%%%%%%%%%%%%%%%%%%%%%%%%%%%%%%%%%%%%%%%%%%%%%%%%%%%%%%%%%%%%%%%%%%%%%%%%%%%%%%%%%%%%%%%%%%%%%%%%%%%%%%%%%%%%%%%%%%%%%%%%%%%%%%%%%%%%%

\usepackage{eurosym}
\usepackage{vmargin}
\usepackage{amsmath}
\usepackage{graphics}
\usepackage{epsfig}
\usepackage{enumerate}
\usepackage{multicol}
\usepackage{subfigure}
\usepackage{fancyhdr}
\usepackage{listings}
\usepackage{framed}
\usepackage{graphicx}
\usepackage{amsmath}
\usepackage{chngpage}

%\usepackage{bigints}
\usepackage{vmargin}

% left top textwidth textheight headheight

% headsep footheight footskip

\setmargins{2.0cm}{2.5cm}{16 cm}{22cm}{0.5cm}{0cm}{1cm}{1cm}

\renewcommand{\baselinestretch}{1.3}

\setcounter{MaxMatrixCols}{10}

\begin{document}

[Total 9]9
The table below sets out incremental claims data for a portfolio of insurance policies.
Accident year
2011
2012
2013
Development year
0
1
2
1,403
1,718
1,912
535
811
142
Past and projected future inflation is given by the following index (measured to the
mid point of the relevant year).
Year Index
2011
2012
2013
2014
2015 100
107
110
113
117
Estimate the outstanding claims using the inflation adjusted chain ladder technique.
[9]



%%%%%%%%%%%%%%%%%%%%%%%%%%%%%%%%%%%%%%%%%%%%%%%%%%%%%%%%%%%%%%%%%%%%%%%%%%%%%%%%%%%%%%%%%%%%%%%%%%












9
Incremental claims in mid 2013 prices are given by:
Accident year
0
2011
2012
2013
Development year
1
1543.3
1766.17
1912
550
811
2
142
Cumulative claims in mid 2013 prices:
Accident year
0
2011
2012
2013
Development year
1
1543.3
1766.17
1912
2093.3
2577.17
2
2235.3
DF 0,1 = (2093.3 + 2577.17) / (1543.3 + 1766.17) = 1.4112441
DF 1,2 = 2235.3 / 2093.3 = 1.067835
Page 9%%%%%%%%%%%%%%%%%%%%%%%%%%%%%%%%%%%%%%%%%%%5 – April 2014 – 
Completed cumulative claims
Accident year
0
Development year
1
2011
2012
2013
2698.30
2
2751.99
2881.34
Incremental claims (mid 2013 prices)
Accident year
0
Development year
1
2011
2012
2013
786.30
2
174.82
183.04
And projecting for inflation, outstanding claims = (174.82 + 786.30) ×
+ 183.04 ×
117
= 1182.02
110
This question was well answered, with many candidates scoring full marks.
\end{document}
