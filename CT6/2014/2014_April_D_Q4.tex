\documentclass[a4paper,12pt]{article}

%%%%%%%%%%%%%%%%%%%%%%%%%%%%%%%%%%%%%%%%%%%%%%%%%%%%%%%%%%%%%%%%%%%%%%%%%%%%%%%%%%%%%%%%%%%%%%%%%%%%%%%%%%%%%%%%%%%%%%%%%%%%%%%%%%%%%%%%%%%%%%%%%%%%%%%%%%%%%%%%%%%%%%%%%%%%%%%%%%%%%%%%%%%%%%%%%%%%%%%%%%%%%%%%%%%%%%%%%%%%%%%%%%%%%%%%%%%%%%%%%%%%%%%%%%%%

\usepackage{eurosym}
\usepackage{vmargin}
\usepackage{amsmath}
\usepackage{graphics}
\usepackage{epsfig}
\usepackage{enumerate}
\usepackage{multicol}
\usepackage{subfigure}
\usepackage{fancyhdr}
\usepackage{listings}
\usepackage{framed}
\usepackage{graphicx}
\usepackage{amsmath}
\usepackage{chngpage}

%\usepackage{bigints}
\usepackage{vmargin}

% left top textwidth textheight headheight

% headsep footheight footskip

\setmargins{2.0cm}{2.5cm}{16 cm}{22cm}{0.5cm}{0cm}{1cm}{1cm}

\renewcommand{\baselinestretch}{1.3}

\setcounter{MaxMatrixCols}{10}

\begin{document}
\begin{enumerate}
4
Individual claim amounts on a portfolio of motor insurance policies follow a Gamma
distribution with parameters \alpha and \lambda. It is known that \lambda = 3 for all drivers, but the
parameter \alpha varies across the population. 70% of drivers have \alpha = 300 and the
remaining 30% have \alpha = 600.
Claims on the portfolio follow a Poisson process with annual rate 500 and the
likelihood of a claim arising is independent of the parameter \alpha.
Calculate the mean and variance of aggregate annual claims on the portfolio.


%%%%%%%%%%%%%%%%%%%%%%%%%%%%%%%%%%%%%%%%%%%%%%%
\newpage

%%%%%%%%%%%%%%%%%%%%%%%%%%%%%%%%%
4
E(X i )
= E(E(X i |\alpha))
⎛ \alpha ⎞ 1
= E ⎜ ⎟ = E ( \alpha )
⎝ \lambda ⎠ 3
=
1
1
(0.7 × 300 + 0.3 × 600) = × 390
3
3
= 130
Var(X i ) = Var(E(X i |\alpha)) + E(Var(X i |\alpha))
⎛ \alpha ⎞
⎛ \alpha ⎞
= Var ⎜ ⎟ + E ⎜ 2 ⎟
⎝ \lambda ⎠
⎝ \lambda ⎠
=
=
1
\lambda
2
Var( \alpha ) +
\lambda 2
E ( \alpha )
1
1
(0.7 × 300 2 + 0.3 × 600 2 − 390 2 ) + × 390
9
9
= 2100 +
Page 4
1
390
= 2143.33
9%%%%%%%%%%%%%%%%%%%%%%%%%%%%%%%%%%%%%%%%%%%5 – April 2014 – 
so overall
E(S) = \lambda E(X) = 500 × 130 = 65,000
Var(S) = \lambda E(X 2 ) = 500 × (2143.33 + 130 2 )
= 9,521,665
There are other approaches which can be taken to calculating the variance, all of which were
given full credit. Whilst most candidates were able to calculate the mean only the better
candidates were able to accurately calculate the variance.
\end{document}
