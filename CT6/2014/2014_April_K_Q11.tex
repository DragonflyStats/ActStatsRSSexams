\documentclass[a4paper,12pt]{article}

%%%%%%%%%%%%%%%%%%%%%%%%%%%%%%%%%%%%%%%%%%%%%%%%%%%%%%%%%%%%%%%%%%%%%%%%%%%%%%%%%%%%%%%%%%%%%%%%%%%%%%%%%%%%%%%%%%%%%%%%%%%%%%%%%%%%%%%%%%%%%%%%%%%%%%%%%%%%%%%%%%%%%%%%%%%%%%%%%%%%%%%%%%%%%%%%%%%%%%%%%%%%%%%%%%%%%%%%%%%%%%%%%%%%%%%%%%%%%%%%%%%%%%%%%%%%

\usepackage{eurosym}
\usepackage{vmargin}
\usepackage{amsmath}
\usepackage{graphics}
\usepackage{epsfig}
\usepackage{enumerate}
\usepackage{multicol}
\usepackage{subfigure}
\usepackage{fancyhdr}
\usepackage{listings}
\usepackage{framed}
\usepackage{graphicx}
\usepackage{amsmath}
\usepackage{chngpage}

%\usepackage{bigints}
\usepackage{vmargin}

% left top textwidth textheight headheight

% headsep footheight footskip

\setmargins{2.0cm}{2.5cm}{16 cm}{22cm}{0.5cm}{0cm}{1cm}{1cm}

\renewcommand{\baselinestretch}{1.3}

\setcounter{MaxMatrixCols}{10}

\begin{document}

%%--- Question 11
Let $\theta$ denote the proportion of insurance policies in a certain portfolio on which a
claim is made. Prior beliefs about $\theta$ are described by a Beta distribution with
parameters $\alpha$ and $\beta$ .
Underwriters are able to estimate the mean \mu and variance \sigma 2 of $\theta$.

\begin{enumerate}[(a)]
\item (i)
Express $\alpha$ and $\beta$   in terms of \mu and \sigma.

A random sample of n policies is taken and it is observed that claims had arisen on d
of them.
\item (ii)
12
(a) Determine the posterior distribution of $\theta$.
(b) Show that the mean of the posterior distribution can be written in the
form of a credibility estimate.

\item (iii) Show that the credibility factor increases as \sigma increases.
\item (iv) Comment on the result in part (iii).
\end{enumerate}

%%%%%%%%%%%%%%%%%%%%%%%%%%%%%%%%%%%%%%%%%%%%%%%%%%%%%%%%%%%%%%%%%%%%%%%%%%%%
\newpage
\begin{itemize}
\item  (i)
We have \mu =
and
\sigma 2 =
\alpha
\alpha+\beta 
so \beta  =
\alpha\beta 
2
( \alpha + \beta  ) ( \alpha + \beta  + 1)
substituting for ߙ ൅ ߚ ൌ
\sigma 2 =
so
(ii)
\sigma 2 =
\alpha (1 − \mu )
\mu
ఈ
ఓ
\beta 
\mu (1 − \mu )
=
( \alpha + \beta  )( \alpha + \beta  + 1) \alpha + \beta  + 1
= \mu ×
gives
\mu (1 − \mu )
\alpha
+ 1
\mu
\mu 2 (1 − \mu )
\alpha+\mu
\mu 2 (1 − \mu ) − \mu\sigma 2
so \alpha =
and \beta  =

\item 
(a) f (\theta| x ) \alpha f ( x |\theta) f (\theta)
\sigma 2
( \mu 2 (1 − \mu ) − \mu\sigma 2 )(1 − \mu )
\mu\sigma 2
=
( \mu (1 − \mu ) − \sigma 2 )(1 − \mu )
\sigma 2
\alpha \theta d (1 − \theta) n − d \theta \alpha− 1 (1 − \theta) \beta − 1
\alpha \theta \alpha + d − 1 (1 − \theta) n − d+ \beta − 1
which is the pdf of a Beta distribution with parameters \alpha + d and
\beta  + n − d .
\item  (b)
The posterior mean is given by
\alpha+ d
\alpha+ d
=
\alpha + d +\beta + n − d
\alpha +\beta + n
where Z =

n
.
\alpha +\beta + n
= \alpha
\alpha+\beta 
d
n
×
+ ×
\alpha +\beta  \alpha +\beta + n n \alpha +\beta + n
= \alpha
d
× (1 − Z ) + Z
n
\alpha +\beta %%%%%%%%%%%%%%%%%%%%%%%%%%%%%%%%%%%%%%%%%%%5 – April 2014 – 
\item This is in the form of a credibility estimate since
the prior distribution and
\item  (iii)
Z =
\alpha
is the mean of
\alpha+\beta 
d
is the MLE.
n
n
decreases as $\alpha + \beta$  increases and increases as $\alpha + \beta$  decreases.
\alpha +\beta + n
From (i) \alpha + \beta  =
=
=
\mu 2 (1 − \mu ) − \mu\sigma 2 + \mu (1 − \mu ) 2 − (1 − \mu ) \sigma 2
\sigma 2
\mu (1 − \mu )( \mu + 1 − \mu ) − \mu\sigma 2 − \sigma 2 + \mu\sigma 2
\sigma 2
\mu (1 − \mu )
\sigma 2
− 1
so increasing \sigma 2 decreases \alpha + \beta  and increases Z .
\item (iv)
Higher $\sigma^2$ implies less certainty in the prior estimate / prior is less reliable and
so should lead to more weight on the observed data – which it does via a
higher Z .
Only the best candidates were able to complete the algebra in part (i). Many candidates
nevertheless scored well on parts (ii) and (iv).
\end{itemize}
\end{document}
