\documentclass[a4paper,12pt]{article}

%%%%%%%%%%%%%%%%%%%%%%%%%%%%%%%%%%%%%%%%%%%%%%%%%%%%%%%%%%%%%%%%%%%%%%%%%%%%%%%%%%%%%%%%%%%%%%%%%%%%%%%%%%%%%%%%%%%%%%%%%%%%%%%%%%%%%%%%%%%%%%%%%%%%%%%%%%%%%%%%%%%%%%%%%%%%%%%%%%%%%%%%%%%%%%%%%%%%%%%%%%%%%%%%%%%%%%%%%%%%%%%%%%%%%%%%%%%%%%%%%%%%%%%%%%%%

\usepackage{eurosym}
\usepackage{vmargin}
\usepackage{amsmath}
\usepackage{graphics}
\usepackage{epsfig}
\usepackage{enumerate}
\usepackage{multicol}
\usepackage{subfigure}
\usepackage{fancyhdr}
\usepackage{listings}
\usepackage{framed}
\usepackage{graphicx}
\usepackage{amsmath}
\usepackage{chngpage}

%\usepackage{bigints}
\usepackage{vmargin}

% left top textwidth textheight headheight

% headsep footheight footskip

\setmargins{2.0cm}{2.5cm}{16 cm}{22cm}{0.5cm}{0cm}{1cm}{1cm}

\renewcommand{\baselinestretch}{1.3}

\setcounter{MaxMatrixCols}{10}

\begin{document}

7
The random variable X follows a Pareto distribution with parameters $\alpha$  and $\lambda$.
\begin{enumerate}
\item (i)
Show that for L, d > 0
L + d
∫ d
xf ( x ) dx =
λ\alpha  ⎡ d\alpha  +\lambda \alpha  ( L + d ) +\lambda  ⎤
−
⎢
⎥
α − 1 ⎣ (\lambda  + d )\alpha  (\lambda  + L + d )\alpha  ⎦
[5]
Claims on a certain type of motor insurance policy follow a Pareto distribution with
mean 16,000 and standard deviation 20,000. The insurance company has an excess of
loss reinsurance policy with a retention level of 40,000 and a maximum amount paid
by the reinsurer of 25,000.
\item (ii)
Determine the mean claim amount paid by the reinsurer on claims that involve
the reinsurer.
[8]
Claim amounts increase by 5%.
\itemg(iii)
CT6 S2014–4
State the new distribution of claim amounts.
\end{enumerate}

\newpage
%%%%%%%%%%%%%%%%%%%%%%%%%%%%7
(i)


\begin{eqnarray*}
\int^{L+d}_{d} xf(x) dx
&=& \lambda^{\alpha} \left[ \frac{d}{(\lambda + d)^\alpha } \;+\; \frac{}{}  \right] \\
&=& \lambda^{\alpha} \left[ \frac{d}{(\lambda + d)^\alpha } \frac{}{} \right] \\
&=& \lambda^{\alpha} \left[ \frac{d}{(\lambda + d)^\alpha } \frac{}{} \right] \\
&=& \lambda^{\alpha} \left[ \frac{}{} \right] \\
\end{eqnarray*}
L  d
 d
xf ( x ) dx
=
\alpha  x
L  d
 d
( \alpha  x )  1
dx
L  d

\alpha  x 
=  
 
  ( \alpha  x )   d
 
\alpha 
L  d
d
( \alpha  x ) 
dx
L  d

 d
( L  d )  
\alpha 



= \alpha 




( \alpha  L  d )     (   1)( \alpha  x )  1   d
 ( \alpha  d )

 d

( L  d )
1
1



= \alpha  


 1
 1 
( \alpha  L  d )
(   1)( \alpha  L  d )
(   1)( \alpha  d ) 
 ( \alpha  d )
  d (   1)  ( \alpha  d ) ( L  d )(   1)  ( \alpha  L  d )  
= \alpha  



(   1)( \alpha  L  d ) 
  (   1)( \alpha  d )
 
=
\alpha    d   \alpha  ( L  d )  \alpha  


 .
  1   ( \alpha  d )  ( \alpha  L  d )   
[5]
(ii)
We first solve for the parameter values:
\alpha
= 16,000
  1
\alpha 2
= 20,000 2
(   1) (   2)
2
2
so 
 \alpha 
2

 = 20,000
  2    1 
so 
20, 000 2
=
= 1.5625
  2 16, 000 2
so  = 1.5625(   2)
so  = 2 
1.5625
= 5.555
0.5625
Page 9Subject CT6 (Statistical Methods Core Technical) – September 2014 – Examiners’ Report
and
\alpha = 16,000  (   1)
= 72,888.89
Now denote by Z the amount paid by the reinsurer.
Then P ( Z > 0) = P ( X > 40,000) = 1  F (40,000)
\alpha


= 

 \alpha  40, 000 

 72,888.89 
= 

 112,888.89 
5.5555
= 0.088004
Now
E ( Z )
65000
( x  40000) f ( x ) dx  

=  40000
=  40000 xf ( x ) dx  40000  40000 f ( x ) dx  25000 P ( X  65000)
65000
72888.89 5.5555
4.5555

65000
25000 f ( x ) dx
65000
 40000  5.5555  72888.89

112888.89 5.5555

65000  5.5555  72888.89 
  40000( F (65000)  F (40000))
137888.89 5.5555

 25000(1  F (65000))
  72888.89  5.5555  72888.89  5.5555 

= 2941.71  40000  
 

  112888.89  

137888.89




 72888.89 
 25000  

 137888.89 
5.5555
= 2941.71  2361.67 + 724.10
= 1304.14
and so E [ Z  Z > 0] =
1304.14
= 14,819.10
0.088004
[8]
\newpage 
(iii)
Pareto with parameters  = 5.5555 and \alpha = 72,888.89  1.05
= 76,533.33
[1]
[Total 14]
Along with question 3, candidates typically found this the hardest question on the paper to
answer. Although many candidates were able to calculate the parameters in part (ii), only
the better candidates were able to work through the integration and calculate the final result.
\end{document}
