\documentclass[a4paper,12pt]{article}

%%%%%%%%%%%%%%%%%%%%%%%%%%%%%%%%%%%%%%%%%%%%%%%%%%%%%%%%%%%%%%%%%%%%%%%%%%%%%%%%%%%%%%%%%%%%%%%%%%%%%%%%%%%%%%%%%%%%%%%%%%%%%%%%%%%%%%%%%%%%%%%%%%%%%%%%%%%%%%%%%%%%%%%%%%%%%%%%%%%%%%%%%%%%%%%%%%%%%%%%%%%%%%%%%%%%%%%%%%%%%%%%%%%%%%%%%%%%%%%%%%%%%%%%%%%%

\usepackage{eurosym}
\usepackage{vmargin}
\usepackage{amsmath}
\usepackage{graphics}
\usepackage{epsfig}
\usepackage{enumerate}
\usepackage{multicol}
\usepackage{subfigure}
\usepackage{fancyhdr}
\usepackage{listings}
\usepackage{framed}
\usepackage{graphicx}
\usepackage{amsmath}
\usepackage{chngpage}

%\usepackage{bigints}
\usepackage{vmargin}

% left top textwidth textheight headheight

% headsep footheight footskip

\setmargins{2.0cm}{2.5cm}{16 cm}{22cm}{0.5cm}{0cm}{1cm}{1cm}

\renewcommand{\baselinestretch}{1.3}

\setcounter{MaxMatrixCols}{10}

\begin{document}
\begin{enumerate}

Institute and Faculty of Actuaries1
2
(i) List six of the characteristics that insurable risks usually have. 
(ii) List two key characteristics of a short term insurance contract. 
[Total 4]
Ruth takes the bus to school every morning. The bus company’s ticket machine is
unreliable and the amount Ruth is charged every morning can be regarded as a
random variable with mean 2 and non-zero standard deviation. The bus company
does offer a “value ticket” which gives a 50% discount in return for a weekly payment
of 5 in advance. There are 5 days in a week and Ruth walks home each day.
Ruth’s mother is worried about Ruth not having enough money to pay for her ticket
and is considering three approaches to paying for bus fares:
A Give Ruth 10 at the start of each week.
B Give Ruth 2 at the start of each day.
C Buy the 50% discount card at the start of the week and then give Ruth 1 at the
start of each day.
Determine the approach that will give the lowest probability of Ruth running out of
money.

3
The table below shows the payoff to a player from a decision problem with three
uncertain states of nature \theta 1 , \theta 2 and \theta 3 and four possible decisions D 1 , D 2 , D 3 and D 4 .
\theta 1
\theta 2
\theta 3
D 1
10
−5
−8
D 2
3
12
−3
D 3
−7
6
13
D 4
9
−7
−10
(i) Determine whether any of the decisions are dominated. 
(ii) Determine the optimal decision using the minimax criteria. 
Now suppose P(\theta 1 ) = 0.5 and P(\theta 2 ) = 0.3 and P(\theta 3 ) = 0.2.
(iii)
CT6 A2014–2
Determine the optimal decision under the Bayes criterion.

\newpage


%%%%%%%%%%%%%%%%%%%%%%%%%%%%%%%%%%%%%%%%%%%%%%%%%%%%%%%%%%%%%%%%%%%%%%%%%%%%%%%%%%%%%%%%5 – April 2014 – 
1
(i)
•
•
•
•
•
•
•
•
• policyholder has an interest in the risk
risk is of a financial nature and reasonably qualifiable
independence of risks
probability of event is relatively small
pool large numbers of potentially similar risks
ultimate limit on liability of insurer
moral hazards eliminated as far as possible
claim amount must bear some relationship to financial loss
sufficient data to reasonably estimate extent of risk / likelihood of
occurence
•
•
•
•
•
• policy lasts for a fixed term
policy lasts for a relatively short period of time
policyholder pays a premium
insurer pays claims that arise during the policy term
option (but no obligation) to renew policy
claim does not bring policy to an end
(ii)
Other sensible points received full credit. This question was generally well answered.
2
A is better than B since Ruth has a capital buffer at the start of the week which can
offset later journeys, whereas under B a high fare on Monday causes Ruth to run out
of funds.
B and C are the same – the net funds available under C are always exactly 1⁄2 of those
available under B.
So overall A gives the lowest probability of running out of cash.
Many candidates did not attempt this question which required a qualitative analysis of the
situation set out. Those candidates who had a good understanding of the basic principles
underlying the material on ruin theory were able to score well.
\end{document}
