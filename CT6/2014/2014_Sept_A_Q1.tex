\documentclass[a4paper,12pt]{article}

%%%%%%%%%%%%%%%%%%%%%%%%%%%%%%%%%%%%%%%%%%%%%%%%%%%%%%%%%%%%%%%%%%%%%%%%%%%%%%%%%%%%%%%%%%%%%%%%%%%%%%%%%%%%%%%%%%%%%%%%%%%%%%%%%%%%%%%%%%%%%%%%%%%%%%%%%%%%%%%%%%%%%%%%%%%%%%%%%%%%%%%%%%%%%%%%%%%%%%%%%%%%%%%%%%%%%%%%%%%%%%%%%%%%%%%%%%%%%%%%%%%%%%%%%%%%

\usepackage{eurosym}
\usepackage{vmargin}
\usepackage{amsmath}
\usepackage{graphics}
\usepackage{epsfig}
\usepackage{enumerate}
\usepackage{multicol}
\usepackage{subfigure}
\usepackage{fancyhdr}
\usepackage{listings}
\usepackage{framed}
\usepackage{graphicx}
\usepackage{amsmath}
\usepackage{chngpage}

%\usepackage{bigints}
\usepackage{vmargin}

% left top textwidth textheight headheight

% headsep footheight footskip

\setmargins{2.0cm}{2.5cm}{16 cm}{22cm}{0.5cm}{0cm}{1cm}{1cm}

\renewcommand{\baselinestretch}{1.3}

\setcounter{MaxMatrixCols}{10}

\begin{document}
\begin{enumerate}

© Institute and Faculty of Actuaries1
An insurance company has a portfolio of 240 insurance policies. The probability of a
claim on the i th policy in a year is p i independently from policy to policy and there is
no possibility of more than one claim. Claim amounts on the i th policy follow an
exponential distribution with mean 100 .
p i
Let X denote the aggregate annual claims on the portfolio.
2
3
Determine the mean and variance of X. [6]
(i) List the three main components of a generalised linear model. 
(ii) Explain what is meant by a saturated model and discuss whether such a model
is useful in practice.


%%%%%%%%%%%%%%%%%%%%%%%%%%%%%%%%%%%%%%%%%%%%%%%%%%%%%%%%%%%%%%%%%%%%%%%%%%%%%%%%%%%%%%%%%%%%%%%%%%%%%%%%%%

\newpage


1
E(X)
=
 p i  E ( X i )
i  1
240
=
 p i 
i  1
100
p i
240
=
 100
i 1
= 24000
Let Y i denote the claim in the i th policy. Then
with probability 1  p i
 0

Y i = 
 p i 
 Exp  100  with probability p i



= p i 
100
= 100
p i
so E(Y i )
and E ( Y i 2 ) = p i  2 
so Var(Y i ) =
100 2
p i 2
=
20000
p i
20000
 100 2
p i
 2

= 10, 000   1 
 p i

 2  p i 
= 10, 000 

 p i 
240
and
Var(X) =
 2  p i 

 p i 
 10, 000 
i  1
240
 2  p i 
= 10, 000  

i  1  p i 
[6]
Full credit was also given to candidates who used standard individual risk model results.
Many candidates scored well here although a disappointing number struggled to derive the
variance.
Page 3%%%%%%%%%%%%%%%%%%%%%%%%%%%%%%%%%%%%%%%%%%%% – September 2014 – Examiners’ Report
\end{document}
