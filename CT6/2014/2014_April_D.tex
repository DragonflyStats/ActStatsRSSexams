\documentclass[a4paper,12pt]{article}

%%%%%%%%%%%%%%%%%%%%%%%%%%%%%%%%%%%%%%%%%%%%%%%%%%%%%%%%%%%%%%%%%%%%%%%%%%%%%%%%%%%%%%%%%%%%%%%%%%%%%%%%%%%%%%%%%%%%%%%%%%%%%%%%%%%%%%%%%%%%%%%%%%%%%%%%%%%%%%%%%%%%%%%%%%%%%%%%%%%%%%%%%%%%%%%%%%%%%%%%%%%%%%%%%%%%%%%%%%%%%%%%%%%%%%%%%%%%%%%%%%%%%%%%%%%%

\usepackage{eurosym}
\usepackage{vmargin}
\usepackage{amsmath}
\usepackage{graphics}
\usepackage{epsfig}
\usepackage{enumerate}
\usepackage{multicol}
\usepackage{subfigure}
\usepackage{fancyhdr}
\usepackage{listings}
\usepackage{framed}
\usepackage{graphicx}
\usepackage{amsmath}
\usepackage{chngpage}

%\usepackage{bigints}
\usepackage{vmargin}

% left top textwidth textheight headheight

% headsep footheight footskip

\setmargins{2.0cm}{2.5cm}{16 cm}{22cm}{0.5cm}{0cm}{1cm}{1cm}

\renewcommand{\baselinestretch}{1.3}

\setcounter{MaxMatrixCols}{10}

\begin{document}
\begin{enumerate}
[Total 9]9
The table below sets out incremental claims data for a portfolio of insurance policies.
Accident year
2011
2012
2013
Development year
0
1
2
1,403
1,718
1,912
535
811
142
Past and projected future inflation is given by the following index (measured to the
mid point of the relevant year).
Year Index
2011
2012
2013
2014
2015 100
107
110
113
117
Estimate the outstanding claims using the inflation adjusted chain ladder technique.
[9]
10
For a certain portfolio of insurance policies the number of claims on the i th policy in
the j th year of cover is denoted by Y ij . The distribution of Y ij is given by
P(Y ij = y) = \theta ij (1 − \theta ij ) y
y = 0, 1, 2, ...
where 0 ≤ \theta ij ≤ 1 are unknown parameters with i = 1, 2, ..., k and j = 1, 2, ..., l.
(i) Derive the maximum likelihood estimate of \theta ij given the single observed data
point y ij .

(ii) Write P(Y ij = y) in exponential family form and specify the parameters.
(iii) Describe the different characteristics of Pearson and deviance residuals.

[Total 10]
CT6 A2014–5


%%%%%%%%%%%%%%%%%%%%%%%%%%%%%%%%%%%%%%%%%%%%%%%%%%%%%%%%%%%%%%%%%%%%%%%%%%%%%%%%%%%%%%%%%%%%%%%%%%












9
Incremental claims in mid 2013 prices are given by:
Accident year
0
2011
2012
2013
Development year
1
1543.3
1766.17
1912
550
811
2
142
Cumulative claims in mid 2013 prices:
Accident year
0
2011
2012
2013
Development year
1
1543.3
1766.17
1912
2093.3
2577.17
2
2235.3
DF 0,1 = (2093.3 + 2577.17) / (1543.3 + 1766.17) = 1.4112441
DF 1,2 = 2235.3 / 2093.3 = 1.067835
Page 9%%%%%%%%%%%%%%%%%%%%%%%%%%%%%%%%%%%%%%%%%%%5 – April 2014 – 
Completed cumulative claims
Accident year
0
Development year
1
2011
2012
2013
2698.30
2
2751.99
2881.34
Incremental claims (mid 2013 prices)
Accident year
0
Development year
1
2011
2012
2013
786.30
2
174.82
183.04
And projecting for inflation, outstanding claims = (174.82 + 786.30) ×
+ 183.04 ×
117
= 1182.02
110
This question was well answered, with many candidates scoring full marks.
10
(i)
The likelihood is given by
L = \theta ij (1 − \theta ij )
y ij
Taking logs gives
l = log L = log \theta ij + y ij log(1 − \theta ij )
Differentiating with respect to \theta ij gives
y ij
1
∂ l
=
−
∂\theta ij \theta ij (1 − \theta ij )
and setting
∂ l
= 0 we have
∂\theta ij
y ij
1
=
\theta ˆ ij 1 − \theta ˆ ij
Page 10
113
110%%%%%%%%%%%%%%%%%%%%%%%%%%%%%%%%%%%%%%%%%%%5 – April 2014 – 
so 1 − \theta ˆ ij = y ij \theta ˆ ij
so 1 = (1 + y ij ) \theta ˆ ij
i.e. \theta ˆ ij =
and since
1
1 + y ij
∂ 2 l
∂\theta ij 2
=−
1
\theta ij 2
−
y ij
(1 − \theta ij ) 2
< 0
(since y ij > 0)
we do have a maximum.
(ii)
P ( Y ij = y ) = \theta ij (1 − \theta ij ) y
= exp[log\theta ij + y log(1 − \theta ij )]
= exp[ y log(1 − \theta ij )] + log\theta ij ]
⎡ y \theta − b ( \theta )
⎤
= exp ⎢
+ c ( y , \phi ) ⎥
⎣ a ( \phi )
⎦
where \theta
= log(1 − \theta ij ) is the natural parameter
b (\theta) = −log\theta ij = −log[1 − e \theta ]
\phi = 1
a (\phi) = 1
c ( y , \phi) = 0
(iii)
The Pearson residuals are often skewed for non normal data which makes the
interpretation of residual plots difficult.
Deviance residuals are usually more likely to be symmetrically distributed and
are preferred for actuarial applications.
This question was, for the most part, answered well. A common mistake in part (i) was to try
to sum across either years or policies when the question specifically referred to a single data
point.
