\documentclass[a4paper,12pt]{article}

%%%%%%%%%%%%%%%%%%%%%%%%%%%%%%%%%%%%%%%%%%%%%%%%%%%%%%%%%%%%%%%%%%%%%%%%%%%%%%%%%%%%%%%%%%%%%%%%%%%%%%%%%%%%%%%%%%%%%%%%%%%%%%%%%%%%%%%%%%%%%%%%%%%%%%%%%%%%%%%%%%%%%%%%%%%%%%%%%%%%%%%%%%%%%%%%%%%%%%%%%%%%%%%%%%%%%%%%%%%%%%%%%%%%%%%%%%%%%%%%%%%%%%%%%%%%

\usepackage{eurosym}
\usepackage{vmargin}
\usepackage{amsmath}
\usepackage{graphics}
\usepackage{epsfig}
\usepackage{enumerate}
\usepackage{multicol}
\usepackage{subfigure}
\usepackage{fancyhdr}
\usepackage{listings}
\usepackage{framed}
\usepackage{graphicx}
\usepackage{amsmath}
\usepackage{chngpage}

%\usepackage{bigints}
\usepackage{vmargin}

% left top textwidth textheight headheight

% headsep footheight footskip

\setmargins{2.0cm}{2.5cm}{16 cm}{22cm}{0.5cm}{0cm}{1cm}{1cm}

\renewcommand{\baselinestretch}{1.3}

\setcounter{MaxMatrixCols}{10}

\begin{document}


A sequence of 100 observations was made from a time series and the following values of the sample auto-covariance function (SACF) were observed:
Lag
1
2
3
4
SACF
0.68
0.55
0.30
0.06
The sample mean and variance of the same observations are 1.35 and 0.9 respectively.
\begin{enumerate}
\item (i) Calculate the first two values of the partial correlation function \phî 1 and \hat{\phi} 2 . 
\item (ii) Estimate the parameters (including $\sigma^2$ ) of the following models which are to
be fitted to the observed data and can be assumed to be stationary.
(a)
(b)
Y t = a 0 + a 1 Y t−1 + e t
Y t = a 0 + a 1 Y t−1 + a 2 Y t−2 + e t
In each case e t is a white noise process with variance \sigma 2 .
\item 
(iii) Explain whether the assumption of stationarity is necessary for the estimation
for each of the models in part (ii).
\item 
(iv) Explain whether each of the models in part (ii) satisfies the Markov property.
\end{itemize}
\newpage
%%%%%%%%%%%%%%%%%%%%%%%%%%%%%%%%%%%%%%%%

12
(i)
0.68
\hat{\phi} 1 = ρ ˆ 1 =
= 0.755556
0 .9
0.55
− 0.755556 2
2
ˆ
ˆ
ρ
−
ρ
0.9
2
1
\hat{\phi} 2 =
=
= 0.093786
1 − ρ ˆ 1 2
1 − 0.755556 2
(ii)
(a)
The Yule-Walker equations for this model give
γ 0 = a 1 γ 1 + \sigma 2
γ 1 = a 1 γ 0
so we have a ˆ 1 =
γ ˆ 1
= ρ ˆ 1 = 0. 75 55 56
γ ˆ 0
Page 13%%%%%%%%%%%%%%%%%%%%%%%%%%%%%%%%%%%%%%%%%%%5 – April 2014 – 
and \sigma ˆ 2 = γ ˆ 0 − a ˆ 1 γ ˆ 1 = γ ˆ 0 (1 − a ˆ 1 ρ ˆ 1 )
= 0.9 − 0.755556 × 0.68 = 0.38622
Finally we let \mu = E ( Y t ) and observe that
\mu = a 0 + a 1 \mu
so \mu =
a 0
1 − a 1
so a ˆ 0 = \hat{\mu} ( 1 − a ˆ 1 ) = 1.35 ( 1 − 0.755556 ) = 0.33 0 0 0
(b)
For this model the Yule-Walker equations are
γ 0 = a 1 γ 1 + a 2 γ 2 + \sigma 2
γ 1 = a 1 γ 0 + a 2 γ 1
γ 2 = a 1 γ 1 + a 2 γ 0
(1)
(2)
(3)
substituting the observed values in (2) and (3) gives
0.68 = 0.9 a ˆ 1 + 0.6 8 a ˆ 2
0.55 = 0.68 a ˆ 1 + 0 .9 a ˆ 2
(4)
(5)
(
(4) ×0.68 − (5) × 0.9 ⇒ 0.68 2 − 0.55 × 0.9 = a ˆ 2 0.6 8 2 − 0 . 9 2
So a ˆ 2 =
)
− 0.0326
= 0.09379
− 0. 3476
And a ˆ 1 =
0.68 ( 1 − 0.09378596 )
0. 9
= 0.68470
substituting into (1)
\sigmâ 2 = 0.9 − 0.68470 × 0.68 − 0.09379 × 0.55 = 0.38283
and finally setting \mu = E ( Y t ) we have \mu = a 0 + a 1 \mu + a 2 \mu
so \mu =
a 0
1 − a 1 − a 2
so a ˆ 0 = \hat{\mu} × ( 1 − a ˆ 1 − a ˆ 2 ) = 1.35 ( 1 − 0.68470 − 0.09379 ) = 0.29905

(iii) Stationarity is necessary for both models since the Yule-Walker equations do not hold without the existence of the auto-covariance function.
(iv) Model (a) does satisfy the Markov property since the current value depends only on the previous value. This does not hold for Model (b).


%Most candidates were able to derive the Yule-Walker equations and therefore scored marks
%on this question. Only the best candidates were able to use these equations to derive
%numerical values of the parameters. Part (iv) was generally well answered.
%Although the question stated that the given values were for the auto-covariance function,
%many candidates calculated as if the given values came from the auto-correlation function.
%The Examiners noted that the core reading does use the abbreviation ACF for the auto-
%correlation function, and therefore gave full credit to candidates who interpreted the question
%in this way. 
The numerical values of the estimated parameters taking this approach are as
follows:
(i)
\hat{\phi} 1 = 0.68
\hat{\phi} 2 = 0. 16 29
(ii)
(a)
a ˆ 0 = 0 .4 32
a ˆ 1 = 0.68
\sigma ˆ 2 = 0. 483 84
(b)
a ˆ 0 = 0. 36 17
a ˆ 1 = 0. 56 92
a ˆ 2 = 0. 16 29
\sigma ˆ 2 = 0. 47 10


\end{document}
