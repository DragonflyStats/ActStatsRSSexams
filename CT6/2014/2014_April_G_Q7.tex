\documentclass[a4paper,12pt]{article}

%%%%%%%%%%%%%%%%%%%%%%%%%%%%%%%%%%%%%%%%%%%%%%%%%%%%%%%%%%%%%%%%%%%%%%%%%%%%%%%%%%%%%%%%%%%%%%%%%%%%%%%%%%%%%%%%%%%%%%%%%%%%%%%%%%%%%%%%%%%%%%%%%%%%%%%%%%%%%%%%%%%%%%%%%%%%%%%%%%%%%%%%%%%%%%%%%%%%%%%%%%%%%%%%%%%%%%%%%%%%%%%%%%%%%%%%%%%%%%%%%%%%%%%%%%%%

\usepackage{eurosym}
\usepackage{vmargin}
\usepackage{amsmath}
\usepackage{graphics}
\usepackage{epsfig}
\usepackage{enumerate}
\usepackage{multicol}
\usepackage{subfigure}
\usepackage{fancyhdr}
\usepackage{listings}
\usepackage{framed}
\usepackage{graphicx}
\usepackage{amsmath}
\usepackage{chngpage}

%\usepackage{bigints}
\usepackage{vmargin}

% left top textwidth textheight headheight

% headsep footheight footskip

\setmargins{2.0cm}{2.5cm}{16 cm}{22cm}{0.5cm}{0cm}{1cm}{1cm}

\renewcommand{\baselinestretch}{1.3}

\setcounter{MaxMatrixCols}{10}

\begin{document}
\begin{enumerate}

[Total 8]
%%--- Question 7
The heights of adult males in a certain population are Normally distributed with
unknown mean \mu and standard deviation \sigma = 15.
Prior beliefs about \mu are described by a Normal distribution with mean 187 and
standard deviation 10.
(i)
Calculate the prior probability that \mu is greater than 180.

A sample of 80 men is taken and the mean height is found to be 182.
8
(ii) Calculate the posterior probability that \mu is greater than 180.

(iii) Comment on your results from parts (i) and (ii).
(i) (a) Write down the Box-Muller algorithm for generating samples from a
standard Normal distribution.
(b) Give an advantage and a disadvantage of the Box-Muller algorithm
relative to the Polar method.

[Total 8]

(ii)
Extend the algorithm in part (i) to generate samples from a Lognormal
distribution with parameters \mu and \sigma 2 .

A portfolio of insurance policies contains n independent policies. The probability of a
claim on a policy in a given year is p and the probability of more than one claim is
zero. Claim amounts follow a Lognormal distribution with parameters \mu and \sigma 2 . The
insurance company is interested in estimating the probability \theta that aggregate claims
exceed a certain fixed level M.
(iii)
Construct an algorithm to simulate aggregate annual claims from this
portfolio.

The insurance company estimates that \theta is around 10%.
(iv)
Calculate the smallest number of simulations the insurance company should
undertake to be able to estimate \theta to within 1% with 95% confidence.

The insurance company is considering the impact on \theta of entering into a reinsurance
arrangement.
(v)
CT6 A2014–4
Explain whether the insurance company should use the same pseudo random
numbers when simulating the impact of reinsurance.


7
(i)
P(\mu > 180) = P(N(187, 10 2 ) > 180)
180 − 187 ⎞
⎛
= P ⎜ N (0,1) >
⎟
10
⎝
⎠
= P(N(0,1) > − 0.7)
= 0.75804
(ii)
We know that \mu|x ~ N ( \mu * , \sigma * 2 )
1 ⎞
⎛ 80 × 182 187 ⎞ ⎛ 80
Where \mu * = ⎜
+ 2 ⎟ ⎜ 2 + 2 ⎟ = 182.14
2
10 ⎠ ⎝ 15 10 ⎠
⎝ 15
And \sigma * 2 =
so
1
80
1
+ 2
2
15 10
= 2.73556 = 1.6540 2
P (\mu > 180) = P ( N (182.14, 1.654 2 ) > 180)
180 − 182.14 ⎞
⎛
= P ⎜ N (0,1) >
⎟
1.654
⎝
⎠
= P ( N (0,1) > -1.29192)
= 0.38 × 0.9032 + 0.62 × 0.90147
= 0.90180
(iii)
The probability has risen, reflecting our much greater certainty over the value
of \mu as a result of taking a large sample.
Page 7%%%%%%%%%%%%%%%%%%%%%%%%%%%%%%%%%%%%%%%%%%%5 – April 2014 – 
This is despite the fact that our mean belief about \mu has fallen, which a priori
might make a lower value of \mu more likely.
The posterior distribution has thinner tails / lower volatility, since we have
increased credibility around the mean
This question was mostly well answered. A small number of candidates were not aware that
the formulae for the Normal / Normal model are given in the tables, and therefore struggled
with the algebra required to derive the posterior distribution.
\end{document}
