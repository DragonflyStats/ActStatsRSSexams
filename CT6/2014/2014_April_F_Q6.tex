\documentclass[a4paper,12pt]{article}

%%%%%%%%%%%%%%%%%%%%%%%%%%%%%%%%%%%%%%%%%%%%%%%%%%%%%%%%%%%%%%%%%%%%%%%%%%%%%%%%%%%%%%%%%%%%%%%%%%%%%%%%%%%%%%%%%%%%%%%%%%%%%%%%%%%%%%%%%%%%%%%%%%%%%%%%%%%%%%%%%%%%%%%%%%%%%%%%%%%%%%%%%%%%%%%%%%%%%%%%%%%%%%%%%%%%%%%%%%%%%%%%%%%%%%%%%%%%%%%%%%%%%%%%%%%%

\usepackage{eurosym}
\usepackage{vmargin}
\usepackage{amsmath}
\usepackage{graphics}
\usepackage{epsfig}
\usepackage{enumerate}
\usepackage{multicol}
\usepackage{subfigure}
\usepackage{fancyhdr}
\usepackage{listings}
\usepackage{framed}
\usepackage{graphicx}
\usepackage{amsmath}
\usepackage{chngpage}

%\usepackage{bigints}
\usepackage{vmargin}

% left top textwidth textheight headheight

% headsep footheight footskip

\setmargins{2.0cm}{2.5cm}{16 cm}{22cm}{0.5cm}{0cm}{1cm}{1cm}

\renewcommand{\baselinestretch}{1.3}

\setcounter{MaxMatrixCols}{10}

\begin{document}



6
Claim amounts arising from a certain type of insurance policy are believed to follow a
Lognormal distribution. One thousand claims are observed and the following
summary statistics are prepared:
mean claim amount 230
standard deviation 110
lower quartile
80
upper quartile
510
\begin{enumerate}
\item (i)
Fit a Lognormal distribution to these claims using:
(a)
(b)
the method of moments.
the method of percentiles.
\item 
(ii)
CT6 A2014–3
Compare the fitted distributions from part (i).
\end{enumerate}


\newpage
6
(i)
(a)
Let the parameters of the Lognormal distribution be \mu and \sigma.
Then we must solve
e
\mu+
\sigma 2
2
= 230
2
(A)
2
e 2 \mu+\sigma ( e \sigma − 1) = 110 2
(B) ÷
(A) 2
⇒ e
2
\sigma 2
110 2
− 1 =
110 2
(B)
230 2
so e \sigma = 1 +
so \sigma 2 = log 1.22873 = 0.205984
230 2
= 1.22873
so \sigma = 0.45385
Substituting into (A) gives
e
\mu+
0.205984
2
= 230
\mu = log(230) −
0.205984
2
= 5.3351
(b)
This time we have
e \mu+ 0.6745 \sigma = 510 (A)
e \mu− 0.6745 \sigma = 80 (B)
logA + logB ⇒ 2\mu = log510 + log80
so \mu = 5.30822
and substituting into (A)
5.30822 + 0.6745\sigma = log510
\sigma =
Page 6
log 510 − 5.30822
= 1.37315
0.6745%%%%%%%%%%%%%%%%%%%%%%%%%%%%%%%%%%%%%%%%%%%5 – April 2014 – 
(ii)
Calculating the upper and lower quartiles from the parameter in (i)(a) gives \[UQ = e 5.3351+0.6745×0.45385 = 282 cf 510\]
\[LQ = e 5.3351−0.6745×0.45385 = 153 cf 80\]
This is not a good fit, suggesting the underlying claims have greater weight in
the tails than a Lognormal distribution.
Most candidates were able to apply the method of moments in part (i) but many struggled to apply the method of percentiles. In particular, it was clear that many candidates could not relate the lognormal distribution back to the underlying normal distribution (this was also a common issue in Q8). Alternative comments on the data were given credit in part (ii).
\end{itemize}
\end{document}
