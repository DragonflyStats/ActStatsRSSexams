10
For a certain portfolio of insurance policies the number of claims on the i th policy in
the j th year of cover is denoted by Y ij . The distribution of Y ij is given by
P(Y ij = y) = \theta  ij (1 − \theta  ij ) y
y = 0, 1, 2, ...
where 0 \leq  \theta  ij \leq  1 are unknown parameters with i = 1, 2, ..., k and j = 1, 2, ..., l.
(i) Derive the maximum likelihood estimate of \theta  ij given the single observed data
point y ij .
[4]
(ii) Write P(Y ij = y) in exponential family form and specify the parameters.
(iii) Describe the different characteristics of Pearson and deviance residuals.
[2]
[Total 10]

%%%%%%%%%%%%%%%%%%%%%%%%%%%%%%%%%%%%%%%%%%%%%%%%%%%%%%%%%%%%%%%%%%%%%%%%%%%%%%%%
\newpage

10
(i)
The likelihood is given by
L = \theta  ij (1 − \theta  ij )
y ij
Taking logs gives
l = log L = log \theta  ij + y ij log(1 − \theta  ij )
Differentiating with respect to \theta  ij gives
y ij
1
∂ l
=
−
∂\theta  ij \theta  ij (1 − \theta  ij )
and setting
∂ l
= 0 we have
∂\theta  ij
y ij
1
=
\theta  ˆ ij 1 − \theta  ˆ ij
Page 10
113
110Subject CT6 (Statistical Methods Core Technical) – April 2014 – Examiners’ Report
so 1 − \theta  ˆ ij = y ij \theta  ˆ ij
so 1 = (1 + y ij ) \theta  ˆ ij
i.e. \theta  ˆ ij =
and since
1
1 + y ij
∂ 2 l
∂\theta  ij 2
=−
1
\theta  ij 2
−
y ij
(1 − \theta  ij ) 2
< 0
(since y ij > 0)
we do have a maximum.
(ii)
P ( Y ij = y ) = \theta  ij (1 − \theta  ij ) y
= exp[log\theta  ij + y log(1 − \theta  ij )]
= exp[ y log(1 − \theta  ij )] + log\theta  ij ]
⎡ y \theta  − b ( \theta  )
⎤
= exp ⎢
+ c ( y , φ ) ⎥
⎣ a ( φ )
⎦
where \theta 
= log(1 − \theta  ij ) is the natural parameter
b (\theta ) = −log\theta  ij = −log[1 − e \theta  ]
φ = 1
a (φ) = 1
c ( y , φ) = 0
(iii)
The Pearson residuals are often skewed for non normal data which makes the
interpretation of residual plots difficult.
Deviance residuals are usually more likely to be symmetrically distributed and
are preferred for actuarial applications.
This question was, for the most part, answered well. A common mistake in part (i) was to try
to sum across either years or policies when the question specifically referred to a single data
point.
