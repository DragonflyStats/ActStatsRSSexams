\documentclass[a4paper,12pt]{article}

%%%%%%%%%%%%%%%%%%%%%%%%%%%%%%%%%%%%%%%%%%%%%%%%%%%%%%%%%%%%%%%%%%%%%%%%%%%%%%%%%%%%%%%%%%%%%%%%%%%%%%%%%%%%%%%%%%%%%%%%%%%%%%%%%%%%%%%%%%%%%%%%%%%%%%%%%%%%%%%%%%%%%%%%%%%%%%%%%%%%%%%%%%%%%%%%%%%%%%%%%%%%%%%%%%%%%%%%%%%%%%%%%%%%%%%%%%%%%%%%%%%%%%%%%%%%

\usepackage{eurosym}
\usepackage{vmargin}
\usepackage{amsmath}
\usepackage{graphics}
\usepackage{epsfig}
\usepackage{enumerate}
\usepackage{multicol}
\usepackage{subfigure}
\usepackage{fancyhdr}
\usepackage{listings}
\usepackage{framed}
\usepackage{graphicx}
\usepackage{amsmath}
\usepackage{chng%%-- Page}

%\usepackage{bigints}
\usepackage{vmargin}

% left top textwidth textheight headheight

% headsep footheight footskip

\setmargins{2.0cm}{2.5cm}{16 cm}{22cm}{0.5cm}{0cm}{1cm}{1cm}

\renewcommand{\baselinestretch}{1.3}

\setcounter{MaxMatrixCols}{10}

\begin{document}

The number of claims, N, in a year on a portfolio of insurance policies has a Poisson
distribution with parameter \lambda. Claims are either large (with probability p) or small
(with probability 1 - p) independently of one another.
Suppose we observe r large claims. Show that the conditional distribution of N − r | r
is Poisson and find its mean.
[7]

Page 3%%%%%%%%%%%%%%%%%%%%%%5 — September 2\theta \theta 7 — Examiners’ Report
3
P ( N − r = k r big claims) = P ( N = r + k ) \times  P ( of r + k claims k are small) / P ( r big claims)
but
∞
P ( r big claims) = \sum  e −\lambda
j = \theta 
\lambda r + j
( r + j )! r
\times 
p (1 − p ) j
( r + j )! r ! j !
\lambda r r ∞ −\lambda \lambda j
=
p \times  \sum  e
(1 − p ) j
r !
j !
j = \theta 
=
\lambda r r −\lambda \lambda (1 − p )
p e e
r !
So
P ( N − r = k r big claims) = P ( N = r + k ) \times  P ( of r + k claims k are small) / P ( r big claims)
e −\lambda
=
\lambda r + k
( r + k )! r
\times 
p (1 − p ) k
( r + k )! r ! k !
e −\lambda e \lambda (1 − p ) p r \lambda r / r !
= e −\lambda (1 − p )
\lambda k (1 − p ) k
k !
which is a probability from a Poisson distribution with parameter \lambda (1 − p ) . Hence
conditional mean of N - r is \lambda (1 − p ) .
Page 4%%%%%%%%%%%%%%%%%%%%%%5 — September 2\theta \theta 7 — Examiners’ Report
