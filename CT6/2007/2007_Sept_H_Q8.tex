PLEASE TURN OVER8
The total claim amount, S, on a portfolio of insurance policies has a compound
Poisson distribution with Poisson parameter 50. Individual loss amounts have an
exponential distribution with mean 75. However, the terms of the policies mean that
the maximum sum payable by the insurer in respect of a single claim is 100.
(i) Find E(S) and Var (S).
(ii) Use the method of moments to fit as an approximation to S:
(a)
(b)
[7]
a normal distribution
a log-normal distribution
[3]
(iii)
9
For each fitted distribution, calculate P(S > 3,000).
[3]
[Total 13]
%%%%%%%%%%%%%%%%%%%%%%%%%%%%%%%%%%%%%%%%%%%
\newpage
8
(i)
Let the individual loss amounts have distribution X . Then
100
E ( X ) =
∫ 0.01333 xe
− 0.01333 x
dx + 100 \times  P ( X > 100)
0
100
= ⎡ − xe − 0.01333 x ⎤ +
⎣
⎦ 0
100
∫ e
− 0.01333 x
∞
dx + 100
0
∫ 0.01333 e
− 0.01333 x
dx
100
100
∞
= − 100 e − 1.333 + ⎡ − 75 e − 0.01333 x ⎤ + 100 ⎡ − e − 0.01333 x ⎤
⎣
⎦ 0
⎣
⎦ 100
= − 100 e − 1.333 − 75 e − 1.333 + 75 + 100 e − 1.333
= 55.2302
Hence E ( S ) = 50 \times  55.2302 = 2761.5
Page 9%%%%%%%%%%%%%%%%%%%%%%5 — September 2007 — Examiners’ Report
E ( X 2 ) =
100
2 − 0.01333 x
∫ 0.01333 x e
dx + 100 2 P ( X > 100)
0
100
= ⎡ − x 2 e − 0.01333 x ⎤ +
⎣
⎦ 0
2 − 1.333
= − 100 e
100
∫ 2 xe
− 0.01333 x
dx + 100 2 e − 1.333
0
100
2 x
⎡
⎤
e − 0.01333 x ⎥ +
+ ⎢ −
⎣ 0.01333
⎦ 0
100
∫
0
2
e − 0.01333 x dx + 100 2 e − 1.333
0.01333
100
200
2
⎡
⎤
=−
e − 1.333 + ⎢ −
e − 0.01333 x ⎥
2
0.01333
⎣ 0.01333
⎦ 0
=−
200
2
2
e − 1.333 −
e − 1.333 +
2
0.01333
0.01333
0.01333 2
= 4330.6
and so
Var ( S ) = 50 \times  4330.6 = 216529 = (465.33) 2
(ii)
(a) The normal distribution is N (2761.5, 465.33 2 )
(b) The Log-Normal distribution has parameters \mu and \sigma with
E ( S ) = e \mu+\sigma
2
/2
2
2
2
Var ( S ) = e 2 \mu+\sigma ( e \sigma − 1) = E ( S ) 2 \times  ( e \sigma − 1)
So substituting gives
2
216529 = 2761.5 2 \times  ( e \sigma − 1)
2
e \sigma =
216529
2761.5 2
+ 1 = 1.028394
\sigma 2 = log(1.028394) = 0.027998
\sigma = 0.167327
And now we can substitute for \sigma to give
Page 10%%%%%%%%%%%%%%%%%%%%%%5 — September 2007 — Examiners’ Report
2761.5 = e \mu+ 0.027998 / 2
\mu = log(2761.5) − 0.027998 / 2 = 7.90953
(iii)
Using the Normal distribution:
3000 − 2761.5 ⎞
⎛
P ( N (2761.5, 465.33 2 ) > 3000) = P ⎜ N (0,1) >
⎟
465.33
⎝
⎠
= P ( N (0,1) > 0.51) = 1 − 0.69497 = 0.30503
From tables.
Using the log-normal distribution,
P (log N (7.90953, 0.167327 2 ) > 3000) = P ( N (7.90953, 0.167327 2 ) > log(3000))
= P ( N (0,1) >
log 3000 − 7.90953
).
0.167327
= P ( N (0,1) > 0.58) = 1 − 0.71904 = 0.28096
n
