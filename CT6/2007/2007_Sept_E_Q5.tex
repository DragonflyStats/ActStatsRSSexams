\documentclass[a4paper,12pt]{article}

%%%%%%%%%%%%%%%%%%%%%%%%%%%%%%%%%%%%%%%%%%%%%%%%%%%%%%%%%%%%%%%%%%%%%%%%%%%%%%%%%%%%%%%%%%%%%%%%%%%%%%%%%%%%%%%%%%%%%%%%%%%%%%%%%%%%%%%%%%%%%%%%%%%%%%%%%%%%%%%%%%%%%%%%%%%%%%%%%%%%%%%%%%%%%%%%%%%%%%%%%%%%%%%%%%%%%%%%%%%%%%%%%%%%%%%%%%%%%%%%%%%%%%%%%%%%

\usepackage{eurosym}
\usepackage{vmargin}
\usepackage{amsmath}
\usepackage{graphics}
\usepackage{epsfig}
\usepackage{enumerate}
\usepackage{multicol}
\usepackage{subfigure}
\usepackage{fancyhdr}
\usepackage{listings}
\usepackage{framed}
\usepackage{graphicx}
\usepackage{amsmath}
\usepackage{chng%%-- Page}

%\usepackage{bigints}
\usepackage{vmargin}

% left top textwidth textheight headheight

% headsep footheight footskip

\setmargins{2.0cm}{2.5cm}{16 cm}{22cm}{0.5cm}{0cm}{1cm}{1cm}

\renewcommand{\baselinestretch}{1.3}

\setcounter{MaxMatrixCols}{10}

\begin{document}

[8]
5
Aggregate annual claims on a portfolio of insurance policies have a compound
Poisson distribution with parameter \lambda. Individual claim amounts have an exponential
distribution with mean 1.
The insurer calculates premiums using a loading of \alpha (so that the annual premium is
1+ \alpha times the annual expected claims) and has initial surplus U.
(i) Show that if the first claim occurs at time t, the probability that this claim
causes ruin is e − U e − (1 +\alpha ) \lambda t .

(ii) Show that the probability of ruin on the first claim is
(iii) Show that if the insurer wishes to set \alpha such that the probability of ruin at the

first claim is less than 1% then it must choose \alpha > 100 e − U − 2 .
[Total 9]

e − U
.
2 +\alpha
[4]

%%%%%%%%%%%%%%%%%%%%%%%%%%%%%%%%%%%%%%%%%%%
5
(i)
Let X be the size of the first claim, so that X has an exponential distribution
with parameter 1. Then for ruin to occur at time t we require
X > U + (1 + \alpha) \lambda t.
∞
P(X > U + (1 + \alpha) \lambda t) =
e − x dx
∫
U + (1 +\alpha ) \lambda t
∞
= ⎡ − e − x ⎤
⎣
⎦ U + (1 +\alpha ) \lambda t
= e - U e - (1+ \alpha ) \lambda t .
[Note that it would be acceptable to quote the cumulative distribution function
for the exponential distribution from the tables rather than calculate the
integral]
(ii)
Let T denote the time until the first claim. Then T has an exponential
distribution with parameter \lambda and
∞
P(Ruin at first claim) =
∫ P (Ruin at first claim ⏐ first claim is at t ) × f T ( t ) dt
0
∞
= ∫ e − U e − ( 1 + \alpha ) \lambda t \lambda e − \lambda t dt
0
∞
=
∫ e
− U
\lambda e − (2 +\alpha ) \lambda t dt
0
∞
⎡
⎤
\lambda
= ⎢ − e − U
e − (2 +\alpha ) \lambda t ⎥
(2 + \alpha ) \lambda
⎣
⎦ 0
=
Page 6
e − U
.
2 +\alpha

%%%%%%%%%%%%%%%%%%%%%%%%%%%%%%%%%%%%%%%%%%%%%%%%%%%%%%%%%%%%555

Subject CT6 (Statistical Methods Core Technical) — September 2007 — Examiners’ Report
(iii)
We require
e − U
< 0.01
2 +\alpha
i.e. e - U < 0.01 × (2 + \alpha)
i.e. 100e - U < 2 + \alpha.
i.e. 100e - U – 2 < \alpha.
