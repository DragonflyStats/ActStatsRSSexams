\documentclass[a4paper,12pt]{article}

%%%%%%%%%%%%%%%%%%%%%%%%%%%%%%%%%%%%%%%%%%%%%%%%%%%%%%%%%%%%%%%%%%%%%%%%%%%%%%%%%%%%%%%%%%%%%%%%%%%%%%%%%%%%%%%%%%%%%%%%%%%%%%%%%%%%%%%%%%%%%%%%%%%%%%%%%%%%%%%%%%%%%%%%%%%%%%%%%%%%%%%%%%%%%%%%%%%%%%%%%%%%%%%%%%%%%%%%%%%%%%%%%%%%%%%%%%%%%%%%%%%%%%%%%%%%

\usepackage{eurosym}
\usepackage{vmargin}
\usepackage{amsmath}
\usepackage{graphics}
\usepackage{epsfig}
\usepackage{enumerate}
\usepackage{multicol}
\usepackage{subfigure}
\usepackage{fancyhdr}
\usepackage{listings}
\usepackage{framed}
\usepackage{graphicx}
\usepackage{amsmath}
\usepackage{chngpage}

%\usepackage{bigints}
\usepackage{vmargin}

% left top textwidth textheight headheight

% headsep footheight footskip

\setmargins{2.0cm}{2.5cm}{16 cm}{22cm}{0.5cm}{0cm}{1cm}{1cm}

\renewcommand{\baselinestretch}{1.3}

\setcounter{MaxMatrixCols}{10}

\begin{document}

\begin{enumerate}

%%%%%%%%%%%%%%%%%%%%%%%%%%%%%%%%%%%%%%%%%%%
10
(i)
The Gamma distribution with mean \mu  and variance \mu  2 /\alpha  has density function
\[f(y) =
\alpha  \alpha 
\mu  \alpha  \Gamma ( \alpha  )
y
\alpha − 1
e
−
y \alpha 
\mu 
( y > 0 )\]
(a) Show that this may be written in the form of an exponential family.
(b) Use the properties of exponential families to confirm that the mean and
[9]
variance of the distribution are \mu  and \mu  2 /\alpha .
(ii) Explain the difference between a continuous covariate and a factor.

(iii) A company is analysing its claims data on a portfolio of motor policies, and uses a gamma distribution to model the claim severities. The company uses
three rating factors:
policyholder age (as a continuous variable);
policyholder gender;
vehicle rating group (as a factor).
(a) Write down the form of the linear predictor when all rating factors are included as main effects.
(b) State how the linear predictor changes if an interaction between
policyholder age and gender is included.
\newpage
%%%%%%%%%%%%%%%%%%%%%%%%%%%%%%%%%%%%%%%%%%%%%%%%%%%%%%%%%%%%%%%%%%%%%%%%%%%%%%%%%%%%%%%%%%%%%%%

(i) (a)
⎡ y \alpha 
⎤
f(y) = exp ⎢ −
− \alpha  log \mu  + ( \alpha  − 1) log y + \alpha  log \alpha  − log \Gamma ( \alpha  ) ⎥
⎣ \mu 
⎦
⎡ ⎛ y
⎤
⎞
= exp ⎢ \alpha  ⎜ − − log \mu  ⎟ + ( \alpha  − 1) log y + \alpha  log \alpha  − log \Gamma ( \alpha  ) ⎥
⎠
⎣ ⎝ \mu 
⎦
which is in the form of an exponential family.
\theta  = −
1
\mu 
b(\theta ) = log \mu 
⎛ 1 ⎞
= log ⎜ − ⎟ = - log( - \theta )
⎝ \theta  ⎠
%%--- Page 10
%%-- Subject CT6 (Statistical Methods Core Technical) — April 2007 — Examiners’ Report
(b)
The mean and variance of the distribution are b ′ ( \theta  ) and a ( \phi  ) b ′′ ( \theta  )
b ′ ( \theta  ) = −
1
= \mu 
\theta 
which confirms that the mean is \mu .
b ′′ ( \theta  ) =
a(\phi ) =
1
\theta 
2
= \mu  2
1
\alpha 
hence the variance is
(ii)
1 2
\mu  , as required
\alpha 
A factor is categorical e.g. male/female.
For a continuous covariate, the value is included. For example, if x is a continuous covariate, a main effect would be $\alpha  + \beta x$.
(iii)
(a)
The linear predictor has the form
\[\alpha  i + \beta j + \gamma x\]
where \alpha  i is the factor for policyholder gender (i = 1,2)
\beta j is the factor for vehicle rating group
x is the policyholder age
(\alpha  1 = 0, \beta 1 = 0)
(b)
The linear predictor becomes
\[\alpha  i + \beta j + \gamma i x\]
\end{document}
