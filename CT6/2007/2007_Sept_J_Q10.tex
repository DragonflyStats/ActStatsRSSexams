
[Total 15]10 
The time series X t is assumed to be stationary and to follow an ARMA (2,1) process
defined by:
X t = 1 +
8
1
1
X t − 1 –
X t − 2 + Ζ t – Ζ t− 1
15
15
7
where Z t are independent N (0 ,1) random variables.
(i) Determine the roots of the characteristic polynomial, and explain how their
values relate to the stationarity of the process.

(ii) (a)
Find the autocorrelation function for lags 0 , 1 and 2.
(b)
Derive the autocorrelation at lag k in the form
\rho k =
A
c
k
+
B
d k
[12]
(iii)
Determine the mean and variance of X t .
\newpage
%%%%%%%%%%%%%%%%%%%%%%%%%%%%%%%%%%%%%%%%%%%%%%%%%%%%%%%%%%%%%%%%%%%%%%%%%%%%%%%%%%%%%%

Page 12%%%%%%%%%%%%%%%%%%%%%%5 — September 20 0 7 — Examiners’ Report
10
(i)
The characteristic equation is given by:
(1 -
8
1 2
1
1
\lambda +
\lambda ) = (1 -
\lambda ) (1 - \lambda ) = 0 
15
15
3
5
which has roots = 3 and 5. They are both greater than 1. Hence, subject to the
initial values having appropriate distributions, this implies (weak) stationarity.
(ii)
(a)
Firstly, note that Cov ( X t , Z t ) = 1 and Cov ( X t , Z t − 1 ) = 8 / 15 − 1 / 7 = 41 / 10 5
We need to generate 3 distinct equations linking γ 0  , γ 1 and γ 2
This can be done as follows:
(A)
γ 0  = Cov ( X t , X t ) = Cov ( 1 + 8 / 15 X t − 1 − 1 / 15 X t − 2 + Z t − 1 / 7 Z t − 1 , X t )
= 8 / 15 γ 1 − 1 / 15 γ 2 + 1 − 1 / 7 \times  41 / 10 5
= 8 / 15 γ 1 − 1 / 15 γ 2 + 694 / 735
(B)
γ 1 = Cov ( X t , X t − 1 ) = Cov ( 1 + 8 / 15 X t − 1 − 1 / 15 X t − 2 + Z t − 1 / 7 Z t − 1 , X t − 1 )
= 8 / 15 γ 0  − 1 / 15 γ 1 − 1 / 7
(C)
γ 2 = Cov ( X t , X t − 2 ) = Cov ( 1 + 8 / 15 X t − 1 − 1 / 15 X t − 2 + Z t − 1 / 7 Z t − 1 , X t − 2 )
= 8 / 15 γ 1 − 1 / 15 γ 0 
Next stage is to solve these equations.
Substituting (C) into (A) gives
γ 0  = ( 8 / 15 ) γ 1 − ( 1 / 15 )(( 8 / 15 ) γ 1 − ( 1 / 15 ) γ 0  ) + 694 / 735
so
( 224 / 225 ) γ 0  = ( 112 / 225 ) γ 1 + 694 / 735
γ 0  = ( 1 / 2 ) γ 1 + 520 5 / 5488
Page 13%%%%%%%%%%%%%%%%%%%%%%5 — September 20 0 7 — Examiners’ Report
Now substituting into (B) gives
γ 1 = 8 / 15 (( 1 / 2 ) γ 1 + 520 5 / 5488 ) − ( 1 / 15 ) γ 1 − 1 / 7
so
( 4 / 5 ) γ 1 = 249 / 686
γ 1 = 1245 / 2744 = 0  . 4537
And
γ 0  = 1 / 2 \times  0  . 4537 + 520 5 / 5488 = 1 . 1753
γ 2 = 8 / 15 \times  0  . 4537 − 1 / 15 \times  1 . 1753 = 0  . 1636
Finally,
\rho 0  = 1, \rho 1 =
(b)
\rho k =
γ 1
γ
= 0 .386, \rho 2 = 2 = 0 .139
γ 0 
γ 0 
8
1
\rho k − 1 -
\rho k − 2
15
15
for k ≥ 2
We will show that the solution has the form:
\rho k
1
1
= A ( ) k + B ( ) k
3
5
Substituting the proposed solution into the recurrence relation gives
8
1
8
1
1
1
1
1
\rho k − 1 -
\rho k − 2 =
[ A ( ) k - 1 + B ( ) k–1 ] -
[ A ( ) k–2 + B ( ) k–2 ]
15
15
15
2
5
15
3
5
1
8
1
1
8
1
= A ( ) k (
\times  3 - \times  9) + B ( ) k (
\times  5 -
\times  25)
3
15
5
5
15
15
1
1
= A ( ) k + B ( ) k
3
5
= \rho k
So the solution does have this form.
Page 14%%%%%%%%%%%%%%%%%%%%%%5 — September 20 0 7 — Examiners’ Report
The values of A and B are fixed by \rho 0  = 1, \rho 1 = 0 .386
∴ A + B = 1
1
1
A + B = 0 .386
3
5
→
1
1
A + (1 – A ) = 0 .386
3
5
A = 1.395
B = - 0 .395
1
1
∴ P k = 1.395 ( ) k – 0 .395 ( ) k
3
5
(iii)
We require mean and variance of X t which must be normally distributed
since Z is normally distributed.
Variance is γ 0  = 1.1753 from (ii) (a)
E ( X t ) = 1 +
∴ E ( X t ) =
8
1
E ( X t ) -
E ( X t )
15
15
15
8
END OF EXAMINERS’ REPORT
Page 15
