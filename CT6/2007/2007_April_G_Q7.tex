\documentclass[a4paper,12pt]{article}

%%%%%%%%%%%%%%%%%%%%%%%%%%%%%%%%%%%%%%%%%%%%%%%%%%%%%%%%%%%%%%%%%%%%%%%%%%%%%%%%%%%%%%%%%%%%%%%%%%%%%%%%%%%%%%%%%%%%%%%%%%%%%%%%%%%%%%%%%%%%%%%%%%%%%%%%%%%%%%%%%%%%%%%%%%%%%%%%%%%%%%%%%%%%%%%%%%%%%%%%%%%%%%%%%%%%%%%%%%%%%%%%%%%%%%%%%%%%%%%%%%%%%%%%%%%%

\usepackage{eurosym}
\usepackage{vmargin}
\usepackage{amsmath}
\usepackage{graphics}
\usepackage{epsfig}
\usepackage{enumerate}
\usepackage{multicol}
\usepackage{subfigure}
\usepackage{fancyhdr}
\usepackage{listings}
\usepackage{framed}
\usepackage{graphicx}
\usepackage{amsmath}
\usepackage{chngpage}

%\usepackage{bigints}
\usepackage{vmargin}

% left top textwidth textheight headheight

% headsep footheight footskip

\setmargins{2.0cm}{2.5cm}{16 cm}{22cm}{0.5cm}{0cm}{1cm}{1cm}

\renewcommand{\baselinestretch}{1.3}

\setcounter{MaxMatrixCols}{10}

\begin{document}

%%%%%%%%%%%%%%%%%%%%%%%%%%%%%%%%%%%%%%%%%%%%%%%%%%%%%%%%%%%%%%%%%%%%%%%%%%%%%%%%%%%%%%%%%55

CT6 A2007—47
The total claims arising from a certain portfolio of insurance policies over a given
month is represented by
⎧ N
⎪ X i
S = ⎨ ∑
i = 1
⎪
⎩ 0
if
N > 0
if N = 0
where N has a Poisson distribution with mean 2 and X 1 , X 2 , ... , X N is a sequence of
independent and identically distributed random variables that are also independent of
N. Their distribution is such that P ( X i = 1) = 1/ 3 and P ( X i = 2) = 2 / 3 . An
aggregate reinsurance contract has been arranged such that the amount paid by the
reinsurer is S - 3 (if S > 3) and zero otherwise.
The aggregate claims paid by the direct insurer and the reinsurer are denoted by
S I and S R , respectively.
Calculate E ( S I ) and E ( S R ) .

\begin{enumerate}
%%%%%%%%%%%%%%%%%%%%%%%%%%%%%%%%%%%%%%%%%%

7
E ( S ) = E ( N ) × E ( X 1 )
= 2 × (1 × 1 + 2 × 2 )
3
3
= 10
3
and S = S I + S R .
We will calculate directly the distribution of S I .
P ( S I = 0) = P ( N = 0) = e − 2 = 0.13534
P ( S I = 1) = P ( N = 1) P ( X 1 = 1) = e − 2
2 1 1
× = 0.09022
1! 3
P ( S I = 2) = P ( N = 1) P ( X 1 = 2) + P ( N = 2) P ( X 1 = 1) P ( X 2 = 1)
2 1 2 − 2 2 2 1 1
× + e
× ×
1! 3
2! 3 3
= 0.21052
= e − 2
P ( S I = 3) = 1 − 0.13534 − 0.09022 − 0.21052 = 0.56392
E ( S I ) = 0 × 0.13534 + 1 × 0.09022 + 2 × 0.21052 + 3 × 0.56392
= 2.20303
and hence
E ( S R ) = E ( S − S I ) = E ( S ) − E ( S I ) = 10 − 2.20303 = 1.1303.
3
Page 7Subject CT6 (Statistical Methods Core Technical) — April 2007 — Examiners’ Report



\end{document}
