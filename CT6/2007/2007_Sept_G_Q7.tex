6\documentclass[a4paper,12pt]{article}

%%%%%%%%%%%%%%%%%%%%%%%%%%%%%%%%%%%%%%%%%%%%%%%%%%%%%%%%%%%%%%%%%%%%%%%%%%%%%%%%%%%%%%%%%%%%%%%%%%%%%%%%%%%%%%%%%%%%%%%%%%%%%%%%%%%%%%%%%%%%%%%%%%%%%%%%%%%%%%%%%%%%%%%%%%%%%%%%%%%%%%%%%%%%%%%%%%%%%%%%%%%%%%%%%%%%%%%%%%%%%%%%%%%%%%%%%%%%%%%%%%%%%%%%%%%%

\usepackage{eurosym}
\usepackage{vmargin}
\usepackage{amsmath}
\usepackage{graphics}
\usepackage{epsfig}
\usepackage{enumerate}
\usepackage{multicol}
\usepackage{subfigure}
\usepackage{fancyhdr}
\usepackage{listings}
\usepackage{framed}
\usepackage{graphicx}
\usepackage{amsmath}
\usepackage{chng%%-- Page}

%\usepackage{bigints}
\usepackage{vmargin}

% left top textwidth textheight headheight

% headsep footheight footskip

\setmargins{2.0cm}{2.5cm}{16 cm}{22cm}{0.5cm}{0cm}{1cm}{1cm}

\renewcommand{\baselinestretch}{1.3}

\setcounter{MaxMatrixCols}{10}

\begin{document}

7
A no claims discount system has 4 levels. The premiums paid by policyholders in
each level are as follows:
Level Premium
0
25
40
50 100%
75%
60%
50%
The rules for moving between the levels are as follows:
• following a claim-free year, a policyholder moves to the next higher level of
discount, or remains at 50% discount
• following a year of one or more claims, a policyholder moves to the next lower
discount rate or remains at 0% discount
It is assumed that claims occur according to a Poisson process with rate λ per year per
policyholder, and that the equilibrium distribution has been reached.
(i)
Show that the average premium paid, if the premium paid by a policyholder in
level 0 is £500, may be written as
⎛ 1 + 0.75 k + 0.6 k 2 + 0.5 k 3 ⎞
500 ⎜
⎟ ⎟
⎜
1 + k + k 2 + k 3
⎝
⎠
where k =
e −λ
1 − e −λ
[5]
(ii) Calculate the average premium paid by policyholders whose claim rate per
year is (a) 0.12, (b) 0.24, (c) 0.36.
[3]
(iii) Comment on the results in (ii), in relation to the effectiveness of the no claims
discount system discriminating between good and bad drivers.
[2]
[Total 10]


%%%%%%%%%%%%%
7
(i)
Let θ = exp( - \lambda )
\pi  = \pi  P
θ
0
⎡ 1 − θ
⎢ 1 − θ
θ
0
P = ⎢
⎢ 0 1 − θ
0
⎢
0 1 − θ
⎣ 0
\pi  0
\pi  25
\pi  40
\pi  50
=
=
=
=
(1 - θ) \pi  0 +
θ \pi  0 +
0 ⎤
0 ⎥ ⎥
θ ⎥
⎥
θ ⎦
(1 - θ) \pi  25
(1 - θ) \pi  40
θ \pi  25 +
θ \pi  40 +
\pi  0 + \pi  25 + \pi  40 + \pi  50 = 1
\pi  25 =
\pi  40 =
\pi  50 =
θ
\pi  0
1 − θ
θ
\pi  25
1 − θ
θ
\pi  40
1 − θ
= k\pi  0
= k 2 \pi  0
= k 3 \pi  0
and hence \pi  0 + k \pi  0 + k 2 \pi  0 + k 3 \pi  0 = 1
Page 8
(1 - θ) \pi  50
θ \pi  50Subject CT6 (Statistical Methods Core Technical) — September 2007 — Examiners’ Report
hence \pi  0 =
1
1 + k + k 2 + k 3
hence the average premium paid is
⎛ 1 + 0.75 k + 0.6 k 2 + 0.5 k 3 ⎞
500 × ⎜
⎟ ⎟
⎜
1 + k + k 2 + k 3
⎝
⎠
θ/(1 - θ) = e -0.12 /(1 – e -0.12 ) = 7.8433
θ/(1 - θ) = e -0.24 /(1 – e -0.24 ) = 3.6866
θ/(1 - θ) = e -0.36 /(1 – e -0.36 ) = 2.3077
(ii) (a)
(b)
(c)
(iii) (a) to (b)
Premium = £257.79
Premium = £270.33
Premium = £288.46
\lambda  increases by 100% but average premium paid increases only by
4.9%
\lambda  increases by 50% but average premium paid increases only by
6.7%
(b) to (c)
The no claims discount system is not effective at discriminating between good
and bad drivers.

\end{document}
