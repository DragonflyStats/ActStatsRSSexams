\documentclass[a4paper,12pt]{article}

%%%%%%%%%%%%%%%%%%%%%%%%%%%%%%%%%%%%%%%%%%%%%%%%%%%%%%%%%%%%%%%%%%%%%%%%%%%%%%%%%%%%%%%%%%%%%%%%%%%%%%%%%%%%%%%%%%%%%%%%%%%%%%%%%%%%%%%%%%%%%%%%%%%%%%%%%%%%%%%%%%%%%%%%%%%%%%%%%%%%%%%%%%%%%%%%%%%%%%%%%%%%%%%%%%%%%%%%%%%%%%%%%%%%%%%%%%%%%%%%%%%%%%%%%%%%

\usepackage{eurosym}
\usepackage{vmargin}
\usepackage{amsmath}
\usepackage{graphics}
\usepackage{epsfig}
\usepackage{enumerate}
\usepackage{multicol}
\usepackage{subfigure}
\usepackage{fancyhdr}
\usepackage{listings}
\usepackage{framed}
\usepackage{graphicx}
\usepackage{amsmath}
\usepackage{chng%%-- Page}

%\usepackage{bigints}
\usepackage{vmargin}

% left top textwidth textheight headheight

% headsep footheight footskip

\setmargins{2.0cm}{2.5cm}{16 cm}{22cm}{0.5cm}{0cm}{1cm}{1cm}

\renewcommand{\baselinestretch}{1.3}

\setcounter{MaxMatrixCols}{10}

\begin{document}




%%%%%%%%%%%%%%%%%%%%%%%%%%%%%%%%%%%%%%%%%%%%%%%%%%%%%%%%%%%%%%%%%%%%%%%%%%%%%%%%%%%5
6
(i) Explain the main advantage of the Polar method compared with the Box-Muller method for generating pairs of uncorrelated pseudo-random values from a standard normal distribution.

(ii) Pseudo-random numbers are generated using the Box-Muller method in order to simulate values of $Y = X$ , where X has a lognormal distribution with parameters $\mu = 5$ and $\sigma = 2$. The quantity of interest is $\theta = E [ Y ]$ .
(a) Calculate the value of Y when the number generated by the Box-Muller method is 0.9095.
(b) The variance of Y has been estimated as 26.3. Calculate how many simulations should be performed in order to ensure that the
discrepancy between Y and θ̂ , measured by the absolute error, is less than 1 with probability at least 0.9.
\newpage
%%-- Page 6%%%%%%%%%%%%%%%%%%%%%%%%%%%%%%%%%%%%%%%%%%%%% — April 2007 — Examiners’ Report
6
(i) The requirement to calculate cos and sin is time consuming for a computer.
The Polar Method avoids this by using the acceptance-rejection method.
(ii) (a)
We must transform the values to a log-normal distribution with the
appropriate mean and variance by calculating
X = e 5 + 2 Z
The required value is Y = X = 915.1 = 30.25
(b)
We require n >
z \alpha 2 / 2 τ 2
ε 2
=
1.645 2 \times 26.3
= 71.17
1
Hence, 72 simulations are required.

\newpage
\end{document}
