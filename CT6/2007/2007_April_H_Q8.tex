\documentclass[a4paper,12pt]{article}

%%%%%%%%%%%%%%%%%%%%%%%%%%%%%%%%%%%%%%%%%%%%%%%%%%%%%%%%%%%%%%%%%%%%%%%%%%%%%%%%%%%%%%%%%%%%%%%%%%%%%%%%%%%%%%%%%%%%%%%%%%%%%%%%%%%%%%%%%%%%%%%%%%%%%%%%%%%%%%%%%%%%%%%%%%%%%%%%%%%%%%%%%%%%%%%%%%%%%%%%%%%%%%%%%%%%%%%%%%%%%%%%%%%%%%%%%%%%%%%%%%%%%%%%%%%%

\usepackage{eurosym}
\usepackage{vmargin}
\usepackage{amsmath}
\usepackage{graphics}
\usepackage{epsfig}
\usepackage{enumerate}
\usepackage{multicol}
\usepackage{subfigure}
\usepackage{fancyhdr}
\usepackage{listings}
\usepackage{framed}
\usepackage{graphicx}
\usepackage{amsmath}
\usepackage{chngpage}

%\usepackage{bigints}
\usepackage{vmargin}

% left top textwidth textheight headheight

% headsep footheight footskip

\setmargins{2.0cm}{2.5cm}{16 cm}{22cm}{0.5cm}{0cm}{1cm}{1cm}

\renewcommand{\baselinestretch}{1.3}

\setcounter{MaxMatrixCols}{10}

\begin{document}
[8]
PLEASE TURN OVER8
A modeller has attempted to fit an ARMA(p,q) model to a set of data using the Box-
Jenkins methodology. The plot of residuals based on this proposed fit is shown
below.
Residuals based on fitted model
120
100
80
60
40
20
0
-20
-40
-60
-80
1
6 11 16 21 26 31 36 41 46 51 56 61 66 71 76 81 86 91 96
Time
(i)
(ii)
Under the assumptions of the model, the residuals should form a white noise
process.
(a) By inspection of the chart, suggest two reasons to suspect that the
residuals do not form a white noise process.
(b) Define what is meant by a turning point.
(c) Perform a significance test on the number of turning points in the data
above. (There are 100 points in the data and 59 turning points.)
[6]
On your suggestion, the original fitted model is discarded, and re-
parameterised to:
X n + 2 = 5 + 0.9( X n + 1 − 5) + e n + 2 + 0.5 e n .
Given the following observations:
X 99 = 2,
e ˆ 99 = − 0.7,
X 100 = 7
e ˆ 100 = 1.4
Use the Box-Jenkins methodology to calculate the forward estimates
X 100 (1), X 100 (2) and X 100 (3) .
[4]
[Total 10]
CT6 A2007—6

\begin{enumerate}

8
(i)
(a)
Magnitude of the residuals increases over time, suggesting that the
variance is increasing over time.
More positive than negative residuals suggesting there is drift in the
process.
(b)
If y i ( i = 1, 2, ... , n ) is a sequence of numbers, it has a turning point at
time k if either y k − 1 < y k and y k > y k + 1 , or y k − 1 > y k and y k < y k + 1 .
(c)
Let T represent the number of turning points, and let N=100 be the
number of data points. Then
E ( T ) = 2 / 3( N − 2) = 2 / 3(100 − 2) = 65.333
Var ( T ) = (16 N − 29) / 90 = 17.45556 = 4.178 2
P ( T ≥ 59) ≈ P ( N (65.333, 4.178 2 ) > 59.5)
65.333 − 59.5
) = P ( N (0,1) > − 1.396) = 0.919
= P ( N (0,1) >
4.178
This is a two-sided test so there is approximately a 16% chance of getting such an extreme number of turning points. This value is not significant at the 5% level, and so this test gives no significant
evidence to suggest that the residuals are not a white noise process.
Full credit should also be given for calculating a confidence interval and
checking if 59 is in this.
(ii)
X 100 (1) = 5 + 0.9( X 100 − 5) + 0 + 0.5 e 99 = 5 + 0.9(7 − 5) − 0.5 × 0.7 = 6.45
X 100 (2) = 5 + 0.9( X 100 (1) − 5) + 0.5 e 100 = 5 + 0.9(6.45 − 5) + 0.5 × 1.4 = 7.005
X 100 (3) = 5 + 0.9( X 100 (2) − 5) + 0.5 e 101 = 5 + 0.9(7.005 − 5) = 6.8045


%%%%%%%%%%%%%%%%%%%%%%%%%%%%%%%%%%%%%%%%%%%%%%%%%%%%%%%%%%%%%%%%%%%%%%%%%%%%%%%%%%%%%%%%%55

CT6 A2007—47
The total claims arising from a certain portfolio of insurance policies over a given
month is represented by
⎧ N
⎪ X i
S = ⎨ ∑
i = 1
⎪
⎩ 0
if
N > 0
if N = 0
where N has a Poisson distribution with mean 2 and X 1 , X 2 , ... , X N is a sequence of
independent and identically distributed random variables that are also independent of
N. Their distribution is such that P ( X i = 1) = 1/ 3 and P ( X i = 2) = 2 / 3 . An
aggregate reinsurance contract has been arranged such that the amount paid by the
reinsurer is S - 3 (if S > 3) and zero otherwise.
The aggregate claims paid by the direct insurer and the reinsurer are denoted by
S I and S R , respectively.
Calculate E ( S I ) and E ( S R ) .
CT6 A2007—5
\end{document}
