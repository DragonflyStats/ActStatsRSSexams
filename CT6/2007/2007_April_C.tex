\documentclass[a4paper,12pt]{article}

%%%%%%%%%%%%%%%%%%%%%%%%%%%%%%%%%%%%%%%%%%%%%%%%%%%%%%%%%%%%%%%%%%%%%%%%%%%%%%%%%%%%%%%%%%%%%%%%%%%%%%%%%%%%%%%%%%%%%%%%%%%%%%%%%%%%%%%%%%%%%%%%%%%%%%%%%%%%%%%%%%%%%%%%%%%%%%%%%%%%%%%%%%%%%%%%%%%%%%%%%%%%%%%%%%%%%%%%%%%%%%%%%%%%%%%%%%%%%%%%%%%%%%%%%%%%

\usepackage{eurosym}
\usepackage{vmargin}
\usepackage{amsmath}
\usepackage{graphics}
\usepackage{epsfig}
\usepackage{enumerate}
\usepackage{multicol}
\usepackage{subfigure}
\usepackage{fancyhdr}
\usepackage{listings}
\usepackage{framed}
\usepackage{graphicx}
\usepackage{amsmath}
\usepackage{chngpage}

%\usepackage{bigints}
\usepackage{vmargin}

% left top textwidth textheight headheight

% headsep footheight footskip

\setmargins{2.0cm}{2.5cm}{16 cm}{22cm}{0.5cm}{0cm}{1cm}{1cm}

\renewcommand{\baselinestretch}{1.3}

\setcounter{MaxMatrixCols}{10}

\begin{document}

\begin{enumerate}
%%%%%%%%%%%%%%%%%%%%%%%%%%%%%%%%%%%%%%%%%%

5
The cumulative cost of claims paid is (Figures in £000s):
Accident Year
2004
2005
2006
Development year
0
1
4,144
4,767
5,903
4,838
5,599
2
5,021
The number of accumulated settled claims is as follows:
Accident Year
2004
2005
2006
Development year
0
1
581
626
674
656
697
2 Ult
684 684
727
788
Grossing up factors for claim numbers
Accident Year
2004
2005
2006
Development year
0
1
2
0.849 0.959
0.861 0.959
0.855
1
% Page 5Subject CT6 (Statistical Methods Core Technical) — April 2007 — Examiners’ Report
Average cost per settled claim
Accident Year
2004
2005
2006
Development year
0
1
7.133
7.615
8.758
7.375
8.033
2 Ult
7.341 7.341
7.996
9.104
Grossing up factors for average claim amounts
Accident Year
2004
2005
2006
Development year
0
1
0.972
0.952
0.962
1.005
1.005
2
1.000
The total ultimate loss is therefore:
Accident Year ACPC Claim
Numbers Projected
Loss
2004
2005
2006 7.341
7.996
9.104 684
727
788 5,021
5,813
7,174
18,008
Claims paid to date
Outstanding claims
16,523
1,485
Assumptions:
Claims fully run-off by end of development year 3.
Projections based on simple average of grossing up factors.
Number of claims relating to each development year are a constant proportion of total
claim numbers from the origin year.
Similarly for claim amounts i.e. same proportion of total claim amount for origin year.
Page 6Subject CT6 (Statistical Methods Core Technical) — April 2007 — Examiners’ Report
6
(i) The requirement to calculate cos and sin is time consuming for a computer.
The Polar Method avoids this by using the acceptance-rejection method.
(ii) (a)
We must transform the values to a log-normal distribution with the
appropriate mean and variance by calculating
X = e 5 + 2 Z
The required value is Y = X = 915.1 = 30.25
(b)
We require n >
z α 2 / 2 τ 2
ε 2
=
1.645 2 × 26.3
= 71.17
1
Hence, 72 simulations are required.


%%%%%%%%%%%%%%%%%%%%%%%%%%%%%%%%%%%%%%%%%%%%%%%%%%%%%%%%%%%%%%%%%%%%%%%%%%%%%%%%%%%5
5
The delay triangles given below relate to a portfolio of motor insurance policies.
The cost of claims settled during each year is given in the table below:
(Figures in £000s)
Development year
Accident
Year
2004
2005
2006
0 1 2
4,144
4,767
5,903 694
832 183
The corresponding number of settled claims is as follows:
Development year
Accident
Year
2004
2005
2006
0 1 2
581
626
674 75
71 28
Calculate the outstanding claims reserve for this portfolio using the average cost per claim method with grossing-up factors, and state the assumptions underlying your result.
%%%%%%%%%%%%%%%%%%%%%%%%%%%%%%%%%%%%%%%%%%%%%%%%%%%%%%%%%%%%%%%
6
(i) Explain the main advantage of the Polar method compared with the Box-Muller method for generating pairs of uncorrelated pseudo-random values from a standard normal distribution.

(ii) Pseudo-random numbers are generated using the Box-Muller method in order to simulate values of $Y = X$ , where X has a lognormal distribution with parameters $\mu = 5$ and $\sigma = 2$. The quantity of interest is $\theta = E [ Y ]$ .
(a) Calculate the value of Y when the number generated by the Box-Muller method is 0.9095.
(b) The variance of Y has been estimated as 26.3. Calculate how many simulations should be performed in order to ensure that the
discrepancy between Y and θ̂ , measured by the absolute error, is less than 1 with probability at least 0.9.
\end{document}
