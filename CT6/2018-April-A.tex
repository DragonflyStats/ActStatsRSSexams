\documentclass[a4paper,12pt]{article}

%%%%%%%%%%%%%%%%%%%%%%%%%%%%%%%%%%%%%%%%%%%%%%%%%%%%%%%%%%%%%%%%%%%%%%%%%%%%%%%%%%%%%%%%%%%%%%%%%%%%%%%%%%%%%%%%%%%%%%%%%%%%%%%%%%%%%%%%%%%%%%%%%%%%%%%%%%%%%%%%%%%%%%%%%%%%%%%%%%%%%%%%%%%%%%%%%%%%%%%%%%%%%%%%%%%%%%%%%%%%%%%%%%%%%%%%%%%%%%%%%%%%%%%%%%%%

\usepackage{eurosym}
\usepackage{vmargin}
\usepackage{amsmath}
\usepackage{graphics}
\usepackage{epsfig}
\usepackage{enumerate}
\usepackage{multicol}
\usepackage{subfigure}
\usepackage{fancyhdr}
\usepackage{listings}
\usepackage{framed}
\usepackage{graphicx}
\usepackage{amsmath}
\usepackage{chngpage}
%\usepackage{bigints}
\usepackage{vmargin}

% left top textwidth textheight headheight

% headsep footheight footskip
\setmargins{2.0cm}{2.5cm}{16 cm}{22cm}{0.5cm}{0cm}{1cm}{1cm}
\renewcommand{\baselinestretch}{1.3}
\setcounter{MaxMatrixCols}{10}
\begin{document}

CT6 A2018 
© Institute and Faculty of Actuaries1
A random variable X follows a Pareto distribution with density function:
5
, x > 0
(1 + x) 6
For a given estimate d of x, the loss function is defined as:
x 4 – 4 d 2 x 2 + d 4
(a)
Show that the expected loss is given by:
2 d 2
E ( L (x, d )) = 1 –
+ d 4
3
(b)
Determine the optimal estimate for d under the Bayes rule.
[5]
2
An insurance company has a portfolio of insurance policies. Claims arise according to a Poisson process, and claim amounts have a probability distribution with parameter q.
(i) State one assumption the insurance company is likely to make when modelling
aggregate claim amounts.
[1]
(ii) Explain what the Maximum Likelihood Estimate (MLE) of q represents.
[2]
(iii) State an alternative to using the MLE.
[1]
(iv) Suggest two complications that may arise for the insurance company when it uses past claims data to determine the MLE of q.[2]
[Total 6]


%%%%%%%%%%%%%%%%%%%%%%%%%%%%%%%%%%%%%%%%%%%%%%%%%%%%%%%%%%%%%%%%%%%5
Q1
(a)
Pareto so clear λ = 1 and α = 5 [1]
Γ ( α − 2 ) Γ ( 1 + 2 ) 2* 2 1
2
  
= =
From tables E X =
Γ ( α )
24 6 [1⁄2]
( )
( )
=
E X 4
Γ ( α − 4 ) Γ ( 1 + 4 )
= 1
Γ ( α )
(
)
1 −
So E X 4 − 4 d 2 X 2 + d 4 =
(b)
[1⁄2]
2 d 2
+ d 4 as required
3
[1⁄2]
Differentiating with respect to d and setting equal to 0
−
4 d
+ 4 d 3 = 0
3
So 4 =
d 2
[1]
4
1
=
,  
d
3
3
Check for minimum: second derivative is 12 d 2 −
[1]
4
, > 0 so this is indeed a
3
minimum.
[1⁄2]
[Total 5]
Many candidates struggled on part (a), as this has not been examined
for some time, although the majority of candidates were able to score
well on part (b), including checking for a minimum.
Q2
(i) The occurrence of the claim and the amount of the claim can be modelled
separately, they are independent.
[1]
(ii) The maximum likelihood estimate yields the highest probability of observing
what has been observed,
[2]
(iii) Method of moments
Method of percentiles
Bayesian estimation
(iv) Insufficient claims
[1]
[1]
[1]
[Max 1]
[1]
Page 3Subject CT6 (Statistical Methods Core Technical) – April 2018 – Examiners’ Report
Large claims not recorded (reinsurance)
Small claims not recorded (policy excess)
Change in nature of policy over time
Complications with modelling future inflation
[1]
[1]
[1]
[1]
[Max 2]
[Total 6]
Candidates with a strong knowledge and understanding of the
bookwork were able to score well here, although many struggled to
articulate their points sufficiently well in parts (i) and (ii); and to give
clearly differentiated examples in part (iv).
