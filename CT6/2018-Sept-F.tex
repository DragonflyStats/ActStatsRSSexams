8
[1]
For a portfolio of insurance policies, claims X i are independent and follow a gamma
distribution, with parameters α = 6 and β, which is unknown.
A random sample of n claims, X 1 ,..., X n is selected, with mean X .
(i) Derive an expression for the estimator of β using the method of moments. [2]
(ii) Explain what the Maximum Likelihood Estimator (MLE) of β represents. [2]
(iii) Derive an expression for the MLE of β, commenting on the result.
(iv) State the Moment Generating Function (MGF) of X.[1]
[5]
Let Y = 2n β X .
(v)

CT6 S2018–5 
Derive the MGF of Y, and hence its distribution, including statement of
parameters.[5]
[Total 15]
%%%%%%%%%%%%%%%%%%%%%%%%%%%%%%%%%%%%%%%%%%%%%%%%%%%%%%%%%%%%%%%%%%%%%%%%%%%%%%%%%%%%%%5
Q8
(i)
(ii)
Mean = α
α
β ⇒ β = X
[2]
The MLE is the estimate that maximises the likelihood of having observed the
sample data.
[2]
n
L ( β ) = ∏
i = 1
(iii)
β α α − 1 − β x
x i e
Γ ( α )
i
n
[11⁄2]
log L ( β ) ∝ n α ln β − β ∑ x i
i = 1
Then differentiate
n
n α
0
− ∑ x i =
β
i = 1
⇒ β = α n
= α = 6
X
X
i
∑ x
[11⁄2]
Check for maximum
d 2 ln L − 6 n
=
< 0
β 2
d β 2 [1]
In this case the MLE and method of moments lead to the same result. [1]
Page 8Subject CT6 (Statistical Methods Core Technical) – September 2018 – Examiners’ Report
(iv)
(v)

t 
 1 − 
 β 
− α
[1]
2 β t ∑ X i
2 n β tX
=
( e tY ) E ( e =
) E ( e =
)
M
E =
Y ( t )
n
∏ E ( e β
2 tX i
)
[11⁄2]
i = 1
by independence
[1⁄2]
− α
 2 β t 
− n α
M Y ( t ) =
( 1 − 2 t )
∏ i   1 − β   =
By the uniqueness property of MGFs
This is a Chi Squared distribution with parameter 2nα
[1]
[1]
[1]
[Total 15]
Many candidates were able to score well on this question, although again marks were
dropped when not showing all the steps in parts (iii) & (v).
