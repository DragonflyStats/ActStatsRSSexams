PLEASE TURN OVER8
Claims on a portfolio of insurance policies arrive as a Poisson process with rate λ .
The claim sizes are independent identically distributed random variables
X 1 , X 2 , ...with:
M
∑ p k = 1 .
P ( X i = k ) = p k for k = 1, 2, ... , M and
k = 1
The premium loading factor is θ.
(i)
Show that the adjustment coefficient R satisfies:
2 θ m 1
1
log(1 + θ ) < R <
M
m 2
where m i = E ( X 1 i ) for i = 1, 2 .
[The inequality e Rx ≤
[7]
x RM
x
for 0 ≤ x ≤ M may be used without
+ 1 −
e
M
M
proof.]
(ii)
9
(a) Determine upper and lower bounds for R if θ = 0.3 and X i is equally
likely to be 2 or 3 (and cannot take any other values).
(b) Hence derive an upper bound on the probability of ruin when the initial
surplus is U.
[3]
[Total 10]
An actuarial student has been working on some claims projections but some of her
workings have been lost. The cumulative claim amounts and projected ultimate
claims are given by the following table:
Accident
Year 0
1
2
3
4 1001
1250
1302
Z
Development Year
1
2
1485
Y
1805
1762
1820
Ultimate
3
W
X
1862.3
2122.5
2278.8
All claims are paid by the end of development year 3.
It is known that ultimate claims for accident years 2 and 3 have been estimated using
the Basic Chain Ladder method.
(i)
Calculate the values of W, X and Y.
[5]
For accident year 4 the student has used the Bornhuetter-Ferguson method using an
earned premium of 2,500 and an expected loss ratio of 90%.
CT6 S2010—410
(ii) Calculate the value of Z.
[4]
(iii) Calculate the outstanding claims reserve for all accident years implied by the
completed table.
%%%%%%%%%%%%%%%%%%%%%%%%%%%%%%%%%%%%%%%%%%%%%%%%%%%%%%%%%%%%%%%%%%%%%%%%%%%%%%%%%%%%%%%%%%%%%%
8
(i)
The adjustment coefficient satisfies the equation
λ + λ (1 + θ ) E ( X 1 ) R = λ M X 1 ( R )
M
That is 1 + (1 + θ ) E ( X 1 ) R = ∑ e Rj p j
j = 1
Applying the inequality given in the question we have
M
1 + (1 + θ ) E ( X 1 ) R ≤ ∑ p j (
j = 1
So 1 + (1 + θ ) E ( X 1 ) R ≤
So (1 + θ ) E ( X 1 ) R ≤
e RM
M
M
∑
j = 1
j RM
j
+ 1 − )
e
M
M
jp j + 1 −
1
M
M
∑
j = 1
jp j =
e RM E ( X 1 )
E ( X 1 )
+ 1 −
M
M
E ( X 1 ) RM
( e
− 1)
M
Page 7Subject CT6 (Statistical Methods Core Technical) — September 2010 — Examiners’ Report
and so
⎛
⎞
⎛ RM R 2 M 2
⎞
R 2 M 2 R 3 M 3
+
+ " − 1 ⎟ = R ⎜ 1 +
+
+ " ⎟
⎜ ⎜ 1 + RM +
⎟
⎜
⎟
2!
3!
2!
3!
⎝
⎠
⎝
⎠
2
2
⎛
⎞
R M
(1 + θ ) R < R ⎜ 1 + RM +
+ " ⎟ = R × e RM
⎜
⎟
2!
⎝
⎠
(1 + θ ) R ≤
1
M
Taking logs, we have
log(1 + θ ) < RM
And so R >
log(1 + θ )
as required.
M
To get the other inequality, we go back to
M
1 + (1 + θ ) E ( X 1 ) R = ∑ e Rj p j
j = 1
M
And so 1 + (1 + θ ) E ( X 1 ) R > ∑ p j (1 + Rj +
j = 1
So we have (1 + θ ) m 1 R > m 1 R +
i.e. θ m 1 >
i.e. R <
(ii)
(a)
R 2 j 2
R 2
) = 1 + RE ( X 1 ) +
E ( X i 2 )
2
2
R 2 m 2
2
Rm 2
2
2 θ m 1
as required.
m 2
In this case we have:
M = 3
And E ( X 1 ) = 2.5 and E ( X 1 2 ) = (4 + 9) / 2 = 6.5
So the inequality in the question gives:
1
2 × 0.3 × 2.5
log1.3 < R <
3
6.5
That is 0.08745 < R < 0.23077
Page 8Subject CT6 (Statistical Methods Core Technical) — September 2010 — Examiners’ Report
(b)
By Lundberg;s inequality ψ ( U ) ≤ e − RU ≤ e − 0.08745 U .
This question was not well answered, with relatively few candidates scoring more than 5
marks.
9
(i)
The development ratio for development year 2 to development year 3 is given
by 1862.3/1820 = 1.023242
Therefore W = 1762 × 1.023242 = 1803.0
Because there is no claims development beyond development year 3
X = 1803.0 also.
The development factor from development year 1 to ultimate is given by
2122.5/1805 = 1.1759003
So the ratio from development year 1 to development year 2 is given by
1.1759003/1.023242 = 1.149190785
But under the definition of the chain ladder approach, this is calculated as:
1.149190785 =
So Y =
(ii)
1762 + 1820
3582
=
Y + 1485
Y + 1485
3582
− 1485 = 1632.0
1.149190785
We require the development ratio from year 0 to year 1; this is given by:
1485 + 1632 + 1805 4922
=
= 1.385308
1001 + 1250 + 1302 3553
The development factor to ultimate is therefore
1 . 385308 × 1 . 149190785 × 1 . 023242 = 1 . 628984285
1
⎛
⎞
And so Z = 2278 . 8 − 2500 × 0 . 9 × ⎜ 1 −
⎟ = 1410 . 0
⎝ 1 . 628984285 ⎠
(iii)
The outstanding claims reserve is
1862.3 + 2122.5 + 2278.8 − 1820 − 1805 − 1410 = 1228.6
This slightly unusual question was nevertheless generally well answered, showing that
candidates understood the principles underlying the calculations. Many candidates scored
full marks here.
%%%%%%%%%%%%%%%%%%%%%%%%%%%%%%%%%%%%%%%%%%%%%%%%%%%%%%%%%%%%%%%%%%%%%%%%%%%%%%%%%%%%%%%%%%%%%%%%%%%%%5
