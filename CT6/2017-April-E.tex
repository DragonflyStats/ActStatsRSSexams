\documentclass[a4paper,12pt]{article}



%%%%%%%%%%%%%%%%%%%%%%%%%%%%%%%%%%%%%%%%%%%%%%%%%%%%%%%%%%%%%%%%%%%%%%%%%%%%%%%%%%%%%%%%%%%%%%%%%%%%%%%%%%%%%%%%%%%%%%%%%%%%%%%%%%%%%%%%%%%%%%%%%%%%%%%%%%%%%%%%%%%%%%%%%%%%%%%%%%%%%%%%%%%%%%%%%%%%%%%%%%%%%%%%%%%%%%%%%%%%%%%%%%%%%%%%%%%%%%%%%%%%%%%%%%%%
  
  
  
  \usepackage{eurosym}

\usepackage{vmargin}

\usepackage{amsmath}

\usepackage{graphics}

\usepackage{epsfig}

\usepackage{enumerate}

\usepackage{multicol}

\usepackage{subfigure}

\usepackage{fancyhdr}

\usepackage{listings}

\usepackage{framed}

\usepackage{graphicx}

\usepackage{amsmath}

\usepackage{chngpage}



%\usepackage{bigints}

\usepackage{vmargin}

% left top textwidth textheight headheight

% headsep footheight footskip

\setmargins{2.0cm}{2.5cm}{16 cm}{22cm}{0.5cm}{0cm}{1cm}{1cm}

\renewcommand{\baselinestretch}{1.3}

\setcounter{MaxMatrixCols}{10}



\begin{document}

\begin{enumerate}
CT6 A2017–5 PLEASE TURN OVER
9 Consider the probability density function of a Gamma distribution where:
  1
( ) 0, 1
Γ( )
x e x f x x
α− −
= > α >
  α
.
Consider the simpler probability density function h(x) where:
  h(x) = βe−βx x > 0 .
Let ( )
( ) max
x
f x
C
h x
= .
(i) Show that C is ( )
1 ( 1) 0
0
1
Γ
xα− e β− x
β α
, where x0 = 1
1
α−
−β
. [4]
(ii) Construct an algorithm which outputs a random variate from f(x) using h(x)
and the Acceptance-Rejection method. [4]
(iii) Determine the value of β that makes the algorithm most efficient, by
maximising the number of accepted values. [6]
[Total 14]
CT6 A2017–6
%%%%%%%%%%%%%%%%%%%%%%%%%%%%%%%%%%%%%%%%%%%%%%%%%%%%%%%%%%%%%%%%%%%%%%%
Q9 (i) The function
 
   
f x 1 1 x x 0
x e x
h x
    
 
[½]
the upper bound of this function is obtained at the same value for x as that of
 
      log log 1 1 log
f x
x x x
h x
   
                
. [1]
However the derivative of  1log x  x  x is 1 1
x
 
   [1]
Subject CT6 (Statistical Methods Core Technical) – April 2017 – Examiners’ Report
Page 10
which becomes zero at 0
1
1
x  

 
[½]
and since the second derivative is 2
1
x
 
 <0 implies that the maximum value
of f / h is attained for 0
1
1
x  

 
. [½]
So C =  
1
0
1 x  exx x  0
 
[½]
(ii) The function    
 
 
  0
1 1
1 1
0
x
x
f x x e g x
Ch x x e
 
 
  [1]
So the rejection algorithm looks like this
1 – Simulate U1 U 0,1 and set 1
Y   1 logU

. [1½]
2 – Simulate U2 U 0,1 if U2  g Y  set accept the value by setting X = Y
otherwise go back to stage 1. [1½]
(iii) The algorithm is most efficient choosing the value  making sup f
h
 
 
 
the
smallest (or minimizing C, or maximizing g(x)) [1]
 
 
 
0
1
1 1 1 1 1
0
sup 1 1 1
Γ Γ 1
f x e x e
h

                       
[1]
Which is minimized if
1
log 1 1
1


                
is minimized i.e. [1]
      
log 1 log 1 ' 1 1 1 0
1 1
  
       
   
[2]
Subject CT6 (Statistical Methods Core Technical) – April 2017 – Examiners’ Report
Page 11
i.e.   1 so   1

[1]
[Total 14]
Candidates mostly scored well in part (i), although those unfamiliar with the
technique of using logs often struggled. Part (ii) was relatively well answered,
although only the best candidates were able to score well in part (iii).
