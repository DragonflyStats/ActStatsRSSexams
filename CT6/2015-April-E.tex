\documentclass[a4paper,12pt]{article}

%%%%%%%%%%%%%%%%%%%%%%%%%%%%%%%%%%%%%%%%%%%%%%%%%%%%%%%%%%%%%%%%%%%%%%%%%%%%%%%%%%%%%%%%%%%%%%%%%%%%%%%%%%%%%%%%%%%%%%%%%%%%%%%%%%%%%%%%%%%%%%%%%%%%%%%%%%%%%%%%%%%%%%%%%%%%%%%%%%%%%%%%%%%%%%%%%%%%%%%%%%%%%%%%%%%%%%%%%%%%%%%%%%%%%%%%%%%%%%%%%%%%%%%%%%%%

\usepackage{eurosym}
\usepackage{vmargin}
\usepackage{amsmath}
\usepackage{graphics}
\usepackage{epsfig}
\usepackage{enumerate}
\usepackage{multicol}
\usepackage{subfigure}
\usepackage{fancyhdr}
\usepackage{listings}
\usepackage{framed}
\usepackage{graphicx}
\usepackage{amsmath}
\usepackage{chngpage}

%\usepackage{bigints}
\usepackage{vmargin}

% left top textwidth textheight headheight

% headsep footheight footskip

\setmargins{2.0cm}{2.5cm}{16 cm}{22cm}{0.5cm}{0cm}{1cm}{1cm}

\renewcommand{\baselinestretch}{1.3}

\setcounter{MaxMatrixCols}{10}

\begin{document}

\begin{enumerate}
%%%%%%%%%%%%%%%%%%%%%%%%%
\item 9
(i)
Consider the loss function given by:
 0 p    g ( x )  p  
L ( g ( x ) p ) = 
 1 otherwise
The expected loss is
E ( L ) = 1  
g 
g 
f ( p  x ) dp
= 1  2 f ( gx )
Page 12Subject CT6 (Statistical Methods Core Technical) – April 2015 – Examiners’ Report
To minimise this expression we need to maximise f ( gx ) which is done by
choosing g at the mode of the posterior distribution.
(ii)
The likelihood function is L ( p )  p 13 (1  p ) 87 p 20 (1  p ) 80+ g
 p 33 (1  p ) 167+ g
the prior likelihood of p is given by f ( p )  p (1  p ) 7
so f ( px )  f ( xp ) f ( p )
 p 33 (1  p ) 167+ g p (1  p ) 7
 p 34 (1  p ) 174+ g
so we have a Beta distribution with  = 35 and  = 175 + g .
(iii)
Under all-or-nothing loss we need to find the mode of
f ( px )  p 34 (1  p ) 174+ g
 f
= 34 p 33 (1  p ) 174+ g  (174 + g ) p 34 (1  p ) 173+ g
 p
= p 33 (1  p ) 173+g [34(1  p )  (174 + g ) p ]
And setting this equal to zero we have
34(1  p ˆ ) = (174  g ) p ˆ
34 = (208  g ) p ˆ
p̂ =
34
208  g
Under quadratic loss the Bayes estimate is given by mean of the posterior
distribution which is given by
p̂ =
35
35
=
35  175  g 210  g
%---------------------------------------------------------------------%
The two are equal if
34
35
=
208  g
210  g
i.e. 34(210 + g ) = 35(208 + g )
i.e. 7140 + 34 g = 7280 + 35 g
i.e. g = 7140  7280 =  140
This would imply there were  40 policies in year 2 which is impossible. Disappointingly few candidates knew the bookwork for part (i). Stronger candidates familiar with Bayes Theory were able to score very well here.
\end{document}
