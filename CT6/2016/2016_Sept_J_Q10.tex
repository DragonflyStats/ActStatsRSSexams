\documentclass[a4paper,12pt]{article}

%%%%%%%%%%%%%%%%%%%%%%%%%%%%%%%%%%%%%%%%%%%%%%%%%%%%%%%%%%%%%%%%%%%%%%%%%%%%%%%%%%%%%%%%%%%%%%%%%%%%%%%%%%%%%%%%%%%%%%%%%%%%%%%%%%%%%%%%%%%%%%%%%%%%%%%%%%%%%%%%%%%%%%%%%%%%%%%%%%%%%%%%%%%%%%%%%%%%%%%%%%%%%%%%%%%%%%%%%%%%%%%%%%%%%%%%%%%%%%%%%%%%%%%%%%%%

\usepackage{eurosym}
\usepackage{vmargin}
\usepackage{amsmath}
\usepackage{graphics}
\usepackage{epsfig}
\usepackage{enumerate}
\usepackage{multicol}
\usepackage{subfigure}
\usepackage{fancyhdr}
\usepackage{listings}
\usepackage{framed}
\usepackage{graphicx}
\usepackage{amsmath}
\usepackage{chng%%-- Page}

%\usepackage{bigints}
\usepackage{vmargin}

% left top textwidth textheight headheight

% headsep footheight footskip

\setmargins{2.0cm}{2.5cm}{16 cm}{22cm}{0.5cm}{0cm}{1cm}{1cm}

\renewcommand{\baselinestretch}{1.3}

\setcounter{MaxMatrixCols}{10}

\begin{document}
Claims on portfolio of insurance policies arise as a Poisson process with parameter
$\lambda = 125$. Individual claim amounts, Xi follow a gamma distribution with parameters
$\alpha = 20$ and $\beta = 0.5$.
The insurance company calculates premiums using a premium loading factor of 15\%
and has an initial surplus of 300.
\begin{enumerate}[(a)]
\item (i) Define the adjustment coefficient R. 
\item (ii) Show that for this portfolio the value of R is 0.00648 correct to three
significant figures. 
\item (iii) (a) Calculate an upper bound for $\psi(300)$.
(b) Calculate an estimate of $\psi_1(300,1)$, using a Normal approximation. 
The parameter $\beta$ now reduces to 0.4.
\item (iv) Explain what would happen to the estimate of $\psi_1(300,1)$, without carrying out
any further calculations. 
\item (v) Propose two ways in which the insurance company could reduce $\psi_1(300,1)$.
\end{enumerate}
%%[Total 15]
%%END OF PAPER

%%%%%%%%%%%%%%%%%%%%%%%%%%%%%%%%%%%%%%%%%%%%%%%%%%%%%%%%%%%%%%%%%%%%%%%%%
%%-- Page 12
Q10 (i) The adjustment coefficient is the unique positive root of the equation
\lambdaMX (R) = \lambda + cR 
(ii) 125* 20 *1.15 5750
0.5
c =   =
   
and
( ) ( ) 20 1 2 X M R R− = − 
So R is the root of
f (R) =1 25(1− 2R)−20 −125−5750R = 0
\begin{enumerate}
\item When R is 0.006475 then f (R) = −0.00282
\item When R is 0.006485 then f (R) = 0.005433 
\item Since the function changes sign between 0.006475 and 0.006485 the unique
positive root must lie between these values hence R is 0.00648 to 3 sf 
\item (iii) By Lundberg’s inequality \psi(300) < exp(−300*.00648) = 0.143 
\item Total claims have a mean claim amount of 125 * 40 = 5000 [½]
And variance 125*(80 + 402 ) = 210 000 
\item So approximately
\psi1 (300) = P(300 + 5750 − N (5000, 210000) < 0 )
(0,1) 6050 5000
210000
\end{enumerate}
P N −  =  > 
 

= P(N (0,1) > 2.291) = 0.011 
(iv) The probability would increase, since both the mean and variance of claim
amounts are higher. 
%%%%%%%%%%%%%%%%%%%%%%%%%%%%%%%%%%%%%%%%%%%%%%% – September 2016 – Examiners’ Report
%%%%-- Page 13
(v) They could use a higher initial surplus or a higher premium loading. 
[Total 15]
%%%%
\newpage
This question was typically answered very well. Candidates who struggled
with part (ii) should note the method used in the answer. Most candidates
were able to give good explanations for parts (iv) and (v) and therefore scored
well.


\end{document}
