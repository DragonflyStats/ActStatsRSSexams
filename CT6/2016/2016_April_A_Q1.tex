\documentclass[a4paper,12pt]{article}

%%%%%%%%%%%%%%%%%%%%%%%%%%%%%%%%%%%%%%%%%%%%%%%%%%%%%%%%%%%%%%%%%%%%%%%%%%%%%%%%%%%%%%%%%%%%%%%%%%%%%%%%%%%%%%%%%%%%%%%%%%%%%%%%%%%%%%%%%%%%%%%%%%%%%%%%%%%%%%%%%%%%%%%%%%%%%%%%%%%%%%%%%%%%%%%%%%%%%%%%%%%%%%%%%%%%%%%%%%%%%%%%%%%%%%%%%%%%%%%%%%%%%%%%%%%%

\usepackage{eurosym}
\usepackage{vmargin}
\usepackage{amsmath}
\usepackage{graphics}
\usepackage{epsfig}
\usepackage{enumerate}
\usepackage{multicol}
\usepackage{subfigure}
\usepackage{fancyhdr}
\usepackage{listings}
\usepackage{framed}
\usepackage{graphicx}
\usepackage{amsmath}
\usepackage{chngpage}
%\usepackage{bigints}
\usepackage{vmargin}

% left top textwidth textheight headheight

% headsep footheight footskip
\setmargins{2.0cm}{2.5cm}{16 cm}{22cm}{0.5cm}{0cm}{1cm}{1cm}
\renewcommand{\baselinestretch}{1.3}
\setcounter{MaxMatrixCols}{10}
\begin{document}
CT6 A2016–2
1 (i) Derive the median of a Pareto distribution with parameters  and . [3]
Let  = 2 and  = 3.
(ii) Comment on the skewness of this Pareto distribution.
%%%%%%%%%%%%%%%%%%%%%%
\newpage
  2 A portfolio of insurance policies has two types of claims:
   Loss amounts for Type I claims are exponentially distributed with mean 120.
 Loss amounts for Type II claims are exponentially distributed with mean 110.
25% of claims are Type I, and 75% are Type II.
\begin{enumerate}
\item (i) Calculate the mean and variance of the loss amount for a randomly chosen claim.
An actuary wants to model randomly chosen claims using an exponential distribution as an approximation.
\item (ii) Explain whether this is a good approximation.
\end{enumerate}

%%%%%%%%%%%%%%%%%%%%%%%%%%%%%%%%%%%%%%%%%%%%%%%%%%%%%%%%%%%%%%
\newpage
Q1 (i) m = P(x ≤ m) =
  [1]
CDF = 1
x
 λ α −   λ + 
so 1
m 2
 λ α   =  λ + 
[1]
So
1
m  2 α 1 = λ  − 
 
[1]
[Total 3]
(ii) median = 3 ( 2 − 1) = 1.2426 [1]
mean = 3/(2 − 1) = 3 [1]
1
2
%%Subject CT6 (Statistical Methods Core Technical) – April 2016 – Examiners’ Report
%%Page 3
So this distribution has positive skew (mean > median) [1]
This is common for a Pareto distribution (or any relevant comment) 
%This straightforward question was well answered by most candidates, although a few erroneously used the formula for the coefficient of skewness, which is not applicable in this case.

\end{document}
