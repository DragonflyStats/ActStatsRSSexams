\documentclass[a4paper,12pt]{article}

%%%%%%%%%%%%%%%%%%%%%%%%%%%%%%%%%%%%%%%%%%%%%%%%%%%%%%%%%%%%%%%%%%%%%%%%%%%%%%%%%%%%%%%%%%%%%%%%%%%%%%%%%%%%%%%%%%%%%%%%%%%%%%%%%%%%%%%%%%%%%%%%%%%%%%%%%%%%%%%%%%%%%%%%%%%%%%%%%%%%%%%%%%%%%%%%%%%%%%%%%%%%%%%%%%%%%%%%%%%%%%%%%%%%%%%%%%%%%%%%%%%%%%%%%%%%

\usepackage{eurosym}
\usepackage{vmargin}
\usepackage{amsmath}
\usepackage{graphics}
\usepackage{epsfig}
\usepackage{enumerate}
\usepackage{multicol}
\usepackage{subfigure}
\usepackage{fancyhdr}
\usepackage{listings}
\usepackage{framed}
\usepackage{graphicx}
\usepackage{amsmath}
\usepackage{chng%%-- Page}
%\usepackage{bigints}
\usepackage{vmargin}

% left top textwidth textheight headheight

% headsep footheight footskip
\setmargins{2.0cm}{2.5cm}{16 cm}{22cm}{0.5cm}{0cm}{1cm}{1cm}
\renewcommand{\baselinestretch}{1.3}
\setcounter{MaxMatrixCols}{10}
\begin{document}

%%%%%%%%%%%%%%%%%%%%%%%%%%%%%%%%%%%%%%%%%%%%%%%%%%%%%%%%%%%%%%%%%%%%%%%%%%%%%%%%%%%%%%%%%%%
2 Andy is playing a game, which involves rolling four-sided fair dice. Each time a dice is rolled, it is equally likely to show one of the numbers: 1, 2, 3 or 4.
Before each roll, he has three strategies:
  a1: Receive 1.5 times the number showing.
a2: Receive half the number showing if it is odd, and twice the number if it is even.
a3: Receive the number showing if it is even, and twice the number if it is odd.
(i) Construct Andy’s payoff matrix. 
(ii) State which, if any, of the decision functions are dominated. 
(iii) Determine Andy’s optimal strategy under the Bayes criterion. 

%%%%%%%%%%%%%%%%%%%%%%%%%%%%%%%%%%%%%%%%%%%%%%%%%%%%%%%%%%%%%%%%%%%%%%%%%%%%%%%%%%%%%%%%%%%

%%%%%%%%%%%%%%%%%%%%%%%%%%%%%%%%%%%%%%%%%%%%%%%%%%%%%%%%%%5
Q2 (i) Let θi be the state of nature when the roll of the die = i.
Then the payoff matrix is:
  θ1 θ2 θ3 θ4
a1 1.5 3 4.5 6
a2 0.5 4 1.5 8
a3 2 2 6 4
[1 mark for first correct row, ½ mark thereafter]
(ii) None of the decision functions is dominated. 
(iii) Since each number is equally likely, this is equivalent to summing up the payoffs for each decision function. 
This is 15, 14 and 14. 
So a1 is the optimal decision under the Bayes criterion. 
This straightforward question was very well answered by most candidates, with many scoring full marks.
%%%%%%%%%%%%%%%%%%%%%%%%%%%%%%%%%%%%%%%%%%%%% – September 2016 – Examiners’ Report
%%-- Page 4
\end{document}
