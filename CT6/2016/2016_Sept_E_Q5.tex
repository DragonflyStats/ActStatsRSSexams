\documentclass[a4paper,12pt]{article}

%%%%%%%%%%%%%%%%%%%%%%%%%%%%%%%%%%%%%%%%%%%%%%%%%%%%%%%%%%%%%%%%%%%%%%%%%%%%%%%%%%%%%%%%%%%%%%%%%%%%%%%%%%%%%%%%%%%%%%%%%%%%%%%%%%%%%%%%%%%%%%%%%%%%%%%%%%%%%%%%%%%%%%%%%%%%%%%%%%%%%%%%%%%%%%%%%%%%%%%%%%%%%%%%%%%%%%%%%%%%%%%%%%%%%%%%%%%%%%%%%%%%%%%%%%%%

\usepackage{eurosym}
\usepackage{vmargin}
\usepackage{amsmath}
\usepackage{graphics}
\usepackage{epsfig}
\usepackage{enumerate}
\usepackage{multicol}
\usepackage{subfigure}
\usepackage{fancyhdr}
\usepackage{listings}
\usepackage{framed}
\usepackage{graphicx}
\usepackage{amsmath}
\usepackage{chngpage}
%\usepackage{bigints}
\usepackage{vmargin}

% left top textwidth textheight headheight

% headsep footheight footskip
\setmargins{2.0cm}{2.5cm}{16 cm}{22cm}{0.5cm}{0cm}{1cm}{1cm}
\renewcommand{\baselinestretch}{1.3}
\setcounter{MaxMatrixCols}{10}
\begin{document} 

5 (i) (a) Explain what is meant by a sequence of independent, identically
distributed (I.I.D.) random variables.
(b) Give one example of a sequence of I.I.D. random variables.

Claim amounts Xi from a portfolio of insurance policies are assumed to be I.I.D. and
exponentially distributed, with parameter \lambda  . In a given year there are n claims.
(ii) Show that the total claim amounts follow a gamma distribution, specifying its parameters. 
In practice the individual claim amounts are not I.I.D. but instead the exponential parameter $\lambda_{i}$ varies between each claim. \lambda  i follows a gamma distribution with parameters \alpha  and \beta .
(iii) Show that the marginal distribution of claim amounts follows a Pareto distribution with parameters \alpha  and \beta . 
[Total 10]
\item

%%-- [Total 10]

\newpage
%%%%%%%%%%%%%%%%%%%%%%%%%%%%%%%%%%%%%%%%%%%%%%%%%%%%%%%%%%%%%%%%%%%%%%%%%%
  Q5 (i) (a) Each realisation of the variable is unaffected by previous outcomes and
in turn does not affect future outcomes. 
The variables all come from the same distribution with the same
parameters. 
(b) E.g. rolling a fair die, tossing a fair coin etc. 

Page 6
(ii) Let
1
n
i
i
S X
=
  = , then
( ) ( )
1
1
[ ] 1 1 i
n n n
n
S X X
i
M t M M t t t
− −
=
        = = =  −   = −    \lambda      \lambda    
Π 
By independence of claim amounts and uniqueness property of MGFs 
This is a gamma distribution with parameters n and \lambda   . 
(iii) ( ) ( ) ( ) ( )
( ) ( ) ( )
, |
  0 0
1
0
, |
  /Γ exp
fX x fX x d f fX x d
exp x d
∞ ∞
\lambda   \lambda   \lambda  
∞
\alpha  \alpha −
= \lambda   \lambda  = \lambda   \lambda   \lambda  
= \beta  \alpha  \lambda   −\beta \lambda   \lambda   −\lambda   \lambda  
 

( ) { ( ) }
0
exp
Γ
x d
\alpha  ∞
= \beta  \lambda  \alpha  − + \beta  \lambda   \lambda  
\alpha   
( ) ( )
( )
( )
( ) { ( ) }
1
1
0
Γ 1
\Γ exp
Γ 1
x
x d
x
∞ \alpha +
  \alpha  \alpha 
\alpha +
  \alpha  + +\beta 
=\beta  \alpha  \lambda   − + \beta  \lambda   \lambda  
+ \beta  \alpha  +  
( ) ( )
( ) 1 ( ) 1
Γ 1
\ Γ , x 0
x x
\alpha 
\alpha 
\alpha + \alpha +
  \alpha  + \alpha \beta  =\beta  \alpha  = >
  +\beta  +\beta 

Since the final integral is the PDF of a Gamma distribution and so equals 1.
This is the PDF of a Pareto distribution with parameters \alpha  and \beta  . 
[Total 10]
Part (i) was poorly answered, with many candidates simply repeating the
words independent and identical. Part (ii) was well answered, although part
(iii) was relatively poorly answered.
\end{document}
