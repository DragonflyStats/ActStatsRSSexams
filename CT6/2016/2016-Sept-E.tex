\documentclass[a4paper,12pt]{article}

%%%%%%%%%%%%%%%%%%%%%%%%%%%%%%%%%%%%%%%%%%%%%%%%%%%%%%%%%%%%%%%%%%%%%%%%%%%%%%%%%%%%%%%%%%%%%%%%%%%%%%%%%%%%%%%%%%%%%%%%%%%%%%%%%%%%%%%%%%%%%%%%%%%%%%%%%%%%%%%%%%%%%%%%%%%%%%%%%%%%%%%%%%%%%%%%%%%%%%%%%%%%%%%%%%%%%%%%%%%%%%%%%%%%%%%%%%%%%%%%%%%%%%%%%%%%

\usepackage{eurosym}
\usepackage{vmargin}
\usepackage{amsmath}
\usepackage{graphics}
\usepackage{epsfig}
\usepackage{enumerate}
\usepackage{multicol}
\usepackage{subfigure}
\usepackage{fancyhdr}
\usepackage{listings}
\usepackage{framed}
\usepackage{graphicx}
\usepackage{amsmath}
\usepackage{chng%%-- Page}

%\usepackage{bigints}
\usepackage{vmargin}

% left top textwidth textheight headheight

% headsep footheight footskip

\setmargins{2.0cm}{2.5cm}{16 cm}{22cm}{0.5cm}{0cm}{1cm}{1cm}

\renewcommand{\baselinestretch}{1.3}

\setcounter{MaxMatrixCols}{10}

\begin{document}

\begin{enumerate}
%%%%%%%%%%%%%%%%%%%%%%%%%

%%-- CT6 S2016–6
9 In order to model the seasonality of a particular data set an actuary is asked to
consider the following model:
  (1 12 )(1 ( ) 2 ) − B − \alpha  + \beta B + \alpha \betaB Xt = \varepsilont
where B is the backshift operator and \varepsilont is a white noise process with variance \sigma2.
The actuary intends to apply a seasonal difference ∇s Xt = Yt.
(i) Explain why s should be 12 in this case (i.e. Yt = Xt – Xt−12). 
(ii) Determine the range of values for \alpha  and \beta for which the process will be
stationary after applying this seasonal difference. 
Assume that after the appropriate seasonal differencing the following sample
autocorrelation values for observations of Yt are \hat{\rho}1 = 0 and \hat{\rho}2 = 0.09.
(iii) Estimate the parameters \alpha  and \beta. 
The actuary observes a sequence of observations x1, x2, …, xT of Xt, with T > 12.
(iv) Derive the next two forecasted values for next two observations xˆT+1 and
xˆT +2, as a function of the existing observations. 
[Total 13]

%%%%%%%%%%%%%%%%%%%%%%%%%%%%%%%%%%%%%%%%%%%%%%%%%%%%%%%%%%%%%%%%%%%%%%%%%%%%%%%%%%
  Q9 (i) The first term in the equation has period 12 and so this removes the periodic
effect. 
(ii) The characteristic polynomial will be 1− (\alpha  + \beta) B + \alpha \betaB2 
with roots 1/ \alpha  and 1/\beta . 
Hence the stationarity holds for \alpha  <1 and \beta <1. 
Subject CT6 %%%%%%%%%%%%%%%%%%%%%%%%%%%%%%%%%%%%%%%%%%%%%%% – September 2016 – Examiners’ Report
%%-- Page 11
(iii) Yt is an AR(2) where a1 = \alpha  + \beta and a2 = −\alpha \beta. Since from the Yule-walker
equations for AR(2) we have
\rho1 = a1 + a2\rho1 
and
\rho2 = a1\rho1 + a2 
which imply that a1 = (1− a2 )\rho1 = 0 since \rho1 = 0 . 
This implies that \alpha  +\beta = 0, \alpha  = −\beta 
and the second equation 2
0.09=\rho2 =a1\rho1+a2 =a2 =\alpha  i.e. \alpha  = −\beta = ±0.3. 
(iv) Since 12 t t t Y X X−
= − we have that
XT +1 = YT +1 + XT −11 [½]
XT +2 = YT +2 + XT −10 [½]
With the forecasted values
xˆT +1 = yˆT +1 + xT −11 [½]
and
xˆT +2 = yˆT +2 + xT −10 [½]
where
yˆT +1 = 0* yT + 0.09 yT −1 = 0.09(xT −1 − xT −13 ) 
And similarly
yˆT +2 = 0.09 (xT − xT −12 ) 
[Total 13]
The performance on this time series question was very good, although only
the stronger candidates were able to score well on part (iv).
Subject CT6 %%%%%%%%%%%%%%%%%%%%%%%%%%%%%%%%%%%%%%%%%%%%%%% – September 2016 – Examiners’ Report
\end{document}
