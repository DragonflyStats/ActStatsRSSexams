\documentclass[a4paper,12pt]{article}
%%%%%%%%%%%%%%%%%%%%%%%%%%%%%%%%%%%%%%%%%%%%%%%%%%%%%%%%%%%%%%%%%%%%%%%%%%%%%%%%%%%%%%%%%%%%%%%%%%%%%%%%%%%%%%%%%%%%%%%%%%%%%%%%%%%%%%%%%%%%%%%%%%%%%%%%%%%%%%%%%%%%%%%%%%%%%%%%%%%%%%%%%%%%%%%%%%%%%%%%%%%%%%%%%%%%%%%%%%%%%%%%%%%%%%%%%%%%%%%%%%%%%%%%%%%%
\usepackage{eurosym}
\usepackage{vmargin}
\usepackage{amsmath}
\usepackage{graphics}
\usepackage{epsfig}
\usepackage{enumerate}
\usepackage{multicol}
\usepackage{subfigure}
\usepackage{fancyhdr}
\usepackage{listings}
\usepackage{framed}
\usepackage{graphicx}
\usepackage{amsmath}
\usepackage{chng%%-- Page}
%\usepackage{bigints}
\usepackage{vmargin}

% left top textwidth textheight headheight

% headsep footheight footskip

\setmargins{2.0cm}{2.5cm}{16 cm}{22cm}{0.5cm}{0cm}{1cm}{1cm}
\renewcommand{\baselinestretch}{1.3}
\setcounter{MaxMatrixCols}{10}
\begin{document}


6 Felicity is a fund manager who is considering investing €1m in a specialist
investment contract where the return depends on the performance of a particular
company. She has a choice between two contracts as follows:
   Long contract: if the company is deemed a “success”, the investment will return
+100\% and if it is deemed a “failure” it will return –75%.
 Short contract: if the company is deemed a “success”, the investment will return
–50\% and if it is deemed a “failure” it will return +50%.
Before she decides which contract to invest in, Felicity will be able to observe the
investment performance of the company’s shares relative to the stock market.
Companies that are “successes” have a 60\% probability of outperforming the stock
market. Companies that are “failures” have a 40\% probability of outperforming the
market.
\begin{enumerate}

\item (i) List Felicity’s four decision functions. 
\item (ii) Calculate the values of the risk function for each decision function and type
of company. 
Two thirds of such companies under consideration are known to be failures.
\item (iii) Determine Felicity’s optimal decision function. 
\end{enumerate}
%%%%%%%%%%%%%%%%%%%%%%%%%%%%%%%%%%%%%%%%%%%%%%%%%%%%%%%%%%%%%%
\newpage 
Q6 (i) There are two possible outcomes of the observation and two possible choices
hence $2 \times 2 = 4$ decision functions.
Decision Fn/Stock Outperforms Underperforms
d1 Long Long
d2 Long Short
d3 Short Long
d4 Short Short
[½ for each row]
[Max 2]
(ii) If good and invest make 100% of €1m = €1m
If bad and invest lose 75% of €1m = −€0.75m 
R(d1|Good) = 1m
R(d1|Bad) = –0.75m 
R(d2|Good) = 1m * 60% outperformance – 0.5m * 40% = 0.4m
R(d2|Bad) = −.75m * 40% + 0.5m * 60% = 0m [1½]
R(d3|Good) = −0.5m * 60% + 1m * 40% = 0.1m
R(d3|Bad) = 0.5m * 40% −.75m * 60% = –0.25m [1½]
R(d4|Good) = –0.5m
R(d4|Bad) = 0.5m 
[Total 6]
(iii) We need to determine the expectation of each Risk function
D2 dominates D3 [½]
E(R(d1)) = 1/3 * 1m – 2/3 * .75m = −1/6m
E(R(d2)) = 1/3 * 2/5m = 2/15m
E(R(d4)) = 1/3 * − 1/2m + 2/3 * 1/2m = 1/6m [1½]
So d4 is the optimal decision function under the Bayes criterion. 
[Total 3]
[TOTAL 11]
% Apart from question 10, this was the most challenging question on the paper.
% Candidates who were able to identify the decision functions and set up the problem correctly generally did well, but many candidates struggled to formulate their answers.
% %%%%%%%%%%%%%%%%%%%%%%%%%%%%%%%%%%%%%%%%%%%%% – April 2016 – Examiners’ Report
\end{document}
