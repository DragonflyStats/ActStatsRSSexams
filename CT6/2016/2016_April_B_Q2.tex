\documentclass[a4paper,12pt]{article}

%%%%%%%%%%%%%%%%%%%%%%%%%%%%%%%%%%%%%%%%%%%%%%%%%%%%%%%%%%%%%%%%%%%%%%%%%%%%%%%%%%%%%%%%%%%%%%%%%%%%%%%%%%%%%%%%%%%%%%%%%%%%%%%%%%%%%%%%%%%%%%%%%%%%%%%%%%%%%%%%%%%%%%%%%%%%%%%%%%%%%%%%%%%%%%%%%%%%%%%%%%%%%%%%%%%%%%%%%%%%%%%%%%%%%%%%%%%%%%%%%%%%%%%%%%%%

\usepackage{eurosym}
\usepackage{vmargin}
\usepackage{amsmath}
\usepackage{graphics}
\usepackage{epsfig}
\usepackage{enumerate}
\usepackage{multicol}
\usepackage{subfigure}
\usepackage{fancyhdr}
\usepackage{listings}
\usepackage{framed}
\usepackage{graphicx}
\usepackage{amsmath}
\usepackage{chngpage}
%\usepackage{bigints}
\usepackage{vmargin}

% left top textwidth textheight headheight

% headsep footheight footskip
\setmargins{2.0cm}{2.5cm}{16 cm}{22cm}{0.5cm}{0cm}{1cm}{1cm}
\renewcommand{\baselinestretch}{1.3}
\setcounter{MaxMatrixCols}{10}
\begin{document}

  2 A portfolio of insurance policies has two types of claims:
   Loss amounts for Type I claims are exponentially distributed with mean 120.
 Loss amounts for Type II claims are exponentially distributed with mean 110.
25\% of claims are Type I, and 75\% are Type II.
\begin{enumerate}
\item (i) Calculate the mean and variance of the loss amount for a randomly chosen claim.
An actuary wants to model randomly chosen claims using an exponential distribution as an approximation.
\item (ii) Explain whether this is a good approximation.
\end{enumerate}

%%%%%%%%%%%%%%%%%%%%%%%%%%%%%%%%%%%%%%%%%%%%%%%%%%%%%%%%%%%%%%
\newpage

%%%%%%%%%%%%%%%%%%%%%%%%%%%%%%%%%%%%%%%%%%%%%%%%%%%%%%%%%%%%%%%%%%%%%
Q2 (i) Mean = 14 \lambda 1 + 34 \lambda 2 =112.5
2 2 2 2 2
E(X ) = 14 (\lambda 1 + \lambda 1 ) + 34 (\lambda 2 + \lambda 2 ) = 25350 [1]
so variance = 25350 – 112.52 = 12,694 [1]
[Total 3]
(ii) 
\begin{itemize}
\item For an exponential, mean = standard deviation and they are pretty close, so yes
this is a good approximation. 
\item Alternatively
In general, the sum of two exponentials is not exponential, so this is not a
good approximation. 
\end{itemize}
%This question was relatively poorly answered, with only the better candidates being able to derive the variance. This is disappointing given how often this type of question occurs.
\end{document}
