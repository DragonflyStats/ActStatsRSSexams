\documentclass[]{report}

\voffset=-1.5cm
\oddsidemargin=0.0cm
\textwidth = 480pt

\usepackage{framed}
\usepackage{subfiles}
\usepackage{graphics}
\usepackage{newlfont}
\usepackage{eurosym}
\usepackage{amsmath,amsthm,amsfonts}
\usepackage{amsmath}
\usepackage{color}
\usepackage{amssymb}
\usepackage{multicol}
\usepackage[dvipsnames]{xcolor}
\usepackage{graphicx}
\begin{document}
3 The table below shows aggregate annual claim statistics for four risks over a period of
six years. Annual aggregate claims for risk i in year j are denoted by Xij.
( ) 6 6 2 2
1 1
1 1 ,
6 5
1 46.8 1227.4
2 30.2 1161.4
3 74.5 1340.3
4 60.7 1414.7
i ij i ij i
j j
Risk i X X S X X
i
i
i
i
= =
  = = −
=
  =
  =
  =
   
(i) Calculate the credibility premium of each risk under the assumptions of
Empirical Bayes Credibility Theory (EBCT) Model 1. [7]
(ii) Comment on why the credibility factor is relatively low in this case. [2]
[Total 9]
%%%%%%%%%%%%%%%%%%%%%%%%%%%%%%%%%5
% CT6 S2016–3 PLEASE TURN OVER

%%%%%%%%%%%%%%%%%%%%%%%%%%%%%%%%%%%%%%%%%%%%%%%%%%%%%%%%%%%%%%%%%%%%%%%
\newpage
Q3 (i) Overall mean is 46.8 30.2 74.5 60.7 53.05
4
X = + + + = 
4
2 2
1
( )) 1 1227.4 1161.4 1340.3 1414.7 1285.95
4 4
( i
  i
  E s S
  =
    θ =  = + + + = [1]
  ( ( )) ( ) ( ( )) 4 2 2
  1
  Var 1 1
  3 i 6
  i
  m X X E S
  =
    θ =  − − θ [1]
  (46.8 53.05)2 (30.2 53.05)2 (74.5 53.05)2 (60.7 53.05)2
  3
  1285.95
  6
  − + − + − + −
  =
    −
  = 145.6 [1]
  So the credibility factor is 6 0.4045
  6 1 285.95145.6
  Z= =
    +
    [1]
  And the credibility premia are:
    (1) 0.4045 * 46.8+ 0.5955 * 53.05 = 50.5 [½]
  (2) 0.4045* 30.2 + 0.5955 * 53.05 = 43.8 [½]
  (3) 0.4045 * 74.5 + 0.5955 * 53.05 = 61.7 [½]
  (4) 0.4045 * 60.7 + 0.5955 * 53.05 = 56.1 [½]
  (ii) The variation within risks is much bigger relative to the variation between
  risks. This suggests that the variability is more explained by claim variability
  than in the underlying parameter, so we put more weight on the information
  provided by the data set as a whole, and less on the individual risks, resulting
  in a low credibility factor. [2]
  [Total 9]
  Many candidates scored full marks on part (i), but scored less well on part (ii).
  Stronger candidates were able to explain in words what was happening,
  beyond simply using mathematical formulae.

  
\end{document}
