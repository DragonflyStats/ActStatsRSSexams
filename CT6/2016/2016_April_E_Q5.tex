\documentclass[a4paper,12pt]{article}
%%%%%%%%%%%%%%%%%%%%%%%%%%%%%%%%%%%%%%%%%%%%%%%%%%%%%%%%%%%%%%%%%%%%%%%%%%%%%%%%%%%%%%%%%%%%%%%%%%%%%%%%%%%%%%%%%%%%%%%%%%%%%%%%%%%%%%%%%%%%%%%%%%%%%%%%%%%%%%%%%%%%%%%%%%%%%%%%%%%%%%%%%%%%%%%%%%%%%%%%%%%%%%%%%%%%%%%%%%%%%%%%%%%%%%%%%%%%%%%%%%%%%%%%%%%%
\usepackage{eurosym}
\usepackage{vmargin}
\usepackage{amsmath}
\usepackage{graphics}
\usepackage{epsfig}
\usepackage{enumerate}
\usepackage{multicol}
\usepackage{subfigure}
\usepackage{fancyhdr}
\usepackage{listings}
\usepackage{framed}
\usepackage{graphicx}
\usepackage{amsmath}
\usepackage{chng%%-- Page}
%\usepackage{bigints}
\usepackage{vmargin}

% left top textwidth textheight headheight

% headsep footheight footskip

\setmargins{2.0cm}{2.5cm}{16 cm}{22cm}{0.5cm}{0cm}{1cm}{1cm}
\renewcommand{\baselinestretch}{1.3}
\setcounter{MaxMatrixCols}{10}
\begin{document}

\begin{enumerate}
5 (i) Explain why insurance companies make use of run-off triangles. 
(ii) The run-off triangle below shows incremental claims incurred on a portfolio of
general insurance policies.
Development Year
Policy Year 0 1 2 3
2011 4,657 3,440 931 572
2012 6,089 5,275 1,381
2013 5,623 4,799
2014 7,224
Calculate the outstanding claims reserve for this portfolio using the basic
chain ladder method. 
\newpage 
%%%%%%%%%%%%%%%%%%%%%%%%%%%%%%%%%%%%%%%%%%%%%%%%%%%%%%%%%%%%%%
  Q5 (i) There is normally a delay between incidents leading to claim and the insurance
pay out 
Insurance companies need to estimate future claims for their reserve 
It makes sense to use historical data to infer future patterns of claims 
[Max 2]
(ii) Cumulate claims
Policy
Year
0 1 2 3
2011 4,657 8,097 9,028 9,600
2012 6,089 11,364 12,745 13,553
2013 5,623 10,422 12,399
2014 7,224 15,690
DF 2,3 = 9600/9028 = 1.063 358 
DF 1,2 = (9028 + 12745) / (8097 + 11364) = 1.118 802 
DF 0,1 = (8097 + 11364 + 10422) / (4657 + 6089 + 5623) = 1.825 585 
Find expected claims 12,745 * 1.063 358 = 13,553 
10,422 * 1.118 802 * 1.063358 = 12,399 
7,224 * 1.825 585 * 1.118802 * 1.063358 = 15,690 
So claim reserve =
  (13,553 – 12,745) + (12,399 – 10,422) + (15,690 – 7,224) = 11,250 (4sf) 
Assuming that claims incurred are equal to claims paid

% This straightforward question on chain ladders was very well answered.
%--------------%%%%%%%%%%%%%%%%%%%%%%%%%%%%%%%%%%%%%%%%%%%%% – April 2016 – Examiners’ Report

\end{document}
