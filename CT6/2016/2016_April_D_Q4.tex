\documentclass[a4paper,12pt]{article}

%%%%%%%%%%%%%%%%%%%%%%%%%%%%%%%%%%%%%%%%%%%%%%%%%%%%%%%%%%%%%%%%%%%%%%%%%%%%%%%%%%%%%%%%%%%%%%%%%%%%%%%%%%%%%%%%%%%%%%%%%%%%%%%%%%%%%%%%%%%%%%%%%%%%%%%%%%%%%%%%%%%%%%%%%%%%%%%%%%%%%%%%%%%%%%%%%%%%%%%%%%%%%%%%%%%%%%%%%%%%%%%%%%%%%%%%%%%%%%%%%%%%%%%%%%%%

\usepackage{eurosym}
\usepackage{vmargin}
\usepackage{amsmath}
\usepackage{graphics}
\usepackage{epsfig}
\usepackage{enumerate}
\usepackage{multicol}
\usepackage{subfigure}
\usepackage{fancyhdr}
\usepackage{listings}
\usepackage{framed}
\usepackage{graphicx}
\usepackage{amsmath}
\usepackage{chngpage}
%\usepackage{bigints}
\usepackage{vmargin}

% left top textwidth textheight headheight

% headsep footheight footskip
\setmargins{2.0cm}{2.5cm}{16 cm}{22cm}{0.5cm}{0cm}{1cm}{1cm}
\renewcommand{\baselinestretch}{1.3}
\setcounter{MaxMatrixCols}{10}
\begin{document}

4 Let us consider that we need to sample from a discrete random variable Y with
distribution function:
  Y 1 2 3
P 1
3 1
3 1
3
(i) Set out a direct method of sampling from Y. [2]
Consider now another random variable �� with distribution:
  X 1 2 3
P 1
2 1
3 1
6
(ii) Set out a direct method of sampling from X. [2]
(iii) (a) Explain how you can apply the acceptance-rejection method to
sample X by rejecting/accepting samples from Y.
(b) Calculate how many samples from Y on average are needed to
generate one sample from X. [6]
[Total 10]
%%%%%%%%%%%%%%%%%%%%%%%%%%%%%%%%%%%%%%%%%%%%%%%%%%%%%%%%%%%%%%%%%%%%%%%%%
Q4 (i) Algorithm
(1) Simulate one �� from U(0,1) 
(2) If 0 1 , 1; 1 2 , 2;else 3
3 3 3
<U ≤ Y = <U ≤ Y = Y = 
[Total 2]
(ii) Similarly for generating samples from ��
Algorithm
Simulate one U from U(0,1) 
If 0 1
2
≤U < , X=1 ; if 1 5
2 6
≤U < , X = 2, else X = 3 
[Total 2]
(iii) The rejection here applied in the same way as in the continuous case we need
to calculate again
1 1 1
max 2 , 3 , 6 max 3 ,1, 3 3 1 1 1 2 6 2
3 3 3
M
 
    =   =   =
     
 

Hence ( )
( )
1, 2 , 1
3 3
f X
M g x
=    
 
if X = 1, 2,3 respectively. 
Algorithm
1 – Simulate Y as in (i). 
2 – Sample U from U(0,1) and then take.
If Y = 1 X = Y [½]
U < 2/3 and if Y = 2 take X = Y [½]
If U < 1/3 and if Y = 3 take X = Y [½]
Subject CT6 (Statistical Methods Core Technical) – April 2016 – Examiners’ Report
Page 5
Otherwise start again. [½]
The average number of samples from Y needed for a single sample from X is
of the order M = 3/2, i.e. 1.5 samples on average. 
[Total 6]
[TOTAL 10]
Most candidates scored well on parts (i) and (ii), although only the better
prepared candidates were able to apply the acceptance-rejection method to
part (iii).
