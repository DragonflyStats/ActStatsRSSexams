\documentclass[a4paper,12pt]{article}



%%%%%%%%%%%%%%%%%%%%%%%%%%%%%%%%%%%%%%%%%%%%%%%%%%%%%%%%%%%%%%%%%%%%%%%%%%%%%%%%%%%%%%%%%%%%%%%%%%%%%%%%%%%%%%%%%%%%%%%%%%%%%%%%%%%%%%%%%%%%%%%%%%%%%%%%%%%%%%%%%%%%%%%%%%%%%%%%%%%%%%%%%%%%%%%%%%%%%%%%%%%%%%%%%%%%%%%%%%%%%%%%%%%%%%%%%%%%%%%%%%%%%%%%%%%%



\usepackage{eurosym}
\usepackage{vmargin}
\usepackage{amsmath}
\usepackage{graphics}
\usepackage{epsfig}
\usepackage{enumerate}
\usepackage{multicol}
\usepackage{subfigure}
\usepackage{fancyhdr}
\usepackage{listings}
\usepackage{framed}
\usepackage{graphicx}
\usepackage{amsmath}
\usepackage{chngpage}



%\usepackage{bigints}

\usepackage{vmargin}

% left top textwidth textheight headheight

% headsep footheight footskip

\setmargins{2.0cm}{2.5cm}{16 cm}{22cm}{0.5cm}{0cm}{1cm}{1cm}
\renewcommand{\baselinestretch}{1.3}
\setcounter{MaxMatrixCols}{10}
\begin{document}
\begin{enumerate}


9 Consider the following time series model:
  \[Yt = 1 + 0.6Yt1 + 0.16Yt2 + t\]
  where t is a white noise process with variance \sigma2.
  \begin{enumerate}[(i)]
\item  (i) Determine whether Yt is stationary and identify it as an ARMA(p,q) process.
\item
  (ii) Calculate 􀜧(Yt). 
 \item (iii) Calculate for the first four lags:
     the autocorrelation values 1, 2, 3, 4 and
   the partial autocorrelation values 1, 2, 3, 4. [7]
\end{enumerate}
  %%%%%%%%%%%%%%%%%%%%%%%%%%%%%%%%%%5
  Page 9
  Q9 (i) The lag polynomial here is 1 − 0.6L − 0.16L2 = (1 − 0.8L)(1 + 0.2L) with roots
  1.25 and –5 therefore it is stationary. 
  Hence an ARMA(2,0) process. 
  [Total 3]
  (ii) From the stationarity condition then
  E(Yt) = \mu = 1 + 0.6\mu + 0.16\mu + 0 
  \mu = 1
  1− 0.76
  = 1
  0.24
  
  [Total 2]
  (iii) From the Yule-Walker equations for autocorrelation function values and for
  lags 1, 2, 3, … we have that \rhok = 0.6\rhok−1 + 0.16\rhok−2. 
  In particular, for k = 1,
  \[\rho_1 = 0.6\rho0 + 0.16\rho_1 = 0.6 + 0.16\rho_1\]
  or \rho_1 =
    0.6
  1− 0.16 =
    0.6
  0.84 = 0.7143 = 5/7 
  \begin{itemize}
\item  For k = 2 we have that
  \rho2 = 0.6\rho_1 + 0.16\rho0 = 0.16\rho_1 + 0.6 =
    0.62
  0.84 + 0.16
  = 0.5886 = 103/175. 
\item  For k = 3
  \rho3 = 0.6\rho2 + 0.16\rho_1 = 0.6 * 0.5885714 + 0.16 * 0.7142857
  = 0.4674 = 409/875 
\item  For k = 4
  \rho4 = 0.6\rho3 + 0.16\rho2 = 0.6 * 0.4674286 + 0.16 * 0.5884714
  = 0.3746 = 1639/4375 
  For the partial autocorrelation function we have that
  ψ1 = \rho_1 = 0.7143 = 5/7 
\end{itemize}  
%   %%%%%%%%%%%%%%%%%%%%%%%%%%%%%%%%%%%%%%%%%%%%%%% – April 2016 – Examiners’ Report
%  Page 10
  ψ2 =
    2
  2 1
  2
  1 1
  \rho − \rho
  − \rho
  = 0.16 = 4/25 
  and ψ3 = ψ4 = 0 since Yt is AR(2). [½]
  [Total 7]
  [TOTAL 12]
%  This straightforward question on time series was well answered by many  candidates.
  
\end{document}
