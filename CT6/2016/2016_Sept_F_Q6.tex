\documentclass[a4paper,12pt]{article}

%%%%%%%%%%%%%%%%%%%%%%%%%%%%%%%%%%%%%%%%%%%%%%%%%%%%%%%%%%%%%%%%%%%%%%%%%%%%%%%%%%%%%%%%%%%%%%%%%%%%%%%%%%%%%%%%%%%%%%%%%%%%%%%%%%%%%%%%%%%%%%%%%%%%%%%%%%%%%%%%%%%%%%%%%%%%%%%%%%%%%%%%%%%%%%%%%%%%%%%%%%%%%%%%%%%%%%%%%%%%%%%%%%%%%%%%%%%%%%%%%%%%%%%%%%%%

\usepackage{eurosym}
\usepackage{vmargin}
\usepackage{amsmath}
\usepackage{graphics}
\usepackage{epsfig}
\usepackage{enumerate}
\usepackage{multicol}
\usepackage{subfigure}
\usepackage{fancyhdr}
\usepackage{listings}
\usepackage{framed}
\usepackage{graphicx}
\usepackage{amsmath}
\usepackage{chngpage}
%\usepackage{bigints}
\usepackage{vmargin}

% left top textwidth textheight headheight

% headsep footheight footskip
\setmargins{2.0cm}{2.5cm}{16 cm}{22cm}{0.5cm}{0cm}{1cm}{1cm}
\renewcommand{\baselinestretch}{1.3}
\setcounter{MaxMatrixCols}{10}
\begin{document} 

6 Assume that the numbers of accidents for three different risks in five years are as
follows:
  Year 1 Year 2 Year 3 Year 4 Year 5 Total
Risk A 1 4 5 0 2 12
Risk B 1 6 4 6 5 22
Risk C 5 6 4 9 4 28
An actuary is modelling each risk according to a Poisson distribution.
(i) Determine the Poisson parameter for each risk using the method of maximum likelihood estimation.
(ii) Test the hypothesis that the three risks have the same claim rate, using the scaled deviances. 

\newpage
%%-- [Total 10]

\newpage
%%%%%%%%%%%%%%%%%%%%%%%%%%%%%%%%%%%%%%%%%%%%%%%%%%%%%%%%%%%%%%
Q6 (i) For risk A with rate μ1 the log-likelihood function is:
  5 5
1 1 1 1 1
1 1
log log i 5 log i !
  i i
L y y
= =
  = μ  − μ −
Subject CT6 (Statistical Methods Core Technical) – September 2016 – Examiners’ Report
Page 7
5
1 1 1
1
= 12log 5 log i !
  i
y
=
  μ − μ − [1½]
And therefore the mle for μ1 is obtained for
1 1
∂logL = 12 −5 = 0
∂μ μ
[1]
i.e. 
μ1 = 2.4 [½]
Similarly we have that 2
22 4.4
5
μ = = and 
3
28 5.6
5
μ = = . [2]
(ii) Under the assumption that these risks share the same rate i.e. μ1 = μ2 = μ3 = μ
then the mle estimate for this is simply ˆ 62
15
μ = . [½]
In order to compare these models we can use the scaled deviances to compare
these models and using the chi-squared test.
The difference in the scaled deviance is chi-square with 3 − 1 = 2 degrees of
freedom. [1]
2(logL1 + logL2 + logL3 − logL)
     
= 2(12logμ1 − 5μ1 + 22logμ2 − 5μ2 + 28logμ3 − 5μ3 − 62logμˆ +15 μˆ ) [1]
With the
5
1
1
log i !
  i
y
= 
+
  5 5
2 3
1 1
log i ! log i !
  i i
y y
= =
   + cancelling out in the difference.
Hence
2( ) 1 2 3 logL + logL + logL − logL
2 12log2.4 22log 4.4 28log5.6 62log 62 5(12 22 28) 15 62
15 5 15
 + + 
=  + + − − + 
 
2 12log2.4 22log4.4 28log5.6 62log 62 6.71034
15
=  + + −  =  
 
. [1½]
Subject CT6 (Statistical Methods Core Technical) – September 2016 – Examiners’ Report
Page 8
This value is above 5.991 which is the critical value at the upper 5% level and
therefore conclude that mean claim rates are different. [1]
[Total 10]
Part (i) was very well answered by most candidates. Fewer candidates
scored full marks in part (ii), despite similar questions having been asked in
several recent examinations.
