\documentclass[]{report}

\voffset=-1.5cm
\oddsidemargin=0.0cm
\textwidth = 480pt

\usepackage{framed}
\usepackage{subfiles}
\usepackage{graphics}
\usepackage{newlfont}
\usepackage{eurosym}
\usepackage{amsmath,amsthm,amsfonts}
\usepackage{amsmath}
\usepackage{color}
\usepackage{amssymb}
\usepackage{multicol}
\usepackage[dvipsnames]{xcolor}
\usepackage{graphicx}
\begin{document}
3 The table below shows aggregate annual claim statistics for four risks over a period of
six years. Annual aggregate claims for risk i in year j are denoted by Xij.
( ) 6 6 2 2
1 1
1 1 ,
6 5
1 46.8 1227.4
2 30.2 1161.4
3 74.5 1340.3
4 60.7 1414.7
i ij i ij i
j j
Risk i X X S X X
i
i
i
i
= =
  = = −
=
  =
  =
  =
   
(i) Calculate the credibility premium of each risk under the assumptions of
Empirical Bayes Credibility Theory (EBCT) Model 1. [7]
(ii) Comment on why the credibility factor is relatively low in this case. [2]
[Total 9]
%%%%%%%%%%%%%%%%%%%%%%%%%%%%%%%%%5
% CT6 S2016–3 PLEASE TURN OVER
%-Question 4 
An insurance company has three types of policyholders: Standard, Premium and Elite.
If a Standard policyholder does not make a claim in a given year, they move to being
a Premium policyholder the following year. If a Standard policyholder makes a claim
in a given year they stay as a Standard policyholder for the following year.
If a Premium policyholder makes a claim in a given year, they move to being a
Standard policyholder for the following year. If a Premium policyholder does not
make a claim in a given year, they move to being an Elite policyholder the following
year.
Similarly if an Elite policyholder makes a claim in a given year, they move to being a
Premium policyholder for the following year. If an Elite policyholder does not make
a claim in a given year, they stay as an Elite policyholder for the following year.
You may assume the probability of more than one claim in a given year is negligible.
The transition probability matrix for the change in the policyholder type for each year
is:
  Type next year
Standard Premium Elite
Current type
Standard 80% 20% 0%
Premium 50% 0% 50%
Elite 0% 90% 10%
(i) Set out three algorithms (one for each possible initial policyholder type) which
simulate the policyholder type for each of the next three years, using the
inverse transform method. 
Assume the starting policyholder type is Standard and that the random numbers drawn
from a Uniform [0,1] distribution for the first simulation are 0.89, 0.64 and 0.12.
(ii) Determine how the policyholder type evolves over a three year period in this
simulation, including statement of how many claims are simulated to have
occurred. 
[Total 9]
%%%%%%%%%%%%%%%%%%%%%%%%%%%%%%%%%%%%%%%%%%%%%%%%%%%%%%%%%%%%%%%%%%%%%%%
\newpage
Q3 (i) Overall mean is 46.8 30.2 74.5 60.7 53.05
4
X = + + + = 
4
2 2
1
( )) 1 1227.4 1161.4 1340.3 1414.7 1285.95
4 4
( i
  i
  E s S
  =
    θ =  = + + + = [1]
  ( ( )) ( ) ( ( )) 4 2 2
  1
  Var 1 1
  3 i 6
  i
  m X X E S
  =
    θ =  − − θ [1]
  (46.8 53.05)2 (30.2 53.05)2 (74.5 53.05)2 (60.7 53.05)2
  3
  1285.95
  6
  − + − + − + −
  =
    −
  = 145.6 [1]
  So the credibility factor is 6 0.4045
  6 1 285.95145.6
  Z= =
    +
    [1]
  And the credibility premia are:
    (1) 0.4045 * 46.8+ 0.5955 * 53.05 = 50.5 [½]
  (2) 0.4045* 30.2 + 0.5955 * 53.05 = 43.8 [½]
  (3) 0.4045 * 74.5 + 0.5955 * 53.05 = 61.7 [½]
  (4) 0.4045 * 60.7 + 0.5955 * 53.05 = 56.1 [½]
  (ii) The variation within risks is much bigger relative to the variation between
  risks. This suggests that the variability is more explained by claim variability
  than in the underlying parameter, so we put more weight on the information
  provided by the data set as a whole, and less on the individual risks, resulting
  in a low credibility factor. [2]
  [Total 9]
  Many candidates scored full marks on part (i), but scored less well on part (ii).
  Stronger candidates were able to explain in words what was happening,
  beyond simply using mathematical formulae.
  Q4 (i) If Standard type the algorithm is:
    Sample U from U(0,1), if U <= 0.8 remain Standard; if U > 0.8 the new state
  is Premium. [1]
  Subject CT6 (Statistical Methods Core Technical) – September 2016 – Examiners’ Report
  Page 5
  If remain Standard repeat, otherwise if move to Premium algorithm: [½]
  If type Premium the algorithm is:
    Sample U from U(0,1), if U <= 0.5 moves to Standard, if 0.5 < U the new state
  is Elite. [1]
  If move to Standard move to the Standard algorithm, otherwise move to the
  Elite algorithm: [½]
  If type Elite the algorithm is:
    Sample U from U(0,1), if U <= 0.9 moves to Premium type; if 0.9 < U stay as
  Elite type. [1]
  If remain Elite type repeat, otherwise move to the Standard algorithm. [½]
  Start in required state and move between algorithms as required until three
  years have been simulated. [½]
  (ii) Start Standard, since U = 0.89 the new type is Premium (hence no claim in the
                                                               previous year!). [1]
  Now type Premium, since U = 0.64 the new type is Elite (no claim in the
                                                          previous year). [1]
  Now type Premium, since U = 0.12 < 0.9 then we have a claim and the new
  state is Premium. [1]
  There is only one simulated claim. [1]
  [Total 9]
  The unfamiliar application of the inverse transform theory confused a few
  candidates, but most scored well on both parts. Only the strongest
  candidates demonstrated their algorithm simulated three years, rather than
  just one.
  
