\documentclass[a4paper,12pt]{article}

%%%%%%%%%%%%%%%%%%%%%%%%%%%%%%%%%%%%%%%%%%%%%%%%%%%%%%%%%%%%%%%%%%%%%%%%%%%%%%%%%%%%%%%%%%%%%%%%%%%%%%%%%%%%%%%%%%%%%%%%%%%%%%%%%%%%%%%%%%%%%%%%%%%%%%%%%%%%%%%%%%%%%%%%%%%%%%%%%%%%%%%%%%%%%%%%%%%%%%%%%%%%%%%%%%%%%%%%%%%%%%%%%%%%%%%%%%%%%%%%%%%%%%%%%%%%

\usepackage{eurosym}
\usepackage{vmargin}
\usepackage{amsmath}
\usepackage{graphics}
\usepackage{epsfig}
\usepackage{enumerate}
\usepackage{multicol}
\usepackage{subfigure}
\usepackage{fancyhdr}
\usepackage{listings}
\usepackage{framed}
\usepackage{graphicx}
\usepackage{amsmath}
\usepackage{chng%%-- Page}

%\usepackage{bigints}
\usepackage{vmargin}

% left top textwidth textheight headheight

% headsep footheight footskip

\setmargins{2.0cm}{2.5cm}{16 cm}{22cm}{0.5cm}{0cm}{1cm}{1cm}

\renewcommand{\baselinestretch}{1.3}

\setcounter{MaxMatrixCols}{10}

\begin{document}
7 Claim amounts, X, arising from a portfolio of insurance policies follow a Pareto
distribution, with parameters α and λ. The insurance company has bought excess of
loss reinsurance cover, with retention M > 0.
The reinsurer only has a record of claims greater than M. Consider the truncated distribution of claim amounts, Z = X – M | X > M.
\begin{enumerate}
(i) Show that Z also follows a Pareto distribution, but with parameters α and
λ + M. [4]
CT6 S2016–5 PLEASE TURN OVER
Claim amounts, X’, have now increased by a factor k, such that a claim incurred is k
times an equivalent claim previously incurred, and k is greater than 1. The retention level M is unchanged.
(ii) Show that the distribution of X’ still follows a Pareto distribution, and
determine its parameters. [4]
The truncated distribution of claim amounts is now Z’, where Z’ = X’ – M | X’ > M.
(iii) State the distribution of Z’, using the results from parts (i) and (ii), including
statement of parameters. [1]
(iv) Comment on whether or not the average claim amount retained by the
insurance company has increased by a factor of k. [2]
\end{enumerate}
%%%%%%%%%%%%%%%%%%%%%%%%%%%
\newpage

Q7 (i) Let Z be the reinsurer claim distribution.
Then ( ) ( )
( )
1
f z M
g z
F M
+
  =
  −
where f(x) and F(x) refer to the underlying claim
distribution [1]
( )
( )
( )
1 ( ) f z M ;F M 1
z M M
α α
α+ α
+ = αλ = − λ
λ+ + λ +
  [1½]
So ( )
( )
( ) ( )
1 ( ) 1
M M
g z
z M z M
α α α
α+ α α+
  αλ λ + α λ + = =
  λ + + λ λ + +
  [1]
This is in the form of a Pareto distribution with parameters α and λ +M . [½]
(ii) Let ( ) ( )
( ) 1
0
;
y
k
Y kX P Y y P kX y dz
z
α
α+
  = < = < = αλ
λ+  [1½]
Let
1 ( ) 1
0 0
;
y y x kz dx k dx
x k k x
k
α αα
α+ α+
  = = αλ = αλ
λ +  λ +  
 
  . [2]
Which is in the form of a Pareto distribution with parameters α and kλ . [½]
(iii) The reinsurer’s distribution of claims is therefore Pareto with parameters α
and kλ +M. [1]
(iv) The average claim retained by the insurer has increased by a factor less than k
since the retention M is unchanged, so on average a greater proportion of
claims get passed on to the reinsurer. [2]
For strong candidates familiar with the bookwork this question was very
straightforward, but few candidates were able to score well here.
\end{document}
