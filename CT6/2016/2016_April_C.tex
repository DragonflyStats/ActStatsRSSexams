\documentclass[a4paper,12pt]{article}
%%%%%%%%%%%%%%%%%%%%%%%%%%%%%%%%%%%%%%%%%%%%%%%%%%%%%%%%%%%%%%%%%%%%%%%%%%%%%%%%%%%%%%%%%%%%%%%%%%%%%%%%%%%%%%%%%%%%%%%%%%%%%%%%%%%%%%%%%%%%%%%%%%%%%%%%%%%%%%%%%%%%%%%%%%%%%%%%%%%%%%%%%%%%%%%%%%%%%%%%%%%%%%%%%%%%%%%%%%%%%%%%%%%%%%%%%%%%%%%%%%%%%%%%%%%%
\usepackage{eurosym}
\usepackage{vmargin}
\usepackage{amsmath}
\usepackage{graphics}
\usepackage{epsfig}
\usepackage{enumerate}
\usepackage{multicol}
\usepackage{subfigure}
\usepackage{fancyhdr}
\usepackage{listings}
\usepackage{framed}
\usepackage{graphicx}
\usepackage{amsmath}
\usepackage{chngpage}
%\usepackage{bigints}
\usepackage{vmargin}

% left top textwidth textheight headheight

% headsep footheight footskip

\setmargins{2.0cm}{2.5cm}{16 cm}{22cm}{0.5cm}{0cm}{1cm}{1cm}
\renewcommand{\baselinestretch}{1.3}
\setcounter{MaxMatrixCols}{10}
\begin{document}

\begin{enumerate}
5 (i) Explain why insurance companies make use of run-off triangles. [2]
(ii) The run-off triangle below shows incremental claims incurred on a portfolio of
general insurance policies.
Development Year
Policy Year 0 1 2 3
2011 4,657 3,440 931 572
2012 6,089 5,275 1,381
2013 5,623 4,799
2014 7,224
Calculate the outstanding claims reserve for this portfolio using the basic
chain ladder method. [7]
[Total 9]
CT6 A2016–4
6 Felicity is a fund manager who is considering investing €1m in a specialist
investment contract where the return depends on the performance of a particular
company. She has a choice between two contracts as follows:
   Long contract: if the company is deemed a “success”, the investment will return
+100% and if it is deemed a “failure” it will return –75%.
 Short contract: if the company is deemed a “success”, the investment will return
–50% and if it is deemed a “failure” it will return +50%.
Before she decides which contract to invest in, Felicity will be able to observe the
investment performance of the company’s shares relative to the stock market.
Companies that are “successes” have a 60% probability of outperforming the stock
market. Companies that are “failures” have a 40% probability of outperforming the
market.
(i) List Felicity’s four decision functions. [2]
(ii) Calculate the values of the risk function for each decision function and type
of company. [6]
Two thirds of such companies under consideration are known to be failures.
(iii) Determine Felicity’s optimal decision function. [3]
[Total 11]
%%%%%%%%%%%%%%%%%%%%%%%%%%%%%%%%%%%%%%%%%%%%%%%%%%%%%%%%%%%%%%
  Q5 (i) There is normally a delay between incidents leading to claim and the insurance
pay out [1]
Insurance companies need to estimate future claims for their reserve [1]
It makes sense to use historical data to infer future patterns of claims [1]
[Max 2]
(ii) Cumulate claims
Policy
Year
0 1 2 3
2011 4,657 8,097 9,028 9,600
2012 6,089 11,364 12,745 13,553
2013 5,623 10,422 12,399
2014 7,224 15,690
DF 2,3 = 9600/9028 = 1.063 358 [1]
DF 1,2 = (9028 + 12745) / (8097 + 11364) = 1.118 802 [1]
DF 0,1 = (8097 + 11364 + 10422) / (4657 + 6089 + 5623) = 1.825 585 [1]
Find expected claims 12,745 * 1.063 358 = 13,553 [1]
10,422 * 1.118 802 * 1.063358 = 12,399 [1]
7,224 * 1.825 585 * 1.118802 * 1.063358 = 15,690 [1]
So claim reserve =
  (13,553 – 12,745) + (12,399 – 10,422) + (15,690 – 7,224) = 11,250 (4sf) [1]
Assuming that claims incurred are equal to claims paid
[Total 7]
[TOTAL 9]
This straightforward question on chain ladders was very well answered.
%--------------Subject CT6 (Statistical Methods Core Technical) – April 2016 – Examiners’ Report
%--------------Page 6
\newpage
Q6 (i) There are two possible outcomes of the observation and two possible choices
hence 2 × 2 = 4 decision functions.
Decision Fn/Stock Outperforms Underperforms
d1 Long Long
d2 Long Short
d3 Short Long
d4 Short Short
[½ for each row]
[Max 2]
(ii) If good and invest make 100% of €1m = €1m
If bad and invest lose 75% of €1m = −€0.75m [1]
R(d1|Good) = 1m
R(d1|Bad) = –0.75m [1]
R(d2|Good) = 1m * 60% outperformance – 0.5m * 40% = 0.4m
R(d2|Bad) = −.75m * 40% + 0.5m * 60% = 0m [1½]
R(d3|Good) = −0.5m * 60% + 1m * 40% = 0.1m
R(d3|Bad) = 0.5m * 40% −.75m * 60% = –0.25m [1½]
R(d4|Good) = –0.5m
R(d4|Bad) = 0.5m [1]
[Total 6]
(iii) We need to determine the expectation of each Risk function
D2 dominates D3 [½]
E(R(d1)) = 1/3 * 1m – 2/3 * .75m = −1/6m
E(R(d2)) = 1/3 * 2/5m = 2/15m
E(R(d4)) = 1/3 * − 1/2m + 2/3 * 1/2m = 1/6m [1½]
So d4 is the optimal decision function under the Bayes criterion. [1]
[Total 3]
[TOTAL 11]
% Apart from question 10, this was the most challenging question on the paper.
% Candidates who were able to identify the decision functions and set up the problem correctly generally did well, but many candidates struggled to formulate their answers.
% Subject CT6 (Statistical Methods Core Technical) – April 2016 – Examiners’ Report
\end{document}
