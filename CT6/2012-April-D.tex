\documentclass[a4paper,12pt]{article}

%%%%%%%%%%%%%%%%%%%%%%%%%%%%%%%%%%%%%%%%%%%%%%%%%%%%%%%%%%%%%%%%%%%%%%%%%%%%%%%%%%%%%%%%%%%%%%%%%%%%%%%%%%%%%%%%%%%%%%%%%%%%%%%%%%%%%%%%%%%%%%%%%%%%%%%%%%%%%%%%%%%%%%%%%%%%%%%%%%%%%%%%%%%%%%%%%%%%%%%%%%%%%%%%%%%%%%%%%%%%%%%%%%%%%%%%%%%%%%%%%%%%%%%%%%%%

\usepackage{eurosym}
\usepackage{vmargin}
\usepackage{amsmath}
\usepackage{graphics}
\usepackage{epsfig}
\usepackage{enumerate}
\usepackage{multicol}
\usepackage{subfigure}
\usepackage{fancyhdr}
\usepackage{listings}
\usepackage{framed}
\usepackage{graphicx}
\usepackage{amsmath}
\usepackage{chngpage}

%\usepackage{bigints}
\usepackage{vmargin}

% left top textwidth textheight headheight

% headsep footheight footskip

\setmargins{2.0cm}{2.5cm}{16 cm}{22cm}{0.5cm}{0cm}{1cm}{1cm}

\renewcommand{\baselinestretch}{1.3}

\setcounter{MaxMatrixCols}{10}

\begin{document}

\begin{enumerate}
%%%%%%%%%%%%%%%%%%%%%%%%%
7
The numbers of claims on three different classes of insurance policies over the last
four years are given in the table below.
Class 1
Class 2
Class 3
Year 1
1
1
5
Year 2
4
6
6
Year 3
5
4
4
Year 4
0
6
9
Total
10
17
24
The number of claims in a given year from a particular class is assumed to follow a
Poisson distribution.
8
(i) Determine the maximum likelihood estimate of the Poisson parameter for each
class of policy based on the data above.
[5]
(ii) Perform a test on the scaled deviance to check whether there is evidence that
the classes of policy have different mean claim rates and state your
conclusion.
[5]
%------%
The table below shows claims paid on a portfolio of general insurance policies. You
may assume that claims are fully run off after three years.
Underwriting year
2008
2009
2010
2011
Development Year
0
1
2
3
450 312 117 41
503 389 162
611 438
555
Past claims inflation has been 5% p.a. However, it is expected that future claims
inflation will be 10% p.a.
Use the inflation adjusted chain ladder method to calculate the outstanding claims on
the portfolio.
[10]
%%%%%%%%%%%%%%%%%%%%%%%%%%%%%%%%%%%%%%%%%%%%%%%%%%%%%%%%%%%%%%%%%%%%%%%%%%5
7
(i)
Suppose that the Poisson rate for risk i is λ i for =1,2,3.
For the first risk, the likelihood is given by:
L = e − 4 λ 1
(4 λ 1 ) 10
10!
And so the log-likelihood is given by
l = log L = − 4 λ 1 + 10 log 4 λ 1 + Constants
Differentiating gives
dl
10
= − 4 +
d λ 1
λ 1
And setting this equal to zero gives a maximum likelihood estimate of
10
λ ˆ 1 =
= 2. 5
4
Since
d 2 l
d λ
2
=−
10
λ i 2
< 0 we do have a maximum.
17
2 4
In the same way λ ˆ 2 =
= 6 .
= 4.25 and λ ˆ 3 =
4
4
Page 9
%%%%%%%%%%%%%%%%%%%%%%%%%

(ii)
Under the assumption that these risks share the same rate i.e. λ 1 = λ 2 = λ 3 = λ
then the mle for this is simply
51
= 4.25
λ ˆ =
12
In order to compare these models we can use the scaled deviances to compare
these models and use the chi-squared test.
The difference in the scaled deviance here should have a chi-square
distribution with 3−1=2 degrees of freedom.
2 ( log L 1 + log L 2 + log L 3 − log L ) = 10 log λ ˆ 1 − 4 λ ˆ 1 + 17 log λ ˆ 2 − 4 λ ˆ 2 + 24log λ ˆ 3 − 4 λ ˆ 3 − 51log λ ˆ + 12 λ ˆ
With the
4 4 4
i = 1 i = 1 i = 1
∑ log y 1 i ! + ∑ log y 2 i ! + ∑ log y 3 i ! cancelling out in the difference.
Hence
2 ( log L 1 + log L 2 + log L 3 − log L )
⎛
51 4 ( 10 + 17 + 24 )
51 ⎞
= 2 ⎜ 10 log 2.5 + 17 log 4.25 + 24 log 6 − 51log −
+ 12 ⎟
12
4
12 ⎠
⎝
51 ⎞
⎛
= 2 ⎜ 10 log 2.5 + 17 log 4.25 + 24 log 6 − 51log ⎟ = 5.939778
12 ⎠
⎝
This value is below 5.991 which is the critical value at the upper 5% level and
therefore there is not a significant improvement by considering different rates
for each risk.
Part (i) was answered very well. Most candidates struggled with part (ii).
Page 10
%%%%%%%%%%%%%%%%%%%%%%%%%
 (Statistical Methods) – April 2012 – Examiners’ Report
8
The claims uplifted to 2011 prices are as follows:
Underwriting
Year
2008
2009
2010
2011
Development Year
1
2
343.98
122.85
408.45
162
438
0
520.93
554.56
641.55
555
3
41
Accumulating gives:
Underwriting
Year
2008
2009
2010
2011
0
520.93
554.56
641.55
555
Development Year
1
2
864.91
987.76
963.01 1125.01
1079.55
3
1028.76
Hence the development factors are given by:
DF 0,1 = 864.91 + 963.01 + 1079.55
= 1.693304
520.93 + 554.56 + 641.55
DF 1,2 = 987.76 + 1125.01
= 1.155833
864.91 + 963.01
DF 2,3 = 1028.76
= 1.041508
987.76
The completed triangle of cumulative claims is:
Underwriting
year
2008
2009
2010
2011
0
520.93
554.56
641.55
555.00
Development Year
1
2
864.91
987.76
963.01 1125.01
1079.55 1247.78
939.78 1086.23
3
1028.76
1171.70
1299.57
1131.32
Dis-accumlating gives (in 2011 prices):
Underwriting
year
2008
2009
2010
2011
0
Development Year
1
2
384.78
168.23
146.45
3
46.70
51.79
45.09
Page 11
%%%%%%%%%%%%%%%%%%%%%%%%%
 (Statistical Methods) – April 2012 – Examiners’ Report
Inflating for future claims growth gives:
Underwriting
year
2008
2009
2010
2011
Development Year
1
2
0
423.26
185.05
177.20
3
51.37
62.67
60.01
And the outstanding claims are:
51.37+62.67+60.01+185.05+177.20+423.26 = 959.56
This question was tackled very well by most candidates
