\documentclass[a4paper,12pt]{article}

%%%%%%%%%%%%%%%%%%%%%%%%%%%%%%%%%%%%%%%%%%%%%%%%%%%%%%%%%%%%%%%%%%%%%%%%%%%%%%%%%%%%%%%%%%%%%%%%%%%%%%%%%%%%%%%%%%%%%%%%%%%%%%%%%%%%%%%%%%%%%%%%%%%%%%%%%%%%%%%%%%%%%%%%%%%%%%%%%%%%%%%%%%%%%%%%%%%%%%%%%%%%%%%%%%%%%%%%%%%%%%%%%%%%%%%%%%%%%%%%%%%%%%%%%%%%

\usepackage{eurosym}
\usepackage{vmargin}
\usepackage{amsmath}
\usepackage{graphics}
\usepackage{epsfig}
\usepackage{enumerate}
\usepackage{multicol}
\usepackage{subfigure}
\usepackage{fancyhdr}
\usepackage{listings}
\usepackage{framed}
\usepackage{graphicx}
\usepackage{amsmath}
\usepackage{chngpage}
%\usepackage{bigints}
\usepackage{vmargin}

% left top textwidth textheight headheight

% headsep footheight footskip
\setmargins{2.0cm}{2.5cm}{16 cm}{22cm}{0.5cm}{0cm}{1cm}{1cm}
\renewcommand{\baselinestretch}{1.3}
\setcounter{MaxMatrixCols}{10}
\begin{document}
CT6 A2016–2
1 (i) Derive the median of a Pareto distribution with parameters  and . [3]
Let  = 2 and  = 3.
(ii) Comment on the skewness of this Pareto distribution. [3]
%%%%%%%%%%%%%%%%%%%%%%
  2 A portfolio of insurance policies has two types of claims:
   Loss amounts for Type I claims are exponentially distributed with mean 120.
 Loss amounts for Type II claims are exponentially distributed with mean 110.
25% of claims are Type I, and 75% are Type II.
(i) Calculate the mean and variance of the loss amount for a randomly chosen claim. [3]
An actuary wants to model randomly chosen claims using an exponential distribution as an approximation.
(ii) Explain whether this is a good approximation. [2]
[Total 5]

%%%%%%%%%%%%%%%%%%%%%%%%%%%%%%%%%%%%%%%%%%%%%%%%%%%%%%%%%%%%%%
  Q1 (i) m = P(x ≤ m) =
  [1]
CDF = 1
x
 λ α −   λ + 
so 1
m 2
 λ α   =  λ + 
[1]
So
1
m  2 α 1 = λ  − 
 
[1]
[Total 3]
(ii) median = 3 ( 2 − 1) = 1.2426 [1]
mean = 3/(2 − 1) = 3 [1]
1
2
Subject CT6 (Statistical Methods Core Technical) – April 2016 – Examiners’ Report
Page 3
So this distribution has positive skew (mean > median) [1]
This is common for a Pareto distribution (or any relevant comment) [1]
[Max 3]
[TOTAL 6]
This straightforward question was well answered by most candidates, although a few erroneously used the formula for the coefficient of skewness, which is not applicable in this case.
Q2 (i) Mean = 14 λ1 + 34 λ2 =112.5 [1]
2 2 2 2 2
E(X ) = 14 (λ1 + λ1 ) + 34 (λ2 + λ2 ) = 25350 [1]
so variance = 25350 – 112.52 = 12,694 [1]
[Total 3]
(ii) For an exponential, mean = standard deviation and they are pretty close, so yes
this is a good approximation. [2]
Alternatively
In general, the sum of two exponentials is not exponential, so this is not a
good approximation. [2]
[Max 2]
[TOTAL 5]
This question was relatively poorly answered, with only the better candidates
being able to derive the variance. This is disappointing given how often this
type of question occurs.
\end{document}
