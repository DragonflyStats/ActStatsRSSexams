\documentclass[a4paper,12pt]{article}

%%%%%%%%%%%%%%%%%%%%%%%%%%%%%%%%%%%%%%%%%%%%%%%%%%%%%%%%%%%%%%%%%%%%%%%%%%%%%%%%%%%%%%%%%%%%%%%%%%%%%%%%%%%%%%%%%%%%%%%%%%%%%%%%%%%%%%%%%%%%%%%%%%%%%%%%%%%%%%%%%%%%%%%%%%%%%%%%%%%%%%%%%%%%%%%%%%%%%%%%%%%%%%%%%%%%%%%%%%%%%%%%%%%%%%%%%%%%%%%%%%%%%%%%%%%%

\usepackage{eurosym}
\usepackage{vmargin}
\usepackage{amsmath}
\usepackage{graphics}
\usepackage{epsfig}
\usepackage{enumerate}
\usepackage{multicol}
\usepackage{subfigure}
\usepackage{fancyhdr}
\usepackage{listings}
\usepackage{framed}
\usepackage{graphicx}
\usepackage{amsmath}
\usepackage{chngpage}

%\usepackage{bigints}
\usepackage{vmargin}

% left top textwidth textheight headheight

% headsep footheight footskip

\setmargins{2.0cm}{2.5cm}{16 cm}{22cm}{0.5cm}{0cm}{1cm}{1cm}

\renewcommand{\baselinestretch}{1.3}

\setcounter{MaxMatrixCols}{10}

\begin{document}

\begin{enumerate}
%%%%%%%%%%%%%%%%%%%%%%%%%%%%%%%%%%%%%%%%%%%%%%%%%%%%%%%%%%%%%%%%%%%%%%%%%%
6
Annual numbers of claims on three different types of insurance policy follow a
Poisson distribution with parameter μ i for i = 1, 2, 3. Data for the last four years is
given in the table below.
Year
7
Type 1 2 3 4 Total
1
2
3 5
2
5 5
5
6 0
4
4 1
5
5 11
16
20
(i) Derive the maximum likelihood estimate of μ 1 and calculate the corresponding
[5]
estimates of μ 2 and μ 3 .
(ii) Test the hypothesis that μ 1 , μ 2 and μ 3 are equal using the scaled deviance. [5]
[Total 10]
%%%%%%%%%%%%%%%%%%%%%%%%%%%%%%%%%%%%%%%%%%%%%%%%%%%%%%%%%%%%%%%%%%%%%%%%%%
5
Determine the randomised strategy Shivon should adopt, and the value of the
game to her.
[2]
[Total 8]
The table below shows the cost of claims settled per calendar year for a set of car
insurance policies, with figures in €000s.
Accident Year
2014
2015
2016
Development Year
0
1
2
5,419
6,234
7,719
908
1,088
239
The corresponding number of settled claims is as follows:
Development Year
1
2
Accident Year 0
2014
2015
2016 760
819
881
98
93
37
(i) Calculate the outstanding claims reserve for this portfolio, using the average
cost per claim method with grossing up factors.
[7]
(ii) State four key assumptions made in part (i).
CT6 S2017–3
[2]
[Total 9]
PLEASE TURN OVER6
7
(i) State the key characteristics of the individual risk model.
[2]
(ii) State all possible values for the number of claims that may arise from a given
risk over the period being modelled.
[1]
(iii) Suggest an example of an insurance contract where the individual risk model
may be suitable, and an example where it is unlikely to be.
[2]
(iv) Describe the differences between the individual risk model and the collective
risk model.
[3]
(v) State the additional assumption required such that the individual risk model
can be reduced to the collective risk model.
[1]
[Total 9]
%%%%%%%%%%%%%%%%%%%%%%%%%%%%%%%%%%%%%%%%%%%%%%%%%%%%%%%%%%%%%%%%%%%%%%%%%%%%%%%%%
Q5
(i)
Cumulative amounts & claims
5
Accident Year
2014
2015
2016
Development Year
0
1
2
5,419 6,327 6,566
6,234 7,322
7,719
[1]
Accident Year
2014
2015
2016
Development Year
0
1
2
760
858
895
819
912
881
[1]
Page 5Subject CT6 (Statistical Methods Core Technical) – September 2017 – Examiners’ Report
Average cost per claim (cumulative)
Accident Year
2014
2015
2016
Development Year
0
1
2
97.191%
100.515%
100%
7.130 263
7.374 126
7.336 313
95.297%
100.515%
100%
7.611 722
8.028509
7.987 3
96.244%
100%
8.761 635
9.103 5
[2]
Claim numbers (cumulative)
Accident Year
2014
2015
2016
Development Year
0
1
84.916%
95.866%
760
858
86.090%
95.866%
819
912
85.503%
881
2
100%
895
100%
951.33
100%
1,030.4
[11⁄2]
Total claims 6,566+7.98734*951.329 +9.103564*1030.373 = 23,545
Claims to date 21,607 so reserve €1,938k
(ii)
[1]
[1⁄2]
Claims fully developed after DY2.
[1⁄2]
The proportions of claim numbers relating to each DY remain constant in
different AYs.
[1]
Cost of claims settled equals amount actually paid out.
[1⁄2]
The average cost per claim figures relating to each DY remain constant in
different AYs.
[1⁄2]
Inflation has been allowed for.
[1⁄2]
[Max 2]
Most candidates scored very well on this straightforward chain ladder
question.
Q6
(i)
A fixed numbers of risks are assumed within the studied portfolio [1⁄2]
The number of risks does not change over the period of insurance cover [1⁄2]
These risks are independent [1⁄2]
Page 6Subject CT6 (Statistical Methods Core Technical) – September 2017 – Examiners’ Report
Claim amounts from these risks are not (necessarily) identically distributed
random variables
[1⁄2]
(ii) The number of claims arising is either 0 or 1. [1]
(iii) e.g. Term assurance (yes), household contents insurance (no) [2]
(iv) In the individual risk model the number of risks is specified and fixed; and the
number of claims from each risk is restricted. This is not the case for the
collective risk model.
[2]
Individual risks are independent in the individual risk model, whilst the
individual claims amounts are independent under the collective risk model. [1]
Claims in collective risk model are identical whereas the claims in each policy
in the individual risk model are not.
[1]
[Max 3]
(v)
When claim amounts are in fact identically distributed.
[1]
Well prepared candidates were able to score highly on this bookwork
question, but a number of students were unable to demonstrate
sufficient knowledge of the individual risk model.
%%%%%%%%%%%%%%%%%%%%%%%%%%%%%%%%%%%%%%%%%%%%%%%%%%%%%%%%%%%%%%%%%%%%%%%%%%%%%%%%%%%%%%%%%%%%%%%%%%5


5
P 1 = 17 + 23 + 21 + 29 + 35 = 125
P 2 = 42 + 51 + 60 + 55 + 37 = 245
P 3 = 43 + 31 + 62 + 98 + 107 = 341
P = 125 + 245 + 341 = 711
X =
(850  125  720  245  900  341)
= 829.18
711
expected claims per policy for risk 1 next year =
so
840 = Z 1 × 850 + (1  Z 1 ) × 829.18
Z 1 =
so
25, 200
= 840
30
840  829.18
= 0.519594
850  829.18
125
E ( s 2 (  ))
125 
Var[ m (  )]
= 0.51969
so E ( s 2 (  ))
125  0.51969  125
=
= 115.57217
Var[ m (  )]
0.51969
so Z 2 =
245
= 0.6794756
245  115.528
Z 3 =
341
= 0.74686
341  115.528
Page 7Subject CT6 (Statistical Methods Core Technical) – April 2015 – Examiners’ Report
So credibility premium per policy are
Type 2: 0.67956 × 720 + (1  0.67956) × 829.18 = 755.0
Type 3: 0.74694 × 900 + (1  0.74694) × 829.18 = 882.1
so overall expected claims
Type 2: 754.98 × 40 = 30,200
Type 3: 882.08 × 110 = 97,028
Candidates with good knowledge of EBCT Model 2 scored well here, however a
disappointing number of candidates were apparently unfamiliar with this method.
6
(i)
The likelihood for type 1 policies is given by
l =
e  4  1  11
1
4
 y ij !
j  1
Taking logarithms gives
4
L = log l =  4  1  11log  1   log y ij !
j  1
so
 L
11
  4 
 1

and setting
i.e.
 4  11
= 0
 ˆ 1
̂ 1 = 11
= 2.75
4
similarly
̂ 3 =
Page 8
 L
= 0 we have
 1
̂ 2 =
16
= 4
4
20
= 5
4Subject CT6 (Statistical Methods Core Technical) – April 2015 – Examiners’ Report
(ii)
Assuming  =  1 =  2 =  3 we have ̂ =
11  16  20
= 3.916667
12
The difference in scaled deviance is given by
 = 2(log L 1  log L 2  log L 3  log L )
= 2(  4  ˆ 1  11 log  ˆ 1  4  ˆ 2  16 log  ˆ 2  4  ˆ 3  20 log  ˆ 3
 12  ˆ  47 log  ˆ )
[The logarithms of factorials cancel]
= 2(47 + 11 log2.75 + 16 log4 + 20 log5 + 47  47 log3.91667)
= 2.6615
Under H 0 :  1 =  2 =  3 we have that  comes from a  2 distribution with
3  1 = 2 degrees of freedom.
The upper 5% point of the  22 distribution is 5.991. The observed value is
below this and so there is no evidence to suggest that the underlying
parameters are different for each risk.
Most candidates were able to score well on part (i), although only the better prepared
candidates were able to complete part (ii).
