\documentclass[a4paper,12pt]{article}

%%%%%%%%%%%%%%%%%%%%%%%%%%%%%%%%%%%%%%%%%%%%%%%%%%%%%%%%%%%%%%%%%%%%%%%%%%%%%%%%%%%%%%%%%%%%%%%%%%%%%%%%%%%%%%%%%%%%%%%%%%%%%%%%%%%%%%%%%%%%%%%%%%%%%%%%%%%%%%%%%%%%%%%%%%%%%%%%%%%%%%%%%%%%%%%%%%%%%%%%%%%%%%%%%%%%%%%%%%%%%%%%%%%%%%%%%%%%%%%%%%%%%%%%%%%%

\usepackage{eurosym}
\usepackage{vmargin}
\usepackage{amsmath}
\usepackage{graphics}
\usepackage{epsfig}
\usepackage{enumerate}
\usepackage{multicol}
\usepackage{subfigure}
\usepackage{fancyhdr}
\usepackage{listings}
\usepackage{framed}
\usepackage{graphicx}
\usepackage{amsmath}
\usepackage{chng%%-- Page}
%\usepackage{bigints}
\usepackage{vmargin}

% left top textwidth textheight headheight

% headsep footheight footskip
\setmargins{2.0cm}{2.5cm}{16 cm}{22cm}{0.5cm}{0cm}{1cm}{1cm}
\renewcommand{\baselinestretch}{1.3}
\setcounter{MaxMatrixCols}{10}
\begin{document} 

%%% - Question 7

PLEASE TURN OVER7
An insurer operates a simple no claims discount system with 5 levels: 0%, 20%, 40%,
50% and 60%.
The rules for moving between levels are:
An introductory discount of 20% is available to new customers.
If no claims are made during a year the policyholder moves up to the next
discount level or remains at the maximum level.
If one or more claims are made during the year, a policyholder at the 50% or 60%
discount level moves to the 20% level and a policyholder at 0%, 20% or 40%
moves to or remains at the 0% level.
The full annual premium is \$600.
When an accident occurs, the distribution of loss is exponential with mean \$1,750. A
policyholder will only claim if the loss is greater than the extra premiums over the
next four years, assuming no further accidents occur.
8
(i) For each discount level, calculate the smallest cost for which a policyholder
will make a claim.

(ii) For each discount level, calculate the probability of a claim being made in the
event of an accident occurring.

(iii) Comment on the results of (ii).
(iv) Currently, equal numbers of customers are in each discount level and the
probability of a policyholder not having an accident each year is 0.9.
Calculate the expected proportions in each discount level next year.

[Total 12]


%%%%%%%%%%%%%%%%%%%%%%%%%%%%%%%%%%%%%%%%%%%%5

7
September 2005
%%%%%%%%%%%%%%%%%%%%%%%%%%%%%%%%%%
(i)
Discount level Premiums if no claim Premiums if claim Difference
0%
20%
40%
50%/60% 480, 360, 300, 240
360, 300, 240, 240
300, 240, 240, 240
240, 240, 240, 240 600, 480, 360, 300
600, 480, 360, 300
600, 480, 360, 300
480, 360, 300, 240 360
600
720
420
(ii)
0%
20%
40%
50%/60%
P(Cost > 360) = exp(
P(Cost > 600) = exp(
P(Cost > 720) = exp(
P(Cost > 420) = exp(
360/1,750) = 0.814
600/1,750) = 0.710
720/1,750) = 0.663
420/1,750) = 0.787
(iii) The amount for which the 50%/60% discount drivers will claim is much lower
than the 20% and 40% categories. It seems illogical for this to be the case as
this leads to a higher probability of a claim in the event of an accident.
Suggest that structure is altered to make claims for these categories less likely.
(iv) The transition matrix is
0.0814 0.9186 0
0
0.0710 0
0.9290 0
0.0663 0
0
0
0.0787 0
0
0.0787 0
0
0
0.9337 0
0
0.9213
0
0.9213
This year the proportions at each level are
(0.2, 0.2, 0.2, 0.2, 0.2)
Next year, the expected proportions are
0.2
0.2
0.2
0.2
0.2
Page 8
(0.0814 + 0.0710 + 0.0663)
(0.9186 + 0.0787 + 0.0787)
0.9290
0.9337
(0.9213 + 0.9213)
= 0.04374
= 0.2152
= 0.1858
= 0.18674
= 0.36852%%%%%%%%%%%%%%%%%%%%%%%%%%%%%%%%%%%%%%%%%%%%%%%%%%%%%%%%%%%%%%%%%
