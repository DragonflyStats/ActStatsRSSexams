\documentclass[a4paper,12pt]{article}

%%%%%%%%%%%%%%%%%%%%%%%%%%%%%%%%%%%%%%%%%%%%%%%%%%%%%%%%%%%%%%%%%%%%%%%%%%%%%%%%%%%%%%%%%%%%%%%%%%%%%%%%%%%%%%%%%%%%%%%%%%%%%%%%%%%%%%%%%%%%%%%%%%%%%%%%%%%%%%%%%%%%%%%%%%%%%%%%%%%%%%%%%%%%%%%%%%%%%%%%%%%%%%%%%%%%%%%%%%%%%%%%%%%%%%%%%%%%%%%%%%%%%%%%%%%%

\usepackage{eurosym}
\usepackage{vmargin}
\usepackage{amsmath}
\usepackage{graphics}
\usepackage{epsfig}
\usepackage{enumerate}
\usepackage{multicol}
\usepackage{subfigure}
\usepackage{fancyhdr}
\usepackage{listings}
\usepackage{framed}
\usepackage{graphicx}
\usepackage{amsmath}
\usepackage{chng%%-- Page}

%\usepackage{bigints}
\usepackage{vmargin}

% left top textwidth textheight headheight

% headsep footheight footskip

\setmargins{2.0cm}{2.5cm}{16 cm}{22cm}{0.5cm}{0cm}{1cm}{1cm}

\renewcommand{\baselinestretch}{1.3}

\setcounter{MaxMatrixCols}{10}

\begin{document}
%%--- Question 6

On 1 January 2001 an insurer in a far off land sells 100 policies, each with a five year
term, to householders wishing to insure against damage caused by fireworks. The
insurer charges annual premiums of £600 payable continuously over the life of the
policy.
The insurer knows that the only likely date a claim will be made is on the day of
St Ignitius feast on 1 August each year, when it is traditional to have an enormous
fireworks display. The annual probability of a claim on each policy is 40\%. Claim
amounts follow a Pareto distribution with parameters = 10 and = 9,000.
\begin{enumerate}
\item (i) Calculate the mean and standard deviation of the annual aggregate claims. 
\item (ii) Denote by (U, t) the probability of ruin before time t given initial surplus $U$.
(a)
(b)
%%%%%%%%%%%%%%%%%%%%%%%%%%%%%%%%%%%%%%%%%%%%%%%%%%%%%%%%%%%%%%%%%%%%%%%%%%%
\item 
Explain why for this portfolio $(U, t 1 ) = (U, t 2 )$ if
$7/12 < t 1 , t 2 < 19/12$.

Estimate (15,000, 1) assuming annual claims are approximately
Normally distributed.
\end{enumerate}
%%%%%%%%%%%%%%%%%%%%%%%%%%%%%%%%%%%%%%%%%%%%%%%%%%%%%%%%%%%%%%%%%%%%%%%%%%%%%%%%%%%%%%%%%%%%%%%%%%%%%%%%%%%%%%%
\begin{itemize}
\item The number of annual claims N follows a binomial distribution:
N ~ B(100, 0.4) then
E(N) = 100
0.4 = 40
and
Var(N) = 100
0.4
0.6 = 24.
\item Let $X$ denote the distribution of the individual claim amounts, so that X ~
Pareto(10, 9,000). Then
E(X) =
9, 000
= 1,000
10 1
and
Var(X) =
9, 000 2 10
9 2 8
= 1,250,000.
\item The annual aggregate claim amount S has
E(S) = E(N)E(X) = 40
1,000 = 40,000
and
\begin{eqnarray*}
Var(S) &=& E(N)Var(X) + Var(N)E(X)^2\\
&=& 40
1,250,000 + 24
1,000^2\\
&=& 74,000,000\\
&=& (8,602.33) 2\\
\end{eqnarray*}
%%-- Page 5%%%%%%%%%%%%%%%%%%%%%%%%%%%%%%%%%%%%%%%%%%%%%
(ii)
(a)
%%%%%%%%%%%%%%%%%%%%%%%%%%%%%%%%%%%%%%%%%%%%%%%%%%%%%%%%%%%%
\item Since claims can only fall on one day of the year, there is effectively only one day of the year on which ruin can occur, namely 1 August (or strictly shortly thereafter). For a year after 1 August, the insurer will be receiving premiums but paying no claims, and hence solvency will be improving. Hence
\[(U, t 1 ) = (U, t 2 ) if 7/12 < t 1 , t 2 < 19/12.\]
\item (b)
We must find (15,000, 1). But ruin will have occurred before time 1
only if it occurs at t = 7/12. Just before the claims occur, the insurers assets will be 7/12 100 600 + 15,000 = 50,000 and ruin will occur if the aggregate claims in the first year exceed this level. \item Assuming that S is approximately normally distributed, we have
P(Ruin) = P(N(40,000, (8,602.33) 2 ) > 50,000)\\
= P N (0, 1)\\
= 1
50, 000 40, 000
8, 602.33
(1.162)
= 0.123.
\end{itemize}
\end{document}
%%%%%%%%%%%%%%%%%%%%%%%%%%%%%%%%%%%%%%%%%%%%%%%%%%%%%
