\documentclass[a4paper,12pt]{article}

%%%%%%%%%%%%%%%%%%%%%%%%%%%%%%%%%%%%%%%%%%%%%%%%%%%%%%%%%%%%%%%%%%%%%%%%%%%%%%%%%%%%%%%%%%%%%%%%%%%%%%%%%%%%%%%%%%%%%%%%%%%%%%%%%%%%%%%%%%%%%%%%%%%%%%%%%%%%%%%%%%%%%%%%%%%%%%%%%%%%%%%%%%%%%%%%%%%%%%%%%%%%%%%%%%%%%%%%%%%%%%%%%%%%%%%%%%%%%%%%%%%%%%%%%%%%

\usepackage{eurosym}
\usepackage{vmargin}
\usepackage{amsmath}
\usepackage{graphics}
\usepackage{epsfig}
\usepackage{enumerate}
\usepackage{multicol}
\usepackage{subfigure}
\usepackage{fancyhdr}
\usepackage{listings}
\usepackage{framed}
\usepackage{graphicx}
\usepackage{amsmath}
\usepackage{chng%%-- Page}

%\usepackage{bigints}
\usepackage{vmargin}

% left top textwidth textheight headheight

% headsep footheight footskip

\setmargins{2.0cm}{2.5cm}{16 cm}{22cm}{0.5cm}{0cm}{1cm}{1cm}

\renewcommand{\baselinestretch}{1.3}

\setcounter{MaxMatrixCols}{10}

\begin{document}



2 An insurer wishes to estimate the expected number of claims, , on a particular type
of policy. Prior beliefs about are represented by a Gamma distribution with density
function
f( ) =
1
( )
For an estimate, d, of
e

( > 0).
the loss function is defined as
d) 2 + d 2 .
L( , d) = (
Show that the expected loss is given by
E(L( , d)) =
(
1)
2 d
2
2 d 2
and hence determine the optimal estimate for
3
(i)
(ii)
4


%%%%%%%%%%%%%%%%%%%%%%%%%%%%%%%%%%%%%%%%%%%%%%%%%%%%%%%%%%
2%%%%%%%
The expected loss is given by
d) 2 + d %2 ]

\begin{eqnarray*}
E[L( , d)] &=& E[(
2 \\
&=& E[ \lambda 
2 d + d 2 + d 2 ]\\
&=& E( 2 )
2dE( ) + 2d 2 .\\
\end{eqnarray*}

Now we know that $ \lambda ~ \Gamma ( , )$. From the tables, we know that E( ) = / and
Var( ) = / 2 . Now
\[E( 2 ) = Var(\lambda ) + E(\lambda ) 2\]
2
=
=
So
2 2
( 1)
2
(
E[L( , d)] =
.
1)
2 d
2
2 d 2
as required.
First set
Then
f(d) = E[(L( , d)].
2
f ( d ) =
4d
and to minimise the expected loss, we must find the value of d* of d for which
f ( d *) = 0. This occurs when
4d* =
so that
%%-- Page 2
%%-- Page 2
d* =
2
2
.

\end{document}
