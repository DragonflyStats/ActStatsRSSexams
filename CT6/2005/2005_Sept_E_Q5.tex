\documentclass[a4paper,12pt]{article}

%%%%%%%%%%%%%%%%%%%%%%%%%%%%%%%%%%%%%%%%%%%%%%%%%%%%%%%%%%%%%%%%%%%%%%%%%%%%%%%%%%%%%%%%%%%%%%%%%%%%%%%%%%%%%%%%%%%%%%%%%%%%%%%%%%%%%%%%%%%%%%%%%%%%%%%%%%%%%%%%%%%%%%%%%%%%%%%%%%%%%%%%%%%%%%%%%%%%%%%%%%%%%%%%%%%%%%%%%%%%%%%%%%%%%%%%%%%%%%%%%%%%%%%%%%%%

\usepackage{eurosym}
\usepackage{vmargin}
\usepackage{amsmath}
\usepackage{graphics}
\usepackage{epsfig}
\usepackage{enumerate}
\usepackage{multicol}
\usepackage{subfigure}
\usepackage{fancyhdr}
\usepackage{listings}
\usepackage{framed}
\usepackage{graphicx}
\usepackage{amsmath}
\usepackage{chng%%-- Page}
%\usepackage{bigints}
\usepackage{vmargin}

% left top textwidth textheight headheight

% headsep footheight footskip
\setmargins{2.0cm}{2.5cm}{16 cm}{22cm}{0.5cm}{0cm}{1cm}{1cm}
\renewcommand{\baselinestretch}{1.3}
\setcounter{MaxMatrixCols}{10}
\begin{document} 

%%% - Question 5

An insurer believes that claims from a particular type of policy follow a Pareto
distribution with parameters = 2.5 and = 300. The insurer wishes to introduce a
deductible such that 25% of losses result in no claim on the insurer.
(i) Calculate the size of the deductible.

(ii) Calculate the average claim amount net of the deductible.

[Total 10]

%%%%%%%%%%%%%%%%%%%%%%%%%%%%%%%%%%%%%%%%%%%%%%%%%%%%%%%%%%%%%%%%%%%%%%%%%%%%%%%%%
5
(i)
September 2005
%%%%%%%%%%%%%%%%%%%%%%%%%%%%%%%%%%
We must find D such that
D
0
f ( x ) dx = 0.25
this means that
0.25 =
D
0
(
dx
1
x )
D
=
= 1
= 1
( x )
( D )
0
2.5
300
300 D
So
1
300
= (1 0.25) 2.5 = 0.8913
300 D
and hence
300 + D =
300
0.8913
so that
D =
(ii)
300
300 = 36.59.
0.8913
The average net claim is given by E[X
D
( x D ) f ( x ) dx =
=
x
D
(
x )
1
D X > D]
dx D
D
(
x )
x
(
x )
D
D
(
x )
1
dx
dx D
(
x )
D
Page 5%%%%%%%%%%%%%%%%%%%%%%%%%%%%%%%%%%%%%%%%%%%%%%%%%%%%%%%%%%%%%%%%%
%%%%%%%%%%%%%%%%%%%%%%%%%%%%%%%%%%
D
= 0
(
D )
(
=
(
=
September 2005
1)(
D )
1)(
x )
0
1
D
D
(
D )
1
300 2.5
1.5 (300 36.59) 1.5
= 168.29.
Hence E[X
6
D X > D] =
168.29
= 224.39
0.75
We first calculate E(100X) and \operatorname{Var}(100X). Taking the expectation first, we have
E(100X) = 100E(X)
= 100
3
32
300
=
2 x 3
32
=
5
1
6 x 2
x 4
4
x 3 5 xdx
5 x 2
2
5
1
300
(31.25 0.75)
32
= 300.
Now for the variance
E((100X) 2 ) = 100 2
= 100 2
E(X 2 )
3
32
5
1
30, 000 6 x 4
=
32
4
Page 6
6 x 3 x 4 5 x 2 dx
x 5
5
5 x 3
3
5
1%%%%%%%%%%%%%%%%%%%%%%%%%%%%%%%%%%%%%%%%%%%%%%%%%%%%%%%%%%%%%%%%%
=
September 2005
%%%%%%%%%%%%%%%%%%%%%%%%%%%%%%%%%%
30, 000
(104.166 0.366)
32
= 98,000
so that
VarX = E(X 2 )
E(x) 2
= 98,000
300 2
= 8000
= (89.44) 2 .
Now let Y denote the cost (if any) of the associated repair. Then Y is independent of X
and
E(Y) = 0.3
200 = 60
and
\operatorname{Var}(Y) = 0.3
200 2
60 2 = 8,400 = (91.65) 2 .
So the mean individual claim amount Z is
E(X + Y) = E(X) + E(Y) = 300 + 60 = 360
and the variance of an individual claim is
\operatorname{Var}(X + Y) = \operatorname{Var}(X) + \operatorname{Var}(Y) = 7,998.75 + 8,400.00 = 16,398.75
The mean and variance of the aggregate claims S are given by the formulae
E(S) = E(N)E(Z)
and
\operatorname{Var}(S) = E(N) \operatorname{Var}(Z) + \operatorname{Var}(N)E(Z) 2
where N is the total number of claims. In our case
E(S) = 50
360 = 18,000
and
\operatorname{Var}(S) = 50
16,398.75 + 50
360 2 = 7,299,937.5 = (2,701.84) 2
Page 7%%%%%%%%%%%%%%%%%%%%%%%%%%%%%%%%%%%%%%%%%%%%%%%%%%%%%%%%%%%%%%%%%
