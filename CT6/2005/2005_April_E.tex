\documentclass[a4paper,12pt]{article}

%%%%%%%%%%%%%%%%%%%%%%%%%%%%%%%%%%%%%%%%%%%%%%%%%%%%%%%%%%%%%%%%%%%%%%%%%%%%%%%%%%%%%%%%%%%%%%%%%%%%%%%%%%%%%%%%%%%%%%%%%%%%%%%%%%%%%%%%%%%%%%%%%%%%%%%%%%%%%%%%%%%%%%%%%%%%%%%%%%%%%%%%%%%%%%%%%%%%%%%%%%%%%%%%%%%%%%%%%%%%%%%%%%%%%%%%%%%%%%%%%%%%%%%%%%%%

\usepackage{eurosym}
\usepackage{vmargin}
\usepackage{amsmath}
\usepackage{graphics}
\usepackage{epsfig}
\usepackage{enumerate}
\usepackage{multicol}
\usepackage{subfigure}
\usepackage{fancyhdr}
\usepackage{listings}
\usepackage{framed}
\usepackage{graphicx}
\usepackage{amsmath}
\usepackage{chngpage}

%\usepackage{bigints}
\usepackage{vmargin}

% left top textwidth textheight headheight

% headsep footheight footskip

\setmargins{2.0cm}{2.5cm}{16 cm}{22cm}{0.5cm}{0cm}{1cm}{1cm}

\renewcommand{\baselinestretch}{1.3}

\setcounter{MaxMatrixCols}{10}

\begin{document}
\begin{enumerate}
%%%%%%%%%%%%%%%%%%%%%%%%%%%%%%%%%%%%%%%%%%%%%%%%%%%%%%%%%%%%%%%%%%%%%%%%%%%%%%%%%%%%%%%%55
9
$\{Y_1 , Y_2 , \ldots , Y_n\}$ are independent claims, which are assumed to be exponentially distributed, with
$E[Y i ] =
i
$.
\begin{enumerate}[(i)]
\item Show that the canonical link function is the inverse link function.
\item It is decided that the canonical link function should not be used, but that the
mean claim sizes should be modelled as follows:
log
(a)
i
i 1, 2, ..., m
=
i
m 1, m 2, ..., n
Show that the log-likelihood can be written as
m
m
( n m )
e
n
y i
e
i 1
y i
i m 1
(b) Derive the maximum likelihood estimators of
(c) Show that the scaled deviance for this model is
1
m
m
2
log
i 1
(iii)
[3]
m
y j
j 1
y i
and .
n
1
n m
n
log
i m 1
y j
j m 1
y i
[12]
For a particular data set, m = 20, n = 44,
1 20
1 44
y i = 14.2,
y i = 18.7.
20 i 1
24 i 21
Calculate the deviance residual for $y_1 = 7$.
\end{enumerate}
%%%%%%%%%%%%%%%%%%%%%%%%%%%%%%%%%%%%%%%%%%%%%%%%%%%%%%%%%%%%%%%%%%%%%%%%%%%%%%%%%%%%%%%%%%%%%
9
(i)
f(y) =
1
y
e
= exp
y
log
which is in the form of an exponential family of distributions, with
Hence the canonical link function is
1
=
1
.
.
%------------------------------------------------------%
n
(ii)
(a)
April 2005
n
1
f ( y i ) =
\begin{itemize}
\item The likelihood is
i 1
yi
e
n
=
log
y i
i
i
i 1
m
\item Hence
n
=
c
i
i
i 1
\item The log-likelihood is

(
e
y i )
i 1
(
e y i )
i m 1
m
m
=
( n m )
e
n
y i
e
i 1
y i
i m 1
m
(b)
c
= m e
y i
i 1
m
c
= 0
m e
y i = 0
i 1
m
y i
= log
i 1
m
n
c
= ( n m ) e
y i
i m 1
n
c
= 0
y i = 0
( n m ) e
i m 1
n
y i
= log

%-------------------------------------------------------------------------------%
(c)
\item The deviance is 2(
i m 1
n m
f
c )
n
f
=
log y i
i 1
Page 10
y i
y i
n
=
log y i
i 1
n
%-----------------------------------------------------------------------------------%
% April 2005
% Examiners Report
\item Hence the deviance is
m
n
2
y i
m
log y i
n
log
i 1
m
n
m
log y i
n
log
i 1
i 1
1
m
m
= 2
log
i 1
1
m
1
m
log
m
n m
1
n m
n
y j
m
log
j 1
i m 1
n m
n
j 1
y i
j m 1
j 1
log
n
y j
j m 1
n
1
n m
y j
n m
j m 1
n
1
y j
y i
i m 1
y j
m
m
y j
n
y i
j 1
j 1
= 2
n
y j
j m 1
y i
i m 1
n
%----------------------------------------%
%%(iii)
\item The deviance residual is sign ( y i
y i ) D i where the deviance is
D i .
i 1
y i = 14.2
D 1 = 2
= 2
log y 1 1 log
1
m
log 7 1 log14.2
m
y i
i 1
y 1
1
m
m
y i
i 1
7
14.2
= 0.4
\item Hence the deviance residual is
\end{itemize}
%%%%%%%%%%%%%%%%%%%%%%%%%%%%%%%%%%%%
end{document}
