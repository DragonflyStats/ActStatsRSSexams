\documentclass[a4paper,12pt]{article}

%%%%%%%%%%%%%%%%%%%%%%%%%%%%%%%%%%%%%%%%%%%%%%%%%%%%%%%%%%%%%%%%%%%%%%%%%%%%%%%%%%%%%%%%%%%%%%%%%%%%%%%%%%%%%%%%%%%%%%%%%%%%%%%%%%%%%%%%%%%%%%%%%%%%%%%%%%%%%%%%%%%%%%%%%%%%%%%%%%%%%%%%%%%%%%%%%%%%%%%%%%%%%%%%%%%%%%%%%%%%%%%%%%%%%%%%%%%%%%%%%%%%%%%%%%%%

\usepackage{eurosym}
\usepackage{vmargin}
\usepackage{amsmath}
\usepackage{graphics}
\usepackage{epsfig}
\usepackage{enumerate}
\usepackage{multicol}
\usepackage{subfigure}
\usepackage{fancyhdr}
\usepackage{listings}
\usepackage{framed}
\usepackage{graphicx}
\usepackage{amsmath}
\usepackage{chng%%-- Page}

%\usepackage{bigints}
\usepackage{vmargin}

% left top textwidth textheight headheight

% headsep footheight footskip

\setmargins{2.0cm}{2.5cm}{16 cm}{22cm}{0.5cm}{0cm}{1cm}{1cm}

\renewcommand{\baselinestretch}{1.3}

\setcounter{MaxMatrixCols}{10}

\begin{document}
\begin{enumerate}
%%%%%%%%%%%%%%%%%%%%%%%%%%%%%%%%%%%%%%%%%%%%%%%%%%%%%%%%%%%%%%%%%%%%%%%%%%%%%%%%%%%%%%%%55

PLEASE TURN OVER10
(i) Explain what a conjugate prior distribution is.
(ii) The random variables X 1 , X 2 ,
f(x) = e
x
, X n are independent and have density function
(x > 0).
Show that the conjugate prior distribution for
(iii)
(a)
The density function of
f( ) =
s
( )
Show that E(1/ ) = s/(
(b)
(iv)

is a Gamma distribution.

is
1
e
s
( > 0).
1).
Hence if X 1 , X 2 , , X n is an independent random sample from an exponential distribution with parameter , show that the posterior
mean of 1/ can be expressed as a weighted average of the prior mean of 1/ and the sample average.

An insurer is considering introducing a new policy to provide insurance
against the failure of toasters within the first five years of purchase. Alan and Beatrice are underwriters working for the insurer. Based on his experience of
similar products, Alan believes that toasters last three years on average. Beatrice believes that six years is the average lifetime. Both are adamant and
are prepared to express their uncertainties about the average lifetime in terms of standard deviations of six months and one year respectively. They decide
to resolve their differences by testing a sample of toasters large enough to ensure the difference in their posterior expectations for the average lifetime
will be less than one year.
Calculate how many toasters they should test, assuming the exponential distribution is a good model for toaster lifetimes.
You may use the fact that if
Var(1/ ) = [E(1/ )] 2
~ ( , s) then
1
2
.
%%%%%%%%%%%%%%%%%%%%%%%%%%%%%%%%%%%%%%%%%%%%%%%%%%%%%%%%%%%%%%%%%%%%%%%%%%%%%%%%
10
April 2005
Examiners Report
(i) If, having take a sample from the distribution parameterised by , the posterior distribution of belongs to the same family as the prior distribution then the
prior is called a conjugate prior.
(ii) We know that the prior distribution of is ( , s). If X is the sample taken from the exponential distribution, then the posterior density satisfies:
f( X)
f(X )f( )
n
=
1
s
x i
e
i 1
n 1
e
s
e
( )
n
x )
i 1 i
( s
n
n , s
pdf of
x i
i 1
This means that the posterior distribution of also follows a Gamma
distribution and therefore the Gamma distribution satisfies the definition of a
conjugate prior.
(iii)
(a)
We know that
~ ( , s). So
E(1/ ) =
=
=
=
=
=
f ( )
d
0
1
e s
d
( )
s
0
2
s
0
s
s
1
s
1
s
e
( )
0
d
1
2
(
e s
d
1)
1
s
1
since the final integral is of the pdf of a (
%%-- Page 12
1, s) distribution.

% %%%%%%%%%%%%%%%%%%%%%%%%%%%%%%%%%%%%%%%%%%%%%
% April 2005
% Examiners Report
n
(b)
Posterior mean is E(1/ ) where
~
n , s
x i . The prior
i 1
s
mean is
1
. The previous result implies that the posterior mean is
given by
n
n
s
x 1
i 1
n 1
s
n 1
=
x 1
i 1
n 1
n
1
=
1
and
(iv)
n 1
n 1
s
1
n
n 1
x 1
i 1
n
n
= 1
n 1
First consider Alan s beliefs. We know from the formula given in the question
that
Var(1/ ) = E(1/ ) 2
1
2
Hence for Alan we have
0.5 2 = 3 2
1
2
which means that
2 =
and hence
9
= 36
0.25
= 38. Using the result for the posterior mean, we have
and hence s = 3
37 = 111. So Alan s prior distribution for
s
= 3
1
is (38, 111).
Similarly for Beatrice, we have
Var(1/ ) = E(1/ ) 2
1
2
%%-- Page 13%%%%%%%%%%%%%%%%%%%%%%%%%%%%%%%%%%%%%%%%%%%%%
April 2005
Examiners Report
Hence
1
1 2 = 6 2
2
which means that
2 =
36
= 36
1
and hence = 38 again. Using the results for the posterior mean, we have
s
= 6 and hence s = 6 37 = 222. So Beatrice s prior distribution for is
1
(38, 222).
We will use the weighted average formula above to calculate the difference in
the posterior means. First note that since both Alan and Beatrice have the
same we have
1
Z A = Z B =
n 1
=
37
.
37 n
So the difference in posterior means is given by
Z A
3 + (1
Z A )
x
Z B
6
(1
Z B )
x = 3
So we need to ensure that n is large enough that
3Z < 1
Z < 1/3
37
< 1/3
37 n
37
3
37 m
3
37
37 < n
n > 74.
% END OF EXAMINERS REPORT
% %%-- Page 14
Z.
\end{document}
