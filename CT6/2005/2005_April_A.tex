\documentclass[a4paper,12pt]{article}

%%%%%%%%%%%%%%%%%%%%%%%%%%%%%%%%%%%%%%%%%%%%%%%%%%%%%%%%%%%%%%%%%%%%%%%%%%%%%%%%%%%%%%%%%%%%%%%%%%%%%%%%%%%%%%%%%%%%%%%%%%%%%%%%%%%%%%%%%%%%%%%%%%%%%%%%%%%%%%%%%%%%%%%%%%%%%%%%%%%%%%%%%%%%%%%%%%%%%%%%%%%%%%%%%%%%%%%%%%%%%%%%%%%%%%%%%%%%%%%%%%%%%%%%%%%%

\usepackage{eurosym}
\usepackage{vmargin}
\usepackage{amsmath}
\usepackage{graphics}
\usepackage{epsfig}
\usepackage{enumerate}
\usepackage{multicol}
\usepackage{subfigure}
\usepackage{fancyhdr}
\usepackage{listings}
\usepackage{framed}
\usepackage{graphicx}
\usepackage{amsmath}
\usepackage{chngpage}

%\usepackage{bigints}
\usepackage{vmargin}

% left top textwidth textheight headheight

% headsep footheight footskip

\setmargins{2.0cm}{2.5cm}{16 cm}{22cm}{0.5cm}{0cm}{1cm}{1cm}

\renewcommand{\baselinestretch}{1.3}

\setcounter{MaxMatrixCols}{10}

\begin{document}
\begin{enumerate}

1 List the three main perils typically covered by employer s liability insurance.
%%%%%%%%%%%%%%%%%%%%%%%%%%%%%%%%%%%%%%%%%%%%
2 An insurer wishes to estimate the expected number of claims, , on a particular type
of policy. Prior beliefs about are represented by a Gamma distribution with density
function
f( ) =
1
( )
For an estimate, d, of
e
[3]
( > 0).
the loss function is defined as
d) 2 + d 2 .
L( , d) = (
Show that the expected loss is given by
E(L( , d)) =
(
1)
2 d
2
2 d 2
and hence determine the optimal estimate for
3
(i)
(ii)
4
[5]
Explain the disadvantages of using truly random, as opposed to pseudo-
random, numbers.
List four methods for the generation of random variates.

%%%%%%%%%%%%%%%%%%%%%%%%%%%%%%%%%%%%%%%%%%%%%%%%%%%%%%%%%%%%%%%%%%%%%%%%%%%%%%%
\newpage
% April 2005
% Examiners Report
The three main perils are:
accidents caused by the negligence of the employer or other employees
exposure to harmful substances
exposure to harmful working conditions
%%%%%%%%%%%%%%%%%%%%%%%%%%%%%%%%%%%%%%%%%%%%%%%%%%%%%%%%%%%%%%%%%%
2%%%%%%%
The expe%cted loss is given by
d) 2 + d %2 ]
E[L( , d)] = E[(
2%%
= E[
2 d + d 2 + d 2 ]
= E( 2 )
2dE( ) + 2d 2 .
Now we know that ~ ( , ). From the tables, we know that E( ) = / and
Var( ) = / 2 . Now
E( 2 ) = Var( ) + E( ) 2
2
=
=
So
2 2
( 1)
2
(
E[L( , d)] =
.
1)
2 d
2
2 d 2
as required.
First set
Then
f(d) = E[(L( , d)].
2
f ( d ) =
4d
and to minimise the expected loss, we must find the value of d* of d for which
f ( d *) = 0. This occurs when
4d* =
so that
Page 2
d* =
2
2
.
% Subject CT6 (Statistical Methods Core Technical)
April 2005
Examiners Report
We can confirm this is a minimum since f ( d ) = 4 > 0.
3
(i)
The stored table of random numbers generated by a physical process may be
too short
a combination of linear congruential generators (LCG) can
produce a sequence which is infinite for practical purposes.
It might not be possible to reproduce exactly the same series of random numbers again with a truly random number generator unless these are stored.
A LCG will generate the same sequence of numbers with the same seed.
Truly random numbers would require either a lengthy table or hardware enhancement compared with a single routine for pseudo random numbers.
(ii)
4
Inverse Transform method.
Acceptance-Rejection Method
Box-Muller algorithm (from the standard normal distribution)
Polar algorithm (from the standard normal distribution)
\end{document}
