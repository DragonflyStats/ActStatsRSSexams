\documentclass[a4paper,12pt]{article}

%%%%%%%%%%%%%%%%%%%%%%%%%%%%%%%%%%%%%%%%%%%%%%%%%%%%%%%%%%%%%%%%%%%%%%%%%%%%%%%%%%%%%%%%%%%%%%%%%%%%%%%%%%%%%%%%%%%%%%%%%%%%%%%%%%%%%%%%%%%%%%%%%%%%%%%%%%%%%%%%%%%%%%%%%%%%%%%%%%%%%%%%%%%%%%%%%%%%%%%%%%%%%%%%%%%%%%%%%%%%%%%%%%%%%%%%%%%%%%%%%%%%%%%%%%%%

\usepackage{eurosym}
\usepackage{vmargin}
\usepackage{amsmath}
\usepackage{graphics}
\usepackage{epsfig}
\usepackage{enumerate}
\usepackage{multicol}
\usepackage{subfigure}
\usepackage{fancyhdr}
\usepackage{listings}
\usepackage{framed}
\usepackage{graphicx}
\usepackage{amsmath}
\usepackage{chng%%-- Page}
%\usepackage{bigints}
\usepackage{vmargin}

% left top textwidth textheight headheight

% headsep footheight footskip
\setmargins{2.0cm}{2.5cm}{16 cm}{22cm}{0.5cm}{0cm}{1cm}{1cm}
\renewcommand{\baselinestretch}{1.3}
\setcounter{MaxMatrixCols}{10}
\begin{document} 

%%% - Question 4

24
XYZ bank are about to offer a new mortgage product to consumers with a poor creditrating. They currently offer a similar product to customers with normal credit
ratings. The normal product charges all customers a Standard Variable Rate (SVR) of 6% which moves up and down in line with short term interest rates. In addition there
is a maximum loan to value of 90%
in other words the customer cannot borrow
more than 90% of the value of the property. For loans above this level an additional Mortgage Indemnity Guarantee insurance premium must be paid
this protects the bank in the event that the borrower defaults and the value of the property has fallen.
There are no other charges on the normal product.
The bank intends to use its experience from the normal business as a basis for setting
terms on the new product.
5
6
(i) List the factors the bank should take into account when setting terms on the new product compared with the normal business.

(ii) Suggest ways in which the bank may mitigate the additional risks associated with this product.


(i) Higher risk of default.
Cost of MIG insurance.
Other insurers.
Cost of underwriting.
Profitability of normal product.
Adjust for underlying economic conditions.
Exceptional defaults.
(ii) Higher SVR.
Lower LTV.
Higher MIG.
Penalty Payments.
Increase charges.
Compulsory insurance.
Maximum loan amount.
Charges/SVR variable according to wish.
Other sensible suggestions were given credit.
Page 4
Max
Expected
Profit
589,750
687,700
684,625Subject CT6 (Statistical Methods Core Technical)
