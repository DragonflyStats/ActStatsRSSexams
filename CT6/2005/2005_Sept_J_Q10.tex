\documentclass[a4paper,12pt]{article}

%%%%%%%%%%%%%%%%%%%%%%%%%%%%%%%%%%%%%%%%%%%%%%%%%%%%%%%%%%%%%%%%%%%%%%%%%%%%%%%%%%%%%%%%%%%%%%%%%%%%%%%%%%%%%%%%%%%%%%%%%%%%%%%%%%%%%%%%%%%%%%%%%%%%%%%%%%%%%%%%%%%%%%%%%%%%%%%%%%%%%%%%%%%%%%%%%%%%%%%%%%%%%%%%%%%%%%%%%%%%%%%%%%%%%%%%%%%%%%%%%%%%%%%%%%%%

\usepackage{eurosym}
\usepackage{vmargin}
\usepackage{amsmath}
\usepackage{graphics}
\usepackage{epsfig}
\usepackage{enumerate}
\usepackage{multicol}
\usepackage{subfigure}
\usepackage{fancyhdr}
\usepackage{listings}
\usepackage{framed}
\usepackage{graphicx}
\usepackage{amsmath}
\usepackage{chng%%-- Page}
%\usepackage{bigints}
\usepackage{vmargin}

% left top textwidth textheight headheight

% headsep footheight footskip
\setmargins{2.0cm}{2.5cm}{16 cm}{22cm}{0.5cm}{0cm}{1cm}{1cm}
\renewcommand{\baselinestretch}{1.3}
\setcounter{MaxMatrixCols}{10}
\begin{document} 

%%% - Question 10

[Total 16]
PLEASE TURN OVER10
The total amounts claimed each year from a portfolio of insurance policies over n
years were x 1 , x 2 , , x n . The insurer believes that annual claims have a normal
distribution with mean
of
2
1 , where
and variance
is assumed to be normal with mean
is unknown. The prior distribution
2
2 .
and variance
(i) Derive the posterior distribution of .
(ii) Using the answer in (a), write down the Bayesian point estimate of
quadratic loss.
(iii)

under

Show that the answer in (b) can be expressed in the form of a credibility
estimate and derive the credibility factor.

The claims experience over five years for two companies was as follows:
Company A
Company B
(iv)
Year 1 2 3 4 5
Amount
Amount 421
343 417
335 438
356 456
366 463
380
Determine the Bayes credibility estimate of the premiums the insurer should
charge for each company based on the modelling assumptions of part (i), a
profit loading of 25% and the following parameters:
2
1
2
2
Company A Company B
400
500 300
350
800 600

(v)
Comment on the effect on the result of increasing
END OF PAPER
CT6 S2005
6
2
1 and
2
2 .

[Total 16]

%%%%%%%%%%%%%%%%%%%%%%%%%%

10
(i)
x 1 ,
(1 / (1.28434 * 1.13207)
1 / (1.73783 * 1.28434 * 1.13207))
, x n are the observed claims:
) 2
(
f( )
e
1
2
2
2
e
2 2
n 2
)
2
1
2
e
p(x )
(
) 2
( x 1
x
2
2
2
i 1
n
1
= e
2
1 i 1
2
n
1
e
2
1 i 1
2
1
= e
p( x)
2
2
(
2 x i )
n
2
n
2
1
) 2
( x i
2
x i
i 1
p(x ) p( )
1
e
2
2
2
(
2
2
1
)
2
2
1
n
2
x i
i 1
n
1
2
x i
n
2
2
2
2
2
1
i 1
2
2
2
2
2
1
2
= e
Page 11%%%%%%%%%%%%%%%%%%%%%%%%%%%%%%%%%%%%%%%%%%%%%%%%%%%%%%%%%%%%%%%%%
2
1
2
= e
2
1
n
2
1
n 2 2
2 2
1 2
2
n 2 2
2 2
1 2
2
September 2005
2
2
2
1
2
1
n
2
2
x i
i 1
2 2
1 2
2
2
%%%%%%%%%%%%%%%%%%%%%%%%%%%%%%%%%%
2
n
x i
i 1
2
2
2
1
n
2
2
e
n
2
1
i 1
x ~ N
(ii)
2
2
x i
2
1
,
2
2
n
2 2
1 2
2
1
n
2
2
Point estimator under quadratic loss is the posterior mean:
2
1
E( x) =
2
1
n
n
2
2
2
2
2
1
n
2
2
x i
i 1
n
x i
(iii)
E( x) = (1
Z)
+ Z
i 1
n
which is in the form of a credibility estimate, and
Z =
n
2
1
2
2
n
2
2
is the credibility factor.
(iv)
Company A Company B
5 5
2
1 500 350
2
2 800 600
0.8889 0.8955
439 356
400 300
n
2
2
Z =
2
2
x
Page 12
2
1
n%%%%%%%%%%%%%%%%%%%%%%%%%%%%%%%%%%%%%%%%%%%%%%%%%%%%%%%%%%%%%%%%%
Credibility Premium
= Zx (1 Z )
Premium (CP + 25%)
(v)
September 2005
434.7 350.1
543.3 437.7
%%%%%%%%%%%%%%%%%%%%%%%%%%%%%%%%%%
2
1 increases Reduces credibility factor and hence credibility premium
moves towards prior mean.
2
2 increases Increases credibility factor and hence credibility premium
moves towards sample mean.
END OF EXAMINERS REPORT
Page 13
