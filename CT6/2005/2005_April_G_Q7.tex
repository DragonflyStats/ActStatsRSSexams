\documentclass[a4paper,12pt]{article}

%%%%%%%%%%%%%%%%%%%%%%%%%%%%%%%%%%%%%%%%%%%%%%%%%%%%%%%%%%%%%%%%%%%%%%%%%%%%%%%%%%%%%%%%%%%%%%%%%%%%%%%%%%%%%%%%%%%%%%%%%%%%%%%%%%%%%%%%%%%%%%%%%%%%%%%%%%%%%%%%%%%%%%%%%%%%%%%%%%%%%%%%%%%%%%%%%%%%%%%%%%%%%%%%%%%%%%%%%%%%%%%%%%%%%%%%%%%%%%%%%%%%%%%%%%%%

\usepackage{eurosym}
\usepackage{vmargin}
\usepackage{amsmath}
\usepackage{graphics}
\usepackage{epsfig}
\usepackage{enumerate}
\usepackage{multicol}
\usepackage{subfigure}
\usepackage{fancyhdr}
\usepackage{listings}
\usepackage{framed}
\usepackage{graphicx}
\usepackage{amsmath}
\usepackage{chng%%-- Page}

%\usepackage{bigints}
\usepackage{vmargin}

% left top textwidth textheight headheight

% headsep footheight footskip

\setmargins{2.0cm}{2.5cm}{16 cm}{22cm}{0.5cm}{0cm}{1cm}{1cm}

\renewcommand{\baselinestretch}{1.3}

\setcounter{MaxMatrixCols}{10}

\begin{document}
7
The no claims discount (NCD) system operated by an insurance company has three levels of discount: 0%, 25% and 50%.
If a policyholder makes a claim they remain at or move down to the 0\% discount level
for two years. Otherwise they move up a discount level in the following year or remain at the maximum 50% level.
The probability of an accident depends on the discount level:
Discount Level Probability of accident
0%
25%
50% 0.25
0.2
0.1
The full premium payable at the 0% discount level is 750.
Losses are assumed to follow a lognormal distribution with mean 1,451 and standard deviation 604.4.
Policyholders will only claim if the loss is greater than the total additional premiums that would have to be paid over the next three years, assuming that no further accidents occur.
%%%%%%%%%%%%%%%%%%%%%%%%%%%%%%%%%%%%%%%%%%%%%%%%%%%%%%%%%%%%%%%%%%%%


%%%%%%%%%%%%%%%%%%%%%%%%%%%%%%%%%%%%%%%%%%%%%%%%%%%%%
\newpage 

%%- Question 7
\begin{itemize}
    \item 

(i)
Denote:
0
0*
1
2
0
0*
1
2
just had a claim
1 claim free year after accident or new customer
25%
50%
Premiums if no claim Premiums if claim Difference
750, 562.50, 375
562.50, 375, 375
375, 375, 375
375, 375, 375 750, 750, 562.50
750, 750, 562.50
750, 750, 562.50
750, 750, 562.50 375
750
937.50
937.50
So minimum claim in state 0 is 375, in state 0* is 750 and in states 1 and 2 is
937.50.
%%-- Page 6%%%%%%%%%%%%%%%%%%%%%%%%%%%%%%%%%%%%%%%%%%%%%
\item (ii)
April 2005
Examiners Report
P(Claim) = P(Claim Accident) . P(Accident)
= P(X > x) P(Accident)
Where X is the loss and x is the minimum loss for which a claim will be made.
E(x) = exp( + 1⁄2 2 ) = 1,451
Var(x) = exp(2( + 1⁄2 2 )) exp(( 2 ) 1) = 604.4 2
Therefore,
exp( 2 ) 1 = 604.4 2 / 1,451 2
exp( 2 ) = 1.1735
2
= 0.16
= 0.4
= 7.2
P(X > 375) = 1 ln 375 7.2
= 0.99927
0.4
P(X > 750) = 1 ln 750 7.2
= 0.9264
0.4
ln 937.50 7.2
= 0.8138
0.4
(X > 937.50) = 1
\item So the transition matrix is
\begin{verbatim}
0.2498 0.7502 0 0
0.2316 0 0.7684 0
0.1628 0 0 0.8372
0.0814 0 0 0.9186
\end{verbatim}
%%%%%%%%%%%%%%%%%%%%%55
\item (iii)
(iv)
0.2498
+ 0.0814 2
0.7502 0
0.7684 0 *
0.8372 1 + 0.9186 2
0 + 0 * + 1 + 2
1 = 0.7502 0.7684
0
0.8372 1
1
\item Average premium across portfolio
750
8
%%-- Examiners Report
= 0
= 0 *
= 1
= 2
= 1
= 0.5766 0
= 2 (1 0.9186)
2 = 10.2850 1
0 + 0.7502 0 + 0.5766 0 + 10.2850 (0.5766 0 ) = 1
8.2556 0 = 1
0 = 0.1211
0 * = 0.0909
1 = 0.7684 0.0909 = 0.0698
2 = 1
0
0 *
1 = 0.7182
0
+ 0.2316 0 * + 0.1628
April 2005
(0.1211 + 0.0909 + 0.0698
0.75 + 0.7182
0.5) = £467.59
\item (v) Intention is to automatically premium rate with NCD system. Small number of categories and the relatively low discount result in high proportion of policyholders in maximum discount category. Many more categories and higher discount levels would be required to correctly rate such a heterogeneous population.
\end{itemize}
\end{document}
