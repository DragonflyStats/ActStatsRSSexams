\documentclass[a4paper,12pt]{article}

%%%%%%%%%%%%%%%%%%%%%%%%%%%%%%%%%%%%%%%%%%%%%%%%%%%%%%%%%%%%%%%%%%%%%%%%%%%%%%%%%%%%%%%%%%%%%%%%%%%%%%%%%%%%%%%%%%%%%%%%%%%%%%%%%%%%%%%%%%%%%%%%%%%%%%%%%%%%%%%%%%%%%%%%%%%%%%%%%%%%%%%%%%%%%%%%%%%%%%%%%%%%%%%%%%%%%%%%%%%%%%%%%%%%%%%%%%%%%%%%%%%%%%%%%%%%

\usepackage{eurosym}
\usepackage{vmargin}
\usepackage{amsmath}
\usepackage{graphics}
\usepackage{epsfig}
\usepackage{enumerate}
\usepackage{multicol}
\usepackage{subfigure}
\usepackage{fancyhdr}
\usepackage{listings}
\usepackage{framed}
\usepackage{graphicx}
\usepackage{amsmath}
\usepackage{chngpage}

%\usepackage{bigints}
\usepackage{vmargin}

% left top textwidth textheight headheight

% headsep footheight footskip

\setmargins{2.0cm}{2.5cm}{16 cm}{22cm}{0.5cm}{0cm}{1cm}{1cm}

\renewcommand{\baselinestretch}{1.3}

\setcounter{MaxMatrixCols}{10}

\begin{document}
\begin{enumerate}

8
(i) Calculate the smallest loss for which a claim will be made for each of the four
states in the NCD system.
[2]
(ii) Determine the transition matrix for this NCD system.
[6]
(iii) Calculate the proportion of policyholders at each discount level when the
system reaches a stable state.
[3]
(iv) Determine the average premium paid once the system reaches a stable state.[1]
(v) Describe the limitations of simple NCD systems such as this one.
(i) Write down the general form of a statistical model for a claims run-off
triangle, defining all terms used.
(ii)
The table below shows the cumulative incurred claims on a portfolio of
insurance policies.
Accident Year
2000
2001
2002
CT6 A2005
[2]
[Total 14]
4
Delay Year
2,748
2,581
3,217
3,819
4,014
3,991
[5]The company decides to apply the Bornhuetter-Ferguson method to calculate
the reserves, with the assumption that the Ultimate Loss Ratio is 85%.
Calculate the reserve for 2002, if the earned premium is 5,012 and the paid
claims are 1,472.
[9]
[Total 14]
%%%%%%%%%%%%%%%%%%%%%%%%%%%%%%%%%%%%%%%%%%%%%%%%%%%%%%%%%%%%%%%%%%%%%%%%%%%%%%%%%%%%%%%%%%
(i) The general form can be written as
C ij = r j s i x i+j + e ij
where C ij
is incremental claims
r j is the development factor for year j, independent of origin year i,
representing proportion of claims paid by development year j
s i is a parameter varying by origin year, representing the exposure
x i+j is a parameter varying by calendar year, representing inflation
e ij
Page 8
is an error termSubject CT6 (Statistical Methods Core Technical)
(ii)
April 2005
Examiners Report
Development factors are
3,991
= 1.04504
3,819
7,833
= 1.46988
5,329
and
1
= 1
f
1
1
1.04504 1.46988
= 0.3490
2002: Emerging liability
= 5,012
0.85
0.3490
= 1,487
Reported liability 3,217
Ultimate liability is 4,704
Reserve = 4,704
1,472
= 3,232
