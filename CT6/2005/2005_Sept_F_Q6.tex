\documentclass[a4paper,12pt]{article}

%%%%%%%%%%%%%%%%%%%%%%%%%%%%%%%%%%%%%%%%%%%%%%%%%%%%%%%%%%%%%%%%%%%%%%%%%%%%%%%%%%%%%%%%%%%%%%%%%%%%%%%%%%%%%%%%%%%%%%%%%%%%%%%%%%%%%%%%%%%%%%%%%%%%%%%%%%%%%%%%%%%%%%%%%%%%%%%%%%%%%%%%%%%%%%%%%%%%%%%%%%%%%%%%%%%%%%%%%%%%%%%%%%%%%%%%%%%%%%%%%%%%%%%%%%%%

\usepackage{eurosym}
\usepackage{vmargin}
\usepackage{amsmath}
\usepackage{graphics}
\usepackage{epsfig}
\usepackage{enumerate}
\usepackage{multicol}
\usepackage{subfigure}
\usepackage{fancyhdr}
\usepackage{listings}
\usepackage{framed}
\usepackage{graphicx}
\usepackage{amsmath}
\usepackage{chng%%-- Page}
%\usepackage{bigints}
\usepackage{vmargin}

% left top textwidth textheight headheight

% headsep footheight footskip
\setmargins{2.0cm}{2.5cm}{16 cm}{22cm}{0.5cm}{0cm}{1cm}{1cm}
\renewcommand{\baselinestretch}{1.3}
\setcounter{MaxMatrixCols}{10}
\begin{document} 

%%% - Question 6

The number of claims on a portfolio of washing machine insurance policies follows a
Poisson distribution with parameter 50. Individual claim amounts for repairs are a
random variable 100X where X has a distribution with probability density function
3
(6 x x 2 5) 1 x 5
f(x) = 32
0
otherwise
In addition, for each claim (independently of the cost of the repair) there is a 30%
chance that an additional fixed amount of \$200 will be payable in respect of water
damage.
\begin{enumerate}
\item (i) Calculate the mean and variance of the total individual claim amounts.
\item (ii) Calculate the mean and variance of the aggregate claims on the portfolio. 
\end{enumerate}
%%%%%%%%%%%%%%%%%%%%%%%%%%%%%%%%%%%

We first calculate E(100X) and Var(100X). Taking the expectation first, we have
E(100X) = 100E(X)
= 100
3
32
300
=
2 x 3
32
=
5
1
6 x 2
x 4
4
x 3 5 xdx
5 x 2
2
5
1
300
(31.25 0.75)
32
= 300.
Now for the variance
E((100X) 2 ) = 100 2
= 100 2
E(X 2 )
3
32
5
1
30, 000 6 x 4
=
32
4
Page 6
6 x 3 x 4 5 x 2 dx
x 5
5
5 x 3
3
5
1Subject CT6 (Statistical Methods Core Technical)
=
September 2005
Examiners Report
30, 000
(104.166 0.366)
32
= 98,000
so that
VarX = E(X 2 )
E(x) 2
= 98,000
300 2
= 8000
= (89.44) 2 .
Now let Y denote the cost (if any) of the associated repair. Then Y is independent of X
and
E(Y) = 0.3
200 = 60
and
Var(Y) = 0.3
200 2
60 2 = 8,400 = (91.65) 2 .
So the mean individual claim amount Z is
E(X + Y) = E(X) + E(Y) = 300 + 60 = 360
and the variance of an individual claim is
Var(X + Y) = Var(X) + Var(Y) = 7,998.75 + 8,400.00 = 16,398.75
The mean and variance of the aggregate claims S are given by the formulae
E(S) = E(N)E(Z)
and
Var(S) = E(N) Var(Z) + Var(N)E(Z) 2
where N is the total number of claims. In our case
E(S) = 50
360 = 18,000
and
Var(S) = 50
16,398.75 + 50
360 2 = 7,299,937.5 = (2,701.84) 2
Page 7Subject CT6 (Statistical Methods Core Technical)
