\documentclass[a4paper,12pt]{article}

%%%%%%%%%%%%%%%%%%%%%%%%%%%%%%%%%%%%%%%%%%%%%%%%%%%%%%%%%%%%%%%%%%%%%%%%%%%%%%%%%%%%%%%%%%%%%%%%%%%%%%%%%%%%%%%%%%%%%%%%%%%%%%%%%%%%%%%%%%%%%%%%%%%%%%%%%%%%%%%%%%%%%%%%%%%%%%%%%%%%%%%%%%%%%%%%%%%%%%%%%%%%%%%%%%%%%%%%%%%%%%%%%%%%%%%%%%%%%%%%%%%%%%%%%%%%

\usepackage{eurosym}
\usepackage{vmargin}
\usepackage{amsmath}
\usepackage{graphics}
\usepackage{epsfig}
\usepackage{enumerate}
\usepackage{multicol}
\usepackage{subfigure}
\usepackage{fancyhdr}
\usepackage{listings}
\usepackage{framed}
\usepackage{graphicx}
\usepackage{amsmath}
\usepackage{chng%%-- Page}
%\usepackage{bigints}
\usepackage{vmargin}

% left top textwidth textheight headheight

% headsep footheight footskip
\setmargins{2.0cm}{2.5cm}{16 cm}{22cm}{0.5cm}{0cm}{1cm}{1cm}
\renewcommand{\baselinestretch}{1.3}
\setcounter{MaxMatrixCols}{10}
\begin{document} 

%%% - Question 3

A manufacturer of specialist products for the retail market must decide which product
to make in the coming year. There are three possible choices basic, deluxe or
supreme
each with different tooling up costs. The manufacturer has fixed
overheads of \$1,300,000.
The revenue and tooling up costs for each product are as follows:
Basic
Deluxe
Supreme
Tooling up costs
\$ Revenue per item sold
\$
100,000
400,000
1,000,000 1.00
1.20
1.50
Last year the manufacturer sold 2,100,000 items and is preparing forecasts of
profitability for the coming year based on three scenarios: Low sales (70% of last
year s level); Medium sales (same as last year) and High sales (15% higher than last
year).
(i) Determine the annual profits in \$ under each possible combination. 
(ii) Determine the minimax solution to this problem. 
(iii) Determine the Bayes criterion solution based on the annual profit given the
probability distribution: P(Low) = 0.25; P(Medium) = 0.6 and P(High) = 0.15.

[Total 7]
CT6 S2005

%%%%%%%%%%%%%%%%%%%%%%%%%%%%%%%%%%%%%%%%%%%%%%%%%%%%%%%%%%%%%%%%%%%%%%%%%%%%%%%

Page 3%%%%%%%%%%%%%%%%%%%%%%%%%%%%%%%%%%%%%%%%%%%%%%%%%%%%%%%%%%%%%%%%%
3
(i)
Basic
Deluxe
Supreme
September 2005
%%%%%%%%%%%%%%%%%%%%%%%%%%%%%%%%%%
Revenue (\$)
Low Medium High
1,470,000
1,764,000
2,205,000 2,100,000
2,520,000
3,150,000 2,415,000
2,898,000
3,622,500
Low Medium High
1,400,000
1,700,000
2,300,000 1,400,000
1,700,000
2,300,000 1,400,000
1,700,000
2,300,000
Low Medium High
70,000
64,000
95,000 700,000
820,000
850,000
Costs (\$)
Basic
Deluxe
Supreme
Profit (\$)
Basic
Deluxe
Supreme
4
1,015,000
1,198,000
1,322,500
Min
70,000 1,015,000
64,000 1,198,000
95,000 1,322,500
(ii) Minimax: Decision is Basic .
(iii) Highest expected profit. Decision is Deluxe .
(i) Higher risk of default.
Cost of MIG insurance.
Other insurers.
Cost of underwriting.
Profitability of normal product.
Adjust for underlying economic conditions.
Exceptional defaults.
(ii) Higher SVR.
Lower LTV.
Higher MIG.
Penalty Payments.
Increase charges.
Compulsory insurance.
Maximum loan amount.
Charges/SVR variable according to wish.
Other sensible suggestions were given credit.
Page 4
Max
Expected
Profit
589,750
687,700
684,625%%%%%%%%%%%%%%%%%%%%%%%%%%%%%%%%%%%%%%%%%%%%%%%%%%%%%%%%%%%%%%%%%
