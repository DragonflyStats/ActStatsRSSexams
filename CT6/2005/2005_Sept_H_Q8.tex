\documentclass[a4paper,12pt]{article}

%%%%%%%%%%%%%%%%%%%%%%%%%%%%%%%%%%%%%%%%%%%%%%%%%%%%%%%%%%%%%%%%%%%%%%%%%%%%%%%%%%%%%%%%%%%%%%%%%%%%%%%%%%%%%%%%%%%%%%%%%%%%%%%%%%%%%%%%%%%%%%%%%%%%%%%%%%%%%%%%%%%%%%%%%%%%%%%%%%%%%%%%%%%%%%%%%%%%%%%%%%%%%%%%%%%%%%%%%%%%%%%%%%%%%%%%%%%%%%%%%%%%%%%%%%%%

\usepackage{eurosym}
\usepackage{vmargin}
\usepackage{amsmath}
\usepackage{graphics}
\usepackage{epsfig}
\usepackage{enumerate}
\usepackage{multicol}
\usepackage{subfigure}
\usepackage{fancyhdr}
\usepackage{listings}
\usepackage{framed}
\usepackage{graphicx}
\usepackage{amsmath}
\usepackage{chng%%-- Page}
%\usepackage{bigints}
\usepackage{vmargin}

% left top textwidth textheight headheight

% headsep footheight footskip
\setmargins{2.0cm}{2.5cm}{16 cm}{22cm}{0.5cm}{0cm}{1cm}{1cm}
\renewcommand{\baselinestretch}{1.3}
\setcounter{MaxMatrixCols}{10}
\begin{document} 

%%% - Question 8


The following time series model is used for the monthly inflation rate (Y t ) in a
particular country:
Y t = 0.4Y t
1
+ 0.2Y t
2
+ Z t + 0.025
where {Z t } is a sequence of uncorrelated identically distributed random variables
whose distributions are normal with mean zero.
(i)
Derive the values of p, d and q, when this model is considered as an
ARIMA(p, d, q) model. 
(ii) Determine whether {Y t } is a stationary process. 
(iii) Assuming an infinite history, calculate the expected value of the rate of
inflation over this. [1]
CT6 S2005
49
(iv) Calculate the autocorrelation function of {Y t }.

(v) Explain how the equivalent infinite-order moving average representation of
{Y t } may be derived.

[Total 13]

%%%%%%%%%%%%%%%%%%%%%%%%%%%%%%%%%%%%%%%%%%%%%%%%%%%%%%%%%%%%%%%%%%%%%%%%%%%%%%%%%%%%%%%5

8
(i)
(1
0.4B
September 2005
Examiners Report
0.2B 2 ) Y t = Z t + 0.025
Characteristic equation
1
0.4z
0.2z 2 = 0
has no root at z = 1, so d = 0.
No functional dependence on Z t 1 , Z t 2 , etc., so q = 0.
Hence this is an ARIMA(2, 0, 0).
(ii) Roots of characteristic equation are 1
so {Y t } is stationary.
(iii) Mean is stationary over time
(1
(iv)
Y t
0.4 0.2)E[Y t ] = 0.025
E[Y t ] = 0.025
= 0.0625.
0.4
0.0625 = 0.4(Y t
k
= E[(Y t
0.0625) + 0.2(Y t
1
0.0625)(Y t
Put k = 1, and note that
0 = 1 and
1 1
= 0.4 + 0.2
2 = 0.4 1 + 0.2 = 0.4
3 = 0.4 2 + 0.2
1
2
0.0625) + Z t
0.0625)] = 0.4
k
1
6 , which are outside ( 1, +1),
1
=
=
k 1
+ 0.2
k 2
1
0.4
= 0.5
0.8
= 0.26
and so on.
(v)
(1
0.4B
Y t
0.2B 2 )(Y t
0.0625 = (1
0.0625) = z t
0.4B
0.2B 2 ) 1 Z t
So we need to invert (1 0.4B 0.2B 2 )
and multiply by Z t to obtain the equivalent moving average process.
Page 9Subject CT6 (Statistical Methods Core Technical)
9
September 2005
Examiners Report
(Figures in £000s)
AY
2001
2002
2003
2004
Inflation factors for each development year
0
1
2
3
1.02499
1.00390
0.99200
1.00000
1.00390
0.99200
1.00000
0.99200
1.00000
1.00000
Other inflation adjustments were given credit providing they were sensible, consistent and
some explanation was given.
Inflation adjusted claim payments in mid 2004
prices
AY
2001
2002
2003
2004
0 1 2 3
1,287.39
1,444.61
1,530.66
1,480 948.69
1,063.42
1,133 625.95
723 378
Inflation adjusted cumulative claim payments in
mid 2004 prices
AY
2001
2002
2003
2004
Column sum
Column sum minus last entry
Development factor
0 1 2 3
1,287.39
1,444.61
1,530.66
1,480 2,236.08
2,508.03
2,663.66 2,862.03
3,231.03 3,240.03
7,407.77
4,744.11
1.28434 6,093.06
2,862.03
1.13207 3,240.03
2,572.0
1,092.0
1.025
1,119.3
2,357.4 3,303.3
731.3
1.050625
768.3 3,739.6
436.3
1.076891
469.8
4,262.66
1.73783
(i)
Outstanding amounts arising from 2004 policies
Accumulated
Disaccumulated
Inflation
Inflation adj by year
Total
For example, 2,572.0 = 1,480
Page 10
1,480.0
1.7378Subject CT6 (Statistical Methods Core Technical)
(ii)
Ultimate amount for 2004 policies
September 2005
5,250
Examiners Report
0.75 = 3,937.50
Outstanding amounts for 2004 policies
1,149.8
1.025
1,178.5
2,482.2
Infl adj by year
Inflation adj by year
Total
770.0
1.050625
809.0
459.4
1.076781
494.7
For example,
1,149.8 = 3,937.5


%%%%%%%%%%%%%%%%%%%%%%%%%%%%%%%%%%%%%%%%%%5
8
(i)
(1
0.4B
September 2005
%%%%%%%%%%%%%%%%%%%%%%%%%%%%%%%%%%
0.2B 2 ) Y t = Z t + 0.025
Characteristic equation
1
0.4z
0.2z 2 = 0
has no root at z = 1, so d = 0.
No functional dependence on Z t 1 , Z t 2 , etc., so q = 0.
Hence this is an ARIMA(2, 0, 0).
(ii) Roots of characteristic equation are 1
so {Y t } is stationary.
(iii) Mean is stationary over time
(1
(iv)
Y t
0.4 0.2)E[Y t ] = 0.025
E[Y t ] = 0.025
= 0.0625.
0.4
0.0625 = 0.4(Y t
k
= E[(Y t
0.0625) + 0.2(Y t
1
0.0625)(Y t
Put k = 1, and note that
0 = 1 and
1 1
= 0.4 + 0.2
2 = 0.4 1 + 0.2 = 0.4
3 = 0.4 2 + 0.2
1
2
0.0625) + Z t
0.0625)] = 0.4
k
1
6 , which are outside ( 1, +1),
1
=
=
k 1
+ 0.2
k 2
1
0.4
= 0.5
0.8
= 0.26
and so on.
(v)
(1
0.4B
Y t
0.2B 2 )(Y t
0.0625 = (1
0.0625) = z t
0.4B
0.2B 2 ) 1 Z t
So we need to invert (1 0.4B 0.2B 2 )
and multiply by Z t to obtain the equivalent moving average process.
Page 9%%%%%%%%%%%%%%%%%%%%%%%%%%%%%%%%%%%%%%%%%%%%%%%%%%%%%%%%%%%%%%%%%
\end{document}
