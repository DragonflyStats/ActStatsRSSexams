1 Give two examples of the main types of liability insurance, stating for each example a
typical insured peril.
[4]
2 A claim amount distribution is normal with unknown mean μ and known standard
deviation £50. Based on past experience a suitable prior distribution for μ is normal
with mean £300 and standard deviation £20.
(i)
(ii)
Calculate the prior probability that μ, the mean of the claim amount
distribution, is less than £270.
[1]
A random sample of 10 current claims has a mean of £270.
(a) Determine the posterior distribution of μ.
(b) Calculate the posterior probability that μ is less than £270 and
comment on your answer.
[5]
[Total 6]

%%%%%%%%%%%%%%%%%%%%%%%%%%%%%%%%%%%%%%%%%%%%%%%%%%%%%%%%%%%%%%%%%%%%%%%%%%%%%%%%%%%%%%%%%%%%%%%%%%%%%%%%%%%%%%%%%%%%%%5
1
Type
Employers’ liability
Motor 3 rd party liability
Public liability
Product liability
Professional Indemnity
Typical perils
• Accidents caused by employer negligence
• Exposure to harmful substances
• Exposure to harmful working conditions
• Road traffic accidents
• Will relate to the type of policy
• Faulty design, manufacture or packaging of
product
• Incorrect or misleading instructions
• Wrong medical diagnosis, error in medical
operation etc.
Comment: Most of the candidates did very well here.
2
(i)
P ( μ < 270) = P ( N (300, 20 2 ) < 270)
270 − 300
)
20
= P ( N (0,1) < − 1.5)
= P ( N (0,1) <
= 1 − 0.93319
= 0.06681
(ii)
(a)
Using the result from page 28 of the tables, the posterior distribution of μ is
normal, with mean
10 × 270 300
+ 2 )
2
20
μ * = 50
(
10
1
( 2 + 2 )
50
20
= £281.54
and variance
σ * 2 =
1
10
1
+ 2
2
50
20
= 153.85 = 12.40 2
So the posterior distribution of μ is N (281.54,12.40 2 ) .
Page 2Subject CT6 (Statistical Methods Core Technical) — April 2008 — Examiners’ Report
(b)
The posterior probability required is given by:
270 − 281.54
)
12.4
= P ( N (0,1) < − 0.931)
= 1 − (0.9 × 0.82381 + 0.1 × 0.82639)
P ( N (281.54,12.40 2 ) < 270) = P ( N (0,1) <
= 0.1759
Comment: The probability that the true mean is less than £270 has risen, as the
sample evidence suggests that the true mean is less than the mean of the prior
distribution. Nevertheless, the sample size is relatively small, and the variance of the
prior distribution is also small, so that a reasonable weight is still given to the prior
information.
Comment: Some candidates did not use tables for generating the posterior
parameters in (ii)(a). A few gave the correct interpretation in (ii)(b).
