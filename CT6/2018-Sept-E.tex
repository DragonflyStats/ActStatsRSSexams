PLEASE TURN OVER9
9
An actuary is modelling a set of data which consists of 100 consecutive observations,
y 9 1 , y 2 , ... , y 100 . The data has the following statistics:
An actuary
An is
actuary
modelling
is modelling
a set of data
a set which
of data consists
which consists
of 100 consecutive
of 100 consecutive
observations,
observations,
100
y 1 , y 2 , 1 ... y 1 , , y y 100
The
, y data
has data
the following
has the following
statistics: statistics:
2 , . ...
100 . The
y =
∑ y i = A = 10.5
100 i = 1
1 100 1 100
y 100

y  y 
A 
 y 10.5

A  10.5

i
i
2 100
100
i
i


1
1
y – y = B = 290
∑ (
)
i
100
100 2
2
y  
y   y  B
 
y  290
 B  290

100 
∑ ( y i – 1 – y ) ( y i – y ) = C = 60
i =  2 y 
  C  60

 y   y y  y    y C   y 60
i = 1
i  1
i
i  1
100
 2
i 100
i
100
i  1
i  2
i  i 1
i
= – 240
) y D =  D 
∑  ( y y i –   2 y –   y y ) y ( y  i y –   y 
y   240

D 
 240

100
i i  = 3 3
100
i  2
i  3
i  2 i
i
(i) Calculate
(i)
Calculate
the values
the of values
the sample
of the auto-correlations
sample auto-correlations
r r 1 1 and r r 2 2 r . .[3]
[3]
1 and r 2 . [3]
(ii) Calculate
(ii)
Calculate
the first two
the first
two partial
sample auto-correlation
auto-
partial auto- values φφ̂ 1 and φφ̂ φφ̂ 2 .  [2]
φφ̂
sample
[2] [2]
The
The is
actuary
considering
is considering
two
two different
models for
for
models
this data:
data:
for this data:
The actuary
actuary
is
considering
two different
different
models
this
a 0 
y a 
y a 1 0   +  a ε t 1 y t  1   t
Model
Model
Model X:
X:
y y t t = 
a X:
0 + t a 1 1 y t t  –1
t
Model Y:

y y t  b t 1 –1
b b 2 1 b y y t 2 t   2 y 1 t  – 2  b t 2 + y t ε  2 t   t
Model
Y:
Model
y y t t =
b b Y:
0  +
0 + y b t b 1 1 
2
where e  t where
is a standard
process,
 t is a white-noise
standard
white-noise
white-noise
process, with
process,
with variance
variance
with σ
variance
σ 2 . .
σ 2 .
where
t is a standard
(iii)
(iii) Estimate the parameters (including σ 2 2 ) for both
Models X and Y, using the
Estimate
(iii) Estimate
the parameters
the parameters
(including
(including
σ ) for both
σ 2 ) for
Models
both X
Models
and Y, X using
and Y,
the using the
method of moments.
[10]
method of
method
moments.
of moments.
[10]
[10]
(iv)
(iv)
 Explain
whether whether
each of
of Models
Models
X
and Y
Markov
[3]
Explain
(iv)
Explain
whether
each
each of X
Models
and
Y X satisfy
satisfy
and Y the
the
satisfy
Markov
the property.
Markov
property. property.
[3]
[3]
[Total
18]
[Total 18]
[Total 18]
END OF PAPER
END OF
END
PAPER
OF PAPER
CT6 S2018–6
CT6 S2017–6
CT6 S2017–6

%%%%%%%%%%%%%%%%%%%%%%%%%%%%%%%%%%%%%%%%%%%%%%%%%%%%%%%%%%%%%%%%%%%%%%%%%%%%%%%
Q9
(i)
C 6
= = 0.2068966
B 29
D
24
r 2 = =
−
=
− 0.8275862
B
29
r =
1
(ii)
φ 11 = ρ =
0.2068966
1
φ 22 =
(iii)
ρ 2 − ρ 1 2
= − 0.9093168
1 − ρ 1 2
[11⁄2]
[11⁄2]
[1]
[1]
Model X:
a =
ρ =
0.2068966
1
1
[1]
^
σ 2 = γ 0 − a 1 γ 1 = γ 0 − a 1 ρ 1 γ 0 = 2.9 * (1 − 0.2068966 2 ) = 2.775862
a 0 = y (1 − a 1 ) = 10.5 * (1 − 0.2068966) = 8.327586
[11⁄2]
[1]
Model Y:
γ 1 =
b 1 γ 0 + b 2 γ 1 ,
γ 2 =
b 1 γ 1 + b 2 γ 0 ,
ρ 1 =
b 1 + b 2 ρ 1
ρ 2 =
b 1 ρ 1 + b 2

b ˆ 2 =
Ψ 22 =
− 0.9093168
ˆ ρ (1 − b =
b =
0.2068966 * (1 − ( − 0.9093168))
= 0.3950312
1
1
2 )
[1]
[1]
[1]
[1]
σ 2 = γ 0 − b 1 γ 1 − b 2 γ 2 = γ 0 ( 1 − b 1 ρ 1 − b 2 ρ 2 )
[11⁄2]
=
2.9 * (1 − 0.2068966 * 0.3950312 − 0.9093168 * 0.8275862) =
0.480621
Page 9Subject CT6 (Statistical Methods Core Technical) – September 2018 – Examiners’ Report
b 0 = y (1 − b 1 − b 2 ) = 10.5 * (1 − 0.3950312 + 0.9093168) = 15.9
(iv)
[1]
Markov property requires that the process depends only on the previous
observation.
[1]
This holds for model X, which is AR1 ...
[1]
... but not for Y, which is AR2
[1]
[Total 18]
Most candidates were able to score well on parts (i), (ii) & (iv) but only the strongest
candidates were able to gain most or all of the marks in part (iii).
END OF EXAMINERS’ REPORT
Page 10
