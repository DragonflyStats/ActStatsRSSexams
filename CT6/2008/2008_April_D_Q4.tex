\documentclass[a4paper,12pt]{article}

%%%%%%%%%%%%%%%%%%%%%%%%%%%%%%%%%%%%%%%%%%%%%%%%%%%%%%%%%%%%%%%%%%%%%%%%%%%%%%%%%%%%%%%%%%%%%%%%%%%%%%%%%%%%%%%%%%%%%%%%%%%%%%%%%%%%%%%%

\usepackage{eurosym}
\usepackage{vmargin}
\usepackage{amsmath}
\usepackage{graphics}
\usepackage{epsfig}
\usepackage{enumerate}
\usepackage{multicol}
\usepackage{subfigure}
\usepackage{fancyhdr}
\usepackage{listings}
\usepackage{framed}
\usepackage{graphicx}
\usepackage{amsmath}
\usepackage{chngpage}

%\usepackage{bigints}
\usepackage{vmargin}
% left top textwidth textheight headheight

% headsep footheight footskip
\setmargins{2.0cm}{2.5cm}{16 cm}{22cm}{0.5cm}{0cm}{1cm}{1cm}
\renewcommand{\baselinestretch}{1.3}
\setcounter{MaxMatrixCols}{10}

\begin{document}




%-----------------------------------------------------------------------%
4
$Y_1 , Y_2 , ..., Y_n$ are independent observations from a normal distribution with $E[Y_i ] = \mu i$
and $Var[Y_i] = \sigma 2$ .
\begin{enumerate}[(i)]
\item  Write the density of $Y_i$ in the form of an exponential family of distributions. 
\item Identify the natural parameter and derive the variance function.
\item Show that the Pearson residual is the same as the deviance residual.
\end{enumerate}
%%%%%%%%%%%%%%%%%%%%%%%%%%%%%%%%%%%%%%%%%%%%%%%%%%%%%%%%%%%%%%
4
\begin{itemize}
\item (i)
f(y i ) =
⎧ ⎪ ( y − \mu ) 2 ⎫ ⎪
exp ⎨ − i 2 i ⎬
2 \sigma
2 \pi\sigma 2
⎩ ⎪
⎭ ⎪
1
2
logf = − 1⁄2 log(2 \pi\sigma ) −
=
\[y i \mu i − 1⁄2 \mu i 2
\sigma 2
( y i 2 − 2 y i \mu i + \mu i 2 )
2 \sigma 2
− 1⁄2 log(2 \pi\sigma 2 ) − 1⁄2
y i 2
\sigma 2\]
which is the form of an exponential family of distributions.
\item (ii)
The natural parameter is \mu i
\theta i = \mu i
b(\theta i ) = 1⁄2 \mu i 2 = 1⁄2 \theta i 2
b ′ ( \theta i ) = \theta i
b ′′ ( \theta i ) = 1
Hence V(\mu i ) = 1
\item (iii)
The scaled deviance is
⎡ ( y − y ) 2
⎤
( y − \hat{\mu} ) 2
2 ∑ ⎢ − i 2 i − 1⁄2 log(2 \pi\sigma^2 ) + i 2 i + 1⁄2 log(2 \pi\sigma 2 ) ⎥
2 \sigma
2 \sigma
⎣ ⎢
⎦ ⎥
=
Page 4
∑
( y i − \hat{\mu} i ) 2
\sigma 2
%%%%%%%%%%%%%%%%%%%%%%%%%%%%%%%%%%%%%%%%%%%%%%%%%%%%%%%%%%%%%%%%%
\item 
Hence the deviance residual is
( y i − \hat{\mu} i ) 2
y i − \hat{\mu} i
\sigma
\sigma
y − \hat{\mu} i
y − \hat{\mu}

\item The Pearson residual is i
= i
\sigma V ( \hat{\mu} i )
\sigma
sign( y i − \hat{\mu} )
2
=

\item Comment: This was a relatively easy question and many scored full marks in (i) and
(iii) here.
\end{itemize}

\end{document}
