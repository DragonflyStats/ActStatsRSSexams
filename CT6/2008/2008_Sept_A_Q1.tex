\documentclass[a4paper,12pt]{article}

%%%%%%%%%%%%%%%%%%%%%%%%%%%%%%%%%%%%%%%%%%%%%%%%%%%%%%%%%%%%%%%%%%%%%%%%%%%%%%%%%%%%%%%%%%%%%%%%%%%%%%%%%%%%%%%%%%%%%%%%%%%%%%%%%%%%%%%%

\usepackage{eurosym}
\usepackage{vmargin}
\usepackage{amsmath}
\usepackage{graphics}
\usepackage{epsfig}
\usepackage{enumerate}
\usepackage{multicol}
\usepackage{subfigure}
\usepackage{fancyhdr}
\usepackage{listings}
\usepackage{framed}
\usepackage{graphicx}
\usepackage{amsmath}
\usepackage{chngpage}

%\usepackage{bigints}
\usepackage{vmargin}
% left top textwidth textheight headheight

% headsep footheight footskip
\setmargins{2.0cm}{2.5cm}{16 cm}{22cm}{0.5cm}{0cm}{1cm}{1cm}
\renewcommand{\baselinestretch}{1.3}
\setcounter{MaxMatrixCols}{10}

\begin{document}

1
Claim amounts on a portfolio of insurance policies have an unknown mean $\mu$. Prior beliefs about \mu are described by a distribution with mean \mu 0 and variance $\sigma_0^2$ . Data
are collected from n claims with mean claim amount x and variance $s^2$ . A credibility
estimate of $\mu$ is to be made, of the form
Zx + (1 − Z ) \mu 0 .
Suggestions for the choice of Z are:
n \sigma 0 2
A
n \sigma 0 2 + s 2
n \sigma 0 2
B
n \sigma 0 2 + n
\sigma 0 2
C
2
3
n + \sigma 0 2
Explain whether each suggestion is an appropriate choice for Z. 


Page 2%%%%%%%%%%%%%%%%%%%%%%%%%%%%%%%%%%%%%%%%%%%%%%%%% — September 2008 — Examiners’ Report
1
2
A This is an appropriate choice – the larger the value of n, the closer Z is to 1 and the more weight is placed on the data. Furthermore, high values of the variance of the
prior (indicating uncertainty in the prior) lead to higher values of Z and so more weight on the sample data. Finally, high variance in the sample reduces the value of Z
and places more reliance on the prior.

B This is not appropriate – the value is a constant independent of the size of the sample, whereas we would expect more weight on the sample the larger the
sample.

C This is not appropriate – the greater the value of n, the lower the value of Z and the less weight is put on the sample. This is the reverse of what we would
expect.


