\documentclass[a4paper,12pt]{article}

%%%%%%%%%%%%%%%%%%%%%%%%%%%%%%%%%%%%%%%%%%%%%%%%%%%%%%%%%%%%%%%%%%%%%%%%%%%%%%%%%%%%%%%%%%%%%%%%%%%%%%%%%%%%%%%%%%%%%%%%%%%%%%%%%%%%%%%%

\usepackage{eurosym}
\usepackage{vmargin}
\usepackage{amsmath}
\usepackage{graphics}
\usepackage{epsfig}
\usepackage{enumerate}
\usepackage{multicol}
\usepackage{subfigure}
\usepackage{fancyhdr}
\usepackage{listings}
\usepackage{framed}
\usepackage{graphicx}
\usepackage{amsmath}
\usepackage{chngpage}

%\usepackage{bigints}
\usepackage{vmargin}
% left top textwidth textheight headheight

% headsep footheight footskip
\setmargins{2.0cm}{2.5cm}{16 cm}{22cm}{0.5cm}{0cm}{1cm}{1cm}
\renewcommand{\baselinestretch}{1.3}
\setcounter{MaxMatrixCols}{10}

\begin{document}
7
Consider the following model applied to some quarterly data:
Y t = e t + \beta 1 e t-1 + \beta 4 e t-4 + \beta 1 \beta 4 e t-5
where e t is a white noise process with mean zero and variance \sigma 2 .
(i) Express in terms of \beta 1 and \beta 4 the roots of the characteristic polynomial of the
MA part, and give conditions for invertibility of the model.

(ii) Derive the autocorrelation function (ACF) for Y t .

For our particular data the sample ACF is:
(iii)



Lag ACF
1
2
3
4
5
6
7 0.73
0.14
0.37
0.59
0.24
0.12
0.07
Explain whether these results confirm the initial belief that the model could be
appropriate for these data.

%%%%%%%%%%%%%%%%%%%%%%%%%%%%%%%%%%%%%%%%%%%%%%%%%%%%%%%%%%%%%%%%%%%%%%%%%%

%%%%%%%%%%%%%%%%%%%%%%%%%%%%%%%%%%%%%%%%%%%%%%%%%%%%%%%%%%%%%%%%%%%%%%%%%%
7
\begin{itemize}
    \item 

(i)
Using the back shift operator it can be seen that
Y t = (1 + \beta 1 B) (1 + \beta 4 B 4 ) e t .
The invertibility conditions are then \beta 1 < 1 and \beta 4 < 1.

\item
(ii)
Since E (Y t ) = 0, γ 0 = E ( y t 2 ) = \sigma 2 (1 + \beta 1 2 + \beta 2 4 + \beta 1 2 \beta 2 4 ) = \sigma 2 (1 + \beta 1 2 )(1 + \beta 2 4 ).
\item  Similarly it can be shown that
\begin{itemize}
\item γ 1 = E (Y t Y t - 1 ) = \sigma 2 \beta 1 (1 + \beta 24 )
\item γ 2 = 0
\item γ 3 = \sigma 2 \beta 1 \beta 4
\item γ 4 = \sigma 2 \beta 4 (1 + \beta 1 2 )
\item γ 5 = γ 3
\item γ k = 0, k > 5
\end{itemize}
So the ACF is
\rho 1 =
\beta 1
1 + \beta 1 2
\rho 2 = 0
\rho 3 = \rho 5 =
\rho 4 =
\beta 1 \beta 4
(1 + \beta 1 2 )(1 + \beta 2 4 )
\beta 4
1 + \beta 24
\rho k = 0, k > 5
(iii)
Since in general the ratio
u
1 + u 2
\leq  0.5 then we see that for our model
\[\rho 1 < 0.5, \rho 3 < 0.25, \rho 4 < 0.5 and \rho 5 < 0.25.\] These do not seem to be
satisfied by the sample ACF. So the model is not appropriate for such data.
Other observations like those listed below can suffice here:
\item 
r(2) is not zero, and neither are r(6) and r(7).
Page 7%%%%%%%%%%%%%%%%%%%%%%% — April 2008 — Examiners’ Report
\item  r(3) is not close to r(5).
\item  r(1)r(4) = 0.43. This should be similar in value to both r(3) and r(5).
Whilst close to r(3) it isn't close to r(5).
Full marks for at least three correct statements.
\item Comment: There were some easy marks here. With the exception of part (iii), many
candidates did well but some dropped many points when the concept of auto-
correlation was not clear.
\end{itemize}
\end{itemize}
\end{document}
