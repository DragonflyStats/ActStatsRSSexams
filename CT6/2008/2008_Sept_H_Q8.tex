\documentclass[a4paper,12pt]{article}

%%%%%%%%%%%%%%%%%%%%%%%%%%%%%%%%%%%%%%%%%%%%%%%%%%%%%%%%%%%%%%%%%%%%%%%%%%%%%%%%%%%%%%%%%%%%%%%%%%%%%%%%%%%%%%%%%%%%%%%%%%%%%%%%%%%%%%%%

\usepackage{eurosym}
\usepackage{vmargin}
\usepackage{amsmath}
\usepackage{graphics}
\usepackage{epsfig}
\usepackage{enumerate}
\usepackage{multicol}
\usepackage{subfigure}
\usepackage{fancyhdr}
\usepackage{listings}
\usepackage{framed}
\usepackage{graphicx}
\usepackage{amsmath}
\usepackage{chngpage}

%\usepackage{bigints}
\usepackage{vmargin}
% left top textwidth textheight headheight

% headsep footheight footskip
\setmargins{2.0cm}{2.5cm}{16 cm}{22cm}{0.5cm}{0cm}{1cm}{1cm}
\renewcommand{\baselinestretch}{1.3}
\setcounter{MaxMatrixCols}{10}

\begin{document}


An insurer has issued two five-year term assurance policies to two individuals
involved in a dangerous sport. Premiums are payable annually in advance, and claims
are paid at the end of the year of death.
Individual
A
B
Annual
Premium
100
50
Sum Assured
1,700
400
Annual
Prob (death)
0.05
0.1
Assume that the probability of death is constant over each of the five years of the
policy. Suppose that the insurer has an initial surplus of U.
(i) Define what is meant by \psi ( U ) and \psi ( U , t ) .
(ii) Assuming U = 1,000

(a) Determine the distribution of S(1), the surplus at the end of the
first year, and hence calculate \psi ( U ,1) .
(b) Determine the possible values of S(2) and hence calculate \psi ( U , 2) .

%%%%%%%%%%%%%%%%%%%%%%%%%%%%%%%%%%%%%%%%%%%%%%%%%%%%%%%%%%%%%%%%%%%%%%%%%5

8
(i)
Let S ( t ) denote the insurer’s surplus at time t . Then
\psi ( U ) = Pr( S ( t ) < 0 for some value of t ) i.e. the probability of ruin at some
time
\psi ( U , t ) = Pr( S ( k ) < 0) for some k < t i.e. the probability that ruin occurs
before time t .
(ii)
(a)
Immediately before the payment of any claims, the insurer has cash
reserves of 1000 + 150 =1150.
The distribution of S (1) is given by:
Deaths
None
A only
B only
A and B
S(1)
1150
1150 – 1700 = - 550
1150 – 400 = 750
1150 – 2100 = - 950
Prob
0.95 \times  0.9 = 0.855
0.9 \times  0.05 = 0.045
0.95 \times  0.1 = 0.095
0.05 \times  0.1 = 0.005
And the probability of ruin is given by 0.045 + 0.005 = 0.05.
(b)
Assuming the surplus process ends if ruin occurs by time 1, then 2
possible values of S (2) are - 550 and - 950.
If there are no deaths in year 1, possible values of S (2) are
No deaths: 1150 + 150 = 1300
Page 7%%%%%%%%%%%%%%%%%%%%%%%%%%%%%%%%%%%%%%%%%%%%%%%%% — September 2008 — Examiners’ Report
A only: 1150 + 150 – 1700 = - 400
B only: 1150 + 150 – 400 = 900
A and B: 1150 + 150 – 1700 – 400 = - 800
If B dies in year 1, the possible values of S (2) are:
A lives: 750 + 100 = 850
A dies: 750 + 100 – 1700 = - 850
The probability of ruin within 2 years is given by:
0.05 + 0.855 \times  (0.05 \times  0.9 + 0.05 \times  0.1) + 0.095 \times  0.05 = 0.0975
Alternatively, note that ruin occurs within 2 years if and only if A dies
during this time, the probability of which is 0.05 + 0.95 \times  0.05 =
0.0975.
