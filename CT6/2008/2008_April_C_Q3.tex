\documentclass[a4paper,12pt]{article}

%%%%%%%%%%%%%%%%%%%%%%%%%%%%%%%%%%%%%%%%%%%%%%%%%%%%%%%%%%%%%%%%%%%%%%%%%%%%%%%%%%%%%%%%%%%%%%%%%%%%%%%%%%%%%%%%%%%%%%%%%%%%%%%%%%%%%%%%

\usepackage{eurosym}
\usepackage{vmargin}
\usepackage{amsmath}
\usepackage{graphics}
\usepackage{epsfig}
\usepackage{enumerate}
\usepackage{multicol}
\usepackage{subfigure}
\usepackage{fancyhdr}
\usepackage{listings}
\usepackage{framed}
\usepackage{graphicx}
\usepackage{amsmath}
\usepackage{chngpage}

%\usepackage{bigints}
\usepackage{vmargin}
% left top textwidth textheight headheight

% headsep footheight footskip
\setmargins{2.0cm}{2.5cm}{16 cm}{22cm}{0.5cm}{0cm}{1cm}{1cm}
\renewcommand{\baselinestretch}{1.3}
\setcounter{MaxMatrixCols}{10}

\begin{document}

3
(i) 

\begin{enumerate}
\item X and Y are independent Poisson random variables with mean $\lambda$. Derive the moment generating function of X, and hence show that X + Y also has a Poisson distribution.

\item 
(ii) An insurer has a portfolio of 100 policies. Annual premiums of 0.2 units per policy are payable annually in advance. Claims, which are paid at the end of the year, are for a fixed sum of 1 unit per claim. Annual claims numbers on each policy are Poisson distributed with mean 0.18.
\medskip 
Calculate how much initial capital is needed in order to ensure that the probability of ruin at the end of the year is less than 1%.
\end{enumerate}
%%%%%%%%%%%%%%%%%%%%%%%%%%%%%%%%%%%%%%%%%%%%%%%%%%%%%%%%%%%%%%%%%%%%%%%%
3
(i)
\begin{eqnarray*}
M_{X} ( t ) &=& E ( e tX )\\
&=& \sum^{\infty}e kt e −\lambda
k = 0
\\
&=& \sum^{\infty}e −\lambda
k = 0
= e −\lambda e \lambda e
\lambda k
k !
( \lambda e t ) k
k !
t \\
&=& e \lambda ( e − 1)
t \\
\end{eqnarray*}
Hence
\begin{eqnarray*}
M X + Y ( t ) &=& E ( e t ( X + Y ) )\\
&=& E ( e tX ) E ( e tY )\\
&=& M X ( t ) M Y ( t )\\
&=& e \lambda ( e − 1) e \lambda ( e − 1)\\
t
t
&=& e 2 \lambda ( e − 1)\\
t
\end{eqnarray*}
which is the MGF of a Poisson distribution with parameter $2\lambda$. Hence $X + Y$ is
Poisson distributed.
(ii)
Using the result above, aggregate claims on the portfolio over the year have a Poisson distribution with parameter 18.
Let the initial capital be U. At the end of the year, the surplus will be $U + 20 – N$ where N is the number of claims.
%%%%%%%%%%%%%%%%%%%%%%%%%%%%%%%%%%%%%%%%%%%%

Now using the tables in the gold book, $P(N \leq 28) = 0.9897$, and
$P(N < = 29) = 0.9941$.
So we need U to be large enough that ruin would only occur if there were 30
or more claims. So we need U + 20 – 29 > 0.
i.e. U > 9.
Comment: Many candidates struggled with (ii) here.



\end{document}
