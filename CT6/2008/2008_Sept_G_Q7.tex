\documentclass[a4paper,12pt]{article}

%%%%%%%%%%%%%%%%%%%%%%%%%%%%%%%%%%%%%%%%%%%%%%%%%%%%%%%%%%%%%%%%%%%%%%%%%%%%%%%%%%%%%%%%%%%%%%%%%%%%%%%%%%%%%%%%%%%%%%%%%%%%%%%%%%%%%%%%

\usepackage{eurosym}
\usepackage{vmargin}
\usepackage{amsmath}
\usepackage{graphics}
\usepackage{epsfig}
\usepackage{enumerate}
\usepackage{multicol}
\usepackage{subfigure}
\usepackage{fancyhdr}
\usepackage{listings}
\usepackage{framed}
\usepackage{graphicx}
\usepackage{amsmath}
\usepackage{chngpage}

%\usepackage{bigints}
\usepackage{vmargin}
% left top textwidth textheight headheight

% headsep footheight footskip
\setmargins{2.0cm}{2.5cm}{16 cm}{22cm}{0.5cm}{0cm}{1cm}{1cm}
\renewcommand{\baselinestretch}{1.3}
\setcounter{MaxMatrixCols}{10}

\begin{document}

PLEASE TURN OVER7
(i)
Let U 1 , U 2 , ..., U n be independent random numbers generated from a U(0,1)
distribution. Write down the Monte Carlo estimator, θ ˆ , for the integral
θ =
8
1
\int  0 ( e
x
− 1) dx .
[1]
(ii) Determine the variance of the estimator θ̂ in (i).
(iii) Calculate the smallest value of n for which the estimator θ̂ has absolute error
less than 0.1 with probability 90%.

[Total 9]

%%%%%%%%%%%%%%%%%%%%%%%%%%%%%%%%%%%%%%%%%%%%%%%%%%%%%%%%%%%%%%%%%%%%%%%%%%%%%%

7
⎛ n e U t
1 n U t
=
(
1)
e
−
⎜ ⎜ ∑
∑
n t = 1
⎝ i = 1 n
θ̂ =
(ii) We need to find the variance of h(U) = e U - 1 where U ~ U(0, 1)
E h(U) =
Page 6
⎞
⎟ ⎟ − 1
⎠
(i)
1
\int  0 ( e
x
− 1) dx = e – 2
E h(U) 2 = \int  0 ( e
1
x
= \int  0 ( e
1
2 x
= e 2 − 1
− 2( e − 1) + 1
2
− 1) 2 dx
− 2 e x + 1) dx%%%%%%%%%%%%%%%%%%%%%%%%%%%%%%%%%%%%%%%%%%%%%%%%% — September 2008 — Examiners’ Report
=
e 2
− 2 e + 2.5
2
varh(U) = E h(U) 2 – ( E h(U)) 2 = 0.242 and var θ̂ =
(iii)
0.242
.
n
From the theory, the required n should satisfy
n ≥
z \alpha  2
0.1 2
0.242
where P ( z < z \alpha  ) = 1 - \alpha  with z ~ N (0, 1). In our case \alpha  = 10% and z \alpha  = 1.64
and so
n ≥
1.64 2
0.1 2
0.242 = 65.09.
The answer is 66.
