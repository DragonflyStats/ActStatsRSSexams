\documentclass[a4paper,12pt]{article}

%%%%%%%%%%%%%%%%%%%%%%%%%%%%%%%%%%%%%%%%%%%%%%%%%%%%%%%%%%%%%%%%%%%%%%%%%%%%%%%%%%%%%%%%%%%%%%%%%%%%%%%%%%%%%%%%%%%%%%%%%%%%%%%%%%%%%%%%

\usepackage{eurosym}
\usepackage{vmargin}
\usepackage{amsmath}
\usepackage{graphics}
\usepackage{epsfig}
\usepackage{enumerate}
\usepackage{multicol}
\usepackage{subfigure}
\usepackage{fancyhdr}
\usepackage{listings}
\usepackage{framed}
\usepackage{graphicx}
\usepackage{amsmath}
\usepackage{chngpage}

%\usepackage{bigints}
\usepackage{vmargin}
% left top textwidth textheight headheight

% headsep footheight footskip
\setmargins{2.0cm}{2.5cm}{16 cm}{22cm}{0.5cm}{0cm}{1cm}{1cm}
\renewcommand{\baselinestretch}{1.3}
\setcounter{MaxMatrixCols}{10}

\begin{document}

5
The following table shows the claim payments for an insurance company in units of
£5,000:
Development year
Accident
year
2004
2005
2006
2007
0 1 2 3
410
575
814
1142 814
940
1066 216
281 79
The inflation for a 12 month period to the middle of each year is given as follows:
2005
5%
2006
5.5%
2007
5.4%
The future inflation from 2007 is estimated to be 8\% per annum.
Claims are fully run-off at the end of the development year 3.
Calculate the amount of outstanding claims arising from accidents in year 2007, using the inflation adjusted chain ladder method.
%%%%%%%%%%%%%%%%%%%%%%%%%%%%%%%%%
6
A portfolio of general insurance policies is made up of two types of policies. The policies are assumed to be independent, and claims are assumed to occur according to a Poisson process. The claim severities are assumed to have exponential distributions.
For the first type of policy, a total of 65 claims are expected each year and the expected size of each claim is £1,200.
For the second type of policy, a total of 20 claims are expected each year and the expected size of each claim is £4,500.
(i)
Calculate the mean and variance of the total cost of annual claims, S, arising from this portfolio.
[3]
The risk premium loading is denoted by θ, so that the annual premium on each policy is (1+θ)×expected annual claims on each policy. The initial reserve is denoted by u.
A normal approximation is used for the distribution of S, and the initial reserve is set by ensuring that
P(S < u + annual premium income) = 0.975.
(ii)
(a) Derive an equation for u in terms of θ.
(b) Determine the annual premium required in order that no initial reserve
is necessary.

%%%%%%%%%%%%%%%%%%%%%%%%%%%%%%%%%%%%%%%%%%%%%%%%%%%%%%%%%%%%%%%%%%%%%%%%%%%%%%%%%%%%%%%%%%%%
5
Multiply the claim payments with the corresponding inflation factors given below:
Development year
2004
2005
2006
2007
1.16757 1.11197 1.05400 1.00000
1.11197 1.05400 1.00000
1.05400 1.00000
1.00000
The resulting table is:
Development year
2004 478.70 905.14
2005 639.38 990.76
2006 857.96 1066.00
2007 1142.00
227.66
281.00
79.00
The inflation adjusted accumulated claim payments in mid 2007 are given below:
Development year
year
0
2004 478.70
2005 639.38
2006 857.96
2007 1142.00
1
1383.84
1630.14
1923.96
2853.75
2
1611.50
1911.14
2248.66
3335.38
3
1690.50
2004.83
2358.90
3498.88
Note only the values of the last row are needed for the answer.
The bolded values show the completed table using the basic chain ladder approach.
The development factors are 2.4989, 1.1688, 1.0490.
%%%%%%%%%%%%%%%%%%%%%%%%%%%%%%%%%%%%%%%%%%%%%%%%%%%%%%%%%%%%%%%%%%%%%%%%%%%%%%
For the answer we only need to work with the projected values at the last row as:
(2853.75 - 1142.00)*1.08+(3335.38 - 2853.75)*1.08 2 +(3498.88 - 3335.38)*1.08 3
= 2616.43
2616.43 * 5000 = £13,082,150
Comment: Many candidates scored full marks here. Some missed the conversion of
the final figure from units to pounds (i.e. multiplying by £5000).
6
(i)
E[S]
= 65 × 1200 + 20 × 4500 = 168000
Var[S] = 65 × 2 × 1200 2 + 20 × 2 × 4500 2
= 187,200,000 + 810,000,000
= 997,200,000
(ii)
(a)
c = annual premium income = (1 + θ) E[S]
⎛ u + c − E [ S ] ⎞
P(S < u + c)  Φ ⎜
⎜ Var[ S ] ⎟ ⎟

⎝
⎠
u + c − E [ S ]
= 1.96
Var[ S ]
u + θE[S] = 1.96 Var[ S ]
u = 1.96 Var[ S ] - θE[S]
= 61893.8 - 168000θ
(b)
61893.8 - 168000θ = 0
θ = 0.3684
No initial reserve is required if θ ≥ 0.3684. i.e. when the premium $ \geq \$229,894$.
Comment: There were some mixed answers for the second part of this question.
Numerical figures varied in (b) due to the rounding of the square root of the
Var(S).
\end{document}
