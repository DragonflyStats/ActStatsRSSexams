\documentclass[a4paper,12pt]{article}

%%%%%%%%%%%%%%%%%%%%%%%%%%%%%%%%%%%%%%%%%%%%%%%%%%%%%%%%%%%%%%%%%%%%%%%%%%%%%%%%%%%%%%%%%%%%%%%%%%%%%%%%%%%%%%%%%%%%%%%%%%%%%%%%%%%%%%%%

\usepackage{eurosym}
\usepackage{vmargin}
\usepackage{amsmath}
\usepackage{graphics}
\usepackage{epsfig}
\usepackage{enumerate}
\usepackage{multicol}
\usepackage{subfigure}
\usepackage{fancyhdr}
\usepackage{listings}
\usepackage{framed}
\usepackage{graphicx}
\usepackage{amsmath}
\usepackage{chngpage}

%\usepackage{bigints}
\usepackage{vmargin}
% left top textwidth textheight headheight

% headsep footheight footskip
\setmargins{2.0cm}{2.5cm}{16 cm}{22cm}{0.5cm}{0cm}{1cm}{1cm}
\renewcommand{\baselinestretch}{1.3}
\setcounter{MaxMatrixCols}{10}

\begin{document}

CT6 S2008—611
Losses on a portfolio of insurance policies in 2006 are assumed to have an exponential distribution with parameter $\lambda$. In 2007 loss amounts have increased by a
factor k (so that a loss incurred in 2007 is k times an equivalent loss incurred in 2006).
(i)
Show that the distribution of loss amounts in 2007 is also exponential and
determine the parameter of the distribution.
[3]
Over the calendar years 2006 and 2007 the insurer had in place an individual excess of-loss reinsurance arrangement with a retention of M. Claims paid by the insurer
were:
2006: 4 amounts of M and 10 claims under M for a total of 13,500.
2007: 6 amounts of M and 12 claims under M for a total of 17,000.
(ii)
Show that the maximum likelihood estimate of \lambda  is:
\hat{\lambda} =
22
17, 000
6 M
13,500 +
+ 4 M +
k
k

(iii)
The insurer is negotiating a new reinsurance arrangement for 2008. The retention was set at 1600 when the current arrangement was put in place in 2006. Loss inflation between 2006 and 2007 was 10% (i.e. k = 1.1) and
further loss inflation of 5% is expected between 2007 and 2008.
(a) Use this information to calculate \lambda ̂ .
(b) The insurer wishes to set the retention M ′ for 2008 such that the expected (net of re-insurance) payment per claim for 2008 is the same
as the expected payment per claim for 2006. Calculate the value of M ′ , using your estimate of \lambda  from (iii)(a).
[10]
[Total 20]
END OF PAPER
CT6 S2008—7

%%%%%%%%%%%%%%%%%%%%%%%%%%%%%%%%%%%%%%%%%%%%%%%
11
(i)
Suppose X is exponentially distributed with parameter \lambda . Then we must show
that Y = kX is also exponentially distributed.
Now
P ( Y < y ) = P ( kX < y )
= P ( X < y / k )
y / k
=
\int 
\lambda  e −\lambda  z dz
0
y
= \int 
0
y
\lambda 
− x
\lambda  e k
dx
making the substitution x = k z
k
\lambda 
\lambda  − x
= \int  e k dx
k
0
Which is the distribution n function of an exponential distribution with parameter \lambda  / k . So Y is exponentially distributed with parameter \lambda  / k .
(ii)
First note that the probability that a loss in 2006 is greater than M is given by e −\lambda  M and likewise the probability that a loss in 2007 is greater than M is
given by e −\lambda  M / k .

The likelihood of the data is given by:
\lambda 
L = C \times  e − 4 \lambda  M \times  ∏ ( \lambda  e −\lambda  x i ) \times  e − 6 \lambda  M / k \times  ∏ ( e −\lambda  y i / k )
k
i
j
Where the x i represent the claims in 2006 and y j represent the claims in
2007.
The log-likelihood is given by
l = log L = C ' − 4 \lambda  M + 10 log \lambda  − \lambda  ∑ x i − 6 \lambda  M / k + 12 log \lambda  − \lambda  / k ∑ y j
= C ' − 4 \lambda  M + 22 log \lambda  − 13,500 \lambda  − 6 \lambda  M / k − 17, 000 \lambda  / k
Differentiating gives
l ' = − 4 M +
22
− 13,500 − 6 M / k − 17, 000 / k
\lambda 
And equating this to zero gives
− 4 M + 22 / \hat{\lambda} − 13,500 − 6 M / k − 17, 000 / k = 0
Page 10%%%%%%%%%%%%%%%%%%%%%%%%%%%%%%%%%%%%%%%%%%%%%%%%% — September 2008 — Examiners’ Report
22 / \hat{\lambda} = 13,500 + 17, 000 / k + 4 M + 6 M / k
22
\hat{\lambda} =
13,500 + 17, 000 / k + 4 M + 6 M / k
We can check this is a maximum by noting that l '' = −
(iii)
22
\lambda  2
< 0
22
= 0.000499
13,500 + 17, 000 /1.1 + 4 \times  1, 600 + 6 \times  1, 600 /1.1
(a) \hat{\lambda} =
(b) Expected payment per claim for 2006 is given by:
M
\int  x \lambda  e
−\lambda  x
∞
dx + M
0
\int  \lambda  e
−\lambda  x
M
M
dx = ⎡ − xe −\lambda  x ⎤ +
⎣
⎦ 0
M
\int  e
−\lambda  x
0
∞
dx + M ⎡ − e −\lambda  x ⎤
⎣
⎦ M
M
⎡ 1
⎤
= − Me −\lambda  M + ⎢ − e −\lambda  x ⎥ + Me −\lambda  M
⎣ \lambda 
⎦ 0
1
= (1 − e −\lambda  M ) (*)
\lambda 
1
(1 − e − 0.000499 \times  1600 )
=
0.000499
= 1102.107
We know that claims for 2008 will have an exponential distribution . We need to choose the retention R so
with parameter $\mu = \lambda$ 
1.05 \times  1.1
that
R
1102.107 = \int  x \mu e
−\mu x
∞
dx + R \int  \mu e −\mu x dx
0
1
= (1 − e −\mu R )
\mu
R
using the result from (*)
−
1.05 \times  1.1
=
(1 − e
0.0004999
0.000499 R
1.05 \times  1.1 )
And so
0.476148392 = 1 − e − 0.00043203463 R
R =−
1
log(1 − 0.476148392)
0.00043203462
=1496.52
Page 11%%%%%%%%%%%%%%%%%%%%%%%%%%%%%%%%%%%%%%%%%%%%%%%%% — September 2008 — Examiners’ Report
END OF EXAMINERS’ REPORT
Page 12
