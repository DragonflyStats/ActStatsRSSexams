\documentclass[a4paper,12pt]{article}
%%%%%%%%%%%%%%%%%%%%%%%%%%%%%%%%%%%%%%%%%%%%%%%%%%%%%%%%%%%%%%%%%%%%%%%%%%%%%%%%%%%%%%%%%%%%%%%%%%%%%%%%%%%%%%%%%%%%%%%%%%%%%%%%%%%%%%%%

\usepackage{eurosym}
\usepackage{vmargin}
\usepackage{amsmath}
\usepackage{graphics}
\usepackage{epsfig}
\usepackage{enumerate}
\usepackage{multicol}
\usepackage{subfigure}
\usepackage{fancyhdr}
\usepackage{listings}
\usepackage{framed}
\usepackage{graphicx}
\usepackage{amsmath}
\usepackage{chngpage}

%\usepackage{bigints}

\usepackage{vmargin}
% left top textwidth textheight headheight

% headsep footheight footskip
\setmargins{2.0cm}{2.5cm}{16 cm}{22cm}{0.5cm}{0cm}{1cm}{1cm}
\renewcommand{\baselinestretch}{1.3}
\setcounter{MaxMatrixCols}{10}
\begin{document}

%%-- Question 9
For each discount level, find the minimum claim amount for which the policyholder will make a claim. 
Assuming that the cost of repair for each accident has an exponential distribution with mean \$600, calculate the probability that a policyholder makes a claim at each level of discount. 

%%%%%%%%%%%%%%%%%%%%%%%%%%%%%%%%%%%%%%%%%%%%%%%%%%%%%%%%%%%%%%%%%%%%%
\begin{enumerate}
    \item (i) Describe the difference between strictly stationary processes and weakly stationary processes.
\item
(ii) Explain why weakly stationary multivariate normal processes are also strictly stationary.
\item 
(iii) Show that the following bivariate time series process, (X n , Y n ) t , is weakly stationary:
\[X_n = 0.5X n-1 + 0.3Y n-1 + e n x\]
\[Y_n = 0.1X n-1 + 0.8Y n-1 + e n y\]
where e n x and e n x are two independent white noise processes.
\item 
(iv)

Determine the positive values of c for which the process
X n = (0.5 + c) X n-1 + 0.3Y n-1 + e n x\]
Y n = 0.1X n-1 + (0.8 + c) Y n-1 + e n y\]
is stationary.
\end{enumerate}

%%%%%%%%%%%%%%%%%%%%%%%%%%%%%%%%%%%%%%%%%%%%%%%%%%%%%%%%%%%%%%%%%%%%%%%%%%%%%%%%%%%%
\newpage
9
\begin{itemize}
\item (i)
Strictly stationary processes have the property that the distribution of $(X t+1 , ... X t+k )$ is the same as that of (X t+s+1 , ... X t+s+k ) for each $t$, $s$ and $k$. For the weakly stationary only the first two moments are needed to satisfy
\[E (X t ) = \mu ∀t\]
and
\[cov(X t , X t+s ) = γ(s)
∀t, s.\]
\item (ii) These two definitions coincide for the multivariate normal processes since the normal distribution is characterised by the first two moments only.

\item 
(iii) In order to confirm that we need to calculate the eigenvalues of the parameter
matrix



⎛ 0.5 0.3 ⎞
A = ⎜
⎟ .
⎝ 0.1 0.8 ⎠
So we need to solve $det(A - \lambdaI) = 0$ which implies the solution of
\[(0.5 - \lambda) (0.8 - \lambda) – 0.03 = 0\]
\[0.37 − 1.3\lambda + \lambda_{2} = 0\]
We see that this equation is satisfied for $\lambda_{1} = 0.8791288$ and $\lambda_{2} = 0.4208712$.
Since they are both smaller than 1, the process is stationary.
\item (iv)
The parameter matrix here is A c = A + cI , and the eigenvalues equation is now $det( A + cI - \lambda_{i} ) = 0$ or $det( A – (\lambda - c ) I ) = 0$.

So the eigenvalues of A c are $\lambda_{1} + c$ and $\lambda_{2} + c$ where $\lambda_{i}$ are those of A .
Since \lambda_{i} are positive then the required values for c are such that \lambda_{1} + c < 1 and $\lambda_{2} + c < 1$.
Hence $0 < c < 1 - \lambda_{1} = 0.1208712$, since \lambda_{1} is the largest of the two.

\item Comment: This was not the easiest question. Some struggled with (ii), (iii) and (iv).
There were quite a few candidates who managed to avoid the calculation of the eigen values of the matrix A by explicitly expressing each $X_n$ and $Y_n$ series as stationary univariate AR(2) processes with some white noise terms .
\end{itemize}
\end{document}
