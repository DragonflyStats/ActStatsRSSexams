\documentclass[a4paper,12pt]{article}

%%%%%%%%%%%%%%%%%%%%%%%%%%%%%%%%%%%%%%%%%%%%%%%%%%%%%%%%%%%%%%%%%%%%%%%%%%%%%%%%%%%%%%%%%%%%%%%%%%%%%%%%%%%%%%%%%%%%%%%%%%%%%%%%%%%%%%%%

\usepackage{eurosym}
\usepackage{vmargin}
\usepackage{amsmath}
\usepackage{graphics}
\usepackage{epsfig}
\usepackage{enumerate}
\usepackage{multicol}
\usepackage{subfigure}
\usepackage{fancyhdr}
\usepackage{listings}
\usepackage{framed}
\usepackage{graphicx}
\usepackage{amsmath}
\usepackage{chngpage}

%\usepackage{bigints}
\usepackage{vmargin}
% left top textwidth textheight headheight

% headsep footheight footskip
\setmargins{2.0cm}{2.5cm}{16 cm}{22cm}{0.5cm}{0cm}{1cm}{1cm}
\renewcommand{\baselinestretch}{1.3}
\setcounter{MaxMatrixCols}{10}

\begin{document}
7
Consider the following model applied to some quarterly data:
Y t = e t + \beta 1 e t-1 + \beta 4 e t-4 + \beta 1 \beta 4 e t-5
where e t is a white noise process with mean zero and variance \sigma 2 .
(i) Express in terms of \beta 1 and \beta 4 the roots of the characteristic polynomial of the
MA part, and give conditions for invertibility of the model.

(ii) Derive the autocorrelation function (ACF) for Y t .

For our particular data the sample ACF is:
(iii)



Lag ACF
1
2
3
4
5
6
7 0.73
0.14
0.37
0.59
0.24
0.12
0.07
Explain whether these results confirm the initial belief that the model could be
appropriate for these data.

%%%%%%%%%%%%%%%%%%%%%%%%%%%%%%%%%%%%%%%%%%%%%%%%%%%%%%%%%%%%%%%%%%%%%%%%%%
%%--- Question 8
The NCD scale policy for an insurance company is:
Level 0
Level 1
Level 2
0%
25%
50%
\begin{itemize}
\item The premium at the Level 0 is \$800. The probability that a policyholder has an accident in a year is 0.2, and it is assumed that a policyholder does not have more than one accident each year.
\item In the event of a claim free year the policyholder moves to the next higher level of discount in the coming year or remains at Level 2.
\item In the event of a claim the policyholder moves to the next lower level of discount in the coming year or remains at Level 0.
\item Following an accident the policyholder decides whether or not to make a claim based on the claim size and the amount of premiums over the period of the next 2 policy
years, assuming no more claims are made.
\end{itemize}

(i)
(ii)

%%%%%%%%%%%%%%%%%%%%%%%%%%%%%%%%%%%%%%%%%%%%%%%%%%%%%%%%%%%%%%%%%%%%%%%%%%
7
(i)
Using the back shift operator it can be seen that
Y t = (1 + \beta 1 B) (1 + \beta 4 B 4 ) e t .
The invertibility conditions are then \beta 1 < 1 and \beta 4 < 1.
(ii)
Since E (Y t ) = 0, γ 0 = E ( y t 2 ) = \sigma 2 (1 + \beta 1 2 + \beta 2 4 + \beta 1 2 \beta 2 4 ) = \sigma 2 (1 + \beta 1 2 )(1 + \beta 2 4 ).
Similarly it can be shown that
\begin{itemize}
\item γ 1 = E (Y t Y t - 1 ) = \sigma 2 \beta 1 (1 + \beta 24 )
\item γ 2 = 0
\item γ 3 = \sigma 2 \beta 1 \beta 4
\item γ 4 = \sigma 2 \beta 4 (1 + \beta 1 2 )
\item γ 5 = γ 3
\item γ k = 0, k > 5
\end{itemize}
So the ACF is
\rho 1 =
\beta 1
1 + \beta 1 2
\rho 2 = 0
\rho 3 = \rho 5 =
\rho 4 =
\beta 1 \beta 4
(1 + \beta 1 2 )(1 + \beta 2 4 )
\beta 4
1 + \beta 24
\rho k = 0, k > 5
(iii)
Since in general the ratio
u
1 + u 2
≤ 0.5 then we see that for our model
\rho 1 < 0.5, \rho 3 < 0.25, \rho 4 < 0.5 and \rho 5 < 0.25. These do not seem to be
satisfied by the sample ACF. So the model is not appropriate for such data.
Other observations like those listed below can suffice here:
\item 
r(2) is not zero, and neither are r(6) and r(7).
Page 7%%%%%%%%%%%%%%%%%%%%%%% — April 2008 — Examiners’ Report
\item  r(3) is not close to r(5).
\item  r(1)r(4) = 0.43. This should be similar in value to both r(3) and r(5).
Whilst close to r(3) it isn't close to r(5).
Full marks for at least three correct statements.
Comment: There were some easy marks here. With the exception of part (iii), many
candidates did well but some dropped many points when the concept of auto-
correlation was not clear.
8
(i)
Level
0
1
2
(ii)
Prem. If. claim
800 600
800 600
600 400
Prem. No claim
600 400
400 400
400 400
Difference
400
600
200
The claims are exponentially distributed with parameter \lambda = 1/ \mu = 1/ 600 and
so Pr(loss > u) = exp( - u \lambda).
Since
P(claim) = P(accident) P(claim|accident)
We derive that for a policyholder at Level 0
P(claim at 0\%) = 0.2P(X > 400) = 0.2 exp( - 400/600) = 0.102683
for Level 1
P(claim at 25\%) = 0.2P(X > 600) = 0.2 exp( - 600/600) = 0.07357589
and for Level 2
P(claim at 50\%) = 0.2P(X > 200) = 0.2 exp( - 200/600) = 0.1433063
(iii)
The transition matrix will be
⎛ 0.1027 0.8973 0.0000 ⎞
⎜
⎟
⎜ 0.0736 0.0000 0.9264 ⎟
⎜ 0.0000 0.1433 0.8567 ⎟
⎝
⎠
\pi P = \pi and hence
\pi 0 = 0.1027\pi 0 + 0.0736\pi 1
\pi 1 = 0.8973\pi 0 + 0.1433\pi 2
\pi 2 = 0.9264\pi 1 + 0.8567\pi 2
Page 8%%%%%%%%%%%%%%%%%%%%%%% — April 2008 — Examiners’ Report
The first and the last equations imply
\pi 0 = 0.0736 / 0.8973\pi 1 = 0.0820\pi 1
and
\pi 2 = 0.9264 / 0.1433\pi 1 = 6.4748\pi 1
From \pi 0 + \pi 1 + \pi 2 = 1 we obtain \pi 1 = 1/7.5568 = 0.1325
with \pi 0 = 0.0820×0.1325 = 0.0109 and \pi 2 = 1 - 0.1325 - 0.0109 = 0.8566.
The average premium is now:
800 \pi 0 + 600 \pi 1 + 400 \pi 2 = 800 × 0.0109 + 600 × 0.1325 + 400 × 0.8566 =
430.85.
Comment: Generally, very good answers with many candidates scoring full marks.
