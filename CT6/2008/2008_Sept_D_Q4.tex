\documentclass[a4paper,12pt]{article}

%%%%%%%%%%%%%%%%%%%%%%%%%%%%%%%%%%%%%%%%%%%%%%%%%%%%%%%%%%%%%%%%%%%%%%%%%%%%%%%%%%%%%%%%%%%%%%%%%%%%%%%%%%%%%%%%%%%%%%%%%%%%%%%%%%%%%%%%

\usepackage{eurosym}
\usepackage{vmargin}
\usepackage{amsmath}
\usepackage{graphics}
\usepackage{epsfig}
\usepackage{enumerate}
\usepackage{multicol}
\usepackage{subfigure}
\usepackage{fancyhdr}
\usepackage{listings}
\usepackage{framed}
\usepackage{graphicx}
\usepackage{amsmath}
\usepackage{chngpage}

%\usepackage{bigints}
\usepackage{vmargin}
% left top textwidth textheight headheight

% headsep footheight footskip
\setmargins{2.0cm}{2.5cm}{16 cm}{22cm}{0.5cm}{0cm}{1cm}{1cm}
\renewcommand{\baselinestretch}{1.3}
\setcounter{MaxMatrixCols}{10}

\begin{document}

CT6 S2008—2
[1]4 An insurance company provides warranties for a certain electrical gadget. At the start
of 2006 there were 4,500 gadgets under warranty, each of which has a probability $q$ of
suffering complete failure in 2006 (independently between gadgets). The prior
distribution of q is beta with mean 0.015 and standard deviation 0.005. Given that 58
gadgets suffer a complete failure in 2006, determine the posterior distribution of $q$.



Page 3%%%%%%%%%%%%%%%%%%%%%%%%%%%%%%%%%%%%%%%%%%%%%%%%% — September 2008 — Examiners’ Report
4
Suppose that q has a beta ( \alpha  , \beta  ) distribution as per the tables, and let X denote the
number of failures in 2006 so that X has a B(4500,q) distribution. Then
f ( q X ) ∝ f ( q ) \times  f ( X q )
∝ q \alpha − 1 (1 − q ) \beta − 1 \times  q x (1 − q ) n − x
∝ q \alpha + x − 1 (1 − q ) \beta + n − x − 1
So the posterior distribution of q is beta with parameters \alpha  + x and \beta  + n − x .
In our case, the parameters of the prior distribution are given by
\alpha 
\alpha \beta 
= 0.015 and
= 0.005 2 .
2
\alpha +\beta 
( \alpha  + \beta  ) ( \alpha  + \beta  + 1)
\alpha  = 0.015( \alpha  + \beta  )
0.985 \alpha  = 0.015 \beta 
3
\alpha =
\beta 
197
And substituting into the second equation gives
3 2
\beta 
197
= 0.005 2
2
⎛ 200 ⎞ ⎛ 200 ⎞
\beta  ⎟ ⎜ 1 +
\beta  ⎟
⎜
⎝ 197 ⎠ ⎝ 197 ⎠
2
3
⎛ 200 ⎞ ⎛ 200 ⎞
= 0.005 2 \times  ⎜
\beta  ⎟
⎟ \times  ⎜ 1 +
197
⎝ 197 ⎠ ⎝ 197 ⎠
200
591 = 1 +
\beta 
197
197 11623
=
= 581.15
200
20
3 11623
\alpha =
\times 
= 8.85
197
20
\beta  = 590 \times 
So the revised parameters are given by:
\alpha  * = 8.85 + 58 = 66.85
\beta  * = 581.15 + 4500 − 58 = 5023.15


\end{document}
