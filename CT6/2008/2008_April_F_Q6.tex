\documentclass[a4paper,12pt]{article}

%%%%%%%%%%%%%%%%%%%%%%%%%%%%%%%%%%%%%%%%%%%%%%%%%%%%%%%%%%%%%%%%%%%%%%%%%%%%%%%%%%%%%%%%%%%%%%%%%%%%%%%%%%%%%%%%%%%%%%%%%%%%%%%%%%%%%%%%

\usepackage{eurosym}
\usepackage{vmargin}
\usepackage{amsmath}
\usepackage{graphics}
\usepackage{epsfig}
\usepackage{enumerate}
\usepackage{multicol}
\usepackage{subfigure}
\usepackage{fancyhdr}
\usepackage{listings}
\usepackage{framed}
\usepackage{graphicx}
\usepackage{amsmath}
\usepackage{chngpage}

%\usepackage{bigints}
\usepackage{vmargin}
% left top textwidth textheight headheight

% headsep footheight footskip
\setmargins{2.0cm}{2.5cm}{16 cm}{22cm}{0.5cm}{0cm}{1cm}{1cm}
\renewcommand{\baselinestretch}{1.3}
\setcounter{MaxMatrixCols}{10}

\begin{document}

%%%%%%%%%%%%%%%%%%%%%%%%%%%%%%%%%
%%-- Question 6
A portfolio of general insurance policies is made up of two types of policies. The policies are assumed to be independent, and claims are assumed to occur according to a Poisson process. The claim severities are assumed to have exponential distributions.
For the first type of policy, a total of 65 claims are expected each year and the expected size of each claim is \$1,200.
For the second type of policy, a total of 20 claims are expected each year and the expected size of each claim is \$4,500.
\begin{enumerate}[(i)]
\item Calculate the mean and variance of the total cost of annual claims, S, arising from this portfolio.

The risk premium loading is denoted by $\theta$, so that the annual premium on each policy is $(1+\theta)$ × expected annual claims on each policy. The initial reserve is denoted by u.
A normal approximation is used for the distribution of S, and the initial reserve is set by ensuring that
\[P(S < u + \mbox{ annual premium income }) = 0.975.\]
\item
(a) Derive an equation for u in terms of $\theta$.
(b) Determine the annual premium required in order that no initial reserve
is necessary.
\end{enumerate}

\end{enumerate}

%%%%%%%%%%%%%%%%%%%%%%%%%%%%%%%%%%%%%%%%%%%%%%%%%%%%%%%%%%%%%%%%%%%%%%%%%%%%%%
6
(i)
E[S]
= 65 × 1200 + 20 × 4500 = 168000
Var[S] = 65 × 2 × 1200 2 + 20 × 2 × 4500 2
= 187,200,000 + 810,000,000
= 997,200,000
(ii)
(a)
c = annual premium income = (1 + \theta) E[S]
⎛ u + c − E [ S ] ⎞
P(S < u + c)  Φ ⎜
⎜ Var[ S ] ⎟ ⎟

⎝
⎠
u + c − E [ S ]
= 1.96
Var[ S ]
u + \thetaE[S] = 1.96 Var[ S ]
u = 1.96 Var[ S ] - \thetaE[S]
= 61893.8 - 168000\theta
(b)
61893.8 - 168000\theta = 0
\theta = 0.3684
No initial reserve is required if $\theta \leq 0.3684$. i.e. when the premium $ \geq \$229,894$.
Comment: There were some mixed answers for the second part of this question.
Numerical figures varied in (b) due to the rounding of the square root of the
Var(S).
\end{document}
