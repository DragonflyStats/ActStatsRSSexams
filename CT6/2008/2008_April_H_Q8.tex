\documentclass[a4paper,12pt]{article}

%%%%%%%%%%%%%%%%%%%%%%%%%%%%%%%%%%%%%%%%%%%%%%%%%%%%%%%%%%%%%%%%%%%%%%%%%%%%%%%%%%%%%%%%%%%%%%%%%%%%%%%%%%%%%%%%%%%%%%%%%%%%%%%%%%%%%%%%

\usepackage{eurosym}
\usepackage{vmargin}
\usepackage{amsmath}
\usepackage{graphics}
\usepackage{epsfig}
\usepackage{enumerate}
\usepackage{multicol}
\usepackage{subfigure}
\usepackage{fancyhdr}
\usepackage{listings}
\usepackage{framed}
\usepackage{graphicx}
\usepackage{amsmath}
\usepackage{chngpage}

%\usepackage{bigints}
\usepackage{vmargin}
% left top textwidth textheight headheight

% headsep footheight footskip
\setmargins{2.0cm}{2.5cm}{16 cm}{22cm}{0.5cm}{0cm}{1cm}{1cm}
\renewcommand{\baselinestretch}{1.3}
\setcounter{MaxMatrixCols}{10}

\begin{document}


%%%%%%%%%%%%%%%%%%%%%%%%%%%%%%%%%%%%%%%%%%%%%%%%%%%%%%%%%%%%%%%%%%%%%%%%%%
%%--- Question 8
The NCD scale policy for an insurance company is:
Level 0
Level 1
Level 2
0%
25%
50%
\begin{itemize}
\item The premium at the Level 0 is \$800. The probability that a policyholder has an accident in a year is 0.2, and it is assumed that a policyholder does not have more than one accident each year.
\item In the event of a claim free year the policyholder moves to the next higher level of discount in the coming year or remains at Level 2.
\item In the event of a claim the policyholder moves to the next lower level of discount in the coming year or remains at Level 0.
\item Following an accident the policyholder decides whether or not to make a claim based on the claim size and the amount of premiums over the period of the next 2 policy
years, assuming no more claims are made.
\end{itemize}

(i)
(ii)
(iii) Write down the transition matrix and calculate the average premium payment for a year when the system has reached the equilibrium.

%%%%%%%%%%%%%%%%%%%%%%%%%%%%%%%%%%%%%%%%%%%%%%%%%%%%%%%%%%%%%%%%%%%%%%%%%%
\newpage
8
(i)
Level
0
1
2
(ii)
Prem. If. claim
800 600
800 600
600 400
Prem. No claim
600 400
400 400
400 400
Difference
400
600
200
The claims are exponentially distributed with parameter $\lambda = 1/ \mu = 1/ 600$ and
so Pr(loss > u) = exp( - u \lambda).
Since
P(claim) = P(accident) P(claim|accident)
\begin{itemize}
\item We derive that for a policyholder at Level 0
P(claim at 0\%) = 0.2P(X > 400) = 0.2 exp( - 400/600) = 0.102683
\item for Level 1
P(claim at 25\%) = 0.2P(X > 600) = 0.2 exp( - 600/600) = 0.07357589
\item and for Level 2
P(claim at 50\%) = 0.2P(X > 200) = 0.2 exp( - 200/600) = 0.1433063
\end{itemize}
%%%%%%%%%%%%%%%%%%%%%%%
(iii)
The transition matrix will be
⎛ 0.1027 0.8973 0.0000 ⎞
⎜
⎟
⎜ 0.0736 0.0000 0.9264 ⎟
⎜ 0.0000 0.1433 0.8567 ⎟
⎝
⎠

$\pi P = \pi$ and hence

\begin{eqnarray*}
\pi_{0}  &=& 0.1027\pi_{0}  + 0.0736\pi_{1} \\
\pi_{1}  &=& 0.8973\pi_{0}  + 0.1433\pi_{2}\\ 
\pi_{2}  &=& 0.9264\pi_{1}  + 0.8567\pi_{2} \\
\end{eqnarray*}
Page 8%%%%%%%%%%%%%%%%%%%%%%% — April 2008 — Examiners’ Report
The first and the last equations imply
\[\pi_{0}  = \frac{0.0736}{0.8973} \pi_{1}  = 0.0820\pi_{1} \]
and
\[\pi_{2}  = \frac{0.9264}{0.1433} \pi_{1}  = 6.4748\pi_{1} \]
From $\pi_{0}  + \pi_{1}  + \pi_{2}  = 1$ we obtain $\pi_{1}  = 1/7.5568 = 0.1325$
with $\pi_{0}  = 0.0820\times 0.1325 = 0.0109$ and $\pi_{2}  = 1 - 0.1325 - 0.0109 = 0.8566$.
The average premium is now:
\[
800 \pi_{0}  + 600 \pi_{1}  + 400 \pi_{2}  = 800 \times  0.0109 + 600 \times  0.1325 + 400 \times  0.8566 =
430.85.\]
Comment: Generally, very good answers with many candidates scoring full marks.
\end{document}
