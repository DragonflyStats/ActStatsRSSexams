\documentclass[a4paper,12pt]{article}

%%%%%%%%%%%%%%%%%%%%%%%%%%%%%%%%%%%%%%%%%%%%%%%%%%%%%%%%%%%%%%%%%%%%%%%%%%%%%%%%%%%%%%%%%%%%%%%%%%%%%%%%%%%%%%%%%%%%%%%%%%%%%%%%%%%%%%%%

\usepackage{eurosym}
\usepackage{vmargin}
\usepackage{amsmath}
\usepackage{graphics}
\usepackage{epsfig}
\usepackage{enumerate}
\usepackage{multicol}
\usepackage{subfigure}
\usepackage{fancyhdr}
\usepackage{listings}
\usepackage{framed}
\usepackage{graphicx}
\usepackage{amsmath}
\usepackage{chngpage}

%\usepackage{bigints}
\usepackage{vmargin}
% left top textwidth textheight headheight

% headsep footheight footskip
\setmargins{2.0cm}{2.5cm}{16 cm}{22cm}{0.5cm}{0cm}{1cm}{1cm}
\renewcommand{\baselinestretch}{1.3}
\setcounter{MaxMatrixCols}{10}

\begin{document}
[Total 10]
CT6 S2008—49
A motor insurance company applies a NCD scale policy of discount levels
Level 0
Level 1
Level 2
0%
20%
50%
Following a claim-free year, the policyholder moves to the next higher level (or
remains at Level 2). If one or more claims are made, the policyholder moves to the
next lower level (or remains at Level 0).
The probability of a policyholder not making a claim in a policy year is 1 - p.
The annual premium at Level 0 is £600.
(i)
(ii)
Derive, in terms of p, the expected proportions of policyholders at each
discount level assuming that the system is in equilibrium .

Calculate the average premium paid in the stable state for the particular values
p = 0.1 and p = 0.3 and comment on the results.
[3]
[Total 10]
CT6 S2008—5

9
(i)
The transition matrix is
0 ⎞
⎛ p 1 − p
⎜
⎟
0
1 − p ⎟
⎜ p
⎜ 0
p 1 − p ⎟ ⎠
⎝
The equilibrium probabilities ( \pi  0 , \pi  1 , \pi  2 ) satisfy
\pi  0 = p \pi  0 + p \pi  1
\pi  1 = \pi  0 (1 - p ) + p \pi  2
\pi  2 = \pi  1 (1 - p ) + (1 - p ) \pi  2
From equations 1 and 3 above we obtain
\pi  0 = p /(1 - p ) \pi  1 and \pi  2 = (1 - p )/ p \pi  1 .
Since \pi  0 + \pi  1 + \pi  2 =1 we get \pi  1 =
and \pi  2 =
(ii)
p (1 − p )
p + (1 − p ) 2
together with \pi  0 =
p 2
p + (1 − p ) 2
(1 − p ) 2
p + (1 − p ) 2
The average premium is now
£600(
p 2
p + (1 − p ) 2
+0.8
p (1 − p )
p + (1 − p ) 2
+0.5
(1 − p ) 2
p + (1 − p ) 2
) and for
p = 0.1 and p = 0.3 this quantity takes values £321.1 and £382 respectively.
Page 8%%%%%%%%%%%%%%%%%%%%%%%%%%%%%%%%%%%%%%%%%%%%%%%%% — September 2008 — Examiners’ Report
The difference in average premiums is small given that the claim probability is
three times higher, suggesting the NCD system does not discriminate well.
