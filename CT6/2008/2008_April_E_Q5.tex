\documentclass[a4paper,12pt]{article}

%%%%%%%%%%%%%%%%%%%%%%%%%%%%%%%%%%%%%%%%%%%%%%%%%%%%%%%%%%%%%%%%%%%%%%%%%%%%%%%%%%%%%%%%%%%%%%%%%%%%%%%%%%%%%%%%%%%%%%%%%%%%%%%%%%%%%%%%

\usepackage{eurosym}
\usepackage{vmargin}
\usepackage{amsmath}
\usepackage{graphics}
\usepackage{epsfig}
\usepackage{enumerate}
\usepackage{multicol}
\usepackage{subfigure}
\usepackage{fancyhdr}
\usepackage{listings}
\usepackage{framed}
\usepackage{graphicx}
\usepackage{amsmath}
\usepackage{chngpage}

%\usepackage{bigints}
\usepackage{vmargin}
% left top textwidth textheight headheight

% headsep footheight footskip
\setmargins{2.0cm}{2.5cm}{16 cm}{22cm}{0.5cm}{0cm}{1cm}{1cm}
\renewcommand{\baselinestretch}{1.3}
\setcounter{MaxMatrixCols}{10}

\begin{document}

5
The following table shows the claim payments for an insurance company in units of
\$5,000:
Development year
Accident
year
2004
2005
2006
2007
0 1 2 3
410
575
814
1142 814
940
1066 216
281 79
The inflation for a 12 month period to the middle of each year is given as follows:
2005
5%
2006
5.5%
2007
5.4%
The future inflation from 2007 is estimated to be 8\% per annum.
Claims are fully run-off at the end of the development year 3.
Calculate the amount of outstanding claims arising from accidents in year 2007, using the inflation adjusted chain ladder method.
%%%%%%%%%%%%%%%%%%%%%%%%%%%%%%%%%

\end{enumerate}
%%%%%%%%%%%%%%%%%%%%%%%%%%%%%%%%%%%%%%%%%%%%%%%%%%%%%%%%%%%%%%%%%%%%%%%%%%%%%%%%%%%%%%%%%%%%
5
Multiply the claim payments with the corresponding inflation factors given below:
Development year
2004
2005
2006
2007
1.16757 1.11197 1.05400 1.00000
1.11197 1.05400 1.00000
1.05400 1.00000
1.00000
The resulting table is:
Development year
2004 478.70 905.14
2005 639.38 990.76
2006 857.96 1066.00
2007 1142.00
227.66
281.00
79.00
The inflation adjusted accumulated claim payments in mid 2007 are given below:
	Development year
	\begin{center}
\begin{tabular}{c|c|c|c|c}
year	&	0	&	1	&	2	&	3	\\ \hline
2004	&	478.7	&	1383.84	&	1611.5	&	1690.5	\\ \hline
2005	&	639.38	&	1630.14	&	1911.14	&	2004.83	\\ \hline
2006	&	857.96	&	1923.96	&	2248.66	&	2358.9	\\ \hline
2007	&	1142	&	2853.75	&	3335.38	&	3498.88	\\ \hline
\end{tabular}
\end{center}
Note only the values of the last row are needed for the answer.
The bolded values show the completed table using the basic chain ladder approach.
The development factors are 2.4989, 1.1688, 1.0490.
%%%%%%%%%%%%%%%%%%%%%%%%%%%%%%%%%%%%%%%%%%%%%%%%%%%%%%%%%%%%%%%%%%%%%%%%%%%%%%
For the answer we only need to work with the projected values at the last row as:
(2853.75 - 1142.00)*1.08+(3335.38 - 2853.75)*1.08 2 +(3498.88 - 3335.38)*1.08 3
= 2616.43
2616.43 * 5000 = \$13,082,150
%Comment: Many candidates scored full marks here. Some missed the conversion of the final figure from units to pounds (i.e. multiplying by \$5000).
%%%%%%%%%%%%%%%%%%%%%%%%%%%%%%%%%%%%%%%%%%%%%%%%%%%%%%%%%%%%%%%%%%%%%%%%%%%%%%

\end{document}
