
9
(i)
The general form of a run-off triangle can be expressed as:
Accident
Year
0
1
2
3
4
5
0 1
C 0,0
C 1,0
C 2,0
C 3,0
C 4,0
C 5,0 C 0,1
C 1,1
C 2,1
C 3,1
C 4,1
Development Year
2
3
C 0,2
C 1,2
C 2,2
C 3,2
C 0,3
C 1,3
C 2,3
4 5
C 0,4
C 1,4 C 0,5
Define a model for each entry, C ij , in general terms and explain each element
of the formula.
[3]
(ii)
The run-off triangles given below relate to a portfolio of motorcycle insurance
policies.
The cost of claims paid during each year is given in the table below:
(Figures in £000s)
Accident
Year
0
2002
2003
2004
2005
2,905
3,315
3,814
4,723
Development Year
1
2
535
578
693
199
159
3
56
The corresponding number of settled claims is as follows:
Accident
Year
0
2002
2003
2004
2005
430
465
501
539
Development Year
1
2
51
58
59
24
24
3
7
Calculate the outstanding claims reserve for this portfolio using the average
cost per claim method with grossing-up factors, and state the assumptions
underlying your result.
[9]
(iii)
Compare the results from the analysis in (ii) with those obtained from the
basic chain ladder method.
[5]
[Total 17]

9
(i)
Each entry can be expressed as:
C ij = r j . s i . x i+j + e ij
where:
r j is the development factor for year j, representing the proportion of claim
payments by year j. Each r j is independent of the accident year i
s i is a parameter varying by origin year, i, representing the exposure, for
example the number of claims incurred in the accident year i
Page 9Subject CT6 (Statistical Methods Core Technical)
April 2006
Examiners Report
x i+j is a parameter varying by calendar year, for example representing
inflation
e ij
(ii)
is an error term
The cumulative cost of claims paid is:
Accident
Year
0
2002
2003
2004
2005
Development Year
1
2
2,905
3,315
3,814
4,723
3,440
3,893
4,507
3,639
4,052
3
3,695
The number of accumulated settled claims is as follows:
(Figures in £000s)
Accident
Year
0
2002
2003
2004
2005
430 (84.0%)
465 (83.8%)
501 (84.2%)
539 (84.0%)
Development Year
1
2
481 (93.9%)
523 (94.3%)
560 (94.1%)
505 (98.6%)
547 (98.6%)
3
512 (100%)
Ult
512
554.8
595.1
641.7
Average cost per settled claim:
Accident
Year
0
2002
2003
2004
2005
Page 10
Development Year
1
2
6.756 (93.6%) 7.152 (99.1%)
7.129 (96.0%) 7.444 (100.3%)
7.613 (94.3%) 8.048 (99.7%)
8.763 (94.7%)
7.206 (99.8%)
7.408 (99.8%)
3
7.217 (100%)
Ult
7.217
7.419
8.072
9.256Subject CT6 (Statistical Methods Core Technical)
April 2006
Examiners Report
The total ultimate loss is therefore:
Accident
Year
Claim
Numbers Projected
Loss
7.217
7.419
8.071
9.256 512
554.8
595.1
641.7 3,695
4,114
4,802
5,938
18,550
Claims paid to date
Outstanding claims 16,977
1,573
ACPC
1
2
3
4
Assumptions:
Claims fully run-off by end of development year 3.
Projections based on simple average of grossing up factors.
Number of claims relating to each development year are a constant proportion
of total claim numbers from the origin year.
Similarly for average claim amounts, i.e. same proportion of total average
claim amount for origin year.
(iii)
Development table:
AF
0
2002
2003
2004
2005
2,905
3,315
3,814
4,723
1
3,440
3,893
4,507
2
3,639
4,052
3
3,695
3,695
4,114
4,800
5,935
DF
10,034
11,840
7,691
3,695 Column sum
7,333
3,639
Column sum minus last entry
1.17999 1.04882 1.01539
Ultimate loss
Claims paid to date
Outstanding claims
18,544
16,977
1,567
Comments on question 9: It was encouraging to see so many candidates achieve full marks
for the bookwork in (i). Parts (ii) and (iii) were generally well done. A small number of
candidates obtained very different answers for (ii) and (iii) but failed to appreciate that this
was due to an arithmetical error rather than the method used.
Page 11Subject CT6 (Statistical Methods Core Technical)
