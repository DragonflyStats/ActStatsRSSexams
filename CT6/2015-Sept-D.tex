7 A shipping insurance company has insured ships for six years, and classifies the
ships it insures into three types.
Let:
  Pij be the number of ships insured in the jth year from type i,
Yij be the corresponding number of claims.
The six years of data are summarised as follows:
  Type (i)
6
1
i ij
j
P P

  Xi 
Yij
j1 Pi
6
Pij
j1
6
Yij
Pij
 Xi






2
Pij
j1
6
Yij
Pij
 X






2
Type 1 648 0.524 691 30.966 692 64.392683
Type 2 981 0.145062 4.689 264 42.240804
Type 3 636 0.370 370 62.449 512 66.467 182
X =
  3 6
1 1
ij
i j
Y
  P
  = 0.297572, where P =
  3
1
i
i
P
 
There are 100 ships of Type 3 to be insured in year seven.
(i) Estimate the number of claims from Type 3 ships in year seven using
empirical Bayes credibility theory (EBCT) Model 2. [6]
The insurance company’s actuary is considering using EBCT Model 1 instead.
(ii) Explain an advantage and a disadvantage of using EBCT Model 1 rather
than EBCT Model 2. [2]
[Total 8]
CT6 S2015–5 PLEASE TURN OVER
8 The run-off triangle below shows cumulative claims incurred on a portfolio of
general insurance policies
Development Year
Policy Year 0 1 2 3
2011 1,528 2,034 2,212 2,310
2012 1,812 2,251 2,951
2013 1,693 1,851
2014 2,125
Annual premiums written in 2014 were 4,023 and the ultimate loss ratio has been
estimated as 91%. Claims paid to date for policy year 2014 are 572.
Estimate the outstanding claims to be paid arising from policies written in 2014
only, using the Bornheutter-Ferguson technique, stating any assumptions that you
make. [9]


%%%%%%%%%%%%%%%%%%%%%%%%%%%%%%%%%%%%%%%%%%%%%%%%%%%%%%%%%%%%%%%
Q7 (i) Z3  j1
6  P3, j
j1
6 P3, j  Es2 / Var m
i
i
P P = 2265
n = 6, N = 3
  *  
1
1 1 /
  1
N
i i
i
P P P P
Nn 
 
 
= 1/17 * (648 * (1  648/2265) + 981 * (1  981/2265) + 636 *
            (1  636/2265))
= 86.831 944
 2    2
, ,
1 1
1 1
1
N n
i j i j i
i j
E s P X X
N n  
         
 
= 1/15{30.966 692 + 4.689 264 + 62.449 512) = 6.540 364 5
    2  2  
* , ,
1 1
Var 1 1
1
N n
i j i j
i j
m PX X E s
P Nn  
           

= 1/86.831 944 * (1/17{64.392683 + 42.240804+66.467182}  6.540 365)
= 0.041 943
so Z3 = 636/(636 + 6.5403645/0.04194341) = 0.803 098
X3,7 = 0.803 098 * 0.370 370 + (1  0.803098) * 0.297 572 = 0.356 036
So Y3,7 = 100 * X3,7 = 35.60
Subject CT6 (Statistical Methods Core Technical) – September 2015 – Examiners’ Report
Page 8
(ii) Disadvantages of Model 1
Does not make use of the risk volumes
Requires more assumptions about the data
Advantages of Model 1
Requires less information (does not take account of risk volumes)
EBCT Model 1 is likely to be computationally more straightforward
This was one of the best answered questions on the paper, with the majority
of candidates able to score most or all of the marks in part (i). Only the
stronger candidates were able to pick up both marks in part (ii), although
again most candidates picked up some marks here.
Q8 Assume claims fully run off by the end of development year 3.
Each year develops in the same way
The weighted average past inflation is repeated
The loss ratio is appropriate
DF 2–3 = 2,310/2,212 = 1.044 304
DF 1–2 = (2,951 + 2,212) / (2,251 + 2,034) = 1.204 901
DF 0–1 = (2,034 + 2,251 + 1,851) / (1,528 + 1,812 + 1,693) = 1.219 154
Ultimate loss for 2014 = 0.91 * 4,023 = 3,660.93
Emerging liability = 3660.93 * (1 – 1/(1.044304 * 1.204901 * 1.219 154))
= 1,274.466
So total amount for policies written in 2014 is
2,125 + 1,274.466 = 3,399.466
so remaining
3,399.466 – 572 = 2,827.5
This straightforward chain ladder question was the best answered question on
the paper, although a disappointing number of candidates slipped up towards
the end. Weaker candidates also failed to state the assumptions required.
Subject CT6 (Statistical Methods Core Technical) – September 2015 – Examiners’ Report
Page 9
