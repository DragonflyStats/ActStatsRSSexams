\documentclass[a4paper,12pt]{article}

%%%%%%%%%%%%%%%%%%%%%%%%%%%%%%%%%%%%%%%%%%%%%%%%%%%%%%%%%%%%%%%%%%%%%%%%%%%%%%%%%%%%%%%%%%%%%%%%%%%%%%%%%%%%%%%%%%%%%%%%%%%%%%%%%%%%%%%%%%%%%%%%%%%%%%%%%%%%%%%%%%%%%%%%%%%%%%%%%%%%%%%%%%%%%%%%%%%%%%%%%%%%%%%%%%%%%%%%%%%%%%%%%%%%%%%%%%%%%%%%%%%%%%%%%%%%

\usepackage{eurosym}
\usepackage{vmargin}
\usepackage{amsmath}
\usepackage{graphics}
\usepackage{epsfig}
\usepackage{enumerate}
\usepackage{multicol}
\usepackage{subfigure}
\usepackage{fancyhdr}
\usepackage{listings}
\usepackage{framed}
\usepackage{graphicx}
\usepackage{amsmath}
\usepackage{chngpage}

%\usepackage{bigints}
\usepackage{vmargin}

% left top textwidth textheight headheight

% headsep footheight footskip

\setmargins{2.0cm}{2.5cm}{16 cm}{22cm}{0.5cm}{0cm}{1cm}{1cm}

\renewcommand{\baselinestretch}{1.3}

\setcounter{MaxMatrixCols}{10}

\begin{document}

 Institute and Faculty of Actuaries1
The potential losses from a decision problem are given in the table below
(i)
\theta  1 D1
5 D2
8 D3
12 D4
3
\theta  2 10 15 7 8
\theta  3 7 12 16 9
\theta  4 17 4 10 12
Find the optimal decision using the minimax criteria.

Now suppose that p ( \theta  1 ) = p ( \theta  2 ) = p ( \theta  3 ) = 0.3 and p ( \theta  4 ) = 0.1.
(ii)
2
Find the optimal decision using the Bayes criteria.

[Total 4]
Page 2%%%%%%%%%%%%%%%%%%%%%%%%%%%%%%%%%%%%%%%%%%%%%%%%%%%%%5 – Examiners’ Report
1
(i)
The maximum losses are:
D1
D2
D3
D4
17
15
16
12
So the minimax decision is to choose D4.
(ii)
The expected losses of the decisions are:
D1
D2
D3
D4
8.3
10.9
11.5
7.2
So the Bayes’ decision is also D4.
This fairly standard question was well answered.
