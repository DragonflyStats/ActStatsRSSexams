\documentclass[a4paper,12pt]{article}

%%%%%%%%%%%%%%%%%%%%%%%%%%%%%%%%%%%%%%%%%%%%%%%%%%%%%%%%%%%%%%%%%%%%%%%%%%%%%%%%%%%%%%%%%%%%%%%%%%%%%%%%%%%%%%%%%%%%%%%%%%%%%%%%%%%%%%%%%%%%%%%%%%%%%%%%%%%%%%%%%%%%%%%%%%%%%%%%%%%%%%%%%%%%%%%%%%%%%%%%%%%%%%%%%%%%%%%%%%%%%%%%%%%%%%%%%%%%%%%%%%%%%%%%%%%%

\usepackage{eurosym}
\usepackage{vmargin}
\usepackage{amsmath}
\usepackage{graphics}
\usepackage{epsfig}
\usepackage{enumerate}
\usepackage{multicol}
\usepackage{subfigure}
\usepackage{fancyhdr}
\usepackage{listings}
\usepackage{framed}
\usepackage{graphicx}
\usepackage{amsmath}
\usepackage{chngpage}

%\usepackage{bigints}
\usepackage{vmargin}

% left top textwidth textheight headheight

% headsep footheight footskip

\setmargins{2.0cm}{2.5cm}{16 cm}{22cm}{0.5cm}{0cm}{1cm}{1cm}

\renewcommand{\baselinestretch}{1.3}

\setcounter{MaxMatrixCols}{10}

\begin{document}

Claim amounts on a certain type of insurance policy depend on a parameter \alpha  which
varies from policy to policy. The mean and variance of the claim amount X given \alpha 
are specified by
E ⎡ ⎣ X \alpha  ⎤ ⎦ = 200 + \alpha 
V [ X \alpha  ] = 10 + 2 \alpha 
The parameter \alpha  follows a normal distribution with mean 20 and variance 4.
Find the unconditional mean and variance of X.

%%%%%%%%%%%%%%%%%%%%%%%%%%%%%%%%5
2
Firstly
E [ X ] = E ⎡ ⎣ E ⎡ ⎣ X \alpha  ⎤ ⎦ ⎤ ⎦ = E [ 200 + \alpha  ] = 200 + E [ \alpha  ] = 220
And secondly
Var ( X ) = Var ⎡ ⎣ E ⎡ ⎣ X \alpha  ⎤ ⎦ ⎤ ⎦ + E [ Var [ X \alpha  ]]
Now
Var [ E ⎡ ⎣ X \alpha  ]] = Var [ 200 + \alpha  ⎤ ⎦ = Var [ \alpha  ] = 4
And
E [ Var ⎣ ⎡ X \alpha  ]] = E [10 + 2 \alpha  ⎦ ⎤ = 10 + 2 \times  20 = 50
Hence
Var [ X ] = 4 + 50 = 54
Again, a fairly routine question that was well answered by most candidates.
Page 3Subject CT6 (Statistical Method) – %%%%%%%%%%%%%%%%%%%%%%%%%%%%%%%%%%5
