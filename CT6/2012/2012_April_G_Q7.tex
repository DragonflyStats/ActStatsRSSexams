\documentclass[a4paper,12pt]{article}

%%%%%%%%%%%%%%%%%%%%%%%%%%%%%%%%%%%%%%%%%%%%%%%%%%%%%%%%%%%%%%%%%%%%%%%%%%%%%%%%%%%%%%%%%%%%%%%%%%%%%%%%%%%%%%%%%%%%%%%%%%%%%%%%%%%%%%%%%%%%%%%%%%%%%%%%%%%%%%%%%%%%%%%%%%%%%%%%%%%%%%%%%%%%%%%%%%%%%%%%%%%%%%%%%%%%%%%%%%%%%%%%%%%%%%%%%%%%%%%%%%%%%%%%%%%%

\usepackage{eurosym}
\usepackage{vmargin}
\usepackage{amsmath}
\usepackage{graphics}
\usepackage{epsfig}
\usepackage{enumerate}
\usepackage{multicol}
\usepackage{subfigure}
\usepackage{fancyhdr}
\usepackage{listings}
\usepackage{framed}
\usepackage{graphicx}
\usepackage{amsmath}
\usepackage{chng%%-- Page}

%\usepackage{bigints}
\usepackage{vmargin}

% left top textwidth textheight headheight

% headsep footheight footskip

\setmargins{2.0cm}{2.5cm}{16 cm}{22cm}{0.5cm}{0cm}{1cm}{1cm}

\renewcommand{\baselinestretch}{1.3}

\setcounter{MaxMatrixCols}{10}

\begin{document}

\begin{enumerate}
%%%%%%%%%%%%%%%%%%%%%%%%%
7
The numbers of claims on three different classes of insurance policies over the last
four years are given in the table below.
Class 1
Class 2
Class 3
Year 1
1
1
5
Year 2
4
6
6
Year 3
5
4
4
Year 4
0
6
9
Total
10
17
24
The number of claims in a given year from a particular class is assumed to follow a
Poisson distribution.
8

\begin{enumerate}[(i)]
\item Determine the maximum likelihood estimate of the Poisson parameter for each class of policy based on the data above.
\item Perform a test on the scaled deviance to check whether there is evidence that the classes of policy have different mean claim rates and state your
conclusion.
\end{enumerate}


%%%%%%%%%%%%%%%%%%%%%%%%%%%%%%%%%%%%%%%%%%%%%%%%%%%%%%%%%%%%%%%%%%%%%%%%%%5
7
(i)
Suppose that the Poisson rate for risk i is \lambda i for =1,2,3.
For the first risk, the likelihood is given by:
L = e − 4 \lambda 1
(4 \lambda 1 ) 10
10!
And so the log-likelihood is given by
l = log L = − 4 \lambda 1 + 10 log 4 \lambda 1 + Constants
Differentiating gives
dl
10
= − 4 +
d \lambda 1
\lambda 1
And setting this equal to zero gives a maximum likelihood estimate of
10
\hat{\lambda} 1 =
= 2. 5
4
Since
d 2 l
d \lambda
2
=−
10
\lambda i 2
< 0 we do have a maximum.
17
2 4
In the same way \hat{\lambda} 2 =
= 6 .
= 4.25 and \hat{\lambda} 3 =
4
4
%%-- Page 9
%%%%%%%%%%%%%%%%%%%%%%%%%

(ii)
Under the assumption that these risks share the same rate i.e. \lambda 1 = \lambda 2 = \lambda 3 = \lambda
then the mle for this is simply
51
= 4.25
\hat{\lambda} =
12

\begin{itemize}
\item In order to compare these models we can use the scaled deviances to compare
these models and use the chi-squared test.
\item The difference in the scaled deviance here should have a chi-square
distribution with 3−1=2 degrees of freedom.
\end{itemize}
%%%%%%%%%%%%
2 ( log L 1 + log L 2 + log L 3 − log L ) = 10 log \hat{\lambda} 1 − 4 \hat{\lambda} 1 + 17 log \hat{\lambda} 2 − 4 \hat{\lambda} 2 + 24log \hat{\lambda} 3 − 4 \hat{\lambda} 3 − 51log \hat{\lambda} + 12 \hat{\lambda}
With the
4 4 4
i = 1 i = 1 i = 1
\sum  log y 1 i ! + \sum  log y 2 i ! + \sum  log y 3 i ! cancelling out in the difference.
Hence
2 ( log L 1 + log L 2 + log L 3 − log L )
⎛
51 4 ( 10 + 17 + 24 )
51 ⎞
= 2 ⎜ 10 log 2.5 + 17 log 4.25 + 24 log 6 − 51log −
+ 12 ⎟
12
4
12 ⎠
⎝
51 ⎞
⎛
= 2 ⎜ 10 log 2.5 + 17 log 4.25 + 24 log 6 − 51log ⎟ = 5.939778
12 ⎠
⎝
This value is below 5.991 which is the critical value at the upper 5% level and
therefore there is not a significant improvement by considering different rates
for each risk.
Part (i) was answered very well. Most candidates struggled with part (ii).
%%-- Page 10
%%%%%%%%%%%%%%%%%%%%%%%%%
%% (Statistical Methods) – April 2012 – Examiners’ Report

\end{document}
