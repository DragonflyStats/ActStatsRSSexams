\documentclass[a4paper,12pt]{article}

%%%%%%%%%%%%%%%%%%%%%%%%%%%%%%%%%%%%%%%%%%%%%%%%%%%%%%%%%%%%%%%%%%%%%%%%%%%%%%%%%%%%%%%%%%%%%%%%%%%%%%%%%%%%%%%%%%%%%%%%%%%%%%%%%%%%%%%%%%%%%%%%%%%%%%%%%%%%%%%%%%%%%%%%%%%%%%%%%%%%%%%%%%%%%%%%%%%%%%%%%%%%%%%%%%%%%%%%%%%%%%%%%%%%%%%%%%%%%%%%%%%%%%%%%%%%

\usepackage{eurosym}
\usepackage{vmargin}
\usepackage{amsmath}
\usepackage{graphics}
\usepackage{epsfig}
\usepackage{enumerate}
\usepackage{multicol}
\usepackage{subfigure}
\usepackage{fancyhdr}
\usepackage{listings}
\usepackage{framed}
\usepackage{graphicx}
\usepackage{amsmath}
\usepackage{chng%%-- Page}

%\usepackage{bigints}
\usepackage{vmargin}

% left top textwidth textheight headheight

% headsep footheight footskip

\setmargins{2.0cm}{2.5cm}{16 cm}{22cm}{0.5cm}{0cm}{1cm}{1cm}

\renewcommand{\baselinestretch}{1.3}

\setcounter{MaxMatrixCols}{10}

\begin{document}

\begin{enumerate}

%%%%%%%%%%%%%%%%%%%%%%%%%%%%%%%%%%%%%%%%%%%%%%%%%%%%%%%%%%%%%%%%%%%

The table below shows claims paid on a portfolio of general insurance policies. You
may assume that claims are fully run off after three years.
Underwriting year
2008
2009
2010
2011
Development Year
0
1
2
3
450 312 117 41
503 389 162
611 438
555
Past claims inflation has been 5% p.a. However, it is expected that future claims
inflation will be 10% p.a.
Use the inflation adjusted chain ladder method to calculate the outstanding claims on
the portfolio.
[10]
%%%%%%%%%%%%%%%%%%%%%%%%%%%%%%%%%%%%%%%%%%%%%%%%%%%%%%%%%%%%%%%%%%%%%%%%%%5

%%-- Page 10
%%%%%%%%%%%%%%%%%%%%%%%%%
%% (Statistical Methods) – April 2012 – Examiners’ Report
\newpage
%%-- Question 8
The claims uplifted to 2011 prices are as follows:
Underwriting
Year
2008
2009
2010
2011
Development Year
1
2
343.98
122.85
408.45
162
438
0
520.93
554.56
641.55
555
3
41
Accumulating gives:
Underwriting
Year
2008
2009
2010
2011
0
520.93
554.56
641.55
555
Development Year
1
2
864.91
987.76
963.01 1125.01
1079.55
3
1028.76
Hence the development factors are given by:
DF 0,1 = 864.91 + 963.01 + 1079.55
= 1.693304
520.93 + 554.56 + 641.55
DF 1,2 = 987.76 + 1125.01
= 1.155833
864.91 + 963.01
DF 2,3 = 1028.76
= 1.041508
987.76
The completed triangle of cumulative claims is:
Underwriting
year
2008
2009
2010
2011
0
520.93
554.56
641.55
555.00
Development Year
1
2
864.91
987.76
963.01 1125.01
1079.55 1247.78
939.78 1086.23
3
1028.76
1171.70
1299.57
1131.32
Dis-accumlating gives (in 2011 prices):
Underwriting
year
2008
2009
2010
2011
0
Development Year
1
2
384.78
168.23
146.45
3
46.70
51.79
45.09
%%-- Page 11
%%%%%%%%%%%%%%%%%%%%%%%%%
% (Statistical Methods) – April 2012 – Examiners’ Report
Inflating for future claims growth gives:
Underwriting
year
2008
2009
2010
2011
Development Year
1
2
0
423.26
185.05
177.20
3
51.37
62.67
60.01
And the outstanding claims are:
51.37+62.67+60.01+185.05+177.20+423.26 = 959.56
% This question was tackled very well by most candidates
\end{document}
