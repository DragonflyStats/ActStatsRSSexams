Individual claim amounts from a particular type of insurance policy follow a normal
distribution with mean 150 and standard deviation 30. Claim numbers on an
individual policy follow a Poisson distribution with parameter 0.25. The insurance
company uses a premium loading of 70% to calculate premiums.
(i)
Calculate the annual premium charged by the insurance company.

The insurance company has an individual excess of loss reinsurance arrangement with
a retention of 200 with a reinsurer who uses a premium loading of 120%.
(ii)
Calculate the probability that an individual claim does not exceed the
retention.

(iii) Calculate the probability for a particular policy that in a given year there are
no claims which exceed the retention.

(iv) Calculate the premium charged by the reinsurer.
(v) Calculate the insurance company’s expected profit.
CT6 S2012–3


[Total 11]


%%%%%%%%%%%%%%%%%%%%%%%%%%%%%%%%%%%%%%%%%%%%%%%%%%%%%%%%%%%%%%%%%%%%%%%%%%%%%%%%%%%%%55

%% PLEASE TURN OVER
%% Question 7
The table below shows claims paid on a portfolio of general insurance policies.
Claims from this portfolio are fully run off after 3 years.
Underwriting year
2008
2009
2010
2011
(i)
Development Year
0
1
2
3
85 42 30 7
103 65 25
93 47
111
Estimate the outstanding claims using the basic chain ladder approach.
[7]
You are asked to investigate the fit of the model by applying the development factors
from part (i) to the claims paid in development year 0 and then comparing the fitted
claim payments to the actual payments.
8
(ii) Construct a table showing the difference between the fitted payments and the
actual payments in the table above.

(iii) Comment on the results of the analysis in part (ii).

[Total 12]

%%%%%%%%%%%%%%%%%%%%%%%%%%%%%%%%%%%%%%%%%%%%%%%%%%%%%%%%%%%%%%%%%%%%%%%%%%%%%%%%%%%%%%%%%%%%%%%%%

7
(i)
The aggregated claims are:
Underwriting year
2008
2009
2010
2011
0
85
103
93
111
Development Year
1
2
3
127 157 164
168 193
140
Hence the development factors are given by:
Page 8
DF 0,1 = 127 + 168 + 140
= 1.548043
85 + 103 + 93
DF 1,2 = 157 + 193
= 1.186441
127 + 168
DF 2,3 = 164
= 1.044586
157%%%%%%%%%%%%%%%%%%%%%%%%%%%%%%%%%%%%%%%%%%%%%%%%%%%%%5 – Examiners’ Report
The completed triangle of cumulative claims is:
Underwriting
year
2008
2009
2010
2011
Development Year
1
2
127
157
168
193
140
166.10
171.83
203.87
0
85
103
93
111
3
164
201.61
173.51
212.96
Dis-accumulating gives:
Underwriting year
2008
2009
2010
2011
0
85
103
93
111
Development Year
1
2
3
42
30
7
65
25
8.61
47
26.10 7.41
60.83 32.04 9.09
And so the outstanding claims are:
8.61+7.41+9.09+26.1+32.04+60.83 = 144.08
(ii)
Applying the development factors to the claims in development year 0 gives:
Underwriting year
2008
2009
2010
2011
0
85
103
93
111
Development Year
1
2
3
131.58 156.12 163.08
159.45 189.18
143.97
Dis-accumulating gives:
Underwriting year
2008
2009
2010
2011
0
85
103
93
111
Development Year
1
2
3
46.58 24.53 6.96
56.45 29.73
50.97
And computing the difference between predicted and actual gives:
Underwriting year
2008
2009
2010
2011
0
0
0
0
0
Development Year
1
2
3
−4.58
5.47 0.04
8.55 −4.73
−3.97
Page 9Subject CT6 (Statistical Method) – %%%%%%%%%%%%%%%%%%%%%%%%%%%%%%%%%%5
(iii)
Overall the model seems a reasonable fit, though some of the individual
differences are quite large in percentage terms – for example the difference of
5.47 is 18% of the observed value.
Part (i) was well answered, though a small number of candidates continue to throw away
simple marks by not computing the single figure for outstanding claims. This was the first
time in some years that the material in part (ii) has been tested, and a number of candidates
performed the comparison on a cumulative basis rather than the incremental basis that the
question asked for.
