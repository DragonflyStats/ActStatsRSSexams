

4
(i)
For the Pareto distribution with parameters \alpha  , \lambda  as per the tables we have:
E ( X ) =
\lambda 
\alpha  − 1
And
Var ( X ) =
Page 4
\alpha \lambda  2
( \alpha  − 1 )
2
( \alpha  − 2)
= E ( X ) 2
\alpha 
\alpha − 2%%%%%%%%%%%%%%%%%%%%%%%%%%%%%%%%%%%%%%%%%%%%%%%%%%%%%5 – Examiners’ Report
And so
( )
2 ⎛ \alpha 
2 ⎛ 2 \alpha  − 2 ⎞
⎞
E X 2 = Var ( X ) + E ( X ) 2 = E ( X ) ⎜
+ 1 ⎟ = E ( X ) ⎜
⎟
⎝ \alpha − 2 ⎠
⎝ \alpha − 2 ⎠
The observed values we are trying to fit are
E ( X ) = 170
( )
E X 2 = 400 2 + 170 2 = 434.626 2
So we have
2 \alpha  − 2 E ( X 2 ) 434.626 2
=
=
= 6.53633
\alpha  − 2 E ( X ) 2
170 2
And so
\alpha =
2 − 2 \times  6.53633
= 2.441
(2 − 6.53633)
And finally \lambda  = 1.441 \times  170 = 244.95
(ii)
We must solve
⎛ 244.95 ⎞
0.5 = 1 − ⎜
⎟
⎝ 244.95 + x ⎠
2.441
Re-arranging and taking roots gives
1
2.441
0.5
= 0.7527965 =
244.95
244.95 + x
And so
x =
244.95 − 244.95 \times  0.7527965
= 80.44
0.7527965
So the median is significantly lower than the mean. This demonstrates how
skew the Pareto distribution is.
Alternative correct (and in some cases quicker) solutions are possible and received full
credit. This question was well answered with many candidates scoring full marks.
Page 5Subject CT6 (Statistical Method) – %%%%%%%%%%%%%%%%%%%%%%%%%%%%%%%%%%5
