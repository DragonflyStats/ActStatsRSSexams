\documentclass[a4paper,12pt]{article}

%%%%%%%%%%%%%%%%%%%%%%%%%%%%%%%%%%%%%%%%%%%%%%%%%%%%%%%%%%%%%%%%%%%%%%%%%%%%%%%%%%%%%%%%%%%%%%%%%%%%%%%%%%%%%%%%%%%%%%%%%%%%%%%%%%%%%%%%%%%%%%%%%%%%%%%%%%%%%%%%%%%%%%%%%%%%%%%%%%%%%%%%%%%%%%%%%%%%%%%%%%%%%%%%%%%%%%%%%%%%%%%%%%%%%%%%%%%%%%%%%%%%%%%%%%%%

\usepackage{eurosym}
\usepackage{vmargin}
\usepackage{amsmath}
\usepackage{graphics}
\usepackage{epsfig}
\usepackage{enumerate}
\usepackage{multicol}
\usepackage{subfigure}
\usepackage{fancyhdr}
\usepackage{listings}
\usepackage{framed}
\usepackage{graphicx}
\usepackage{amsmath}
\usepackage{chng%%-- Page}

%\usepackage{bigints}
\usepackage{vmargin}

% left top textwidth textheight headheight

% headsep footheight footskip

\setmargins{2.0cm}{2.5cm}{16 cm}{22cm}{0.5cm}{0cm}{1cm}{1cm}

\renewcommand{\baselinestretch}{1.3}

\setcounter{MaxMatrixCols}{10}

\begin{document}
A proportion p of packets of a rather dull breakfast cereal contain an exciting toy (independently from packet to packet). An actuary has been persuaded by his children to begin buying packets of this cereal. His prior beliefs about p before
opening any packets are given by a uniform distribution on the interval [0,1]. It turns out the first toy is found in the n 1 th packet of cereal.

\begin{enumerate}[(i)]
\item Specify the posterior distribution of p after the first toy is found.

A further toy was found after opening another n 2 packets, another toy after opening
another n 3 packets and so on until the fifth toy was found after opening a grand total
of n_{1}  + n 2 + n 3 + n 4 + n 5 packets.
\item (ii) Specify the posterior distribution of p after the fifth toy is found.
\item (iii) Show the Bayes’ estimate of p under quadratic loss is not the same as the
maximum likelihood estimate and comment on this result.
\end{itemize}
%%------ [Total 10]
%%------ CT6 A2012–3



%%%%%%%%%%%%%%%%%%%%%%%%%%%%%%%%%%%%%%%%%%%%%%%%%%%%%%%%%%%%%%%%%%%%%%%%%%%%%
\newpage


%%%%%%%%%%%%%%%%%%%%%%%%%%%%%%%%%%%%%%%%%%%%%%%%%%%%%%%%%%%%%%%%%%%%%%%%%%%%%%%%%%%%%%%%%%%%%55
6
\begin{itemize}
\item (i)
The posterior distribution has a likelihood given by \[f ( p n 1 ) \propto  f ( n 1 p ) f ( p )
\propto  (1 − p ) n 1 − 1 p \times 1\]
Which is the pdf of a Beta distribution with parameters \alpha = 2 and \beta = n 1 .
(ii)
\item Now the posterior distribution has likelihood given by
\[f ( p n 1 , n 2 , ... , n 5 ) \propto  f ( n 1 , n 2 , ... , n 5 p ) f ( p )\]
\[ \propto  (1 − p ) n 1 − 1 p \times (1 − p ) n 2 − 1 p \times " \times (1 − p ) n_{5}  _ 1 \times p\]
\[\propto  (1 − p ) n_{1}  + n 2 + " + n_{5}  _ 5 \times p 5\]
Which is the pdf of a Beta distribution with parameters \alpha = 6 and
\beta = n_{1}  + n 2 + " + n_{5}  _ 4 .
\item (iii)
Under squared error loss the Bayes estimate is given by the mean of the posterior distribution which in this case is
p ˆ =
\alpha
6
=
\alpha + \beta n_{1}  + n 2 + " + n 5 + 2
\item The maximum likelihood estimate is given by maximising the likelihood which is
\[L \propto  (1 − p ) n_{1}  + n 2 + " + n_{5}  _ 5 \times p 5\]
The log-likelihood is given by
\[l = log L = log C + ( n_{1}  + " + n_{5}  _ 5 ) log(1 − p ) + 5log p\]
And so
dl
1
5
= − ( n_{1}  + " + n_{5}  _ 5 ) \times
+
dp
1 − p p
And setting this expression to zero gives
( n_{1}  + " + n_{5}  _ 5 ) p ˆ = 5(1 − p ˆ )
And so ( n_{1}  + " + n 5 ) p ˆ = 5

%%%%%%%%%%%%%%%%%%%%%%%%%%%%%%%%%%%%%%%%%%%%%%%%%%%%%%%%%%%%%%%%%%%%%%%%%%%%%%%%%%%%%%%%%%%%%55
i.e. p̂ =
5
n_{1}  + " + n 5
\item So the two estimates are not the same. This is perhaps a little surprising given
that we started with an uninformative prior, but arises because the estimates are calculated in two different ways – i.e. one maximises the likelihood and the other minimises the expected squared error. If we wanted the two to be the same we should use an “all-or-nothing” loss function.
\end{itemize}
%A reasonably well answered question. Weaker candidates failed to identify the geometric distribution in part (i). Stronger candidates demonstrated a good understanding of loss functions in part (iii).

%%%%%%%%%%%%%%%%%%%%%%%%%%%%%%%%%%%%%%%%%%%%%%%%%%%%%%%%%%%%%%%%%%%%%%%%%%%%%%%%%%%%%%%%%%%%%55
\end{document}
