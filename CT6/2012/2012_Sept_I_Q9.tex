\documentclass[a4paper,12pt]{article}

%%%%%%%%%%%%%%%%%%%%%%%%%%%%%%%%%%%%%%%%%%%%%%%%%%%%%%%%%%%%%%%%%%%%%%%%%%%%%%%%%%%%%%%%%%%%%%%%%%%%%%%%%%%%%%%%%%%%%%%%%%%%%%%%%%%%%%%%%%%%%%%%%%%%%%%%%%%%%%%%%%%%%%%%%%%%%%%%%%%%%%%%%%%%%%%%%%%%%%%%%%%%%%%%%%%%%%%%%%%%%%%%%%%%%%%%%%%%%%%%%%%%%%%%%%%%

\usepackage{eurosym}

\usepackage{vmargin}

\usepackage{amsmath}

\usepackage{graphics}

\usepackage{epsfig}

\usepackage{enumerate}

\usepackage{multicol}

\usepackage{subfigure}

\usepackage{fancyhdr}

\usepackage{listings}

\usepackage{framed}

\usepackage{graphicx}

\usepackage{amsmath}

\usepackage{chngpage}

%\usepackage{bigints}

\usepackage{vmargin}

% left top textwidth textheight headheight

% headsep footheight footskip

\setmargins{2.0cm}{2.5cm}{16 cm}{22cm}{0.5cm}{0cm}{1cm}{1cm}

\renewcommand{\baselinestretch}{1.3}

\setcounter{MaxMatrixCols}{10}

\begin{document}

In order to model a particular seasonal data set an actuary is considering using a
model of the form
(
)
(1 − B 3 ) 1 − ( \alpha  + \beta  ) B + \alpha \beta  B 2 X t = e t
where B is the backward shift operator and e t is a white noise process with variance
σ 2 .
(i)
Show that for a suitable choice of s the seasonal difference series
Y t = X t − X t − s is stationary for a range of values of \alpha  and \beta  which you should
specify.

After appropriate seasonal differencing the following sample autocorrelation values
for the series Y t are observed: \hat{\rho} 1 = 0.2 and \hat{\rho} 2 = 0.7.
(ii)
Estimate the parameters \alpha  and \beta  based on this information.
[7]
[HINT: let X = \alpha  + \beta , Y = \alpha \beta  and find a quadratic equation with roots
\alpha  and \beta . ]
(iii)
CT6 S2012–5
Forecast the next two observations x̂ 101 and x̂ 102 based on the parameters
estimated in part (ii) and the observed values x 1 , x 2 , ... , x 100 of X t .

[Total 14]

%%%%%%%%%%%%%%%%%%%%%%%%%%%%%%%%%%%%%%%%%%%%%%%%%%%%%%%%%%%%%%%%%%%%%%%%%%%%%%%%%%%%%%%%%%%%


9
(i)
The order s will be 3 i.e. Y t = ∇ 3 X t = X t − X t − 3
The characteristic polynomial will be 1 − ( \alpha  + \beta  )z + \alpha \beta  z 2 with roots 1/ \alpha  and
1/ \beta  .
Hence the process is stationary for \alpha  < 1 and \beta  < 1 .
(ii)
The Yule-Walker equations for the differenced equations give:
ρ 1 − ( \alpha  + \beta  ) + \alpha \beta ρ 1 = 0
ρ 2 − ( \alpha  + \beta  ) ρ 1 + \alpha \beta  = 0
Substituting the observed values of the auto-correlation gives:
0.2 − ( \alpha  + \beta  ) + 0.2 \alpha \beta  = 0
0.7 − 0.2 ( \alpha  + \beta  ) + \alpha \beta  = 0
Let X = \alpha  + \beta  and let Y = \alpha \beta  then we have
0.2 − X + 0.2 Y = 0
0.7 − 0.2 X + Y = 0
The first equation gives X = 0.2 + 0.2 Y and substituting into the second gives:
0.7 − 0.04 − 0.04 Y + Y = 0
So 0.96 Y = − 0.66 and so Y=-0.6875 and X = 0.0625
This means that \alpha  and \beta  are the roots of the quadratic equation
x 2 − 0.0625 x − 0.6875 = 0
Which are
0.0625 \pm  0.0625 2 + 4 \times  0.6875
2
i.e. 0.860995 and −0.79849
Page 12
%%%%%%%%%%%%%%%%%%%%%%%%%%%%%%%%%%%%%%%%%%%%%%%%%%%%%5 – Examiners’ Report
(iii)
Since Y t = X t − X t − 3 we have that
X 101 = Y 101 + X 98
and
X 102 = Y 102 + X 99
With the forecasted values
x ˆ 101 = y ˆ 101 + x 98
and
x ˆ 102 = y ˆ 102 + x 99
where
y ˆ 101 = 0.0625 y 100 + 0.6875 y 99 = 0.0625 ( x 100 − x 97 ) + 0.6875( x 99 − x 96 )
and
y ˆ 102 = 0.0625 y ˆ 101 + 0.6875 ( x 100 − x 97 )
Many candidates struggled with this question. In particular many failed to identify quickly
that s=3 in part (i) leads to difficult algebra in part (ii). Those who did identify that s=3
were generally able to write down the Yule Walker equations and make some progress in part
(ii) though only the better candidates were able to find the numerical values required.
