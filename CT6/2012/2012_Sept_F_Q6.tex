

%%%%%%%%%%%%%%%%%%%%%%%%%%%%%%%%%%%%%%%%%%%%%%%%%%%%%%%%%%%%%%%5
6
(i) The annual premium charged is 0.25 \times  150 \times  1.7 = 63.75
(ii) Let X be an individual claim. Then
( (
)
P ( X < 200 ) = P N 150,30 2 < 200
)
200 − 150 ⎞
⎛
= P ⎜ N ( 0,1 ) <
⎟
30
⎝
⎠
= P ( N ( 0,1 ) < 1.667)
= ( 0.95154 \times  0.3 + 0.7 \times  0.95254 )
= 0.95224
(iii)
We need to calculate:
∞
p = \sum  P ( j claims) \times  P (all claims below retention) 
j = 0
∞
= \sum  e
− 0.25
j = 0
(0.25) j
\times  (0.95224) j
j !
∞
(0.25 \times  0.95224) j
j !
j = 0
= e − 0.25 \times  \sum 
= e − 0.25 \times  e 0.25 \times  0.95224
= 0.9881
(iv)
We need to first calculate the mean claim amount paid by the reinsurer. This
is given by
∞
I =
∫ ( x − 200 ) f ( x ) dx
200
Where f(x) is the pdf of the Normal distribution with mean 150 and standard
deviation 30.
Page 7Subject CT6 (Statistical Method) – %%%%%%%%%%%%%%%%%%%%%%%%%%%%%%%%%%5
Using the formula on p18 of the tables, we have:
∞
I =
∫ xf ( x ) dx − 200 P ( X > 200)
200
= 150 \times  ⎡ ⎣ \Phi ( ∞ ) − \Phi(1.667) ⎤ ⎦ − 30 \times  ( \Phi ( ∞ ) − \Phi ( 1.667 ) ) − 200 \times  (1 − 0.95224)
= 150 ( 1 − 0.95224 ) − 30 \times  ( 0 − 0.09942 ) − 200 \times  0.0.04776
= 0.5946
So the reinsurer charges 0.25 \times  0.5946 \times  2.2 = 0.32703
%%%%%%%%%%%%%%%%%%%%%%%%%%%%%%%%%%%%%%%%%%%%%%%%%%%%%%%%%%%%%%%%%%%%%%%%%%%%%%%%%%%%%
(v)
The direct insurers expected profit is given by:
63.75 − 0.32703 − 0.25 \times  ( 150 − 0.5946 ) = 26.07
Comment: Answers were mixed here. Parts (i) and (ii) were generally well done. Only the
best candidates completed part (iii) with most being unable to condition on the number of
claims. On part (iv) most candidates wrote down the integral that needed to be evaluated,
but only the better candidates were able to use the formula from the tables to evaluate it. A
number of candidates struggled to compute the values of the probability density function of
the Normal distribution.


\end{document}
