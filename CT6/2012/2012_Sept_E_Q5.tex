\documentclass[a4paper,12pt]{article}



%%%%%%%%%%%%%%%%%%%%%%%%%%%%%%%%%%%%%%%%%%%%%%%%%%%%%%%%%%%%%%%%%%%%%%%%%%%%%%%%%%%%%%%%%%%%%%%%%%%%%%%%%%%%%%%%%%%%%%%%%%%%%%%%%%%%%%%%%%%%%%%%%%%%%%%%%%%%%%%%%%%%%%%%%%%%%%%%%%%%%%%%%%%%%%%%%%%%%%%%%%%%%%%%%%%%%%%%%%%%%%%%%%%%%%%%%%%%%%%%%%%%%%%%%%%%



\usepackage{eurosym}

\usepackage{vmargin}

\usepackage{amsmath}

\usepackage{graphics}

\usepackage{epsfig}

\usepackage{enumerate}

\usepackage{multicol}

\usepackage{subfigure}

\usepackage{fancyhdr}

\usepackage{listings}

\usepackage{framed}

\usepackage{graphicx}

\usepackage{amsmath}

\usepackage{chngpage}



%\usepackage{bigints}

\usepackage{vmargin}



% left top textwidth textheight headheight



% headsep footheight footskip



\setmargins{2.0cm}{2.5cm}{16 cm}{22cm}{0.5cm}{0cm}{1cm}{1cm}



\renewcommand{\baselinestretch}{1.3}



\setcounter{MaxMatrixCols}{10}



\begin{document}


5
Claims arising on a particular type of insurance policy are believed to follow a Pareto
distribution. Data for the last several years shows the mean claim size is 170 and the
standard deviation is 400.
(i) Fit a Pareto distribution to this data using the method of moments.

(ii) Calculate the median claim using the fitted parameters and comment on the
result.

[Total 7]
A discrete probability distribution is defined by
⎛ n ⎞
n − ny
f ( y , \mu  ) = ⎜ ⎟ \mu  ny ( 1 − \mu  )
⎝ ny ⎠
1 2
y = 0, , , ... .,1
n n
where \mu  is a parameter between 0 and 1.
6
(i) Explain why this distribution belongs to an exponential family. 
(ii) State the three main components that need to be taken into account when
constructing a generalised linear model. 
(iii) Suggest a natural choice of link function if the response variable followed the
distribution defined above.

(iv) Suggest a natural choice of link function if instead the response variable
followed a lognormal distribution.

[Total 10]



%%%%%%%%%%%%%%%%%%%%%%%%%%%%%%%%%%%%%%%%%%%%%%%%%%%%%%%%%%%%%%%%%%%%%%%%%%%%%%%%%%%%%%%%%%%%%%%

5
(i)
From the definition
⎡
⎛ n ⎞ ⎤
f ( y , \mu  ) = exp ⎢ n ( y log \mu  + ( 1 − y ) log ( 1 − \mu  ) ) + log ⎜ ⎟ ⎥
⎝ ny ⎠ ⎦
⎣
⎡ ⎛
⎛ n ⎞ ⎤
⎞
\mu 
= exp ⎢ n ⎜ y log(
) + log ( 1 − \mu  ) ⎟ + log ⎜ ⎟ ⎥
1 − \mu 
⎠
⎝ ny ⎠ ⎦
⎣ ⎝
⎡ y \theta  − b ( \theta  )
⎤
= exp ⎢
+ c ( y , \Phi ) ⎥
⎣ a ( \Phi )
⎦
Which is the right form for a member of an exponential family where
⎛ \mu  ⎞
\theta  = log ⎜
⎟
⎝ 1 − \mu  ⎠
\Phi= n
a ( \Phi ) =
1
\Phi
(
b ( \theta  ) = log 1 + e \theta 
)
⎛ \Phi ⎞
c ( y , \Phi ) = log ⎜ ⎟
⎝ \Phi y ⎠
Hence the distribution does belong to an exponential family.
(ii)
The three main components are:
•
•
•
the distribution of the response variable
a linear predictor of the covariates
link function between the response variable and the linear predictor
(iii) In this case we have a binomial distribution and therefore the natural choice of
⎛ \mu  ⎞
link function is g ( \mu  ) = log ⎜
⎟ .
⎝ 1 − \mu  ⎠
(iv) We could apply a log transform to the response and then apply a simple linear
regression. Hence the link function is log ( \mu  ) .
This was well answered, though a number of candidates lost some marks through failing to
carefully define all of the parameters involved in the characterisation as a member of the
exponential family.
Page 6%%%%%%%%%%%%%%%%%%%%%%%%%%%%%%%%%%%%%%%%%%%%%%%%%%%%%5 – Examiners’ Report
