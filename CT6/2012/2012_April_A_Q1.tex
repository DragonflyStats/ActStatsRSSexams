\documentclass[a4paper,12pt]{article}

%%%%%%%%%%%%%%%%%%%%%%%%%%%%%%%%%%%%%%%%%%%%%%%%%%%%%%%%%%%%%%%%%%%%%%%%%%%%%%%%%%%%%%%%%%%%%%%%%%%%%%%%%%%%%%%%%%%%%%%%%%%%%%%%%%%%%%%%%%%%%%%%%%%%%%%%%%%%%%%%%%%%%%%%%%%%%%%%%%%%%%%%%%%%%%%%%%%%%%%%%%%%%%%%%%%%%%%%%%%%%%%%%%%%%%%%%%%%%%%%%%%%%%%%%%%%

\usepackage{eurosym}
\usepackage{vmargin}
\usepackage{amsmath}
\usepackage{graphics}
\usepackage{epsfig}
\usepackage{enumerate}
\usepackage{multicol}
\usepackage{subfigure}
\usepackage{fancyhdr}
\usepackage{listings}
\usepackage{framed}
\usepackage{graphicx}
\usepackage{amsmath}
\usepackage{chngpage}

%\usepackage{bigints}
\usepackage{vmargin}

% left top textwidth textheight headheight

% headsep footheight footskip

\setmargins{2.0cm}{2.5cm}{16 cm}{22cm}{0.5cm}{0cm}{1cm}{1cm}

\renewcommand{\baselinestretch}{1.3}

\setcounter{MaxMatrixCols}{10}

\begin{document}

\begin{enumerate}
%%%%%%%%%%%%%%%%%%%%%%%%%
\item 

Define what it means for a random variable to belong to an exponential
family.
[1]
Show that if a random variable has the exponential distribution it belongs to an
exponential family.

\item 

\end{enumerate}
\newpage

%%%%%%%%%%%%%%%%%%%%%%%%%%%%%%%%%%%%%%%%%%%%%%%%%%%%%%%%%%%%%%%%%%%%%%%%%%%%%55
1
(i)
A random variable Y belongs to an exponential family if the pdf of Y can be
written in the form
⎡ y \theta  − b ( \theta  )
⎤
f ( y ; \theta  , \phi  ) = exp ⎢
− c ( y , \phi  ) ⎥
⎣ a ( \phi  )
⎦
Where a, b and c are functions.
(ii)
Suppose that the parameter of the exponential distribution Y is \lambda . Then
f ( y ) = \lambda  exp ( −\lambda  y )
= exp [ log \lambda  − \lambda  y ]
⎡ \lambda  y − log \lambda  ⎤
= exp ⎢
− 1 ⎥ ⎦
⎣
Which is of the required form with
\theta =\lambda 
a ( \phi  ) = − 1
b ( \theta  ) = log \theta 
c ( y , \phi  ) = 0
Alternative solution: \theta  = −\lambda ; a(\phi ) = 1; b(\theta ) = −log(−\theta ); c(y,\phi ) = 0
This question was answered well.
\end{document}
