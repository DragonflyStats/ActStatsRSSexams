\documentclass[a4paper,12pt]{article}

%%%%%%%%%%%%%%%%%%%%%%%%%%%%%%%%%%%%%%%%%%%%%%%%%%%%%%%%%%%%%%%%%%%%%%%%%%%%%%%%%%%%%%%%%%%%%%%%%%%%%%%%%%%%%%%%%%%%%%%%%%%%%%%%%%%%%%%%%%%%%%%%%%%%%%%%%%%%%%%%%%%%%%%%%%%%%%%%%%%%%%%%%%%%%%%%%%%%%%%%%%%%%%%%%%%%%%%%%%%%%%%%%%%%%%%%%%%%%%%%%%%%%%%%%%%%

\usepackage{eurosym}
\usepackage{vmargin}
\usepackage{amsmath}
\usepackage{graphics}
\usepackage{epsfig}

\usepackage{enumerate}
\usepackage{multicol}
\usepackage{subfigure}

\usepackage{fancyhdr}
\usepackage{listings}

\usepackage{framed}
\usepackage{graphicx}

\usepackage{amsmath}
\usepackage{chngpage}

%\usepackage{bigints}

\usepackage{vmargin}

% left top textwidth textheight headheight

% headsep footheight footskip

\setmargins{2.0cm}{2.5cm}{16 cm}{22cm}{0.5cm}{0cm}{1cm}{1cm}

\renewcommand{\baselinestretch}{1.3}

\setcounter{MaxMatrixCols}{10}

\begin{document}

%%-- Question 10
Claims occur on a portfolio of insurance policies according to a Poisson process.
Individual claim amounts are either 1 (with probability 0.7) or 8 (with probability
0.3). The insurance company uses a premium loading of 60% to calculate premiums
and buys excess of loss reinsurance with a retention of M (1<M<8) from a reinsurer.
The reinsurer uses a premium loading of 120%.
\begin{enumerate}
\item (i) Calculate the smallest value of M that the insurance company should consider
if it wishes to expect to make a profit on this portfolio.

\item (ii) Derive the adjustment coefficient equation for the insurance company. 
\item (iii) Calculate the adjustment coefficient (correct to 2 decimal places) if M=4. 
The same reinsurer also offers proportional reinsurance with the same premium
loading such that the reinsurer pays a proportion \alpha  of each claim.
(iv) Show that the insurance company may either purchase excess of loss
3(8 − M )
reinsurance with retention M or proportional reinsurance with \alpha  =
31
for the same premium.

\item (v) Determine whether the adjustment coefficient with proportional reinsurance is
higher or lower than that with excess of loss reinsurance when M=4.

\item (vi) Comment on the implications of part (v).
\end{enumerate}
%%%%%%%%%%%%%%%%%%%%%%%%%%%%%%%%%%%%%%%%%%%%%%%%%%%%%%%%%%%%%%%%%%%%%%%%%%%%%%%%%%%%%%%%%%%%%%%%%%%%%%%%%%%%%%%%%%%%%%%%%55
10
(i)
The insurer charges a premium of \lambda  \times  ( 1 \times  0.7 + 8 \times  0.3 ) \times  1.6 = 4.96 \lambda 
Where \lambda  is the rate of the Poisson process. Expected claims outgo (net of
reinsurance) is given by \lambda  \times  ( 1 \times  0.7 + M \times  0.3 ) = \lambda  (0.7 + 0.3 M )
The premiums charged by the reinsurer are
\[\lambda  \times  ( 0.3 \times  ( 8 − M ) \times  2.2 ) = 0.66 \lambda  (8 − M )\]
So the expected profit is positive if:
\[4.96 \lambda  − \lambda  ( 0.7 + 0.3 M ) − 0.66 \lambda  ( 8 − M ) > 0\]
i.e.
− 1.02 + 0.36 M > 0
Page 13Subject CT6 (Statistical Method) – %%%%%%%%%%%%%%%%%%%%%%%%%%%%%%%%%%5
i.e.
M >
(ii)
1.02
= 2.833
0.36
The adjustment coefficient is equation is:
\[M X ( R ) − 1 − cR = 0\]
Comment: Or alternatively \lambda  + c net R = \lambda  M Y ( R ) , where c net is the overall net
premium.
Where X is the distribution of net claim payments by the direct insurer. This
gives:
\[0.7 e R + 0.3 e MR − 1 − ( 4.96 − 0.66 ( 8 − M ) ) R = 0\]
(iii)
With M = 4 this equation becomes:
\[f ( R ) : = 0.7 e R + 0.3 e 4 R − 1 − 2.32 R = 0\]
We shall find R by trial and error
f(0.1) = −0.0108<0
f(0.2) = 0.058644209>0
f(0.15) = 0.01191961>0
f(0.125) = −0.002179<0
f(0.135 )= 0.002778077>0
So the root lies between 0.125 and 0.135 and so R=0.13 (to 2 decimal places)
(iv)
The premium charged by the reinsurer for the proportional reinsurance is
\lambda  \times  \alpha  \times  2.2 \times  ( 0.7 + 0.3 \times  8 ) = 6.82 \alpha \lambda  .
Equating the premiums for the two types of reinsurance we get
\[6.82 \alpha \lambda  = 0.66 \lambda  (8 − M )\]
i.e.
\alpha =
(v)
Page 14
0.66(8 − M ) 3(8 − M )
=
6.82
31
In this case \alpha  = 0.387096774 and the premium charged by the reinsurer is
2.64\lambda  .%%%%%%%%%%%%%%%%%%%%%%%%%%%%%%%%%%%%%%%%%%%%%%%%%%%%%5 – Examiners’ Report
The adjustment coefficient equation for the insurer is given by
g ( R ) : = 0.7 e 0.612903226 R + 0.3 e 4.903225806 R − 1 − 2.32 R = 0
Again by trial and error
g ( 0.125 ) = 0.019471559 > 0
g ( 0.135 ) = 0.028745229 > 0
g ( 0.001 ) = − 0.00041625635 < 0
So the root lies between 0.001 and 0.125 and is therefore less than in the
excess of loss case.
(vi)
By Lundberg’s inequality the adjustment coefficient is an inverse measure of risk – that is, the higher the coefficient the lower the probability of ruin. The
excess of loss reinsurance is therefore more effective at reducing the probability of ruin than the proportional reinsurance.
Many candidates really struggled with this question, and in particular with the re-insurance arrangement and its impact on the claims paid and net premiums received by the insurer. A
not insignificant number assumed that the insurer would reduce the premiums it charged the customer as a result of the reinsurance. Only the best candidates managed to accurately
produce the equations satisfied by the adjustment coefficient and go on to find the numerical values.
\end{document}
