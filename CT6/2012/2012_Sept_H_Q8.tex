\documentclass[a4paper,12pt]{article}

%%%%%%%%%%%%%%%%%%%%%%%%%%%%%%%%%%%%%%%%%%%%%%%%%%%%%%%%%%%%%%%%%%%%%%%%%%%%%%%%%%%%%%%%%%%%%%%%%%%%%%%%%%%%%%%%%%%%%%%%%%%%%%%%%%%%%%%%%%%%%%%%%%%%%%%%%%%%%%%%%%%%%%%%%%%%%%%%%%%%%%%%%%%%%%%%%%%%%%%%%%%%%%%%%%%%%%%%%%%%%%%%%%%%%%%%%%%%%%%%%%%%%%%%%%%%

\usepackage{eurosym}

\usepackage{vmargin}

\usepackage{amsmath}

\usepackage{graphics}

\usepackage{epsfig}

\usepackage{enumerate}

\usepackage{multicol}

\usepackage{subfigure}

\usepackage{fancyhdr}

\usepackage{listings}

\usepackage{framed}

\usepackage{graphicx}

\usepackage{amsmath}

\usepackage{chngpage}

%\usepackage{bigints}

\usepackage{vmargin}

% left top textwidth textheight headheight

% headsep footheight footskip

\setmargins{2.0cm}{2.5cm}{16 cm}{22cm}{0.5cm}{0cm}{1cm}{1cm}

\renewcommand{\baselinestretch}{1.3}

\setcounter{MaxMatrixCols}{10}

\begin{document}

An insurer classifies the buildings it insures into one of three types. For Type 1
buildings, the number of claims per building per year follows a Poisson distribution
with parameter \lambda . Data are available for the last five years as follows:
Year
Number of type 1 buildings covered
Number of claims
(i)
1
89
15
2
112
23
3
153
29
4
178
41
5
165
50
Determine the maximum likelihood estimate of \lambda  based on the data above. 
The insurer also has data for the other two types of building for all five years. Define
P ij = the number of buildings insured in the jth year from type i and
Y ij = the corresponding number of claims.
CT6 S2012–4The five years of data can be summarised as follows:
5
Type(i)
5
P i = \sum  P ij
X i = \sum 
P
j = 1 i
j = 1
Type 1
Type 2
Type 3
697
295
515
3
Y ij
0.226686
0.237288
0.330097
5
X = \sum \sum 
i = 1 j = 1
Y ij
P
⎛ Y ij
⎞
\sum  P ij ⎜ ⎜ P ij − X i ⎟ ⎟
j = 1 ⎝
⎠
1.527016
0.96605
4.53253
5
2
⎛ Y ij
⎞
\sum  P ij ⎜ ⎜ P ij − X ⎟ ⎟
j = 1 ⎝
⎠
2.502737
1.178133
6.775614
5
2
3
= 0.264101 where P = \sum  P i
i = 1
There are 191 buildings of Type 1 to be insured in year six.
(ii)
(iii)
9
Estimate the number of claims from Type 1 buildings in year six using
Empirical Bayes Credibility Theory model 2.

Explain the main differences between the approaches in parts (i) and (ii). 
[Total 13]

%%%%%%%%%%%%%%%%%%%%%%%%%%%%%%%%%%%%%%%%%%%%%%%%%%%%%%%%%%%%%%%%%%%%%%%%%%%%%%%%%%%%%%%%%%%%%%%%%%%%%%%%%%

8
(i)
Let N i be the number of type 1 buildings covered in year i. Set
N = N 1 + " + N 5 = 697. Let the number of claims in year i be denoted by M i
and set M = M 1 + " + M 5 . Then under the conditions in the question
M ~ Poisson ( N\lambda  ) .
The likelihood is given by
L = Ce − 697 \lambda 
(697 \lambda  ) m
m !
Where m=158 is the total number of claims over the 5 years. The log-
likelihood is given by
l = log L = D − 697 \lambda  + m log 697 \lambda 
Differentiating gives
dl
m
= − 697 +
d \lambda 
\lambda 
And setting this equal to zero we get
m
158
\lambda  ˆ =
=
= 0.226686
697 6 97
This is a maximum since
(ii)
d \lambda 
2
=−
m
\lambda  2
< 0 .
We first need to calculate
P * =
=
Page 10
d 2 l
3
⎛ P ⎞
1
P i ⎜ 1 − i ⎟
\sum 
5 \times  3 − 1 i = 1 ⎝
P ⎠
1 ⎡
697 ⎞
295 ⎞
515 ⎞ ⎤
⎛
⎛
⎛
697 ⎜ 1 −
⎟ + 295 ⎜ 1 −
⎟ + 515 ⎜ 1 −
⎟ ⎥
⎢
14 ⎣
⎝ 1507 ⎠
⎝ 1507 ⎠
⎝ 1507 ⎠ ⎦%%%%%%%%%%%%%%%%%%%%%%%%%%%%%%%%%%%%%%%%%%%%%%%%%%%%%5 – Examiners’ Report
= 67.9207
The estimators are given by
E ( m ( \theta  ) ) = X = 0.264101
⎞
1 3 1 5 ⎛ Y ij
−
E s ( \theta  ) = \sum 
P
X
⎜
⎟
ij ⎜
i ⎟
P
3 i = 1 5 − 1 \sum 
ij
j = 1 ⎝
⎠
(
=
)
2
2
1
\times  ( 1.527016 + 0.96605 + 4.53253 ) = 0.585466
12
2
⎛
⎞
3 5
⎛ Y ij
⎞
1 ⎜ 1
2
(
)
Var ( m ( \theta  ) ) = *
P
X
E
s
−
−
\theta 
⎟
( ) ⎟ ⎟
ij ⎜
\sum \sum 
⎜
⎟
⎜
P 3 \times  5 − 1 i = 1 j = 1 ⎝ P ij
⎠
⎝
⎠
=
1
⎛ 1
⎞
\times  ⎜ ( 2.502737 + 1.178133 + 6.775614 ) − 0.585466 ⎟
67.9207 ⎝ 14
⎠
= 0.00237668
And the credibility factor for type 1 policies is given by
P 1
Z 1 =
P 1 +
E ( s ( \theta  ) )
Var ( m ( \theta  ) )
2
=
697
= 0.73887
0.585466
697 +
0.00237668
Number of claims per unit risk is then given by
0.73887 \times  0.226686 + ( 1 − 0.73887 ) \times  0.264101 = 0.2364571
And so expected claims are 0.2364571 \times  191 = 45.16
(iii)
The main differences are:
• The approach in (i) uses only the data from type 1 policies; the approach in
(ii) uses a weighted average of the data from type 1 policies and the overall
data.
• The approach in (i) makes a precise distributional assumption about claims
(i.e. that they are Poisson distributed). This assumption is not used in
approach (ii).
Part (i) was often not well answered, with many weaker candidates not reflecting the fact that
the number of buildings covered impacts the parameter of the Poisson distribution for the
Page 11Subject CT6 (Statistical Method) – %%%%%%%%%%%%%%%%%%%%%%%%%%%%%%%%%%5
number of claims. Parts (ii) and (iii) were generally well answered, which was pleasing
given that this was the first appearance of EBCT Model 2 since its return to the syllabus.
