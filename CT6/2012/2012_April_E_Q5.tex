\documentclass[a4paper,12pt]{article}

%%%%%%%%%%%%%%%%%%%%%%%%%%%%%%%%%%%%%%%%%%%%%%%%%%%%%%%%%%%%%%%%%%%%%%%%%%%%%%%%%%%%%%%%%%%%%%%%%%%%%%%%%%%%%%%%%%%%%%%%%%%%%%%%%%%%%%%%%%%%%%%%%%%%%%%%%%%%%%%%%%%%%%%%%%%%%%%%%%%%%%%%%%%%%%%%%%%%%%%%%%%%%%%%%%%%%%%%%%%%%%%%%%%%%%%%%%%%%%%%%%%%%%%%%%%%

\usepackage{eurosym}
\usepackage{vmargin}
\usepackage{amsmath}
\usepackage{graphics}
\usepackage{epsfig}
\usepackage{enumerate}
\usepackage{multicol}
\usepackage{subfigure}
\usepackage{fancyhdr}
\usepackage{listings}
\usepackage{framed}
\usepackage{graphicx}
\usepackage{amsmath}
\usepackage{chng%%-- Page}

%\usepackage{bigints}
\usepackage{vmargin}

% left top textwidth textheight headheight

% headsep footheight footskip

\setmargins{2.0cm}{2.5cm}{16 cm}{22cm}{0.5cm}{0cm}{1cm}{1cm}

\renewcommand{\baselinestretch}{1.3}

\setcounter{MaxMatrixCols}{10}

\begin{document}

The total claim amount per annum on a particular insurance policy follows a normal distribution with unknown mean $\theta$ and variance 200 2 . Prior beliefs about $\theta$ are described by a normal distribution with mean 600 and variance 50 2 . Claim amounts
$\{x 1 , x 2 , ... ., x n\}$ are observed over n years.

\begin{enumerate}[(a)]
\item State the posterior distribution of $\theta$ .
\item Show that the mean of the posterior distribution of $\theta$ can be written in the
form of a credibility estimate.


\item Now suppose that $n=5$ and that total claims over the five years were 3,400.

Calculate the posterior probability that $\theta$ is greater than 600.
\end{enumerate}
%---------------------------------------------------------------------------------%



%%%%%%%%%%%%%%%%%%%%%%%%%%%%%%%%%%%%%%%%%%%%%%%%%%%%%%%%%%%%%%%%%%%%%%%%%%%%%
\newpage
5
(i)
The posterior distribution of \theta is Normal with variance given by
\sigma * 2 =
1
1 ⎞
⎛ n
+ 2 ⎟
⎜
2
50 ⎠
⎝ 200
And mean given by
600 ⎞
⎛ nx
μ * = \sigma * 2 ⎜
+ 2 ⎟
2
50 ⎠
⎝ 200
(ii)
Set
Z = \sigma * 2
n
200 2
Then
Z =
n
200 2
1 ⎞
⎛ n
+ 2 ⎟
⎜
2
50 ⎠
⎝ 200
=
n
( n + 16)
And
1 − Z =
1
50 2
1 ⎞
⎛ n
+ 2 ⎟
⎜
2
50 ⎠
⎝ 200
And so
μ * = Zx + ( 1 − Z ) 600
%%-- Page 6
= \sigma * 2
1
50 2
%%%%%%%%%%%%%%%%%%%%%%%%%%%%%%%%%%%%%%%%%%%%%%%%%%%%%%%%%%%%%%%%%%%%%%%%%%%%%%%%%%%%%%%%%%%%%55

Which is in the form of a credibility estimate with 600 being the prior mean, x being the observed sample mean and Z being the credibility factor.

(iii)
In this case we have
\sigma * 2 =
1
1 ⎞
⎛ n
+ 2 ⎟
⎜
2
50 ⎠
⎝ 200
=
1
1 ⎞
⎛ 5
+ 2 ⎟
⎜
2
50 ⎠
⎝ 200
= 43.64 2
and
600 ⎞
⎛ nx
2 ⎛ 3400 600 ⎞
43.64
μ * = \sigma * 2 ⎜
+
=
+
⎟
⎜
⎟ = 619.0476
⎝ 200 2 50 2 ⎠
⎝ 200 2 50 2 ⎠
So
( (
)
P ( \theta > 600 ) = P N 619.0476, 43.64 2 > 600
)
600 − 619.0476 ⎞
⎛
= P ⎜ N ( 0,1 ) >
⎟ = P ( N ( 0,1 ) > − 0.436)
43.64
⎝
⎠
= 0.6 \times 0.67003 + 0.4 \times 0.66640
= 0 . 669


% This question was well answered. Some candidates attempted to derive the answer to part (i) from first principles which was not required. Parts (ii) and (iii) were generally answered well.
%%-- Page 7

%%%%%%%%%%%%%%%%%%%%%%%%%%%%%%%%%%%%%%%%%%%%%%%%%%%%%%%%%%%%%%%%%%%
\end{document}
