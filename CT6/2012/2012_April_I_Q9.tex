\documentclass[a4paper,12pt]{article}

%%%%%%%%%%%%%%%%%%%%%%%%%%%%%%%%%%%%%%%%%%%%%%%%%%%%%%%%%%%%%%%%%%%%%%%%%%%%%%%%%%%%%%%%%%%%%%%%%%%%%%%%%%%%%%%%%%%%%%%%%%%%%%%%%%%%%%%%%%%%%%%%%%%%%%%%%%%%%%%%%%%%%%%%%%%%%%%%%%%%%%%%%%%%%%%%%%%%%%%%%%%%%%%%%%%%%%%%%%%%%%%%%%%%%%%%%%%%%%%%%%%%%%%%%%%%

\usepackage{eurosym}
\usepackage{vmargin}
\usepackage{amsmath}
\usepackage{graphics}
\usepackage{epsfig}
\usepackage{enumerate}
\usepackage{multicol}
\usepackage{subfigure}
\usepackage{fancyhdr}
\usepackage{listings}
\usepackage{framed}
\usepackage{graphicx}
\usepackage{amsmath}
\usepackage{chng%%-- Page}

%\usepackage{bigints}
\usepackage{vmargin}

% left top textwidth textheight headheight

% headsep footheight footskip

\setmargins{2.0cm}{2.5cm}{16 cm}{22cm}{0.5cm}{0cm}{1cm}{1cm}

\renewcommand{\baselinestretch}{1.3}

\setcounter{MaxMatrixCols}{10}

\begin{document}

\begin{enumerate}
%%%%%%%%%%%%%%%%%%%%%%%%%
\item 9
Consider the time series model
(1 − \alpha B ) 3 X t = e t
where B is the backwards shift operator and e t is a white noise process with variance
\sigma 2 .
(i)
Determine for which values of \alpha the process is stationary.

Now assume that \alpha = 0.4.
(ii)
(iii)
10
(a) Write down the Yule-Walker equations.
(b) Calculate the first two values of the auto-correlation function \rho  1 and
\rho  2 .

Describe the behaviour of \rho  k and the partial autocorrelation function \phi k as
k →∞ .

[Total 12]

\end{enumerate}
%%%%%%%%%%%%%%%%%%%%%%%%%%%%%%%%%%%%%%%%%%%%%%%%%%%%%%%%%%%%%%%%%%%%%%%%%%%%%%%%%%%%%%%%%%%%%%%%%%%%%%%%%%%5
\new%%-- Page 
9
(i)
The characteristic polynomial is (1 − \alpha Y ) 3 = 0 .
This has a triple root at 1
\alpha
and so the process is stationary when
i.e. $\alpha < 1$ .
(ii)
Expanding the cubic equation and rearranging gives:
X t − 3 \alpha X t − 1 + 3 \alpha_{2} X t − 2 − \alpha_{3} X t − 3 = e t
So the Yule-Walker equations give:

\[\rho  0 − 3 \alpha\rho  1 + 3 \alpha_{2} \rho  2 − \alpha_{3} \rho  3 = \sigma 2
\rho  1 − 3 \alpha + 3 \alpha_{2} \rho  1 − \alpha_{3} \rho  2 = 0 (A)
\rho  2 − 3 \alpha\rho  1 + 3 \alpha_{2} − \alpha_{3} \rho  1 = 0 (B)
\rho  3 − 3 \alpha\rho  2 + 3 \alpha_{2} \rho  1 − \alpha_{3} \rho  0 = 0
(
)\]
So re-writing we have from (A) \rho  1 1 + 3 \alpha_{2} − 3 \alpha = \alpha_{3} \rho  2
And substituting into (B) gives
(
)
\rho  1 1 + 3 \alpha_{2} − 3 \alpha
\alpha
i.e.
%%-- Page 12
3
− 3 \alpha\rho  1 + 3 \alpha_{2} − \alpha_{3} \rho  1 = 0
\rho  1 (1 + 3 \alpha_{2} − 3 \alpha_{4} − \alpha 6 )
\alpha_{3}
=
3 \alpha − 3 \alpha_{5}
\alpha_{3}
1
> 1
\alpha

%%%%%%%%%%%%%%%%%%%%%%%%%%%%%%%%%%%%%
i.e.
\rho  1 =
3 \alpha (1 − \alpha_{4} )
(1 + 3 \alpha_{2} − 3 \alpha_{4} − \alpha 6 )
= 0.83573487
And so
\rho  2 =
(
)
3 \alpha_{1} − \alpha_{4} (1 + 3 \alpha_{2} )
2
4
6
(1 + 3 \alpha − 3 \alpha − \alpha ) \alpha
3
−
3
\alpha_{2}
= 0.576368876
\newpage

\subsection*{Alternative solution:}
Express the Yule-Walker equations in terms of the covariances:
X t = 1.2 X t − 1 − 0.48 X t − 2 + 0.064 X t − 3 + e t
\gamma_{0} = 1.2 \gamma_{1} − 0.48 \gamma_{2} + 0.064 \gamma_{3} + \sigma 2
\gamma_{1} = 1.2 \gamma_{0} − 0.48 \gamma_{1} + 0.064 \gamma_{2}
\gamma_{2} = 1.2 \gamma_{1} − 0.48 \gamma_{0} + 0.064 \gamma_{1}
\gamma_{3} = 1.2 \gamma_{2} − 0.48 \gamma_{1} + 0.064 \gamma_{0}
Or in general:
\gamma_{0} = 1.2 \gamma_{1} − 0.48 \gamma_{2} + 0.064 \gamma_{3} + \sigma 2
\gamma k = 1.2 \gamma k − 1 − 0.48 \gamma k − 2 + 0.064 \gamma k − 3 k \geq  1
Simplifying the second and third equations:
148 \gamma_{1} = 1.2 \gamma_{0} + 0.064 \gamma_{2} ⇒ \gamma_{1} =
30 \gamma
37 0
8 \gamma
+ 185
2
\gamma_{2} = 1.264 \gamma_{1} + 0.064 \gamma_{1}
To obtain:
\gamma_{2} =
200 \gamma
347 0
\lambda 1 =
290 \gamma
347 0
Dividing both by \gamma_{0} gives the same solutions as above.
(iii)
The series is an AR(3) series. The asymptotic behaviour is therefore that \rho  k
decays exponentially to zero
whilst \phi k is zero for k>3.
The latter parts of this question were not particularly well answered. Candidates generally
showed an understanding of how to solve the problem, but made a number of arithmetic and
algebraic slips.
%%%%%%%%%%%%%%%%%%%%%%%%%%%%%%%%%%%%%%%%%%%%%%%%%%%%%%%%%%%%%%

\end{document}
