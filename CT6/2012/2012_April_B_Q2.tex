\documentclass[a4paper,12pt]{article}

%%%%%%%%%%%%%%%%%%%%%%%%%%%%%%%%%%%%%%%%%%%%%%%%%%%%%%%%%%%%%%%%%%%%%%%%%%%%%%%%%%%%%%%%%%%%%%%%%%%%%%%%%%%%%%%%%%%%%%%%%%%%%%%%%%%%%%%%%%%%%%%%%%%%%%%%%%%%%%%%%%%%%%%%%%%%%%%%%%%%%%%%%%%%%%%%%%%%%%%%%%%%%%%%%%%%%%%%%%%%%%%%%%%%%%%%%%%%%%%%%%%%%%%%%%%%

\usepackage{eurosym}
\usepackage{vmargin}
\usepackage{amsmath}
\usepackage{graphics}
\usepackage{epsfig}
\usepackage{enumerate}
\usepackage{multicol}
\usepackage{subfigure}
\usepackage{fancyhdr}
\usepackage{listings}
\usepackage{framed}
\usepackage{graphicx}
\usepackage{amsmath}
\usepackage{chngpage}

%\usepackage{bigints}
\usepackage{vmargin}

% left top textwidth textheight headheight

% headsep footheight footskip

\setmargins{2.0cm}{2.5cm}{16 cm}{22cm}{0.5cm}{0cm}{1cm}{1cm}

\renewcommand{\baselinestretch}{1.3}

\setcounter{MaxMatrixCols}{10}

\begin{document}

\begin{enumerate}
%%%%%%%%%%%%%%%%%%%%%%%%%


\item 
A statistician is told that one of two dice has been chosen and rolled, and he is told the result of the roll. One dice is a conventional dice, but the other has three sides numbered 2 and three sides numbered 4. If the statistician correctly identifies the dice he wins a prize of 1.
\begin{enumerate}
\item (i) Determine the total number of decision functions available to the statistician.'
\item 
(ii) (a)
Identify the most natural candidate for the decision function.
\item 
Calculate the expected payoff for this function assuming that each of
the two dice are equally likely to be chosen.
\end{enumerate}
\end{enumerate}
\newpage


%%%%%%%%%%%%%%%%%%%%%%%%%
2
\begin{enumerate}
\item (i)
The decision function must nominate a choice of die for each potential
outcome from the observation.
There are 6 possible outcomes from the die roll and hence $2 \times  2 \times  2 \times  2 \times  2 \times 
2 = 64$ possible decision functions.
\item (ii)
The most natural candidate is to nominate the conventional die on rolls of
1,3,5,6 and the special die on rolls of 2 or 4.
The expected payoff from this approach is:
2 ⎞
⎛ 4
0.5 \times  ⎜ \times  1 + \times  0 ⎟ + 0.5 \times  1 = 0.83333
6 ⎠
⎝ 6
\end{enumerate}
% Candidates who understood what a decision function is scored well. However, many
% candidates struggled to make any headway with this question. For part (i) some candidates
% observed that if the dice roll is 1,3,5 or 6 it is obvious that you must chose the conventional
% dice. Therefore a choice is only needed on a roll of a 2 or a 4 giving a total of 2 \times  2 = 4
% functions. This was given full credit if carefully explained.
\end{document}
