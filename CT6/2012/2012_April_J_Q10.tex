\documentclass[a4paper,12pt]{article}

%%%%%%%%%%%%%%%%%%%%%%%%%%%%%%%%%%%%%%%%%%%%%%%%%%%%%%%%%%%%%%%%%%%%%%%%%%%%%%%%%%%%%%%%%%%%%%%%%%%%%%%%%%%%%%%%%%%%%%%%%%%%%%%%%%%%%%%%%%%%%%%%%%%%%%%%%%%%%%%%%%%%%%%%%%%%%%%%%%%%%%%%%%%%%%%%%%%%%%%%%%%%%%%%%%%%%%%%%%%%%%%%%%%%%%%%%%%%%%%%%%%%%%%%%%%%

\usepackage{eurosym}
\usepackage{vmargin}
\usepackage{amsmath}
\usepackage{graphics}
\usepackage{epsfig}
\usepackage{enumerate}
\usepackage{multicol}
\usepackage{subfigure}
\usepackage{fancyhdr}
\usepackage{listings}
\usepackage{framed}
\usepackage{graphicx}
\usepackage{amsmath}
\usepackage{chng%%-- Page}

%\usepackage{bigints}
\usepackage{vmargin}

% left top textwidth textheight headheight

% headsep footheight footskip

\setmargins{2.0cm}{2.5cm}{16 cm}{22cm}{0.5cm}{0cm}{1cm}{1cm}

\renewcommand{\baselinestretch}{1.3}

\setcounter{MaxMatrixCols}{10}

\begin{document}
Let X 1 and X 2 be random variables with moment generating functions M X 1 ( t ) and
M X 2 ( t ) respectively. A new random variable Y is formed by choosing a sample
from X 1 with probability p or a sample from X 2 with probability 1 − p .
(i)
Show that the moment generating function of Y is given by
M Y ( t ) = pM X 1 ( t ) + ( 1 − p ) M X 2 ( t )
[2]
A portfolio of insurance policies consists of two different types of policy. Claims on
type 1 policies arrive according to a Poisson process with parameter \lambda  1 and claim
amounts have a distribution X 1 . Claims on type 2 policies arrive according to a
Poisson process with parameter \lambda  2 and claim amounts have a distribution X 2 .
(ii)
Show that aggregate claims on the whole portfolio follow a compound Poisson
distribution, specifying the claim rate and the claim size distribution.
[6]
Now suppose that \lambda  1 = 10 and \lambda  2 = 15 and that the claim sizes are exponentially
distributed with mean 50 for type 1 policies and mean 70 for type 2 policies.
(iii)
CT6 A2012–5
Construct an algorithm for simulating total claims on the whole portfolio. [6]
[Total 14]

%%%%%%%%%%%%%%%%%%%%%%%%%%%%%%%%%%%%%%%%%%%%%%%%%%%%%%%%%%%%%%%%%%%%%%%%%%%%%%%%%%%%%%%%%%%%%%%%%%%%%%%%%%%%%
\newpage

10
(i)
( )
( )
( )
M Y ( t ) = E e tY = pE e tX 1 + ( 1 − p ) E e tX 2
= pM X 1 ( t ) + ( 1 − p ) M X 2 ( t )
(ii)
Let S 1 , S 2 denote aggregate claims on the type 1 and type 2 policies
respectively, and let N 1 , N 2 denote the number of claims from type 1 and type
2 policies respectively. Let S = S 1 + S 2 denote the aggregate claims on the
combined portfolio. We know that S 1 , S 2 follow compound Poisson processes
and so
M S i ( t ) = M N i (log M X i ( t )) = exp( \lambda  i ( M X i ( t ) − 1))
Now
M S ( t ) = M S 1 + S 2 ( t ) = M S 1 ( t ) M S 2 ( t )
( (
) )
= exp \lambda  1 M X 1 ( t ) − 1 exp( \lambda  2 ( M X 2 ( t ) − 1))
⎡
⎛ \lambda  1
⎞ ⎤
\lambda  2
= exp ⎢ ( \lambda  1 + \lambda  2 ) ⎜
M X ( t ) +
M X ( t ) − 1 ⎟ ⎥
1
2
\lambda  1 + \lambda  2
⎝ \lambda  1 + \lambda  2
⎠ ⎦
⎣
= exp(( \lambda  1 + \lambda  2 )( pM X 1 ( t ) + ( 1 − p ) M X 2 ( t ) − 1) where p =
\lambda  1
\lambda  1 + \lambda  2
= exp(( \lambda  1 + \lambda  2 )( M Y ( t ) − 1))
Where Y is defined as in part (i). This is of the form M N ( logM Y ( t ) ) where N
is a Poisson distribution with parameter \lambda  1 + \lambda  2 . Hence S has a compound
Poisson distribution with rate \lambda  1 + \lambda  2 and where individual claim amounts are
\lambda  1
taken from distribution X 1 with probability p =
and from distribution
\lambda  1 + \lambda  2
\lambda  2
X 2 with probability 1 − p =
.
\lambda  1 + \lambda  2
Page 14Subject CT6 (Statistical Methods) – April 2012 – Examiners’ Report
(iii)
Step 1
We first begin by generating a random sample from N ~ P ( 25 ) as follows:
Let u be a random sample from a Uniform distribution on (0,1).
Find the positive integer i such that P ( N \leq  i − 1 ) < u \leq  P ( N \leq  i ) (using the
cumulative Poisson tables)
Then i is the simulated number of claims.
Step 2
Now we simulate the individual claim amount
Generate v a sample from a Uniform distribution on (0,1).
10
10
=
= 0.4 then we have a type 1 claim otherwise we have a
10 + 15 25
type 2 claim. Let the claim type be j.
If v \leq 
Put \mu  1 = 50 and \mu  2 = 70 . Generate w a sample from a uniform distribution on
(0,1).
The simulated claim Z is given by setting
F X j ( Z ) = w
⎛ 1 ⎞
So 1 − exp ⎜
Z ⎟ = w
⎜ \mu 
⎟
⎝ j ⎠
So Z = − \mu  j ln( 1 − w )
Step 3
Repeat Step 2 for a total of i samples and add the results.
Alternative algorithm: simulate the two results separately and add together at
the end.
This question was not answered well. In particular, many candidates did not attempt part
(iii). Of those that did, most had a good attempt at step 2, but very few got step 1 (to deduce
the simulated number of claims).

\end{document}
