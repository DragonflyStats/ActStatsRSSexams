\documentclass[a4paper,12pt]{article}

%%%%%%%%%%%%%%%%%%%%%%%%%%%%%%%%%%%%%%%%%%%%%%%%%%%%%%%%%%%%%%%%%%%%%%%%%%%%%%%%%%%%%%%%%%%%%%%%%%%%%%%%%%%%%%%%%%%%%%%%%%%%%%%%%%%%%%%%%%%%%%%%%%%%%%%%%%%%%%%%%%%%%%%%%%%%%%%%%%%%%%%%%%%%%%%%%%%%%%%%%%%%%%%%%%%%%%%%%%%%%%%%%%%%%%%%%%%%%%%%%%%%%%%%%%%%

\usepackage{eurosym}
\usepackage{vmargin}
\usepackage{amsmath}
\usepackage{graphics}
\usepackage{epsfig}
\usepackage{enumerate}
\usepackage{multicol}
\usepackage{subfigure}
\usepackage{fancyhdr}
\usepackage{listings}
\usepackage{framed}
\usepackage{graphicx}
\usepackage{amsmath}
\usepackage{chng%%-- Page}

%\usepackage{bigints}
\usepackage{vmargin}

% left top textwidth textheight headheight

% headsep footheight footskip

\setmargins{2.0cm}{2.5cm}{16 cm}{22cm}{0.5cm}{0cm}{1cm}{1cm}

\renewcommand{\baselinestretch}{1.3}

\setcounter{MaxMatrixCols}{10}

\begin{document}


4
%----------%
Claims on a particular type of insurance policy follow a compound Poisson process with annual claim rate per policy 0.2. Individual claim amounts are exponentially distributed with mean 100. In addition, for a given claim there is a probability of 30%
that an extra claim handling expense of 30 is incurred (independently of the claim size). The insurer charges an annual premium of 35 per policy.
Use a normal approximation to estimate how many policies the insurer must sell so that the insurer has a 95\% probability of making a profit on the portfolio in the year.
%%%%%%%%%%%%%%%%%%%%%%%%%%%%%%%%%%%%%%%%%%%%%%%%%%%%%%%%%%%%%%

\newpage

4
\begin{itemize}
\item Let the individual total claim costs be denoted by X. 
Then $X=Y+Z$ where Y is the cost of the claim and Z is the claim handling expense.
Then
\[E ( X ) = E ( Y ) + E ( Z ) = 100 + 0.3 \times 30 = 109\]
And
( ) (
) ( )
E X 2 = E Y 2 + 2 YZ + Z 2 = E Y 2 + 2 E ( Y ) E ( Z ) + E ( Z 2 )
\item  Using the independence of Y and Z. Now
( )
E Y 2 = 2 E ( Y ) 2 = 2 \times 100 2 = 20000
and
( )
E Z 2 = 0.3 \times 30 2 = 270
So that
\[
E X 2 = 20000 + 2 \times 100 \times 9 + 270 = 22070 = 148.56 2\]
\item  Now if there are n policies in the portfolio, total claim amounts S will have an approximately Normal distribution with mean $0.2 \times n \times 109 = 21.8 n$ and variance 0.2 \times n \times 148.56 2 .
The premium income will be 35n.

\item  We need to solve for n in the following equation:
( (
)
)
P N 21.8 n , 66.44 2 n > 35 n < 0.05
i.e.
13.2 n ⎞
⎛
P ⎜ N ( 0,1 ) >
⎟ < 0.05
66.44 n ⎠
⎝
% %%-- Page 5
%%%%%%%%%%%%%%%%%%%%%%%%%
% (Statistical Methods) – April 2012 – Examiners’ Report
\item  So
0.198675496 n > 1.6449
n > 68.55
i.e. at least 69 policies must be sold.
\item  Most candidates struggled with this question. Many did not calculate the variance correctly and a lot did not correctly use the number of policies, n, as a multiplier for the mean and variance of the claims. Others used n and the claim rate when calculating the additional
claim handling expense.
\end{itemize}
\end{document}
