\documentclass[a4paper,12pt]{article}

%%%%%%%%%%%%%%%%%%%%%%%%%%%%%%%%%%%%%%%%%%%%%%%%%%%%%%%%%%%%%%%%%%%%%%%%%%%%%%%%%%%%%%%%%%%%%%%%%%%%%%%%%%%%%%%%%%%%%%%%%%%%%%%%%%%%%%%%%%%%%%%%%%%%%%%%%%%%%%%%%%%%%%%%%%%%%%%%%%%%%%%%%%%%%%%%%%%%%%%%%%%%%%%%%%%%%%%%%%%%%%%%%%%%%%%%%%%%%%%%%%%%%%%%%%%%

\usepackage{eurosym}
\usepackage{vmargin}
\usepackage{amsmath}
\usepackage{graphics}
\usepackage{epsfig}
\usepackage{enumerate}
\usepackage{multicol}
\usepackage{subfigure}
\usepackage{fancyhdr}
\usepackage{listings}
\usepackage{framed}
\usepackage{graphicx}
\usepackage{amsmath}
\usepackage{chngpage}

%\usepackage{bigints}
\usepackage{vmargin}

% left top textwidth textheight headheight

% headsep footheight footskip

\setmargins{2.0cm}{2.5cm}{16 cm}{22cm}{0.5cm}{0cm}{1cm}{1cm}

\renewcommand{\baselinestretch}{1.3}

\setcounter{MaxMatrixCols}{10}

\begin{document}


3

An actuary needs to generate samples from the standard normal distribution for use in
a simulation model he is constructing.
(i) Describe the polar algorithm for generating pairs of samples from the standard
normal distribution given pairs of samples from a uniform distribution on
[0,1].

(ii) Calculate the probability that a pair of samples from a uniform distribution on
[0,1] results in an acceptable pair of samples from the standard normal
distribution under the algorithm in (i).

[Total 6]
CT6 S2012–24


%%%%%%%%%%%%%%%%%%%%%%%%%%%%%%%%%%%%%%%%%%%%%%%%%%%%%%%%%%%%%%%%%%%%%%%%%%
3
(i)
Polar algorithm:
(1) Generate independently U 1 and U 2 from U ( 0,1 )
(2)
(3) Set V 1 = 2 U 1 − 1 , V 2 = 2 U 2 − 1 and S = V 1 2 + V 2 2
If S > 1 go to step 1
Otherwise set:
Z 1 = −
(ii)
2 ln S
2 ln S
V 1 and Z 2 = −
V 2
S
S
The acceptance probability is obtained from the condition S < 1 . So the
required probability is obtained as P ( V 1 2 + V 2 2 < 1) where V i are independently
drawn from U ( − 1,1 ) .
Simple geometrical arguments show that the required probability is equivalent
to the event that a uniform draw from the points of the square defined by
V 1 \in  [ − 1,1] and V 2 \in  [ − 1,1] falls within the circle with centre at the origin of
coordinates (0, 0) , and radius 1.
The probability of this event is equivalent to the ratios of the areas:
P =
π 1 2
2
2
=
π
= 0.7854 .
4
Part (i) was mostly well answered, though some candidates lost marks as they did not specify
how to transform U(0,1) random samples into U( − 1,1) random samples. Very few candidates
adopted the geometric approach in (ii).
