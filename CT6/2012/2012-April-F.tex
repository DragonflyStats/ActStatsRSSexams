\documentclass[a4paper,12pt]{article}

%%%%%%%%%%%%%%%%%%%%%%%%%%%%%%%%%%%%%%%%%%%%%%%%%%%%%%%%%%%%%%%%%%%%%%%%%%%%%%%%%%%%%%%%%%%%%%%%%%%%%%%%%%%%%%%%%%%%%%%%%%%%%%%%%%%%%%%%%%%%%%%%%%%%%%%%%%%%%%%%%%%%%%%%%%%%%%%%%%%%%%%%%%%%%%%%%%%%%%%%%%%%%%%%%%%%%%%%%%%%%%%%%%%%%%%%%%%%%%%%%%%%%%%%%%%%

\usepackage{eurosym}
\usepackage{vmargin}
\usepackage{amsmath}
\usepackage{graphics}
\usepackage{epsfig}
\usepackage{enumerate}
\usepackage{multicol}
\usepackage{subfigure}
\usepackage{fancyhdr}
\usepackage{listings}
\usepackage{framed}
\usepackage{graphicx}
\usepackage{amsmath}
\usepackage{chngpage}

%\usepackage{bigints}
\usepackage{vmargin}

% left top textwidth textheight headheight

% headsep footheight footskip

\setmargins{2.0cm}{2.5cm}{16 cm}{22cm}{0.5cm}{0cm}{1cm}{1cm}

\renewcommand{\baselinestretch}{1.3}

\setcounter{MaxMatrixCols}{10}

\begin{document}

\begin{enumerate}
%%%%%%%%%%%%%%%%%%%%%%%%%%%%%%%%%%%%%%%%%%%%%%%%%%
11
Claims on a portfolio of insurance policies arrive as a Poisson process with parameter
100. Individual claim amounts follow a normal distribution with mean 30 and
variance 5 2 . The insurer calculates premiums using a premium loading of 20% and
has initial surplus of 100.
\begin{enumerate}
\item (i) Define carefully the ruin probabilities ψ (100) , ψ (100,1) and ψ 1 (100,1) .
[3]
\item (item) Define the adjustment coefficient R.
[1]
\item  Show that for this portfolio the value of R is 0.011 correct to 3 decimal places.
[5]
\item  Calculate an upper bound for ψ ( 100 ) and an estimate of ψ 1 (100,1) .
\item Comment on the results in (iv).
\end{enumerate}

%%%%%%%%%%%%%%%%%%%%%%%%%%%%%%%%%%%%%%%%%%%%%%%%%%%%%%%%%%%%%%%%%%%%%%%%%%%%%%%%%%%%%%%%%%%%%%%%%%%%
11
(i)
Let S ( t ) denote cumulative claims to time t. Let the annual rate of premium
income be c and let the insurer’s initial surplus be U=100.
Then the surplus at time t is given by:
U ( t ) = U + ct − S ( t )
And the relevant probabilities are defined by:
ψ ( 100 ) = P ( U ( t ) < 0 for some t > 0)
ψ ( 100,1 ) = P ( U ( t ) < 0 for some t with 0 < t ≤ 1)
ψ 1 ( 100,1 ) = P ( U ( 1 ) < 0)
(item)
The adjustment coefficient is the unique positive root of the equation
λ M X ( R ) = λ + cR
Where λ is the rate of the Poisson process (i.e. 100) and X is the normal
distribution with mean 30 and standard deviation 5.
(itemi)
In this case we have:
c = 100 × 30 × 1.2 = 3600
And
M X ( R ) = exp(30 R + 12.5 R 2 )
So R is the root of
(
)
100 exp 30 R + 12.5 R 2 − 100 − 3600 R = 0
Denote the left hand side of this equation by f(R).
When R = 0.0115 we have
f ( 0.0115 ) = 100 exp ( 0.346653125 ) − 100 − 41.4 = 0.032604592 > 0
And when R = 0.0105 we have
f ( 0.0105 ) = 100 exp ( 0.316378125 ) − 100 − 37.8 = − 0.585099862 < 0
Page 16
%%%%%%%%%%%%%%%%%%%%%%%%%
 (Statistical Methods) – April 2012 – Examiners’ Report
Since the function changes sign between 0.0105 and 0.0115 the unique
positive root must lie between these values and hence the root is 0.011 correct
to 3 decimal places.
(iv)
By Lundberg’s inequality ψ ( 100 ) < exp ( − 100 × 0.011 ) = 0.33287
Claims in the first year are approximately Normal, with mean 100 × 30 = 3000
(
)
And variance given by 100 × 25 + 30 2 = 92500
So approximately
ψ 1 ( 100,1 ) = P (100 + 3600 − N ( 3000,92500 ) < 0)
3700 − 3000 ⎞
⎛
= P ( N ( 3000,92500 ) > 3700 ) = P ⎜ N ( 0,1 ) >
⎟
92500 ⎠
⎝
= P ( N ( 0,1 ) > 2.302)
= 1 − ( 0.98928 × 0.8 + 0.98956 × 0.2 )
= 0 . 0107 .
(v)
The probability of ruin is much smaller in the first year than the long-term
bound provided by Lundberg’s inequality. This suggests that either the bound
in Lundberg’s inequality may not be that tight or that there is significant
probability of ruin at times greater than 1 year.
In part (i) many candidates lost straightforward marks by failing to give sufficiently precise
definitions. In particular, many candidates gave solutions along the lines of P(U(t)<0, t>0).
It isn’t clear whether this refers to all positive values of t or just some positive value.
Most candidates got part (item).
For part (itemi), many candidates were able to show that when R=0.011 the two sides of the
equation are approximately equal. Very few were able to give a precise demonstration that
the root is at R=0.011 by considering where the curve cross the axis. Candidates for future
exams should note this technique carefully.
For part (iv) most candidates got the upper bound for R.
Part (v) was well answered by stronger candidates.
END OF EXAMINERS’ REPORT
Page 17
