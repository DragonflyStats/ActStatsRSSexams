
5
The cumulative claim amounts incurred on a portfolio of motor insurance policies are
as follows:
Accident Year
Development Year
0 1 2 3
2014 3,907 5,606 6,061 6,145
2015 4,831 7,319 7,470 2016 6,042 8,282 2017 7,061
The cumulative number of reported claims are as follows
Accident Year
(i)
Development Year
0 1 2 3
2014 435 469 528 534
2015 485 525 541 2016 509 558 2017 544
Estimate the ultimate number of claims, for each accident year, using the
chain-ladder technique.
[4]
(ii) Estimate the ultimate average incurred cost per claim, for each accident year,
using the grossing-up method.
[5]
(iii) Calculate the total reserve required, using the results from (i) and (ii),
assuming that claims paid to date are 19,544.
[2]
[Total 11]

CT6 S2018–3 
PLEASE TURN OVER6
In a two-player zero sum game, the matrix below shows the value to Player 1. Player
1’s strategies are labelled I to VI, where Player 2’s strategies are labelled A to F.
A B C D E F
I 13 29 8 12 16 23
II 18 22 21 22 29 31
III 18 22 31 31 27 37
IV 11 22 12 21 21 26
V 18 16 19 14 19 28
VI 23 22 19 23 30 34
(i)
(ii)
Show, by eliminating dominated strategies, that the game can be reduced to the
following 3 x 3 matrix.
[3]
a b c
α 13 29 8
β 18 22 31
γ 23 22 19
Explain whether or not this new 3 x 3 matrix has any saddle points.
[2]
Now consider a randomised strategy for Player 2, denoted X, whereby strategy ‘a’ is
chosen with probability p and strategy ‘c’ is chosen with probability 1 – p, 0 < p < 1.
(iii) Find the range of values for p such that X dominates strategy ‘b’.
(iv) Solve the game and determine the value to Player 1 given that ‘b’ is
dominated.[3]
[Total 11]

%%%%%%%%%%%%%%%%%%%%%%%%%%%%%%%%%%%%%%%%%%%%%%%%%%%%%%%%%%%%%%%%%%%%%%%%%%%%%%%%
Q5
(i)
df1 = 534 / 528 = 1.011 364 ...
df2 = (528+541)/(469+525) = 1.075 453 ...
df3 = (469 + 525 + 558) / (435 + 485 + 509) = 1.086 074 ...
Accident
Year
2014
2015
2016
2017
[1]
[1⁄2]
[1⁄2]
Development Year
0
435
485
509
544
1
469
525
558
2
528
541
3
534
547.15
606.92
642.62
[2]
(ii)
Cost per claim:
Accident
Year
2014
2015
2016
2017
Development Year
0
8.982
9.961
11.870
12.980
1
11.953
13.941
14.842
2
11.479
13.808
3
11.507
[2]
Grossing up table:
Accident
Year
0
2014
78.050%
2015
71.962%
2016
81.811%
2017
77.274%
Development Year
1
103.872%
100.716%
102.294%
2
99.754%
99.754%
3
100%
[2]
Page 5Subject CT6 (Statistical Methods Core Technical) – September 2018 – Examiners’ Report
Ultimate average cost per claim:
Accident
Development Year
Year
0
1
2
2014
8.982
11.953
11.479
2015
9.961
13.941
13.808
2016
11.870
14.842
2017
12.980
3
11.507
13.842
14.509
16.797
[1]
Ultimate total claim amounts:
Accident
Year
0
2014
3,907
2015
4,831
2016
6,042
2017
7,061
Development Year
1
5,606
7,319
8,282
2
6,061
7,470
3
6,145
7,574
8,806
10,794
[1]
Outstanding = (6145 +7574 + 8806 + 10794) – 19544 = 13,775
[1]
[Total 11]
This question posed few problems for well-prepared candidates.
Q6
(i) For Player 1, Strategy VI dominates strategies IV & V
For Player 2, Strategy C dominates Strategy F ...
... and Strategies D & E with IV & V eliminated
With D, E & F eliminated, Strategy III dominates Strategy II
(ii) There are no saddle points, since the maximum in any column is never equal
to the minimum of any row.
[2]
(iii) ‘b’ is always dominated for α.
[1⁄2]
For β, we require 18p + 31(1 – p) < 22
For γ, we require 23p + 19(1 – p) < 22
So 9/13 < p < 3⁄4
[1]
[1⁄2]
[1]
(iv)
[1]
[1⁄2]
[1]
[1⁄2]
Now α is also dominated for Player 1, so we are left with
a
c
β
18
31
γ
23
19
[1]
Hence Player 2 should choose a randomised strategy with
18p + 31(1 – p) = 23p + 19(1 – p)
p = 12/17 (i.e. choose strategy a with probability 12/17)
with value 21.8
[1]
[1⁄2]
[1⁄2]
Page 6Subject CT6 (Statistical Methods Core Technical) – September 2018 – Examiners’ Report
[Total 11]
Performance in this question was mixed, with many candidates dropping marks in part (i) for
insufficient explanation of their answer; and fewer candidates were able to score well in
parts (iii) & (iv).
