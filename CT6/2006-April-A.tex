
1
April 2006
Examiners Report                   
1 List the main perils typically insured against under a household buildings policy.
2 A No Claims Discount (NCD) system has 3 levels of discount
Level 0
Level 1
Level 2
[3]
no discount
discount = p
discount = 2p
where 0 < p < 0.5.
The probability of a policyholder NOT making a claim each year is 0.9.
In the event of a claim, the policyholder moves to, or remains at level 0. Otherwise,
the policyholder moves to the next higher level (or remains at level 2).
The premium paid in level 0 is £1,000.
Derive an expression in terms of p for the average premium paid by a policyholder
once the steady state has been reached.
[6]
Fire
Flood
Storm
Theft
Explosions
Lightning
Damage caused by measures taken to put out a fire.
Comments on question 1: This straightforward bookwork question was poorly done with
relatively few candidates scoring full marks. Credit was given for any other reasonable
suggestion not included on the list above.
%%%%%%%%%%%%%%%%%%%%%%%%%%%%%%%%%%%%%%%%%%%%%%%%%%%%%%%%%%%%%%%%%%%%%%%%%%%%%%%%%%%%%%%%%%%%%%%%%%%%%%%%%%%%%%%%
2
The transition matrix is
0.1 0.9
0.1
0.1
0.1 (
0.9
0
0
=
+
0
0
0
2 )
0.9
0.9
1 +
=
0 = 0.1
1 = 0.09
2 = 0.81
0
1
Average premium paid is
[0.1 + 0.09(1 p) + 0.81(1
= [1 1.62p]
0.09p
2p)]
1,000
1,000
p]
1,000
Comments on question 2: Most candidates obtained full marks. A few incorrectly identified
the transition matrix or failed to solve the simultaneous equations.
Page 2Subject CT6 (Statistical Methods Core Technical)
%%%%%%%%%%%%%%%%%%%%%%%%%%%%%%%%%%%%%%%%%%%%%%%%%%%%%%%%%%%%%%%%%%%%%%%%%%%%%%%%%%%%%%%%%%%%%%%%%%%%%%%%%%%%%%%%%%5
