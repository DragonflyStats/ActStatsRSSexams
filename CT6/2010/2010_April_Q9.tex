%%%%%%%%%%%%%%%%%%%%%%%%%%%%%%%%%%%%%%%%%%%%%

\documentclass[a4paper,12pt]{article}

%%%%%%%%%%%%%%%%%%%%%%%%%%%%%%%%%%%%%%%%%%%%%%%%%%%%%%%%%%%%%%%%%%%%%%%%%%%%%%%%%%%%%%%%%%%%%%%%%%%%%%%%%%%%%%%%%%%%%%%%%%%%%%%%%%%%%%%%%%%%%%%%%%%%%%%%%%%%%%%%%%%%%%%%%%%%%%%%%%%%%%%%%%%%%%%%%%%%%%%%%%%%%%%%%%%%%%%%%%%%%%%%%%%%%%%%%%%%%%%%%%%%%%%%%%%%

\usepackage{eurosym}
\usepackage{vmargin}
\usepackage{amsmath}
\usepackage{graphics}
\usepackage{epsfig}
\usepackage{enumerate}
\usepackage{multicol}
\usepackage{subfigure}
\usepackage{fancyhdr}
\usepackage{listings}
\usepackage{framed}
\usepackage{graphicx}
\usepackage{amsmath}
\usepackage{chng%%-- Page}
%\usepackage{bigints}
\usepackage{vmargin}

% left top textwidth textheight headheight

% headsep footheight footskip
\setmargins{2.0cm}{2.5cm}{16 cm}{22cm}{0.5cm}{0cm}{1cm}{1cm}
\renewcommand{\baselinestretch}{1.3}
\setcounter{MaxMatrixCols}{10}
\begin{document} 


%%% ---- Question 9

On the tapas menu in a local Spanish restaurant customers can order a dish of 20 roasted chillies for £5. There is always a mixture of hot chillies and mild chillies on
the dish and these cannot be distinguished except by taste. The restaurant produces two types of dish: one containing 4 hot and 16 mild chillies and one containing 8 hot
and 12 mild chillies. When the dish is served, the waiter allows the customer to taste one chilli and then offers a 50% discount to customers who correctly guess whether
the dish contains 4 hot chillies or 8 hot chillies.

A hungry actuary who regularly visits the restaurant is trying to work out the best
strategy for guessing the number of hot chillies.
(i) List the four possible decision functions the actuary could use.
(ii) Calculate the values of the risk function for the two different chilli dishes and each decision function.
(iii) Determine the optimum strategy for the actuary using the Bayes criterion and work out the average price he will pay for a dish of chillies if the restaurant
produces equal numbers of the two types of dish.

\newpage

%%%%%%%%%%%%%%%%%%%%%%%%%%%%%%%%%%%%%%%%%%%%%%%%%%%%%%%%%%%%%%%%%%%%%%%%%%%%%%%%%%%%%%%%%%%%%
9
(i)
The possibilities are (where H denotes trying a hot chilli and M denotes trying
a mild chilli)
d 1 ( H ) = 4 H and d 1 ( M ) = 4 H
d 2 ( H ) = 4 H and d 2 ( M ) = 8 H
d 3 ( H ) = 8 H and d 3 ( M ) = 4 H
d 4 ( H ) = 8 H and d 4 ( M ) = 8 H
(ii)
Page 8
Under 4H we have P ( H ) = 0.2 and P ( M ) = 0.8
Under 8H we have P ( H ) = 0.4 and P ( M ) = 0.6

%%%%%%%%%%%%%%%%%%%%%%%%%%%%%%%%%%%%%%%% — April 2010 — Examiners’ Report
We can present the game so that the loss to the actuary is what he has to pay
for the plate of chillis (i.e. the loss is either 2.5 or 5). Under this approach we
have:
R ( d 1 , 4 H ) = P ( H 4 H ) \times L ( d 1 ( H ), 4 H ) + P ( M 4 H ) \times L ( d 1 ( M ), 4 H ) = 0.2 \times 2.5 + 0.8 \times 2.5 = 2.5
R ( d 1 ,8 H ) = P ( H 8 H ) \times L ( d 1 ( H ),8 H ) + P ( M 8 H ) \times L ( d 1 ( M ),8 H ) = 0.4 \times 5 + 0.6 \times 5 = 5
R ( d 2 , 4 H ) = P ( H 4 H ) \times L ( d 2 ( H ), 4 H ) + P ( M 4 H ) \times L ( d 2 ( M ), 4 H ) = 0.2 \times 2.5 + 0.8 \times 5 = 4.5
R ( d 2 ,8 H ) = P ( H 8 H ) \times L ( d 2 ( H ),8 H ) + P ( M 8 H ) \times L ( d 2 ( M ),8 H ) = 0.4 \times 5 + 0.6 \times 2.5 = 3.5
R ( d 3 , 4 H ) = P ( H 4 H ) \times L ( d 3 ( H ), 4 H ) + P ( M 4 H ) \times L ( d 3 ( M ), 4 H ) = 0.2 \times 5 + 0.8 \times 2.5 = 3
R ( d 3 ,8 H ) = P ( H 8 H ) \times L ( d 3 ( H ),8 H ) + P ( M 8 H ) \times L ( d 3 ( M ),8 H ) = 0.4 \times 2.5 + 0.6 \times 5 = 4
R ( d 4 , 4 H ) = P ( H 4 H ) \times L ( d 4 ( H ), 4 H ) + P ( M 4 H ) \times L ( d 4 ( M ), 4 H ) = 0.2 \times 5 + 0.8 \times 5 = 5
R ( d 4 ,8 H ) = P ( H 8 H ) \times L ( d 4 ( H ),8 H ) + P ( M 8 H ) \times L ( d 4 ( M ),8 H ) = 0.4 \times 2.5 + 0.6 \times 2.5 = 2.5
(iii)
The payoff matrix for the player is:
4H
8H
Expected Loss
d 1 d 2 d 3 d 4
2.5
5.0
3.75 4.5
3.5
4.0 3
4
3.5 5
2.5
3.75
So the Bayes criterion strategy is d 3 .
Under this approach, the average price paid is £3.50.
Comment: Alternative solutions are possible here. The question was not answered as well as
those relating to the same material in previous years. In particular, many weaker candidates
were unable to fully specify the possible decision functions and therefore made little headway
with this question.

