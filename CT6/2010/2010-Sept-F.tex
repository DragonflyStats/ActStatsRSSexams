PLEASE TURN OVER11
A time series model is specified by
Y t = 2 \alpha Y t − 1 − \alpha 2 Y t − 2 + e t
where e t is a white noise process with variance \sigma 2 .
(i) Determine the values of \alpha for which the process is stationary.
(ii) Derive the auto-covariances \gamma 0 and \gamma 1 for this process and find a general
recursive expression for \gamma k for k ≥ 2.
(iii)

[10]
Show that the auto-covariance function can be written in the form:
\gamma k = A \alpha k + kB \alpha k
for some values of A, B which you should specify in terms of the constants \alpha
and \sigma 2 .

[Total 17]
END OF PAPER
CT6 S2010—6

%%%%%%%%%%%%%%%%%%%%%%%%%%%%%%%%%%%%%%%%%%%%%%%%%%%%%%%%%%%%%%%%%%%%%%%%%%%%%%%%%%%%%%%%
11
(i)
Let B be the backward shift operator. Then the time series has the form:
(1 − 2 \alpha B + \alpha 2 B 2 ) Y t = e t
(1 − \alpha B ) 2 Y t = e t
And the roots of the characteristic equation will have modulus greater than 1
ands so the series will be stationary provided that \alpha < 1 .
%%-- Page 11%%%%%%%%%%%%%%%%%%%%%%%%%%%%%%%%%%%%%%%%%%%%% — September 2010 — Examiners’ Report
(ii)
Firstly, note that Cov ( Y t , e t ) = Cov ( e t , e t ) = \sigma 2
So, taking the covariance of the defining equation with Y t we get:
\gamma 0 = 2 \alpha\gamma 1 − \alpha 2 \gamma 2 + \sigma 2 (A)
Taking the covariance with Y t − 1 we get
\gamma 1 = 2 \alpha\gamma 0 − \alpha 2 \gamma 1
i.e. (1 + \alpha 2 ) \gamma 1 = 2 \alpha\gamma 0
(B)
Finally, taking the covariance with Y t − 2 gives:
\gamma 2 = 2 \alpha\gamma 1 − \alpha 2 \gamma 0 (C)
In general, for k ≥ 2 we have \gamma k = 2 \alpha\gamma k − 1 − \alpha 2 \gamma k − 2
Substituting the expression for \gamma 2 in (C) into (A) gives:
\gamma 0 = 2 \alpha\gamma 1 − \alpha 2 (2 \alpha\gamma 1 − \alpha 2 \gamma 0 ) + \sigma 2
So that
(1 − \alpha 4 ) \gamma 0 = 2 \alpha (1 − \alpha 2 ) \gamma 1 + \sigma 2
And now substituting the expression for \gamma 1 in (B) we get
(1 − \alpha 4 ) \gamma 0 = 2 \alpha (1 − \alpha 2 ) \times
2 \alpha\gamma 0
2
(1 + \alpha )
+ \sigma 2
2
2
⎛
4 4 \alpha (1 − \alpha ) ⎞
2
1
−
\alpha
−
⎜ ⎜
⎟ ⎟ \gamma 0 = \sigma
2
1 + \alpha
⎝
⎠
(1 + \alpha 2 − \alpha 4 − \alpha 6 − 4 \alpha 2 + 4 \alpha 4 ) \gamma 0 = (1 + \alpha 2 ) \sigma 2
So \gamma 0 =
(1 + \alpha 2 )
4
6
(1 − 3 \alpha + 3 \alpha − \alpha )
And so \gamma 1 =
%%-- Page 12
2
\sigma 2 =
(1 + \alpha 2 )
2 3
(1 − \alpha )
\sigma 2
⎛ 2 \alpha (1 + \alpha 2 ) ⎞ 2
2 \alpha
=
\sigma
=
\sigma 2
⎜
⎟
2
2 3
2 ⎟
2 3
⎜
1 + \alpha
(1 − \alpha )
⎝ (1 − \alpha ) (1 + \alpha ) ⎠
2 \alpha\gamma 0%%%%%%%%%%%%%%%%%%%%%%%%%%%%%%%%%%%%%%%%%%%%% — September 2010 — Examiners’ Report
And more generally \gamma k = 2 \alpha\gamma k − 1 − \alpha 2 \gamma k − 2 (D)
(iii)
Suppose \gamma k − 1 = A \alpha k − 1 + ( k − 1) B \alpha k − 1 and \gamma k − 2 = A \alpha k − 2 + ( k − 2) B \alpha k − 2 and
substitute into (D).
\gamma k = 2 \alpha A \alpha k − 1 + 2 \alpha ( k − 1) B \alpha k − 1 − \alpha 2 A \alpha k − 2 − ( k − 2) \alpha 2 B \alpha k − 2
= A ( 2 \alpha k − \alpha k ) + B ( 2 \alpha k ( k − 1 ) − ( k − 2 ) \alpha k ) = A \alpha k + Bk \alpha k
Which is of the correct form, so the general form of the expression holds.
Setting k = 0 we get \gamma 0 = A
So A =
(1 + \alpha 2 )
2 3
(1 − \alpha )
\sigma 2
Setting k = 1 we get \gamma 1 = ( A + B ) \alpha
⎛
⎞ 2 (1 + \alpha 2 ) 2 ⎛ 1 − \alpha 2 ⎞ 2
\gamma 1
\sigma 2
2 \alpha
So B = − A = ⎜ ⎜
2 3 ⎟
⎟ \sigma − (1 − \alpha 2 ) 3 \sigma = ⎜ ⎜ (1 − \alpha 2 ) 3 ⎟ ⎟ \sigma = (1 − \alpha 2 ) 2
\alpha
⎝ \alpha (1 − \alpha ) ⎠
⎝
⎠
[Alternatively, solve using the formula on %%-- Page 4 of the Tables:
We have g k = 2 \alpha g k − 1 + \alpha 2 g k − 2 = 0
Using the Tables formula, the roots are \lambda 1 = \lambda 2 = \alpha so we have a solution
of the form g k = ( A + Bk ) \lambda k = ( A + Bk ) \alpha k
Set k = 0 and k = 1 to get the same equations as before.]
Another good differentiator, with strong candidates scoring well, and weaker candidates
struggling with parts (ii) and (iii) in particular.
END OF EXAMINERS’ REPORT
%%-- Page 13
