PLEASE TURN OVER11
A time series model is specified by
Y t = 2 α Y t − 1 − α 2 Y t − 2 + e t
where e t is a white noise process with variance σ 2 .
(i) Determine the values of α for which the process is stationary.
(ii) Derive the auto-covariances γ 0 and γ 1 for this process and find a general
recursive expression for γ k for k ≥ 2.
(iii)
[2]
[10]
Show that the auto-covariance function can be written in the form:
γ k = A α k + kB α k
for some values of A, B which you should specify in terms of the constants α
and σ 2 .
[5]
[Total 17]
END OF PAPER
CT6 S2010—6

%%%%%%%%%%%%%%%%%%%%%%%%%%%%%%%%%%%%%%%%%%%%%%%%%%%%%%%%%%%%%%%%%%%%%%%%%%%%%%%%%%%%%%%%
11
(i)
Let B be the backward shift operator. Then the time series has the form:
(1 − 2 α B + α 2 B 2 ) Y t = e t
(1 − α B ) 2 Y t = e t
And the roots of the characteristic equation will have modulus greater than 1
ands so the series will be stationary provided that α < 1 .
Page 11Subject CT6 (Statistical Methods Core Technical) — September 2010 — Examiners’ Report
(ii)
Firstly, note that Cov ( Y t , e t ) = Cov ( e t , e t ) = σ 2
So, taking the covariance of the defining equation with Y t we get:
γ 0 = 2 αγ 1 − α 2 γ 2 + σ 2 (A)
Taking the covariance with Y t − 1 we get
γ 1 = 2 αγ 0 − α 2 γ 1
i.e. (1 + α 2 ) γ 1 = 2 αγ 0
(B)
Finally, taking the covariance with Y t − 2 gives:
γ 2 = 2 αγ 1 − α 2 γ 0 (C)
In general, for k ≥ 2 we have γ k = 2 αγ k − 1 − α 2 γ k − 2
Substituting the expression for γ 2 in (C) into (A) gives:
γ 0 = 2 αγ 1 − α 2 (2 αγ 1 − α 2 γ 0 ) + σ 2
So that
(1 − α 4 ) γ 0 = 2 α (1 − α 2 ) γ 1 + σ 2
And now substituting the expression for γ 1 in (B) we get
(1 − α 4 ) γ 0 = 2 α (1 − α 2 ) ×
2 αγ 0
2
(1 + α )
+ σ 2
2
2
⎛
4 4 α (1 − α ) ⎞
2
1
−
α
−
⎜ ⎜
⎟ ⎟ γ 0 = σ
2
1 + α
⎝
⎠
(1 + α 2 − α 4 − α 6 − 4 α 2 + 4 α 4 ) γ 0 = (1 + α 2 ) σ 2
So γ 0 =
(1 + α 2 )
4
6
(1 − 3 α + 3 α − α )
And so γ 1 =
Page 12
2
σ 2 =
(1 + α 2 )
2 3
(1 − α )
σ 2
⎛ 2 α (1 + α 2 ) ⎞ 2
2 α
=
σ
=
σ 2
⎜
⎟
2
2 3
2 ⎟
2 3
⎜
1 + α
(1 − α )
⎝ (1 − α ) (1 + α ) ⎠
2 αγ 0Subject CT6 (Statistical Methods Core Technical) — September 2010 — Examiners’ Report
And more generally γ k = 2 αγ k − 1 − α 2 γ k − 2 (D)
(iii)
Suppose γ k − 1 = A α k − 1 + ( k − 1) B α k − 1 and γ k − 2 = A α k − 2 + ( k − 2) B α k − 2 and
substitute into (D).
γ k = 2 α A α k − 1 + 2 α ( k − 1) B α k − 1 − α 2 A α k − 2 − ( k − 2) α 2 B α k − 2
= A ( 2 α k − α k ) + B ( 2 α k ( k − 1 ) − ( k − 2 ) α k ) = A α k + Bk α k
Which is of the correct form, so the general form of the expression holds.
Setting k = 0 we get γ 0 = A
So A =
(1 + α 2 )
2 3
(1 − α )
σ 2
Setting k = 1 we get γ 1 = ( A + B ) α
⎛
⎞ 2 (1 + α 2 ) 2 ⎛ 1 − α 2 ⎞ 2
γ 1
σ 2
2 α
So B = − A = ⎜ ⎜
2 3 ⎟
⎟ σ − (1 − α 2 ) 3 σ = ⎜ ⎜ (1 − α 2 ) 3 ⎟ ⎟ σ = (1 − α 2 ) 2
α
⎝ α (1 − α ) ⎠
⎝
⎠
[Alternatively, solve using the formula on page 4 of the Tables:
We have g k = 2 α g k − 1 + α 2 g k − 2 = 0
Using the Tables formula, the roots are λ 1 = λ 2 = α so we have a solution
of the form g k = ( A + Bk ) λ k = ( A + Bk ) α k
Set k = 0 and k = 1 to get the same equations as before.]
Another good differentiator, with strong candidates scoring well, and weaker candidates
struggling with parts (ii) and (iii) in particular.
END OF EXAMINERS’ REPORT
Page 13
