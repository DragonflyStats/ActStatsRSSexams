\documentclass[a4paper,12pt]{article}

%%%%%%%%%%%%%%%%%%%%%%%%%%%%%%%%%%%%%%%%%%%%%%%%%%%%%%%%%%%%%%%%%%%%%%%%%%%%%%%%%%%%%%%%%%%%%%%%%%%%%%%%%%%%%%%%%%%%%%%%%%%%%%%%%%%%%%%%%%%%%%%%%%%%%%%%%%%%%%%%%%%%%%%%%%%%%%%%%%%%%%%%%%%%%%%%%%%%%%%%%%%%%%%%%%%%%%%%%%%%%%%%%%%%%%%%%%%%%%%%%%%%%%%%%%%%

\usepackage{eurosym}
\usepackage{vmargin}
\usepackage{amsmath}
\usepackage{graphics}
\usepackage{epsfig}
\usepackage{enumerate}
\usepackage{multicol}
\usepackage{subfigure}
\usepackage{fancyhdr}
\usepackage{listings}
\usepackage{framed}
\usepackage{graphicx}
\usepackage{amsmath}
\usepackage{chngpage}
%\usepackage{bigints}
\usepackage{vmargin}

% left top textwidth textheight headheight

% headsep footheight footskip
\setmargins{2.0cm}{2.5cm}{16 cm}{22cm}{0.5cm}{0cm}{1cm}{1cm}
\renewcommand{\baselinestretch}{1.3}
\setcounter{MaxMatrixCols}{10}
\begin{document} 

An office worker receives a random number of e-mails each day. The number of emails per day follows a Poisson distribution with unknown mean $\mu$. Prior beliefs about μ are specified by a gamma distribution with mean 50 and standard deviation
15. The worker receives a total of 630 e-mails over a period of ten days.
Calculate the Bayesian estimate of μ under all or nothing loss.
CT6 S2010—2
[7]5
The table below shows aggregate annual claim statistics for three risks over a period
of seven years. Annual aggregate claims for risk i in year j are denoted by X ij .
X i =
Risk, i
1 7
= ∑ X ij − X i
6 j = 1
(
)
2
335.1
65.1
33.9
(i) Calculate the credibility premium of each risk under the assumptions of EBCT
Model 1.
[6]
(ii) Explain why the credibility factor is relatively high in this case.
[2]
[Total 8]
The probability density function of a gamma distribution is given in the following
parameterised form:
f ( x ) =
(i)
(ii)
7
S i 2
127.9
88.9
149.7
i = 1
i = 2
i = 3
6
1 7
∑ X ij
7 j = 1
α α
μ α Γ ( α )
x
α− 1
e
−
x α
μ
for x > 0.
Express this density in the form of a member of the exponential family,
specifying all the parameters.
[6]
Hence show that the mean and variance of the distribution are given by $\mu$ and
μ 2
respectively.
[3]
α
[Total 9]
%%%%%%%%%%%%%%%%%%%%%%%%%%%%%%%%%%%%%%%%%%%%%%%%%%%%%%%%%%%%%%%%%%%%%%%%%%%%%%%%%%

\newpage

4
Let the prior distribution of $\mu$  have a Gamma distribution with parameters $\alpha$ and $\lambda$ 
as per the tables.

Then
${ displaystyle \frac{a}{\lambda} \;=\; 50 }$ and ${ displaystyle \frac{a}{\lambda^2} \;=\; 15^2 }$

Then dividing the first by the second 

\[ \lambda  = \frac{50}{15^2} = 0.2222\]

And so \[  \alpha = 50 \times 0.2222 = 11.1111 \[


The posterior distribution of $\mu$  is then given by
\begin{eqnarray*}
f ( \mu  x ) &\propto&  f ( x \mu  ) f ( \mu  )\\
&\propto&  e − 10 \mu  ×\mu  630 ×\mu  10.11111 e − 0.22222 \mu \\
&\propto&    \mu  640.11111 e − 10.22222 \mu \\
\end{eqnarray*}

Which is the pdf of a Gamma distribution with parameters $\alpha^{\prime} = 641.1111$ and
$\lambda^{\prime}= 10 . 22222$
Now under all or nothing loss, the Bayesian estimate is given by the mode of the
posterior distribution. So we must find the maximum of
\[f ( x ) = x 640.11111 e − 10.2222 x \](we may ignore constants here)
Differentiating:
\begin{eqnarray*}
f '( x ) &=& e − 10.22222 x \left( − 10.2222 x 640.11111 + 640.1111 x 639.11111 \right) \\
&=& x 639.1111 e − 10.22222 x ( − 10.2222 x + 640.11111)\\
\end{eqnarray*}

And setting this equal to zero we get
x =
640.111111
= 62.62
10.22222
%% Alternatively, credit was given for differentiating the log of the posterior (which is simpler).
%% This question was well answered by most candidates.

\end{document}
