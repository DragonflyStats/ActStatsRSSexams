1
An actuary is using a simulation technique to estimate the probability θ that a claim
on an insurance policy exceeds a given amount. The actuary has carried out 50
simulations and has produced an estimate that θ ˆ = 0.47 . The variance of the
simulated values is 0.15.
Calculate the minimum number of simulations that the actuary will have to perform in
order to estimate θ to within 0.01 with 95% confidence.
[4]
2
Claims on a portfolio of insurance policies follow a compound Poisson process with
annual claim rate λ. Individual claim amounts are independent and follow an
exponential distribution with mean μ. Premiums are received continuously and are
set using a premium loading of θ. The insurer’s initial surplus is U.
Derive an expression for the adjustment coefficient, R, for this portfolio in terms of μ
and θ.
[4]
3
4
An underwriter has suggested that losses on a certain class of policies follow a
Weibull distribution. She estimates that the 10 th percentile loss is 20 and the 90 th
percentile loss is 95.
(i) Calculate the parameters of the Weibull distribution that fit these percentiles.
[3]
(ii) Calculate the 99.5 th percentile loss.
[2]
[Total 5]
%%%%%%%%%%%%%%%%%%%%%%%%%%%%%%%%%%%%%%%%%%%%%%%%%%%%%%%%%%%%%%%%%%%%%%%%%%%%%%%
Institute and Faculty of ActuariesSubject CT6 (Statistical Methods Core Technical) — September 2010 — Examiners’ Report
1
2
We know that, approximately, θ − θ ˆ ≈ N (0, τ ) where τ 2 can be approximated by
n
0.15.
⎡
⎤
θ − θ ˆ
⎢
Then P − 1.96 ≤
≤ 1.96 ⎥ = 0.95
⎢
⎥
0.15
⎢ ⎣
⎥ ⎦
n
And we require 1.96 ×
That is n ≥
1.96 2 × 0.15
0.01 2
0.15
≤ 0.01
n
= 5762.4 i.e. n must be at least 5763
This question was generally poorly answered.
2
The adjustment coefficient satisfies the equation:
λ + λμ (1 + θ ) R = λ M X ( R )
Where X is exponentially distributed with mean μ so that M X ( t ) =
So we have 1 + μ (1 + θ ) R =
1
1 − μ R
and so 1 − μ R + R μ (1 + θ )(1 − μ R ) = 1
−μ R + μ R + μθ R − μ 2 R 2 (1 + θ ) = 0
Dividing through by μ R gives
μ R (1 + θ ) = θ
So R =
θ
.
μ (1 + θ )
This question was well answered by most candidates.
Page 2
1/ μ
1
=
1/ μ − t 1 − μ tSubject CT6 (Statistical Methods Core Technical) — September 2010 — Examiners’ Report
3
(i)
Let the parameters be c and γ as per the tables.
Then we have:
γ
1 − e − c × 20 = 0.1 so
γ
e − c × 20 = 0.9 and so c × 20 γ = − log 0.9 (A)
And similarly c × 95 γ = − log 0.1 (B)
γ
log 0.9
⎛ 20 ⎞
= 0.0457575
(A) divided by (B) gives ⎜ ⎟ =
log 0.1
⎝ 95 ⎠
So γ =
log 0.0457575
= 1.9795337
⎛ 20 ⎞
log ⎜ ⎟
⎝ 95 ⎠
And substituting into (A) we have c = −
(ii)
log 0.9
20 1.9795337
= 0.000280056
The 99.5 th percentile loss is given by
1.9795337
1 − e − 0.00280056 x
= 0.995
So that − 0.000280056 x 1.9795337 = log 0.005
⎛ log 0.005 ⎞
log ⎜
⎟
⎝ − 0.000280056 ⎠ = 4.97486366
log x =
1.9795337
So x = e 4.97486366 = 144.73
Most candidates scored well on this question.
