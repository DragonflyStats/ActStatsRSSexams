%%%%%%%%%%%%%%%%%%%%%%%%%%%%%%%%%%%%%%%%%%%%%

\documentclass[a4paper,12pt]{article}

%%%%%%%%%%%%%%%%%%%%%%%%%%%%%%%%%%%%%%%%%%%%%%%%%%%%%%%%%%%%%%%%%%%%%%%%%%%%%%%%%%%%%%%%%%%%%%%%%%%%%%%%%%%%%%%%%%%%%%%%%%%%%%%%%%%%%%%%%%%%%%%%%%%%%%%%%%%%%%%%%%%%%%%%%%%%%%%%%%%%%%%%%%%%%%%%%%%%%%%%%%%%%%%%%%%%%%%%%%%%%%%%%%%%%%%%%%%%%%%%%%%%%%%%%%%%

\usepackage{eurosym}
\usepackage{vmargin}
\usepackage{amsmath}
\usepackage{graphics}
\usepackage{epsfig}
\usepackage{enumerate}
\usepackage{multicol}
\usepackage{subfigure}
\usepackage{fancyhdr}
\usepackage{listings}
\usepackage{framed}
\usepackage{graphicx}
\usepackage{amsmath}
\usepackage{chng%%-- Page}
%\usepackage{bigints}
\usepackage{vmargin}

% left top textwidth textheight headheight

% headsep footheight footskip
\setmargins{2.0cm}{2.5cm}{16 cm}{22cm}{0.5cm}{0cm}{1cm}{1cm}
\renewcommand{\baselinestretch}{1.3}
\setcounter{MaxMatrixCols}{10}
\begin{document} 



%%- Question 8
%%%%%%%%%%%%%%%%%%%%%%%%%%%%%%%%%%%%%%%%%%%%%%%%%%%%%%%%%%%%%%%%%%%%%%%%%%%%%%

The table below shows the incremental claims paid on a portfolio of insurance
policies together with an extract from an index of prices. Claims are fully paid by the
end of development year 3.
Accident Year
2006
2007
2008
2009
0
103
88
110
132
Development Year
1
2
32
21
35
29
16
3 Year Price index
(mid year)
13 2006
2007
2008
2009 100
104
109
111
Calculate the reserve for unpaid claims using the inflation-adjusted chain ladder
approach, assuming that future claims inflation will be 3% p.a.


%%%%%%%%%%%%%%%%%%%%%%%%%%%%%%%%%%%%%%%%%%%%%%%%%%%%%%%%%%%%%%%%%%%%%%%%%%%%%%
8
(i)
Adjusting the incremental data for inflation to mid 2008 prices gives:
Accident Year 0
2006
2007
2008
2009 114.3
93.9
112.0
132
Development Year
1
2
34.2
21.4
35
29.5
16
3
13
Cumulating gives
Accident Year 0
2006
2007
2008
2009 114.3
93.9
112.0
132.0
Development Year
1
2
148.5
115.3
147.0
178.0
131.3
3
191.0
The development factors are:
Year 0 to year 1 148.5 + 115.3 + 147.0
= 1.2827
114.3 + 93.9 + 112.0
Year 1 to year 2 178.0 + 131.3
= 1.1726
148.5 + 115.3
Page 7%%%%%%%%%%%%%%%%%%%%%%%%%%%%%%%%%%%%%%%% — April 2010 — Examiners’ Report
Year 2 to year 3
191.0
= 1.0730
178.0
The completed table at mid 2008 prices is:
Accident Year
0
2006
2007
2008
2009
Development Year
1
2
169.3
172.4
198.5
3
140.9
185.0
213.0
Differencing gives:
Accident Year
2006
2007
2008
2009
0
Development Year
1
2
37.3
25.4
29.2
3
9.6
12.6
14.5
And so the total reserve is
(37.3 + 25.4 + 9.6) \times 1.03 + (29.2 + 12.6) \times 1.03 2 + 14.5 \times 1.03 3 = 134.7
Comment: This was a straightforward question with many candidates scoring full marks
here.
