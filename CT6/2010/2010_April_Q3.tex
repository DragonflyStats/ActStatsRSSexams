
%%% - Question 3

%%%%%%%%%%%%%%%%%%%%%%%%%%%%%%%%%%%%%%%%%%%%%

\documentclass[a4paper,12pt]{article}

%%%%%%%%%%%%%%%%%%%%%%%%%%%%%%%%%%%%%%%%%%%%%%%%%%%%%%%%%%%%%%%%%%%%%%%%%%%%%%%%%%%%%%%%%%%%%%%%%%%%%%%%%%%%%%%%%%%%%%%%%%%%%%%%%%%%%%%%%%%%%%%%%%%%%%%%%%%%%%%%%%%%%%%%%%%%%%%%%%%%%%%%%%%%%%%%%%%%%%%%%%%%%%%%%%%%%%%%%%%%%%%%%%%%%%%%%%%%%%%%%%%%%%%%%%%%

\usepackage{eurosym}
\usepackage{vmargin}
\usepackage{amsmath}
\usepackage{graphics}
\usepackage{epsfig}
\usepackage{enumerate}
\usepackage{multicol}
\usepackage{subfigure}
\usepackage{fancyhdr}
\usepackage{listings}
\usepackage{framed}
\usepackage{graphicx}
\usepackage{amsmath}
\usepackage{chng%%-- Page}
%\usepackage{bigints}
\usepackage{vmargin}

% left top textwidth textheight headheight

% headsep footheight footskip
\setmargins{2.0cm}{2.5cm}{16 cm}{22cm}{0.5cm}{0cm}{1cm}{1cm}
\renewcommand{\baselinestretch}{1.3}
\setcounter{MaxMatrixCols}{10}
\begin{document} 

3
The following two models have been suggested for representing some quarterly data
with underlying seasonality.
Model 1
Model 2
Y t = \alpha Y t − 4 + e t
Y t = \beta e t − 4 + e t
Where e t is a white noise process in each case.
(i)
Determine the auto-correlation function for each model.

The observed quarterly data is used to calculate the sample auto-correlation.
(ii)
State the features of the sample auto-correlation that would lead you to prefer
Model 1.

[Total 5]
CT6 A2010—24
%%%%%%%%%%%%%%%%%%%%%%%%%%%%%%%%%%%%%%%%%%%%%

%%%%%%%%%%%%%%%%%%%%%%%%%%%%%%%%%%%%
3
(i)
Model 1
In general we have γ k = \alphaγ k − 4
Taking covariance with Y t , Y t − 1 , Y t − 2 , Y t − 3 we get:
γ 1 = \alphaγ 3
γ 2 = \alphaγ 2
γ 3 = \alphaγ 1
For \alpha ≠ 0 these equations imply that ρ k = 0 unless k is divisible by 4.
Page 2%%%%%%%%%%%%%%%%%%%%%%%%%%%%%%%%%%%%%%%% — April 2010 — Examiners’ Report
So we have ρ 4 k = \alpha k and all other autocorrelations are zero.
Model 2
Here we have γ k = 0 unless k=4 and k=0. In these cases
.
So ρ 0 = 1 ,
(ii)
and all the other autocorrelations are zero.
Model 1 is preferred in situations where the sample auto-correlation is non-
zero and decays exponentially.
Comment: This question was reasonably well attempted, although a number of candidates
dropped marks especially in calculating the ACF of Model 1.
