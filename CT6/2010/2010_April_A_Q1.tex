\documentclass[a4paper,12pt]{article}

%%%%%%%%%%%%%%%%%%%%%%%%%%%%%%%%%%%%%%%%%%%%%%%%%%%%%%%%%%%%%%%%%%%%%%%%%%%%%%%%%%%%%%%%%%%%%%%%%%%%%%%%%%%%%%%%%%%%%%%%%%%%%%%%%%%%%%%%%%%%%%%%%%%%%%%%%%%%%%%%%%%%%%%%%%%%%%%%%%%%%%%%%%%%%%%%%%%%%%%%%%%%%%%%%%%%%%%%%%%%%%%%%%%%%%%%%%%%%%%%%%%%%%%%%%%%

\usepackage{eurosym}
\usepackage{vmargin}
\usepackage{amsmath}
\usepackage{graphics}
\usepackage{epsfig}
\usepackage{enumerate}
\usepackage{multicol}
\usepackage{subfigure}
\usepackage{fancyhdr}
\usepackage{listings}
\usepackage{framed}
\usepackage{graphicx}
\usepackage{amsmath}
\usepackage{chng%%-- Page}
%\usepackage{bigints}
\usepackage{vmargin}

% left top textwidth textheight headheight

% headsep footheight footskip
\setmargins{2.0cm}{2.5cm}{16 cm}{22cm}{0.5cm}{0cm}{1cm}{1cm}
\renewcommand{\baselinestretch}{1.3}
\setcounter{MaxMatrixCols}{10}
\begin{document} 

%%% - Question 1

1
A coin is biased so that the probability of throwing a head is an unknown constant $p$. It is known that p must be either 0.4 or 0.75. Prior beliefs about $p$ are given by the distribution:

P(p = 0.4) = 0.6
P(p = 0.75) = 0.4

The coin is tossed 6 times and 4 heads are observed.
Find the posterior distribution of $p$.

%%%%%%%%%%%%%%%%%%%%%%%%%

%%- Question 1

\[P ( p = 0.4 | 4 H ) &=&
\frac{P (4 H p = 0.4) P ( p = 0.4)}{P (4 H )}\]
But 

\begin{eqnarray*}
P(4H) &=& P (4 H |p = 0.4) P ( p = 0.4) + P (4 H p = 0.75) P ( p = 0.75)\\
&=& { 6 \choose 4} 0.4^4 0.6^2 \times 0.6 + { 6 \choose 4} 0.75^4 0.25^2 \times 0.4\\
&=&  0.082944 + 0.11865\\
&=& 0.201596\\
\end{eqnarray}

So 
\begin{eqnarray*}
P ( p = 0.4 |4 H ) &=& \frac{0.082944}{0.201596}\\
&=& 0.411436
\end{eqnarray*}

So the posterior distribution of $p$ is given by 
$P(p = 0.4) = 0.411436$ and 
$P(p = 0.75) = 0.588564$



Comment: This question was intended to be a straightforward application of Bayes’
Theorem. However, the question was generally not well answered, with many candidates
unable to find the posterior distribution in this slightly unfamiliar scenario.
\end{document}
