\documentclass[a4paper,12pt]{article}

%%%%%%%%%%%%%%%%%%%%%%%%%%%%%%%%%%%%%%%%%%%%%%%%%%%%%%%%%%%%%%%%%%%%%%%%%%%%%%%%%%%%%%%%%%%%%%%%%%%%%%%%%%%%%%%%%%%%%%%%%%%%%%%%%%%%%%%%%%%%%%%%%%%%%%%%%%%%%%%%%%%%%%%%%%%%%%%%%%%%%%%%%%%%%%%%%%%%%%%%%%%%%%%%%%%%%%%%%%%%%%%%%%%%%%%%%%%%%%%%%%%%%%%%%%%%

\usepackage{eurosym}
\usepackage{vmargin}
\usepackage{amsmath}
\usepackage{graphics}
\usepackage{epsfig}
\usepackage{enumerate}
\usepackage{multicol}
\usepackage{subfigure}
\usepackage{fancyhdr}
\usepackage{listings}
\usepackage{framed}
\usepackage{graphicx}
\usepackage{amsmath}
\usepackage{chngpage}
%\usepackage{bigints}
\usepackage{vmargin}

% left top textwidth textheight headheight

% headsep footheight footskip
\setmargins{2.0cm}{2.5cm}{16 cm}{22cm}{0.5cm}{0cm}{1cm}{1cm}
\renewcommand{\baselinestretch}{1.3}
\setcounter{MaxMatrixCols}{10}



\begin{document}
An insurance company has a portfolio of policies under which individual loss amounts follow an exponential distribution with mean 1/ λ . There is an individual excess of loss reinsurance arrangement in place with retention level 100. In one year,
the insurer observes:
•
•
85 claims for amounts below 100 with mean claim amount 42; and
39 claims for amounts above the retention level.
(i) Calculate the maximum likelihood estimate of λ .
(ii) Show that the estimate of λ produced by applying the method of moments to
the distribution of amounts paid by the insurer is 0.011164.
[5]
[Total 10]
% CT6 S2010—3
% [5]
% PLEASE TURN OVER
%%%%%%%%%%%%%%%%%%%%%%%%%%%%%%%%%%%%%%%%%%%%%%%%%%%%%%%%%%%%
8
Claims on a portfolio of insurance policies arrive as a Poisson process with rate λ .
The claim sizes are independent identically distributed random variables
X 1 , X 2 , ...with:
M
∑ p k = 1 .
P ( X i = k ) = p k for k = 1, 2, ... , M and
k = 1
The premium loading factor is θ.
(i)
Show that the adjustment coefficient R satisfies:
2 θ m 1
1
log(1 + θ ) < R <
M
m 2
where m i = E ( X 1 i ) for i = 1, 2 .
[The inequality e Rx ≤
[7]
x RM
x
for 0 ≤ x ≤ M may be used without
+ 1 −
e
M
M
proof.]
(ii)
9
(a) Determine upper and lower bounds for R if θ = 0.3 and X i is equally likely to be 2 or 3 (and cannot take any other values).
(b) Hence derive an upper bound on the probability of ruin when the initial
surplus is U.
[3]
[Total 10]
An actuarial student has been working on some claims projections but some of her workings have been lost. The cumulative claim amounts and projected ultimate
claims are given by the following table:
Accident
Year 0
1
2
3
4 1001
1250
1302
Z
Development Year
1
2
1485
Y
1805
1762
1820
Ultimate
3
W
X
1862.3
2122.5
2278.8
All claims are paid by the end of development year 3.
It is known that ultimate claims for accident years 2 and 3 have been estimated using the Basic Chain Ladder method.
(i)
Calculate the values of W, X and Y.
[5]
For accident year 4 the student has used the Bornhuetter-Ferguson method using an earned premium of 2,500 and an expected loss ratio of 90%.
CT6 S2010—410
(ii) Calculate the value of Z.
[4]
(iii) Calculate the outstanding claims reserve for all accident years implied by the completed table.
%%%%%%%%%%%%%%%%%%%%%%%%%%%%%%%%%%%%%%%%%%%%%%%%%%%%%%%%%%%%%%%%%%%%%%%%%%%%%%%%%%%%%%%%%%%%%%%
6
(i)
We must write f(x) in the form:
⎡ x θ − b ( θ )
⎤
f ( x ) = exp ⎢
+ c ( x , φ ) ⎥
⎣ a ( φ )
⎦
For some parameters θ , φ and functions a,b and c.
α α
f ( x ) =
μ α Γ ( α )
x
α− 1
e
−
x α
μ
⎡ ⎛ x
⎤
⎞
= exp ⎢ ⎜ − − log μ ⎟ α + ( α − 1) log x + α log α − log Γ ( α ) ⎥
⎠
⎣ ⎝ μ
⎦
Which is of the required form with:
θ=−
1
μ
φ=α
a ( φ ) =
1
φ
% Page 5Subject CT6 (Statistical Methods Core Technical) — September 2010 — Examiners’ Report
b ( θ ) = − log( −θ ) = log μ
c ( x , φ ) = ( φ − 1) log x + φ log φ − log Γ ( φ )
(ii)
The mean and variance for members of the exponential family are given by
b '( θ ) and a ( φ ) b ''( θ ) .
In this case b '( θ ) = −
1
=μ
θ
b ''( θ ) = θ − 2 = μ 2 so the variance is μ 2 / α as required.
Generally well answered, though many candidates did not score full marks on part (i)
because they failed to specify all the parameters involved.
7
(i)
First note that the probability of a claim exceeding 100 is e − 100 λ .
The likelihood function for the given data is:
L = C × λ 85 e − 85 × 42 ×λ × ( e − 100 ×λ ) 39
Where C is some constant. Taking logarithms gives
l = log L = C ' + 85 log λ − 85 × 42 × λ − 100 × 39 × λ
Differentiating with respect to λ gives
∂ l 85
=
− 85 × 42 − 100 × 39
∂λ λ
Setting this expression equal to zero we get:
λ ˆ =
85
= 0.011379
85 × 42 + 100 × 39
And this gives a maximum since
Page 6
∂ 2 l
∂λ
2
=−
85
λ 2
< 0
% Subject CT6 (Statistical Methods Core Technical) — September 2010 — Examiners’ Report
(ii)
We must first calculate the mean amount paid by the insurer per claim. This is
100
∫
0
100
x λ e −λ x dx + 100 P ( X > 100) = ⎡ − xe −λ x ⎤ +
⎣
⎦ 0
= − 100 e
=
− 100 λ
(
100
∫ e
−λ x
dx + 100 e − 100 λ
0
100
⎡ 1
⎤
+ ⎢ − e −λ x ⎥ + 100 e − 100 λ
⎣ λ
⎦ 0
1
1 − e − 100 λ
λ
)
So we must show the given value of λ results in the actual average paid by the
85 × 42 + 39 × 100
insurer. This is
= 60.24
85 + 39
Substituting for λ in the expression derived above, we get
(
)
1
0.6725435
1 − e − 100 × 0.0111654 =
= 60.24 as required.
0.011164
0.011164
Stronger candidates scored well on this question, whereas the weaker candidates struggled
with the calculations required in part (ii).
\end{document}
