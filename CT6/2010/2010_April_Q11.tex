%%%%%%%%%%%%%%%%%%%%%%%%%%%%%%%%%%%%%%%%%%%%%

\documentclass[a4paper,12pt]{article}

%%%%%%%%%%%%%%%%%%%%%%%%%%%%%%%%%%%%%%%%%%%%%%%%%%%%%%%%%%%%%%%%%%%%%%%%%%%%%%%%%%%%%%%%%%%%%%%%%%%%%%%%%%%%%%%%%%%%%%%%%%%%%%%%%%%%%%%%%%%%%%%%%%%%%%%%%%%%%%%%%%%%%%%%%%%%%%%%%%%%%%%%%%%%%%%%%%%%%%%%%%%%%%%%%%%%%%%%%%%%%%%%%%%%%%%%%%%%%%%%%%%%%%%%%%%%

\usepackage{eurosym}
\usepackage{vmargin}
\usepackage{amsmath}
\usepackage{graphics}
\usepackage{epsfig}
\usepackage{enumerate}
\usepackage{multicol}
\usepackage{subfigure}
\usepackage{fancyhdr}
\usepackage{listings}
\usepackage{framed}
\usepackage{graphicx}
\usepackage{amsmath}
\usepackage{chng%%-- Page}
%\usepackage{bigints}
\usepackage{vmargin}

% left top textwidth textheight headheight

% headsep footheight footskip
\setmargins{2.0cm}{2.5cm}{16 cm}{22cm}{0.5cm}{0cm}{1cm}{1cm}
\renewcommand{\baselinestretch}{1.3}
\setcounter{MaxMatrixCols}{10}
\begin{document} 


CT6 A2010—611
An actuary has, for three years, recorded the volume of unsolicited advertising that he
receives. He believes that the number of items that he receives follows a Poisson
distribution with a mean which varies according to which quarter of the year it is. He
has recorded Y ij the number of items received in the i th quarter of the j th year ( i = 1,
2, 3, 4 and j = 1, 2, 3). The actuary wishes to estimate the number of items that he
will receive in the 1st quarter of year 4. He has recorded the following data:
Y i 1
Y i 2
Y i 3
Y i = 1
3 \sum
j
i = 1
i = 2
i = 3
i = 4
(i)
98
82
75
132
117
102
83
152
124
95
88
148
113
93
82
144
Y ij
\sum ( Y ij − Y i )
2
j
362
206
86
224
Estimate Y 1,4 the number of items that the actuary expects to receive in the
first quarter of year 4 using the assumptions of EBCT model 1.
[5]
The actuary believes that, in fact, the volume of items has been increasing at the rate
of 10% per annum.
(ii) Suggest how the approach in (i) can be adjusted to produce a revised estimate
taking this growth into account.

(iii) Calculate the maximum likelihood estimate of Y 1,4 (based on the quarter 1 data
already observed and the 10% p.a. increase described above).
[5]
(iv) Compare the assumptions underlying the approach in (i) and (ii) with those
underlying the approach in (iii).

%%%%%%%%%%%%%%%%%%%%%%%%%%%%%%%%%%%%%%%%%%%%%%%%%%%%%%%%%%%%%%%%%%%%%%%%%%%%%%%%%%%%%5


11
An actuary has, for three years, recorded the volume of unsolicited advertising that he
receives. He believes that the number of items that he receives follows a Poisson
distribution with a mean which varies according to which quarter of the year it is. He
has recorded Y ij the number of items received in the i th quarter of the j th year ( i = 1,
2, 3, 4 and j = 1, 2, 3). The actuary wishes to estimate the number of items that he
will receive in the 1st quarter of year 4. He has recorded the following data:
Y i 1
Y i 2
Y i 3
Y i = 1
3 \sum
j
i = 1
i = 2
i = 3
i = 4
(i)
98
82
75
132
117
102
83
152
124
95
88
148
113
93
82
144
Y ij
\sum ( Y ij − Y i )
2
j
362
206
86
224
Estimate Y 1,4 the number of items that the actuary expects to receive in the first quarter of year 4 using the assumptions of EBCT model 1.
[5]
The actuary believes that, in fact, the volume of items has been increasing at the rate of 10% per annum.
(ii) Suggest how the approach in (i) can be adjusted to produce a revised estimate taking this growth into account.

(iii) Calculate the maximum likelihood estimate of Y 1,4 (based on the quarter 1 data already observed and the 10% p.a. increase described above).
[5]
(iv) Compare the assumptions underlying the approach in (i) and (ii) with those underlying the approach in (iii).

%%%%%%%%%%%%%%%%%%%%%%%%%%%%%%%%%%%%%%%%%%%%%%%%%%%%%%%%%%%%%%%%%%%%%%%%%%%%%%%%%5


%%--- Question 11



%%%%%%%%%%%%%%%%%%%%%%%%%%%%%%%%%%%%%%%%%%%%%%%%%%%%%%%%%%%%%%%%%%%%%%%%%%%%%%%%%%%%%%%%%%%%%%

11
(i)
The overall mean is given by Y =
(
E s ( θ )
2
)
113 + 93 + 82 + 144
= 108
4
⎞ 362 + 206 + 86 + 224
1 4 ⎛ 1 3
= 109.75
= \sum ⎜ \sum ( Y ij − Y i ) 2 ⎟ =
⎟
4 i = 1 ⎜ 2 j = 1
8
⎝
⎠
2
1 4
1
Var ( m ( θ )) = \sum ( Y i − Y ) − E ( S 2 ( θ ))
3 i = 1
3
=
(113 − 108) 2 + (93 − 108) 2 + (82 − 108) 2 + (144 − 108) 2 109.75
−
3
3
= 704.083
So the credibility factor is Z =
3
= 0.950608
109.75
3 +
704.083
And the estimate for next quarter is
0.950608 \times 113 + (1 − 0.950608) \times 108 = 112.75
(ii)
(iii)
The average number of pieces of mail is assumed to be growing each year.
We need to adjust the data to take account of this. Two approaches are:
• Convert the data into “Year 4” values by increasing by 10% p.a. and then
applying the methodology above; OR
• Recognise the lower volume of data in earlier years, by applying a risk
volume to each year and using EBCT model 2. If the risk volume for year
4 is 1, then the risk volume for year 3 is 1/1.1 and year 2 is 1/1.21 etc.
Let the mean number of items in quarter 1 of year 1 be given by \lambda . Then the
likelihood is given by:
L ∝ e −\lambda \lambda Y 11 e − 1.1 \lambda (1.1 \lambda ) Y 12 e − 1.1 \lambda (1.1 2 \lambda ) Y 13
2
And so the log likelihood is
l = log L = C − \lambda (1 + 1.1 + 1.1 2 ) + ( Y 11 + Y 12 + Y 13 ) log \lambda
Page 11%%%%%%%%%%%%%%%%%%%%%%%%%%%%%%%%%%%%%%%% — April 2010 — Examiners’ Report
Differentiating
Y + Y + Y
\frac{\partial}{\partial}  l
= − (1 + 1.1 + 1.1 2 ) + 11 12 13
\frac{\partial}{\partial} \lambda
\lambda
And setting this equal to zero gives:
Y + Y + Y
98 + 117 + 124
\hat{lambda} = 11 12 13
=
= 102.417
3.31
1 + 1.1 + 1.1 2
So the estimate for Q1 in year 4 is 1.1 3 \times \hat{lambda} = 1.331 \times 102.417 = 136.32
(iv)
The main difference is that the maximum likelihood estimate approach considers the data for Q1 in isolation, whereas the EBCT approach assumes
that data from other quarters come from a related distribution and so can tell us something about Q1.
Specifically, the EBCT approach assumes that the mean volume of unsolicited mail for each quarter is itself a sample from a common distribution. Hence
whilst each quarter has a different mean, they provide some information about the population from which the mean is drawn.
Comment: The same comment as in the previous question is valid here. This question was
not very well answered question but with various alternative answers in (ii) and (iv).
END OF EXAMINERS’ REPORT
Page 12
