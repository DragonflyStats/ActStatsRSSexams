%%%%%%%%%%%%%%%%%%%%%%%%%%%%%%%%%%%%%%%%%%%%%

\documentclass[a4paper,12pt]{article}

%%%%%%%%%%%%%%%%%%%%%%%%%%%%%%%%%%%%%%%%%%%%%%%%%%%%%%%%%%%%%%%%%%%%%%%%%%%%%%%%%%%%%%%%%%%%%%%%%%%%%%%%%%%%%%%%%%%%%%%%%%%%%%%%%%%%%%%%%%%%%%%%%%%%%%%%%%%%%%%%%%%%%%%%%%%%%%%%%%%%%%%%%%%%%%%%%%%%%%%%%%%%%%%%%%%%%%%%%%%%%%%%%%%%%%%%%%%%%%%%%%%%%%%%%%%%

\usepackage{eurosym}
\usepackage{vmargin}
\usepackage{amsmath}
\usepackage{graphics}
\usepackage{epsfig}
\usepackage{enumerate}
\usepackage{multicol}
\usepackage{subfigure}
\usepackage{fancyhdr}
\usepackage{listings}
\usepackage{framed}
\usepackage{graphicx}
\usepackage{amsmath}
\usepackage{chng%%-- Page}
%\usepackage{bigints}
\usepackage{vmargin}

% left top textwidth textheight headheight

% headsep footheight footskip
\setmargins{2.0cm}{2.5cm}{16 cm}{22cm}{0.5cm}{0cm}{1cm}{1cm}
\renewcommand{\baselinestretch}{1.3}
\setcounter{MaxMatrixCols}{10}
\begin{document} 


%%% ---- Question 10


10
Claims on a portfolio of insurance policies arrive as a Poisson process with annual
rate \lambda. Individual claims are for a fixed amount of 100 and the insurer uses a
premium loading of 15%. The insurer is considering entering a proportional
reinsurance agreement with a reinsurer who uses a premium loading of 20%. The
insurer will retain a proportion \alpha of each risk.
(i) Write down and simplify the equation defining the adjustment coefficient R
for the insurer.

(ii) By considering R as a function of \alpha and differentiating show that
(120 \alpha − 5)
(iii)
Explain why setting
for \alpha .
(iv)
dR
dR ⎞ 100 \alpha R
⎛
+ 120 R = ⎜ 100 R + 100 \alpha
⎟ e
d \alpha
d \alpha ⎠
⎝

dR
= 0 and solving for \alpha may give an optimal value
d\alpha

Use the method suggested in part (iii) to find an optimal choice for \alpha .

[Total 13]

10
(i)
Let X denote the individual claim amounts net of re-insurance. Then
X = 100 \alpha and M X ( t ) = e 100 \alpha t .
The insurer’s annual net premium income is
100 \times \lambda \times 1.15 − 100 \times (1 − \alpha ) \times \lambda \times 1.2 = \lambda (120 \alpha − 5)
So the adjustment coefficient R satisfies
\lambda + \lambda (120 \alpha − 5) R = \lambda e 100 \alpha R
That is 1 + (120 \alpha − 5) R = e 100 \alpha R
Page 9%%%%%%%%%%%%%%%%%%%%%%%%%%%%%%%%%%%%%%%% — April 2010 — Examiners’ Report
(ii)
Differentiating this equation with respect to \alpha we get
120 R + (120 \alpha − 5)
and
dR
d ⎡ 100 \alpha R ⎤
=
e
⎦
d \alpha d \alpha ⎣
d ⎡ 100 \alpha R ⎤
d
= e 100 \alpha R
e
[ 100 \alpha R ]
⎣
⎦
d \alpha
d \alpha
dR ⎞
⎛
= e 100 \alpha R ⎜ 100 R + 100 \alpha
⎟
d \alpha ⎠
⎝
So putting these together, we have:
(120 \alpha − 5)
(iii)
dR
dR ⎞ 100 \alpha R
⎛
+ 120 R = ⎜ 100 R + 100 \alpha
⎟ e
d \alpha
d \alpha ⎠
⎝
Firstly, by Lundberg’s inequality the higher the value of R the lower the upper
bound on the probability of ruin.
So we wish to choose \alpha so that R is a maximum.
That is, we need
(iv)
dR
= 0 .
d \alpha
dR
= 0 in the equation in (ii) we get
d \alpha
Putting
120 R = 100 Re 100 \alpha R
i.e. e 100 \alpha R = 1.2
i.e. R =
log1.2
100 \alpha
substituting into the equation in (i) gives
1 + (120 \alpha − 5) \times
log1.2
= 1.2
100 \alpha
i.e. 100 \alpha + (120 \alpha − 5) log1.2 = 120 \alpha
− 20 \alpha + 120 \alpha log1.2 = 5 log1.2
So \alpha =
Page 10
5log1.2
= 0.48526
120 log1.2 − 20

%%%%%%%%%%%%%%%%%%%%%%%%%%%%%%%%%%%%%%%% — April 2010 — Examiners’ Report
Comment: This question on material new to the syllabus for 2010 was not answered well
with many candidates struggling, especially in the last part.

\end{document}
