\documentclass[a4paper,12pt]{article}

%%%%%%%%%%%%%%%%%%%%%%%%%%%%%%%%%%%%%%%%%%%%%%%%%%%%%%%%%%%%%%%%%%%%%%%%%%%%%%%%%%%%%%%%%%%%%%%%%%%%%%%%%%%%%%%%%%%%%%%%%%%%%%%%%%%%%%%%%%%%%%%%%%%%%%%%%%%%%%%%%%%%%%%%%%%%%%%%%%%%%%%%%%%%%%%%%%%%%%%%%%%%%%%%%%%%%%%%%%%%%%%%%%%%%%%%%%%%%%%%%%%%%%%%%%%%

\usepackage{eurosym}
\usepackage{vmargin}
\usepackage{amsmath}
\usepackage{graphics}
\usepackage{epsfig}
\usepackage{enumerate}
\usepackage{multicol}
\usepackage{subfigure}
\usepackage{fancyhdr}
\usepackage{listings}
\usepackage{framed}
\usepackage{graphicx}
\usepackage{amsmath}
\usepackage{chngpage}
%\usepackage{bigints}
\usepackage{vmargin}

% left top textwidth textheight headheight

% headsep footheight footskip
\setmargins{2.0cm}{2.5cm}{16 cm}{22cm}{0.5cm}{0cm}{1cm}{1cm}
\renewcommand{\baselinestretch}{1.3}
\setcounter{MaxMatrixCols}{10}
\begin{document} 
5
The table below shows aggregate annual claim statistics for three risks over a period
of seven years. Annual aggregate claims for risk i in year j are denoted by X ij .
X i =
Risk, i
1 7
= ∑ X ij − X i
6 j = 1
(
)
2
335.1
65.1
33.9
(i) Calculate the credibility premium of each risk under the assumptions of EBCT
Model 1.
[6]
(ii) Explain why the credibility factor is relatively high in this case.
[2]
[Total 8]
The probability density function of a gamma distribution is given in the following
parameterised form:
f ( x ) =
(i)
(ii)
7
S i 2
127.9
88.9
149.7
i = 1
i = 2
i = 3
6
1 7
∑ X ij
7 j = 1
α α
\mu  α Γ ( α )
x
α− 1
e
−
x α
\mu 
for x > 0.
Express this density in the form of a member of the exponential family,
specifying all the parameters.
[6]
Hence show that the mean and variance of the distribution are given by \mu  and
\mu  2
respectively.
[3]
α
[Total 9]
%%%%%%%%%%%%%%%%%%%%%%%%%%%%%%%%%%%%%%%%%%%%%%%%%%%%%%%%%%%%%%%%%%%%%%%%%%%%%%%%%%

\newpage

4
Let the prior distribution of \mu  have a Gamma distribution with parameters α and \lambda 
as per the tables.
Then
α
α
= 50 and 2 = 15 2
\lambda 
\lambda 
Then dividing the first by the second \lambda  =
50
15 2
= 0.22222
And so α = 50 × 0.22222 = 11.111111

%%--- Page 3Subject CT6 (Statistical Methods Core Technical) — September 2010 — Examiners’ Report

The posterior distribution of \mu  is then given by
f ( \mu  x ) ∝ f ( x \mu  ) f ( \mu  )
∝ e − 10 \mu  ×\mu  630 ×\mu  10.11111 e − 0.22222 \mu 
∝ \mu  640.11111 e − 10.22222 \mu 
Which is the pdf of a Gamma distribution with parameters α ' = 641.11111 and
\lambda  '= 10 . 22222
Now under all or nothing loss, the Bayesian estimate is given by the mode of the
posterior distribution. So we must find the maximum of
f ( x ) = x 640.11111 e − 10.2222 x (we may ignore constants here)
Differentiating:
(
f '( x ) = e − 10.22222 x − 10.2222 x 640.11111 + 640.1111 x 639.11111
)
= x 639.1111 e − 10.22222 x ( − 10.2222 x + 640.11111)
And setting this equal to zero we get
x =
640.111111
= 62.62
10.22222
Alternatively, credit was given for differentiating the log of the posterior (which is simpler).
This question was well answered by most candidates.

%%%%%%%%%%%%%%%%%%%%%%%%%%%%%%%%%%%%%%%%%%%%%%%%%%%%%%%%%%%%%%%%%%%%%%%%%%%%%%%%%%%%%%%%%%%%%
5
(i)
The overall mean is given by X =
(
)
E s 2 ( θ ) =
127.9 + 88.9 + 149.7
= 122.1 67
3
⎞ 335 . 1 + 65 . 1 + 33 . 9
1 3 ⎛ 1 7
⎜ ∑ ( X ij − X i ) 2 ⎟ =
= 144 . 7
∑
⎜
⎟
3 i = 1 ⎝ 6 j = 1
3
⎠
1 3
( X i − X ) 2 − 1 E ( S 2 ( θ ))
∑
2 i = 1
7
2
( 127 . 9 − 122 . 1 ) + ( 88 . 9 − 122 . 1 ) 2 + ( 149 . 7 − 122 . 1 ) 2 144 . 7
=
−
2
7
= 928 . 14
Var ( m ( θ )) =
So the credibility factor is Z =
Page 4
7
7 + 144 . 7
= 0 . 978213
928 . 14
%%----- Subject CT6 (Statistical Methods Core Technical) — September 2010 — Examiners’ Report
And the credibility premia for the risks are:
For risk 1 : 0 . 978213 × 127 . 9 + ( 1 − 0 . 978213 ) × 122 . 167 = 127 . 8
For risk 2 : 0 . 978213 × 88 . 9 + ( 1 − 0 . 978213 ) × 122 . 167 = 89 . 6
For risk 3 : 0 . 978213 × 149 . 7 + ( 1 − 0 . 978213 ) × 122 . 167 = 149 . 1
(ii)
The data show that the variation within risks is relatively low (the S i 2 are low, especially for the 2 nd and 3 rd risks) but there seems to be quite a high variation between the average claims on the risks.
With the S i 2 being low, this variation cannot be explained just by variability in the claims, and must be due to variability in the underlying parameter.
This means that we can put relatively little weight on the information provided by the data set as a whole, and must put more on the data from the individual risks, leading to a relatively high credibility factor.
Most candidates scored well on part (i). Only the better candidates were able to give a clear explanation in part (ii).
\end{document}
