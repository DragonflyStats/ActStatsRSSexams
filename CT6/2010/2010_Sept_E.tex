\documentclass[a4paper,12pt]{article}

%%%%%%%%%%%%%%%%%%%%%%%%%%%%%%%%%%%%%%%%%%%%%%%%%%%%%%%%%%%%%%%%%%%%%%%%%%%%%%%%%%%%%%%%%%%%%%%%%%%%%%%%%%%%%%%%%%%%%%%%%%%%%%%%%%%%%%%%%%%%%%%%%%%%%%%%%%%%%%%%%%%%%%%%%%%%%%%%%%%%%%%%%%%%%%%%%%%%%%%%%%%%%%%%%%%%%%%%%%%%%%%%%%%%%%%%%%%%%%%%%%%%%%%%%%%%

\usepackage{eurosym}
\usepackage{vmargin}
\usepackage{amsmath}
\usepackage{graphics}
\usepackage{epsfig}
\usepackage{enumerate}
\usepackage{multicol}
\usepackage{subfigure}
\usepackage{fancyhdr}
\usepackage{listings}
\usepackage{framed}
\usepackage{graphicx}
\usepackage{amsmath}
\usepackage{chngpage}
%\usepackage{bigints}
\usepackage{vmargin}

% left top textwidth textheight headheight

% headsep footheight footskip
\setmargins{2.0cm}{2.5cm}{16 cm}{22cm}{0.5cm}{0cm}{1cm}{1cm}
\renewcommand{\baselinestretch}{1.3}
\setcounter{MaxMatrixCols}{10}
\begin{document} [1]
\item 
An insurance company has a portfolio of 10,000 policies covering buildings against the risk of flood damage.
\begin{enumerate}[(i)]
\item State the conditions under which the annual number of claims on the portfolio can be modelled by a binomial distribution B(n, p) with n = 10,000.

These conditions are satisfied and p = 0.03. Individual claim amounts follow a normal distribution with mean 400 and standard deviation 50. The insurer wishes to take out proportional reinsurance with the retention α set such that the probability of
aggregate payments on the portfolio after reinsurance exceeding 120,000 is 1%.
\item %(ii)
Calculate α assuming that aggregate annual claims can be approximated by a
normal distribution.
[7]
This reinsurance arrangement is set up with a reinsurer who uses a premium loading
of 15%.
\item %-- (iii)
Calculate the annual premium charged by the reinsurer.

As an alternative, the reinsurer has offered an individual excess of loss reinsurance
arrangement with a retention of M for the same annual premium. The reinsurer uses
the same 15\% loading to calculate premiums for this arrangement.
\item %--(iv)
Show that the retention M is approximately 358.50.
\end{enumerate}
[You may wish to use the following formula which is given on page 18 of the Tables:
If f(x) is the PDF of the N(μ, σ 2 ) distribution then
U
∫ L
xf ( x ) dx = μ [ Φ ( U ') − Φ ( L ') ] − σ [ φ ( U ') − φ ( L ') ]
where L ' =
L −μ
U −μ
.
and U ' =
σ
σ
Here Φ(z) is the cumulative density function of the N(0, 1) distribution and
φ ( z ) =
e
CT6 S2010—5
−
z 2
2
2 π
.]
[Total 16]
%%%%%%%%%%%%%%%%%%%%%%%%%%%%%%%%%%%%%%%%%%%%%%%%%%%%%%%%%%%%%%%%%%%%%%%%%%%%%%%%%%%%%%%%%%%%%%%%%%
\newpage
% Page 9Subject CT6 (Statistical Methods Core Technical) — September 2010 — Examiners’ Report
10
(i)
We require:
•
•
•
(ii)
\begin{itemize}
\item The risk of flood damage is a constant p for each building.
\item There can only be one claim per policy per year.
\item The risk of flood damage is independent from building to building.
\item Let the individual claim amounts net of re-insurance be X. 
\item Then
E ( α X ) = α E ( X ) = 400 α
And Var ( α X ) = α 2 Var ( X ) = (50 α ) 2
\item So if Y represents the aggregate annual claims net of re-insurance, then we
have:
E ( Y ) = 10, 000 × 0.03 × 400 α = 120, 000 α
and
Var ( Y ) = 10, 000 × 0.03 × (50 α ) 2 + 10, 000 × 0.03 × 0.97 × (400 α ) 2 = 47,310, 000 α 2
= (6,878.23 α ) 2
\item We require α to be chosen so that
P ( Y > 120, 000) = 0.01
i.e. P ( N (0,1) >
120, 000 − 120, 000 α
) = 0.01
6,878.23 α
120, 000 − 120, 000 α
= 2.3263
6,878.23 α
i.e. α =
(iii)
120 , 000
= 0 . 8823476 ; α = 88 . 2 % to 3 sf
120 , 000 + 2 . 3263 × 6 , 878 . 23
\item The mean claim amount for the re-insurer is ( 1 − 0 . 882 ) × 400 = 47 . 20
\item The annual premiums for reinsurance are 10 , 000 × 0 . 03 × 47 . 20 × 1 . 15 = 16 , 284
%% Page 10Subject CT6 (Statistical Methods Core Technical) — September 2010 — Examiners’ Report
\end{itemize}
(iv)
We must show that using a retention of 358.50 to calculate the premium for the individual excess of loss arrangement gives the same result as the proportional reinsurance arrangement in part (ii).
We first calculate the mean claim amount paid by re-insurer. This is equal to
∞
∫
( x − 358.50) f ( x ) dx
358.50
⎛
358.50 − 400 ⎤
358.50 − 400 ⎤
⎡
⎡
⎛ 358.50 − 400 ⎞ ⎞
= 400 ⎢ 1 − Φ (
) ⎥ − 50 ⎢ 0 − φ (
) ⎥ − 358.50 × ⎜ 1 − Φ ⎜
⎟ ⎟
50
50
50
⎣
⎦
⎣
⎦
⎝
⎠ ⎠
⎝
This gives
400 [ 1 − Φ ( − 0.83) ] + 50 φ ( − 0.83) − 358.50 × (1 − Φ ( − 0.83))
1 −
= 400 × 0.79673 + 50 ×
e
2 π
= 47.20
0.83 2
2
− 358.50 × 0.79673
Then the aggregate premium charged will be 10, 000 × 0.03 × 47.20 × 1.15 =
16,284 which is the same as under the first arrangement as required.
Carrying forward more than 3 significant figures from the result in (ii) gives a slightly
different value in (iii). To full accuracy, the solution in (iii) becomes 16,236 resulting in a
minor discrepancy between the answers in (iii) and (iv). This appears not to have concerned
candidates who were generally happy to observe that the results in (iii) and (iv) were
approximately equal. The examiners gave credit for either approach.

%%This question was a good differentiator – the better prepared candidates were able to score well whilst weaker candidates struggled.
\end{document}
