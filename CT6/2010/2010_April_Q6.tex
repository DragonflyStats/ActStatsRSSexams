\documentclass[a4paper,12pt]{article}

%%%%%%%%%%%%%%%%%%%%%%%%%%%%%%%%%%%%%%%%%%%%%%%%%%%%%%%%%%%%%%%%%%%%%%%%%%%%%%%%%%%%%%%%%%%%%%%%%%%%%%%%%%%%%%%%%%%%%%%%%%%%%%%%%%%%%%%%%%%%%%%%%%%%%%%%%%%%%%%%%%%%%%%%%%%%%%%%%%%%%%%%%%%%%%%%%%%%%%%%%%%%%%%%%%%%%%%%%%%%%%%%%%%%%%%%%%%%%%%%%%%%%%%%%%%%

\usepackage{eurosym}
\usepackage{vmargin}
\usepackage{amsmath}
\usepackage{graphics}
\usepackage{epsfig}
\usepackage{enumerate}
\usepackage{multicol}
\usepackage{subfigure}
\usepackage{fancyhdr}
\usepackage{listings}
\usepackage{framed}
\usepackage{graphicx}
\usepackage{amsmath}
\usepackage{chng%%-- Page}
%\usepackage{bigints}
\usepackage{vmargin}

% left top textwidth textheight headheight

% headsep footheight footskip
\setmargins{2.0cm}{2.5cm}{16 cm}{22cm}{0.5cm}{0cm}{1cm}{1cm}
\renewcommand{\baselinestretch}{1.3}
\setcounter{MaxMatrixCols}{10}
\begin{document} 


%%- Question 6
6
Observations y 1 , y 2 , ... , y n are made from a random walk process given by:
Y 0 = 0 and Y t = a + Y t − 1 + e t for t > 0
where e t is a white noise process with variance \sigma^2 .
7

\begin{enumerate}
\item (i) Derive expressions for E(Y_{t} ) and Var(Y_{t} ) and explain why the process is not
stationary.
\item 
(ii) Show that γ t , $s = Cov ( Y t , Y t − s )$ for s < t is linear in s.
\item
(iii) Explain how you would use the observed data to estimate the parameters a
and \sigma.
\item 
(iv) Derive expressions for the one-step and two-step forecasts for Y n + 1 and Y n + 2 .
\end{enumerate}

%%%%%%%%%%%%%%%%%%%%%%%%%%%%%%%%%%%%%%%%%%%%%%%%%%%%%%%%%%%%%%%%%%%%%%%%%%%%%%%
6
(i)
We can deduce that Y t = at + \sum e i
i = 1
and so $E ( Y t ) = at$ and
$Var ( Y t ) = t \sigma^2$ .
Since these expressions depend on t the process is not stationary.
(ii)
As s < t we have
t t − s t − s
i = 1 j = 1 j = 1
Cov ( Y t , Y t − s ) = Cov ( at + \sum e i , as + \sum e j ) = Var ( \sum e j ) = ( t − s ) \sigma^2
Which is linear in s as required.
(iii)
First note that the differenced series:
\[X t = Y t − Y t − 1 = a + e t\]
is essentially a white noise process. So estimates of a and \sigma^2 can be found by constructing the sample differences series x i = y i − y i − 1 for i = 1, 2, ... , n and
taking the mean and sample variance(or its square for estimating )
respectively.
(iv)
In this case \hat{y} n (1) = a ˆ + y n + 0 = a ˆ + y n
And \hat{y} n (2) = a ˆ + \hat{y} n (1) + 0 = 2 a ˆ + y n

%%%%%%%%%%%%%%%%%%%%%%%%%%%%%%%%55
\newpage
Comment: This was a rather hard question with many candidates confirming the non-
stationarity in (i) but not finding the general solution. In part (ii) a good number of answers
failed to score full marks. Alternative answers to (i) and (ii) could have been obtained by
increasing t iteratively and noticing a pattern developing.


%% Page 5%%%%%%%%%%%%%%%%%%%%%%%%%%%%%%%%%%%%%%%% — April 2010 — Examiners’ Report
\end{document}
