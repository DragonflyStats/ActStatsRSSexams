\documentclass[a4paper,12pt]{article}

%%%%%%%%%%%%%%%%%%%%%%%%%%%%%%%%%%%%%%%%%%%%%%%%%%%%%%%%%%%%%%%%%%%%%%%%%%%%%%%%%%%%%%%%%%%%%%%%%%%%%%%%%%%%%%%%%%%%%%%%%%%%%%%%%%%%%%%%%%%%%%%%%%%%%%%%%%%%%%%%%%%%%%%%%%%%%%%%%%%%%%%%%%%%%%%%%%%%%%%%%%%%%%%%%%%%%%%%%%%%%%%%%%%%%%%%%%%%%%%%%%%%%%%%%%%%

\usepackage{eurosym}
\usepackage{vmargin}
\usepackage{amsmath}
\usepackage{graphics}
\usepackage{epsfig}
\usepackage{enumerate}
\usepackage{multicol}
\usepackage{subfigure}
\usepackage{fancyhdr}
\usepackage{listings}
\usepackage{framed}
\usepackage{graphicx}
\usepackage{amsmath}
\usepackage{chng%%-- Page}
%\usepackage{bigints}
\usepackage{vmargin}

% left top textwidth textheight headheight

% headsep footheight footskip
\setmargins{2.0cm}{2.5cm}{16 cm}{22cm}{0.5cm}{0cm}{1cm}{1cm}
\renewcommand{\baselinestretch}{1.3}
\setcounter{MaxMatrixCols}{10}
\begin{document} 

%%- Question 4
%%%%%%%%%%%%%%%%%%%%%%%%%%%%%%%%%%%%%%%%%%%%%%%%%%%%%%%%%%%%%%%%%%%%%%%%%%%%%%

The number of claims N on a portfolio of insurance policies follows a binomial
distribution with parameters n and p. Individual claim amounts follow an exponential
distribution with mean 1/ \lambda . The insurer has in place an individual excess of loss
reinsurance arrangement with retention M.
(i)
Derive an expression, involving M and \lambda, for the probability that an individual
claim involves the reinsurer.

Let I i be an indicator variable taking the value 1 if the ith claim involves the reinsurer
and 0 otherwise.
(ii)
Evaluate the moment generating function M I i ( t ) .
Let K be the number of claims involving the re-insurer so that K = I 1 +
(iii)


+ I N .
(a) Find the moment generating function of K.
(b) Deduce that K follows a binomial distribution with parameters that you
should specify.


%%%%%%%%%%%%%%%%%%%%%%%%%%%%%%%%%%%%%%%%%%%%%%%%%%%%%%%%%%%%%

4
(i)
Let X represent the distribution of individual claims. Let denote the
probability that an individual claim involves the reinsurer. Then
∞
\pi  = P ( X > M ) =
∫ \lambda e
−\lambda x
dx
M
∞
= ⎡ − e −\lambda x ⎤
⎣
⎦ M
= e −\lambda M
(ii) M I i ( t ) = E ( e tI i ) = \pi  e t + 1 − \pi 
(iii) Using the results for the moment generating function of a compound
distribution, we have
M K ( t ) = M N (log M I i ( t ))
(
= pM I i ( t ) + 1 − p
)
n
( )
( )
= p ( \pi  e t + 1 − \pi  ) + 1 − p
= p \pi  e t + p − p \pi  + 1 − p
n
n
Page 3%%%%%%%%%%%%%%%%%%%%%%%%%%%%%%%%%%%%%%%% — April 2010 — Examiners’ Report
(
= p \pi  e t + 1 − p \pi 
)
n
Which is the MGF of a binomial distribution with parameters n and p\pi  .
Hence, by the uniqueness of MGFs K has a binomial distribution with
parameters n and p\pi  .
Comment: This question was well answered.

