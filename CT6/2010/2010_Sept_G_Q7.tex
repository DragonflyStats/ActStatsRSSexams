\documentclass[a4paper,12pt]{article}

%%%%%%%%%%%%%%%%%%%%%%%%%%%%%%%%%%%%%%%%%%%%%%%%%%%%%%%%%%%%%%%%%%%%%%%%%%%%%%%%%%%%%%%%%%%%%%%%%%%%%%%%%%%%%%%%%%%%%%%%%%%%%%%%%%%%%%%%%%%%%%%%%%%%%%%%%%%%%%%%%%%%%%%%%%%%%%%%%%%%%%%%%%%%%%%%%%%%%%%%%%%%%%%%%%%%%%%%%%%%%%%%%%%%%%%%%%%%%%%%%%%%%%%%%%%%

\usepackage{eurosym}
\usepackage{vmargin}
\usepackage{amsmath}
\usepackage{graphics}
\usepackage{epsfig}
\usepackage{enumerate}
\usepackage{multicol}
\usepackage{subfigure}
\usepackage{fancyhdr}
\usepackage{listings}
\usepackage{framed}
\usepackage{graphicx}
\usepackage{amsmath}
\usepackage{chngpage}
%\usepackage{bigints}
\usepackage{vmargin}

% left top textwidth textheight headheight

% headsep footheight footskip
\setmargins{2.0cm}{2.5cm}{16 cm}{22cm}{0.5cm}{0cm}{1cm}{1cm}
\renewcommand{\baselinestretch}{1.3}
\setcounter{MaxMatrixCols}{10}



\begin{document}
An insurance company has a portfolio of policies under which individual loss amounts follow an exponential distribution with mean $1/ \lambda$  . There is an individual excess of loss reinsurance arrangement in place with retention level 100. In one year,
the insurer observes:
\begin{enumerate}
\item 85 claims for amounts below 100 with mean claim amount 42; and
\item 39 claims for amounts above the retention level.
\end{enumerate}

\begin{enumerate}[(a)]
\item (i) Calculate the maximum likelihood estimate of \lambda  .
\item (ii) Show that the estimate of \lambda  produced by applying the method of moments to
the distribution of amounts paid by the insurer is 0.011164.
\end{enumerate}
% CT6 S2010—3
% [5]
% PLEASE TURN OVER

6
(i)
We must write f(x) in the form:
⎡ x \theta  − b ( \theta  )
⎤
f ( x ) = exp ⎢
+ c ( x ,\phi  ) ⎥
⎣ a (\phi  )
⎦
For some parameters \theta  ,\phi  and functions a,b and c.
\alpha  \alpha 
f ( x ) =
μ \alpha  Γ ( \alpha  )
x
\alpha − 1
e
−
x \alpha 
μ
⎡ ⎛ x
⎤
⎞
= exp ⎢ ⎜ − − log μ ⎟ \alpha  + ( \alpha  − 1) log x + \alpha  log \alpha  − log Γ ( \alpha  ) ⎥
⎠
⎣ ⎝ μ
⎦
Which is of the required form with:
\theta =−
1
μ
φ=\alpha 
a (\phi  ) =
1
φ
% Page 5Subject CT6 (Statistical Methods Core Technical) — September 2010 — Examiners’ Report
b ( \theta  ) = − log( −\theta  ) = log μ
c ( x ,\phi  ) = (\phi  − 1) log x +\phi  log\phi  − log Γ (\phi  )
(ii)
The mean and variance for members of the exponential family are given by
b '( \theta  ) and a (\phi  ) b ''( \theta  ) .
In this case b '( \theta  ) = −
1
=μ
\theta 
b ''( \theta  ) = \theta  − 2 = μ 2 so the variance is μ 2 / \alpha  as required.
Generally well answered, though many candidates did not score full marks on part (i)
because they failed to specify all the parameters involved.
7
(i)
First note that the probability of a claim exceeding 100 is e − 100 \lambda  .
The likelihood function for the given data is:
L = C × \lambda  85 e − 85 × 42 ×\lambda  × ( e − 100 ×\lambda  ) 39
Where C is some constant. Taking logarithms gives
l = log L = C ' + 85 log \lambda  − 85 × 42 × \lambda  − 100 × 39 × \lambda 
Differentiating with respect to \lambda  gives
∂ l 85
=
− 85 × 42 − 100 × 39
∂\lambda  \lambda 
Setting this expression equal to zero we get:
\lambda  ˆ =
85
= 0.011379
85 × 42 + 100 × 39
And this gives a maximum since
Page 6
∂ 2 l
∂\lambda 
2
=−
85
\lambda  2
< 0
% Subject CT6 (Statistical Methods Core Technical) — September 2010 — Examiners’ Report
(ii)
We must first calculate the mean amount paid by the insurer per claim. This is
100
∫
0
100
x \lambda  e −\lambda  x dx + 100 P ( X > 100) = ⎡ − xe −\lambda  x ⎤ +
⎣
⎦ 0
= − 100 e
=
− 100 \lambda 
(
100
∫ e
−\lambda  x
dx + 100 e − 100 \lambda 
0
100
⎡ 1
⎤
+ ⎢ − e −\lambda  x ⎥ + 100 e − 100 \lambda 
⎣ \lambda 
⎦ 0
1
1 − e − 100 \lambda 
\lambda 
)
So we must show the given value of \lambda  results in the actual average paid by the
85 × 42 + 39 × 100
insurer. This is
= 60.24
85 + 39
Substituting for \lambda  in the expression derived above, we get
(
)
1
0.6725435
1 − e − 100 × 0.0111654 =
= 60.24 as required.
0.011164
0.011164
Stronger candidates scored well on this question, whereas the weaker candidates struggled
with the calculations required in part (ii).
\end{document}
