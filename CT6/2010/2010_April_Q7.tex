\documentclass[a4paper,12pt]{article}

%%%%%%%%%%%%%%%%%%%%%%%%%%%%%%%%%%%%%%%%%%%%%%%%%%%%%%%%%%%%%%%%%%%%%%%%%%%%%%%%%%%%%%%%%%%%%%%%%%%%%%%%%%%%%%%%%%%%%%%%%%%%%%%%%%%%%%%%%%%%%%%%%%%%%%%%%%%%%%%%%%%%%%%%%%%%%%%%%%%%%%%%%%%%%%%%%%%%%%%%%%%%%%%%%%%%%%%%%%%%%%%%%%%%%%%%%%%%%%%%%%%%%%%%%%%%

\usepackage{eurosym}
\usepackage{vmargin}
\usepackage{amsmath}
\usepackage{graphics}
\usepackage{epsfig}
\usepackage{enumerate}
\usepackage{multicol}
\usepackage{subfigure}
\usepackage{fancyhdr}
\usepackage{listings}
\usepackage{framed}
\usepackage{graphicx}
\usepackage{amsmath}
\usepackage{chng%%-- Page}
%\usepackage{bigints}
\usepackage{vmargin}

% left top textwidth textheight headheight

% headsep footheight footskip
\setmargins{2.0cm}{2.5cm}{16 cm}{22cm}{0.5cm}{0cm}{1cm}{1cm}
\renewcommand{\baselinestretch}{1.3}
\setcounter{MaxMatrixCols}{10}
\begin{document} 

%%- Question 7
%%%%%%%%%%%%%%%%%%%%%%%%%%%%%%%%%%%%%%%%%%%%%%%%%%%%%%%%%%%%%%%%%%%%%%%%%%%%%%
The truncated exponential distribution on the interval (0, c) is defined by the
probability density function:
f ( x ) = ae −\lambda x for 0 < x < c
where \lambda is a parameter and a is a constant.
(i) Derive an expression for a in terms of \lambda and c.

(ii) Construct an algorithm for generating samples from this distribution using the
inverse transform method.

% --------------------------------------------------------%

%- Part 3

Suppose that $0 < \lambda < c$ and consider the truncated Normal distribution defined by the
probability density function:


\[  g ( x ) = \frac{  e^{ − x^2 / 2} }{ \sqrt{2 \pi}  \left[ \Phi ( c ) − 0.5 \right] }   \]


for $0 < x < c$

where $ \Phi ( z )$ is the cumulative density function of the standard Normal distribution.
(iii)
Extend the algorithm in (ii) to use samples generated from $f(x)$ to produce
samples from $g(x)$ using the acceptance / rejection method.




%%%%%%%%%%%%%%%%%%%%%%%%%%%%%%%%%%%%%%%%%%%%%%%%%%%%%%%%%%%%%%%%%%%%%%%%%%%

7
(i)
c
7
(i)

%%-- 1

\[ \int^{c}_{0} a e^{-lambda x}dx \;=\; 1\]

\begin{eqnarray*}
\int^{c}_{0} a e^{-lambda x}dx &=& \left[ - \frac{a}{\lambda} e^{-lambda x} \left] ^{c}_{0} \\
&=& \frac{a}{\lambda} \left( 1 \;-\; e^{-lambda c} \right) \\
\edn{eqnarray*}

So \[ a \;=\; \frac{\lambda}{ 1 \;-\; e^{-lambda c}} \]

%-------------------%

(ii)
The distribution function is
\begin{eqnarray}
F(x) &=& \int^{x}_{0} ae^{ −\lambda y} dy \\
&=& \left[ − \frac{a}{\lambda} e^{ −\lambda y} \right]^{x}_{0} \\
&=& \frac{a}{\lambda} \left(1 − e^{ −\lambda x} \right) \\
&=& \frac{1 − e^{ −\lambda x}}{1 − e^{ −\lambda c}\\
\end{eqnarray*}




\begin{array*} 
F(x) &=& \int^{x}_{0} a e^{-lambda y}dy \\
&=& \left[  - \frac{a}{\lambda}  e^{-lambda y} \right]^{x}_{0}\\
&=& \frac{a}{\lambda} \left( 1 \;-\; e^{-lambda cx \right) \\
&=& a \frac{1}{\lambda} \left( 1 \;-\; e^{-lambda cx \right) \\
&=& \left(\frac{\lambda}{ 1 \;-\; e^{-lambda c}}\right) \times \frac{1}{\lambda} \left( 1 \;-\; e^{-lambda cx \right) \\
&=& \left(\frac{1 \;-\; e^{-lambda x}}}{ 1 \;-\; e^{-lambda c}}\right)
\end{eqnarray*}

%%%%%%%%%%%%%%%%%%%%%%%%%%%%
%--------------------------%

The required transformation is therefore given by:
$u = F ( x )$
\[1 − e^{ −\lambda x} = u (1 − e^{ −\lambda c}) \]

\[x = − frac{(log 1 − u (1 − e^{ −\lambda c})
) } {\lambda} \]

%------------%

So the algorithm is:
•
•
Generate u from U(0, 1)
Set x = −
(
log 1 − u (1 − e − c \lambda )
)
\lambda
g ( x )
(1 − e − c \lambda ) e − x / 2 +\lambda x
= Max
We need to find M = Max
2 \pi  [ Φ ( c ) − 0.5] \lambda
0 < x < c f ( x )
0 < x < c
2
(iii)
Let h ( x ) = e − x /2 +\lambda x then we can simply find the maximum of h(x) since it
differs only by a constant.
2
Then h '( x ) = h ( x ) \times ( − x + \lambda )
So
h '( x ) = 0 when x = \lambda
h ''( x ) = h '( x )( \lambda − x ) − h ( x )
And so h ''( \lambda ) = − h ( \lambda ) < 0 so we do have a maximum.
(1 − e − c \lambda ) e \lambda / 2
Hence M =
2 \pi  [ Φ ( c ) − 0.5] \lambda
2
Page 6%%%%%%%%%%%%%%%%%%%%%%%%%%%%%%%%%%%%%%%% — April 2010 — Examiners’ Report
And so
2
2
g ( x )
= e − x /2 +\lambda x −\lambda /2
Mf ( x )
So the algorithm is:
•
Generate u from U(0,1)
(
log 1 − u (1 − e − c \lambda )
• Set x = −
• Generate v from U(0,1)
• If v < e − x
)
\lambda
2
/2 +\lambda x −\lambda 2 /2
return x as the random sample, otherwise begin again.
Comment: This was found particularly hard especially part (iii), where some harder
calculations are involved. Many candidates just described the general theory of rejection
algorithms without being able to apply it explicitly here.
