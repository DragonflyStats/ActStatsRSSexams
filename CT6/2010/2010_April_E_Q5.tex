\documentclass[a4paper,12pt]{article}

%%%%%%%%%%%%%%%%%%%%%%%%%%%%%%%%%%%%%%%%%%%%%%%%%%%%%%%%%%%%%%%%%%%%%%%%%%%%%%%%%%%%%%%%%%%%%%%%%%%%%%%%%%%%%%%%%%%%%%%%%%%%%%%%%%%%%%%%%%%%%%%%%%%%%%%%%%%%%%%%%%%%%%%%%%%%%%%%%%%%%%%%%%%%%%%%%%%%%%%%%%%%%%%%%%%%%%%%%%%%%%%%%%%%%%%%%%%%%%%%%%%%%%%%%%%%

\usepackage{eurosym}
\usepackage{vmargin}
\usepackage{amsmath}
\usepackage{graphics}
\usepackage{epsfig}
\usepackage{enumerate}
\usepackage{multicol}
\usepackage{subfigure}
\usepackage{fancyhdr}
\usepackage{listings}
\usepackage{framed}
\usepackage{graphicx}
\usepackage{amsmath}
\usepackage{chng%%-- Page}
%\usepackage{bigints}
\usepackage{vmargin}

% left top textwidth textheight headheight

% headsep footheight footskip
\setmargins{2.0cm}{2.5cm}{16 cm}{22cm}{0.5cm}{0cm}{1cm}{1cm}
\renewcommand{\baselinestretch}{1.3}
\setcounter{MaxMatrixCols}{10}
\begin{document} 

%%- Question 5
%%%%%%%%%%%%%%%%%%%%%%%%%%%%%%%%%%%%%%%%%%%%%%%%%%%%%%%%%%%%%%%%%%%%%%%%%%%%%%

[Total 7]

%%- Question 5
An insurance company has issued life insurance policies to 1,000 individuals. Each
life has a probability q of dying in the coming year. In a warm year, q = 0.001 and in
a cold year q = 0.005. The probability of a warm year is 50% and the probability of a
cold year is 50%. Let N be the aggregate number of claims across the portfolio in the
coming year.
\begin{enumerate}
\item (i) Calculate the mean and variance of N.
\item (ii) Calculate the alternative values for the mean and variance of N assuming that
q is a constant 0.003.
\item 
(iii) Comment on the results of (i) and (ii).

\end{enumerate}
%%%%%%%%%%%%%%%%%%%%%5

%%-- CT6 April 2010 .... Question 5
(i)
\begin{eqnarray*}
E ( N ) &=& E \left[ E ( N | q ) \right]\\
&=& E \left[ 1000 q \right]\\
&=& 1000 \times ( 0.5 \times 0.001 + 0.5 \times 0.005 )\\
&=& 1000 \times 0.003 \\
&=& 3\\
\end{eqnarray*}

\begin{eqnarray*}
Var ( N ) 
&=& Var \left[ E ( N |q ) \right] + E  \left[ Var ( N| q ) \right]\\
&=& Var (1000 q ) + E (1000 q (1 − q )) \\
\end{eqnarray*}

Now
\begin{itemize}
\item $E ( q ) = 0.003$ and 
\item $E ( q 2 ) = 0.5 \times 0.001 2 + 0.5 \times 0.005 2 = 0.000013$
\end{itemize}

%----------------------%
So 

\begin{eqnarray*}
Var ( q ) &=& 0.000013 − 0.003^2 \\
&=& 0.000004\\
\end{eqnarray*}

\begin{eqnarray*}
E ( q (1 − q )) &=&  0.5 \times 0.001 \times 0.999 + 0.5 \times 0.005 \times 0.995 \\ &=&  0.002987  \\
\end{eqnarray*}



So Var ( N ) = 1000 2 \times 0.000004 + 1000 \times 0.002987 = 6.987
(ii)
In this case N ~ B (1000, 0.003) and so
E ( N ) = 1000 \times 0.003 = 3
and
Var ( N ) = 1000 \times 0.003 \times 0.997 = 2.991
(iii)
Page 4
The simplification in (ii) results in the same mean number of deaths, but a
very significantly lower variance.%%%%%%%%%%%%%%%%%%%%%%%%%%%%%%%%%%%%%%%% — April 2010 — Examiners’ Report
This is because in (i) there is a tendency for deaths to occur at the same time,
or not at all (as a result of the weather) whereas in (ii) deaths are genuinely
independent.
Comment: Many good answers here although some candidates struggled to articulate why
the variance in (ii) was lower.

\end{document}
