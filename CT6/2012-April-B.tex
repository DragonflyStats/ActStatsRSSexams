\documentclass[a4paper,12pt]{article}

%%%%%%%%%%%%%%%%%%%%%%%%%%%%%%%%%%%%%%%%%%%%%%%%%%%%%%%%%%%%%%%%%%%%%%%%%%%%%%%%%%%%%%%%%%%%%%%%%%%%%%%%%%%%%%%%%%%%%%%%%%%%%%%%%%%%%%%%%%%%%%%%%%%%%%%%%%%%%%%%%%%%%%%%%%%%%%%%%%%%%%%%%%%%%%%%%%%%%%%%%%%%%%%%%%%%%%%%%%%%%%%%%%%%%%%%%%%%%%%%%%%%%%%%%%%%

\usepackage{eurosym}
\usepackage{vmargin}
\usepackage{amsmath}
\usepackage{graphics}
\usepackage{epsfig}
\usepackage{enumerate}
\usepackage{multicol}
\usepackage{subfigure}
\usepackage{fancyhdr}
\usepackage{listings}
\usepackage{framed}
\usepackage{graphicx}
\usepackage{amsmath}
\usepackage{chngpage}

%\usepackage{bigints}
\usepackage{vmargin}

% left top textwidth textheight headheight

% headsep footheight footskip

\setmargins{2.0cm}{2.5cm}{16 cm}{22cm}{0.5cm}{0cm}{1cm}{1cm}

\renewcommand{\baselinestretch}{1.3}

\setcounter{MaxMatrixCols}{10}

\begin{document}

\begin{enumerate}
Claim amounts on a certain type of insurance policy follow a distribution with density
f ( x ) = 3 cx 2 e − cx for x > 0
3
where c is an unknown positive constant. The insurer has in place individual excess
of loss reinsurance with an excess of 50. The following ten payments are made by the
insurer:
Losses below the retention: 23, 37, 41, 11, 19, 33
Losses above the retention: 50, 50, 50, 50
Calculate the maximum likelihood estimate of c.
4
[6]
Claims on a particular type of insurance policy follow a compound Poisson process
with annual claim rate per policy 0.2. Individual claim amounts are exponentially
distributed with mean 100. In addition, for a given claim there is a probability of 30%
that an extra claim handling expense of 30 is incurred (independently of the claim
size). The insurer charges an annual premium of 35 per policy.
Use a normal approximation to estimate how many policies the insurer must sell so
that the insurer has a 95% probability of making a profit on the portfolio in the year.
[6]
%%%%%%%%%%%%%%%%%%%%%%%%%%%%%%%%%%%%%%%%%%%%%%%%%%%%%%%%%%%%%%%%%%%%%%%%%%%%%%%%%%
3
The likelihood function is given by:
6
3
L = D × ∏ 3 cx i 2 e − cx i × e − 4 × 50
3
c
i = 1
where D is a constant.
Where the x i are the claims below the retention.
6
6
l = log L = log D + ∑ log 3 cx i − c ∑ x i 3 − 4 × 50 3 c
2
i = 1
i = 1
6 6
i = 1 i = 1
= log D + 6 log 3 + 6 log c + ∑ log x i 2 − c ∑ x i 3 − 4 × 50 3 c
Differentiating we get
dl 6 6 3
= − ∑ x i − 500000
dc c i = 1
Page 4
%%%%%%%%%%%%%%%%%%%%%%%%%
 (Statistical Methods) – April 2012 – Examiners’ Report
So our estimate is given by
c ˆ =
6
6
∑ i = 1 x 1 3 + 500000
=
6
= 8.8775 × 10 − 6
175868 + 500000
This question was answered well.
4
Let the individual total claim costs be denoted by X. Then X=Y+Z where Y is the cost
of the claim and Z is the claim handling expense.
Then
E ( X ) = E ( Y ) + E ( Z ) = 100 + 0.3 × 30 = 109
And
( ) (
) ( )
E X 2 = E Y 2 + 2 YZ + Z 2 = E Y 2 + 2 E ( Y ) E ( Z ) + E ( Z 2 )
Using the independence of Y and Z. Now
( )
E Y 2 = 2 E ( Y ) 2 = 2 × 100 2 = 20000
and
( )
E Z 2 = 0.3 × 30 2 = 270
So that
( )
E X 2 = 20000 + 2 × 100 × 9 + 270 = 22070 = 148.56 2
Now if there are n policies in the portfolio, total claim amounts S will have an
approximately Normal distribution with mean 0.2 × n × 109 = 21.8 n and variance
0.2 × n × 148.56 2 .
The premium income will be 35n.
We need to solve for n in the following equation:
( (
)
)
P N 21.8 n , 66.44 2 n > 35 n < 0.05
i.e.
13.2 n ⎞
⎛
P ⎜ N ( 0,1 ) >
⎟ < 0.05
66.44 n ⎠
⎝
Page 5
%%%%%%%%%%%%%%%%%%%%%%%%%
 (Statistical Methods) – April 2012 – Examiners’ Report
So
0.198675496 n > 1.6449
n > 68.55
i.e. at least 69 policies must be sold.
Most candidates struggled with this question. Many did not calculate the variance correctly
and a lot did not correctly use the number of policies, n, as a multiplier for the mean and
variance of the claims. Others used n and the claim rate when calculating the additional
claim handling expense.
