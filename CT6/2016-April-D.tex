7 Claims on a portfolio of insurance policies arrive as a Poisson process with
parameter λ, claim amounts having a Normal distribution with mean μ and variance
σ2, and there is a loading θ on premiums. The insurance company has an initial
surplus of U.
(i) Explain carefully the meaning of (U), (U,t) and (U,1). [3]
(ii) State four factors that affect the size of (U,t), for a given t. [2]
(iii) Explain, for each factor, what happens to (U,t) when the factor increases. [4]
Sarah, the insurance company’s actuary, prefers to consider the probability of ruin
in discrete rather than continuous time.
(iv) Explain an advantage and disadvantage of Sarah’s approach. [2]
[Total 11]
CT6 A2016–5 PLEASE TURN OVER
8 (i) Show that:
  2
2 2
1 (ln ) 2 2
2 ½
2
1 ln ln
2
b x
a
e dx e b a
 
  
           
                
 . [4]
A general insurance company writes claims, whose amounts have a lognormal
distribution, with mean 300 and standard deviation 400. The insurance company
purchases excess of loss reinsurance with retention 500 per claim.
(ii) Calculate the average expected claim size payable by the insurance
company. [6]
Next year, claim inflation is 10%, but the retention amount remains the same.
(iii) Explain whether the average expected claim size payable by the insurance
company next year would increase by 10%. [2]
[Total 12]


Page 7
Q7 (i) Ψ(U) = P(U(t) < 0) , for some t, 0 < t < ∞ [1]
Ψ(U,t) = P(U(τ) < 0) , for some τ, 0 < τ ≤ t [1]
Ψ(U,1) = P(U(t) < 0) , for some t, 0 < t ≤1 [1]
[Total 3]
(ii) λ, μ, σ2, θ and initial surplus U [½ each – Max 2]
(iii) higher λ increases Ψ(U, t) as the process is faster – claims and premiums
come in quicker
higher μ increases Ψ(U, t) as claims amounts are larger, relative to the surplus
held
higher θ reduces Ψ(U, t) as premiums increase at a quicker rate, so more of a
buffer
higher U reduces Ψ(U, t) as more of a buffer to withstand claims
higher σ2 will typically increase Ψ(U, t), assuming that expected premiums
are higher than expected claims, since the likelihood of more
extreme claims increase, [but may reduce Ψ(U, t) if expected claims
                          are higher than expected premiums.]
[1 each – Max 4]
(iv) Advantage – easier to measure, more useful for reporting [1]
Disadvantage – less information, artificial, can miss time when ruin occurs [1]
[Total 2]
[TOTAL 11]
This straightforward question on Ruin Theory was very well answered by
most candidates.
Q8 (i) I =
  2
2
1 (ln )
2
2
1
2
b x
a
e dx
− −μ
σ
πσ

Put y lnx, so dy = 1 dx
x
and dx = eydy [1]
I =
  2
2
1 ( ) ln 2
ln 2
1
2
b y y
a
e edy
− −μ
σ
πσ
 [½]
=
  ln 2 2 2
ln 2 2
1 exp 1 ( 2 2 )
2 2
b
a
− y − μy + μ − yσ dy
πσ  σ 
 [½]
Subject CT6 (Statistical Methods Core Technical) – April 2016 – Examiners’ Report
Page 8
=
  ln 2 2 2 4
ln 2 2
1 exp 1 (( ) 2 )
2 2
b
a
− y −μ −σ − μσ −σ dy
πσ  σ 
 [1]
=
  2 2
2 2
1 ( ) ½ ln 2
ln 2
1
2
b y
a
e e dy
−μ−σ
μ+ σ σ
πσ  [½]
= 2 2 2
eμ+½σ ln b ln a   − μ − σ   − μ − σ   Φ  −Φ     σ   σ  
[½]
[Total 4]
(ii)
e(μ+σ2 /2) = 300
e(2μ+σ2 ) (eσ2 −1) = 4002
2 2
2
2
400 25 ( 1) ln9 300
eσ − = → σ =  
 
[1]
μ = ln 300 − σ2 / 2 = ln180 [1]
Average claim payable is
E(X|X < 500) + 500 * P(X ≥ 500) [1]
( )
ln 500 2 E(X|X 500) 300 * 300 * 0 150
 − μ − σ  < = Φ   = Φ =  σ 
[1]
P(X ≥ 500) = 1−Φ  ln 500 − μ  = 0.1561  σ 
[1]
So average claim is 150 + 500 * 0.1561 = 228.0 (4sf) [1]
[Total 6]
(iii) The insurance company’s expected claims would increase by less than 10%,
[½]
since the chances of high claims has increased due to the standard deviation
remaining the same, hence the reinsurer will pick up a greater share of the
claims. [1½]
[Total 2]
[TOTAL 12]
Candidates who had learned the bookwork underlying part (i) were able to
score well here. Most candidates did well on parts (ii) and (iii).
Subject CT6 (Statistical Methods Core Technical) – April 2016 – Examiners’ Report
