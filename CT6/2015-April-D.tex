

The following time series model is being used to model monthly data:
Y t  Y t  1  Y t  12  Y t  13  e t   1 e t  1   12 e t  12   1  12 e t  13
where e t is a white noise process with variance  2 .
8
(i) Perform two differencing transformations and show that the result is a moving
average process which you may assume to be stationary.
[3]
(ii) Explain why this transformation is called seasonal differencing.
(iii) Derive the auto-correlation function of the model generated in part (i).
[8]

%%%%%%%%%%%%%%%%%%%%%%%%%%%%%%%%%%%%%%%%%%%%%%%%%%%%%%%%%%%%%%%%%%%%%%%%%%%%%%%%%%%%%%%%%%%%%%%%%%%%%%%

[Total 12]
[1]
The number of claims, N, in a given year on a particular type of insurance policy is
given by:
P(N = n) = 0.8  0.2 n
n = 0, 1, 2, ...
Individual claim amounts are independent from claim to claim and follow a Pareto
distribution with parameters  = 5 and  = 1,000.
(i) Calculate the mean and variance of the aggregate annual claims per policy. [4]
(ii) Calculate the probability that aggregate annual claims exceed 400 using:
(a)
(b)
a Normal approximation.
a Lognormal approximation.
[6]
(iii)
CT6 A2015–4
Explain which approximation in part (ii) you believe is more reliable.
[2]
[Total 12]9
Let p be an unknown parameter and let f(px) be the probability density of the
posterior distribution of p given information x.
(i)
Show that under all-or-nothing loss the Bayes estimate of p is the mode of
f(px).
[2]
John is setting up an insurance company to insure luxury yachts. In year 1 he will
insure 100 yachts and in year 2 he will insure 100 + g yachts where g is an integer.
If there is a claim the insurance company pays a fixed sum of $1m per claim.
The probability of a claim on a policy in a given year is p. You may assume that the
probability of more than one claim on a policy in any given year is zero. Prior beliefs
about p are described by a Beta distribution with parameters  = 2 and  = 8.
In year 1 total claims are $13m and in year 2 they are $20m.
10
(ii) Derive the posterior distribution of p in terms of g.
[4]
(iii) Show that it is not possible in this case for the Bayes estimate of p to be the
same under quadratic loss and all-or-nothing loss.
[6]
[Total 12]
Claims on a certain portfolio of insurance policies arise as a Poisson process with
annual rate . Individual claim amounts are independent from claim to claim and
follow an exponential distribution with mean . The insurance company has
purchased excess of loss reinsurance with retention M from a reinsurer who calculates
premiums using a premium loading of . Denote by X i the amount paid by the
reinsurer on the i th claim (so that X i = 0 if the i th claim amount is below M).
(i)
(ii)
Explain why the claims arrival process for the reinsurer is also a Poisson
process and specify its parameter.
Show that
M X i ( t )  1  e
(iii)
(iv)
 M


 t
.
1   t
[4]
(a) Determine E(X i ).
(b) Write down and simplify the equation for the reinsurer’s adjustment
coefficient.
[6]
Comment on your results to part (iii).
END OF PAPER
CT6 A2015–5
[3]
[2]
[Total 15]

7
(i)
Set X t = (1  B 12 )(1  B) Y t where B is the background shift operator
i.e. X t = Y t  Y t 1  Y t  12 + Y t 13
then we have X t = e t +  1 e t 1 +  12 e t 12 +  1  12 e t 13
= (1 +  1 B)(1 +  12 B 12 )e t
which is a moving average process [of order 13].
(ii)
This is called seasonal differencing because it compares the monthly change in
Y t with the corresponding monthly change at the same time last year.
Page 9Subject CT6 (Statistical Methods Core Technical) – April 2015 – Examiners’ Report
(iii)
We can see that
 0 = Cov(X t , X t ) = (1   1 2   12 2   1 2  12 2 )  2 = (1   1 2 )(1   12 2 )  2
 1 = Cov(X t , X t 1 ) = Cov(e t +  1 e t 1 +  12 e t 12 +  1  12 e t 13 ;
e t 1 +  1 e t 2 +  12 e t 13 +  1  12 e t 14 )
= (  1   1  12 2 )  2   1 (1   12 2 )  2
 11 = Cov(X t , X t 11 ) = Cov(e t +  1 e t 1 +  12 e t 12 +  1  12 e t 13 ;
e t 11 +  1 e t 12 +  12 e t 23 +  1  12 e t 24 )
  1  12  2
 12 = (  12   1 2  12 )  2   12 (1   1 2 )  2
 13   1  12  2
and  s = 0 for all other values of s.
so  1 =
2
 1 (1   12
)
2
(1   1 2 )(1   12
)
 11 =  13 =
 12 =
=
 1
1   1 2
 1  12
2
(1   1 2 )(1   12
)
 12 (1   1 2 )
2
(1   1 2 )(1   12
)
=
 12
2
1   12
and  0 = 1 and  s = 0 for all other s.
Most candidates were able to identify this as a moving average process, however only the
strongest candidates were able to work through the algebra to derive the auto-correlation
function.
Page 10Subject CT6 (Statistical Methods Core Technical) – April 2015 – Examiners’ Report
8
(i)
First note that N has a type 2 negative binomial distribution with parameters
p = 0.8 and k = 1. Hence
E(N) =
0.2
= 0.25
0.8
Var(N) =
0.2
0.8 2
= 0.3125
Let X denote the distribution of an individual claim. Then
E ( X ) =

1000
=
= 250
4
  1
Var( X ) = 250 2 ×
5
= 104,166.666 = (322.75) 2
3
Now let S denote aggregate annual claims. Then
E ( S )
= E ( N ) E ( X ) = 0.25 × 250 = 62.5
Var( S ) = E ( N ) Var( X ) + Var( N ) E ( X ) 2
= 0.3125 × 250 2 + 0.25 × 104,166.666
= 45,572.92 = 213.478 2
(ii)
(a)
P ( S > 400) = P ( N (62.5, 213.478 2 ) > 400)
400  62.5 

= P  N (0,1) 

213.478 

= P ( N (0,1) > 1.581)
= 1  [0.94295 × 0.9 + 0.1 × 0.94408]
= 0.0569
(b)
Let  and  be the parameters of the underlying Normal distribution.
Then
e

 2
2
= 62.5
e 2  ( e   1) = 213.478 2
2
2
(A)
(B)
Page 11Subject CT6 (Statistical Methods Core Technical) – April 2015 – Examiners’ Report
(B) ÷ (A) 2  e   1 =
2
213.478 2
62.5 2
= 11.66665
 2 = log12.66665 = 2.53897 = 1.5934 2
substituting into (A)  + 2.53897 = log62.5
2
so  = log62.5  2.53897 = 2.8657
2
and so P ( S > 400) = P ( N (2.8657, 1.5934 2 ) > log 400)
log 400  2.8657 

= P  N (0,1) 

1.5934


= P ( N (0,1) > 1.9617)
= 1  (0.17 × 0.97558 + 0.83 × 0.97500)
= 0.0249
(iii)
The Pareto distribution is significantly skewed and the Normal approximation
is not. The Normal approximation in (ii)(b) has variance 213.48 2 and mean
62.5, so negative values of S (which are impossible in reality) are less than 1
standard deviation from the mean.
The approximation in (ii)(b) will therefore be more reliable.
This question was well answered by the majority of candidates. Full credit was given to
alternative correct answers in part (iii).
