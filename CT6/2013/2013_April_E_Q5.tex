\documentclass[a4paper,12pt]{article}

%%%%%%%%%%%%%%%%%%%%%%%%%%%%%%%%%%%%%%%%%%%%%%%%%%%%%%%%%%%%%%%%%%%%%%%%%%%%%%%%%%%%%%%%%%%%%%%%%%%%%%%%%%%%%%%%%%%%%%%%%%%%%%%%%%%%%%%%%%%%%%%%%%%%%%%%%%%%%%%%%%%%%%%%%%%%%%%%%%%%%%%%%%%%%%%%%%%%%%%%%%%%%%%%%%%%%%%%%%%%%%%%%%%%%%%%%%%%%%%%%%%%%%%%%%%%

\usepackage{eurosym}
\usepackage{vmargin}
\usepackage{amsmath}
\usepackage{graphics}
\usepackage{epsfig}
\usepackage{enumerate}
\usepackage{multicol}
\usepackage{subfigure}
\usepackage{fancyhdr}
\usepackage{listings}
\usepackage{framed}
\usepackage{graphicx}
\usepackage{amsmath}
\usepackage{chngpage}

%\usepackage{bigints}
\usepackage{vmargin}

% left top textwidth textheight headheight

% headsep footheight footskip

\setmargins{2.0cm}{2.5cm}{16 cm}{22cm}{0.5cm}{0cm}{1cm}{1cm}

\renewcommand{\baselinestretch}{1.3}

\setcounter{MaxMatrixCols}{10}

\begin{document}
CT6 A2013–25
The following table shows incremental claims data from a portfolio of insurance
policies for the accident years 2010, 2011 and 2012. Claims from this type of policy
are fully run off after the end of development year two.
Incremental
Claims
Accident year
2010
2011
2012
Development year
0
1
2
2,328
1,749
2,117
1,484
1,188
384
Estimate the total claims outstanding using the basic chain ladder technique.

\newpage
%%%%%%%%%%%%%%%%%
5
First accumulate claims:
Cumulative
Claims
Accident year
2010
2011
2012
0
2,328
1,749
2,117
Development year
1
2
3,812
2,937
4,196
DY1 = (3,812 + 2,937) / (2,328 + 1,749) = 1.655 384
DY2 = 4,196 / 3,812 = 1.100 735
Now complete lower half of table:
Cumulative
Claims
Accident year
2009
2011
2012
0
2,328
1,749
2,117
Development year
1
2
3,812
2,937
3,504.45
4,196
3,232.86
3,857.47
So estimated amount of outstanding claims is:
(3,232.86 – 2,937) + (3,857.47 – 2,117) = 2,036.3.
Most candidates scored full marks on this straightforward application of chain ladder theory.
Page 5Subject %%%%%%%%%%%%%%%%%%%%%%%%%%%%%%%%%%%%%5 – April 


\end{document}
