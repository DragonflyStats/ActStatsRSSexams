PLEASE TURN OVER7
An insurance company believes that individual claim amounts from house insurance
policies follow a gamma distribution with distribution function given by:
f ( y ) =
\alpha \alpha
μ \alpha Γ( \alpha )
y
\alpha− 1
e
\alpha
− y
μ
for y > 0
where \alpha and μ are positive parameters.
(i)
Show that the gamma distribution can be written in exponential family form,
giving the natural parameter and the canonical link function.

The insurance company has data for claim amounts from previous claims. It believes
that the claim amount is primarily influenced by two variables:
x i the type of geographical area in which the house is situated. This can take
one of 4 values.
y i the category of the age of the house where the three categories are 0–29
years, 30-59 years and 60 years +.
It wishes to model claim amounts using this data and the generalised linear model
from part (i) with canonical link function. The insurance company is investigating
models which take into account these variables and has the following table of values:
Model
A
B
C
D
(ii)
CT6 A2013–4
Choice of
predictor
1
Age
Age +location
Age * location
Scaled Deviance
900
789
544
541
Explain, by analysing the scaled deviances, which model the insurance
company should use.

[Total 11]8
An insurance company has a portfolio of 1,000 car insurance policies. Claims arise
on individual policies according to a Poisson process with annual rate μ . The
insurance company believes that μ follows a gamma distribution with parameters
\alpha = 2 and \lambda = 8.
(i)
(a) Show that the average annual number of claims per policy is 0.25.
(b) Show that the variance of the number of annual claims per policy is
0.28125.

Individual claim amounts follow a gamma distribution with density
− x
x
f ( x ) =
e 1000 for x > 0.
1, 000, 000
(ii)
Calculate the mean and variance of the annual aggregate claims for the whole
portfolio.

The insurance company has agreed an aggregate excess of loss reinsurance contract
with a retention of £0.55m (this means that the reinsurance company will pay the
excess above £0.55m if the aggregate claims on the portfolio in a given year exceed
£0.55m).
(iii)
Calculate, using a Normal approximation, the probability of aggregate claims
exceeding the retention in any year.

For each of the last three years, the total claim amount has in fact exceeded the
retention.
(iv)
CT6 A2013–5
Comment on this outcome in light of the calculation in part (iii).


7
(i)
From the definition of the gamma density given in the question
f ( y ) =
\alpha \alpha
μ \alpha Γ( \alpha )
y
\alpha− 1
e
\alpha
− y
μ
⎡ ⎛ y
⎤
⎞
= exp ⎢ ⎜ − − log μ ⎟ \alpha + ( \alpha − 1 ) log y + \alpha log \alpha − log Γ( \alpha ) ) ⎥
⎠
⎣ ⎝ μ
⎦
⎡ ( y θ − b ( θ ) )
⎤
= exp ⎢
+ c ( y , φ ) ⎥
⎣ a ( φ )
⎦
where:
θ=−
1
μ
φ=\alpha
a ( φ ) =
1
φ
b ( θ ) = − log ( −θ )
c ( y , φ ) = ( φ − 1 ) log y + φ log φ − log Γ( φ ).
Hence the distribution has the right form for a member of an exponential
family.
1
The natural parameter is − 1 . The canonical link function is .
μ
μ
(ii)
Using the information given, we can calculate the deviance differences and
compare that with the differences of the degrees of freedom for each of the
nested models. If the decrease in the deviances is greater than twice the
difference in degrees of freedom this suggests an improvement.
Model Scaled
Deviance Degrees of
freedom
1
Age
Age +location
Age * location 900
789
544
541 12
10
7
1
Difference
in scaled
deviance
111
245
3
Page 7Subject %%%%%%%%%%%%%%%%%%%%%%%%%%%%%%%%%%%%%5 – April 2013 – Examiners’ Report
From the table we can see that the interaction model does not indicate any
improvement hence the recommended model would be Age +location.
Again well prepared candidates were able to score highly on this question, however weaker
candidates dropped marks as a result of not specifying a full parameterisation in part (i). In
part (ii) full credit was given to candidates who used the chi-squared test rather than the
approximation set out above..
8
(i)
(ii)
(a) E(N) = E[E(N|μ)]
= E[μ] = 2/8 = 0.25
(b) var(N) = E[var(N|μ)] + var[E(N|μ)]
= E[μ] + var[μ]
= 2/8 + 2/8 2 = 0.28125
Let Y be aggregate claims from one policy.
Individual claim is gamma with \alpha = 2 and \lambda = 0.001 .
E ( Y ) = E ( X ) E ( N ) = 2000 × 0.25 = 500.
Var ( Y ) = E ( N ) Var ( X ) + Var ( N ) E ( X ) 2
= 0.25 × 2000000 + 9
32
× 2000 2 = 1, 625, 000.
So the mean and variance of total claims are 500,000 and 1,625,000,000
respectively.
(iii)
Our approximate distribution for S is S ~ N(500,000 , 1625000000).
550000 − 500000 ⎞
⎛
P ( S > 550000 ) = P ⎜ Z >
⎟ = P ( Z > 1.24035 ) = 0.1074 .
1625000000 ⎠
⎝
(iv)
The prob three years in a row is 0.1074 3 = 0.00124 .
The probability of this happening is very low. It is more likely that the
insurance company’s belief about the distribution of claims amounts is
incorrect.
The normal approximation tails off quickly and so underestimates the
probability of extreme events
Part (i) was straightforward, however some candidates failed to show sufficient working to
gain full marks. A surprising number of candidates were unfamiliar with the standard
bookwork underlying part (ii). Credit was given for any sensible comments in part (iv).
Page 8Subject %%%%%%%%%%%%%%%%%%%%%%%%%%%%%%%%%%%%%5 – April 2013 – Examiners’ Report
