\documentclass[a4paper,12pt]{article}

%%%%%%%%%%%%%%%%%%%%%%%%%%%%%%%%%%%%%%%%%%%%%%%%%%%%%%%%%%%%%%%%%%%%%%%%%%%%%%%%%%%%%%%%%%%%%%%%%%%%%%%%%%%%%%%%%%%%%%%%%%%%%%%%%%%%%%%%%%%%%%%%%%%%%%%%%%%%%%%%%%%%%%%%%%%%%%%%%%%%%%%%%%%%%%%%%%%%%%%%%%%%%%%%%%%%%%%%%%%%%%%%%%%%%%%%%%%%%%%%%%%%%%%%%%%%

\usepackage{eurosym}
\usepackage{vmargin}
\usepackage{amsmath}
\usepackage{graphics}
\usepackage{epsfig}
\usepackage{enumerate}
\usepackage{multicol}
\usepackage{subfigure}
\usepackage{fancyhdr}
\usepackage{listings}
\usepackage{framed}
\usepackage{graphicx}
\usepackage{amsmath}
\usepackage{chngpage}

%\usepackage{bigints}
\usepackage{vmargin}

% left top textwidth textheight headheight

% headsep footheight footskip

\setmargins{2.0cm}{2.5cm}{16 cm}{22cm}{0.5cm}{0cm}{1cm}{1cm}

\renewcommand{\baselinestretch}{1.3}

\setcounter{MaxMatrixCols}{10}

\begin{document}
6

Claim numbers on a portfolio of insurance policies follow a Poisson process with
parameter $\lambda$ . Individual claim amounts X follow a distribution with moments
\[m i = E ( X i ) for i = 1, 2, 3, ... .\] Let S denote the aggregate claims for the portfolio.
You may assume that the mean of S is \lambda m 1 and the variance of S is \lambda m 2 .
\begin{enumerate}
\item (i)
Derive the third central moment of S and show that the coefficient of
\lambda m 3
skewness of S is
.
3
2
( \lambda m 2 ) 
\item (ii) Show that S is positively skewed regardless of the distribution of X. 

\item (iii) Show that the distribution of S tends to symmetry as $\lambda \rightarrow \infty$ .
\end{enumerate}

%%%%%%%%%%%%%%%%%
\newpage

6
\begin{itemize}
\item (i)
We have:
\[M S ( t ) = M N (log M X ( t )) = e
\lambda ( M_X ( t ) − 1)\]
\item  Let us work with the cumulant generating function:
\[C_{S} ( t ) = log M_S ( t ) = \lambda M X ( t ) − \lambda .\]
The third central moment is given by $C_{S} \prime \prime \prime  (0)$.
Now:
\[C_S \prime \prime \prime  ( t ) = \lambda M \prime \prime \prime  X ( t )\]
and so
\[C_{S} \prime  \prime \prime  ( 0 ) = \lambda M \prime \prime \prime  X ( 0 ) = \lambda m 3 .\]
\item  Hence the coefficient of skewness is given by:
E (( S − E ( S ) ) 3
( Var ( S ) )

\item (ii)
3
\lambda m 3
=
(
2
3
\lambda m 2 2
.
)
( )
\item  Since X takes only positive values we have m 3 = E X 3 > 0.
Both $\lambda$ and $m 2 = E ( X 2 )$ are also always positive.
This means the coefficient of skewness is always positive.
\item  (iii)
Re-writing the equation for the coefficient of skewness we have:
\lambda m 3
(
3
\lambda m 2 2
)
=
m 3
\lambda
0.5
m 2 1.5
→ 0 as \lambda \rightarrow \infty .
\item  Hence the distribution of S tends to symmetry as \lambda \rightarrow \infty .
Well prepared candidates who knew their bookwork were able to answer this question well,
however weaker candidates struggled with part (i) and gave unconvincing answers to part
(ii) & (iii).
Page 6Subject %%%%%%%%%%%%%%%%%%%%%%%%%%%%%%%%%%%%%5 – April 2013 – Examiners’ Report
\end{itemize}
\end{document}
