\documentclass[a4paper,12pt]{article}

%%%%%%%%%%%%%%%%%%%%%%%%%%%%%%%%%%%%%%%%%%%%%%%%%%%%%%%%%%%%%%%%%%%%%%%%%%%%%%%%%%%%%%%%%%%%%%%%%%%%%%%%%%%%%%%%%%%%%%%%%%%%%%%%%%%%%%%%%%%%%%%%%%%%%%%%%%%%%%%%%%%%%%%%%%%%%%%%%%%%%%%%%%%%%%%%%%%%%%%%%%%%%%%%%%%%%%%%%%%%%%%%%%%%%%%%%%%%%%%%%%%%%%%%%%%%

\usepackage{eurosym}
\usepackage{vmargin}
\usepackage{amsmath}
\usepackage{graphics}
\usepackage{epsfig}
\usepackage{enumerate}
\usepackage{multicol}
\usepackage{subfigure}
\usepackage{fancyhdr}
\usepackage{listings}
\usepackage{framed}
\usepackage{graphicx}
\usepackage{amsmath}
\usepackage{chngpage}

%\usepackage{bigints}
\usepackage{vmargin}

% left top textwidth textheight headheight

% headsep footheight footskip

\setmargins{2.0cm}{2.5cm}{16 cm}{22cm}{0.5cm}{0cm}{1cm}{1cm}

\renewcommand{\baselinestretch}{1.3}

\setcounter{MaxMatrixCols}{10}

\begin{document}
[Total 12]8
The number of claims per month Y arising on a certain portfolio of insurance policies
is to be modelled using a modified geometric distribution with probability density
given by
p ( y \alpha ) =
\alpha y − 1
(1 + \alpha ) y
y = 1, 2,3, ...
where \alpha is an unknown positive parameter. The most recent four months have
resulted in claim numbers of 8, 6, 10 and 9.
\begin{enumerate}
\item (i) Derive the maximum likelihood estimate of \alpha .
\item (ii) Show that Y belongs to an exponential family of distributions and suggest its
natural parameter.
\end{enumerate}

%%%%%%%%%%%%%%%%%%%%%%%%%%%%%
\newpage


8
(i)
We have 4 years of observations such that y 1  y 2  y 3  y 4  33 . The
likelihood function is then:
4  y i  1
i  1 (1   ) y i
L  

 33  4
(1   ) 33

 29
(1   ) 33
The log-likelihood is then:
l  29 log   33log(1   )
Taking its derivative w.r.t.  and equation it to zero we have:
29
33

 0
 1  
29(1   )  33 
which implies that 29  4 
therefore  ˆ 
29
 7.25.
4
Differentiating the log likelihood again gives 
29

2

33
 1    2
which is
negative at  ˆ  7.25.
(ii)
We have:
p ( y ) 
 y  1
(1   ) y
 exp[ y log   y log(1   )  log  ]


  
 exp  y log 
  log  
 1   


 ( y   b (  ))

 exp 
 c ( y ,  ) 
 a (  )

where
  
  log 
 , the natural parameter
 1   
  1
Page 11Subject CT6 (Statistical Methods Core Technical) – September 2013 – Examiners’ Report
a (  )  
b (  )  log     log(1  e  )
c ( y ,  )  0
This question was mostly well answered. Only the best candidates showed that the estimate
was a maximum by evaluating the second derivative of the log-likelihood at the value of the
estimate. In part (ii) some candidates failed to score full marks as a result of not specifying
all the parameters.
