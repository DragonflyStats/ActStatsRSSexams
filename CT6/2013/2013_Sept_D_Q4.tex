\documentclass[a4paper,12pt]{article}

%%%%%%%%%%%%%%%%%%%%%%%%%%%%%%%%%%%%%%%%%%%%%%%%%%%%%%%%%%%%%%%%%%%%%%%%%%%%%%%%%%%%%%%%%%%%%%%%%%%%%%%%%%%%%%%%%%%%%%%%%%%%%%%%%%%%%%%%%%%%%%%%%%%%%%%%%%%%%%%%%%%%%%%%%%%%%%%%%%%%%%%%%%%%%%%%%%%%%%%%%%%%%%%%%%%%%%%%%%%%%%%%%%%%%%%%%%%%%%%%%%%%%%%%%%%%

\usepackage{eurosym}
\usepackage{vmargin}
\usepackage{amsmath}
\usepackage{graphics}
\usepackage{epsfig}
\usepackage{enumerate}
\usepackage{multicol}
\usepackage{subfigure}
\usepackage{fancyhdr}
\usepackage{listings}
\usepackage{framed}
\usepackage{graphicx}
\usepackage{amsmath}
\usepackage{chngpage}

%\usepackage{bigints}
\usepackage{vmargin}

% left top textwidth textheight headheight

% headsep footheight footskip

\setmargins{2.0cm}{2.5cm}{16 cm}{22cm}{0.5cm}{0cm}{1cm}{1cm}

\renewcommand{\baselinestretch}{1.3}

\setcounter{MaxMatrixCols}{10}

\begin{document}
3
[6]
Andy is a famous weight lifter who will be competing at the Olympic Games. He has
taken out special insurance which pays out if he is injured. If the injury is so serious
that his career is ended the policy pays $1m and is terminated. If he is injured but
recovers the insurance payment is $0.1m and the policy continues.
The insurance company’s underwriters believe that the probability of an injury in any
year is 0.2, and that the probability of more than one injury in a year can be ignored.
If Andy is injured, there is a 75% chance that he will recover.
Annual premiums are paid in advance, and the insurance company pays claims at the
end of the year. Assume that this is the only policy that the insurance company
writes, and that it has an initial surplus of $0.1m.
(i) Define what is meant by ψ ($0.1m,1) and ψ ($0.1m).
(ii) Calculate the annual premium charged assuming the insurance company uses a
premium loading of 30%.

(iii) Determine ψ ($0.1m, 2).
CT6 S2013–2


[Total 8]4
The table below shows the probability distribution of a discrete random variable X.
Value
Probability
(i)
1
0.3
2
0.3
3
0.4
Construct an algorithm to generate random samples from X.

The random variable Z takes values from X with probability 0.2 and values from an
exponential distribution Y with probability 0.8. The upper quartile point of the
distribution of Y is 2.5.
5
(ii) Calculate the expected value of Y.
[3]
(iii) Extend the algorithm in part (i) to generate random samples from Z.


%%%%%%%%%%%%%%%%%%%%%%%%%%%%%%%%%%%%%%%%%%%%%%%%%%%%%%%%%%%%%%%%%%%%%%%%%%
3
(i)
If U  t   U  ct  S  t  where U  U  0   $0.1 m then
Ψ  $0.1 m ,1   Pr  U  t   0 for some t   0,1  given U  0   $0.1 m 
and
Ψ  $0.1 m   Pr( U  t   0 for some t  0 given U  0   $0.1 m )
(ii)
The premium charged will be:
1.3  0.2  (0.25  $1m  0.75  $0.1m)  $0.0845m.
Page 4Subject CT6 (Statistical Methods Core Technical) – September 2013 – Examiners’ Report
(iii)
The possibilities are tabulated below, where N means not injured, R means
injured but recovered and X means injured but career ending:
Year 1 Year 2 Probability Ruin?
N N 0.8  0.8 =0.64 No
N R 0.8  0.15 = 0.12 No
N X 0.8  0.05 = 0.04 Yes
R N 0.15  0.8 = 0.12 No
R R 0.15  0.15 = 0.0225 No
R X 0.15  0.05 = 0.0075 Yes
X N/A 0.05 Yes
Summing the cases where ruin occurs we have:
 ($0.1m, 2)  0.04  0.0075  0.05  0.0975
Many candidates lost marks in part (i) by not giving a sufficiently precise definition to score
full marks. For part (iii) candidates who worked through the possibilities methodically
generally scored well. A number of candidates unnecessarily used approximate methods in
part (iii).
4
(i)
The algorithm is as follows:
Step 1 Generate u from the uniform distribution on [0,1].
Step 2 If 0 < u < 0.3 set X = 1.
If 0.3 <= u < 0.6 set X = 2.
Otherwise set X = 3.
(ii)
We need to solve P ( Y  2.5)  0.75
but
P ( Y  2.5)  1  e  2.5 
so 1  e  2.5   0.75
so e  2.5   0.25
so 
log(0.25)
 0.554517744 and the mean of Y is 1.803368801.
 2.5
Page 5Subject CT6 (Statistical Methods Core Technical) – September 2013 – Examiners’ Report
(iii)
The extended algorithm is:
Step 1 Generate v from the uniform distribution on [0,1].
Step 2 If v< 0.2 then generate a sample from X as in (i) and finish,
otherwise go to step 3.
Step 3 Generate u from the uniform distribution on [0,1].
Step 4 Set 1  e  x  u
i.e. x 
log(1  u )
 0.55451744
This question was answered well.


