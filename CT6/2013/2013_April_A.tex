CT6 A2013
© Institute and Faculty of Actuaries1 Use the U(0,1) random numbers 0.238 and 0.655 to generate two observations from
the Weibull distribution with parameters c = 0.002 and γ = 1.1 .

2 Claims on a certain type of insurance policy are believed to follow an exponential
distribution. The upper quartile claim size is 240.
Calculate the mean claim size.
3

An actuary has a tendency to be late for work. If he gets up late then he arrives at
work X minutes late where X is exponentially distributed with mean 15. If he gets up
on time then he arrives at work Y minutes late where Y is uniformly distributed on
[0,25]. The office manager believes that the actuary gets up late one third of the time.
Calculate the posterior probability that the actuary did in fact get up late given that he
arrives more than 20 minutes late at work.

4
(i)
Explain what is meant by a two player zero-sum game.

Sally and Fiona agree to play a game. The rules of the game are as follows:
•
•
•
•
Each player chooses either the number 10 or the number 40.
Neither player knows the other player’s choice before selecting her number.
If both players choose the same number, Fiona pays Sally the sum of the numbers.
If the players choose differently, Sally pays Fiona the sum of the numbers.
Sally decides to adopt a randomised strategy where she chooses 10 with probability p
and 40 with probability 1 − p.
(ii)
(a) Determine the value of p for which Sally’s expected payoff is the same
regardless of what Fiona chooses.
(b) Explain why this strategy is optimal for Sally.
(c) Calculate Sally’s expected payout each time the game is played,
assuming that she follows this strategy.

[Total 6]



%%%%%%%%%%%%%%%%%%%%%%%%%%%%%%%%%%%%%%%%%%%%%%%%%%%%%%%%%%%%%%%%%%%%%%%%%%%%%%%%%%%%%%%%%%%
1
Using the inverse transform method we need to set:
γ
u = 1 − e − cx .
i.e.
− cx γ = log(1 − u ).
i.e.
1
⎛ log(1 − u ) ⎞ γ
x =⎜
⎟ .
− c
⎝
⎠
Using the equation above with the parameters c = 0.002 and γ = 1.1 we get:
u = 0.238 gives x = 86.96
u = 0.655 gives x = 300.73
This routine question was well answered, although a few candidates struggled with the
algebra.
2
We need to solve:
1 − e − 240 \lambda = 0.75
e − 240 \lambda = 0.25
so
\lambda=
log(0.25)
= 0.005776
− 240
and so the mean is
1
= 173.12.
0.005776
This straightforward question was well answered.
Page 3Subject %%%%%%%%%%%%%%%%%%%%%%%%%%%%%%%%%%%%%5 – April 2013 – Examiners’ Report
3
Let L be the state getting up late and let M be the state of getting up on time.
Let Z be the number of minutes late.
According to Bayes’ theorem:
P ( L Z > 20) =
P ( Z > 20 L ) P ( L )
P ( Z > 20)
but
P ( Z > 20 L ) = e
−
20
15
= 0.263597138
and
P ( Z > 20 ) = P ( Z > 20 L ) P ( L ) + P ( Z > 20 M ) P ( M )
= 0.263597138 × 1 + 0.2 × 2 = 0.221199046
3
3
and so
P ( L Z > 20) =
0.263597138 × 1
3 = 0.3972 .
0.221199046
This question was well answered by most candidates however weaker candidates were unable
to apply Bayes' Theorem.
4
(i) A game with 2 players where whatever one player loses in the game the other
player wins, and vice versa.
(ii) (a)
Value to Sally
Fiona
10
40
Sally
10
40
20
−50
−50 80
Sally chooses 10 with probability p.
Page 4Subject %%%%%%%%%%%%%%%%%%%%%%%%%%%%%%%%%%%%%5 – April 2013 – Examiners’ Report
Then for the expected payoffs to be equal regardless of Fiona’s choice
we must have:
20 p − 50 ( 1 − p ) = − 50 p + 80 ( 1 − p )
so
so
200 p = 130
p = 0.65
(b) This strategy is optimal for Sally because it produces the same
expected payoff regardless of what Fiona does. Under any other
randomized strategy Fiona can adopt a strategy that minimizes Sally’s
expected payoff.
(c) Value = 20 * 0.65 – 50 * 0.35 = −4.5.
This question was generally well answered, although a few candidates were thrown by a less
familiar application of decision theory.
