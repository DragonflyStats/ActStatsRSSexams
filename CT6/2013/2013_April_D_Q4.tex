
\documentclass[a4paper,12pt]{article}

%%%%%%%%%%%%%%%%%%%%%%%%%%%%%%%%%%%%%%%%%%%%%%%%%%%%%%%%%%%%%%%%%%%%%%%%%%%%%%%%%%%%%%%%%%%%%%%%%%%%%%%%%%%%%%%%%%%%%%%%%%%%%%%%%%%%%%%%%%%%%%%%%%%%%%%%%%%%%%%%%%%%%%%%%%%%%%%%%%%%%%%%%%%%%%%%%%%%%%%%%%%%%%%%%%%%%%%%%%%%%%%%%%%%%%%%%%%%%%%%%%%%%%%%%%%%

\usepackage{eurosym}
\usepackage{vmargin}
\usepackage{amsmath}
\usepackage{graphics}
\usepackage{epsfig}
\usepackage{enumerate}
\usepackage{multicol}
\usepackage{subfigure}
\usepackage{fancyhdr}
\usepackage{listings}
\usepackage{framed}
\usepackage{graphicx}
\usepackage{amsmath}
\usepackage{chngpage}

%\usepackage{bigints}
\usepackage{vmargin}

% left top textwidth textheight headheight
% headsep footheight footskip
\setmargins{2.0cm}{2.5cm}{16 cm}{22cm}{0.5cm}{0cm}{1cm}{1cm}

\renewcommand{\baselinestretch}{1.3}

\setcounter{MaxMatrixCols}{10}

\begin{document}
4
(i)
Explain what is meant by a two player zero-sum game.

Sally and Fiona agree to play a game. The rules of the game are as follows:
•
•
•
•
Each player chooses either the number 10 or the number 40.
Neither player knows the other player’s choice before selecting her number.
If both players choose the same number, Fiona pays Sally the sum of the numbers.
If the players choose differently, Sally pays Fiona the sum of the numbers.
Sally decides to adopt a randomised strategy where she chooses 10 with probability p
and 40 with probability 1 − p.
(ii)
(a) Determine the value of p for which Sally’s expected payoff is the same
regardless of what Fiona chooses.
(b) Explain why this strategy is optimal for Sally.
(c) Calculate Sally’s expected payout each time the game is played,
assuming that she follows this strategy.

[Total 6]



%%%%%%%%%%%%%%%%%%%%%%%%%%%%%%%%%%%%%%%%%%%%%%%%%%%%%%%%%%%%%%%%%%%%%%%%%%%%%%%%%%%%%%%%%%%
\newpage
4
(i) A game with 2 players where whatever one player loses in the game the other
player wins, and vice versa.
(ii) (a)
Value to Sally
Fiona
10
40
Sally
10
40
20
−50
−50 80
Sally chooses 10 with probability p.
Page 4Subject %%%%%%%%%%%%%%%%%%%%%%%%%%%%%%%%%%%%%5 – April 2013 – Examiners’ Report
Then for the expected payoffs to be equal regardless of Fiona’s choice
we must have:
20 p − 50 ( 1 − p ) = − 50 p + 80 ( 1 − p )
so
so
200 p = 130
p = 0.65
(b) This strategy is optimal for Sally because it produces the same
expected payoff regardless of what Fiona does. Under any other
randomized strategy Fiona can adopt a strategy that minimizes Sally’s
expected payoff.
(c) Value = 20 * 0.65 – 50 * 0.35 = −4.5.
This question was generally well answered, although a few candidates were thrown by a less
familiar application of decision theory.
\end{document}
