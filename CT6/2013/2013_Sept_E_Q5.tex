\documentclass[a4paper,12pt]{article}

%%%%%%%%%%%%%%%%%%%%%%%%%%%%%%%%%%%%%%%%%%%%%%%%%%%%%%%%%%%%%%%%%%%%%%%%%%%%%%%%%%%%%%%%%%%%%%%%%%%%%%%%%%%%%%%%%%%%%%%%%%%%%%%%%%%%%%%%%%%%%%%%%%%%%%%%%%%%%%%%%%%%%%%%%%%%%%%%%%%%%%%%%%%%%%%%%%%%%%%%%%%%%%%%%%%%%%%%%%%%%%%%%%%%%%%%%%%%%%%%%%%%%%%%%%%%

\usepackage{eurosym}
\usepackage{vmargin}
\usepackage{amsmath}
\usepackage{graphics}
\usepackage{epsfig}
\usepackage{enumerate}
\usepackage{multicol}
\usepackage{subfigure}
\usepackage{fancyhdr}
\usepackage{listings}
\usepackage{framed}
\usepackage{graphicx}
\usepackage{amsmath}
\usepackage{chngpage}

%\usepackage{bigints}
\usepackage{vmargin}

% left top textwidth textheight headheight

% headsep footheight footskip

\setmargins{2.0cm}{2.5cm}{16 cm}{22cm}{0.5cm}{0cm}{1cm}{1cm}

\renewcommand{\baselinestretch}{1.3}

\setcounter{MaxMatrixCols}{10}

\begin{document}
An insurance company has a portfolio of life insurance policies for 2,000 workers at a
factory. The policies pay out £5,000 if a worker dies in an industrial accident and
£2,000 if a worker dies for any other reason. For each worker, the probability of
death in any year is 0.02 and 25% of deaths are the result of industrial accidents. The
insurance company charges an annual premium of £74.25 per worker.
(i)
Calculate the premium loading used by the insurance company.

The insurance company is considering adopting one of the following three approaches
to reinsurance:
A None.
B 30% proportional reinsurance at a cost of £27 per worker.
C Individual excess of loss reinsurance with retention £3,000 and a
premium of £15 per worker.
(ii) Find the optimal decision under the Bayes criterion. 
(iii) Find the optimal decision under the minimax criterion. 
(iv) Comment on your answer to part (iii).
CT6 S2013–3


\newpage


%%%%%%%%%%%%%%%%%%%%%

5
(i)
The premium loading  is given by:
74.25  (1   )  (0.75  2000  0.25  5000)  0.02  55(1   )
and so

(ii)
74.25
 1  35%.
55
Under A expected profit is:
2000  74.25  2000  0.02  (0.75  2000  0.25  5000)
 38,500.
Under B expected profit is:
2, 000  74.25  2, 000  27  0.7  2, 000  0.02  (0.75  2, 000  0.25  5, 000)
 17,500.
Under C expected profit is:
2000  74.25  2000  15  2000  0.02  (0.75  2000  0.25  3000)
 28, 500
so the optimal course under the Bayes criterion is no reinsurance.
Page 6Subject CT6 (Statistical Methods Core Technical) – September 2013 – Examiners’ Report
(iii)
Under the minimax we need to consider the worst case scenario – which is that
all 2,000 workers die in industrial accidents.
Under this outcome, the losses are:
Under A:
Under B:
Under C:
2000  74.25  2000  5000   9,851,500
2000  (74.25  27)  2000  5000  0.7   6,905,500
2000  (74.25  15)  2000  3000 = 5,881,500
so the optimal decision under the minimax criterion is C.
(iv)
The approach in (iii) puts all the weight on what is at first seems a pretty
unlikely scenario – so that our decision making is driven by something fairly
remote.
That said, the workers are all in the same factory, so it is not inconceivable
that a single catastrophe could result in a large number of claims all at the
same time – i.e. the lives are not independent.
This question was well answered. There are a number of alternative approaches available
(for example working on a per policy basis) which all give the same results, and all of which
were given full credit. Candidates made a range of comments in part (iv) and all sensible
answers were given credit.
\end{document}
