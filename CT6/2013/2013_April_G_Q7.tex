\documentclass[a4paper,12pt]{article}

%%%%%%%%%%%%%%%%%%%%%%%%%%%%%%%%%%%%%%%%%%%%%%%%%%%%%%%%%%%%%%%%%%%%%%%%%%%%%%%%%%%%%%%%%%%%%%%%%%%%%%%%%%%%%%%%%%%%%%%%%%%%%%%%%%%%%%%%%%%%%%%%%%%%%%%%%%%%%%%%%%%%%%%%%%%%%%%%%%%%%%%%%%%%%%%%%%%%%%%%%%%%%%%%%%%%%%%%%%%%%%%%%%%%%%%%%%%%%%%%%%%%%%%%%%%%

\usepackage{eurosym}
\usepackage{vmargin}
\usepackage{amsmath}
\usepackage{graphics}
\usepackage{epsfig}
\usepackage{enumerate}
\usepackage{multicol}
\usepackage{subfigure}
\usepackage{fancyhdr}
\usepackage{listings}
\usepackage{framed}
\usepackage{graphicx}
\usepackage{amsmath}
\usepackage{chngpage}

%\usepackage{bigints}
\usepackage{vmargin}

% left top textwidth textheight headheight

% headsep footheight footskip

\setmargins{2.0cm}{2.5cm}{16 cm}{22cm}{0.5cm}{0cm}{1cm}{1cm}

\renewcommand{\baselinestretch}{1.3}

\setcounter{MaxMatrixCols}{10}

\begin{document}
An insurance company believes that individual claim amounts from house insurance
policies follow a gamma distribution with distribution function given by:
f ( y ) =
\alpha \alpha
\mu  \alpha \Gamma ( \alpha )
y
\alpha− 1
e
\alpha
− y
\mu 
for y > 0
where \alpha and \mu  are positive parameters.
(i)
Show that the gamma distribution can be written in exponential family form,
giving the natural parameter and the canonical link function.

The insurance company has data for claim amounts from previous claims. It believes
that the claim amount is primarily influenced by two variables:
x i the type of geographical area in which the house is situated. This can take
one of 4 values.
y i the category of the age of the house where the three categories are 0–29
years, 30-59 years and 60 years +.
It wishes to model claim amounts using this data and the generalised linear model
from part (i) with canonical link function. The insurance company is investigating
models which take into account these variables and has the following table of values:
Model
A
B
C
D
(ii)
CT6 A2013–4
Choice of
predictor
1
Age
Age +location
Age * location
Scaled Deviance
900
789
544
541
Explain, by analysing the scaled deviances, which model the insurance
company should use.

%%%%%%%%%%%%%%%%%%%%%%%%%%%%%%%%%%%%%%%%
\newpage

7
(i)
From the definition of the gamma density given in the question
f ( y ) =
\alpha \alpha
\mu  \alpha \Gamma ( \alpha )
y
\alpha− 1
e
\alpha
− y
\mu 
⎡ ⎛ y
⎤
⎞
= exp ⎢ ⎜ − − log \mu  ⎟ \alpha + ( \alpha − 1 ) log y + \alpha log \alpha − log \Gamma ( \alpha ) ) ⎥
⎠
⎣ ⎝ \mu 
⎦
⎡ ( y \theta  − b ( \theta  ) )
⎤
= exp ⎢
+ c ( y , \phi  ) ⎥
⎣ a ( \phi  )
⎦
where:
\theta =−
1
\mu 
\phi =\alpha
a ( \phi  ) =
1
\phi 
b ( \theta  ) = − log ( −\theta  )
c ( y , \phi  ) = ( \phi  − 1 ) log y + \phi  log \phi  − log \Gamma ( \phi  ).
Hence the distribution has the right form for a member of an exponential
family.
1
The natural parameter is − 1 . The canonical link function is .
\mu 
\mu 
(ii)
Using the information given, we can calculate the deviance differences and
compare that with the differences of the degrees of freedom for each of the
nested models. If the decrease in the deviances is greater than twice the
difference in degrees of freedom this suggests an improvement.
Model Scaled
Deviance Degrees of
freedom
1
Age
Age +location
Age * location 900
789
544
541 12
10
7
1
Difference
in scaled
deviance
111
245
3

From the table we can see that the interaction model does not indicate any
improvement hence the recommended model would be Age +location.
Again well prepared candidates were able to score highly on this question, however weaker
candidates dropped marks as a result of not specifying a full parameterisation in part (i). In
part (ii) full credit was given to candidates who used the chi-squared test rather than the
approximation set out above..
\end{document}
