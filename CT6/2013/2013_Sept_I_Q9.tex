\documentclass[a4paper,12pt]{article}

%%%%%%%%%%%%%%%%%%%%%%%%%%%%%%%%%%%%%%%%%%%%%%%%%%%%%%%%%%%%%%%%%%%%%%%%%%%%%%%%%%%%%%%%%%%%%%%%%%%%%%%%%%%%%%%%%%%%%%%%%%%%%%%%%%%%%%%%%%%%%%%%%%%%%%%%%%%%%%%%%%%%%%%%%%%%%%%%%%%%%%%%%%%%%%%%%%%%%%%%%%%%%%%%%%%%%%%%%%%%%%%%%%%%%%%%%%%%%%%%%%%%%%%%%%%%

\usepackage{eurosym}
\usepackage{vmargin}
\usepackage{amsmath}
\usepackage{graphics}
\usepackage{epsfig}
\usepackage{enumerate}
\usepackage{multicol}
\usepackage{subfigure}
\usepackage{fancyhdr}
\usepackage{listings}
\usepackage{framed}
\usepackage{graphicx}
\usepackage{amsmath}
\usepackage{chngpage}

%\usepackage{bigints}
\usepackage{vmargin}

% left top textwidth textheight headheight

% headsep footheight footskip

\setmargins{2.0cm}{2.5cm}{16 cm}{22cm}{0.5cm}{0cm}{1cm}{1cm}

\renewcommand{\baselinestretch}{1.3}

\setcounter{MaxMatrixCols}{10}

\begin{document}
9

[Total 10]
(i) State the three main stages in the Box-Jenkins approach to fitting an ARIMA
time series model.
[3]
(ii) Explain, with reasons, which ARIMA time series would fit the observed data
in the charts below.

ACF

PACF
Now consider the time series model given by
X t = \alpha 1 X t − 1 + \alpha 2 X t − 2 +\beta 1 e t − 1 + e t
where e t is a white noise process with variance \sigma 2 .
(iii) Derive the Yule-Walker equations for this model.
(iv) Explain whether the partial auto-correlation function for this model can ever
give a zero value.

%%%%%%%%%%%%%%%%%%%%%%%%%%%%%%%%%%%%%%%%%%%%%%%%%%%%%%%%%%%%%%%%%%%%%%%%%%%%%%%%%
\newpage
9
(i)
The three main stages are:
(a)
(b)
(c)
tentative model identification
model fitting
diagnostics
(ii) Since the auto-correlation is non-zero for the first lag only and the partial auto-
correlation function decays exponentially it is likely that the observed data
comes from an MA(1) (or equivalently a ARMA(0,1) or ARIMA(0,0,1)
model).
(iii) First note that for this model:
Cov( X t , e t )   2
and
Cov( X t , e t  1 )   1 Cov( X t  1 , e t  1 )   1  2  (  1   1 )  2 .
Taking the covariance of the defining equation with X t we get:
 0   1  1   2  2   1 (  1   1 )  2   2 .
Taking the covariance with X t  1 we get:
 1   1  0   2  1   1  2 .
Taking the covariance with X t  2 we get:
 2   1  1   2  0

%%--- Page 12Subject CT6 (Statistical Methods Core Technical) – September 2013 – Examiners’ Report
and in general
 n   1  n  1   2  n  2 for n  2.
(iv)
The presence of the term  1 e t  1 means that the PACF will decay exponentially to zero, but it will never get there, so that the PACF will always be non-zero.
Many candidates struggled with this question, with only the best accurately calculating the covariance of X t with e t  1 . The chart on the printed examination paper was not clear, and the
examiners took a generous approach to marking part (ii) where candidates had struggled interpreting the chart.
\end{document}
