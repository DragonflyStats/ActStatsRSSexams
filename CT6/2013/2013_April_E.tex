[Total 13]10
An insurance company has a portfolio of building insurance policies. The company
classifies buildings into three types and believes that the number of claims on
buildings of each type follows a Poisson distribution with parameters as shown:
Type Parameter
1
2
3 \lambda
2\lambda
5 \lambda
where \lambda is an unknown positive constant.
Actual claim numbers over the last five years have been as follows. Here X ij
represents the number of claims from the ith type in the jth year:
Number of claims X ij
Year (j)
5
Type(i)
1
2
3
5 4 3 2 1 ∑ ( X ij − X i ) 2
23
56
87 17
39
115 9
44
141 21
29
92 12
35
84 139.2
417.2
2322.8
j = 1
(i) Derive the maximum likelihood estimate of \lambda . 
(ii) Estimate the average number of claims per year for each type of building
using EBCT Model 1. 
(iii) Comment on the results of parts (i) and (ii). 
(iv) Explain the main weakness of the model in part (ii).
CT6 A2013–7

[Total 15]

%%%%%%%%%%%%%%%%%%%%%%%%%%%%%%%%%%%%%%%%%%%%%%%%%%%%%%%%%%%%%%%%%%%%%%%%%%%%%%%%%%%%%%%%%%%%%%%%%%%%%%%%%%%%%%%%%%%%%


10
(i)
The likelihood function is given by:
5
L ∝ ∏ e −\lambda \lambda
x 1, i
i = 1
5
∏ e − 2 \lambda (2 \lambda )
x 2, i
i = 1
5
∏ e − 5 \lambda (5 \lambda )
x 3, i
.
i = 1
Where x j , i is the number of claims on the jth type in the ith year.
The log likelihood is given by:
l = log L = C − 5 \lambda − 10 \lambda − 25 \lambda + (log \lambda ) ∑ x j , i .
i , j
= C − 40 \lambda + 804(log \lambda ) .
Differentiating gives:
dl
804
.
= − 40 +
d \lambda
\lambda
and setting this equal to zero gives:
804
\lambda ˆ =
= 20.1.
40
This is a maximum since:
d 2 l
d \lambda
(ii)
2
=−
804
\lambda 2
< 0.
The mean number of claims for the various types are:
X 1 = 16.4 and X 2 = 40.6 and X 3 = 103.8.
With overall mean X = 53.6.
Page 11Subject %%%%%%%%%%%%%%%%%%%%%%%%%%%%%%%%%%%%%5 – April 2013 – Examiners’ Report
So we have parameter estimates:
E ( m ( θ ) ) = X = 53.6.
1 3 ⎡ 1 5
E s ( θ ) = ∑ ⎢ ∑ X ij − X i
3 i = 1 ⎢ 4 j = 1
⎣
(
=
2
)
(
)
2 ⎤
⎥
⎥ ⎦
1
( 139.2 + 417.2 + 2322.8 ) = 239.9333333.
12
2 1
1 3
Var ( m ( θ ) ) = ∑ ( X i − X ) − E ( s 2 ( θ ) )
2 i = 1
5
= 0.5 ⎡ (16.4 − 53.6) 2 + (40.6 − 53.6) 2 + (103.8 − 53.6) 2 ⎤ − 0.2 × 239.93333333
⎣
⎦
=1988.4533333.
And so:
5
Z =
5 +
E ( s ( θ ) )
Var ( m ( θ ) )
2
=
5
= 0.976436003 .
239.9333333
5 +
1988.4533333
and the expected claims from the three types are:
(iii)
Type Credibility Premium
1
2
3 0.976436002 × 16.4 + 0.023563998 × 53.6 = 17.3
0.976436002 × 40.6 + 0.023563998 × 53.6 = 40.9
0.976436002 × 103.8 + 0.023563998 × 53.6 = 102.6
The corresponding estimates based on our computed \lambdâ are 20.1, 40.2 and
100.5.
The estimates are remarkably similar. The biggest difference is for type 1
buildings, where the maximum likelihood estimate gives a lower weight to the
data from that risk, but the credibility estimate gives greater weight.
(iv)
The main limitation is that the model in (ii) does not take account of the
volume of buildings covered, which will probably vary from year to year.
Again well prepared candidates found this question relatively straightforward. Weaker
candidates were unable to construct the likelihood function in part (i). A disappointing
number of candidates were unable to accurately render the standard formulae in part (ii).
