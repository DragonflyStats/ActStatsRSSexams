CT6 A2013–25
The following table shows incremental claims data from a portfolio of insurance
policies for the accident years 2010, 2011 and 2012. Claims from this type of policy
are fully run off after the end of development year two.
Incremental
Claims
Accident year
2010
2011
2012
Development year
0
1
2
2,328
1,749
2,117
1,484
1,188
384
Estimate the total claims outstanding using the basic chain ladder technique.
6

Claim numbers on a portfolio of insurance policies follow a Poisson process with
parameter \lambda . Individual claim amounts X follow a distribution with moments
m i = E ( X i ) for i = 1, 2, 3, ... . Let S denote the aggregate claims for the portfolio.
You may assume that the mean of S is \lambda m 1 and the variance of S is \lambda m 2 .
(i)
Derive the third central moment of S and show that the coefficient of
\lambda m 3
skewness of S is
.
3
2
( \lambda m 2 ) 
(ii) Show that S is positively skewed regardless of the distribution of X. 
(iii) Show that the distribution of S tends to symmetry as \lambda → ∞ .
CT6 A2013–3

[Total 8]
%%%%%%%%%%%%%%%%%
5
First accumulate claims:
Cumulative
Claims
Accident year
2010
2011
2012
0
2,328
1,749
2,117
Development year
1
2
3,812
2,937
4,196
DY1 = (3,812 + 2,937) / (2,328 + 1,749) = 1.655 384
DY2 = 4,196 / 3,812 = 1.100 735
Now complete lower half of table:
Cumulative
Claims
Accident year
2009
2011
2012
0
2,328
1,749
2,117
Development year
1
2
3,812
2,937
3,504.45
4,196
3,232.86
3,857.47
So estimated amount of outstanding claims is:
(3,232.86 – 2,937) + (3,857.47 – 2,117) = 2,036.3.
Most candidates scored full marks on this straightforward application of chain ladder theory.
Page 5Subject %%%%%%%%%%%%%%%%%%%%%%%%%%%%%%%%%%%%%5 – April 2013 – Examiners’ Report
6
(i)
We have:
M S ( t ) = M N (log M X ( t )) = e
\lambda ( M X ( t ) − 1)
Let us work with the cumulant generating function:
C S ( t ) = log M S ( t ) = \lambda M X ( t ) − \lambda .
The third central moment is given by C S ′′′ (0).
Now:
C S ′′′ ( t ) = \lambda M ′′′ X ( t )
and so
C S ′ ′′ ( 0 ) = \lambda M ′′′ X ( 0 ) = \lambda m 3 .
Hence the coefficient of skewness is given by:
E (( S − E ( S ) ) 3
( Var ( S ) )
(ii)
3
\lambda m 3
=
(
2
3
\lambda m 2 2
.
)
( )
Since X takes only positive values we have m 3 = E X 3 > 0.
Both \lambda and m 2 = E ( X 2 ) are also always positive.
This means the coefficient of skewness is always positive.
(iii)
Re-writing the equation for the coefficient of skewness we have:
\lambda m 3
(
3
\lambda m 2 2
)
=
m 3
\lambda
0.5
m 2 1.5
→ 0 as \lambda → ∞ .
Hence the distribution of S tends to symmetry as \lambda → ∞ .
Well prepared candidates who knew their bookwork were able to answer this question well,
however weaker candidates struggled with part (i) and gave unconvincing answers to part
(ii) & (iii).
Page 6Subject %%%%%%%%%%%%%%%%%%%%%%%%%%%%%%%%%%%%%5 – April 2013 – Examiners’ Report
