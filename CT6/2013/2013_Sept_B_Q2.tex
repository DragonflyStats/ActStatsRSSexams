\documentclass[a4paper,12pt]{article}

%%%%%%%%%%%%%%%%%%%%%%%%%%%%%%%%%%%%%%%%%%%%%%%%%%%%%%%%%%%%%%%%%%%%%%%%%%%%%%%%%%%%%%%%%%%%%%%%%%%%%%%%%%%%%%%%%%%%%%%%%%%%%%%%%%%%%%%%%%%%%%%%%%%%%%%%%%%%%%%%%%%%%%%%%%%%%%%%%%%%%%%%%%%%%%%%%%%%%%%%%%%%%%%%%%%%%%%%%%%%%%%%%%%%%%%%%%%%%%%%%%%%%%%%%%%%

\usepackage{eurosym}
\usepackage{vmargin}
\usepackage{amsmath}
\usepackage{graphics}
\usepackage{epsfig}
\usepackage{enumerate}
\usepackage{multicol}
\usepackage{subfigure}
\usepackage{fancyhdr}
\usepackage{listings}
\usepackage{framed}
\usepackage{graphicx}
\usepackage{amsmath}
\usepackage{chngpage}

%\usepackage{bigints}
\usepackage{vmargin}

% left top textwidth textheight headheight

% headsep footheight footskip

\setmargins{2.0cm}{2.5cm}{16 cm}{22cm}{0.5cm}{0cm}{1cm}{1cm}

\renewcommand{\baselinestretch}{1.3}

\setcounter{MaxMatrixCols}{10}

\begin{document}

2
Claim amounts on a certain type of insurance policy follow an exponential
distribution with mean 100. The insurance company purchases a special type of
reinsurance policy so that for a given claim X the reinsurance company pays
0
0.5 X − 40
X − 120
if 0 < X < 80;
if 80 ≤ X < 160;
if X ≥ 160.
Calculate the expected amount paid by the reinsurance company on a randomly
chosen claim.


%%%%%%%%%%%%%%%%%%%%%%%%%%%%%%%%%%%%%%%%%%%%%%%%%%%%%%%%%%%%%%%%%%%%%%%%%%%
1
The posterior distribution of p is given by
f ( p k claims)  f ( k claims p )  f ( p )
 p k ( 1  p ) n  k p  1 ( 1  p )  1
 p  k  1 (1  p )  n  k  1
which is the pdf of a Beta distribution with parameters   k and   n  k .
Using the fact given in the questions, the mode of the posterior distribution (which is
the estimate of p under all or nothing loss is given by:
p̂ =
=
  k  1
  k  1

 k  n  k  2  n  2
  1
 2
k
n

 
 2  n  2 n  n  2
= (1  Z ) 
where Z
=
  1
k
 Z 
n
 2
n
.
 n  2
This is in the form of a credibility estimate since
under all or nothing loss and k
n
  1
is the prior estimate of p
 2
is the estimate of p derived from the data.
The first part of this question was answered well. Most candidates didn’t recognise the need
to base the prior estimate on the mode of the prior distribution and therefore didn’t manage
to express the posterior estimate as a credibility estimate.
Page 3Subject CT6 (Statistical Methods Core Technical) – September 2013 – Examiners’ Report
2
The mean amount paid by the reinsurance company is given by:

160
 1

 0.01 x
 0.01 x
   2 x  40    0.01 e dx   ( x  120)  0.01 e dx .
80
160
The first integral is (using integration by parts):
160
  1
  0.01 x 
+
   2 x  40  e


 
 80
160

80
1  0.01 x
e
dx
2
160
=  40 e  1.6   50 e  0.01 x 

 80
= 50 e  0.8  90 e  1.6  4.2957.
The second integral is:

  ( x  120) e  0.01 x 
+

 160

 e
 0.01 x
dx
160
= 40 e  1.6  1 00 e  1.6
= 28.26551.
So total mean claim = 4.29576 + 28.26551 = 32.56.
This question was generally answered well. Weaker candidates could not integrate by parts
accurately.
\end{document}
