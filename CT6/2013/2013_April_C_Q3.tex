
\documentclass[a4paper,12pt]{article}

%%%%%%%%%%%%%%%%%%%%%%%%%%%%%%%%%%%%%%%%%%%%%%%%%%%%%%%%%%%%%%%%%%%%%%%%%%%%%%%%%%%%%%%%%%%%%%%%%%%%%%%%%%%%%%%%%%%%%%%%%%%%%%%%%%%%%%%%%%%%%%%%%%%%%%%%%%%%%%%%%%%%%%%%%%%%%%%%%%%%%%%%%%%%%%%%%%%%%%%%%%%%%%%%%%%%%%%%%%%%%%%%%%%%%%%%%%%%%%%%%%%%%%%%%%%%

\usepackage{eurosym}
\usepackage{vmargin}
\usepackage{amsmath}
\usepackage{graphics}
\usepackage{epsfig}
\usepackage{enumerate}
\usepackage{multicol}
\usepackage{subfigure}
\usepackage{fancyhdr}
\usepackage{listings}
\usepackage{framed}
\usepackage{graphicx}
\usepackage{amsmath}
\usepackage{chngpage}

%\usepackage{bigints}
\usepackage{vmargin}

% left top textwidth textheight headheight
% headsep footheight footskip
\setmargins{2.0cm}{2.5cm}{16 cm}{22cm}{0.5cm}{0cm}{1cm}{1cm}

\renewcommand{\baselinestretch}{1.3}

\setcounter{MaxMatrixCols}{10}

\begin{document}
3

An actuary has a tendency to be late for work. If he gets up late then he arrives at
work X minutes late where X is exponentially distributed with mean 15. If he gets up
on time then he arrives at work Y minutes late where Y is uniformly distributed on
[0,25]. The office manager believes that the actuary gets up late one third of the time.
Calculate the posterior probability that the actuary did in fact get up late given that he
arrives more than 20 minutes late at work.



%%%%%%%%%%%%%%%%%%%%%%%%%%%%%%%%%%%%%%%%%%%%%%%%%%%%%%%%%%%%%%%%%%%%%%%%%%%%%%%%%%%%%%%%%%%

3
Let L be the state getting up late and let M be the state of getting up on time.
Let Z be the number of minutes late.
According to Bayes’ theorem:
P ( L Z > 20) =
P ( Z > 20 L ) P ( L )
P ( Z > 20)
but
P ( Z > 20 L ) = e
−
20
15
= 0.263597138
and
P ( Z > 20 ) = P ( Z > 20 L ) P ( L ) + P ( Z > 20 M ) P ( M )
= 0.263597138 × 1 + 0.2 × 2 = 0.221199046
3
3
and so
P ( L Z > 20) =
0.263597138 × 1
3 = 0.3972 .
0.221199046
This question was well answered by most candidates however weaker candidates were unable
to apply Bayes' Theorem.
