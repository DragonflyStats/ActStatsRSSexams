\documentclass[a4paper,12pt]{article}

%%%%%%%%%%%%%%%%%%%%%%%%%%%%%%%%%%%%%%%%%%%%%%%%%%%%%%%%%%%%%%%%%%%%%%%%%%%%%%%%%%%%%%%%%%%%%%%%%%%%%%%%%%%%%%%%%%%%%%%%%%%%%%%%%%%%%%%%%%%%%%%%%%%%%%%%%%%%%%%%%%%%%%%%%%%%%%%%%%%%%%%%%%%%%%%%%%%%%%%%%%%%%%%%%%%%%%%%%%%%%%%%%%%%%%%%%%%%%%%%%%%%%%%%%%%%

\usepackage{eurosym}
\usepackage{vmargin}
\usepackage{amsmath}
\usepackage{graphics}
\usepackage{epsfig}
\usepackage{enumerate}
\usepackage{multicol}
\usepackage{subfigure}
\usepackage{fancyhdr}
\usepackage{listings}
\usepackage{framed}
\usepackage{graphicx}
\usepackage{amsmath}
\usepackage{chngpage}

%\usepackage{bigints}
\usepackage{vmargin}

% left top textwidth textheight headheight

% headsep footheight footskip

\setmargins{2.0cm}{2.5cm}{16 cm}{22cm}{0.5cm}{0cm}{1cm}{1cm}

\renewcommand{\baselinestretch}{1.3}

\setcounter{MaxMatrixCols}{10}

\begin{document}

PLEASE TURN OVER11
An actuary is considering the time series model defined by
X t = \alpha X t − 1 + e t
where e t is a sequence of independent Normally distributed random variables with
mean 0 variance \sigma 2 . The series begins with the fixed value X 0 = 0.
(i)
Show that the conditional distribution of X t given X t − 1 is Normal and hence
show that the likelihood of making observations x 1 , x 2 , ... , x n from this model
is:
n
L \propto  ∏
i = 1
1
e
2 π\sigma
−
( x i −\alpha x i − 1 ) 2
2 \sigma 2
.

(ii) Show that the maximum likelihood estimate of \alpha can also be regarded as a
least squares estimate.

(iii) Find the maximum likelihood estimates of \alpha and \sigma 2 .
(iv) Derive the Yule-Walker equations for the model and hence derive estimates of
\alpha and \sigma 2 based on observed values of the autocovariance function.

(v) Comment on the difference between the estimates of \alpha in parts (iii) and (iv).

[Total 15]
END OF PAPER
CT6 A2013–8


%%%%%%%%%%%%%%%%%%%%%%%%
Page 12Subject %%%%%%%%%%%%%%%%%%%%%%%%%%%%%%%%%%%%%5 – April 2013 – Examiners’ Report
11
(i)
X t − \alpha X t − 1 = e t ~ N ( 0 , \sigma 2 ) .
So X t X t − 1 ~ N ( \alpha X t − 1 , \sigma 2 )
and so the likelihood is given by:
n
L \propto  ∏ P ( X i = x i x i − 1 ) × P ( x 0 )
i = 1
n
1
e
2 π\sigma
L \propto  ∏
i = 1
(ii)
−
( x i −\alpha x i − 1 ) 2
2 \sigma 2
× 1
We can see that maximising the likelihood with respect to \alpha is the same as
minimising the expression:
\sum  i = 1 ( x i −\alpha x i − 1 ) 2
n
L \propto  \sigma − n e
−
2 \sigma 2
n
\sum  ( x i − \alpha x i − 1 ) 2 .
i = 1
(iii)
The log-likelihood is given by:
l = logL = − n log \sigma −
n
1
2 \sigma
2
\sum  ( x i − \alpha x i − 1 ) 2 + Constant
i = 1
Differentiating with respect to \alpha gives:
\partial  l
1
= 2
\partial \alpha 2 \sigma
and setting
n
\sum  2 x i − 1 ( x i − \alpha x i − 1 )
i = 1
=
1
\sigma
2
n
\alpha
n
\sum  x i x i − 1 − \sigma 2 \sum  x i 2 − 1
i = 1
i = 1
\partial  l
= 0 we have:
\partial \alpha
\sum  x i x i − 1 .
\hat{\alpha} = i = n 1
\sum  i = 1 x i 2 − 1
n
Page 13Subject %%%%%%%%%%%%%%%%%%%%%%%%%%%%%%%%%%%%%5 – April 2013 – Examiners’ Report
Differentiating with respect to \sigma we have:
\partial  l
n 1 n
= − + 3 \sum  ( x i − \alpha x i − 1 ) 2 .
\partial \sigma
\sigma \sigma i = 1
Setting this expression equal to zero we have:
\sigma ˆ 2 =
(iv)
1 n
( x i − \hat{\alpha} x i − 1 ) 2 .
\sum 
n i = 1
The Yule Walker equations are:
γ 0 = cov (\alpha X t−1 + e t , X t ) = \alpha cov(X t−1 , X t ) + cov (e t , X t ) = \alpha γ 1 + \sigma 2
γ 1 = cov (\alpha X t−1 + e t , X t−1 ) = \alpha cov(X t−1 , X t − 1 ) + cov (e t , X t−1 ) = \alpha γ 0
Using these to estimate the parameters we get:
\sigma ˆ 2 = γ ˆ 0 − \hat{\alpha} γ ˆ 1 .
n
(v)
γ ˆ
\hat{\alpha} = 1 =
γ ˆ 0
\sum  ( x i − x ) ( x i − 1 − x )
i = 1
n
\sum  ( x i − x )
.
2
i = 1
The difference between them is that in the second approach we need to
centralise the data around the mean x .
This question was relatively well answered for a time series question. It was clear that some
candidates had learnt the bookwork, but struggled with this more unfamiliar application of
time series. In particular only the best candidates accurately completed the differentiation
needed in part (iii).

\end{document}
