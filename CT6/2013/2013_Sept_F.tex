10
The number of service requests received by an IT engineer on any given day follows a
Poisson distribution with mean \mu . Prior beliefs about \mu follow a gamma distribution
with parameters \alpha and \lambda . Over a period of n days the actual numbers of service
requests received are x 1 , x 2 , ... , x n .
(i) Derive the posterior distribution of \mu .
(ii) Show that the Bayes estimate of \mu under quadratic loss can be written as a
credibility estimate and state the credibility factor.

[3]
Now suppose that \alpha = 10 , \lambda = 2 and that the IT worker receives 42 requests in
6 days.
(iii)
Calculate the Bayes estimate of $\mu$ under quadratic loss.

Three quarters of requests can be resolved by telling users to restart their machine,
and the time taken to do so follows a Pareto distribution with density
f ( x ) =
3 × 20 3
(20 + x ) 4
for x>0.
One quarter of requests are much harder to resolve, and the time taken to resolve these
follows a Weibull distribution with density
f ( x ) = 0.4 × 0.5 x − 0.5 e − 0.4 x
(iv)
0.5
for x>0.
(a) Calculate the probability that a randomly chosen request takes more
than 30 minutes to resolve.
(b) Calculate the average time spent on each request.
(c) Calculate the expected total amount of time the IT worker spends
dealing with service requests each day, using the estimate of \mu from
part (iii).

The IT worker’s line manager is carefully considering his staffing requirements. He
decides to model the time taken on each request approximately using an exponential
distribution.
(v)
CT6 S2013–6
(a) Fit an exponential distribution to the time taken per request using the
method of moments.
(b) Calculate the probability that a randomly chosen request takes more
than 30 minutes to resolve using this approximation.
(c) Comment briefly on your answer to part (v)(b).
The IT engineer needs to devote more of his time to a separate project, so his firm
have hired an assistant to help him. The assistant is just as fast at dealing with the
straightforward requests, and the time taken to resolve these still follows the Pareto
distribution given above. He is significantly slower at dealing with the difficult
requests, and the time taken to resolve these now follows a Weibull distribution with
density:
f ( x ) = c × 0.5 x − 0.5 e − cx
0.5
for x>0
where c is a positive parameter. The line manager again fits an exponential
distribution as an approximation to the time taken to service each request using the
method of moments. His approximation results in an estimate that the probability that
a random service request takes longer than 30 minutes to resolve is 10%.
(vi)
Determine the value of c.
END OF PAPER
CT6 S2013–7

[Total 17]

%%%%%%%%%%%%%%%%%%%%%%%%%%%%%%%%%%%%%%%%%%%%%%%%%%%%%%%%%%%%%%%%%%%%%%%%%%%%%%%%5
10
(i)
Firstly:
f (  x 1 , x 2 ,  , x n )  f ( x 1 , x 2 ,  , x n ) f (  )
n
  e

i  1
 x i    1 

 e
x i ! Γ(  )
n
 x i  1  (  n ) 
e
  i  1
which is the pdf of another gamma distribution. So the posterior distribution
n
is gamma with parameters    x i and   n .
i  1
(ii)
Under quadratic loss the Bayes estimate is the mean of the posterior
distribution, so:
   x i
n
 ˆ 
i  1
 n
which can be written as
 x i n


 ˆ  
 i  1 
  n
 n
n
n
 (1  Z )

 Zx

Page 13Subject CT6 (Statistical Methods Core Technical) – September 2013 – Examiners’ Report
n
. This is in the form of a credibility estimate since the mean
 n
of the prior distribution is  and we have written the posterior mean as a

weighted average of the prior mean and the mean of the observed data.
where Z 
(iii)
In this case we have:
 ˆ 
(iv)
(a)
10  42
 6.5.
2  6
Let S be the time taken to resolve a single query. Then for a simple
query:
3
 20 
3
P ( S  30 simple)  
  0.4  0.064.
 20  30 
For a complicated query we have
0.5
P ( S  30 complicated)  e  0.4  30
 0.111817.
And finally
P ( S  30)  0.75  0.064  0.25  0.111817  0.07595 .
(b)
Mean time for simple calls is
20
 10 .
3  1
Mean time for complicated calls is
 1
1 

0.5  Γ(3)  0.4  2  2  0.4  2  12.5 .

Γ  1 
0.4

 0.5 
Overall mean is 0.75  10  0.25  12.5  10.625 .
(v)
(c) Overall total time is 6.5  10.625  69.0625 .
(a) The parameter of the exponential distribution is
1
 0.094117647 .
10.625
(b) The probability of taking more than 30 minutes using this
approximation is P ( S  30)  e
(c)
Page 14

30
10.625
 0.059396 .
This compares to the true value of 0.07595. The exponential
distribution underestimates this tail probability since it has less fat tails
than the Pareto and Weibull distributions.Subject CT6 (Statistical Methods Core Technical) – September 2013 – Examiners’ Report
(vi)
We first find the parameter  of the exponential distribution being used. This
is given by:
P ( S  30)  e  30   0.1
so

log 0.1
 0.076752836
 30
and the mean of the exponential distribution is 13.0288.
1
1   0.5

The mean of the given Weibull distribution is Γ  1 
 2 c  2 .
  c
0.5


The overall mean is then given by 0.75  10  0.25  2 c  2  7.5  0. 5 c  2 .
Equating this to 13.0288 gives:
7.5  0.5 c  2  13.0288
c  2  11.0576
c  0.300725.
Parts (i) to (iv) were well answered. Parts (v) and (vi) were attempted by only the more able
candidates.
END OF EXAMINERS’ REPORT
Page 15
