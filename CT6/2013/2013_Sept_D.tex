7
[10]
An insurance company offers dental insurance to the employees of a small firm. The
annual number of claims follows a Poisson process with rate 20. Individual loss
amounts follow an exponential distribution with mean 100. In order to increase the
take-up rate, the insurance company has guaranteed to pay a minimum amount of £50
per qualifying claim. Let S be the total claim amount on the portfolio for a given
year.
(i)
Show that the mean and variance of S are 2,213.06 and 413,918.40
respectively.
∞
[You may use without proof the result that if I n =
∫ y
n
[7]
\lambda e −\lambda y dy
M
n
I n − 1 ]
\lambda
then I n = M n e −\lambda M +
(ii) (a) Fit a log-normal distribution for S using the method of moments.
(b) Estimate the probability that S is greater than 4,000.
[3]
Sarah, the insurance company’s actuary, has instead approximated S by a Normal
distribution.
(iii)
CT6 S2013–4
Explain, without performing any further calculations, whether the probability
that she calculates that S exceeds 4,000 will be greater or smaller than the
calculation in part (ii).
%%%%%%%%%%%%%%%%%%%%%%%%%%%%%%%%%%%%%%%%%%%%%%%%%%%%%%%%%%%%%%%%%%%%%%%%%%%%%%%%

[Total 12]8
The number of claims per month Y arising on a certain portfolio of insurance policies
is to be modelled using a modified geometric distribution with probability density
given by
p ( y \alpha ) =
\alpha y − 1
(1 + \alpha ) y
y = 1, 2,3, ...
where \alpha is an unknown positive parameter. The most recent four months have
resulted in claim numbers of 8, 6, 10 and 9.
(i) Derive the maximum likelihood estimate of \alpha .
(ii) Show that Y belongs to an exponential family of distributions and suggest its
natural parameter.


%%%%%%%%%%%%%%%%%%%%%%%%%%%%%
\newpage

7
(i)
Let X i be the amount paid on the i th claim:
Then
50 
0 50
E ( X i )  50  f ( y ) dy   yf ( y ) dy
50
 50  0.01 e  0.01 y dy  I 1
0
Using the notation given in the question.
Page 8
151.0
178.5
So the total claims are:
2010
2011
2012
UltSubject CT6 (Statistical Methods Core Technical) – September 2013 – Examiners’ Report
Now

I 1  50 e  0.5  100  0.01 e  0.01 y dy
50

 50 e  0.5  100   e  0.01 y 

 50
 50 e  0.5  100 e  0.5  150 e  0.5
So
50
E ( X i )  50   e  0.01 y   150 e  0.5

 0
  50 e  0.5  50  150 e  0.5  50  100 e  0.5  110.653066
And
E ( X i 2 )
 50
2
50 
0 50
2
 f  y  dy   y f  y  dy
50
 2,500  0.01 e  0.01 y dy  I 2
0
50
 2,500   e  0.01 y   50 2 e  0.5  200 I 1

 0
  2,500 e  0.5  2,500  2,500 e  0.5  200  150 e  0.5
 2,500  200  150 e  0.5  20, 695.91979
So finally we have:
E ( S )  20  110.653066  2213.06
Var( S )  20  20695.91979  413918.40.
Page 9Subject CT6 (Statistical Methods Core Technical) – September 2013 – Examiners’ Report
(ii)
We need to solve:
e 
2
/2
 2213.06 (1)
and
e 2 
2
 e  1   413918.40.
 2
(2)
Dividing (2) by the square of (1) we have:
2
e   1 
413918.40
2213.06 2
 0.084514
 2  log(1.084514)  0.081132
and substituting into (1) we have:
  log(2213.06) 
0.081132
 7.6615655.
2
Finally:
P ( S  4000)  P ( N (7.6615655, 0.081132)  log(4000))

8.29404964  7.6615655 
 P  N (0,1) 
  P ( N (0,1)  2.2205)
0.081132


 0.95  (1  0.98679)  0.05  (1  0.98713)
= 0.01319
(iii)
The probability will be lower.
This is because the log normal distribution has a “fat tail” and hence gives
more weight to extreme outcomes.
Only the best candidates were able to derive the value of the variance in part (i) despite the
formula for integration by parts being given in the question paper. The remaining parts were
well answered.
Page 10Subject CT6 (Statistical Methods Core Technical) – September 2013 – Examiners’ Report
8
(i)
We have 4 years of observations such that y 1  y 2  y 3  y 4  33 . The
likelihood function is then:
4  y i  1
i  1 (1   ) y i
L  

 33  4
(1   ) 33

 29
(1   ) 33
The log-likelihood is then:
l  29 log   33log(1   )
Taking its derivative w.r.t.  and equation it to zero we have:
29
33

 0
 1  
29(1   )  33 
which implies that 29  4 
therefore  ˆ 
29
 7.25.
4
Differentiating the log likelihood again gives 
29

2

33
 1    2
which is
negative at  ˆ  7.25.
(ii)
We have:
p ( y ) 
 y  1
(1   ) y
 exp[ y log   y log(1   )  log  ]


  
 exp  y log 
  log  
 1   


 ( y   b (  ))

 exp 
 c ( y ,  ) 
 a (  )

where
  
  log 
 , the natural parameter
 1   
  1
Page 11Subject CT6 (Statistical Methods Core Technical) – September 2013 – Examiners’ Report
a (  )  
b (  )  log     log(1  e  )
c ( y ,  )  0
This question was mostly well answered. Only the best candidates showed that the estimate
was a maximum by evaluating the second derivative of the log-likelihood at the value of the
estimate. In part (ii) some candidates failed to score full marks as a result of not specifying
all the parameters.
