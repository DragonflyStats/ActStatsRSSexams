
\documentclass[a4paper,12pt]{article}

%%%%%%%%%%%%%%%%%%%%%%%%%%%%%%%%%%%%%%%%%%%%%%%%%%%%%%%%%%%%%%%%%%%%%%%%%%%%%%%%%%%%%%%%%%%%%%%%%%%%%%%%%%%%%%%%%%%%%%%%%%%%%%%%%%%%%%%%%%%%%%%%%%%%%%%%%%%%%%%%%%%%%%%%%%%%%%%%%%%%%%%%%%%%%%%%%%%%%%%%%%%%%%%%%%%%%%%%%%%%%%%%%%%%%%%%%%%%%%%%%%%%%%%%%%%%

\usepackage{eurosym}
\usepackage{vmargin}
\usepackage{amsmath}
\usepackage{graphics}
\usepackage{epsfig}
\usepackage{enumerate}
\usepackage{multicol}
\usepackage{subfigure}
\usepackage{fancyhdr}
\usepackage{listings}
\usepackage{framed}
\usepackage{graphicx}
\usepackage{amsmath}
\usepackage{chngpage}

%\usepackage{bigints}
\usepackage{vmargin}

% left top textwidth textheight headheight

% headsep footheight footskip

\setmargins{2.0cm}{2.5cm}{16 cm}{22cm}{0.5cm}{0cm}{1cm}{1cm}

\renewcommand{\baselinestretch}{1.3}

\setcounter{MaxMatrixCols}{10}

\begin{document}


[Total 11]8
An insurance company has a portfolio of 1,000 car insurance policies. Claims arise
on individual policies according to a Poisson process with annual rate \mu  . The
insurance company believes that \mu  follows a gamma distribution with parameters
\alpha = 2 and \lambda = 8.
\begin{enumerate}
\item (i)
(a) Show that the average annual number of claims per policy is 0.25.
(b) Show that the variance of the number of annual claims per policy is
0.28125.

Individual claim amounts follow a gamma distribution with density
− x
x
f ( x ) =
e 1000 for x > 0.
1, 000, 000
\item (ii)
Calculate the mean and variance of the annual aggregate claims for the whole
portfolio.

The insurance company has agreed an aggregate excess of loss reinsurance contract
with a retention of \$0.55m (this means that the reinsurance company will pay the
excess above \$0.55m if the aggregate claims on the portfolio in a given year exceed
\$0.55m).
\item (iii)
Calculate, using a Normal approximation, the probability of aggregate claims
exceeding the retention in any year.

For each of the last three years, the total claim amount has in fact exceeded the
retention.
\item (iv)

Comment on this outcome in light of the calculation in part (iii).
\end{enumerate}

\newpage


8
\begin{itemize}
\item (i)
(ii)
(a)
\begin{eqnarray*}
E(N) &=& E[E(N|\mu )]\\
&=& E[\mu ] \\
&=& 2/8 \\
&=& 0.25\\
\end{eqnarray*}
(b)
\begin{eqnarray*}
var(N) &=& E[var(N|\mu )] + var[E(N|\mu )]\\
&=& E[\mu ] + var[\mu ]\\
&=& 2/8 + 2/8 2 \\ 
&=& 0.28125\\
\end{eqnarray*}

\item Let Y be aggregate claims from one policy.
Individual claim is gamma with \alpha = 2 and \lambda = 0.001 .
\[E ( Y ) = E ( X ) E ( N ) = 2000 × 0.25 = 500.\]
Var ( Y ) = E ( N ) Var ( X ) + Var ( N ) E ( X ) 2
= 0.25 × 2000000 + 9
32
× 2000 2 = 1, 625, 000.
So the mean and variance of total claims are 500,000 and 1,625,000,000
respectively.
\item (iii)
Our approximate distribution for S is S ~ N(500,000 , 1625000000).
550000 − 500000 ⎞
⎛
P ( S > 550000 ) = P ⎜ Z >
⎟ = P ( Z > 1.24035 ) = 0.1074 .
1625000000 ⎠
⎝
\item (iv)
The prob three years in a row is $0.1074^3 = 0.00124$ .
The probability of this happening is very low. It is more likely that the
insurance company’s belief about the distribution of claims amounts is
incorrect.
The normal approximation tails off quickly and so underestimates the
probability of extreme events
\item Part (i) was straightforward, however some candidates failed to show sufficient working to
gain full marks. A surprising number of candidates were unfamiliar with the standard
bookwork underlying part (ii). Credit was given for any sensible comments in part (iv).
\end{itemize}


\end{document}
