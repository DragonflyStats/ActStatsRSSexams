\documentclass[a4paper,12pt]{article}

%%%%%%%%%%%%%%%%%%%%%%%%%%%%%%%%%%%%%%%%%%%%%%%%%%%%%%%%%%%%%%%%%%%%%%%%%%%%%%%%%%%%%%%%%%%%%%%%%%%%%%%%%%%%%%%%%%%%%%%%%%%%%%%%%%%%%%%%%%%%%%%%%%%%%%%%%%%%%%%%%%%%%%%%%%%%%%%%%%%%%%%%%%%%%%%%%%%%%%%%%%%%%%%%%%%%%%%%%%%%%%%%%%%%%%%%%%%%%%%%%%%%%%%%%%%%

\usepackage{eurosym}
\usepackage{vmargin}
\usepackage{amsmath}
\usepackage{graphics}
\usepackage{epsfig}
\usepackage{enumerate}
\usepackage{multicol}
\usepackage{subfigure}
\usepackage{fancyhdr}
\usepackage{listings}
\usepackage{framed}
\usepackage{graphicx}
\usepackage{amsmath}
\usepackage{chngpage}

%\usepackage{bigints}
\usepackage{vmargin}

% left top textwidth textheight headheight

% headsep footheight footskip

\setmargins{2.0cm}{2.5cm}{16 cm}{22cm}{0.5cm}{0cm}{1cm}{1cm}

\renewcommand{\baselinestretch}{1.3}

\setcounter{MaxMatrixCols}{10}

\begin{document}

© Institute and Faculty of Actuaries1
An insurance company has a portfolio of n policies. The probability of a claim in a
given year on each policy is p independently from policy to policy, and the possibility
of more than one claim can be ignored. Prior beliefs about p are specified by a Beta
distribution with parameters \alpha and \beta. In one year the insurance company has a total
of k claims on the portfolio.
Calculate the posterior estimate of p under all or nothing loss and show that it can be
written in the form of a credibility estimate.

[You may use without proof the fact that the mode of a Beta distribution with
\alpha − 1
.]
parameters \alpha and \beta is
\alpha +\beta− 2


%%%%%%%%%%%%%%%%%%%%%%%%%%%%%%%%%%%%%%%%%%%%%%%%%%%%%%%%%%%%%%

\newpage
1
The posterior distribution of p is given by
f ( p k claims)  f ( k claims p )  f ( p )
 p k ( 1  p ) n  k p  1 ( 1  p )  1
 p  k  1 (1  p )  n  k  1
which is the pdf of a Beta distribution with parameters   k and   n  k .
Using the fact given in the questions, the mode of the posterior distribution (which is
the estimate of p under all or nothing loss is given by:
p̂ =
=
  k  1
  k  1

 k  n  k  2  n  2
  1
 2
k
n

 
 2  n  2 n  n  2
= (1  Z ) 
where Z
=
  1
k
 Z 
n
 2
n
.
 n  2
This is in the form of a credibility estimate since
under all or nothing loss and k
n
  1
is the prior estimate of p
 2
is the estimate of p derived from the data.
The first part of this question was answered well. Most candidates didn’t recognise the need
to base the prior estimate on the mode of the prior distribution and therefore didn’t manage
to express the posterior estimate as a credibility estimate.
\end{document}
