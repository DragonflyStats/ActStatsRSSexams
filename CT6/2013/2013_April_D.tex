
[Total 12]
PLEASE TURN OVER9
Claims on a portfolio of insurance policies arise as a Poisson process with rate \lambda .
The mean claim amount is μ . The insurance company calculates premiums using a
loading of θ and has an initial surplus of U.
(i)
Explain how the parameters \lambda , μ , θ and U affect ψ ( U ) , the probability of
ultimate ruin.

Now suppose that \lambda =50, μ =200 and θ =30%. There are three models under
consideration for the distribution of individual claim amounts:
A
B
C
fixed claims of 200
exponential with mean 200
gamma with mean 200 and variance 800
Let the corresponding adjustment coefficients be R A , R B and R C .
(ii)
Find the numerical value of R B and show that R B is less than both R A and
R C .

n
(iii)
CT6 A2013–6
⎛ x ⎞
Use the fact that ⎜ 1 + ⎟ ≈ e x for large n to show that R A and R C are
⎝ n ⎠
approximately equal.


%%%%%%%%%%%%%%%%%%%%%%%%%%%%%%%%%%%%%%%%%%%%%%%%%%%%%%%%%%%%%%%%%%%%%%%%%%%%%%%%%%%%%%%%%
9
(i)
ψ ( U )
does not depend on \lambda . This parameter affects the speed with which
the process runs, but does not affect the ultimate probability of ruin.
ψ ( U ) is higher for higher values of μ since the significance of the starting
capital falls as μ rises, providing proportionately less of a buffer.
ψ ( U ) is lower for higher values of θ since the higher θ is the higher the
premiums with no change to claim amounts, so that there is a larger buffer
against ruin.
ψ ( U ) is lower for higher values of U since the higher U is the higher the
larger the buffer against ruin given by the initial capital.
(ii)
The adjustment coefficients are the solutions to:
M X ( R ) = 1 + 200 × 1.3 × R = 1 + 260 R
for the various choices of the moment generating function.
Our first task is to find the parameters in the gamma distribution in C.
Denoting these by \alpha and \beta we have:
\alpha
\alpha
= 200 and 2 = 800
\beta
\beta
Dividing the second by the first we get 1 = 4 so \beta = 0.25 and \alpha = 50.
\beta
Solving for R B we have:
0.005
= 1 + 260 R B .
0.005 − R B
1 = ( 1 + 260 R B ) (1 − 200 R B ).
1 = 1 + 60 R B − 52, 000 R B 2
R B =
60
= 0.001153846.
52, 000
Consider the three functions:
A = e 200 R − 1 − 260 R
Page 9Subject %%%%%%%%%%%%%%%%%%%%%%%%%%%%%%%%%%%%%5 – April 2013 – Examiners’ Report
B =
0.005
− 1 − 260 R
0.005 − R
50
⎛ 0.25 ⎞
C = ⎜
⎟ − 1 − 260 R .
⎝ 0.25 − R ⎠
We can tabulate the values of these functions as follows:
R A B C
0.0001
0.0012 −0.00579
−0.04075 −0.00559
0.003789 −0.00579
−0.0400
So the second function has changed sign, but the first and third have not which
gives the required result.
(iii)
We know that R C satisfies:
⎛ 0.25 ⎞
⎜
⎟
⎝ 0.25 − R C ⎠
50
= 1 + 260 R C .
We can re-write this as:
⎛ 0.25 − R C ⎞
⎜ 0.25 ⎟
⎝
⎠
− 50
= 1 + 260 R C .
1
= 1 + 260 R C .
R
(1 − C ) 50
0. 25
1
⎛ 200 R C ⎞
⎜ 1 − 50 ⎟
⎝
⎠
50
= 1 + 2 60 R C .
But due to the approximation given in the question, the denominator of the left
hand side is approximately e − 200 R C .
So we have, approximately:
1
e
i.e.
Page 10
− 200 R C
= 1 + 260 R C .
e 200 R C = 1 + 260 R C .Subject %%%%%%%%%%%%%%%%%%%%%%%%%%%%%%%%%%%%%5 – April 2013 – Examiners’ Report
Which is the equation satisfied by R A . Hence R C and R A are approximately
equal.
Most candidates scored well in part (i), although many simply stated how the probability of
ruin changes without explaining why. Well prepared candidates scored well on part (ii),
noting the method in previous examinations for finding the root of the equation; however very
few candidates scored well on part (iii) which was stretching.
