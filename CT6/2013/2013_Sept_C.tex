[Total 9]
An insurance company has a portfolio of life insurance policies for 2,000 workers at a
factory. The policies pay out £5,000 if a worker dies in an industrial accident and
£2,000 if a worker dies for any other reason. For each worker, the probability of
death in any year is 0.02 and 25% of deaths are the result of industrial accidents. The
insurance company charges an annual premium of £74.25 per worker.
(i)
Calculate the premium loading used by the insurance company.

The insurance company is considering adopting one of the following three approaches
to reinsurance:
A None.
B 30% proportional reinsurance at a cost of £27 per worker.
C Individual excess of loss reinsurance with retention £3,000 and a
premium of £15 per worker.
(ii) Find the optimal decision under the Bayes criterion. 
(iii) Find the optimal decision under the minimax criterion. 
(iv) Comment on your answer to part (iii).
CT6 S2013–3

[Total 10]
6
The tables below show cumulative data for the number of claims and the total claim
amounts arising from a portfolio of insurance policies.
Claim Numbers
Development Year
0
1
2
2010
2011
2012
87
117
99
132
156
151
Total Claim Amounts
Development Year
0
1
2
2010
2011
2012
43,290
68,900
74,250
87,430 126,310
125,290
Claims are fully run off after two development years.
Estimate the outstanding claims using the average cost per claim method with
grossing up factors.



%%%%%%%%%%%%%%%%%%%%%

5
(i)
The premium loading  is given by:
74.25  (1   )  (0.75  2000  0.25  5000)  0.02  55(1   )
and so

(ii)
74.25
 1  35%.
55
Under A expected profit is:
2000  74.25  2000  0.02  (0.75  2000  0.25  5000)
 38,500.
Under B expected profit is:
2, 000  74.25  2, 000  27  0.7  2, 000  0.02  (0.75  2, 000  0.25  5, 000)
 17,500.
Under C expected profit is:
2000  74.25  2000  15  2000  0.02  (0.75  2000  0.25  3000)
 28, 500
so the optimal course under the Bayes criterion is no reinsurance.
Page 6Subject CT6 (Statistical Methods Core Technical) – September 2013 – Examiners’ Report
(iii)
Under the minimax we need to consider the worst case scenario – which is that
all 2,000 workers die in industrial accidents.
Under this outcome, the losses are:
Under A:
Under B:
Under C:
2000  74.25  2000  5000   9,851,500
2000  (74.25  27)  2000  5000  0.7   6,905,500
2000  (74.25  15)  2000  3000 = 5,881,500
so the optimal decision under the minimax criterion is C.
(iv)
The approach in (iii) puts all the weight on what is at first seems a pretty
unlikely scenario – so that our decision making is driven by something fairly
remote.
That said, the workers are all in the same factory, so it is not inconceivable
that a single catastrophe could result in a large number of claims all at the
same time – i.e. the lives are not independent.
This question was well answered. There are a number of alternative approaches available
(for example working on a per policy basis) which all give the same results, and all of which
were given full credit. Candidates made a range of comments in part (iv) and all sensible
answers were given credit.
6
The average cost per claim is given in the table:
0
2010
2011
2012
497.59
588.89
750.00
1 2
662.35
803.14 836.49
The grossing up factors for average costs are given in the table below (the underlined
figures are the simple averages):
0
2010
2011
2012
497.6
59.49%
588.9
58.06%
750.0
58.77%
1
662.3
79.18%
803.1
79.18%
2
836.5
100.00%
Ult
836.5
1014.3
1276.1
Page 7Subject CT6 (Statistical Methods Core Technical) – September 2013 – Examiners’ Report
The grossing up factors for claim numbers are as follows:
0
2010
87.0
57.62%
117.0
65.56%
99.0
61.59%
2011
2012
1
2
132.0
87.42%
156.0
87.42%
151.0
100.00%
160.7
Average
amount Number Total
836.5
1014.3
1276.1 151.0
178.5
160.7 126310
181006
205126
512442
So the outstanding claims are:
512,442 – 126,310 – 125,290 – 74,250 = 186,592.
This question was well answered.
