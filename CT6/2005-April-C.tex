\documentclass[a4paper,12pt]{article}

%%%%%%%%%%%%%%%%%%%%%%%%%%%%%%%%%%%%%%%%%%%%%%%%%%%%%%%%%%%%%%%%%%%%%%%%%%%%%%%%%%%%%%%%%%%%%%%%%%%%%%%%%%%%%%%%%%%%%%%%%%%%%%%%%%%%%%%%%%%%%%%%%%%%%%%%%%%%%%%%%%%%%%%%%%%%%%%%%%%%%%%%%%%%%%%%%%%%%%%%%%%%%%%%%%%%%%%%%%%%%%%%%%%%%%%%%%%%%%%%%%%%%%%%%%%%

\usepackage{eurosym}
\usepackage{vmargin}
\usepackage{amsmath}
\usepackage{graphics}
\usepackage{epsfig}
\usepackage{enumerate}
\usepackage{multicol}
\usepackage{subfigure}
\usepackage{fancyhdr}
\usepackage{listings}
\usepackage{framed}
\usepackage{graphicx}
\usepackage{amsmath}
\usepackage{chngpage}

%\usepackage{bigints}
\usepackage{vmargin}

% left top textwidth textheight headheight

% headsep footheight footskip

\setmargins{2.0cm}{2.5cm}{16 cm}{22cm}{0.5cm}{0cm}{1cm}{1cm}

\renewcommand{\baselinestretch}{1.3}

\setcounter{MaxMatrixCols}{10}

\begin{document}
\begin{enumerate}

6
[8]
On 1 January 2001 an insurer in a far off land sells 100 policies, each with a five year
term, to householders wishing to insure against damage caused by fireworks. The
insurer charges annual premiums of £600 payable continuously over the life of the
policy.
The insurer knows that the only likely date a claim will be made is on the day of
St Ignitius feast on 1 August each year, when it is traditional to have an enormous
fireworks display. The annual probability of a claim on each policy is 40%. Claim
amounts follow a Pareto distribution with parameters = 10 and = 9,000.
(i) Calculate the mean and standard deviation of the annual aggregate claims. 
(ii) Denote by (U, t) the probability of ruin before time t given initial surplus $U$.
(a)
(b)
%%%%%%%%%%%%%%%%%%%%%%%%%%%%%%%%%%%%%%%%%%%%%%%%%%%%%%%%%%%%%%%%%%%%%%%%%%%
3
Explain why for this portfolio (U, t 1 ) = (U, t 2 ) if
7/12 < t 1 , t 2 < 19/12.
[1]
Estimate (15,000, 1) assuming annual claims are approximately
Normally distributed.
[4]
[Total 9]
PLEASE TURN OVER7
The no claims discount (NCD) system operated by an insurance company has three levels of discount: 0%, 25% and 50%.
If a policyholder makes a claim they remain at or move down to the 0\% discount level
for two years. Otherwise they move up a discount level in the following year or remain at the maximum 50% level.
The probability of an accident depends on the discount level:
Discount Level Probability of accident
0%
25%
50% 0.25
0.2
0.1
The full premium payable at the 0% discount level is 750.
Losses are assumed to follow a lognormal distribution with mean 1,451 and standard deviation 604.4.
Policyholders will only claim if the loss is greater than the total additional premiums that would have to be paid over the next three years, assuming that no further accidents occur.
%%%%%%%%%%%%%%%%%%%%%%%%%%%%%%%%%%%%%%%%%%%%%%%%%%%%%%%%%%%%%%%%%%%%%%%%%%%%%%%%%%%%%%%%%%%%%%%%%%%%%%%%%%%%%%%5
The number of annual claims N follows a binomial distribution:
N ~ B(100, 0.4) then
E(N) = 100
0.4 = 40
and
Var(N) = 100
0.4
0.6 = 24.
Let X denote the distribution of the individual claim amounts, so that X ~
Pareto(10, 9,000). Then
E(X) =
9, 000
= 1,000
10 1
and
Var(X) =
9, 000 2 10
9 2 8
= 1,250,000.
The annual aggregate claim amount S has
E(S) = E(N)E(X) = 40
1,000 = 40,000
and
Var(S) = E(N)Var(X) + Var(N)E(X) 2
= 40
1,250,000 + 24
1,000 2
= 74,000,000
= (8,602.33) 2
Page 5Subject CT6 (Statistical Methods Core Technical)
(ii)
(a)
%%%%%%%%%%%%%%%%%%%%%%%%%%%%%%%%%%%%%%%%%%%%%%%%%%%%%%%%%%%%
Since claims can only fall on one day of the year, there is %%effectively only one day of the year on which ruin can occur, namely 1 August (or strictly shortly thereafter). For a year after 1 August, the insurer will be receiving premiums but paying no claims, and hence solvency will be improving. Hence
\[(U, t 1 ) = (U, t 2 ) if 7/12 < t 1 , t 2 < 19/12.\]
(b)
We must find (15,000, 1). But ruin will have occurred before time 1
only if it occurs at t = 7/12. Just before the claims occur, the insurers assets will be 7/12 100 600 + 15,000 = 50,000 and ruin will occur if the aggregate claims in the first year exceed this level. Assuming that S is approximately normally distributed, we have
P(Ruin) = P(N(40,000, (8,602.33) 2 ) > 50,000)\\
= P N (0, 1)\\
= 1
50, 000 40, 000
8, 602.33
(1.162)
= 0.123.
%%%%%%%%%%%%%%%%%%%%%%%%%%%%%%%%%%%%%%%%%%%%%%%%%%%%%
7
(i)
Denote:
0
0*
1
2
0
0*
1
2
just had a claim
1 claim free year after accident or new customer
25%
50%
Premiums if no claim Premiums if claim Difference
750, 562.50, 375
562.50, 375, 375
375, 375, 375
375, 375, 375 750, 750, 562.50
750, 750, 562.50
750, 750, 562.50
750, 750, 562.50 375
750
937.50
937.50
So minimum claim in state 0 is 375, in state 0* is 750 and in states 1 and 2 is
937.50.
Page 6Subject CT6 (Statistical Methods Core Technical)
(ii)
April 2005
Examiners Report
P(Claim) = P(Claim Accident) . P(Accident)
= P(X > x) P(Accident)
Where X is the loss and x is the minimum loss for which a claim will be made.
E(x) = exp( + 1⁄2 2 ) = 1,451
Var(x) = exp(2( + 1⁄2 2 )) exp(( 2 ) 1) = 604.4 2
Therefore,
exp( 2 ) 1 = 604.4 2 / 1,451 2
exp( 2 ) = 1.1735
2
= 0.16
= 0.4
= 7.2
P(X > 375) = 1 ln 375 7.2
= 0.99927
0.4
P(X > 750) = 1 ln 750 7.2
= 0.9264
0.4
ln 937.50 7.2
= 0.8138
0.4
(X > 937.50) = 1
So the transition matrix is
0.2498 0.7502 0
0
0.2316 0 0.7684 0
0.1628 0 0 0.8372
0.0814 0 0 0.9186
%%%%%%%%%%%%%%%%%%%%%55
(iii)
(iv)
0.2498
+ 0.0814 2
0.7502 0
0.7684 0 *
0.8372 1 + 0.9186 2
0 + 0 * + 1 + 2
1 = 0.7502 0.7684
0
0.8372 1
1
Average premium across portfolio
750
8
Examiners Report
= 0
= 0 *
= 1
= 2
= 1
= 0.5766 0
= 2 (1 0.9186)
2 = 10.2850 1
0 + 0.7502 0 + 0.5766 0 + 10.2850 (0.5766 0 ) = 1
8.2556 0 = 1
0 = 0.1211
0 * = 0.0909
1 = 0.7684 0.0909 = 0.0698
2 = 1
0
0 *
1 = 0.7182
0
+ 0.2316 0 * + 0.1628
April 2005
(0.1211 + 0.0909 + 0.0698
0.75 + 0.7182
0.5) = £467.59
(v) Intention is to automatically premium rate with NCD system. Small number of categories and the relatively low discount result in high proportion of policyholders in maximum discount category. Many more categories and higher discount levels would be required to correctly rate such a heterogeneous population.
\end{document}
