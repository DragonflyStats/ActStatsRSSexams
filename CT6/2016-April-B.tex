3 A child playing a game believes that a six sided die is unfair, and that he has a
probability p > 1/6 of predicting the outcome of any given throw. His mother is less
sure, and her prior beliefs about p are as follows:
   a 1/3 chance that p = 2/6 and
 a 2/3 chance that p = 1/6
The child accurately predicts the results of 4 out of 10 dice throws.
Calculate the posterior probability that p = 1/6. [6]
CT6 A2016–3 PLEASE TURN OVER
4 Let us consider that we need to sample from a discrete random variable Y with
distribution function:
  Y 1 2 3
P 1
3 1
3 1
3
(i) Set out a direct method of sampling from Y. [2]
Consider now another random variable 􀜺 with distribution:
  X 1 2 3
P 1
2 1
3 1
6
(ii) Set out a direct method of sampling from X. [2]
(iii) (a) Explain how you can apply the acceptance-rejection method to
sample X by rejecting/accepting samples from Y.
(b) Calculate how many samples from Y on average are needed to
generate one sample from X. [6]
[Total 10]
%%%%%%%%%%%%%%%%%%%%%%%%%%%%%%%%%%%%%%%%%%%%%%%%%%%%%%%%%%%%%%%%%%%%%%%%%
Q3 Let X be the number of correct predictions, so X ~ Bin (10, p) [1]
P(p = 1/6 | X = 4) = P(p = 1/6 and X = 4) / P(X = 4)
= P(X = 4 | p = 1/6) * P(p = 1/6) / P(X = 4) [1]
P(X = 4 | p = 1/6) = 10 ( )4 ( )6
C4 1 / 6 5 / 6 = 0.0542 659 … [1]
P(X = 4 | p = 2/6) = 10 ( )4 ( )6
C4 2 / 6 4 / 6 = 0.227 607 580… [1]
P(X = 4) = P(p = 1/6) * P(X = 4|p = 1/6) + P(p = 2/6) * P(X = 4|p = 2/6)
= 2/3 * 0.054 265 9 … + 1/3 * 0.227 607 58
= 0.112 046 … [1]
Subject CT6 (Statistical Methods Core Technical) – April 2016 – Examiners’ Report
Page 4
So posterior prob = 0.0542659 * (2/3)/0.112046 = 0.323 [1]
[TOTAL 6]
Candidates who were well prepared on Bayes’ Theorem and previous exam
questions on this topic did very well here, although a number of candidates
struggled.
Q4 (i) Algorithm
(1) Simulate one 􀜷 from U(0,1) [1]
(2) If 0 1 , 1; 1 2 , 2;else 3
3 3 3
<U ≤ Y = <U ≤ Y = Y = [1]
[Total 2]
(ii) Similarly for generating samples from 􀜺
Algorithm
Simulate one U from U(0,1) [1]
If 0 1
2
≤U < , X=1 ; if 1 5
2 6
≤U < , X = 2, else X = 3 [1]
[Total 2]
(iii) The rejection here applied in the same way as in the continuous case we need
to calculate again
1 1 1
max 2 , 3 , 6 max 3 ,1, 3 3 1 1 1 2 6 2
3 3 3
M
 
    =   =   =
     
 
[1]
Hence ( )
( )
1, 2 , 1
3 3
f X
M g x
=    
 
if X = 1, 2,3 respectively. [1]
Algorithm
1 – Simulate Y as in (i). [1]
2 – Sample U from U(0,1) and then take.
If Y = 1 X = Y [½]
U < 2/3 and if Y = 2 take X = Y [½]
If U < 1/3 and if Y = 3 take X = Y [½]
Subject CT6 (Statistical Methods Core Technical) – April 2016 – Examiners’ Report
Page 5
Otherwise start again. [½]
The average number of samples from Y needed for a single sample from X is
of the order M = 3/2, i.e. 1.5 samples on average. [1]
[Total 6]
[TOTAL 10]
Most candidates scored well on parts (i) and (ii), although only the better
prepared candidates were able to apply the acceptance-rejection method to
part (iii).
