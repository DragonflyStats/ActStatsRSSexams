9
For each discount level, find the minimum claim amount for which the
policyholder will make a claim. [2]
Assuming that the cost of repair for each accident has an exponential
distribution with mean £600, calculate the probability that a policyholder
makes a claim at each level of discount. [5]
(iii) Write down the transition matrix and calculate the average premium payment
for a year when the system has reached the equilibrium.
[6]
[Total 13]
(i) Describe the difference between strictly stationary processes and weakly
stationary processes.
[2]
(ii) Explain why weakly stationary multivariate normal processes are also strictly
stationary.
[1]
(iii) Show that the following bivariate time series process, (X n , Y n ) t , is weakly
stationary:
X n = 0.5X n-1 + 0.3Y n-1 + e n x
Y n = 0.1X n-1 + 0.8Y n-1 + e n y
where e n x and e n x are two independent white noise processes.
(iv)
[5]
Determine the positive values of c for which the process
X n = (0.5 + c) X n-1 + 0.3Y n-1 + e n x
Y n = 0.1X n-1 + (0.8 + c) Y n-1 + e n y
is stationary.
CT6 A2008—5
[6]
%%%%%%%%%%%%%%%%%%%%%%%%%%%%%%%%%%%%%%%%%%%%%%%%%%%%%%%%%%%%%%%%%%%%%%%%%%%%%%%%%%%%
9
(i)
Strictly stationary processes have the property that the distribution of
(X t+1 , ... X t+k ) is the same as that of (X t+s+1 , ... X t+s+k ) for each t, s and k. For
the weakly stationary only the first two moments are needed to satisfy
E (X t ) = μ ∀t
and
cov(X t , X t+s ) = γ(s)
∀t, s.
(ii) These two definitions coincide for the multivariate normal processes since the
normal distribution is characterised by the first two moments only.
(iii) In order to confirm that we need to calculate the eigenvalues of the parameter
matrix
⎛ 0.5 0.3 ⎞
A = ⎜
⎟ .
⎝ 0.1 0.8 ⎠
So we need to solve det(A - λI) = 0 which implies the solution of
(0.5 - λ) (0.8 - λ) – 0.03 = 0
0.37 − 1.3λ + λ 2 = 0
We see that this equation is satisfied for λ 1 = 0.8791288 and λ 2 = 0.4208712.
Since they are both smaller than 1, the process is stationary.
(iv)
The parameter matrix here is A c = A + cI , and the eigenvalues equation is now
det( A + cI - λ I ) = 0 or det( A – (λ - c ) I ) = 0.
Page 9Subject CT6 (Statistical Methods Core Technical) — April 2008 — Examiners’ Report
So the eigenvalues of A c are λ 1 + c and λ 2 + c where λ i are those of A .
Since λ i are positive then the required values for c are such that λ 1 + c < 1 and
λ 2 + c < 1.
Hence 0 < c < 1 - λ 1 = 0.1208712, since λ 1 is the largest of the two.
Comment: This was not the easiest question. Some struggled with (ii), (iii) and (iv).
There were quite a few candidates who managed to avoid the calculation of the eigen
values of the matrix A by explicitly expressing each X_n and Y_n series as stationary
univariate AR(2) processes with some white noise terms .
