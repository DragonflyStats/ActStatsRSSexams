CT6 S2016–4
5 (i) (a) Explain what is meant by a sequence of independent, identically
distributed (I.I.D.) random variables.
(b) Give one example of a sequence of I.I.D. random variables.
[3]
Claim amounts Xi from a portfolio of insurance policies are assumed to be I.I.D. and
exponentially distributed, with parameter λ. In a given year there are n claims.
(ii) Show that the total claim amounts follow a gamma distribution, specifying its
parameters. [2]
In practice the individual claim amounts are not I.I.D. but instead the exponential
parameter λi varies between each claim. λi follows a gamma distribution with
parameters α and β.
(iii) Show that the marginal distribution of claim amounts follows a Pareto
distribution with parameters α and β. [5]
[Total 10]
6 Assume that the numbers of accidents for three different risks in five years are as
follows:
  Year 1 Year 2 Year 3 Year 4 Year 5 Total
Risk A 1 4 5 0 2 12
Risk B 1 6 4 6 5 22
Risk C 5 6 4 9 4 28
An actuary is modelling each risk according to a Poisson distribution.
(i) Determine the Poisson parameter for each risk using the method of maximum
likelihood estimation. [5]
(ii) Test the hypothesis that the three risks have the same claim rate, using the
scaled deviances. [5]
[Total 10]


%%%%%%%%%%%%%%%%%%%%%%%%%%%%%%%%%%%%%%%%%%%%%%%%%%%%%%%%%%%%%%%%%%%%%%%%%%
  Q5 (i) (a) Each realisation of the variable is unaffected by previous outcomes and
in turn does not affect future outcomes. [1]
The variables all come from the same distribution with the same
parameters. [1]
(b) E.g. rolling a fair die, tossing a fair coin etc. [1]
Subject CT6 (Statistical Methods Core Technical) – September 2016 – Examiners’ Report
Page 6
(ii) Let
1
n
i
i
S X
=
  = , then
( ) ( )
1
1
[ ] 1 1 i
n n n
n
S X X
i
M t M M t t t
− −
=
        = = =  −   = −    λ    λ  
Π [1]
By independence of claim amounts and uniqueness property of MGFs [½]
This is a gamma distribution with parameters n and λ . [½]
(iii) ( ) ( ) ( ) ( )
( ) ( ) ( )
, |
  0 0
1
0
, |
  /Γ exp
fX x fX x d f fX x d
exp x d
∞ ∞
λ λ λ
∞
α α−
= λ λ= λ λ λ
= β α λ −βλ λ −λ λ
 

( ) { ( ) }
0
exp
Γ
x d
α ∞
= β λα − + β λ λ
α  [1½]
( ) ( )
( )
( )
( ) { ( ) }
1
1
0
Γ 1
\Γ exp
Γ 1
x
x d
x
∞ α+
  α α
α+
  α + +β
=β α λ − + β λ λ
+ β α +  [1]
( ) ( )
( ) 1 ( ) 1
Γ 1
\ Γ , x 0
x x
α
α
α+ α+
  α + αβ =β α = >
  +β +β
[1½]
Since the final integral is the PDF of a Gamma distribution and so equals 1.
This is the PDF of a Pareto distribution with parameters α and β . [1]
[Total 10]
Part (i) was poorly answered, with many candidates simply repeating the
words independent and identical. Part (ii) was well answered, although part
(iii) was relatively poorly answered.
Q6 (i) For risk A with rate μ1 the log-likelihood function is:
  5 5
1 1 1 1 1
1 1
log log i 5 log i !
  i i
L y y
= =
  = μ  − μ −
Subject CT6 (Statistical Methods Core Technical) – September 2016 – Examiners’ Report
Page 7
5
1 1 1
1
= 12log 5 log i !
  i
y
=
  μ − μ − [1½]
And therefore the mle for μ1 is obtained for
1 1
∂logL = 12 −5 = 0
∂μ μ
[1]
i.e. 
μ1 = 2.4 [½]
Similarly we have that 2
22 4.4
5
μ = = and 
3
28 5.6
5
μ = = . [2]
(ii) Under the assumption that these risks share the same rate i.e. μ1 = μ2 = μ3 = μ
then the mle estimate for this is simply ˆ 62
15
μ = . [½]
In order to compare these models we can use the scaled deviances to compare
these models and using the chi-squared test.
The difference in the scaled deviance is chi-square with 3 − 1 = 2 degrees of
freedom. [1]
2(logL1 + logL2 + logL3 − logL)
     
= 2(12logμ1 − 5μ1 + 22logμ2 − 5μ2 + 28logμ3 − 5μ3 − 62logμˆ +15 μˆ ) [1]
With the
5
1
1
log i !
  i
y
= 
+
  5 5
2 3
1 1
log i ! log i !
  i i
y y
= =
   + cancelling out in the difference.
Hence
2( ) 1 2 3 logL + logL + logL − logL
2 12log2.4 22log 4.4 28log5.6 62log 62 5(12 22 28) 15 62
15 5 15
 + + 
=  + + − − + 
 
2 12log2.4 22log4.4 28log5.6 62log 62 6.71034
15
=  + + −  =  
 
. [1½]
Subject CT6 (Statistical Methods Core Technical) – September 2016 – Examiners’ Report
Page 8
This value is above 5.991 which is the critical value at the upper 5% level and
therefore conclude that mean claim rates are different. [1]
[Total 10]
Part (i) was very well answered by most candidates. Fewer candidates
scored full marks in part (ii), despite similar questions having been asked in
several recent examinations.
