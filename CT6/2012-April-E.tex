\documentclass[a4paper,12pt]{article}

%%%%%%%%%%%%%%%%%%%%%%%%%%%%%%%%%%%%%%%%%%%%%%%%%%%%%%%%%%%%%%%%%%%%%%%%%%%%%%%%%%%%%%%%%%%%%%%%%%%%%%%%%%%%%%%%%%%%%%%%%%%%%%%%%%%%%%%%%%%%%%%%%%%%%%%%%%%%%%%%%%%%%%%%%%%%%%%%%%%%%%%%%%%%%%%%%%%%%%%%%%%%%%%%%%%%%%%%%%%%%%%%%%%%%%%%%%%%%%%%%%%%%%%%%%%%

\usepackage{eurosym}
\usepackage{vmargin}
\usepackage{amsmath}
\usepackage{graphics}
\usepackage{epsfig}
\usepackage{enumerate}
\usepackage{multicol}
\usepackage{subfigure}
\usepackage{fancyhdr}
\usepackage{listings}
\usepackage{framed}
\usepackage{graphicx}
\usepackage{amsmath}
\usepackage{chngpage}

%\usepackage{bigints}
\usepackage{vmargin}

% left top textwidth textheight headheight

% headsep footheight footskip

\setmargins{2.0cm}{2.5cm}{16 cm}{22cm}{0.5cm}{0cm}{1cm}{1cm}

\renewcommand{\baselinestretch}{1.3}

\setcounter{MaxMatrixCols}{10}

\begin{document}

\begin{enumerate}
%%%%%%%%%%%%%%%%%%%%%%%%%
\item 9
Consider the time series model
(1 − α B ) 3 X t = e t
where B is the backwards shift operator and e t is a white noise process with variance
σ 2 .
(i)
Determine for which values of α the process is stationary.
[2]
Now assume that α = 0.4.
(ii)
(iii)
10
(a) Write down the Yule-Walker equations.
(b) Calculate the first two values of the auto-correlation function ρ 1 and
ρ 2 .
[7]
Describe the behaviour of ρ k and the partial autocorrelation function φ k as
k →∞ .
[3]
[Total 12]
\item Let X 1 and X 2 be random variables with moment generating functions M X 1 ( t ) and
M X 2 ( t ) respectively. A new random variable Y is formed by choosing a sample
from X 1 with probability p or a sample from X 2 with probability 1 − p .
\begin{enumerate}[(a)]
\item 
Show that the moment generating function of Y is given by
M Y ( t ) = pM X 1 ( t ) + ( 1 − p ) M X 2 ( t )

A portfolio of insurance policies consists of two different types of policy. Claims on
type 1 policies arrive according to a Poisson process with parameter λ 1 and claim
amounts have a distribution X 1 . Claims on type 2 policies arrive according to a
Poisson process with parameter λ 2 and claim amounts have a distribution X 2 .
\item 
Show that aggregate claims on the whole portfolio follow a compound Poisson
distribution, specifying the claim rate and the claim size distribution.
Now suppose that λ 1 = 10 and λ 2 = 15 and that the claim sizes are exponentially
distributed with mean 50 for type 1 policies and mean 70 for type 2 policies.
\item 
CT6 A2012–5
Construct an algorithm for simulating total claims on the whole portfolio. 
\end{enumerate}
\end{enumerate}
%%%%%%%%%%%%%%%%%%%%%%%%%%%%%%%%%%%%%%%%%%%%%%%%%%%%%%%%%%%%%%%%%%%%%%%%%%%%%%%%%%%%%%%%%%%%%%%%%%%%%%%%%%%5
\newpage 
9
(i)
The characteristic polynomial is (1 − α Y ) 3 = 0 .
This has a triple root at 1
α
and so the process is stationary when
i.e. α < 1 .
(ii)
Expanding the cubic equation and rearranging gives:
X t − 3 α X t − 1 + 3 α 2 X t − 2 − α 3 X t − 3 = e t
So the Yule-Walker equations give:
ρ 0 − 3 αρ 1 + 3 α 2 ρ 2 − α 3 ρ 3 = σ 2
ρ 1 − 3 α + 3 α 2 ρ 1 − α 3 ρ 2 = 0 (A)
ρ 2 − 3 αρ 1 + 3 α 2 − α 3 ρ 1 = 0 (B)
ρ 3 − 3 αρ 2 + 3 α 2 ρ 1 − α 3 ρ 0 = 0
(
)
So re-writing we have from (A) ρ 1 1 + 3 α 2 − 3 α = α 3 ρ 2
And substituting into (B) gives
(
)
ρ 1 1 + 3 α 2 − 3 α
α
i.e.
Page 12
3
− 3 αρ 1 + 3 α 2 − α 3 ρ 1 = 0
ρ 1 (1 + 3 α 2 − 3 α 4 − α 6 )
α 3
=
3 α − 3 α 5
α 3
1
> 1
αSubject CT6 (Statistical Methods) – April 2012 – Examiners’ Report
i.e.
ρ 1 =
3 α (1 − α 4 )
(1 + 3 α 2 − 3 α 4 − α 6 )
= 0.83573487
And so
ρ 2 =
(
)
3 α 1 − α 4 (1 + 3 α 2 )
2
4
6
(1 + 3 α − 3 α − α ) α
3
−
3
α 2
= 0.576368876
Alternative solution:
Express the Yule-Walker equations in terms of the covariances:
X t = 1.2 X t − 1 − 0.48 X t − 2 + 0.064 X t − 3 + e t
γ 0 = 1.2 γ 1 − 0.48 γ 2 + 0.064 γ 3 + σ 2
γ 1 = 1.2 γ 0 − 0.48 γ 1 + 0.064 γ 2
γ 2 = 1.2 γ 1 − 0.48 γ 0 + 0.064 γ 1
γ 3 = 1.2 γ 2 − 0.48 γ 1 + 0.064 γ 0
Or in general:
γ 0 = 1.2 γ 1 − 0.48 γ 2 + 0.064 γ 3 + σ 2
γ k = 1.2 γ k − 1 − 0.48 γ k − 2 + 0.064 γ k − 3 k ≥ 1
Simplifying the second and third equations:
148 γ 1 = 1.2 γ 0 + 0.064 γ 2 ⇒ γ 1 =
30 γ
37 0
8 γ
+ 185
2
γ 2 = 1.264 γ 1 + 0.064 γ 1
To obtain:
γ 2 =
200 γ
347 0
λ 1 =
290 γ
347 0
Dividing both by γ 0 gives the same solutions as above.
(iii)
The series is an AR(3) series. The asymptotic behaviour is therefore that ρ k
decays exponentially to zero
whilst φ k is zero for k>3.
The latter parts of this question were not particularly well answered. Candidates generally
showed an understanding of how to solve the problem, but made a number of arithmetic and
algebraic slips.
%%%%%%%%%%%%%%%%%%%%%%%%%%%%%%%%%%%%%%%%%%%%%%%%%%%%%%%%%%%%%%%%%%%%%%%%%%%%%%%%%%%%%%%%%%%%%%%%%%%%%%%%%%%5
\newpage
%%% Page 13Subject CT6 (Statistical Methods) – April 2012 – Examiners’ Report
10
(i)
( )
( )
( )
M Y ( t ) = E e tY = pE e tX 1 + ( 1 − p ) E e tX 2
= pM X 1 ( t ) + ( 1 − p ) M X 2 ( t )
(ii)
Let S 1 , S 2 denote aggregate claims on the type 1 and type 2 policies
respectively, and let N 1 , N 2 denote the number of claims from type 1 and type
2 policies respectively. Let S = S 1 + S 2 denote the aggregate claims on the
combined portfolio. We know that S 1 , S 2 follow compound Poisson processes
and so
M S i ( t ) = M N i (log M X i ( t )) = exp( λ i ( M X i ( t ) − 1))
Now
M S ( t ) = M S 1 + S 2 ( t ) = M S 1 ( t ) M S 2 ( t )
( (
) )
= exp λ 1 M X 1 ( t ) − 1 exp( λ 2 ( M X 2 ( t ) − 1))
⎡
⎛ λ 1
⎞ ⎤
λ 2
= exp ⎢ ( λ 1 + λ 2 ) ⎜
M X ( t ) +
M X ( t ) − 1 ⎟ ⎥
1
2
λ 1 + λ 2
⎝ λ 1 + λ 2
⎠ ⎦
⎣
= exp(( λ 1 + λ 2 )( pM X 1 ( t ) + ( 1 − p ) M X 2 ( t ) − 1) where p =
λ 1
λ 1 + λ 2
= exp(( λ 1 + λ 2 )( M Y ( t ) − 1))
Where Y is defined as in part (i). This is of the form M N ( logM Y ( t ) ) where N
is a Poisson distribution with parameter λ 1 + λ 2 . Hence S has a compound
Poisson distribution with rate λ 1 + λ 2 and where individual claim amounts are
λ 1
taken from distribution X 1 with probability p =
and from distribution
λ 1 + λ 2
λ 2
X 2 with probability 1 − p =
.
λ 1 + λ 2
%%%%%%%%%%%%%%%%%%%%%%%%%%%%%%%%%%%%%%%%%%%%%%%%%%%%%%%%%%%%%%%%%%%%%%%%%%%%%%%%
Page 14Subject CT6 (Statistical Methods) – April 2012 – Examiners’ Report
(iii)
Step 1
We first begin by generating a random sample from N ~ P ( 25 ) as follows:
Let u be a random sample from a Uniform distribution on (0,1).
Find the positive integer i such that P ( N ≤ i − 1 ) < u ≤ P ( N ≤ i ) (using the
cumulative Poisson tables)
Then i is the simulated number of claims.
Step 2
Now we simulate the individual claim amount
Generate v a sample from a Uniform distribution on (0,1).
10
10
=
= 0.4 then we have a type 1 claim otherwise we have a
10 + 15 25
type 2 claim. Let the claim type be j.
If v ≤
Put μ 1 = 50 and μ 2 = 70 . Generate w a sample from a uniform distribution on
(0,1).
The simulated claim Z is given by setting
F X j ( Z ) = w
⎛ 1 ⎞
So 1 − exp ⎜
Z ⎟ = w
⎜ μ
⎟
⎝ j ⎠
So Z = − μ j ln( 1 − w )
Step 3
Repeat Step 2 for a total of i samples and add the results.
Alternative algorithm: simulate the two results separately and add together at
the end.
This question was not answered well. In particular, many candidates did not attempt part
(iii). Of those that did, most had a good attempt at step 2, but very few got step 1 (to deduce
the simulated number of claims).

\end{document}
