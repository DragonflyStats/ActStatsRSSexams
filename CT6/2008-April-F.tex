[Total 14]
PLEASE TURN OVER10
A bicycle wheel manufacturer claims that its products are virtually indestructible in
accidents and therefore offers a guarantee to purchasers of pairs of its wheels. There
are 250 bicycles covered, each of which has a probability p of being involved in an
accident (independently). Despite the manufacturer’s publicity, if a bicycle is
involved in an accident, there is in fact a probability of 0.1 for each wheel
(independently) that the wheel will need to be replaced at a cost of £100. Let S denote
the total cost of replacement wheels in a year.
(i)
Show that the moment generating function of S is given by
⎡ pe 200 t + 18 pe 100 t + 81 p
⎤
M S ( t ) = ⎢
+ 1 − p ⎥
100
⎣ ⎢
⎦ ⎥
(ii)
250
.
Show that E ( S ) = 5, 000 p and Var ( S ) = 550, 000 p − 100, 000 p 2
[4]
[6]
Suppose instead that the manufacturer models the cost of replacement wheels
as a random variable T based on a portfolio of 500 wheels, each of which
(independently) has a probability of 0.1p of requiring replacement.
(iii) Derive expressions for E(T) and Var(T) in terms of p.
(iv) Suppose p = 0.05.
(a)
(b)
(b)
Calculate the mean and variance of S and T.
Calculate the probabilities that S and T exceed £500.
Comment on the differences.
END OF PAPER
CT6 A2008—6
[2]
[5]
[Total 17]
%%%%%%%%%%%%%%%%%%%%%%%%%%%%%%%%%%%%%%%%%%%%%%%%%%%%%%%%%%%%%%%%%%%%%%%%%%%%%%%%%%%%%%%%%%%%%%%%%%%%%%%%%%%%%%%%%%%%%%
10
(i)
Let N denote the annual number of accidents. Then N ~ B (250, p ) and (from the tables) M N ( t ) = ( pe t + 1 − p ) 250
If there is an accident, then the total cost of replacement wheels, X , has the following distribution:
Number of wheels requiring replacement
Cost of replacement X
Probability
0
£0
0.81
1
£100
0.18
2
£200
0.01
And M X ( t ) = 0.01 e 200 t + 0.18 e 100 t + 0.81 .
So
M S ( t ) = M N (log M X ( t ))
= ( pe log M X ( t ) + 1 − p ) 250
= ( pM X ( t ) + 1 − p ) 250
= ( p (0.01 e 200 t + 0.18 e 100 t + 0.81) + 1 − p ) 250
⎛ pe 200 t + 18 pe 100 t + 81 p
⎞
= ⎜
+ 1 − p ⎟
⎜
⎟
100
⎝
⎠
(ii)
250
E ( S ) = M S ' (0)
⎛ pe 200 t + 18 pe 100 t + 81 p
⎞
+ 1 − p ⎟
M S '( t ) = 250 × ⎜
⎜
⎟
100
⎝
⎠
E ( S ) = M S '(0) = 250 × 1 × 20 p = 5000 p
Page 10
249
× (2 pe 200 t + 18 pe 100 t )Subject CT6 (Statistical Methods Core Technical) — April 2008 — Examiners’ Report
E ( S 2 ) = M S ''(0)
⎛ pe 200 t + 18 pe 100 t + 81 p
M S ''( t ) = 250 × 249 × ⎜
+ 1 −
⎜
100
⎝
⎛ pe 200 t + 18 pe 100 t + 81 p
+ 250 × ⎜
+ 1 −
⎜
100
⎝
⎞
p ⎟
⎟
⎠
⎞
p ⎟
⎟
⎠
248
× (2 pe 200 t + 18 pe 100 t ) 2
249
× (400 pe 200 t + 1800 pe 100 t )
M S ''(0) = 250 × 249 × (20 p ) 2 + 250 × 1 × 2200 p = 24,900, 000 p 2 + 550, 000 p
Var ( S ) = E ( S 2 ) − E ( S ) 2 = 24,900, 000 p 2 + 550, 000 p − (5000 p ) 2 = 550, 000 p − 100, 000 p 2 .
Alternatively, we note that
E ( N ) = 250 p and Var( N ) = 250 p (1 - p )
E ( X ) = 0.01 × 200 + 0.18 × 100 = 20
Var( X ) = 40000 × 0.01 + 10000 × 0.18 – 20 × 20 = 1800
and
E ( S ) = E ( N ) E ( X ) = 250 p × 20 = 5000 p
Var( S ) = E ( N ) Var( X ) +Var( N ) × E ( X ) × E ( X )
Var( S ) = 250 p × 1800 + 250 p (1 - p ) × 20 × 20
= 450,000 p + 100,000 p (1 - p ) = 550,000 p - 100,000 p 2 .
(iii)
Let W denote the total number of wheels needing replacement. Then
W ~ B (500,0.1 p ) and T = 100 W
Hence
E ( T ) = 100 E ( W ) = 100 × 500 × 0.1 p = 5000 p
and
Var ( T ) = Var (100 W ) = 100 2 Var ( W ) = 100 2 × 500 × 0.1 p × (1 − 0.1 p )
= 500, 000 p (1 − 0.1 p )
(iv)
(a)
If p = 0.05 then
E ( S ) = E ( T ) = 250.
Var ( S ) = 550 , 000 × 0 . 05 − 100 , 000 × 0 . 05 × 0 . 05 = 27 , 250 = 165 . 08 2
Var ( T ) = 500 , 000 × 0 . 05 × 0 . 995 = 24 , 875 = 157 . 72 2
Page 11Subject CT6 (Statistical Methods Core Technical) — April 2008 — Examiners’ Report
(b)
And so assuming both can be approximated by a normal distribution, and allowing for a continuity correction
550 − 250
) = P ( N ( 0 , 1 ) > 1 . 817 )
165 . 08
= 1 − ( 0 . 7 × 0 . 96562 + 0 . 3 × 0 . 96485 )
P ( S > 550 ) ≈ P ( N ( 0 , 1 ) >
= 0 . 034611
550 − 250
) = P ( N ( 0 , 1 ) > 1 . 902 )
157 . 72
= 1 − ( 0 . 8 × 0 . 97128 + 0 . 2 × 0 . 97193 )
P ( T > 500 ) ≈ P ( N ( 0 , 1 ) >
= 0 . 02859
(c)
The two distributions have the same mean, but different variances – the
variance of S is slightly higher than that of T . This leads to a higher
probability for such a loss under S than under the approximation T .
Though the probabilities are both small in absolute terms, that for S is
20% higher than that for T . Effectively, fewer accidents are needed
under S to give a high loss, because each accident can lead to two
wheels being replaced, whereas under T only one wheel can be
damaged per accident.
Comment: This was a challenging question with many students scoring well here and
some trying to fudge the answers for (i).
END OF EXAMINERS’ REPORT
Page 12
