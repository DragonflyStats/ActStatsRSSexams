\documentclass[a4paper,12pt]{article}

%%%%%%%%%%%%%%%%%%%%%%%%%%%%%%%%%%%%%%%%%%%%%%%%%%%%%%%%%%%%%%%%%%%%%%%%%%%%%%%%%%%%%%%%%%%%%%%%%%%%%%%%%%%%%%%%%%%%%%%%%%%%%%%%%%%%%%%%%%%%%%%%%%%%%%%%%%%%%%%%%%%%%%%%%%%%%%%%%%%%%%%%%%%%%%%%%%%%%%%%%%%%%%%%%%%%%%%%%%%%%%%%%%%%%%%%%%%%%%%%%%%%%%%%%%%%

\usepackage{eurosym}
\usepackage{vmargin}
\usepackage{amsmath}
\usepackage{graphics}
\usepackage{epsfig}
\usepackage{enumerate}
\usepackage{multicol}
\usepackage{subfigure}
\usepackage{fancyhdr}
\usepackage{listings}
\usepackage{framed}
\usepackage{graphicx}
\usepackage{amsmath}
\usepackage{chngpage}

%\usepackage{bigints}
\usepackage{vmargin}

% left top textwidth textheight headheight

% headsep footheight footskip

\setmargins{2.0cm}{2.5cm}{16 cm}{22cm}{0.5cm}{0cm}{1cm}{1cm}

\renewcommand{\baselinestretch}{1.3}

\setcounter{MaxMatrixCols}{10}

\begin{document}

\begin{enumerate}
%%%%%%%%%%%%%%%%%%%%%%%%%
\item 10
(i)
A Poisson process is characterised by the probability of a single claim arising
in a small time interval dt being dt (with no probability of more than one
claim).
For the reinsurer, the probability of a claim arising in a small time interval dt
is given by
dt × P ( X i > M )
=  dt  

M
1
1   x
e dx


x
=  dt    
   e   M
=  dt  e

M
=    
  e
M

 dt


so we have a Poisson process with parameter  e
Page 14

M

.Subject CT6 (Statistical Methods Core Technical) – April 2015 – Examiners’ Report
(ii)
M X i ( t ) = E ( e tX i )
x

   e t ( x  M ) 1 e  dx
  M


M

= e   
 1  e 
0 t
= 1  e

M

 

M
 1

1  Mt  x     t  
e e
dx


= 1  e
= 1  e
= 1  e
= 1  e
= 1  e
=1  e
(iii)
(a)
1

 1  
 x   t  


e    
 e  Mt  
 1  t

  
  M
M


 1
 1  
 M   t  

 

 e  Mt 
 e    
 1  t

  
 
 M
  M
  M
 
1 
 1 

 1  t  
 M
   t  


 1  t  

M

 e


M


1
1  t 
 t
1   t
Now M X i ( t ) = 1  e
So
M X ' i ( t )

= e
and so E(X i ) =
M


M


 t
1   t
 

 t

  

2
 1   t (1   t )

M X ' i (0) =
 e

M

Page 15Subject CT6 (Statistical Methods Core Technical) – April 2015 – Examiners’ Report
(b)
The reinsurers annual rate of premium income is given by ( 1 + )
E(X i ). So the adjustment coefficient satisfies
 + (1 + ) E(X i )R =  M X i ( R )
i.e.
(iv)
1 + (1 + ) E(X i )R = M X i ( R )

M

R = 1  e
i.e. 1 + (1 + )  e
i.e. (1 + ) R =
i.e. (1 + ) R(1  R) = R
i.e. (1 + ) (1  R) = 1
i.e. 1 +   R  (R) = 1
i.e. R(1 + ) = 
i.e. R =

M


 R
1   R
 R
1   R

(1   ) 
R does not depend on the retention M.
This is a surprising result at first glance, but arises because of the memoryless
feature of the exponential distribution
i.e. X i  MX i > M is exponential with parameter
1

so the reinsurers claim process is just a slower version of the insurers.
Full credit was given for alternative solutions to part (i) and part (iv). This question was
relatively poorly answered, although those candidates who were able to make a good attempt
tended to score well.
\end{document}
