7 Claim amounts, X, arising from a portfolio of insurance policies follow a Pareto
distribution, with parameters α and λ. The insurance company has bought excess of
loss reinsurance cover, with retention M > 0.
The reinsurer only has a record of claims greater than M. Consider the truncated
distribution of claim amounts, Z = X – M | X > M.
(i) Show that Z also follows a Pareto distribution, but with parameters α and
λ + M. [4]
CT6 S2016–5 PLEASE TURN OVER
Claim amounts, X’, have now increased by a factor k, such that a claim incurred is k
times an equivalent claim previously incurred, and k is greater than 1. The retention
level M is unchanged.
(ii) Show that the distribution of X’ still follows a Pareto distribution, and
determine its parameters. [4]
The truncated distribution of claim amounts is now Z’, where Z’ = X’ – M | X’ > M.
(iii) State the distribution of Z’, using the results from parts (i) and (ii), including
statement of parameters. [1]
(iv) Comment on whether or not the average claim amount retained by the
insurance company has increased by a factor of k. [2]
[Total 11]
8 The table below shows incremental claim amounts paid on a portfolio of general
insurance policies, where claims are assumed to fully run off after three years.
Underwriting
Year
Development Year
0 1 2 3
2012 504 286 110 35
2013 621 302 120
2014 685 340
2015 801
Past and projected future inflation is given by the following index (measured to the
                                                                     mid-point of the relevant year).
Year
Index
2012 100
2013 103
2014 105
2015 106
2016 105
2017 107
2018 110
Estimate the outstanding claims reserve using the inflation-adjusted chain ladder
technique. [12]

Q7 (i) Let Z be the reinsurer claim distribution.
Then ( ) ( )
( )
1
f z M
g z
F M
+
  =
  −
where f(x) and F(x) refer to the underlying claim
distribution [1]
( )
( )
( )
1 ( ) f z M ;F M 1
z M M
α α
α+ α
+ = αλ = − λ
λ+ + λ +
  [1½]
So ( )
( )
( ) ( )
1 ( ) 1
M M
g z
z M z M
α α α
α+ α α+
  αλ λ + α λ + = =
  λ + + λ λ + +
  [1]
This is in the form of a Pareto distribution with parameters α and λ +M . [½]
(ii) Let ( ) ( )
( ) 1
0
;
y
k
Y kX P Y y P kX y dz
z
α
α+
  = < = < = αλ
λ+  [1½]
Let
1 ( ) 1
0 0
;
y y x kz dx k dx
x k k x
k
α αα
α+ α+
  = = αλ = αλ
λ +  λ +  
 
  . [2]
Which is in the form of a Pareto distribution with parameters α and kλ . [½]
(iii) The reinsurer’s distribution of claims is therefore Pareto with parameters α
and kλ +M. [1]
(iv) The average claim retained by the insurer has increased by a factor less than k
since the retention M is unchanged, so on average a greater proportion of
claims get passed on to the reinsurer. [2]
For strong candidates familiar with the bookwork this question was very
straightforward, but few candidates were able to score well here.
Subject CT6 (Statistical Methods Core Technical) – September 2016 – Examiners’ Report
Page 9
Q8 Adjusting for past inflation to 2015 prices gives
Underwriting
Year
Development Year
0
1 2 3
2012 504 * 106/100
= 534.24
286 * 106/103
= 294.33
110 * 106/105
= 111.05
35
2013 621 * 106/103
= 639.09
302 * 106/105
= 304.88
120
2014 685 * 106/105
= 691.52
340
2015 801
[2]
Cumulative figures (2015 prices)
Underwriting
Year
Development Year
0
1 2 3
2012 534.24 828.57 939.62 974.62
2013 639.09 943.97 1,063.97
2014 691.52 1,031.52
2015 801
[1]
Development factors:
  Year 2 to 3: 974.62 / 939.62 = 1.037249 [1]
Year 1 to 2: (939.62 + 1063.97) / (828.57 + 943.97) = 1.130350 [1]
Year 0 to 1: (828.57 + 943.97 + 1031.52) / (534.24 + 639.09 + 691.52) = 1.503638
[1]
Projected cumulative figures (2015 prices)
Underwriting
Year
Development Year
0 1 2
3
2012 534.24 828.57 939.62 974.62
2013 639.09 943.97 1,063.97 1,103.60
2014 691.52 1,031.52 1,165.98 1,209.41
2015 801 1,204.41 1,361.40 1,412.12
[2]
Subject CT6 (Statistical Methods Core Technical) – September 2016 – Examiners’ Report
Page 10
Projected incremental figures (2015 prices)
Underwriting
Year
Development Year
0
1 2 3
2012
2013 39.63
2014 134.46 43.43
2015 403.41 156.99 50.72
[1]
Adjusting for future inflation
Underwriting
Year
Development Year
0
1 2 3
2012
2013 39.63 * 105 / 106
= 39.26
2014 134.46 * 105 / 106
= 133.19
43.43 * 107 / 106
= 43.84
2015 403.41* 105 / 106
= 399.60
156.99 * 107 / 106
= 158.47
50.72 * 110 / 106
= 52.63
[2]
The estimated reserve is the sum of these: 827.0 [1]
[Total 12]
Most candidates scored very well on this straightforward chain ladder
question.
