\documentclass[a4paper,12pt]{article}
%%%%%%%%%%%%%%%%%%%%%%%%%%%%%%%%%%%%%%%%%%%%%%%%%%%%%%%%%%%%%%%%%%%%%%%%%%%%%%%%%%%%%%%%%%%%%%%%%%%%%%%%%%%%%%%%%%%%%%%%%%%%%%%%%%%%%%%%%%%%%%%%%%%%%%%%%%%%%%%%%%%%%%%%%%%%%%%%%%%%%%%%%%%%%%%%%%%%%%%%%%%%%%%%%%%%%%%%%%%%%%%%%%%%%%%%%%%%%%%%%%%%%%%%%%%%
\usepackage{eurosym}
\usepackage{vmargin}
\usepackage{amsmath}
\usepackage{graphics}
\usepackage{epsfig}
\usepackage{enumerate}
\usepackage{multicol}
\usepackage{subfigure}
\usepackage{fancyhdr}
\usepackage{listings}
\usepackage{framed}
\usepackage{graphicx}
\usepackage{amsmath}
\usepackage{chngpage}
%\usepackage{bigints}
\usepackage{vmargin}

% left top textwidth textheight headheight

% headsep footheight footskip

\setmargins{2.0cm}{2.5cm}{16 cm}{22cm}{0.5cm}{0cm}{1cm}{1cm}
\renewcommand{\baselinestretch}{1.3}
\setcounter{MaxMatrixCols}{10}
\begin{document}

\begin{enumerate}
3
An insurance company has collected data on the number of claims arising from
certain risks over the last n years. The number of claims from the i th risk in the j th
year is denoted by X_{ij} for i = 1, 2, ..., N and j = 1, 2, ..., n.
The distribution of X_{ij} depends on an unknown parameter q i . The X_{ij} are independent
identically distributed random variables given q i .
\begin{enumerate}
\item (i) Describe briefly what is meant by each of the following: m(q), s 2 (q), E(s 2 (q)),
var(m(q)), and Z, when using Empirical Bayes Credibility Theory (EBCT)
Model 1.
\item 
(ii) Explain how the value of Z depends on the following factors: n, E(s 2 (q)),
var(m(q)).
\end{enumerate}
%%%%%%%%%%%%%%%%%%%%%%%%%%%%%%%%
\newpage
Q3
\begin{itemize}
    \item 

(i)
m ( \theta ) is the average claim amount for each risk for a given value of $\theta_i$

s 2 ( \theta ) is the variance of the claim amount for each risk given a value of $\theta_i$
E ( s 2 ( \theta ) ) is the average variability of data values from year to year for a single
risk, I 
var( m ( \theta ) ) is the variability of the average data values for different risks 
\item Z is the credibility factor for the EBCT 1 model / weight placed on the sample
mean

(ii)
(a)
(b)
Z increases as n increases, since we place more weight on the data for
that risk

(
)
\item As E s 2 ( \theta ) increases, Z decreases since the variance of the data
from the individual risk is high and so we place more weight on the
collective data.
[2]
(c)
(
)
\item As var m ( \theta ) increases, Z increases since it implies that the means of
the individual risks are very different, so we place more weight on the
individual risk data compared to the collective.
%%%%%%%%%%%%%%%%%%%%%%%%

Most candidates scored well here, although some lost marks through a
combination of only using formulae in part (i) and giving insufficient
explanations in part (ii).
\end{itemize}
\end{document}
