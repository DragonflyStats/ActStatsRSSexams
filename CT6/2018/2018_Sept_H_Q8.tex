\documentclass[a4paper,12pt]{article}
%%%%%%%%%%%%%%%%%%%%%%%%%%%%%%%%%%%%%%%%%%%%%%%%%%%%%%%%%%%%%%%%%%%%%%%%%%%%%%%%%%%%%%%%%%%%%%%%%%%%%%%%%%%%%%%%%%%%%%%%%%%%%%%%%%%%%%%%%%%%%%%%%%%%%%%%%%%%%%%%%%%%%%%%%%%%%%%%%%%%%%%%%%%%%%%%%%%%%%%%%%%%%%%%%%%%%%%%%%%%%%%%%%%%%%%%%%%%%%%%%%%%%%%%%%%%
\usepackage{eurosym}
\usepackage{vmargin}
\usepackage{amsmath}
\usepackage{graphics}
\usepackage{epsfig}
\usepackage{enumerate}
\usepackage{multicol}
\usepackage{subfigure}
\usepackage{fancyhdr}
\usepackage{listings}
\usepackage{framed}
\usepackage{graphicx}
\usepackage{amsmath}
\usepackage{chngpage}
%\usepackage{bigints}
\usepackage{vmargin}

% left top textwidth textheight headheight

% headsep footheight footskip

\setmargins{2.0cm}{2.5cm}{16 cm}{22cm}{0.5cm}{0cm}{1cm}{1cm}
\renewcommand{\baselinestretch}{1.3}
\setcounter{MaxMatrixCols}{10}
\begin{document}

%%--- Question 8

For a portfolio of insurance policies, claims X i are independent and follow a gamma
distribution, with parameters $\alpha  = 6$ and $\beta$ , which is unknown.
A random sample of n claims, X 1 ,..., X n is selected, with mean X .
\begin{enumerate}[(a)]
\item(i) Derive an expression for the estimator of $\beta$ using the method of moments. 
\item (ii) Explain what the Maximum Likelihood Estimator (MLE) of $\beta$  represents. 
\item (iii) Derive an expression for the MLE of $\beta$, commenting on the result.
\item (iv) State the Moment Generating Function (MGF) of X.
\end{enumerate}
Let Y = 2n \beta  X .
(v)


Derive the MGF of Y, and hence its distribution, including statement of
parameters.[5]
[Total 15]
%%%%%%%%%%%%%%%%%%%%%%%%%%%%%%%%%%%%%%%%%%%%%%%%%%%%%%%%%%%%%%
\newpage 
Q8
(i)
(ii)
Mean = \alpha 
\alpha 
\beta  ⇒ \beta  = X

The MLE is the estimate that maximises the likelihood of having observed the
sample data.

n
L ( \beta  ) = ∏
i = 1
(iii)
\beta  \alpha  \alpha  − 1 − \beta  x
x i e
Γ ( \alpha  )
i
n
[11⁄2]
log L ( \beta  ) \propto n \alpha  ln \beta  − \beta  ∑ x i
i = 1
Then differentiate
n
n \alpha 
0
− ∑ x i =
\beta 
i = 1
⇒ \beta  = \alpha  n
= \alpha  = 6
X
X
i
∑ x
[11⁄2]
Check for maximum
d 2 ln L − 6 n
=
< 0
\beta  2
d \beta  2 
In this case the MLE and method of moments lead to the same result. 
%Page 8Subject CT6 (Statistical Methods Core Technical) – September 2018 – Examiners’ Report
(iv)
(v)

t 
 1 − 
 \beta  
− \alpha 

2 \beta  t ∑ X i
2 n \beta  tX
=
( e tY ) E ( e =
) E ( e =
)
M
E =
Y ( t )
n
∏ E ( e \beta 
2 tX i
)
[11⁄2]
i = 1
by independence
[1⁄2]
− \alpha 
 2 \beta  t 
− n \alpha 
M Y ( t ) =
( 1 − 2 t )
∏ i   1 − \beta    =
By the uniqueness property of MGFs
This is a Chi Squared distribution with parameter 2n\alpha 



[Total 15]
% Many candidates were able to score well on this question, although again marks were dropped when not showing all the steps in parts (iii) & (v).
\end{document}
