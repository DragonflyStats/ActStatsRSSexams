\documentclass[a4paper,12pt]{article}
%%%%%%%%%%%%%%%%%%%%%%%%%%%%%%%%%%%%%%%%%%%%%%%%%%%%%%%%%%%%%%%%%%%%%%%%%%%%%%%%%%%%%%%%%%%%%%%%%%%%%%%%%%%%%%%%%%%%%%%%%%%%%%%%%%%%%%%%%%%%%%%%%%%%%%%%%%%%%%%%%%%%%%%%%%%%%%%%%%%%%%%%%%%%%%%%%%%%%%%%%%%%%%%%%%%%%%%%%%%%%%%%%%%%%%%%%%%%%%%%%%%%%%%%%%%%
\usepackage{eurosym}
\usepackage{vmargin}
\usepackage{amsmath}
\usepackage{graphics}
\usepackage{epsfig}
\usepackage{enumerate}
\usepackage{multicol}
\usepackage{subfigure}
\usepackage{fancyhdr}
\usepackage{listings}
\usepackage{framed}
\usepackage{graphicx}
\usepackage{amsmath}
\usepackage{chngpage}
%\usepackage{bigints}
\usepackage{vmargin}

% left top textwidth textheight headheight

% headsep footheight footskip

\setmargins{2.0cm}{2.5cm}{16 cm}{22cm}{0.5cm}{0cm}{1cm}{1cm}
\renewcommand{\baselinestretch}{1.3}
\setcounter{MaxMatrixCols}{10}
\begin{document}

\begin{enumerate}
3
[1]
[Total 8]
(i) State the fundamental difference between Bayesian estimation and Classical
estimation.[2]
(ii) State three different loss functions which may be used under Bayesian
estimation, indicating for each its link to the posterior distribution.
[3]
The proportion, θ , of the population of a particular country who use online banking is
being estimated. Of a sample of 500 people, 326 do use online banking.
An actuary is estimating θ using a suitable uniform distribution as a prior.
(iii)

CT6 S2018–2
(a) Determine the posterior distribution of θ .
(b) Calculate an estimate of θ using the loss function that minimises the
mean of the posterior distribution.
[4]
[Total 9]4
An insurance company has a portfolio of policies, where claim amounts follow a
Pareto distribution with parameters α = 3 and λ = 100. The insurance company has
entered into an excess of loss reinsurance agreement with a retention of M, such that
90% of claims are still paid in full by the insurer.
(i) Calculate M.[4]
(ii) Calculate the average claim amount paid by the reinsurer, on claims which
involve the reinsurer.
[6]
[Total 10]

%%%%%%%%%%%%%%%%%%%%%%%%%%%%%%%%%%
Q3
(i) Under Bayesian estimation the parameter being estimated is itself considered
to be a random variable.
[2]
(ii) Quadratic loss function – the mean of the posterior distribution
Absolute error function – the median of the posterior distribution
All-or-nothing loss function – the mode of the posterior distribution
A suitable prior is U(0,1), so f θ ( θ ) = 1
Distribution of X| θ is Bin(500, θ), so the likelihood function is:
(iii)
=
L ( θ )
C 326 θ 326 (1 − θ ) 174
500
[1]
[1]
[1]
[2]
Apply Bayes, PDF of posterior is proportional to θ (1 − θ )
[1]
Hence the distribution of θ|X is beta(327,175), and the estimate of θ under
quadratic loss is 327/(327+175) = 65.1%.
[1]
[4]
[Total 9]
326
174
Again many candidates were able to score well here. A few candidates tripped up by
assuming the posterior was binomial rather than beta.
Q4
3
(i)
 100 
P ( X < M ) =−
1 
0.9
 =
 100 + M 
 100

 100 + M
(ii)
[11⁄2]
1
1
100 − 100 * 0.1 3

3
=
0.1
⇒
M
=
= 115.4

1

0.1 3
[21⁄2]
Let Y be the claim amount paid by the reinsurer, so that
X <= M
X > M
E ( Z )
E ( Y | X > M ) =
P ( X > M )
 0
Y =
 X − M
∞
∞
E ( Z ) =
∫ ( x − M ) f ( x ) dx =
∫ ( x − M )
M
M
dv
3*100 3
u =
( x − M ); =
dx (100 + x ) 4
[1]
3*100 3
dx
(100 + x ) 4
[1]
[1]
Page 4Subject CT6 (Statistical Methods Core Technical) – September 2018 – Examiners’ Report
∞
∞

100 3 
100 3
E ( Z ) =  − ( x − M )
dx
 +
(100 + x ) 3  M M ∫ (100 + x ) 3

[1]
∞
 − 100 3 


100 3
=
0 + 
=
=
10.772

2 
2 
 2(100 + x )  M  2(100 + M ) 
10.772
E ( Y | X > M =
)
= 107.7
0.1
[1]
[1]
[Total 10]
Many candidates got full marks on part (i), but only the strongest candidates were able to
score well on part (ii).
