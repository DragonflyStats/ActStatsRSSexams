\documentclass[a4paper,12pt]{article}

%%%%%%%%%%%%%%%%%%%%%%%%%%%%%%%%%%%%%%%%%%%%%%%%%%%%%%%%%%%%%%%%%%%%%%%%%%%%%%%%%%%%%%%%%%%%%%%%%%%%%%%%%%%%%%%%%%%%%%%%%%%%%%%%%%%%%%%%%%%%%%%%%%%%%%%%%%%%%%%%%%%%%%%%%%%%%%%%%%%%%%%%%%%%%%%%%%%%%%%%%%%%%%%%%%%%%%%%%%%%%%%%%%%%%%%%%%%%%%%%%%%%%%%%%%%%

\usepackage{eurosym}
\usepackage{vmargin}
\usepackage{amsmath}
\usepackage{graphics}
\usepackage{epsfig}
\usepackage{enumerate}
\usepackage{multicol}
\usepackage{subfigure}
\usepackage{fancyhdr}
\usepackage{listings}
\usepackage{framed}
\usepackage{graphicx}
\usepackage{amsmath}
\usepackage{chngpage}
%\usepackage{bigints}
\usepackage{vmargin}

% left top textwidth textheight headheight

% headsep footheight footskip
\setmargins{2.0cm}{2.5cm}{16 cm}{22cm}{0.5cm}{0cm}{1cm}{1cm}
\renewcommand{\baselinestretch}{1.3}
\setcounter{MaxMatrixCols}{10}
\begin{document}

6
Tarik and Liam are playing a zero-sum two-person game. 
\begin{itemize}
    \item From a deck of three cards numbered 1, 2 and 3 Tarik selects a card, making sure Liam does not know which one it is.
      \item Liam then proceeds to guess which card Tarik has picked. If Liam is wrong, he
continues to guess until he has guessed correctly, at which point the game ends.
  \item After each guess, if Liam’s guess is lower than the number on Tarik’s card, Tarik 
says “Low”, but if Liam’s guess is higher than the number on Tarik’s card, Tarik says
“High”.
  \item At the end of each game, Liam pays Tarik \$1 for each guess he made.
\end{itemize}
You should assume that Liam will never make a guess that contradicts the information
provided by Tarik – for example, if Liam guesses “2” first, and Tarik says “Low”,
Liam would then always guess “3”, rather than “1”.
Consider strategy A, where Liam will guess “1” first, and then guesses “2” if 1 is not
correct.

\begin{enumerate}
\item (i) Set out the four other strategies in addition to A (labelled B to E) which Liam
could adopt.
\item 
(ii) Construct the payoff matrix to Tarik.
\item 
 Explain whether or not there is a saddle point.
\end{enumerate}

%%%%%%%%%%%%%%%%%%%%%%%%%%%%%%%%%%%%%%%%%%%%%%%%%%%%%%%%%%%%%%\newpage

Q6
(i)
Guess 1, then guess 2 if “Low”, then guess 3 if “Low” again. (A)
Guess 1, then guess 3 if “Low”, then guess 2 if “High” (B)
Guess 2, then guess 1 if “High”, or guess 3 if “Low” (C)
Guess 3, then guess 2 if “High”, then guess 1if “High” again (D)
Guess 3, then guess 1 if “High”, then guess 2 if “Low” (E)

(ii)
Tarik \ Liam
1
2
3
A
1
2
3
B
1
3
2
C
2
1
2
D
3
2
1
E
2
3
1
[3]
(iii)
There is no saddle point.

This is because there is no element in the matrix which is both the highest in
the row and lowest in the column, and vice versa.

Candidates with a good understanding of the relevant theory were able
to score very well here, although some candidates struggled to
formulate the strategies required in part (i).

\end{document}
