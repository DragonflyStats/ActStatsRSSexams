\documentclass[a4paper,12pt]{article}
%%%%%%%%%%%%%%%%%%%%%%%%%%%%%%%%%%%%%%%%%%%%%%%%%%%%%%%%%%%%%%%%%%%%%%%%%%%%%%%%%%%%%%%%%%%%%%%%%%%%%%%%%%%%%%%%%%%%%%%%%%%%%%%%%%%%%%%%%%%%%%%%%%%%%%%%%%%%%%%%%%%%%%%%%%%%%%%%%%%%%%%%%%%%%%%%%%%%%%%%%%%%%%%%%%%%%%%%%%%%%%%%%%%%%%%%%%%%%%%%%%%%%%%%%%%%
\usepackage{eurosym}
\usepackage{vmargin}
\usepackage{amsmath}
\usepackage{graphics}
\usepackage{epsfig}
\usepackage{enumerate}
\usepackage{multicol}
\usepackage{subfigure}
\usepackage{fancyhdr}
\usepackage{listings}
\usepackage{framed}
\usepackage{graphicx}
\usepackage{amsmath}
\usepackage{chngpage}
%\usepackage{bigints}
\usepackage{vmargin}

% left top textwidth textheight headheight

% headsep footheight footskip

\setmargins{2.0cm}{2.5cm}{16 cm}{22cm}{0.5cm}{0cm}{1cm}{1cm}
\renewcommand{\baselinestretch}{1.3}
\setcounter{MaxMatrixCols}{10}
\begin{document}

The table below shows the cumulative incurred claims by year for a portfolio of general insurance policies, with all figures in £m. Claims paid to date total 13.5. The ultimate loss ratio is expected to be in line with the 2013 accident year, and claims are
assumed to be fully developed by the end of Development Year 3.

Development Year
Accident Year 0 1 2 3 Earned Premiums
2013 3.01 3.38 3.85 4.00 4.32
2014 3.30 3.67 4.15 2015 3.32 3.86 2016 3.74
4.41
4.55
4.68
Calculate the total reserve required to meet the outstanding claims, using the
Bornheutter-Ferguson method.
%%%%%%%%%%%%%%
\newpage
Q4
\begin{itemize}
\item Ultimate loss ratio = 4/4.32 = 92.5926% 
\item DF3 = 4/3.85 = 1.03896 
\item DF2 = (3.85+4.15)/(3.38+3.67) = 1.134752 
\item DF1 = (3.38+3.67+3.86)/(3.01+3.3+3.32) = 1.132918 
\item Adjusted expected ultimate claim for
AY2 = 4.15+0.925926*4.41*(1 – 1/1.03896) = 4.3031 
\item Adjusted expected ultimate claim for AY3
= 3.86+0.925926*4.55*(1 – 1/(1.03896*1.134751)) = 4.4995 
\item Adjusted expected ultimate claim for AY4
= 3.74+0.925926*4.68*(1–1/(1.03896*1.134751*1.132918)) = 4.8290 \end{itemize}
So reserve = 4 + 4.3031 + 4.4995 + 4.8290 – 13.5 = 4.13m

Most candidates scored very well on this straightforward chain-ladder
question, although some candidates did not appear to know the method
required.
\end{document}
