\documentclass[a4paper,12pt]{article}

%%%%%%%%%%%%%%%%%%%%%%%%%%%%%%%%%%%%%%%%%%%%%%%%%%%%%%%%%%%%%%%%%%%%%%%%%%%%%%%%%%%%%%%%%%%%%%%%%%%%%%%%%%%%%%%%%%%%%%%%%%%%%%%%%%%%%%%%%%%%%%%%%%%%%%%%%%%%%%%%%%%%%%%%%%%%%%%%%%%%%%%%%%%%%%%%%%%%%%%%%%%%%%%%%%%%%%%%%%%%%%%%%%%%%%%%%%%%%%%%%%%%%%%%%%%%

\usepackage{eurosym}
\usepackage{vmargin}
\usepackage{amsmath}
\usepackage{graphics}
\usepackage{epsfig}
\usepackage{enumerate}
\usepackage{multicol}
\usepackage{subfigure}
\usepackage{fancyhdr}
\usepackage{listings}
\usepackage{framed}
\usepackage{graphicx}
\usepackage{amsmath}
\usepackage{chngpage}
%\usepackage{bigints}
\usepackage{vmargin}

% left top textwidth textheight headheight

% headsep footheight footskip
\setmargins{2.0cm}{2.5cm}{16 cm}{22cm}{0.5cm}{0cm}{1cm}{1cm}
\renewcommand{\baselinestretch}{1.3}
\setcounter{MaxMatrixCols}{10}
\begin{document} 

2
\begin{enumerate}
\item (i)
State the three main components of a generalised linear model.
[3]
Consider the discrete random variable Y, with the following probability density
function
⎛ n ⎞
⎟ \mu ny ( 1− \mu ) n−ny
f ( y, \mu ) = ⎜⎜
⎟
ny
⎝
⎠
(ii)
1 2
y = 0, , ,...,1
n n
\item Show that Y belongs to the exponential family of distributions, specifying each
component.
\item (iii)
State the canonical link function in this case.
\end{enumerate}
%%%%%%%%%%%%%%%%%%%%%%%%%%%%%%%%%%%%%%%%%%%%%%%%%%%%%%%%%%%%%%%%%%%%%%%%%%%
\newpage 
\noindent \textbf{Solutions}

Q2
(i)
(ii)
The distribution of the response variable.
A linear predictor as a function of the covariates.
A link function between the response variable and the linear predictor.

 n  
=
f ( y , \mu ) exp  n ( y ln \mu + (1 − y ) ln ( 1 − \mu ) ) + ln   
 ny  

 

 n  
 \mu 
= exp  n  y ln 
 + ln ( 1 − \mu )  + ln  ny  
  
 1 − \mu 
   

Hence
 \mu 
θ = ln 

 1 − \mu 
\phi  = n
1
a ( \phi  ) =
\phi 
(iii)
[1]
[1]

[1⁄2]
[1⁄2]
b =
( θ ) ln ( 1 + e θ )
 \phi  
c ( y , \phi  ) = ln  
 \phi  y 
 \mu 
Link function: g ( \mu ) = ln 

 1 − \mu  [1]
%%%%%%%%%%%%%%%%%%%%%%%%%%%%%%
Well prepared candidates were able to score highly here, although many candidates gave
insufficiently detailed answers in part (i), and didn’t give expressions in part (ii) as a function
of the correct variable e.g. leaving b(theta) as a function of mu.


\end{document}
