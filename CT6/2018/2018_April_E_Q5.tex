\documentclass[a4paper,12pt]{article}

%%%%%%%%%%%%%%%%%%%%%%%%%%%%%%%%%%%%%%%%%%%%%%%%%%%%%%%%%%%%%%%%%%%%%%%%%%%%%%%%%%%%%%%%%%%%%%%%%%%%%%%%%%%%%%%%%%%%%%%%%%%%%%%%%%%%%%%%%%%%%%%%%%%%%%%%%%%%%%%%%%%%%%%%%%%%%%%%%%%%%%%%%%%%%%%%%%%%%%%%%%%%%%%%%%%%%%%%%%%%%%%%%%%%%%%%%%%%%%%%%%%%%%%%%%%%

\usepackage{eurosym}
\usepackage{vmargin}
\usepackage{amsmath}
\usepackage{graphics}
\usepackage{epsfig}
\usepackage{enumerate}
\usepackage{multicol}
\usepackage{subfigure}
\usepackage{fancyhdr}
\usepackage{listings}
\usepackage{framed}
\usepackage{graphicx}
\usepackage{amsmath}
\usepackage{chngpage}
%\usepackage{bigints}
\usepackage{vmargin}

% left top textwidth textheight headheight

% headsep footheight footskip
\setmargins{2.0cm}{2.5cm}{16 cm}{22cm}{0.5cm}{0cm}{1cm}{1cm}
\renewcommand{\baselinestretch}{1.3}
\setcounter{MaxMatrixCols}{10}
\begin{document}
%%- Question 5
For three different risks, an actuary is modelling the monthly claim numbers with
three different Poisson distributions.
\begin{center}
\begin{tabular}{|c|c|c|}
Risk & Exposure & Number of claims \\ \hline
Risk 1 & 36 months & 20 \\ \hline
Risk 2 & 30 months & 18\\ \hline
Risk 3 & 24 months & 16\\ \hline
\end{tabular}
\end{center}
\begin{enumerate}[(a)]
\item (i) Derive the maximum likelihood estimates of the parameter for each of the
three individual Poisson distributions fitted.

\item (ii)
 Test the hypothesis that the three risks have the same monthly claim rate. 
\end{enumerate}

%%%%%%%%%%%%%%%%%%%%%%%%%%%%%%%%%%%%%%%%%%%%%%%%%%%%%%%%%%%%%%%%%%%%%%%%%%%%
\newpage
Q5
\begin{itemize}
\item (i)
For risk one, let x i , y i , z i be the numbers of claims in month i for risks one,
two & three respectively. 
\item Let $\mu_{I}$ , \mu_{II} , \mu_{III} be the monthly rate for these three
risks then the likelihood function is:
\[
log L ( \mu_{I} , \mu_{II} , \mu_{III} ) = log L ( \mu_{I} ) + log L ( \mu_{II} ) + log L ( \mu_{III} )\]
[1⁄2]
where
\[
log L ( \mu=
I )
36
= i 1 = i 1
30
30
36
20 log \mu_{I} − 36 \mu_{I} − \sum  log x i !
[1]
= i 1
30
i ! 18log \mu_{II} − 30 \mu_{II} − \sum  log y i !
\sum  y i log \mu_{II} − \sum  \mu_{II} − \sum  log y =
= i 1
log L ( \mu_{III} =
)
36
x i !
\sum  x i log \mu_{I} − \sum  \mu_{I} − \sum  log =
= i 1
30
log L ( \mu_{II} =
)
36
24
= i 1 = i 1
24
= i 1
24 24
= i 1 = i 1 = i 1
i ! 16 log \mu_{III} − 24 \mu_{III} − \sum  log z i !
\sum  z i log \mu_{III} − \sum  \mu_{III} − \sum  log z =
= i 1
\]
\item After differentiating and equating to zero we have
\[
\frac{\partial}{\partial}  log L ( \mu_{I} , \mu_{II} , \mu_{III} )
\frac{\partial}{\partial} \mu_{I}
=
− 0.36 +
20
\mu_{I}\]


Second derivative is − 20 / \mu_{I} 2 < 0
20 5
=
so \mu  I =
36 9
[1]
[1⁄2]
18 3  16 2
=
=
, \mu_{III} =
\item Similarly we can see that \mu 
II =
3 0 5
24 3
%%%%%%%%%%%%%%%%%%%%%%%%%%%%%%%%%%%%%
\item (ii)
For testing whether the three models are the same we carry out the likelihood
ratio test.
\item We fit the same rate to the three risks using this log likelihood function
36
30
24


\[\log L =
( \mu )   \sum  x i + \sum  y i + \sum  z i   log \mu − 90 \mu − \sum  log x i ! − \sum  log y i ! − \sum  log z i !
= i 1 = i 1 = i 1\]


[1]
ˆ
=
and similar to the above the corresponding MLE is \mu
54 27
=
90 45
[1⁄2]
2 ( log L ( \mu_{I} , \mu_{II} , \mu_{III} ) − log L ( \mu ) )
20
18
16
54


= 2  20 log − 20 + 18log − 18 + 16 log − 16 − 54 log + 54 
36
30
24
90



 20 
 18 
 16 
 54  
= 2 *  20 * log   + 18 * log   + 16 * log   − 54 * log   
 36 
 30 
 24 
 90  

= 0.2930949
[11⁄2]
\item The difference in the parameters between the models is 3 – 1 = 2, therefore we compare this test statistics against the $\chi^2_{2}$ which at the 5\% upper level has
critical value 5.991>. 0.2931 . 

\item Therefore there is insufficient evidence to reject the hypothesis that the three risks have a common rate.
\item
Variants of this question have been seen many times before and so candidates were able to score very well here, although only the
strongest candidates were able to pick up full marks.
\end{itemize}
\end{document}
