\documentclass[a4paper,12pt]{article}

%%%%%%%%%%%%%%%%%%%%%%%%%%%%%%%%%%%%%%%%%%%%%%%%%%%%%%%%%%%%%%%%%%%%%%%%%%%%%%%%%%%%%%%%%%%%%%%%%%%%%%%%%%%%%%%%%%%%%%%%%%%%%%%%%%%%%%%%%%%%%%%%%%%%%%%%%%%%%%%%%%%%%%%%%%%%%%%%%%%%%%%%%%%%%%%%%%%%%%%%%%%%%%%%%%%%%%%%%%%%%%%%%%%%%%%%%%%%%%%%%%%%%%%%%%%%

\usepackage{eurosym}
\usepackage{vmargin}
\usepackage{amsmath}
\usepackage{graphics}
\usepackage{epsfig}
\usepackage{enumerate}
\usepackage{multicol}
\usepackage{subfigure}
\usepackage{fancyhdr}
\usepackage{listings}
\usepackage{framed}
\usepackage{graphicx}
\usepackage{amsmath}
\usepackage{chngpage}

%\usepackage{bigints}
\usepackage{vmargin}

% left top textwidth textheight headheight

% headsep footheight footskip

\setmargins{2.0cm}{2.5cm}{16 cm}{22cm}{0.5cm}{0cm}{1cm}{1cm}

\renewcommand{\baselinestretch}{1.3}

\setcounter{MaxMatrixCols}{10}

\begin{document}

\begin{enumerate}

[Total 13]10
Consider the following probability density function:
h(x) = λe −λx x > 0
(i)
Set out an algorithm for sampling from h(x) using the inverse transform
method.
Now also consider the following density function, the “half-normal” distribution:
−x 2
2 2 \sigma 2
f (x) =
e
, x > 0
\sigma π
Let M be the maximum of
f (x)
h(x)
2
λ 2 \sigma 2
e 2
(ii) Show that M =
(iii) Set out an acceptance-rejection algorithm, using h(x), which generates samples
from the half normal distribution f(x).
(iv) Determine the value of l for which the algorithm is the most efficient (i.e. on
average require fewest samples from h(x) to generate samples from f(x)).
(v) Show that the samples from part (iii) can be used to generate samples from the
normal distribution with mean m and variance s 2 .
[Total 14]

λ\sigma π

END OF PAPER
CT6 A2018–7 
PLEASE TURN OVER
%%%%%%%%%%%%%%%%%%%%%%%%%%%%%%%%%%%%%%%%%%%%%%%%%%%%%%%%%%%%%%%%%%%%%%%%%%%%%%%%%%%%%%%%%%%%%%%%%%%
Q10
(i)
F ( x ) = 1 − e −λ x 
First, sample u from U ∼ U ( 0,1 ) 
1
− log ( 1 − U )
Then X =
λ 
− x 2
2 2 \sigma 2 +λ x
f
sup
sup
M
e
=
=
(ii)
h x > 0 λ\sigma π
the maximum being achieved at
and M =
2
λ\sigma π
x =\sigma 2 λ ,
λ 2 \sigma 2
e 2
g ( x )
therefore =


− x 2
(iii)

f ( x )
= e 2 \sigma
Mh ( x )
+λ x −
2
λ 2 \sigma 2
2

The rejection algorithm is then:
Sample x from f ( x ) as in (i) [Step 1] 
Sample u from U(0,1) [Step 2] 
− x 2
If u< e 2 \sigma
(iv)
+λ x −
2
λ 2 \sigma 2
2
then set y = x, otherwise go to step 1.
The optimal λ is the one minimizing M ,


this can be found by taking log M and differentiating
'

2 λ 2 \sigma 2 
1
log M =  − log λ + log
+
= − + λ\sigma 2


2  
λ
\sigma π

'
[11⁄2]
Page 11 %%%%%%%%%%%%%%%%%%%%%%%%%%%%%%%%%%%%%%%%%%%%%%% – April 2018 – Examiners’ Report
which becomes zero for λ =
1
\sigma

(v)
Y is a sample as in (iii)
Sample U from U ( 0,1 ) 
X = \mu + Y , if U > 0.5 
X = \mu − Y , if U <= 0.5

[Total 14]
Most candidates scored well in parts (i) to (iii), although a number
failed to use the cumulative distribution in part (i).
Only the best prepared candidates were able to score well in parts (iv)
and (v).
END OF EXAMINERS’ REPORT
Page 12
