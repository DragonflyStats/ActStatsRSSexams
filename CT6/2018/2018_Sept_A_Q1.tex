\documentclass[a4paper,12pt]{article}

%%%%%%%%%%%%%%%%%%%%%%%%%%%%%%%%%%%%%%%%%%%%%%%%%%%%%%%%%%%%%%%%%%%%%%%%%%%%%%%%%%%%%%%%%%%%%%%%%%%%%%%%%%%%%%%%%%%%%%%%%%%%%%%%%%%%%%%%%%%%%%%%%%%%%%%%%%%%%%%%%%%%%%%%%%%%%%%%%%%%%%%%%%%%%%%%%%%%%%%%%%%%%%%%%%%%%%%%%%%%%%%%%%%%%%%%%%%%%%%%%%%%%%%%%%%%

\usepackage{eurosym}
\usepackage{vmargin}
\usepackage{amsmath}
\usepackage{graphics}
\usepackage{epsfig}
\usepackage{enumerate}
\usepackage{multicol}
\usepackage{subfigure}
\usepackage{fancyhdr}
\usepackage{listings}
\usepackage{framed}
\usepackage{graphicx}
\usepackage{amsmath}
\usepackage{chngpage}
%\usepackage{bigints}
\usepackage{vmargin}

% left top textwidth textheight headheight

% headsep footheight footskip
\setmargins{2.0cm}{2.5cm}{16 cm}{22cm}{0.5cm}{0cm}{1cm}{1cm}
\renewcommand{\baselinestretch}{1.3}
\setcounter{MaxMatrixCols}{10}
\begin{document} 
%%© Institute and Faculty of Actuaries1
\begin{enumerate}
\item An actuary is simulating claim sizes, X, for a particular insurance policy. X follows
an exponential distribution with varying parameter λ, where λ can take one of three
possible values.
The table shows the distribution function of this external factor, and its impact on λ.
Probability
λ
0.3 0.3 0.4
1 2 3
Set out an algorithm to generate samples from X, using the inverse transform method.

%%%%%%%%%%%%%%%%%%%%%%%%%%%%%%%%%%%%%%%%%%%%%%%%%%%%%%%%%%%%%%%%%%%%%%%%%%%
\newpage 
\noindent \textbf{Solutions}
Q1
For a given λ, we know that one can sample from X as
1
X = − log U
U  U (0,1)
[2]
λ
Algorithm:
1. Generate U 1  U (0,1)
2.
 1

=
λ  2
 3

[1⁄2]
U 1 ∈ ( 0, 0.3 ]
U 1 ∈ ( 0.3, 0.6 ]
U 1 ∈ ( 0.6,1 ]
[1]
3. Generate U 2  U (0,1)
[1]
4. X = −
1
λ
log U 2
[1⁄2]
[Total 5]

%%This question was generally well answered, but a disappointing number of candidates used the probability density function, rather than the cumulative density function.


\end{document}
