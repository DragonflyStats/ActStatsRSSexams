\documentclass[a4paper,12pt]{article}
%%%%%%%%%%%%%%%%%%%%%%%%%%%%%%%%%%%%%%%%%%%%%%%%%%%%%%%%%%%%%%%%%%%%%%%%%%%%%%%%%%%%%%%%%%%%%%%%%%%%%%%%%%%%%%%%%%%%%%%%%%%%%%%%%%%%%%%%%%%%%%%%%%%%%%%%%%%%%%%%%%%%%%%%%%%%%%%%%%%%%%%%%%%%%%%%%%%%%%%%%%%%%%%%%%%%%%%%%%%%%%%%%%%%%%%%%%%%%%%%%%%%%%%%%%%%
\usepackage{eurosym}
\usepackage{vmargin}
\usepackage{amsmath}
\usepackage{graphics}
\usepackage{epsfig}
\usepackage{enumerate}
\usepackage{multicol}
\usepackage{subfigure}
\usepackage{fancyhdr}
\usepackage{listings}
\usepackage{framed}
\usepackage{graphicx}
\usepackage{amsmath}
\usepackage{chngpage}
%\usepackage{bigints}
\usepackage{vmargin}

% left top textwidth textheight headheight

% headsep footheight footskip

\setmargins{2.0cm}{2.5cm}{16 cm}{22cm}{0.5cm}{0cm}{1cm}{1cm}
\renewcommand{\baselinestretch}{1.3}
\setcounter{MaxMatrixCols}{10}
\begin{document}

\begin{enumerate}
3
[1]
[Total 8]
(i) State the fundamental difference between Bayesian estimation and Classical
estimation.[2]
(ii) State three different loss functions which may be used under Bayesian
estimation, indicating for each its link to the posterior distribution.
[3]
The proportion, \theta  , of the population of a particular country who use online banking is
being estimated. Of a sample of 500 people, 326 do use online banking.
An actuary is estimating \theta  using a suitable uniform distribution as a prior.
(iii)

CT6 S2018–2
(a) Determine the posterior distribution of \theta  .
(b) Calculate an estimate of \theta  using the loss function that minimises the
mean of the posterior distribution.
[4]

%%%%%%%%%%%%%%%%%%%%%%%%%%%%%%%%%%
\newpage 
Q3
(i) Under Bayesian estimation the parameter being estimated is itself considered
to be a random variable.
[2]
(ii) Quadratic loss function – the mean of the posterior distribution
Absolute error function – the median of the posterior distribution
All-or-nothing loss function – the mode of the posterior distribution
A suitable prior is U(0,1), so f \theta  ( \theta  ) = 1
Distribution of X| \theta  is Bin(500, \theta ), so the likelihood function is:
(iii)
=
L ( \theta  )
C 326 \theta  326 (1 − \theta  ) 174
500
%%%%%%%%%%%%%%%%%%%%%%%%%

Apply Bayes, PDF of posterior is proportional to \theta  (1 − \theta  )
[1]
Hence the distribution of \theta |X is beta(327,175), and the estimate of \theta  under
quadratic loss is 327/(327+175) = 65.1%.
[1]
[4]
[Total 9]
326
174
Again many candidates were able to score well here. A few candidates tripped up by
assuming the posterior was binomial rather than beta.

\end{document}
