\documentclass[a4paper,12pt]{article}

%%%%%%%%%%%%%%%%%%%%%%%%%%%%%%%%%%%%%%%%%%%%%%%%%%%%%%%%%%%%%%%%%%%%%%%%%%%%%%%%%%%%%%%%%%%%%%%%%%%%%%%%%%%%%%%%%%%%%%%%%%%%%%%%%%%%%%%%%%%%%%%%%%%%%%%%%%%%%%%%%%%%%%%%%%%%%%%%%%%%%%%%%%%%%%%%%%%%%%%%%%%%%%%%%%%%%%%%%%%%%%%%%%%%%%%%%%%%%%%%%%%%%%%%%%%%

\usepackage{eurosym}
\usepackage{vmargin}
\usepackage{amsmath}
\usepackage{graphics}
\usepackage{epsfig}
\usepackage{enumerate}
\usepackage{multicol}
\usepackage{subfigure}
\usepackage{fancyhdr}
\usepackage{listings}
\usepackage{framed}
\usepackage{graphicx}
\usepackage{amsmath}
\usepackage{chngpage}
%\usepackage{bigints}
\usepackage{vmargin}

% left top textwidth textheight headheight

% headsep footheight footskip
\setmargins{2.0cm}{2.5cm}{16 cm}{22cm}{0.5cm}{0cm}{1cm}{1cm}
\renewcommand{\baselinestretch}{1.3}
\setcounter{MaxMatrixCols}{10}
\begin{document} 

%%[Total 12]
%%- Question 88
Claim events on a portfolio of insurance policies follow a Poisson process with
parameter l. Individual claim amounts, X, follow a Normal distribution with
parameters m = 500 and s 2 = 200.
The insurance company calculates premiums using a premium loading factor of 20%.
\begin{enumerate}
    \item (i)
Show that the adjustment coefficient, r = 0.000708 to three significant figures.

The insurance company’s initial surplus is 5,000.
    \item (ii)
Calculate an upper bound on the probability of ruin, using Lundberg’s
inequality.
The insurance company actuary believes that claim amounts are better modelled using
an exponential distribution. You may assume that the mean m is unchanged, and is
now equal to the standard deviation.
    \item (iii) Calculate the new upper bound of the probability of ruin.
    \item (iv) Give a reason why claim amounts on insurance policies are not usually
modelled using a Normal distribution, and suggest an alternative distribution,
other than the exponential.
\end{enumerate}


[Total 12]

CT6 A2018–5 
\newpage
\begin{itemize}
%%-- Q8
(i)
r is the unique positive root of the equation:
\[\lambda + cr = \lambda M X ( r )\]

c 1.2 \lambda E ( X ) , so the equation simplifies to
Here =
1 + 1.2 E ( X ) r =
M X ( r )

Page 8 %%%%%%%%%%%%%%%%%%%%%%%%%%%%%%%%%%%%%%%%%%%%%%% – April 2018 – Examiners’ Report
M X ( r ) =
1
\mu r + \sigma 2 r 2
e 2
(from tables)

E ( X ) = 500 (from question)
So 1 + 600 r
1
500 r + 200 r 2
2
− e

= 0

\item At r = 0.000 7075, LHS is +0.000 03, and at r = 0.000 708 5, LHS is
–0.000 08

\item So the adjustment coefficient must be 0.000 708 to 3sf
− 0.000 708*5000
=
By Lundberg’s inequality, upper bound given
by e − RU e =
0.029
(ii)
(iii)
\lambda = 0.002
MGF is
(iv)


\lambda
\lambda − r

0.002
1 + 600 r =
0.002 − r 
(1 + 600 r )(0.002 −=
r ) 0.002 ⇒ 1.2 r − r − 600 =
r 2 0 
r (0.2 − 600 r ) = 0 ⇒ r = 1 
3000
e − RU = 0.189 
\item Normal distributions allow the possibility of negative claim amounts  Normal distributions do not have “fat tails”, commonly observed in insurance
claims

Normal distributions are not positively skewed, unlike typical claim amounts

[Max 1]
Any sensible alternative (gamma, Pareto, Weibull etc.)

%%%%%%%%%%%%%%%%%%%%%%%%%%%%%%%%%%%%%%%

% Most candidates are now familiar with the method required in part (i), and were able to score well throughout this question.
\end{itemize}
\end{document}
