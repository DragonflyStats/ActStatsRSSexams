CT6 A2016–6
10 (i) State the general expression of the exponential families of distributions and
use this to derive the relevant expressions for the mean and the variance of
these distributions. [6]
(ii) Extend the result in (i) to obtain an expression for the third central moment.
[4]
(iii) Show that the following density function belongs to the exponential family
of distributions: [4]
f(x) = 1 .
( )
x x e

  


  
(iv) Using the results in (i) and (ii) obtain the second and third central moments
for this distribution. [4]
[Total 18]
END OF PAPER

%%%%%%%%%%%%%%%%%%%%%%%%%%%%%%%%%%%%%%%%%%%%%%%%%
Q10 (i) The general form for the
f(x) =
  exp ( ) ( , )
( )
x b c x
a
 θ− θ   + ϕ   ϕ 
[1]
where a, b, c, are functions and θ and ϕ are called the natural and scale
parameters respectively.
Since
1 =  f (x)dx = exp ( )( ) ( , )
x b c x dx a
 θ − θ + ϕ   ϕ  
then differentiating with respect to θ we have that
∂
exp ( ) ( , )
( )
x b c x dx
a
 θ − θ   + ϕ   ϕ 
∂θ

= a(1 ) (x b ( )) f (x)dx
− ′ θ ϕ  = 0. [2]
Hence E(X) =  xf (x)dx = b′(θ) f (x)dx = b′(θ). [1]
Similarly differentiating again w.r.t. θ both sides of (x −b′(θ)) f (x)dx = 0 we
have
a(1 ) (x b ( ))2 b ( ) f (x)dx
 − ′ θ − ′′ θ   ϕ   = 0 [1]
hence
(x −b′(θ))2 f (x)dx = Var(X) = a(ϕ) b′′(θ) f (x)dx = a(ϕ)b′′(θ). [1]
[Total 6]
Subject CT6 (Statistical Methods Core Technical) – April 2016 – Examiners’ Report
Page 11
(ii) Differentiating further w.r.t. θ both sides of this identity
a(1 ) (x b ( ))2 b ( ) f (x)dx
 − ′ θ − ′′ θ   ϕ   = 0
we have
a(1 ) a(1 ) (x b ( ))2 b ( ) (x b ( )) f (x)dx
 − ′ θ − ′′ θ  − ′ θ ϕ  ϕ  
+ a(2 ) (x b ( )) b ( ) b ( ) f (x)dx
 − ′ θ ′′ θ − ′′′ θ   ϕ   = 0 [1½]
Since
(x −b′(θ)) f (x)dx = 0 = b′′(θ)(x −b′(θ)) f (x)dx
3
2
1 ( ()) ()
( )
x b fxdx
a
− ′ θ
ϕ  = 0 + b′′′(θ) f (x)dx
= 0 + b′′′(θ) [1½]
Therefore
E(X − E(X))3 = (x − b′(θ))3 f (x)dx = a(ϕ)2 b′′′(θ) [1]
[Total 4]
(iii) f(x) =
  exp ( ) ( , )
( )
x b c x
a
 θ− θ   + ϕ   ϕ 
= exp x log ( 1) log x log log ( )
  
 − − μα + α − + α α − Γ α   μ  
[1]
Hence
θ =
  − 1
μ
[½]
ϕ = α [½]
a(ϕ) =
  1
ϕ
= 1
α
[½]
b(θ) = − log (− θ) [1]
Subject CT6 (Statistical Methods Core Technical) – April 2016 – Examiners’ Report
Page 12
c(x, ϕ) = (ϕ − 1) log x + ϕ log ϕ − logΓ(ϕ) [½]
[Total 4]
(iv) b(θ) = − log (− θ) hence b′(θ) = − 1θ = μ and b′′(θ) = 2
1
θ
= μ2 [1½]
E(X − E(X))2 = a(ϕ)b′′(θ) =
  μ2
α. [1]
Similarly ( ) ( ( )) ( ) ( )
3 3 3 2
3 2
b′′′ θ = −2 = 2μ , hence E X − E X = a φ b′′′ θ = 2μ
θ α
. [1½]
[Total 4]
[TOTAL 18]
The hardest and most challenging question on the paper. Most candidates
were able to score well on part (ii), but only the best prepared candidates
scored well on the whole question. Full credit was given for alternative
solutions, including the use of MGFs and CGFs.
END OF EXAMINERS’ REPORT
