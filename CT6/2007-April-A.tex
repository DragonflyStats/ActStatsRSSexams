\documentclass[a4paper,12pt]{article}

%%%%%%%%%%%%%%%%%%%%%%%%%%%%%%%%%%%%%%%%%%%%%%%%%%%%%%%%%%%%%%%%%%%%%%%%%%%%%%%%%%%%%%%%%%%%%%%%%%%%%%%%%%%%%%%%%%%%%%%%%%%%%%%%%%%%%%%%%%%%%%%%%%%%%%%%%%%%%%%%%%%%%%%%%%%%%%%%%%%%%%%%%%%%%%%%%%%%%%%%%%%%%%%%%%%%%%%%%%%%%%%%%%%%%%%%%%%%%%%%%%%%%%%%%%%%

\usepackage{eurosym}
\usepackage{vmargin}
\usepackage{amsmath}
\usepackage{graphics}
\usepackage{epsfig}
\usepackage{enumerate}
\usepackage{multicol}
\usepackage{subfigure}
\usepackage{fancyhdr}
\usepackage{listings}
\usepackage{framed}
\usepackage{graphicx}
\usepackage{amsmath}
\usepackage{chngpage}

%\usepackage{bigints}
\usepackage{vmargin}

% left top textwidth textheight headheight

% headsep footheight footskip

\setmargins{2.0cm}{2.5cm}{16 cm}{22cm}{0.5cm}{0cm}{1cm}{1cm}

\renewcommand{\baselinestretch}{1.3}

\setcounter{MaxMatrixCols}{10}

\begin{document}

\begin{enumerate}
%%%%%%%%%%%%%%%%%%%%%%%%%%%%%%%%%%%%%%%%%%
 Institute of Actuaries1
2
3
(i) State two conditions for a risk to be insurable.
[2]
(ii) Describe briefly three distinct examples of financial loss insurance policies. [3]
[Total 5]
(i) Explain the concept of cointegrated time series.
(ii) Give two examples of circumstances when it is reasonable to expect that two
processes may be cointegrated.
[2]
[Total 5]
Page 2Subject CT6 (Statistical Methods Core Technical) — April 2007 — Examiners’ Report
1
(i)
The policyholder must have an interest in the risk being insured, to distinguish
between insurance and a wager.
The risk must be of a financial and reasonably quantifiable nature.
(ii)
Pecuniary loss — protects against bad debts or other failure of a third party.
Fidelity guarantee — protects against losses caused by dishonest actions of
employees.
Business interruption cover — protects against losses made as a result of not
being able to conduct business.
2
(i)
Two time series X, Y are cointegrated if X and Y are I(1) random processes
and there exists a non-zero vector (α, β) such that αX + βY is stationary.
I(1) means that ∇X and ∇Y are stationary
(α, β) is called a cointegrating vector
(ii)
Examples:
One of the processes is driving the other
Both are being driven by the same underlying process
%%%%%%%%%%%%%%%%%%%%%%%%%%%%%%%%%%%%%%%%%%%%%%%%%%%%%%%%%%%%%%%%%%%%%%%%%%%%
\end{document}
