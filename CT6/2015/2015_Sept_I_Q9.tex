\documentclass[a4paper,12pt]{article}

%%%%%%%%%%%%%%%%%%%%%%%%%%%%%%%%%%%%%%%%%%%%%%%%%%%%%%%%%%%%%%%%%%%%%%%%%%%%%%%%%%%%%%%%%%%%%%%%%%%%%%%%%%%%%%%%%%%%%%%%%%%%%%%%%%%%%%%%%%%%%%%%%%%%%%%%%%%%%%%%%%%%%%%%%%%%%%%%%%%%%%%%%%%%%%%%%%%%%%%%%%%%%%%%%%%%%%%%%%%%%%%%%%%%%%%%%%%%%%%%%%%%%%%%%%%%

\usepackage{eurosym}
\usepackage{vmargin}
\usepackage{amsmath}
\usepackage{graphics}
\usepackage{epsfig}
\usepackage{enumerate}
\usepackage{multicol}
\usepackage{subfigure}
\usepackage{fancyhdr}
\usepackage{listings}
\usepackage{framed}
\usepackage{graphicx}
\usepackage{amsmath}
\usepackage{chngpage}
%\usepackage{bigints}
\usepackage{vmargin}

% left top textwidth textheight headheight

% headsep footheight footskip
\setmargins{2.0cm}{2.5cm}{16 cm}{22cm}{0.5cm}{0cm}{1cm}{1cm}
\renewcommand{\baselinestretch}{1.3}
\setcounter{MaxMatrixCols}{10}
\begin{document}
9 A random variable X follows a gamma distribution with parameters  and .
(i) Derive the moment generating function (MGF) of X. [3]
(ii) Derive the coefficient of skewness of X. [8]
[Total 11]
%%%%%%%%%%%%%%%%%%%%%%%%%%%%%%%%%%%%%
10 Claims on a portfolio of insurance policies arrive as a Poisson process with  rate . Individual claims are uniformly distributed between 0 and 50, and the
insurance company uses a premium loading of 12%.
(i) Show that the insurance company’s adjustment coefficient is 0.0066 to four
decimal places. [3]
The insurance company has entered into an excess of loss insurance agreement with a retention amount of M and with a reinsurer who uses a premium loading of 15%.
(ii) (a) Determine the mean amount per claim paid by the reinsurer as a
function of M.
(b) Determine the minimum retention level Mmin for the insurance
company, assuming that expected net premium income needs to be greater than expected net claims.
[6]
The insurance company manages to negotiate a lower reinsurance premium loading.
(iii) Explain what happens to the minimum retention level Mmin, without doing
any further calculations. [2]
[Total 11]
CT6 S2015–6
\newpage
%%%%%%%%%%%%%%%%%%%%%%%%%%%%%%%%%%%%%%%%%%%%%%%%%%%%%%%%%%%%
  Q9 (i) M(t) = 1
0
[ ] = * 1
( )
E etx etx x e xdx

  
  
= 1 ( )
0
1
( )
x e t xdx

   
  
=
          1
0
1
Γ
t x e t xdx
t
 
   


 
   
The integral is PDF of  (,  - t)  = 1
 M(t) =
  t
   
     
= 1 t ,t .
          
(ii) Coefficient =  
3
3
[( ) ]
Var( )
E x
x
  , given E[x]  
E[(x )3] = E[x3  3x2 + 3x2  3]
= E[x3]  3 E[x2] + 32 E[x]  3
= E[x3]  3 E[x2] + 23
= E[x3]  3 *
  

* E[x2] + 2(E[x])3
M(t) = 1 t         
from (i)
M(t) =  *  1


 

 
1
*
  1
1 t         
=
  1
* 1 t           
M(t) =
  

* [( + 1)] *  1


 

 
1
*
  2
1 t         
= 2
*( 1)

*
  2
1 t         
%%%%%%%%%%%%%%%%%%%%%%%%%%%%
Page 10
M(t) = 2
*( 1)

* [( +2)] *
   1 1     
*
  3
1 t         
= 3
*( 1)*(  2)

*
  3
1 t         
E[x3] = M(0) = 3
*( 1)*(  2)

E[x2] = M''(0) = 2
( 1)

So
E[(x  )3] = *( 1) *(  2)
3
 3

* ( 1)
2
 2* 


 

 
3
=  3 2 3 2 3 
3 3
1 3 2 3 3 2 2

           
 
Coefficient =
  2
3

2

 

 
3 = 2

3
2
= 2

Most candidates were able to score well on part (i), but many candidates
struggled with part (ii), particularly those who had forgotten or were unfamiliar
with CT3 concepts. Many stronger candidates made use of the cumulant
generating function in part (ii), which simplified the algebra considerably, and
were awarded full credit.
Q10 (i)   cR   MX R
c  1 EX  1.12 *25  28 
  50  50 
0
1
0.02
50
R
Rx
X
e
M R e dx
R

  
Subject CT6 (Statistical Methods Core Technical) – September 2015 – Examiners’ Report
Page 11
So (dividing by lambda)
 50 1
1 28 0
50
e R
R
R

  
At R = 0.00665, fn = -0.000 115
At R = 0.00655 fn = 0.000 209
(ii) (a)    
50 2
0.02 0.02 50 25 0.01 2
2 M
M
E Z   x M dx  x Mx  M  M
(b) cnet  1 EX  1.15EZ
 28 1.1525M 0.01M2 
 0.751.15M 0.0115M2 
net claims =      
 
2
2
25 25 0.01
0.01
E X E Z M M
M M
     
  
Need income > claims so
 0.75 + 1.15M  0.0115M2 > M – 0.01M2
 0.0015M2 + 0.15M – 0.75 > 0
M > 5.279
(iii) It decreases Mmin since reinsurance is less of a drag.
Again most candidates were able to score well on part (i), but only stronger
candidates were able to apply reinsurance theory to score well on parts (ii)
and (iii).
%- Subject CT6 (Statistical Methods Core Technical) – September 2015 – Examiners’ Report
\end{document}
