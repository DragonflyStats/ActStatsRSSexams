\documentclass[a4paper,12pt]{article}

%%%%%%%%%%%%%%%%%%%%%%%%%%%%%%%%%%%%%%%%%%%%%%%%%%%%%%%%%%%%%%%%%%%%%%%%%%%%%%%%%%%%%%%%%%%%%%%%%%%%%%%%%%%%%%%%%%%%%%%%%%%%%%%%%%%%%%%%%%%%%%%%%%%%%%%%%%%%%%%%%%%%%%%%%%%%%%%%%%%%%%%%%%%%%%%%%%%%%%%%%%%%%%%%%%%%%%%%%%%%%%%%%%%%%%%%%%%%%%%%%%%%%%%%%%%%

\usepackage{eurosym}
\usepackage{vmargin}
\usepackage{amsmath}
\usepackage{graphics}
\usepackage{epsfig}
\usepackage{enumerate}
\usepackage{multicol}
\usepackage{subfigure}
\usepackage{fancyhdr}
\usepackage{listings}
\usepackage{framed}
\usepackage{graphicx}
\usepackage{amsmath}
\usepackage{chngpage}

%\usepackage{bigints}
\usepackage{vmargin}

% left top textwidth textheight headheight

% headsep footheight footskip

\setmargins{2.0cm}{2.5cm}{16 cm}{22cm}{0.5cm}{0cm}{1cm}{1cm}

\renewcommand{\baselinestretch}{1.3}

\setcounter{MaxMatrixCols}{10}

\begin{document}

\begin{enumerate}
3
(i)
(a)
To protect itself from the risk of large claims.
(b)
(ii)
 Excess of loss reinsurance where the reinsurer pays any amount of a claim above the retention.
 Proportional reinsurance where the reinsurer pays a fixed proportion of any claim.
We must first find the parameters  and  of the Pareto distribution.
 2

= 270 and
= 340 2
2
  1
(   1) (   2)

 2

= 340 2
2
  2 (   1)
so
340 2

=
= 1.585733882
 2
270 2
so  =
2  1.585733882
1.585733882  1
= 5.4145
and  = 270 × 4.4145 = 1191.920375
%%%%%%%%%%%%%%%%%%%%%%%%%%%%%%%%%%%%%%%%%%%%%%%%%%%%%%%%%%%%%%%%%%%%%%%%%%%%%%%%%%%%%%%%%%%%%%%5
We need to find M such that P(X > M) = 0.05

i.e.
  

 = 0.05
  M 

 M

= 1

0.05
= 1
0.05  (   M )
1 

  1  0.05  



M = 
1
0.05 
1 

5.4145

1191.920375   1  0.05




=
1
0.05 5.4145
= 880.8
%This question was the best answered on the paper with most candidates scoring well. Some candidates were unable to manipulate the Pareto distribution.
\newpage
%%%%%%%%%%%%%%%%%%%%%%%%%%%%%%%%%%%%%%%%%%%%%%%%%%%%%%%%%%%%%%%%%%%%%%%%%%%%%%%%%%%%%%%%%%5
%%%%%%%%%%%%%%%%%%%%%%%

\end{document}
