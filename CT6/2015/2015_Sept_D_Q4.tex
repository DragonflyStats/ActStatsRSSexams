\documentclass[a4paper,12pt]{article}

%%%%%%%%%%%%%%%%%%%%%%%%%%%%%%%%%%%%%%%%%%%%%%%%%%%%%%%%%%%%%%%%%%%%%%%%%%%%%%%%%%%%%%%%%%%%%%%%%%%%%%%%%%%%%%%%%%%%%%%%%%%%%%%%%%%%%%%%%%%%%%%%%%%%%%%%%%%%%%%%%%%%%%%%%%%%%%%%%%%%%%%%%%%%%%%%%%%%%%%%%%%%%%%%%%%%%%%%%%%%%%%%%%%%%%%%%%%%%%%%%%%%%%%%%%%%

\usepackage{eurosym}
\usepackage{vmargin}
\usepackage{amsmath}
\usepackage{graphics}
\usepackage{epsfig}
\usepackage{enumerate}
\usepackage{multicol}
\usepackage{subfigure}
\usepackage{fancyhdr}
\usepackage{listings}
\usepackage{framed}
\usepackage{graphicx}
\usepackage{amsmath}
\usepackage{chngpage}

%\usepackage{bigints}

\usepackage{vmargin}

% left top textwidth textheight headheight
% headsep footheight footskip
\setmargins{2.0cm}{2.5cm}{16 cm}{22cm}{0.5cm}{0cm}{1cm}{1cm}
\renewcommand{\baselinestretch}{1.3}

\setcounter{MaxMatrixCols}{10}

\begin{document}

\begin{enumerate}


%%%%%%%%%%%%%%%%%%%%%%%%%%%%%%%%%%%%%%%%%%%%%%%%%%%%%%%%%%%%%%%%%%%

4 A small island is holding a vote on independence. Two recent survey results are
shown below:
  Poll Sample size Support for
independence
A 10 5
B 20 11
You should assume that the samples are independent.
A politician is using a suitable uniform distribution as the prior distribution in order
to estimate the proportion  in favour of independence.
(i) Calculate an estimate of  under the quadratic loss function. [3]
A rival politician decides to use instead a beta distribution as the prior, with
parameters  and , where  = .
(ii) Determine the new estimate of  under the “all-or-nothing” loss function in
terms of . [4]
[Total 7]
%%%%%%%%%%%%%%%%%%%%%%%%%%%%%%%%%%%%%%

Q4 (i) A suitable distribution is U(0,1) as theta must lie between 0 & 1.
f() = 1 for 0    1
Binomial distribution so likelihood function
L() = (30 C 16) 16 (1  )14
Bayes theorem: PDF (posterior) = PDF (prior) * likelihood
PDF (Posterior) proportional to 16 (1 – )14
So distribution of theta|sample is Beta (17,15)
Under quadratic loss estimate of theta is mean so
17/(17 + 15) = 0.53125
(ii) prior PDF proportional to 1 * (1 – )1
so using Bayes again posterior proportional to 15+ * (1 – )13+
Under all-or-nothing we need the mode of the posterior
Subject CT6 (Statistical Methods Core Technical) – September 2015 – Examiners’ Report
Page 5
Take logs (15 + ) log + (13 + ) log (1  )
differentiate (15 + ) /   (13 + ) / (1  ) = 0
so (15 + ) (1 – ) = (13 + )  so  (2 + 28) = 15 + 
so  = (15 + ) / (28 + 2)
since f(0) = f(1) = 0 this must be a maximum
Well prepared candidates had little difficulty with this straightforward question
on Bayes’ theory. Most candidates were able to pick up at least a few marks
by showing knowledge of the basic theory.
\end{document}
