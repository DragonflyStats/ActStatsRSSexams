\documentclass[a4paper,12pt]{article}

%%%%%%%%%%%%%%%%%%%%%%%%%%%%%%%%%%%%%%%%%%%%%%%%%%%%%%%%%%%%%%%%%%%%%%%%%%%%%%%%%%%%%%%%%%%%%%%%%%%%%%%%%%%%%%%%%%%%%%%%%%%%%%%%%%%%%%%%%%%%%%%%%%%%%%%%%%%%%%%%%%%%%%%%%%%%%%%%%%%%%%%%%%%%%%%%%%%%%%%%%%%%%%%%%%%%%%%%%%%%%%%%%%%%%%%%%%%%%%%%%%%%%%%%%%%%

\usepackage{eurosym}
\usepackage{vmargin}
\usepackage{amsmath}
\usepackage{graphics}
\usepackage{epsfig}
\usepackage{enumerate}
\usepackage{multicol}
\usepackage{subfigure}
\usepackage{fancyhdr}
\usepackage{listings}
\usepackage{framed}
\usepackage{graphicx}
\usepackage{amsmath}
\usepackage{chngpage}
%\usepackage{bigints}
\usepackage{vmargin}

% left top textwidth textheight headheight

% headsep footheight footskip
\setmargins{2.0cm}{2.5cm}{16 cm}{22cm}{0.5cm}{0cm}{1cm}{1cm}
\renewcommand{\baselinestretch}{1.3}
\setcounter{MaxMatrixCols}{10}
\begin{document}
11 Consider the following pair of equations:
  Xt = 0.5Xt1 + Yt + 1t
  
  Yt = 0.5Yt1 + Xt + 2
  t
  where 1t
   and 2
  t are independent white noise processes.
  (i) (a) Show that these equations can be represented as
  M
  Xt
  Yt
  
  
  
  
  
  
   N
  Xt1
  Yt1
  
  
  
  
  
  
  
  t
  1
  t
  2
  
  
  
  
  
  
  where M and N are matrices to be determined.
  (b) Determine the values of  for which these equations represent a
  stationary bivariate time series model. [9]
  (ii) Show that the following set of equations represents a VAR(p) (vector auto
                                                                     regressive) process, by specifying the order and the relevant parameters:
    Xt = Xt1 + Yt1 + 1t
  
  Yt = Xt1  Xt2 + 2
  t
  [3]
  [Total 12]
  12 An actuary needs to sample from a particular claim size distribution with the
  following density function:
    f(x) = kex 2, 0  x  2
  kex , x  2
  
   
   
  (i) Calculate the value of ��. ��2��
  (ii) Set out an algorithm for sampling from f(x) using the inverse transform
  method. [5]
  (iii) Set out an algorithm for sampling from f(x) using the acceptance – rejection
  method, by using samples from the exponential distribution with mean 1. [6]
  [Total 13]
  END OF PAPER
  
  %%%%%%%%%%%%%%%%%%%%%%%%%%%%%%%%%%%%%%%%%%%%%%%%%%%%%%%%%%%%%%%%%%%%%%
    Page 12
  Q11 (i) (a) It follows that
  1
  1
  1 2
  1 0.5 0
  1 0 0.5
  t t t
  t t t
  X X
  Y Y
  
  
          
                      
  (b) Multiplying both sides by
  1
  2
  1 1 1
  1 (1 ) 1
      
           
  we then have
     
  1
  1
  2 1 2 2
  1 1 1 1
  2 1 1 1 1
  t t t
  t t t
  X X
  Y Y
  
  
         
                       
  
  
  .
  Which is a stationary VAR(1) model if the eigenvalues of
  A1 = 2
  1 1
  2(1 ) 1
   
       
  are those  such that
  1
  det 0
  1
      
        
  or 1,2 = 1  
  then the eigenvalues of A1 are less than one in absolute value if
  2
  1 1
  2(1 )
   
  
   
  i.e.
  1 1
  2(1 )
  
   
  and
  1 1
  2(1 )
  
   
  which implies that 1
  2
    or 3
  2
   
  Subject CT6 (Statistical Methods Core Technical) – September 2015 – Examiners’ Report
  Page 13
  (ii) Here we have a VAR(2) where
  A1 =
    0
   
      
  A2 =
    0 0
  0
   
     
  since
  t
  t
  X
  Y
   
   
   
  =
    1
  1 2
  1 2 2
  0 0
  0 0
  t t t
  t t t
  X X
  Y Y
   
   
         
                      
  
  
  .
  This was by far the poorest answered question on the paper, with many
  candidates scoring few marks. Only the strongest candidates were able to
  derive the required eigenvalues and score well.
  Q12 (i) Here k1 = 2
  2
  0 2
  x
  e dx e xdx
        = ��2 ��������
  ������
  ��
  ��
  = 2(1  e1) + e2  0
  = 1.399576
  So k = 1.3995761 = 0.714502
  (ii) The distribution function here is
       
  2 2
  0
  1 1 2
  2
  2 1 , 0 2
  2 1 21 , 2
  x u x
  x
  u x
  k e du k e x
  k e e du k e e e x
   
      
    
        
        
    
                 
  
  
  At �� = 2, (2) = 2 ∗ �� ∗ (1 – exp(−1)) = 0.9033029, therefore the inversion
  function is
  X = F1(U) =
    1 2
  2ln 1 , 0 0.9033029
  2
  ln 2(1 ) , 0.9033029
  U U
  k
  e e U U
  k
   
             
  
              
  The inversion algorithm is then:
    Sample U from U(0,1)
  Take X = F1(U) as above.
  Subject CT6 (Statistical Methods Core Technical) – September 2015 – Examiners’ Report
  Page 14
  (iii) M =
    0
  max ( ) x x
  f x
   e
  = 2 1
  0
  max ( ) , 0 2
  , 2
  x
  x x
  f x ke x ke
  e k x
  
   
  
     
   
  The rejection function is then
  h(x) = ( )
  x
  f x
  Me =
    1 2
  1
  , 0 2
  , 2
  x
  e e x
  e x
  
  
  
    
    
  The algorithm is then:
    1  Simulate U1 from U(0,1), so that Y = logU1 is Exp(1)
  2  Simulate U2 from U(0,1)
  If U2 < h(Y) take X = Y otherwise start again.
  Most candidates were able to score well in part (i), and also at least partially
  in part (iii). Only the better prepared candidates were able to apply the
  inversion theory to part (ii), and only the strongest candidates correctly
  included the term 2(1 – e–1) for the second half of the distribution.
\end{document}
