\documentclass[a4paper,12pt]{article}

%%%%%%%%%%%%%%%%%%%%%%%%%%%%%%%%%%%%%%%%%%%%%%%%%%%%%%%%%%%%%%%%%%%%%%%%%%%%%%%%%%%%%%%%%%%%%%%%%%%%%%%%%%%%%%%%%%%%%%%%%%%%%%%%%%%%%%%%%%%%%%%%%%%%%%%%%%%%%%%%%%%%%%%%%%%%%%%%%%%%%%%%%%%%%%%%%%%%%%%%%%%%%%%%%%%%%%%%%%%%%%%%%%%%%%%%%%%%%%%%%%%%%%%%%%%%

\usepackage{eurosym}
\usepackage{vmargin}
\usepackage{amsmath}
\usepackage{graphics}
\usepackage{epsfig}
\usepackage{enumerate}
\usepackage{multicol}
\usepackage{subfigure}
\usepackage{fancyhdr}
\usepackage{listings}
\usepackage{framed}
\usepackage{graphicx}
\usepackage{amsmath}
\usepackage{chngpage}

%\usepackage{bigints}
\usepackage{vmargin}

% left top textwidth textheight headheight

% headsep footheight footskip

\setmargins{2.0cm}{2.5cm}{16 cm}{22cm}{0.5cm}{0cm}{1cm}{1cm}

\renewcommand{\baselinestretch}{1.3}

\setcounter{MaxMatrixCols}{10}

\begin{document}

\begin{enumerate}
%%%%%%%%%%%%%%%%%%%%%%%%%
\item 
% 1
Claim amounts on a portfolio of insurance policies follow a Weibull distribution. The
median claim amount is £1,000 and 90\% of claims are less than £5,000.
Estimate the parameters of the Weibull distribution, using the method of moments. 

\item 
% 2

\begin{enumerate}[(a)]  
% (i)
\item Explain why a sequence of pseudo-random numbers are often preferred to
truly random numbers for Monte Carlo simulation.

\item An actuary is generating pairs of standard Normal variates using the Polar algorithm
and pairs of pseudo-random variates from a U(0,1) distribution.
\item 
Determine the pairs of standard Normal variates generated by the following
pairs of pseudo-random variates where possible.
(a) 0.062, 0.293
(b) 0.984, 0.794
(c) 0.008, 0.961
[3]
\end{enumerate}
\end{enumerate}
%%%%%%%%%%%%%%%%%%%%%%%%%%%%%%%%%%%%%%%%%%%%%%%%%%%%%%%%%%%%%%%%%%
\newpage

1
The matrix below shows the losses to Player A in a two player zero sum game. The
strategies for Player A are denoted I, II, III and IV.
I
Player B
2
1
2
3
10
8
3
Player A
II
III IV
6
X
9 3
Y
4
4
6
7
(i) Determine the values of X and Y for which there are dominated strategies for
Player A.
[4]
(ii) Determine whether there exist values of X and Y which give rise to a saddle
point.
[3]
[Total 7]
The table below shows cumulative claim amounts incurred on a portfolio of insurance
policies.
Accident Year
0
2011
2012
2013
2014
1,509
1,542
1,734
1,773
Development Year
1
2
1,969
2,186
1,924
2,106
2,985
3
2,207
Annual premiums written in 2014 were 4,013 and the ultimate loss ratio has been
estimated as 93.5%. Claims can be assumed to be fully run off by the end of
development year 3.
Estimate the total claims arising from policies written in 2014 only, using the
Bornhuetter-Ferguson method.
3
(i)
(a)
(b)

Explain why an insurance company might purchase reinsurance.
Describe two types of reinsurance.
[3]
The claim amounts on a particular type of insurance policy follow a Pareto
distribution with mean 270 and standard deviation 340.
(ii)

Determine the lowest retention amount such that under excess of loss
reinsurance the probability of a claim involving the reinsurer is 5%.
[4]
%%%%%%%%%%%%%%%%%%%%%%%%%%%%%%%%%%%%%%%%%%%%%%%%%%%%%%%%%%%%%%
\newpage
%%%%%%%%%%%%%%%%%%%%%%%%%%%%%%%%%%%%%%%%%%%%%%%%%%%%%%%%%%%%%%%%%%%%%%%%%%%%%%%%%%%%%%%%%%%%%%%%%%%%%%%%
1
(i)
III will dominate II if X  6.
III will dominate IV if X  Y.
I will dominate IV if Y  8
I does not dominate II and vice versa (since I is better if Player B chooses 1
and II is better if Player B chooses 2).
III cannot dominate I (since I is better if Player B chooses 1).
IV cannot dominate any strategy since it gives the worst outcome if Player B
chooses 3.
Similarly II cannot dominate any strategy since it gives the worst result if
Player B chooses 1.
So there will exist dominated strategies if X  6 or X  Y or Y  8.
(ii)
\begin{itemize}
\item A saddle point exists if an entry is both the largest in its column and the
smallest in its row.
\item This can only occur in row 2 (the smallest values in rows 1 and 3 are not the
largest in their columns).
\item X cannot give a saddle since this would require X  8 (to be the smallest in
row 2) but then X would not be the largest in its column.
\item Equally Y cannot give a saddle point as this would require that Y  8 in which
case Y would not be the largest in its column.
\item 
So there are no values of X and Y which give a saddle point.
\item 
This straightforward question was relatively poorly answered, with many candidates
seemingly put off by the unfamiliar nature of the question.
\end{itemize}

\end{document}
