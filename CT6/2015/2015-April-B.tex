\documentclass[a4paper,12pt]{article}

%%%%%%%%%%%%%%%%%%%%%%%%%%%%%%%%%%%%%%%%%%%%%%%%%%%%%%%%%%%%%%%%%%%%%%%%%%%%%%%%%%%%%%%%%%%%%%%%%%%%%%%%%%%%%%%%%%%%%%%%%%%%%%%%%%%%%%%%%%%%%%%%%%%%%%%%%%%%%%%%%%%%%%%%%%%%%%%%%%%%%%%%%%%%%%%%%%%%%%%%%%%%%%%%%%%%%%%%%%%%%%%%%%%%%%%%%%%%%%%%%%%%%%%%%%%%

\usepackage{eurosym}
\usepackage{vmargin}
\usepackage{amsmath}
\usepackage{graphics}
\usepackage{epsfig}
\usepackage{enumerate}
\usepackage{multicol}
\usepackage{subfigure}
\usepackage{fancyhdr}
\usepackage{listings}
\usepackage{framed}
\usepackage{graphicx}
\usepackage{amsmath}
\usepackage{chngpage}

%\usepackage{bigints}
\usepackage{vmargin}

% left top textwidth textheight headheight

% headsep footheight footskip

\setmargins{2.0cm}{2.5cm}{16 cm}{22cm}{0.5cm}{0cm}{1cm}{1cm}

\renewcommand{\baselinestretch}{1.3}

\setcounter{MaxMatrixCols}{10}

\begin{document}

\begin{enumerate}
3
(i)
(a)
To protect itself from the risk of large claims.
(b)
(ii)
 Excess of loss reinsurance where the reinsurer pays any amount of a claim above the retention.
 Proportional reinsurance where the reinsurer pays a fixed proportion of any claim.
We must first find the parameters  and  of the Pareto distribution.
 2

= 270 and
= 340 2
2
  1
(   1) (   2)

 2

= 340 2
2
  2 (   1)
so
340 2

=
= 1.585733882
 2
270 2
so  =
2  1.585733882
1.585733882  1
= 5.4145
and  = 270 × 4.4145 = 1191.920375
%%%%%%%%%%%%%%%%%%%%%%%%%%%%%%%%%%%%%%%%%%%%%%%%%%%%%%%%%%%%%%%%%%%%%%%%%%%%%%%%%%%%%%%%%%%%%%%5
We need to find M such that P(X > M) = 0.05

i.e.
  

 = 0.05
  M 

 M

= 1

0.05
= 1
0.05  (   M )
1 

  1  0.05  



M = 
1
0.05 
1 

5.4145

1191.920375   1  0.05




=
1
0.05 5.4145
= 880.8
%This question was the best answered on the paper with most candidates scoring well. Some candidates were unable to manipulate the Pareto distribution.
\newpage
%%%%%%%%%%%%%%%%%%%%%%%%%%%%%%%%%%%%%%%%%%%%%%%%%%%%%%%%%%%%%%%%%%%%%%%%%%%%%%%%%%%%%%%%%%5
%%%%%%%%%%%%%%%%%%%%%%%

%%- Question 4

Let $X$ be a random variable with density $f(x) = e^{x}$ for $x > 0$.



\begin{enumerate}[(i)]
\item 
(i) Construct an algorithm for generating random samples from X. 
A sequence of simulated observations is required from the density function
$h(x) = 2xe^{-x^2} x > 0.$
\item (ii) Construct a procedure using the Acceptance-Rejection method to obtain the required observations. 
\item (iii) Calculate the expected number of pseudo-random numbers required to generate 10 observations from h using the algorithm in part (ii). 
\end{enumerate}

4
(i)
\begin{itemize}
\item X has an exponential distribution with parameter 1.
\item Let u be a sample from a $U(0,1)$ distribution.
\item Then using the inverse transform method we set
u = F(x) = 1  e x
i.e. 1  u = e x
i.e. x = log(1  u)
\item so the algorithm is
Step 1
Step 2
Generate u from U(0,1).
Set x = log(1  u).
\end{itemize}
%%--- Page 5Subject CT6 %%%%%%%%%%%%%%%%%%%%%%%%%%%%%%%%%%%%%%%%%%%%%%% – April 2015 – Examiners’ Report
(ii)
We first find C = max
x  0
= max
h ( x )
f ( x )
2 xe  x
x  0
2
e  x
= max 2 xe x  x
2
x  0
To find the maximum consider g(x) = 2 xe x  x
then
log g(x) = log2 + logx + x  x 2
d log g ( x )
dx = 1  1  2 x
x
2
setting this equal to zero we have
1
 1  2 x = 0
x
i.e.
1 + x  2x 2 = 0
2x 2  x  1 = 0
(2x + 1)(x 1) = 0
x = 1 or x = 1⁄2
but x > 0 so C = 2 × 1 × e 11 = 2
so now set w(x) =
2
h ( x )
= xe x  x
2 f ( x )
and our algorithm is
\begin{description}
\item[Step 1] Generate $u$ from $U(0,1)$ distribution.
\item[Step 2] Generate $x = -log(1 - u)$.
\item[Step 3] Generate $v$ from $U(0,1)$.
\item[Step 4] If $v > w(x)$ return to step 1 else return $x$.
\end{description}
%%%%%%%%%%%%%%%%%%%%%%%
%%--- Subject CT6 %%%%%%%%%%%%%%%%%%%%%%%%%%%%%%%%%%%%%%%%%%%%%%% – April 2015 – Examiners’ Report
(iii)
1
1
=
so on average 2 simulations are needed to
c
2
return 1 value, so in this case we need 20 simulations. Each simulation requires 2 pseudo-random numbers so on average we will need 40 pseudo-random numbers.
We accept a proportion
% Most candidates were able to produce good quality answers to part (i), although the majority
% struggled with part (ii). Candidates with the confidence to apply the acceptance-rejection method scored well.
\end{document}
