\documentclass[a4paper,12pt]{article}

%%%%%%%%%%%%%%%%%%%%%%%%%%%%%%%%%%%%%%%%%%%%%%%%%%%%%%%%%%%%%%%%%%%%%%%%%%%%%%%%%%%%%%%%%%%%%%%%%%%%%%%%%%%%%%%%%%%%%%%%%%%%%%%%%%%%%%%%%%%%%%%%%%%%%%%%%%%%%%%%%%%%%%%%%%%%%%%%%%%%%%%%%%%%%%%%%%%%%%%%%%%%%%%%%%%%%%%%%%%%%%%%%%%%%%%%%%%%%%%%%%%%%%%%%%%%

\usepackage{eurosym}
\usepackage{vmargin}
\usepackage{amsmath}
\usepackage{graphics}
\usepackage{epsfig}
\usepackage{enumerate}
\usepackage{multicol}
\usepackage{subfigure}
\usepackage{fancyhdr}
\usepackage{listings}
\usepackage{framed}
\usepackage{graphicx}
\usepackage{amsmath}
\usepackage{chngpage}

%\usepackage{bigints}
\usepackage{vmargin}

% left top textwidth textheight headheight

% headsep footheight footskip

\setmargins{2.0cm}{2.5cm}{16 cm}{22cm}{0.5cm}{0cm}{1cm}{1cm}

\renewcommand{\baselinestretch}{1.3}

\setcounter{MaxMatrixCols}{10}

\begin{document}

\begin{enumerate}
%%%%%%%%%%%%%%%%%%%%%%%%%


\item 
% 2

\begin{enumerate}[(a)]  
% (i)
\item Explain why a sequence of pseudo-random numbers are often preferred to
truly random numbers for Monte Carlo simulation.

\item An actuary is generating pairs of standard Normal variates using the Polar algorithm
and pairs of pseudo-random variates from a U(0,1) distribution.
\item 
Determine the pairs of standard Normal variates generated by the following
pairs of pseudo-random variates where possible.
(a) 0.062, 0.293
(b) 0.984, 0.794
(c) 0.008, 0.961
\end{enumerate}
\end{enumerate}
%%%%%%%%%%%%%%%%%%%%%%%%%%%%%%%%%%%%%%%%%%%%%%%%%%%%%%%%%%%%%%%%%%
\newpage

%%%%%%%%%%%%%%%%%%%%%%%%%%%%%%%%%%%%%%%%%%%%%%%%%%%%%%%%%%%%%%%%%%%%%%%%%%%%%%%%%5
Q2
(i)
Pseudo random numbers can be reproduced [1]
Only single routine required, rather than lengthy table/hardware [1]
Difficult to generate very large set of truly random numbers
(ii)
(a)
[1]
[Max 2]
v 1  2*0.062  1  .876, v 2  2*0.293  1  .414,
s  v 1 2  v 2 2  0.938772
[1]
z 1  v 1 *  2ln  s  / s   .3 2 14
z 2  v 2 *  2ln  s  / s   .1 5 19 [1]
(b)  2*0.984  1  2   2*0.794  1  2  1 , so no variate [1⁄2]
(c)  2*0.008  1  2   2*0.961  1  2  1 , so no variate [1⁄2]
Page 3Subject CT6 (Statistical Methods Core Technical) – September 2017 – Examiners’ Report
Many candidates scored well here, although a number failed to make
the adjustment to convert a U(0,1) into a U(-1,1).


%%%%%%%%%%%%%%%%%%%%%%%%%%%%%%%%%%%%%%%%%%%%%%%%%%%%%%%%%%%%%%
%%%%%%%%%%%%%%%%%%%%%%%%%%%%%%%%%%%%%%%%%%%%%%%%%%%%%%%%%%%
2
The development factors are:
DF 2,3 = 2207
= 1.0480
2106
DF 1,2 = (2106  2985)
= 1.2253
(1969  2186)
DF 0,1 = (1969  2186  1924)
= 1.2704
(1509  1542  1734)
Page 3Subject CT6 (Statistical Methods Core Technical) – April 2015 – Examiners’ Report
The initial ultimate loss for 2014 is 0.935 × 4013 = 3752.16.
 1 
Total emerging liability = initial UL ×  1  
f 

1


= 3752.16 ×  1 

 1.0480  1.2253  1.2704 
= 1452.01
Total claims = 1452.01 + 1773
= 3225.01
This standard chain ladder question was very well answered by the majority of candidates.



\newpage


An insurance company has for five years insured three different types of risk. The
number of policies in the j th year for the i th type of risk is denoted by P ij for i = 1, 2, 3
and j = 1, 2, 3, 4, 5. The average claim size per policy over all five years for the i th
type of risk is denoted by X i . The values of P ij and X i are tabulated below.
Risk type i Year 1
1
2
3 17
42
43
Number of policies
Year 2
Year 3
Year 4
23
51
31
21
60
62
29
55
98
Year 5 Mean claim size
X i
35
37
107 850
720
900
The insurance company will be insuring 30 policies of type 1 next year and has
calculated the aggregate expected claims to be 25,200 using the assumptions of
Empirical Bayes Credibility Theory Model 2.
Calculate the expected annual claims next year for risks 2 and 3 assuming the number
of policies will be 40 and 110 respectively.
\newpage

\end{document}
