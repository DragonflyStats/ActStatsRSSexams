\documentclass[a4paper,12pt]{article}



%%%%%%%%%%%%%%%%%%%%%%%%%%%%%%%%%%%%%%%%%%%%%%%%%%%%%%%%%%%%%%%%%%%%%%%%%%%%%%%%%%%%%%%%%%%%%%%%%%%%%%%%%%%%%%%%%%%%%%%%%%%%%%%%%%%%%%%%%%%%%%%%%%%%%%%%%%%%%%%%%%%%%%%%%%%%%%%%%%%%%%%%%%%%%%%%%%%%%%%%%%%%%%%%%%%%%%%%%%%%%%%%%%%%%%%%%%%%%%%%%%%%%%%%%%%%



\usepackage{eurosym}
\usepackage{vmargin}
\usepackage{amsmath}
\usepackage{graphics}
\usepackage{epsfig}
\usepackage{enumerate}
\usepackage{multicol}
\usepackage{subfigure}
\usepackage{fancyhdr}
\usepackage{listings}
\usepackage{framed}
\usepackage{graphicx}
\usepackage{amsmath}
\usepackage{chngpage}



%\usepackage{bigints}

\usepackage{vmargin}

% left top textwidth textheight headheight

% headsep footheight footskip

\setmargins{2.0cm}{2.5cm}{16 cm}{22cm}{0.5cm}{0cm}{1cm}{1cm}
\renewcommand{\baselinestretch}{1.3}
\setcounter{MaxMatrixCols}{10}
\begin{document}
\begin{enumerate}

%%- Question 5 
\item Claims X each year from a portfolio of insurance policies are normally distributed
with mean \theta and variance \tau2. Prior information is that \theta is normally distributed with
known mean \mu and known variance \sigma2.
Aggregate claims over the last n years have been xi for i = 1 to n, and you should
assume that these are independent.
\begin{enumerate}
\item (i) Derive the posterior distribution of \theta. 
\item (ii) Write down the Bayesian estimate of \theta under quadratic loss. 
\item (iii) Show that the estimate in your answer to part (ii) can be expressed in the
form of a credibility estimate, including statement of the credibility factor Z.
\end{enumerate}

\item % Question 6

\begin{enumerate}
\item (i) Explain what is meant by a saturated model. 
\item(ii) State the definition of the scaled deviance in a fitting under generalised
linear modelling. 
\item (iii) (a) Define both Pearson and deviance residuals.
(b) Explain how these two types of residuals are generally different.
(c) State in which case they are the same. 
\end{enumerate}
\end{enumerate}
%%%%%%%%%%%%%%%%%%%%%%%%%%%%%%%%%%%%%%%%%%%%%%%%%%%%%%%%%%%%%%%%%%%%%%%%%%%%%
\newpage


Q5 (i) Given
f(\theta) 
2
2 2
exp ( ) exp 1
2 2
\theta  \mu
  
\sigma \sigma
(\theta2  2\mu\theta)
and
p(x\theta) 
2
2
2 2
1 1
exp ( ) exp 1 ( 2 )
2 2
n n
i
i
i i
x x
 
 \theta
   \theta  \theta
\tau \tau  
 2
2
exp 1 ( 2 )
2
 n\theta  nx\theta
\tau
1
=
  n
i
i
x nx

 
 
 
 ,
we want
p(\thetax)  p(x\theta) p(\theta)
 2
2 2 2 2
exp 1
2 2
   n  \theta   \mu  nx  \theta         \sigma \tau   \sigma \tau  
 exp  \tau2  n\sigma2
2\sigma2\tau2 \theta2  2 \mu\tau2  nx\sigma2
\tau2  n\sigma2

 

 
\theta
 

 








2 2 2 2 2
2 2 2 2 exp
2
n nx
n
 \tau  \sigma  \mu\tau  \sigma    \theta   
 \sigma \tau  \tau  \sigma  
 \thetax ~
  2 2 2 2
2 2 2 2 2 2 N n x,
n n n
 \tau \sigma \sigma \tau 
 \mu  
 \tau  \sigma \tau  \sigma \tau  \sigma 
 %%%%%%%%%%%%%%%%%%%%%%%%%%%%%%%%%%%%%%%%%%%%%%% – September 2015 – Examiners’ Report
Page 6
(ii) The posterior mean (the point estimator under quadratic loss) is
E(\thetax) =
  2 2
2 2 2 2
n x
n n
\tau \sigma
\mu 
\tau  \sigma \tau  \sigma
(iii) E(\thetax) = (1  Z) \mu + Zx
where
Z =
  2
2 2
n
n
\sigma
\tau  \sigma
= 2
2
n
n \tau

\sigma
is the credibility factor. Hence E(\thetax) can be expressed in the form of a
credibility estimate.
Again well prepared candidates who had learnt the relevant bookwork scored
very well on this question. Some candidates attempted to “fudge” the result or
only quoted the result, making no attempt to derive it, and hence scored
poorly.
Q6 (i) The saturated model is one where the number of parameters is the same as the
data points,
i.e. the fitted values are the same as the fitted data.
(ii) The scaled deviance is twice the difference between the log likelihood values
between the model in consideration and the saturated model.
(iii) (a) Pearson residuals are
var( ˆ )
y \muˆ
\mu
where ˆ \mu is the fitted response
estimator.
The deviance residuals are sign(y \muˆ)di where di is the contribution of
the i-th to the total deviances,
i.e. 2
di is the scaled deviance.
(b) The Pearson residuals tend to be skewed in non normal data
while the deviance residuals tend to be symmetric and hence the
normal assumption is more appropriate.
For that reason the latter is preferred in actuarial applications.
 %%%%%%%%%%%%%%%%%%%%%%%%%%%%%%%%%%%%%%%%%%%%%%% – September 2015 – Examiners’ Report
Page 7
(c) In the normal data, normal residuals these are identical.
Most candidates were able to score at least some of the marks here, but only
the stronger candidates had sufficient recall and understanding of the full
detail of the bookwork in order to score very well.
