\documentclass[a4paper,12pt]{article}

%%%%%%%%%%%%%%%%%%%%%%%%%%%%%%%%%%%%%%%%%%%%%%%%%%%%%%%%%%%%%%%%%%%%%%%%%%%%%%%%%%%%%%%%%%%%%%%%%%%%%%%%%%%%%%%%%%%%%%%%%%%%%%%%%%%%%%%%%%%%%%%%%%%%%%%%%%%%%%%%%%%%%%%%%%%%%%%%%%%%%%%%%%%%%%%%%%%%%%%%%%%%%%%%%%%%%%%%%%%%%%%%%%%%%%%%%%%%%%%%%%%%%%%%%%%%

\usepackage{eurosym}
\usepackage{vmargin}
\usepackage{amsmath}
\usepackage{graphics}
\usepackage{epsfig}
\usepackage{enumerate}
\usepackage{multicol}
\usepackage{subfigure}
\usepackage{fancyhdr}
\usepackage{listings}
\usepackage{framed}
\usepackage{graphicx}
\usepackage{amsmath}
\usepackage{chngpage}

%\usepackage{bigints}

\usepackage{vmargin}

% left top textwidth textheight headheight
% headsep footheight footskip
\setmargins{2.0cm}{2.5cm}{16 cm}{22cm}{0.5cm}{0cm}{1cm}{1cm}
\renewcommand{\baselinestretch}{1.3}

\setcounter{MaxMatrixCols}{10}

\begin{document}

\begin{enumerate}

%%%%%%%%%%%%%%%%%%%%%%%%%%%%%%%%%%%%%%%%%%%%%%%%%%
  [Total 4]
3 A particular industry always generates total profits of \$1bn in each year, shared between two companies: Raspberry Inc. and Robots Ltd. Every year the companies each need to choose between two distinct business approaches: cautious and
aggressive.
If both companies adopt the same approach in a given year, Raspberry Inc. captures
70% of the total profits. If they adopt different approaches, Robots Ltd. captures 80% of the total profits if Raspberry Inc. is cautious, and 60% of the total profits if
Raspberry Inc. is aggressive. Neither company knows what the other company’s
approach will be before adopting its own approach.
\begin{enumerate}
    \item (i) Explain why the above can be thought of as a zero-sum two person game. 
Raspberry Inc. decides to adopt a randomised strategy to setting its approach each
year.
    \item (ii) Explain what is meant by a randomised strategy. 
    \item (iii) Determine Raspberry Inc.’s optimal randomised strategy. 
\end{enumerate}


%%%%%%%%%%%%%%%%%%%%%%%%%%%%%%%%%%%%%%
  Q3 (i) Because total profits are fixed, whatever one company makes the other can be
thought of as having lost, and vice-versa.
(ii) A randomised strategy is where the player randomly chooses between
different strategies, rather than adopting a fixed approach.
(iii) Raspberry Inc. will randomly choose the cautious approach with probability p,
and the aggressive approach with probability (1  p).
Robots/Raspberry Cautious Aggressive
Cautious 700 400
Aggressive 200 700
In order to determine the optimal strategy we need to equate the payoffs:
  700p + 400(1  p) = 200p + 700(1  p)
800p = 300
so p = 3/8
So Raspberry Inc. should adopt the Cautious approach 3/8 of the time.
\newpage
Candidates familiar with zero-sum two person games were able to score very
well on this question. Weaker candidates were unfamiliar with randomised
strategies.
\end{document}
