
\documentclass[a4paper,12pt]{article}

%%%%%%%%%%%%%%%%%%%%%%%%%%%%%%%%%%%%%%%%%%%%%%%%%%%%%%%%%%%%%%%%%%%%%%%%%%%%%%%%%%%%%%%%%%%%%%%%%%%%%%%%%%%%%%%%%%%%%%%%%%%%%%%%%%%%%%%%%%%%%%%%%%%%%%%%%%%%%%%%%%%%%%%%%%%%%%%%%%%%%%%%%%%%%%%%%%%%%%%%%%%%%%%%%%%%%%%%%%%%%%%%%%%%%%%%%%%%%%%%%%%%%%%%%%%%

\usepackage{eurosym}
\usepackage{vmargin}
\usepackage{amsmath}
\usepackage{graphics}
\usepackage{epsfig}
\usepackage{enumerate}
\usepackage{multicol}
\usepackage{subfigure}
\usepackage{fancyhdr}
\usepackage{listings}
\usepackage{framed}
\usepackage{graphicx}
\usepackage{amsmath}
\usepackage{chngpage}

%\usepackage{bigints}
\usepackage{vmargin}

% left top textwidth textheight headheight

% headsep footheight footskip

\setmargins{2.0cm}{2.5cm}{16 cm}{22cm}{0.5cm}{0cm}{1cm}{1cm}

\renewcommand{\baselinestretch}{1.3}

\setcounter{MaxMatrixCols}{10}

\begin{document}
\begin{enumerate}

The following time series model is being used to model monthly data:
Y t  Y t  1  Y t  12  Y t  13  e t   1 e t  1   12 e t  12   1  12 e t  13
where e t is a white noise process with variance  2 .
8
\begin{enumerate}[(a)]
\item (i) Perform two differencing transformations and show that the result is a moving
average process which you may assume to be stationary.
\item (ii) Explain why this transformation is called seasonal differencing.
\item (iii) Derive the auto-correlation function of the model generated in part (i).
\end{enumerate}

\newpage

%%%%%%%%%%%%%%%%%%%%%%%%%%%%%%%%%%%%%%%%%%%%%%%%%%%%%%%%%%%%%%%%%%%%%%%%%%%%%%%%%%%%%%%%%%%%%%%%%%%%%%%

[Total 12]
[1]
The number of claims, N, in a given year on a particular type of insurance policy is
given by:
P(N = n) = 0.8  0.2 n
n = 0, 1, 2, ...
Individual claim amounts are independent from claim to claim and follow a Pareto
distribution with parameters  = 5 and  = 1,000.
(i) Calculate the mean and variance of the aggregate annual claims per policy. [4]
(ii) Calculate the probability that aggregate annual claims exceed 400 using:
(a)
(b)
a Normal approximation.
a Lognormal approximation.
[6]
(iii)
CT6 A2015–4
Explain which approximation in part (ii) you believe is more reliable.
[2]

10
(ii) Derive the posterior distribution of p in terms of g.
[4]
(iii) Show that it is not possible in this case for the Bayes estimate of p to be the
same under quadratic loss and all-or-nothing loss.


%%%%%%%%%%%%%%%%%%%%%%%%%%%%%%%%%%%%%%%%%%%%%%%%% CT6 A2015–5
\newpage

%%%%%%%%%%%%%%%%%%%%%%%%%%%%%%%%%%%%%%%%%%%%%%%%%%%%%%%%%%%%%% CT6 April 2015 – Examiners’ Report
8
(i)
First note that N has a type 2 negative binomial distribution with parameters
p = 0.8 and k = 1. Hence
E(N) =
0.2
= 0.25
0.8
Var(N) =
0.2
0.8 2
= 0.3125
Let X denote the distribution of an individual claim. Then
E ( X ) =

1000
=
= 250
4
  1
Var( X ) = 250 2 ×
5
= 104,166.666 = (322.75) 2
3
%%%%%%%%%%%%%%%%%%%%%%%%%%%%%%%%%%%%%%%%%%%%%%%%%%%%%%%%%%%%%%%%%%%%%%%%%%
Now let S denote aggregate annual claims. Then
E ( S )
= E ( N ) E ( X ) = 0.25 × 250 = 62.5
Var( S ) = E ( N ) Var( X ) + Var( N ) E ( X ) 2
= 0.3125 × 250 2 + 0.25 × 104,166.666
= 45,572.92 = 213.478 2
(ii)
(a)
P ( S > 400) = P ( N (62.5, 213.478 2 ) > 400)
400  62.5 

= P  N (0,1) 

213.478 

= P ( N (0,1) > 1.581)
= 1  [0.94295 × 0.9 + 0.1 × 0.94408]
= 0.0569
%%%%%%%%%%%%%%%%%%%%%%%%%%%%%%%%%%%%%%%%%%%%%%%%%%%%%%%%%%%%%%%%%%%%%%%%%%%%
(b)
Let  and  be the parameters of the underlying Normal distribution.
Then
e

 2
2
= 62.5
e 2  ( e   1) = 213.478 2
2
2
(A)
(B)
(B) ÷ (A) 2  e   1 =
2
213.478 2
62.5 2
= 11.66665
 2 = log12.66665 = 2.53897 = 1.5934 2
substituting into (A)  + 2.53897 = log62.5
2
so  = log62.5  2.53897 = 2.8657
2
and so P ( S > 400) = P ( N (2.8657, 1.5934 2 ) > log 400)
log 400  2.8657 

= P  N (0,1) 

1.5934


= P ( N (0,1) > 1.9617)
= 1 -(0.17 × 0.97558 + 0.83 × 0.97500)
= 0.0249
%%%%%%%%%%%%%%%%%%%%%%%%%%%%%%%%%%%%%%%%%%%%%%
(iii)
The Pareto distribution is significantly skewed and the Normal approximation is not. The Normal approximation in (ii)(b) has variance 213.48 2 and mean 62.5, so negative values of S (which are impossible in reality) are less than 1
standard deviation from the mean.
The approximation in (ii)(b) will therefore be more reliable.
This question was well answered by the majority of candidates. Full credit was given to
alternative correct answers in part (iii).


\end{document}
