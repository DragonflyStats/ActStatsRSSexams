\documentclass[a4paper,12pt]{article}

\usepackage{eurosym}
\usepackage{vmargin}
\usepackage{amsmath}
\usepackage{graphics}
\usepackage{epsfig}
\usepackage{enumerate}
\usepackage{multicol}
\usepackage{subfigure}
\usepackage{fancyhdr}
\usepackage{listings}
\usepackage{framed}
\usepackage{graphicx}
\usepackage{amsmath}

\usepackage{chngpage}



%\usepackage{bigints}

\usepackage{vmargin}



% left top textwidth textheight headheight

% headsep footheight footskip

\setmargins{2.0cm}{2.5cm}{16 cm}{22cm}{0.5cm}{0cm}{1cm}{1cm}

\renewcommand{\baselinestretch}{1.3}

\setcounter{MaxMatrixCols}{10}



\begin{document}

\newpage
8 The run-off triangle below shows cumulative claims incurred on a portfolio of
general insurance policies
Development Year
Policy Year 0 1 2 3
2011 1,528 2,034 2,212 2,310
2012 1,812 2,251 2,951
2013 1,693 1,851
2014 2,125
Annual premiums written in 2014 were 4,023 and the ultimate loss ratio has been estimated as 91%. Claims paid to date for policy year 2014 are 572.
Estimate the outstanding claims to be paid arising from policies written in 2014 only, using the Bornheutter-Ferguson technique, stating any assumptions that you
make. [9]


%%%%%%%%%%%%%%%%%%%%%%%%%%%%%%%%%%%%%%%%%%%%%%%%%%%%%%%%%%%%%%%
\newpage

%%%%%%%%%%%%%%%%%%%%%%%%%%%%%%%%%%%%%%%%%%%%%%%%%%%%%%%%%%%%%%%%%%%%%%%%%%%%%%%%%%%%%%%%%%
Q8 Assume claims fully run off by the end of development year 3.
Each year develops in the same way
The weighted average past inflation is repeated
The loss ratio is appropriate
DF 2–3 = 2,310/2,212 = 1.044 304
DF 1–2 = (2,951 + 2,212) / (2,251 + 2,034) = 1.204 901
DF 0–1 = (2,034 + 2,251 + 1,851) / (1,528 + 1,812 + 1,693) = 1.219 154
Ultimate loss for 2014 = 0.91 * 4,023 = 3,660.93
Emerging liability = 3660.93 * (1 – 1/(1.044304 * 1.204901 * 1.219 154))
= 1,274.466
So total amount for policies written in 2014 is
2,125 + 1,274.466 = 3,399.466
so remaining
3,399.466 – 572 = 2,827.5
This straightforward chain ladder question was the best answered question on
the paper, although a disappointing number of candidates slipped up towards
the end. Weaker candidates also failed to state the assumptions required.
 %%%%%%%%%%%%%%%%%%%%%%%%%%%%%%%%%%%%%%%%%%%%%%% – September 2015 – Examiners’ Report
\end{document}
