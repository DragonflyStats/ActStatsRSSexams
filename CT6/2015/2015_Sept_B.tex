\documentclass[a4paper,12pt]{article}

%%%%%%%%%%%%%%%%%%%%%%%%%%%%%%%%%%%%%%%%%%%%%%%%%%%%%%%%%%%%%%%%%%%%%%%%%%%%%%%%%%%%%%%%%%%%%%%%%%%%%%%%%%%%%%%%%%%%%%%%%%%%%%%%%%%%%%%%%%%%%%%%%%%%%%%%%%%%%%%%%%%%%%%%%%%%%%%%%%%%%%%%%%%%%%%%%%%%%%%%%%%%%%%%%%%%%%%%%%%%%%%%%%%%%%%%%%%%%%%%%%%%%%%%%%%%

\usepackage{eurosym}
\usepackage{vmargin}
\usepackage{amsmath}
\usepackage{graphics}
\usepackage{epsfig}
\usepackage{enumerate}
\usepackage{multicol}
\usepackage{subfigure}
\usepackage{fancyhdr}
\usepackage{listings}
\usepackage{framed}
\usepackage{graphicx}
\usepackage{amsmath}
\usepackage{chngpage}

%\usepackage{bigints}

\usepackage{vmargin}

% left top textwidth textheight headheight
% headsep footheight footskip
\setmargins{2.0cm}{2.5cm}{16 cm}{22cm}{0.5cm}{0cm}{1cm}{1cm}
\renewcommand{\baselinestretch}{1.3}

\setcounter{MaxMatrixCols}{10}

\begin{document}

\begin{enumerate}

%%%%%%%%%%%%%%%%%%%%%%%%%%%%%%%%%%%%%%%%%%%%%%%%%%
  [Total 4]
3 A particular industry always generates total profits of $1bn in each year, shared
between two companies: Raspberry Inc. and Robots Ltd. Every year the companies
each need to choose between two distinct business approaches: cautious and
aggressive.
If both companies adopt the same approach in a given year, Raspberry Inc. captures
70% of the total profits. If they adopt different approaches, Robots Ltd. captures
80% of the total profits if Raspberry Inc. is cautious, and 60% of the total profits if
Raspberry Inc. is aggressive. Neither company knows what the other company’s
approach will be before adopting its own approach.
(i) Explain why the above can be thought of as a zero-sum two person game. 
Raspberry Inc. decides to adopt a randomised strategy to setting its approach each
year.
(ii) Explain what is meant by a randomised strategy. 
(iii) Determine Raspberry Inc.’s optimal randomised strategy. [4]
%%%%%%%%%%%%%%%%%%%%%%%%%%%%%%%%%%%%%%%%%%%%%%%%%%%%%%%%%%%%%%%%%%%

4 A small island is holding a vote on independence. Two recent survey results are
shown below:
  Poll Sample size Support for
independence
A 10 5
B 20 11
You should assume that the samples are independent.
A politician is using a suitable uniform distribution as the prior distribution in order
to estimate the proportion  in favour of independence.
(i) Calculate an estimate of  under the quadratic loss function. [3]
A rival politician decides to use instead a beta distribution as the prior, with
parameters  and , where  = .
(ii) Determine the new estimate of  under the “all-or-nothing” loss function in
terms of . [4]
[Total 7]
%%%%%%%%%%%%%%%%%%%%%%%%%%%%%%%%%%%%%%
  Q3 (i) Because total profits are fixed, whatever one company makes the other can be
thought of as having lost, and vice-versa.
(ii) A randomised strategy is where the player randomly chooses between
different strategies, rather than adopting a fixed approach.
(iii) Raspberry Inc. will randomly choose the cautious approach with probability p,
and the aggressive approach with probability (1  p).
Robots/Raspberry Cautious Aggressive
Cautious 700 400
Aggressive 200 700
In order to determine the optimal strategy we need to equate the payoffs:
  700p + 400(1  p) = 200p + 700(1  p)
800p = 300
so p = 3/8
So Raspberry Inc. should adopt the Cautious approach 3/8 of the time.
Candidates familiar with zero-sum two person games were able to score very
well on this question. Weaker candidates were unfamiliar with randomised
strategies.
Q4 (i) A suitable distribution is U(0,1) as theta must lie between 0 & 1.
f() = 1 for 0    1
Binomial distribution so likelihood function
L() = (30 C 16) 16 (1  )14
Bayes theorem: PDF (posterior) = PDF (prior) * likelihood
PDF (Posterior) proportional to 16 (1 – )14
So distribution of theta|sample is Beta (17,15)
Under quadratic loss estimate of theta is mean so
17/(17 + 15) = 0.53125
(ii) prior PDF proportional to 1 * (1 – )1
so using Bayes again posterior proportional to 15+ * (1 – )13+
Under all-or-nothing we need the mode of the posterior
Subject CT6 (Statistical Methods Core Technical) – September 2015 – Examiners’ Report
Page 5
Take logs (15 + ) log + (13 + ) log (1  )
differentiate (15 + ) /   (13 + ) / (1  ) = 0
so (15 + ) (1 – ) = (13 + )  so  (2 + 28) = 15 + 
so  = (15 + ) / (28 + 2)
since f(0) = f(1) = 0 this must be a maximum
Well prepared candidates had little difficulty with this straightforward question
on Bayes’ theory. Most candidates were able to pick up at least a few marks
by showing knowledge of the basic theory.
