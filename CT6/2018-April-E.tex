[5]
PLEASE TURN OVER9
A health actuary is modelling the impact of a new infection which occurs in hospitals.
He is studying 100 long term patients in different hospitals across a country.
Infections occur according to a Poisson process, and the additional cost incurred due
to an infection is a random variable, X, with mean 250 and variance 200.
Based on previous infections, the health actuary believes that in all hospitals across
the country, the following applies:
Patient Type
Proportion of patients Poisson parameter
High resistance 1 in 2 0.1
Moderate resistance 1 in 3 0.3
Low resistance 1 in 6 0.9
There is no way of knowing in advance which particular patients are more resistant.
For a given patient, let l i represent the Poisson parameter, and let S i represent the
additional cost.
(i) Explain whether this is a heterogeneous or homogeneous group of risks.
[2]
(ii) (a) Calculate the mean and variance of l i .
(b) State the formulae for the mean and variance of S i , conditional on l i , in
terms of the moments of X.
(c) Show that the unconditional mean of S i is 75, and determine the
variance of S i .
(d)
Calculate the mean and variance of the aggregate additional costs for
all 100 patients.
[5]
In another country, a health actuary is modelling the same infection, again for 100
patients, but in a single hospital where it is believed each patient has the same level
of resistance, and hence the Poisson parameter for each patient is the same. It is not
known whether the Poisson parameter is 0.2 or 0.4, each being equally likely. The
additional costs due to infection have the same distribution as above.
(iii) Calculate the mean and variance of the aggregate additional costs.
(iv)
 Comment on your answers to parts (ii) (d) and (iii).
CT6 A2018–6
[4]
%%%%%%%%%%%%%%%%%%%%%%%%%%%%
Q9
(i)
(ii)
This is a heterogeneous group, ...
... since the parameters vary by patient, not at the overall level. [1]
[1]
1
1
= 0.3
(a) E =
( λ i ) 0.5*0.1 + *0.3 + *0.9
3
6 [1⁄2]
( )
1
1
2
E =
λ i 2 0.5*0.1 2 + *0.3 2 + *0.9
=
0.17
3
6
So Var ( λ i =
) 0.17 − 0.3 2 = 0.08 [1]
(b) E [ S i | λ i ] = [ λ i m 1 ] ,   var [ S i | λ i ]   =
[ λ i m 2 ] [1]
(c) E ( S i ) = E   E [ S i | λ i ]   = E [ λ i m 1 ] = 0.3 m 1 = 0.3 * 250 = 75 [1⁄2]
[11⁄2]
(d) For the whole portfolio, since the variables are independent and identically
distributed, the mean and variance are just 100*75 and 100*23,810 i.e. 7,500
and 2,381,000 respectively.
[1⁄2]
(iii)
Now homogeneous
(
)
=
− 0.3 2 0.01
So Var
( λ i ) 0.5* 0.2 2 + 0.4 2 = [1]
 n 
E  =
=
=
250 7500
( S 1 ) 0.3*100*
∑ S i  nE
  i = 1   [1]

  n
 n 
 n
 
 
Var
=
 ∑ S i  E  Var   ∑ S i | λ    + Var  E   ∑ S i | λ   
 
   i 1
  i 1 =
 
=
=
 i 1
  
  
= E [ n λ m 2 ] + var [ n λ m 1 ]
= 0.3*100* 200 + 0.01*100 2 * 250 2
= 6256000
[2]
(iv)
The mean is the same, but the variance is much higher. This makes sense
since the parameter uncertainty is at the overall level rather than at the
individual level, so affects the aggregate claim variance much more.
[2]
[Total 13]
Page 10Subject CT6 (Statistical Methods Core Technical) – April 2018 – Examiners’ Report
This question was the poorest answered in the entire paper, despite
being closely related to the example given in the Core Reading. Since
this topic has not come up for some time, it is likely that many
candidates were under prepared in this area.
