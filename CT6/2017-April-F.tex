\documentclass[a4paper,12pt]{article}



%%%%%%%%%%%%%%%%%%%%%%%%%%%%%%%%%%%%%%%%%%%%%%%%%%%%%%%%%%%%%%%%%%%%%%%%%%%%%%%%%%%%%%%%%%%%%%%%%%%%%%%%%%%%%%%%%%%%%%%%%%%%%%%%%%%%%%%%%%%%%%%%%%%%%%%%%%%%%%%%%%%%%%%%%%%%%%%%%%%%%%%%%%%%%%%%%%%%%%%%%%%%%%%%%%%%%%%%%%%%%%%%%%%%%%%%%%%%%%%%%%%%%%%%%%%%
  
  
  
  \usepackage{eurosym}

\usepackage{vmargin}

\usepackage{amsmath}

\usepackage{graphics}

\usepackage{epsfig}

\usepackage{enumerate}

\usepackage{multicol}

\usepackage{subfigure}

\usepackage{fancyhdr}
\usepackage{listings}
\usepackage{framed}
\usepackage{graphicx}
\usepackage{amsmath}
\usepackage{chngpage}

%\usepackage{bigints}

\usepackage{vmargin}



% left top textwidth textheight headheight

% headsep footheight footskip

\setmargins{2.0cm}{2.5cm}{16 cm}{22cm}{0.5cm}{0cm}{1cm}{1cm}

\renewcommand{\baselinestretch}{1.3}

\setcounter{MaxMatrixCols}{10}

\begin{document}

\begin{enumerate}

10 Total annual claim amounts S on a portfolio of insurance policies come from two
independent types of policies:
  Type I, which have claim amounts uniformly distributed between 3,000 and 4,000.
Type II, which have claim amounts following an Exponential distribution with mean
3,600.
Claims occur according to a Poisson process, with mean 15 per annum for Type 1
claims and mean 25 per annum for Type 2 claims.
The insurance company uses a premium loading factor of 7% and checks for ruin at
the end of each year.
(i) Calculate the mean and standard deviation of S. [3]
(ii) Calculate the minimum initial surplus Um required such that the probability of
ruin at the end of the first year is less than 0.015, using a Normal
approximation for the distribution of S. [4]
Regulatory reforms mean the insurance company is trying to reduce this probability of
ruin to less than 0.005. The insurance company is therefore purchasing proportional
reinsurance from a reinsurer, who uses a premium loading factor of 17% in its
premiums.
The insurance company retains a proportionα of each claim, and denotes by SI the
aggregate annual claims it retains net of reinsurance. The insurance company
continues to hold initial surplus Um.
(iii) Calculate the maximum proportion αmax that the insurance company can retain
in order to keep the probability of ruin less than 0.005, using a Normal
approximation for the distribution of SI. [6]
The insurance company is concerned that αmax is too low, reducing its profits, and
intends to retain a higher proportion.
(iv) Suggest other ways in which the insurance company can reduce the
probability of ruin. [3]
[Total 16]
END OF PAPER

%%%%%%%%%%%%%%%%%%%%%%%%%%%%%%%%%%%%%%%%%%%%%%%%%%%%%%%%%%%%%%%%%%%%%%%%%%%%%%%%
  Q10 (i) ES1 15*3500  52,500, ES2   25*3600  90,000, so ES  142,500
[1]
   2 2
1
Var 15* 1 4000 3000 3500 185,000,000
12
S      
 
[½]
   2 
Var S2  25* 2*3600  648,000,000 [½]
By independence
Var S   Var S1   Var S2   833000000 [½]
And so sd S   Var S   28,862 [½]
(ii) U 1 U  c*1 S 1 U 1.07ES   S [1]
        
 
0.07
1 0 1.07
U ES
P U P S U E S P Z
sd S
  
       
 
[1½]
For this to be less than 0.015, need
 
 
0.07
2.1701, m 2.1701* 28,862 0.07 *142,500 52,658
U ES
U
sd S

    [1½]
(iii) Now
U1 Um  cnet*1 SI 1 Um 1.07ES  1.171 ES   SI 
Um  1.17  0.1 ES   SI [1½]
Also SI ~ N ES ,2Var S  [1]
Subject CT6 (Statistical Methods Core Technical) – April 2017 – Examiners’ Report
Page 12
So PU1  0  PSI  Um  1.17  0.1 E S 
   
 
0.17 0.1
*
  Um E S
P Z
sd S
        
  
[1]
We need
   
 
0.17 0.1 52,655 0.17 0.1*142,500
2.5758,
* * 28,862
Um E S
sd S
   

 
> 2.5758
[1]
So
2.5758* 28,862 * 142,500 * 0.17  52,655  0.1*142,500
So
  76.6% [1½]
(iv) Use a higher initial surplus, [1]
increase its premium loading, [1]
insure more type I claims relative to Type II claims since they have a smaller
variance. [1]
[Total 16]
Candidates familiar with reinsurance theory were able to score highly here.
END OF EXAMINERS’ REPORT
