CT6 S2018–4
[3]7
Claims on a portfolio of insurance policies arise as a Poisson process with parameter
l. Individual claim amounts are taken from a distribution X and we define
m i = E(X i ) for i = 1, 2, .... The insurance company calculates premiums using a
premium loading of q.
(i) Define the adjustment coefficient R.[1]
(ii) 2θm 1
Show that R can be approximated as
, by truncating the series expansion
m 2
of M X (t).[3]
Now suppose that X follows an exponential distribution with parameter γ.
(iii)
Show that R =
θγ
.[3]
( 1 + θ )
The insurance company uses a premium loading of 12%, and the mean claim amount
is 200.
(iv)
Calculate R, commenting on the difference with the approximation to R shown
in part (ii).
[3]
The initial surplus is 5,000.
(v) Calculate an upper bound for the ultimate probability of ruin.
(vi) Suggest two methods by which the insurance company can reduce the
probability of ruin.
[2]
[Total 13]


%%%%%%%%%%%%%%%%%%%%%%%%%%%%%%%%%%%%%%%%%%%%%%%%%%%%%%%%%%%%%%%%%%%%%%%%%%%%%%%%%%%
Q7
(i)
The adjustment coefficient is the unique positive solution to
λM X (R) − λ − λ(1 + θ) E(X) R = 0
[1]
Cancelling the λ terms we have
(ii)
M X (R) = E(e RX ) = 1 + (1 + θ) E(X)R


R 2 X 2
E  1 + RX +
+ ...  = 1 + (1 + θ) E(X)R


2


[1⁄2]
[1⁄2]
And truncating the expression we get
E(1 + RX + R 2 X 2 /2) = 1 + (1 + θ) E(X)R [1⁄2]
i.e. 1 + Rm 1 + R 2 m 2 /2 = 1 + (1 + θ) m 1 R [1⁄2]
i.e. R 2 m 2 = 2θm 1 R [1⁄2]
i.e. R =
2 θ m 1
m 2
[1⁄2]
1 + (1 + θ) E(X)R = M X (R)
(iii)
( γ )
 1 + ( 1 + θ ) R  1 − R =
γ   (
γ ) 1
 
1 + (1 + θ ) R = 1 − R
− 1
γ
1 + θ
γ 2
R
(
0
R − θ R =
2
[2]
γ
)
R
0
γ ( 1 + θ ) γ − θ =
Rejecting R = 0; we therefore have
Page 7Subject CT6 (Statistical Methods Core Technical) – September 2018 – Examiners’ Report
R =
(iv)
θγ
(1 + θ )
[1]
m 1 is 200, m 2 = Var(X) + [E(X)] 2 = 200 2 + 200 2 = 80000
[1]
Approximation is therefore 2 * 12% * 200 / 80,000 = 0.000 6 [1⁄2]
True value is (12% * 1/200) / 1.12 = 0.000 536 [1⁄2]
As expected, the approximation is an upper bound for R.
− 0.000536*5000
=
e − RU e =
6.9%
(v) Use Lundberg, [1]
[1]
(vi) Can reduce by holding higher initial surplus / reinsurance / higher premium
loading etc.
[2]
[Total 13]
This question was generally well answered, although candidates lost marks when not clearly
showing their steps in parts (ii) & (iii).
