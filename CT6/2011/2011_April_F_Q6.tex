6
The double exponential distribution with parameter \lambda  > 0 has density given by
g(x) = 1⁄2\lambda e – \lambda |x| x ∈ \ .
(i) Construct an algorithm for generating samples from this distribution.

(ii) Construct an algorithm for producing samples from a N(0,1) distribution using
samples from the double exponential distribution and the acceptance-rejection
method.
[6]
[Total 9]



Page 4%%%%%%%%%%%%%%%%%%%%%%%%%%%%%%%%%%%5 — Examiners’ Report, April 2011
6
(i)
We first note that the distribution function for this double exponential is given
by
⎧
0 . 5 e \lambda  x
x < 0
G ( x ) = ⎨
− \lambda  x
⎩ 0 . 5 ( 1 − e ) + 0 . 5 x ≥ 0
And so the inverse function is given by
log 2 u
⎧
⎪
\lambda 
G − 1 ( u ) = ⎨
log( 2 ( 1 − u ))
⎪ −
\lambda 
⎩
u < 0 . 5
u ≥ 0 . 5
And so our algorithm is:
(ii)
(A) Generate u from U (0,1)
(B) If u < 0.5 set x =
log 2 u
log( 2 ( 1 − u ))
otherwise set x = −
\lambda 
\lambda 
We must first find M = Sup
f ( x )
where f ( x ) is the pdf of the N (0,1)
g ( x )
distribution.
Sup
f ( x )
2
e
= Sup
g ( x )
\lambda  2 π
−
x 2
+\lambda  | x |
2
And using the symmetry around 0 we can concentrate on positive values of x
x 2
− +\lambda  x
f ( x )
2
Sup
e 2
= Sup
g ( x )
\lambda  2 π
And the exponential expression is maximised when −
x 2
+ \lambda  x is maximised.
2
Differentiating, this occurs when − \lambda  + x = 0 i.e. x = \lambda 
f ( x )
2
=
e
Hence M = Sup
g ( x ) \lambda  2 π
x 2
\lambda  2
2 .
So define
\lambda  2
− + \lambda  x −
f ( x )
2
h ( x ) =
= e 2
Mg ( x )
Page 5%%%%%%%%%%%%%%%%%%%%%%%%%%%%%%%%%%%5 — Examiners’ Report, April 2011
So algorithm is as follows:
(A)
(B)
(C)
Generate x as in part (i)
Generate u from U(0,1)
If u ≤ h(x) then set y = x otherwise return to (A)
This question was poorly answered. Only the strongest candidates treated the
modulus in the pdf correctly. There was also difficulty deriving both parts of the
inverse function for both the u ≥ 0.5 and the u < 0.5 case.
