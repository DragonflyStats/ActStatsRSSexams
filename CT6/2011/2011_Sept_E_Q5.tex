[Total 5]5
An insurance company covers pedigree cats against the costs of medical treatment.
The cost of claims from a policy in a year is assumed to have a normal distribution
with mean μ (which varies from policy to policy) and known variance 25 2 . It is
assumed that μ = \alpha  + \betax where \alpha  and \beta are fixed constants and x is the age of the cat.
You are given the following data for the pairs (y i , x i ) for i = 1, 2, ..., 50 where y i is the
cost of claims last year for the ith policy and x i is the age of the corresponding cat.
50 50 50 50
i = 1 i = 1 i = 1 i = 1
\sum  x i = 637 \sum  y i = 5, 492 \sum  y i x i = 74,532 \sum  x i 2 = 8,312
Calculate the maximum likelihood estimates of \alpha  and \beta.


%%%%%%%%%%%%%%%%%%%%%%%%%%%%%%%%%%

5
The likelihood is given by
50
l = C \times  ∏
⎛ y −μ ⎞
− 1⁄2 ⎜ i i ⎟
e ⎝ 25 ⎠
2
i = 1
So the log-likelihood is given by
L = log l = D −
= D −
1 50
( y i − \alpha  − \beta x i ) 2
\sum 
1250 i = 1
50
50
50
50
⎞
1 ⎛ 50 2
2
2
y
y
x
x
y
x i 2 ⎟
2
2
50
2
−
\alpha 
+
\alpha \beta
+
\alpha 
−
\beta
+
\beta
⎜ ⎜ \sum  i
\sum 
\sum 
\sum 
\sum 
i
i
i i
⎟
1250 ⎝ i = 1
i = 1
i = 1
i = 1
i = 1
⎠
We can ignore the factor of 1,250.
50
50
\frac{\partial}{\partial} L
= 2 \sum  y i − 2 \beta \sum  x i − 100 \alpha  = 10,984 − 1, 274 \beta − 100 \alpha 
\frac{\partial}{\partial}\alpha 
i = 1
i = 1
50
50
50
\frac{\partial}{\partial} L
= − 2 \alpha  \sum  x i + 2 \sum  x i y i − 2 \beta \sum  x i 2 = 149, 064 − 1, 274 \alpha  − 16, 624 \beta
\frac{\partial}{\partial}\beta
i = 1
i = 1
i = 1
Setting both partial derivatives to zero and solving:
100\alpha  + 1,274\beta = 10,984
(AA)
− 1,274\alpha  − 16,624\beta = −149,064 (BB)
(AA) \times  12.74 + (BB) gives −393.24\beta = −9,127.84 so that \beta = 23.212
And so \alpha  = 0.01(10,984 − 1274 \times  23.212) = −185.88
This requires some calculations to produce the mle estimates and only the stronger
candidates were able to carry the algebra through to the end. Alternatively, solutions for
and could also be obtained using the least-squares linear regression expressions given in
the tables. This approach gave full credit provided it was accompanied by an explanation of
why it produces the same estimates.
