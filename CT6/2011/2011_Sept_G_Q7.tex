%%%%%%%%%%%%%%%%%%%%%%%%%%%%
%%- Question 7

A portfolio of insurance policies contains two types of risk. Type I risks make up
80% of claims and give rise to loss amounts which follow a normal distribution with
mean 100 and variance 400. Type II risks give rise to loss amounts which are
normally distributed with mean 115 and variance 900.
\item (i) Calculate the mean and variance of the loss amount for a randomly chosen
claim.

\item (ii) Explain whether the loss amount for a randomly chosen claim follows a
normal distribution.

The insurance company has in place an excess of loss reinsurance arrangement with
retention 130.
\item (iii)
\item (iv)
8
Calculate the probability that a randomly chosen claim from the portfolio
results in a payment by the reinsurer.

Calculate the proportion of claims involving the reinsurer that arise from Type
II risks.


%%%%%%%%%%%%%%%%%%%%%%%%%%%%%%%%%%%%%%%

7
(i)
Let the loss amount be X. Then
E(X) = 0.8 \times  100 + 0.2 \times  115 = 103
E(X 2 ) = 0.8 \times  (100 2 + 400) + 0.2 (115 2 + 900) = 11,145
Var(X) = E(X 2 ) − E(X) 2 = 11,145 – 103 2 = 536
(ii) No, the loss distribution is not Normal. To see this, note that (for example) the
pdf of the combined distribution will have local maxima at both 100 and 115.
[Consider the case where the variances are very small to see this]
(iii) Pr(X > 130) = 0.8 \times  Pr(N(100, 20 2 ) > 130) + 0.2 \times  Pr(N(115,30 2 ) > 130)
130 − 100 ⎞
130 − 115 ⎞
⎛
= 0.8 \times  Pr ⎛ ⎜ N (0,1) >
⎟ + 0.2 \times  Pr ⎜ N (0,1) >
⎟ Pr X
20
30
⎝
⎠
⎝
⎠
130
0.8 Pr N 100, 20
130
0.2 Pr N 115, 30
130
Page 6%%%%%%%%%%%%%%%%%%%%%%%%%%%%, September 2011
= 0.8 \times  Pr(N(0,1) > 1.5) + 0.2 \times  Pr(N(0,1) > 0.5)
= 0.8 \times  (1 − 0.93319) + 0.2 \times  (1 − 0.69146)
= 0.115156
(iv)
The relevant proportion is given by:
0.2 \times  (1 − 0.69146)
= 53.6%
0.115156
Many weaker candidates struggled with this question, with a large number incorrectly
asserting the loss distribution was Normal in part (ii).
