
\documentclass[a4paper,12pt]{article}

%%%%%%%%%%%%%%%%%%%%%%%%%%%%%%%%%%%%%%%%%%%%%%%%%%%%%%%%%%%%%%%%%%%%%%%%%%%%%%%%%%%%%%%%%%%%%%%%%%%%%%%%%%%%%%%%%%%%%%%%%%%%%%%%%%%%%%%%%%%%%%%%%%%%%%%%%%%%%%%%%%%%%%%%%%%%%%%%%%%%%%%%%%%%%%%%%%%%%%%%%%%%%%%%%%%%%%%%%%%%%%%%%%%%%%%%%%%%%%%%%%%%%%%%%%%%

\usepackage{eurosym}
\usepackage{vmargin}
\usepackage{amsmath}
\usepackage{graphics}
\usepackage{epsfig}
\usepackage{enumerate}
\usepackage{multicol}
\usepackage{subfigure}
\usepackage{fancyhdr}
\usepackage{listings}
\usepackage{framed}
\usepackage{graphicx}
\usepackage{amsmath}
\usepackage{chngpage}

%\usepackage{bigints}
\usepackage{vmargin}

% left top textwidth textheight headheight

% headsep footheight footskip

\setmargins{2.0cm}{2.5cm}{16 cm}{22cm}{0.5cm}{0cm}{1cm}{1cm}

\renewcommand{\baselinestretch}{1.3}

\setcounter{MaxMatrixCols}{10}

\begin{document}
Claims on a portfolio of insurance policies arise as a Poisson process with parameter
\lambda . Individual claim amounts are taken from a distribution X and we define m i = E(X i )
for i = 1, 2, .... The insurance company calculates premiums using a premium
loading of \theta .
(i) Define the adjustment coefficient R.
(ii) (a)
Show that R can be approximated by

2 \theta  m 1
by truncating the series
m 2
expansion of M X (t).
(b)
Show that there is another approximation to R which is a solution of
the equation m 3 y 2 + 3m 2 y − 6\theta m 1 = 0.
[6]
Now suppose that X has an exponential distribution with mean 10 and that \theta  = 0.3.
(iii)
Calculate the approximations to R in (ii) and (iii) and compare them to the true
value of R.
[6]
[Total 13]
CT6 A2011—410


%%%%%%%%%%%%%%%%%%%%%%%%%%%%%%%%%%%%%%%%%%%%%%%%%%%%%%%%%%%%%%%%%%%%%%%%%%%%%%%%%%%%%%%%%%%%%%%%%%%%%%%%


Page 10%%%%%%%%%%%%%%%%%%%%%%%%%%%%%%%%%%%5 — Examiners’ Report, April 2011
10
(i)
Denote the insurers profits by Z
Under A:
Premium income = 200 \times  40 \times  1.4 = 11200
Expected claims = 200 \times  40 = 8000
So E(Z) = 11200 − 8000 = 3200
Under B
We need first to calculate the expected loss for the insurer. Denote the
insurer’s loss by X. Then
E(X) =
60
\int  0
∞
0.025 xe − 0.025 x dx + 60 \times  \int  0.025 e − 0.025 x dx
= [ − xe − 0.025 x ] 60
0 + \int 
60
60 − 0.025 x
∞
e
dx + 60 \times  [ − e − 0.025 x ] 60
0
= − 60 e − 1.5 + [ − 40 e − 0.025 x ] 0 60 + 60 e − 1.5
= 40 − 40e−1.5 = 31.07479359
So the expected loss for the re-insurer is 40 − 31.07479359 = 8.925206406
Premium income = 11200 − 200 \times  1.55 \times  8.925206406 = 8433.186014
Expected claims = 200 \times  31.07479359 = 6214.958718
So E(Z) = 8433.186014 − 6214.958718 = 2218.227
Under C
Premium income = 200 \times  40 \times  1.4 − 200 \times  40 \times  0.25 \times  1.45 = 8300
Expected claims = 200 \times  40 \times  0.75 = 6000
So E(Z) = 8300 − 6000 = 2300
Page 11%%%%%%%%%%%%%%%%%%%%%%%%%%%%%%%%%%%5 — Examiners’ Report, April 2011
(ii)
We now need to find the variance of the total claim amount paid by the
insurer. Denote this by Y. Then
Under A
Var(Y) = 200Var(X) + 200E(X)2
= 200 \times  402 + 200 \times  402 = 640,000 = 8002
So
9200 − 8000 ⎞
⎛
Pr(Z < 2000) = Pr(Y > 9200) = Pr ⎜ N (0,1) >
⎟
800
⎝
⎠
= Pr(N(0,1) > 1.5) = (1 − 0.93319) = 0.06681
Under B
We first need to find E(X2) as defined above.
E(X2) =
60
\int  0
∞
0.025 x 2 e − 0.025 x dx + 60 2 \int  0.025 e − 0.025 x dx
60
= [ − x 2 e − 0.025 x ] 60
0 + \int 
= − 3600 e − 1.5 +
=
60
0
2 xe − 0.025 x dx + 3600 e − 1.5
60
2
0.025 xe − 0.025 x dx + 3600 e − 1.5
\int 
0
0.025
2
( E ( X ) − 60 e − 1.5 ) = 1414.958718
0.025
And so
Var(X) = 1414.958718 – 31.07479359 2 = 449.3159219
And therefore
Var(Y) = 200Var(X) + 200E(X)2
= 200 \times  449.3159219 + 200 \times  31.074793592 = 282991.7438
Finally
Pr(Z < 2000) = Pr(Y > 6433.186014)
6433 . 186014 − 6214 . 958718 ⎞
⎛
= Pr ⎜ N ( 0 , 1 ) >
⎟
531 . 97
⎝
⎠
Page 12%%%%%%%%%%%%%%%%%%%%%%%%%%%%%%%%%%%5 — Examiners’ Report, April 2011
= Pr(N(0,1) > 0.41023) = 1 − 0.65918 = 0.34082
Under C
Var(Y) = 200Var(X) + 200E(X)2
Var(Y) = 200 \times  0.752 \times  402 + 200 \times  (0.75 \times  30)2 = 360000 = 6002
So
6300 − 6000 ⎞
⎛
Pr(Z < 2000) = Pr(Y > 6300) = Pr ⎜ N (0,1) >
⎟
600
⎝
⎠
= Pr(N(0,1) > 0.5) = (1 − 0.69146) = 0.30854
This question received a wide range of quality of answers. Most candidates calculated
A and C correctly, but many failed to produce a reasonable answer for B. Common
errors for A and C included not re-calculating the variance and using the wrong
claim amount when calculating the probability. Some candidates were unable to
calculate the normal distribution probability correctly after deriving the correct claim
and variance values. For arrangement B many struggled to evaluate the integral
correctly.
