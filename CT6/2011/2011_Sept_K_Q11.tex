
%%%%%%%%%%%%%%%%%%%%%%%%%%%%
11
Five years ago, an insurance company began to issue insurance policies covering
medical expenses for dogs. The insurance company classifies dogs into three risk
categories: large pedigree (category 1), small pedigree (category 2) and non-pedigree
(category 3). The number of claims n ij in the ith category in the jth year is assumed to
have a Poisson distribution with unknown parameter \theta  i . Data on the number of
claims in each category over the last 5 years is set out as follows:
Category 1
Category 2
Category 3
1 2 Year
3 4 5 \sum  j = 1 n ij \sum  j = 1 n ij 2
30
37
26 43
49
31 49
58
18 58
52
37 60
64
32 240
260
144 12,144
13,934
4,354
5
5
Prior beliefs about \theta  1 are given by a gamma distribution with mean 50 and variance
25.
\item (i) Find the Bayes estimate of \theta  1 under quadratic loss. 
\item (ii) Calculate the expected claims for year 6 of each category under the
assumptions of Empirical Bayes Credibility Theory Model 1 [6]
\item (iii) Explain the main differences between the approach in (i) and that in  (ii). 
\item (iv) Explain why the assumption of a Poisson distribution with a constant
parameter may not be appropriate and describe how each approach might be
generalised.

[Total 17]
END OF PAPER
CT6 S2011—6

%%%%%%%%%%%%%%%%%%%%%%%%%%%%%%%%%%%%%%%%%%%%%%%%%%%%%%%%%%%%%%%%%%%%%%%%%%%%%%%%%%%%%%%%%%%%%%%%%%%5
11
(i)
We need to find the parameters of the Gamma distribution, say \alpha  and \lambda . Then
\alpha 
E ( X )
\lambda  = \lambda  = 50 = 2
=
\alpha 
Var ( X )
25
\lambda  2
And hence \alpha  = E(X) \times  \lambda  = 50 \times  2 = 100
The posterior distribution is given by:
f(\theta  1 |x) ∝f (x|\theta  1 ) f(\theta  1 )
⎛ 5
n ⎞
∝ ⎜ ∏ e −\theta  1 \theta  1 1 j ⎟ \times  \theta  1 \alpha − 1 e −\lambda \theta  1
⎜ j = 1
⎟
⎝
⎠
\alpha + \sum  j = 1 n 1 j − 1
5
∝ e − ( \lambda + 5) \theta  1 \theta  1
Which is the pdf of a gamma distribution with parameters
\alpha  +
\sum  j = 1 n 1 j
5
= 100 + 240 = 340 and \lambda  + 5 = 7.
Under quadratic loss the Bayes estimate is the mean of the posterior
340
distribution. So we have an estimate of
= 48.57.
7
(ii)
We have n 1 =
240
260
144
= 48 and n 2 =
= 52 and n 3 =
= 28.8.
5
5
5
This gives n =
5
\sum  ( n 1 j − n 1 ) 2
j = 1
=
48 + 52 + 28.8
= 42.9333
3
\sum  j = 1 n 1 j − 2 \sum  j = 1 n 1 j \times  n 1 + 5 \times  n 1 2
5
5
= 12,144 − 2 \times  240 \times  48 + 5 \times  48 2 = 624
Similarly
5
\sum  ( n 2 j − n 2 ) 2 = 13,934 − 2 \times  260 \times  52 + 5 \times  52 2 = 414
j = 1
5
\sum  ( n 3 j − n 3 ) 2 = 4,354 − 2 \times  144 \times  28.8 + 5 \times  28.8 2 = 206.8
j = 1
Page 12%%%%%%%%%%%%%%%%%%%%%%%%%%%%, September 2011
So
E ( s 2 ( \theta  )) =
1 \times  1 (624
3 4
+ 414 + 206.8) = 103.733
and
Var( m ( \theta  )) =
1 ((48
2
− 42.9333) 2 + (52 − 42.9333) 2 + (28.8 − 42.9333) 2 )
−
So Z =
1 \times 
5
103.7333 = 133.06667
5
= 0.86512
103.733
5 +
133.06667
So expected claims for next year are:
Cat 1 0.13488 \times  42.9333 + 0.86512 \times  48 = 47.32
Cat 2 0.13488 \times  42.9333 + 0.86512 \times  52 = 50.78
Cat 3 0.13488 \times  42.9333 + 0.86512 \times  28.8 = 30.71
This question contained a minor typographical error in the summary statistics. Based on the
5
figures given in the question a direct calculation of
\sum  n 1 2 j gives the correct figure 12,114
j = 1
and not 12,144 which is given in the question. Candidates who used 12114 will have
produced slightly different results as follows:
5
\sum  ( n 1 j − n 1 )
2
= 594
j = 1
E ( s 2 ( \theta  )) = 101.2333
Var ( S ( \theta  ) ) = 133.5666
Z = 0.86837
And the final three figures will change from 47.32, 50.78, 30.71 to 47.33, 50.81 and 30.66
respectively. Candidates producing these figures scored full marks.
(iii)
The main differences are that:
• The approach under (i) makes use of prior information about the
distribution of \theta  1 whereas the approach in (ii) does not.
• The approach under (i) uses only the information from the first category to
produce a posterior estimate, whereas the approach under (ii) assumes that
information from the other categories can give some information about
category 1.
Page 13%%%%%%%%%%%%%%%%%%%%%%%%%%%%, September 2011
•
(iv)
The approach under (i) makes precise distributional assumptions about the number of claims (i.e. that they are Poisson distributed) whereas the
approach under (ii) makes no such assumptions.
The insurance policies were newly introduced 5 years ago, and it is therefore
likely that the volume of policies written has increased (or at least not been
constant) over time. The assumption that the number of claims has a Poisson
distribution with a fixed mean is therefore unlikely to be accurate, as one
would expect the mean number of claims to be proportional to the number of
policies.
Let P ij be the number of policies in force for risk i in year j . Then the models
can be amended as follows:
The approach in (i) can be taken assuming that that the mean number of claims
in the Poisson distribution is P ij \theta  i .
The approach in (ii) can be generalised by using EBCT Model 2 which
explicitly incorporates an adjustment for the volume of risk.
This long question was answered well generally. A bit of care was needed in the final two
parts where only the better candidates were able to give a full discussion of the assumptions
underlying the models and how the models could be amended.
END OF EXAMINERS’ REPORT
Page 14
