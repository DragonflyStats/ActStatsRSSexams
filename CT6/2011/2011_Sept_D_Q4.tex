
Claims on a portfolio of insurance policies follow a Poisson process with parameter \lambda .
The insurance company calculates premiums using a premium loading of \theta  and has an
initial surplus of U.
\item (i) Define the surplus process U(t). 
\item (ii) Define the probabilities \psi (U, t) and \psi (U). 
\item (iii) Explain how \psi (U, t) and \psi (U) depend on \lambda .
CT6 S2011—2



%%%%%%%%%%%%%%%%%%%%%%%%%%%%%%%%%%%%%%%%%%%%%%%%%%%%%%%%%%%%%%%%%%%%%%%%%%%%%%%%%%%%%%%%%%%%%%%


4
(i)
Let S(t) denote the total claims up to time t and suppose individual claim
amounts follow a distribution X.
Then U(t) = U + \lambda t(1 + \theta ) E(X) − S(t).
(ii)
\psi (U, t) = Pr(U(s) < 0 for some s ∈ [0, t])
\psi (U) = Pr(U(t) < 0 for some t > 0)
(iii)
The probability of ruin by time t will increase as \lambda  increases. This is because
claims and premiums arrive at a faster rate, so that if ruin occurs it will occur
earlier, which leads to an increase in \psi (U, t).
The probability of ultimate ruin does not depend on how quickly the claims
arrive. We are not interested in the time when ruin occurs as we are looking
over an infinite time horizon.
This is another standard theory question. Many candidates lost marks by not specifying the
probabilities carefully enough in part (ii) – for example \psi  (U) = Pr(U(t) < 0) does not fully
specify the probability since no information is given about t.
Page 4%%%%%%%%%%%%%%%%%%%%%%%%%%%%, September 2011
