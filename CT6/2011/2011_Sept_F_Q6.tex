
\documentclass[a4paper,12pt]{article}

%%%%%%%%%%%%%%%%%%%%%%%%%%%%%%%%%%%%%%%%%%%%%%%%%%%%%%%%%%%%%%%%%%%%%%%%%%%%%%%%%%%%%%%%%%%%%%%%%%%%%%%%%%%%%%%%%%%%%%%%%%%%%%%%%%%%%%%%%%%%%%%%%%%%%%%%%%%%%%%%%%%%%%%%%%%%%%%%%%%%%%%%%%%%%%%%%%%%%%%%%%%%%%%%%%%%%%%%%%%%%%%%%%%%%%%%%%%%%%%%%%%%%%%%%%%%

\usepackage{eurosym}
\usepackage{vmargin}
\usepackage{amsmath}
\usepackage{graphics}
\usepackage{epsfig}
\usepackage{enumerate}
\usepackage{multicol}
\usepackage{subfigure}
\usepackage{fancyhdr}
\usepackage{listings}
\usepackage{framed}
\usepackage{graphicx}
\usepackage{amsmath}
\usepackage{chngpage}

%\usepackage{bigints}
\usepackage{vmargin}

% left top textwidth textheight headheight

% headsep footheight footskip

\setmargins{2.0cm}{2.5cm}{16 cm}{22cm}{0.5cm}{0cm}{1cm}{1cm}

\renewcommand{\baselinestretch}{1.3}

\setcounter{MaxMatrixCols}{10}

\begin{document}6
[6]
Let X 1 and X 2 be two independent exponentially distributed random variables with
parameters \lambda  1 and \lambda  2 respectively. The random variable X is related to X 1 and X 2
such that a single observation from X is chosen from X 1 with probability p and from
X 2 with probability 1 − p.
\item (i)
Show that the density function of X is
pf 1 (x) + (1 − p) f 2 (x).
where f i (x) is the density function of X i . 
\item (ii) Construct an algorithm for generating samples from X. 
\item (iii) Describe how the algorithm in \item (ii) could be generalised for k independent
components p 1 f 1 (x) + ...+ p k f k (x) where p 1 + ... + p k =1, each p i ≥ 0 and f i (x)
is the density of an exponential distribution with parameter \lambda  i .

[Total 8]
CT6 S2011—3

%%%%%%%%%%%%%%%%%%%%%%%%%%%%%%%%%%%%%%%%%%%%%%%%%%%

6
(i)
F X ( x ) = P ( X \leq  x ) = P ( X = X 1 ∩ X 1 \leq  x ) + P ( X = X 2 ∩ X 2 \leq  x )
= pF 1 ( x ) + (1 − p ) F 2 ( x )
and so f X ( x ) = F X ′ ( x ) = pF 1 ′ ( x ) + (1 − p ) F 2 ′ ( x ) = pf 1 ( x ) + (1 − p ) f 2 ( x )
(ii)
We need to combine an algorithm for determining whether to sample from X 1
or X 2 with an algorithm for generating a sample from the appropriate
exponential distribution.
Page 5%%%%%%%%%%%%%%%%%%%%%%%%%%%%, September 2011
If u is generated from a U(0,1) distribution then F i − 1 ( u ) is exponentially
log ( 1 − u )
distributed with mean 1 . But F i ( x ) = 1 − e −\lambda  i x so that F i − 1 = −
\lambda  i
\lambda 
i
So the algorithm is as follows:
(iii)
(A) Generate u 1 and u 2 from U(0,1)
(B) If u 1 < p then set i = 1 otherwise set i = 2.
(C) Set x = x = −
log1 − u 2 log(1 − u 2 )
−
\lambda  i
\lambda  i
The algorithm will be as follows:
(A) Generate u 1 and u u 2 from U(0,1)
(B) Set q 0 = 0, q j = p 1 + ... + p j for j = 1, 2, ..., k
(C) If q j−1 \leq  u 1 < q j then set i = j.
(D) Set x = −
log1 − u 2
log(1 − u 2 )
x =−
\lambda  i
\lambda  i
A number of candidates struggled to generate the correct algorithm. Some attempted to use
the inversion method in parts (ii) and (iii) but the method shown above is much easier.

