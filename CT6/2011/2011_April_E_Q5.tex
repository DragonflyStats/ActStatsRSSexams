CT6 A2011—25
The number of claims under an insurance policy in a year is either 0 (with probability
40\%) or 1 (with probability 20\%) or 2 (with probability 40\%). Individual claim
amounts are equally likely to be 50 or 20. The insurance company calculates
premiums using a premium loading of 50\% and is considering operating one of the
following arrangements:
(A) Making no changes.
(B) Introducing a policy excess of 10 (per claim) in return for a reduction of 5 in
premiums.
(C) Effecting an individual excess of loss reinsurance arrangement with retention
30 for a premium of 10.
Construct a table of the insurance company’s profits under all the possible outcomes
for each of (A) (B) and (C) and hence determine the optimal arrangement using the
Bayes criteria.
[8]

%%%%%%%%%%%%%%%%%%%%%%%%%%%%%%%%%%%%%%%%%%%%%%%%


5
Premiums charged to the policyholder are
1.5 \times  (0.2 \times  (0.5 \times  50 + 0.5 \times 20) + 0.4 \times  (0.25 \times  100 + 0.5 \times 70 + 0.25 \times  40)) = 52.5
The completed table is
Arrangement Net Premium
52.5
A
47.5
B
42.5
C
None
0
0
0
1 L
50
40
30
Claims
1 S 1S 1L
20
70
10
50
20
50
2 S
40
20
40
2 L
100
80
60
So the completed table of profits for the insurer is:
Probability
A
B
C
None
0.4
52.5
47.5
42.5
1 L
0.1
2.5
7.5
12.5
Profit
1 S
1S 1L
0.1
0.2
32.5 −17.5
37.5 −2.5
22.5 −7.5
2 S
2 L
0.1
0.1
12.5 −47.5
27.5 −32.5
2.5 −17.5
E(P)
17.5
22.5
17.5
So the insurer should introduce the policy excess (arrangement B )
This question was answered well. Some candidates gave incorrect completed tables
without showing any working, and the examiners were therefore unable to give
partial credit in these cases.
