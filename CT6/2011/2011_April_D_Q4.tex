
\documentclass[a4paper,12pt]{article}

%%%%%%%%%%%%%%%%%%%%%%%%%%%%%%%%%%%%%%%%%%%%%%%%%%%%%%%%%%%%%%%%%%%%%%%%%%%%%%%%%%%%%%%%%%%%%%%%%%%%%%%%%%%%%%%%%%%%%%%%%%%%%%%%%%%%%%%%%%%%%%%%%%%%%%%%%%%%%%%%%%%%%%%%%%%%%%%%%%%%%%%%%%%%%%%%%%%%%%%%%%%%%%%%%%%%%%%%%%%%%%%%%%%%%%%%%%%%%%%%%%%%%%%%%%%%

\usepackage{eurosym}
\usepackage{vmargin}
\usepackage{amsmath}
\usepackage{graphics}
\usepackage{epsfig}
\usepackage{enumerate}
\usepackage{multicol}
\usepackage{subfigure}
\usepackage{fancyhdr}
\usepackage{listings}
\usepackage{framed}
\usepackage{graphicx}
\usepackage{amsmath}
\usepackage{chngpage}

%\usepackage{bigints}
\usepackage{vmargin}

% left top textwidth textheight headheight

% headsep footheight footskip

\setmargins{2.0cm}{2.5cm}{16 cm}{22cm}{0.5cm}{0cm}{1cm}{1cm}

\renewcommand{\baselinestretch}{1.3}

\setcounter{MaxMatrixCols}{10}

\begin{document}
4
(i) Show that the posterior distribution of \theta  given y 1 is of the same form as the
prior distribution, specifying the parameters involved.

(ii) Write down the posterior distribution of \theta  given y 1 , ..., y n .

[Total 6]
The annual number of claims on an insurance policy within a certain portfolio follows
a Poisson distribution with mean \mu . The parameter \mu  varies from policy to policy and
can be considered as a random variable that follows an exponential distribution with
mean 1 .
\lambda 
Find the unconditional distribution of the annual number of claims on a randomly
chosen policy from the portfolio.
[6]

%%%%%%%%%%%%%%%%%%%%%%%%%%%%%%%%%%%%%%%%%%%%%%%%%%%%%%%%%%%%%%%%%%%%%%%%%%%%%%%%%%%%%%%%%%%%%%%%%%%%%%%%%%%5


4
Let the annual number of claims be denoted by N . Then
P ( N = k ) =
=
\int^{\infty} 
∞ −\mu 
e
0
∞
\int  0
P ( N = k |\mu ) f (\mu ) d \mu 
\mu  k −\lambda \mu 
\lambda  e d \mu 
k !
\lambda  ∞ k − (1 +\lambda  ) \mu 
\mu  e
d \mu 
k ! \int  0
\lambda  \Gamma  ( k + 1) ∞ (1 + \lambda  ) k + 1 k − (1 +\lambda  ) \mu 
= ×
\mu  e
d \mu 
k ! (1 + \lambda  ) k + 1 \int  0 \Gamma  ( k + 1)
=
=
\lambda 
(1 + \lambda  ) k + 1
× 1
Page 3%%%%%%%%%%%%%%%%%%%%%%%%%%%%%%%%%%%5 — Examiners’ Report, April 2011
Where the final integral is 1 since the integrand is the pdf of a Gamma distribution.
So
P ( N = k ) =
\lambda 
(1 + \lambda  )
k + 1
=
\lambda 
1
, for k = 0, 1, 2, ...
×
1 + \lambda  (1 + \lambda  ) k
Which means that N has a geometric distribution with parameter p =
\lambda 
. This is
1 + \lambda 
equivalent to a Type II negative binomial with k=1


%% This was the worst answered question on this paper, with few candidates able to write down the first integral. Candidates who attempted the algebra often did not recognise the resultant distribution.


\end{document}
