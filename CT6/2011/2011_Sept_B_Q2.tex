
[Total 4]
An accountant is using a psychic octopus to predict the outcome of tosses of a fair
coin. He claims that the octopus has a probability p > 0.5 of successfully predicting
the outcome of any given coin toss. His actuarial colleague is extremely sceptical and
summarises his prior beliefs about p as follows: there is an 80% chance that p = 0.5
and a 20% chance that p is uniformly distributed on the interval [0.5,1]. The octopus
successfully predicts the results of 7 out of 8 coin tosses.
Calculate the posterior probability that p = 0.5.

Page 2%%%%%%%%%%%%%%%%%%%%%%%%%%%%, September 2011
2
By Bayes theorem
Pr( p = 0.5 | X = 7) =
And
1
∫ 0.5
Pr( X = 7 | p = 0.5) \times  Pr( p = 0.5)
Pr( X = 7 | p = 0.5) \times  Pr( p = 0.5) + ∫
f ( x ) Pr( X = 7 | p = x ) dx = ∫
1
0.5
1
0.5
f ( x ) Pr( X = 7 | p = x ) dx
0.4 \times  8 \times  x 7 (1 − x ) dx
9 ⎤ 1
⎡ x 8 x
= 3.2 ⎢ − ⎥
⎣ ⎢ 8 9 ⎥ ⎦ 0.5
⎛ 0.5 8 0.5 9 ⎞
⎛ 1 1 ⎞
= 3.2 ⎜ − ⎟ − 3.2 ⎜
−
⎟
⎜ 8
9 ⎟ ⎠
⎝ 8 9 ⎠
⎝
1 1
0.5 0.5
7
∫ f ( x ) Pr( X = 7 p = x ) dx = ∫ 0.4 \times  8 \times  x ( 1 − x ) dx
= 0.043576389
And so
0.8 \times  8 \times  0.5 8
0.025
Pr ( p = 0.5| X ) =
=
= 0.364557
8
0.8 \times  8 \times  0.5 + 0.043576389 0.025 + 0.043576389
Many candidates struggled to apply Bayes’ theorem, and many of those that did struggled
with the mixed prior distribution. Candidates found this one of the harder questions on the
paper.
