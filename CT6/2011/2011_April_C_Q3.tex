
\documentclass[a4paper,12pt]{article}

%%%%%%%%%%%%%%%%%%%%%%%%%%%%%%%%%%%%%%%%%%%%%%%%%%%%%%%%%%%%%%%%%%%%%%%%%%%%%%%%%%%%%%%%%%%%%%%%%%%%%%%%%%%%%%%%%%%%%%%%%%%%%%%%%%%%%%%%%%%%%%%%%%%%%%%%%%%%%%%%%%%%%%%%%%%%%%%%%%%%%%%%%%%%%%%%%%%%%%%%%%%%%%%%%%%%%%%%%%%%%%%%%%%%%%%%%%%%%%%%%%%%%%%%%%%%

\usepackage{eurosym}
\usepackage{vmargin}
\usepackage{amsmath}
\usepackage{graphics}
\usepackage{epsfig}
\usepackage{enumerate}
\usepackage{multicol}
\usepackage{subfigure}
\usepackage{fancyhdr}
\usepackage{listings}
\usepackage{framed}
\usepackage{graphicx}
\usepackage{amsmath}
\usepackage{chngpage}

%\usepackage{bigints}
\usepackage{vmargin}

% left top textwidth textheight headheight

% headsep footheight footskip

\setmargins{2.0cm}{2.5cm}{16 cm}{22cm}{0.5cm}{0cm}{1cm}{1cm}

\renewcommand{\baselinestretch}{1.3}

\setcounter{MaxMatrixCols}{10}

\begin{document}

3
(i) Give a brief interpretation of E[s 2 (\theta )] and V[m(\theta ))] under the assumptions of
Empirical Bayes Credibility Theory Model 1.

(ii) Explain how the value of the credibility factor Z depends on E[s 2 (\theta )] and
V[m(\theta )].

[Total 5]
Let y 1 , ..., y n be samples from a uniform distribution on the interval [0, \theta ] where \theta  > 0
is an unknown constant. Prior beliefs about \theta  are given by a distribution with density
⎪ ⎧\alpha\beta \alpha  \theta  − (1 +\alpha  )
f ( \theta  ) = ⎨
⎩ 0
\theta >\beta 
otherwise
where \alpha  and\beta are positive constants.

%%%%%%%%%%%%%%%%%%%%%%%%%%%%%%%%%%%%%%%%%%%%%%%%%%%%%%%%%%%5

3
First note that f ( y 1|\theta ) =
(i)
1
for \theta  > y 1
\theta 
f (\theta | y 1) \propto  f ( y 1|\theta ) f (\theta )
=
1
\alpha \beta \alpha  \theta −(1+\alpha ) for \theta  >\beta and \theta  > y 1
\theta 
\propto  \theta −(2+\alpha ) for \theta  > max(\beta , y 1)
\propto  (\alpha  + 1)\beta \alpha + 1 \theta −(1+1+\alpha ) for \theta  >\beta where\beta = max(\beta , y 1)
Which is of the same form with parameters \alpha  + 1 and\beta .

%%%%%%%%%%%%%%%%%%%%%%%%%%%%%%%%%%%%%%%%%%%%%%%%%%%%%%%%%%%%%
Alternatively, we can derive this formally as:
f ( y 1 \theta  ) f ( \theta  )
f ( \theta  y 1 ) =
\int  f ( y 1 \theta  ) f ( \theta  ) d \theta 
\theta 
This gives:
f ( \theta  y 1 ) =
(ii)
1
\theta 
× \alpha\beta \alpha  \theta  − ( 1 + \alpha  )
∞
\int \beta \alpha \beta 
\alpha 
\theta  − \alpha  d \theta 
=
\alpha\beta \alpha  \theta  − ( 2 + \alpha  )
= ( \alpha  + 1 )\beta \alpha  + 1 \theta  − ( 2 + \alpha  )
− 1
\alpha 
\alpha  + 1 \beta 
%%%%%%%%%%%%%%%%%%%%%%%%%%%%%%%%%%%%%%%%%%%%%%%%%%%%%%%%%%%%%%%%%%%%%%%%%%%%%%
The posterior distribution has the same form with parameters \alpha  + n and
max(\beta , y 1, ..., yn ).
This question received a wide range of quality of answers. Only the strongest
candidates stated the correct range for the posterior in part (i). A number of
candidates assumed a sample of size n in part (i) and therefore failed to differentiate
between parts (i) and (ii).
\end{document}
