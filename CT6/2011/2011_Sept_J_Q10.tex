10

[Total 14]
The table below shows cumulative claims paid on a portfolio of motor insurance
policies.
Accident Year
2007
2008
2009
2010
0
120
140
135
138
Development Year
1
2
134
146
180
185
149
3
148
All claims are fully run off by the end of development year 3.
\item (i)
Calculate the total reserve for outstanding claims using the basic chain ladder
technique.
[7]
An actuarial student suggests an alternative approach to projecting the claims as
follows:
• For each of development years 1 to 3 calculate the observed development factor
separately for each accident year.
• Then project claims assuming the development factor for a given year is the
maximum of the observed development factors for the relevant accident year.
• For example for the development factor from development year 1 to development
year 2 we can observe actual factors for accident years 2007 and 2008. To project
claims, we assume that the development factor for development year 1 to
development year 2 is the maximum of the two observed factors.
\item (ii)
\item (iii)
Calculate the increase in the reserve for outstanding claims if claims are
projected in this way.
Discuss why the method in \item (ii) may not be appropriate.
CT6 S2011—5


%%%%%%%%%%%%%%%%%%%%%%%%%%%%%%%%%%%%%%%%%%%%%%%%%%%%%%%%%%%%%%%%%%%%%%%%%%%%%%%%%

10
(i)
The development factors are:
r 0,1 = 134 + 180 + 149
463
=
= 1.172151899
120 + 140 + 135
395
r 1,2 = 146 + 185
331
=
= 1.054140127
134 + 180
314
r 2,3 = 148
= 1.01369863
146
The ultimate claims are therefore:
For AY2008: 185 \times  1.01369863 = 187.53
Page 10%%%%%%%%%%%%%%%%%%%%%%%%%%%%, September 2011
For AY2009: 149 \times  1.05414027 \times  1.01369863 = 159.22
For AY2010: 138 \times  1.172151899 \times 1.054140127 \times  1.01369863 = 172.85
So the outstanding claim reserve is
187.53 + 159.22 + 172.85 − 185 − 149 − 138 = 47.60
(ii)
The individual development factors are as follows:
Accident Year
2007
2008
2009
Max
0 to 1
1.1167
1.2857
1.1037
1.2857
Development Factor
1 to 2
1.0896
1.0278 2 to 3
1.0137
1.0896 1.0137
The ultimate claims are therefore:
For AY2008: 185 \times  1.0137 = 187.53
For AY2009: 149 \times  1.0896 \times  1.0137 = 164.57
For AY2010: 138 \times  1.2857 \times  1.0896 \times  1.0137 = 195.97
So the outstanding claim reserve is
187.53 + 164.57 + 195.97 − 185 − 149 − 138 = 76.07
This is an increase of 28.47 which is 59.8% higher.
(iii)
Selecting the maximum DF in each column increases the reserves by 60%.
Better to take a weighted average of each column as per usual chain ladder
approach, UNLESS we know something in particular why we should give full
credence to the 1.286 factor (which is much larger than the other two factors
in column 2/1) and the 1.09 factor (which is much larger than the 1.028 factor
in column 3/2)
This question was well answered. Some candidates dropped marks in part (iii).
Page 11%%%%%%%%%%%%%%%%%%%%%%%%%%%%, September 2011
