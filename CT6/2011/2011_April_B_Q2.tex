
\documentclass[a4paper,12pt]{article}

%%%%%%%%%%%%%%%%%%%%%%%%%%%%%%%%%%%%%%%%%%%%%%%%%%%%%%%%%%%%%%%%%%%%%%%%%%%%%%%%%%%%%%%%%%%%%%%%%%%%%%%%%%%%%%%%%%%%%%%%%%%%%%%%%%%%%%%%%%%%%%%%%%%%%%%%%%%%%%%%%%%%%%%%%%%%%%%%%%%%%%%%%%%%%%%%%%%%%%%%%%%%%%%%%%%%%%%%%%%%%%%%%%%%%%%%%%%%%%%%%%%%%%%%%%%%

\usepackage{eurosym}
\usepackage{vmargin}
\usepackage{amsmath}
\usepackage{graphics}
\usepackage{epsfig}
\usepackage{enumerate}
\usepackage{multicol}
\usepackage{subfigure}
\usepackage{fancyhdr}
\usepackage{listings}
\usepackage{framed}
\usepackage{graphicx}
\usepackage{amsmath}
\usepackage{chngpage}

%\usepackage{bigints}
\usepackage{vmargin}

% left top textwidth textheight headheight

% headsep footheight footskip

\setmargins{2.0cm}{2.5cm}{16 cm}{22cm}{0.5cm}{0cm}{1cm}{1cm}

\renewcommand{\baselinestretch}{1.3}

\setcounter{MaxMatrixCols}{10}

\begin{document}



2 An insurance company has collected data for the number of claims arising from
certain risks over the last 10 years. The number of claims in the jth year from the ith
risk is denoted by X_{ij} for i = 1, 2, ..., n and j = 1, 2, ..., 10. The distribution of X_{ij} for
j = 1 2, ..., 10 depends on an unknown parameter \theta  i and given \theta  i the X_{ij} are
independent identically distributed random variables.

%%%%%%%%%%%%%%%%%%%%%%%%%%%%%%%%%%%%%


2
(i)
E ( s 2 (\theta )) represents the average variability of claim numbers from year to year
for a single risk.
V ( m (\theta )) represents the variability of the average claim numbers for different
risks i.e. the variability of the means from risk to risk.
(ii)
The credibility factor is given by
n
Z =
n +
E ( s 2 ( \theta  ))
V ( m ( \theta  ))
We can see that it is the relative values of E ( s 2(\theta )) and V ( m (\theta )) that matter. In
particular, if E ( s 2(\theta )) is high relative to V ( m (\theta )), this means that there is more
variability from year to year than from risk to risk. More credibility can be
placed on the data from other risks leading to a lower value of Z .
On the other hand, if V ( m (\theta )) is relatively higher this means there is greater
variation from risk to risk, so that we can place less reliance on the data as a
whole leading to a higher value of Z .
This question was generally well answered. Weaker students were not able to
give clear, concise descriptions of the quantities in part (i).
