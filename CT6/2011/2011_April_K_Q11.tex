
\documentclass[a4paper,12pt]{article}

%%%%%%%%%%%%%%%%%%%%%%%%%%%%%%%%%%%%%%%%%%%%%%%%%%%%%%%%%%%%%%%%%%%%%%%%%%%%%%%%%%%%%%%%%%%%%%%%%%%%%%%%%%%%%%%%%%%%%%%%%%%%%%%%%%%%%%%%%%%%%%%%%%%%%%%%%%%%%%%%%%%%%%%%%%%%%%%%%%%%%%%%%%%%%%%%%%%%%%%%%%%%%%%%%%%%%%%%%%%%%%%%%%%%%%%%%%%%%%%%%%%%%%%%%%%%

\usepackage{eurosym}
\usepackage{vmargin}
\usepackage{amsmath}
\usepackage{graphics}
\usepackage{epsfig}
\usepackage{enumerate}
\usepackage{multicol}
\usepackage{subfigure}
\usepackage{fancyhdr}
\usepackage{listings}
\usepackage{framed}
\usepackage{graphicx}
\usepackage{amsmath}
\usepackage{chngpage}

%\usepackage{bigints}
\usepackage{vmargin}

% left top textwidth textheight headheight

% headsep footheight footskip

\setmargins{2.0cm}{2.5cm}{16 cm}{22cm}{0.5cm}{0cm}{1cm}{1cm}

\renewcommand{\baselinestretch}{1.3}

\setcounter{MaxMatrixCols}{10}

\begin{document}11
The number of claims on a portfolio of insurance policies has a Poisson distribution
with mean 200. Individual claim amounts are exponentially distributed with mean 40.
The insurance company calculates premiums using a premium loading of 40% and is
considering entering into one of the following re-insurance arrangements:
(A) No reinsurance.
(B) Individual excess of loss insurance with retention 60 with a reinsurance
company that calculates premiums using a premium loading of 55%.
(C) Proportional reinsurance with retention 75% with a reinsurance company that
calculates premiums using a premium loading of 45%.
(i) Find the insurance company’s expected profit under each arrangement.
(ii) Find the probability that the insurer makes a profit of less than 2000 under
each of the arrangements using a normal approximation.
[8]
[Total 14]
[6]
The table below shows cumulative claims paid on a portfolio of insurance policies.
Accident Year
2007
2008
2009
2010
0
240
260
270
276
Development Year
1
2
281.4
302
320
322
312.9
3
305
All claims are fully run off by the end of development year 3.
(i)
Calculate the total reserve for outstanding claims using the basic chain ladder
technique.
[7]
An actuary is considering modelling the future claims assuming that individual
development factors are lognormally distributed with the following parameters:
Parameter
\mu 
\sigma
(ii)
(iii)
0 to 1
0.171251
0.032148
Development Year
1 to 2
0.035850
0.045606
2 to 3
0.008787
0.046853
Show that under these assumptions the cumulative development factor to
ultimate is also lognormally distributed.

Calculate a 99% upper confidence limit for the outstanding claims relating to
the 2010 accident year.
[5]
[Total 15]
END OF PAPER
CT6 A2011—5


%%%%%%%%%%%%%%%%%%%%%%%%%%%%%%%%%%%%%%%%%%%%%%%%%%%%%%

11
(i)
The development factors are:
r0,1 = 281.4 + 320 + 312.9 914.3
=
= 1.187403
240 + 260 + 270
770
r1,2 = 302 + 322
624
=
= 1.037579
281.4 + 320 601.4
r2,3 = 305
= 1.009934
302
And the ultimate claims are:
For AY 2008: 322 \times  1.009934 = 325.20
For AY 2009: 312.9 \times  1.037579 \times  1.009934 = 327.88
For AY 2010: 276 \times  1.187403 \times  1.037579 \times  1.009934 = 343.42
So outstanding claims reserve is
325.20 + 327.88 + 343.42 − 322 − 312.9 − 276 = 85.60
Page 13%%%%%%%%%%%%%%%%%%%%%%%%%%%%%%%%%%%5 — Examiners’ Report, April 2011
(ii)
Suppose that Ri,i+1 ~ log N(\mu i, \sigma i 2 ) for i = 0, 1, 2.
Then log Ri,i+1 ~ N(\mu i, \sigma i 2 ).
So (for example)
log Ri,i+1 Ri+1,i+2 = log Ri,i+1 + log Ri+1,i+2 ~ N(\mu i + \mu i+1, \sigma i 2 + \sigma i 2 + 1 )
Which means that the product Ri,i+1 Ri+1,i+2 is also log-normally distributed.

Since any product of log-normally distributed development factors is also lognormally distributed the development factors to ultimate must also be log-
normally distributed.
(iii)
Using the results from (ii) the development factor to ultimate for AY 2010 is
log-normally distributed with parameters:
\[\mu  = 0.171251 + 0.035850 + 0.008787 = 0.215889\]
\[\sigma^{2} = 0.0321482 + 0.0456062 + 0.0468532 = 0.0728602\]
So an upper 99\% confidence limit for the development factor to ultimate is given by exp(0.215889 + 0.07286 \times  2.3263) = 1.47018
So an upper 99\% confidence limit for total claims is
1.47018 \times  276 = 405.77
So an upper 99\% confidence limit for outstanding claims is
405.77 − 276 = 129.77
%%%%%%%%%%%%%%%%%%%%%%%%%%%%%%%%%%%%%%%%%%%%%%%%%%%%%%%%%%%%%%%%
\newpage
Part (i) was answered very well by the majority of candidates. Parts (ii) and (iii) were answered poorly. Many candidates failed to produce sufficient detail in part (ii) for instance calculating the parameters of the distribution rather than explaining why it was a log-normal. For part (iii) few candidates were able to calculate the parameters
of the distribution correctly. A further common mistake was to calculate a two-sided confidence interval.

\end{document}
