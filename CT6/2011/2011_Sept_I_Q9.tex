Claim events on a portfolio of insurance policies follow a Poisson process with
parameter \lambda . Individual claim amounts follow a distribution X with density
f(x) = 0.01 2 xe −0.01x
x > 0.
The insurance company calculates premiums using a premium loading of 45%.
\item (i)
Derive the moment generating function M X (t).
CT6 S2011—4
\item (ii) Determine the adjustment coefficient and hence derive an upper bound on the
probability of ruin if the insurance company has initial surplus U.

\item (iii) Find the surplus required to ensure the probability of ruin is less than 1% using
the upper bound in \item (ii).

Suppose instead that individual claims are for a fixed amount of 200.
\item (iv)
Determine whether the adjustment coefficient is higher or lower than in \item (ii)
and comment on your conclusion.

\end{enumerate}



Page 8%%%%%%%%%%%%%%%%%%%%%%%%%%%%, September 2011
9
(i)
M X (t) = E(e tX ) =
=
∞
∫ 0 0.01
2
∞ tx
∫ 0 e
f ( x ) dx
xe ( t − 0.01) x dx
∞
⎡ 0.01 2 xe ( t − 0.01) x ⎤
∞ 0.01 2 e ( t − 0.01) x
dx
−
= ⎢
⎥
∫
0
t
0.01
t
0.01
−
−
⎣ ⎢
⎦ ⎥ 0
∞
⎡ 0.01 2 e ( t − 0.01) x ⎤
= 0 − 0 − ⎢
provided that t < 0.01
2 ⎥
⎣ ⎢ ( t − 0.01) ⎦ ⎥ 0
=
(ii)
0.01 2
( t − 0.01) 2
again provided that t < 0.01
The adjustment coefficient is the unique positive solution of
M X (R) = 1 + 1.45E(X)R
But E(X) = M ′ X (0) =
=
− 2 \times  0.01 2
( t − 0.01) 3
=
t = 0
d ⎡ 0.01 2 ⎤
⎢
⎥
dt ⎣ ⎢ ( t − 0.01) 2 ⎦ ⎥
t = 0
− 2
= 200
− 0.01
So we need to solve
0.01 2
( R − 0.01) 2
= 1 + 290R
i.e. 0.01 2 = (1 + 290R) (R − 0.01) 2 = (1 + 290R) (0.01 2 − 0.02R + R 2 )
i.e. 0.012 = 0.01 2 + 0.029R − 0.02R − 5.8R 2 + R 2 + 290R 3
i.e. 290R 2 − 4.8R + 0.009 = 0
R =
4.8 \pm  4.8 2 − 4 \times  290 \times  0.009
2 \times  290
i.e. R = 0.00215578 or R = 0.0143959
So taking the smaller root we have R = 0.00215578 since that is less than 0.01
Page 9%%%%%%%%%%%%%%%%%%%%%%%%%%%%, September 2011
The upper bound for the probability of ruin is given by Lundberg’s inequality
as
\psi (U) \leq  e −RU = e −0.00215578U
(iii)
We want \psi (U) \leq  e −0.00215578U \leq  0.01
i.e. −0.00215578U \leq  log 0.01
log 0.01
i.e. U ≥
= 2136.20
− 0.00215578
(iv)
This time the adjustment coefficient is the solution to:
e 200R = 1 + 290R
So the question is whether y = e 200R crosses the line y = 1 + 290R before or
after y = 0.01 2 (0.01 − R) −2 crosses the same line
But when R = 0.00215578 we have
e 200R = e 200\times 0.00215578 = 1.539 < 1 + 290R = 1.625.
So y = e 200R has not yet crossed the given line, and the second scenario has a
larger adjustment coefficient that the first.
This means the second risk has a lower probability of ruin, which is to be
expected since although the mean claim amounts are the same in each
scenario, the claim amounts in the first scenario are more variable suggesting a
greater risk.
This was found one of the more challenging questions on the paper. In part (i), the final
expression could be quoted from the tables but for full marks candidates had to show it from
the definitions. Special care is needed here in calculations as decimal places of R can affect
the final figures.
