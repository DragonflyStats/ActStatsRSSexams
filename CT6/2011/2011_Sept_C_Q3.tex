
3

Loss amounts under a class of insurance policies follow an exponential distribution
with mean 100. The insurance company wishes to enter into an individual excess of
loss reinsurance arrangement with retention level M set such that 8 out of 10 claims
will not involve the reinsurer.
\item (i)
Find the retention M.

For a given claim, let X I denote the amount paid by the insurer and X R the amount
paid by the reinsurer.
\item (ii)
4
Calculate E(X I ) and E(X R ).

[Total 5]



%%%%%%%%%%%%%%%%%%%%%%%%%%%%%%%%%%%%%%%%%%%%%%%%%%%%%%%%%%%%%%%%%%%%%%%%%%%%
3
(i)
We must solve
M
∫ 0
0.01 e − 0.01 x dx = 0.8
[ − e − 0.01 x ] 0 M = 0.8
1 − e − 0.01 M = 0.8
M =
log 0.2
= 160.9437912
− 0.01
Page 3%%%%%%%%%%%%%%%%%%%%%%%%%%%%, September 2011
(ii)
We have
M
E ( X I ) = ∫ 0.01 x e − 0.01 x dx + MP ( X > M )
0
= [ − xe − 0.01 x ] 0 M + ∫
M − 0.01 x
e
dx + Me − 0.01 M
0
M
= − Me
− 0.01 M
⎡ − e − 0.01 x ⎤
− 0.01 M
+ ⎢
⎥ + Me
0.01
⎣ ⎢
⎦ ⎥ 0
= − 100 e − 0.01 M + 100
= − 100 e − 1.6094 + 100 = 80
And hence E(X R ) = E(X) − E(X I ) = 100 − 80 = 20
This standard question was generally well answered. Alternatively, one could calculate
E(X R ) first and then apply E(X I ) = E(X) − E(X R ).


