9
∞
\int  −∞ f ( y , \theta  + \phi  t , \phi  ) dy = 1.
(ii) Show that E(Y) = b ′ ( \theta  ) and Var(Y) = \phi  b ′′ ( \theta  ) using the result in (i).
(iii) Verify that the result in (i) holds if Y has a Poisson distribution.


[Total 11]

%%%%%%%%%%%%%%%%%%%%%%%%%%%%%%%%%%%%%%%%%%%%%%%%%%%%%%%%%%%%%%%%%%%%%%%%%%%%%%%%%%%%%%%%55

9
(i)
The adjustment coefficient is the unique positive solution to
\lambda MX(R) − \lambda  − \lambda (1 + \theta ) E(X) R = 0
(ii)
Cancelling the \lambda  terms we have
(a)
MX(R) = E(eRX) = 1 + (1 + \theta ) E(X)R
⎛
⎞
R 2 X 2
E ⎜ 1 + RX +
+ ... ⎟ = 1 + (1 + \theta ) E(X)R
⎜
⎟
2
⎝
⎠
And truncating the expression we get
E(1 + RX + R2X2/2) = 1 + (1 + \theta ) E(X)R
i.e. 1 + Rm1 + R2m2/2 = 1 + (1 + \theta ) m1R
i.e. R2m2 = 2\theta m1R
i.e. R =
(b)
2 \theta  m 1
m 2
Once more we have
⎞
⎛
R 2 X 2 R 3 X 3
⎜
E ⎜ 1 + RX +
+
+ .... ⎟ ⎟ = 1 + (1 + \theta ) E(X)R
2 !
3 !
⎠
⎝
And truncating the expression we get
⎛
R 2 X 2 R 3 X 3 ⎞
E ⎜ 1 + RX +
+
⎟ = 1 + (1 + \theta ) E(X)R
⎜
2
6 ⎟ ⎠
⎝
i.e. 1 + Rm 1 +
R 2 m 2 R 3 m 3
+
= 1 + (1 + \theta )m1R
2
6
i.e. 3R2m2 + R3m3 = 6\theta m1R
i.e. m3R2 + 3Rm2 − 6\theta m1 = 0
As required
Page 9%%%%%%%%%%%%%%%%%%%%%%%%%%%%%%%%%%%5 — Examiners’ Report, April 2011
(iii)
In this case m1 = 10 and m2 = 200 and m3 = 6000
So the estimate from (ii) (a) is R =
2 \theta  m 1 2 \times  0.3 \times  10
6
=
= 0.03
=
200
200
m 2
The estimate from (ii) (b) is the solution to 6000R2 + 600R − 18 = 0
Which is given by
− b ± b 2 − 4 ac
− 600 + 600 2 + 4 \times  6000 \times  18
=
= 0.024161984
2 a
12000
[The negative root of the equation is −0.12416]
The true value of R is given by the solution to
MX(R) = E(eRX) = 1 + (1 + \theta ) E(X)R
\mu 
(1 + \theta  ) R
= 1 +
where \mu  = 1 is the parameter of the
10
\mu − R
\mu 
exponential distribution.
That is
And so
\mu 2 = \mu (\mu  − R) + (1 + \theta ) R(\mu  − R)
\mu 2 = \mu 2 − \mu R + \mu R − R2 + \mu \theta R − \theta R2
0 = \mu \theta  − R(1 + \theta )
R =
0.1 \times  0.3
\mu \theta 
= 0.0230769
=
1.3
1 + \theta 
So the first estimate gives a greater error than the second (the error is 30% for
the first approximation and only about 4.7% for the second). This is as we
would expect since we took more terms before truncating.
Candidates generally answered part (i) and part (ii) well, although part (ii) b) caused
problems and many candidates did not give sufficient detail. In (iii) many candidates
produced an answer wrongly using a denominator of 100, or calculated the estimate but then
did not explain the difference in estimates adequately.
