
\documentclass[a4paper,12pt]{article}

%%%%%%%%%%%%%%%%%%%%%%%%%%%%%%%%%%%%%%%%%%%%%%%%%%%%%%%%%%%%%%%%%%%%%%%%%%%%%%%%%%%%%%%%%%%%%%%%%%%%%%%%%%%%%%%%%%%%%%%%%%%%%%%%%%%%%%%%%%%%%%%%%%%%%%%%%%%%%%%%%%%%%%%%%%%%%%%%%%%%%%%%%%%%%%%%%%%%%%%%%%%%%%%%%%%%%%%%%%%%%%%%%%%%%%%%%%%%%%%%%%%%%%%%%%%%

\usepackage{eurosym}
\usepackage{vmargin}
\usepackage{amsmath}
\usepackage{graphics}
\usepackage{epsfig}
\usepackage{enumerate}
\usepackage{multicol}
\usepackage{subfigure}
\usepackage{fancyhdr}
\usepackage{listings}
\usepackage{framed}
\usepackage{graphicx}
\usepackage{amsmath}
\usepackage{chngpage}

%\usepackage{bigints}
\usepackage{vmargin}

% left top textwidth textheight headheight

% headsep footheight footskip

\setmargins{2.0cm}{2.5cm}{16 cm}{22cm}{0.5cm}{0cm}{1cm}{1cm}

\renewcommand{\baselinestretch}{1.3}

\setcounter{MaxMatrixCols}{10}

\begin{document}
7

Consider the time series
\[Y t = 0.7 + 0.4Y t− 1 + 0.12Y t− 2 + e t\]
where e t is a white noise process with variance \sigma 2 .
\begin{enumerate}
\item (i) Identify the model as an ARIMA(p,d,q) process. 
\item (ii) Determine whether Y t is a stationary process. 
\item (iii) Calculate E(Y t ). 
\item (iv) Calculate the auto-correlations \rho  1 , \rho  2 , \rho  3 and \rho  4 .
\end{enumerate}
%%%%%%%%%%%%%%%%%%%%%%%%%%%%%%%%%%%%%%%%%%%%%%%%%%%%%%%%%%%%%%%%%%%%%%%%%%%%%%%%%%%%%%%%%%%%%%%%%%%%%%

7
(i) The model is ARIMA(2,0,0) provided that the model is stationary.
(ii) The lag polynomial is 1 − 0.4L − 0.12L2 = (1 − 0.6L)(1 + 0.2L)
Since the roots
1
0.6
1 are both greater than one in absolute value the
and − 0.2
process is stationary.
(iii)
Since the process is stationary we know that E(Yt) is equal to some constant \mu 
independent of t.
Taking expectations on both sides of the equation defining Yt gives
E(Yt) = 0.7 + 0.4E(Yt−1) + 0.12E(Yt−2)
\mu  = 0.7 + 0.4\mu  + 0.12\mu 
\mu  =
(iv)
0.7
= 1.45833333
1 − 0.4 − 0.12
The auto-covariance function is not affected by the constant term of 0.7 in the
equation, and this term can be ignored.
The Yule-Walker equations are
γ0 = 0.4γ1 + 0.12γ2 + \sigma^{2}
γ1 = 0.4γ0 + 0.12γ1 (A)
γ2 = 0.4γ1 + 0.12γ0 (B)
γs = 0.4γs−1 + 0.12γs−2 for s > 2
Dividing both sides of (A) by γ0 and noting that \rho  s =
\rho 1 = 0.4 + 0.12\rho 1 so that \rho 1 =
Page 6
0.4
= 0.45454545
0.88
γ s
we have
γ 0%%%%%%%%%%%%%%%%%%%%%%%%%%%%%%%%%%%5 — Examiners’ Report, April 2011
Substituting this result into (B) we have
\rho 2 = 0.4 \times  0.45454545 + 0.12 = 0.30181818
And using the final result we have
\rho 3 = 0.4 \times  0.3018181818 + 0.12 \times  0.45454545 = 0.1752727
and
\rho 4 = 0.4 \times  0.1752727 + 0.12 \times  0.30181818 = 0.10632727
Expressed as fractions:
5 \rho  =
\rho  1 = 11
2
83
275
241 \rho  =
\rho  3 = 1375
4
731
6875
This straightforward question was answered well.
\end{document}
