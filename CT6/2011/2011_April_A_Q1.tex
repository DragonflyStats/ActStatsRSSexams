
\documentclass[a4paper,12pt]{article}

%%%%%%%%%%%%%%%%%%%%%%%%%%%%%%%%%%%%%%%%%%%%%%%%%%%%%%%%%%%%%%%%%%%%%%%%%%%%%%%%%%%%%%%%%%%%%%%%%%%%%%%%%%%%%%%%%%%%%%%%%%%%%%%%%%%%%%%%%%%%%%%%%%%%%%%%%%%%%%%%%%%%%%%%%%%%%%%%%%%%%%%%%%%%%%%%%%%%%%%%%%%%%%%%%%%%%%%%%%%%%%%%%%%%%%%%%%%%%%%%%%%%%%%%%%%%

\usepackage{eurosym}
\usepackage{vmargin}
\usepackage{amsmath}
\usepackage{graphics}
\usepackage{epsfig}
\usepackage{enumerate}
\usepackage{multicol}
\usepackage{subfigure}
\usepackage{fancyhdr}
\usepackage{listings}
\usepackage{framed}
\usepackage{graphicx}
\usepackage{amsmath}
\usepackage{chngpage}

%\usepackage{bigints}
\usepackage{vmargin}

% left top textwidth textheight headheight

% headsep footheight footskip

\setmargins{2.0cm}{2.5cm}{16 cm}{22cm}{0.5cm}{0cm}{1cm}{1cm}

\renewcommand{\baselinestretch}{1.3}

\setcounter{MaxMatrixCols}{10}

\begin{document}
1 Give two examples of exercises where Monte-Carlo simulation should be performed
using the same choice of random numbers, explaining your reasoning in each case. 
%%%%%%%%%%%%%%%%%%%%%%%%%%%%%%%5
1
Example 1: Testing sensitivity to parameter variation – we want the results to change
as a result of changes to the parameter not as a result of variations in the random
numbers.
Example 2: Performance evaluation. When comparing two or more schemes which
might be adopted we want differences in results to arise from differences between the
schemes rather than as a result of variations in the random numbers.
Alternative Example: The same set of simulations could be used for the numerical
evaluation of derivatives
\theta  ( \alpha  + \delta  ) − \theta  ( \alpha  )
\theta  ' ( \alpha  ) =
\delta 
This question was generally answered well, although weaker candidates explained
how to obtain random numbers or perform Monte-Carlo simulation instead of
explaining why you would want to use the same random numbers.
\end{document}
