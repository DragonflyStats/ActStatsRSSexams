
1 Give two examples of exercises where Monte-Carlo simulation should be performed
using the same choice of random numbers, explaining your reasoning in each case. 
%%%%%%%%%%%%%%%%%%%%%%%%%%%%%%%5
1
Example 1: Testing sensitivity to parameter variation – we want the results to change as a result of changes to the parameter not as a result of variations in the random
numbers.

Example 2: Performance evaluation. When comparing two or more schemes which might be adopted we want differences in results to arise from differences between the
schemes rather than as a result of variations in the random numbers.

Alternative Example: The same set of simulations could be used for the numerical evaluation of derivatives
\theta  ( \alpha  + \delta  ) − \theta  ( \alpha  )
\theta  ' ( \alpha  ) =
\delta 
This question was generally answered well, although weaker candidates explained how to obtain random numbers or perform Monte-Carlo simulation instead of
explaining why you would want to use the same random numbers.

%%%%%%%%%%%%%%%%%%%%%%%%%%%%%%%%%%%%%%%%%%%%%%%%%%%%%%%%%%%%%%%%%%%%%%%%%%%%%%%%%%%%%%%%%55

\end{document}
