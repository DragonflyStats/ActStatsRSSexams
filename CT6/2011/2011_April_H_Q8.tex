PLEASE TURN OVER8
Suppose that Y is a random variable belonging to a special subset of the exponential
family where the density function of Y has the form
⎡ y \theta  − b ( \theta  )
⎤
f ( y , \theta  , \phi  ) = exp ⎢
+ c ( y , \phi  ) ⎥
\phi 
⎣
⎦
For some constants \theta  and \phi  and functions b and c.
(i)
Show that the moment generating function of Y is given by
⎡ b ( \theta  + t \phi  ) − b ( \theta  ) ⎤
M Y ( t ) = exp ⎢
⎥
\phi 
⎣
⎦

Hint: Note that the function f(y, \theta  + \phi t, \phi ) is the density of another random variable of
the same family and hence

%%%%%%%%%%%%%%%%%%%%%%%%%%%%%%%%%%%%%%%%%%%%%%%%%%%%

8
(i)
MY(t) = \int  e ty f ( y , \theta  , \phi  ) dy
⎡ y \theta  − b ( \theta  )
⎤
+ c ( y , \phi  ) ⎥ dy
\phi 
⎣
⎦
= \int  exp( ty ) exp ⎢
⎡ y ( \theta  + t \phi  ) − b ( \theta  )
⎤
+ c ( y , \phi  ) ⎥ dy
\phi 
⎣
⎦
= \int  exp ⎢
⎡ b ( \theta  + t \phi  ) − b ( \theta  ) ⎤
⎡ y ( \theta  + \phi  t ) − b ( \theta  + \phi  t )
⎤
exp ⎢
+ c ( y , \phi  ) ⎥ dy
⎥
\int 
\phi 
\phi 
⎣
⎦
⎣
⎦
= exp ⎢
⎡ b ( \theta  + t \phi  ) − b ( \theta  ) ⎤
⎥ \times  1
\phi 
⎣
⎦
= exp ⎢
using the hint to evaluate the second integral.
(ii)
dM Y ( t )
d ⎛ b ( \theta  + t \phi  ) − b ( \theta  ) ⎞
=
⎜
⎟ M Y ( t )
dt
\phi 
dt ⎝
⎠
=
\phi  b ′ ( \theta  + t \phi  )
M Y ( t )
\phi 
= b ′ ( \theta  + t \phi  ) M Y ( t )
Page 7%%%%%%%%%%%%%%%%%%%%%%%%%%%%%%%%%%%5 — Examiners’ Report, April 2011
And E ( Y ) = M Y ′ (0) = b ′ ( \theta  ) M Y (0) = b ′ ( \theta  ) \times  1 = b ′ ( \theta  )
d 2 M Y ( t )
dt 2
= \phi  b ′′ ( \theta  + t \phi  ) M Y ( t ) + b ′ ( \theta  + t \phi  ) M Y ′ ( t )
So E ( Y 2) = M Y ′′ (0) = \phi  b ′′ ( \theta  ) M Y (0) + b ′ ( \theta  ) M Y ′ (0)
= \phi  b ′′ ( \theta  ) + b ′ ( \theta  ) 2
So Var( Y ) = E ( Y 2) − E ( Y )2 = \phi  b ′′ ( \theta  ) + b ′ ( \theta  ) 2 − b ′ ( \theta  ) 2 = \phi  b ′′ ( \theta  )
Credit given for alternative approaches (e.g. CGF).
(iii)
For the Poisson distribution with parameter \mu  we have
f(y, \theta , \phi ) =
\mu  y e −\mu 
= exp[y log \mu  − \mu  − log y!]
y !
Which is of the form in the question with \theta  = log \mu , \phi  = 1 and b(\theta ) = e\theta  and
c(y, \phi ) = − log y!
So the result from (i) gives
⎡ b ( \theta  + t \phi  ) − b ( \theta  ) ⎤
MY(t) = exp ⎢
⎥
\phi 
⎣
⎦
⎡ e log \mu + t − e log \mu  ⎤
= exp ⎢
⎥ = exp[\mu (et − 1)]
1
⎣ ⎢
⎦ ⎥
which is indeed the MGF of the Poisson distribution as shown on p7 of the
tables.
This question was generally well done, with many candidates who could not complete
the derivation in part (i) nevertheless able to use the result to score well in parts (ii)
and (iii). For part (iii) some candidates calculated the first two moments rather than
showing that the MGF of the Poisson distribution has the form given in part (i).
Page 8%%%%%%%%%%%%%%%%%%%%%%%%%%%%%%%%%%%5 — Examiners’ Report, April 2011
