5
[9]
For three different risks, an actuary is modelling the monthly claim numbers with
three different Poisson distributions.
Risk Exposure Number of claims
Risk 1 36 months 20
Risk 2 30 months 18
Risk 3 24 months 16
(i) Derive the maximum likelihood estimates of the parameter for each of the
three individual Poisson distributions fitted.
[5]
(ii)
 Test the hypothesis that the three risks have the same monthly claim rate. [5]
[Total 10]
CT6 A2018–3 
PLEASE TURN OVER6
Tarik and Liam are playing a zero-sum two-person game. From a deck of three cards
numbered 1, 2 and 3 Tarik selects a card, making sure Liam does not know which one
it is. Liam then proceeds to guess which card Tarik has picked. If Liam is wrong, he
continues to guess until he has guessed correctly, at which point the game ends.
After each guess, if Liam’s guess is lower than the number on Tarik’s card, Tarik
says “Low”, but if Liam’s guess is higher than the number on Tarik’s card, Tarik says
“High”.
At the end of each game, Liam pays Tarik $1 for each guess he made.
You should assume that Liam will never make a guess that contradicts the information
provided by Tarik – for example, if Liam guesses “2” first, and Tarik says “Low”,
Liam would then always guess “3”, rather than “1”.
Consider strategy A, where Liam will guess “1” first, and then guesses “2” if 1 is not
correct.
%%%%%%%%%%%%%%%%%%%%%%%%%%%%%%%%%%%%%%%%%%%%%%%%%%%%%%%%%%%%%%%%%%%%%%%%%%%%
Q5
(i)
For risk one, let x i , y i , z i be the numbers of claims in month i for risks one,
two & three respectively. Let μ I , μ II , μ III be the monthly rate for these three
risks then the likelihood function is:
log L ( μ I , μ II , μ III ) = log L ( μ I ) + log L ( μ II ) + log L ( μ III )
[1⁄2]
where
log L ( μ=
I )
36
= i 1 = i 1
30
30
36
20 log μ I − 36 μ I − ∑ log x i !
[1]
= i 1
30
i ! 18log μ II − 30 μ II − ∑ log y i !
∑ y i log μ II − ∑ μ II − ∑ log y =
= i 1
log L ( μ III =
)
36
x i !
∑ x i log μ I − ∑ μ I − ∑ log =
= i 1
30
log L ( μ II =
)
36
24
= i 1 = i 1
24
= i 1
24 24
= i 1 = i 1 = i 1
i ! 16 log μ III − 24 μ III − ∑ log z i !
∑ z i log μ III − ∑ μ III − ∑ log z =
= i 1
After differentiating and equating to zero we have
Page 5Subject CT6 (Statistical Methods Core Technical) – April 2018 – Examiners’ Report
∂ log L ( μ I , μ II , μ III )
∂μ I
=
− 0.36 +
20
μ I
[1]

Second derivative is − 20 / μ I 2 < 0
20 5
=
so μ  I =
36 9
[1]
[1⁄2]
18 3  16 2
=
=
, μ III =
Similarly we can see that μ 
II =
3 0 5
24 3
[1]
[Total 5]
(ii)
For testing whether the three models are the same we carry out the likelihood
ratio test.
We fit the same rate to the three risks using this log likelihood function
36
30
24


log L =
( μ )   ∑ x i + ∑ y i + ∑ z i   log μ − 90 μ − ∑ log x i ! − ∑ log y i ! − ∑ log z i !
= i 1 = i 1 = i 1


[1]
ˆ
=
and similar to the above the corresponding MLE is μ
54 27
=
90 45
[1⁄2]
2 ( log L ( μ I , μ II , μ III ) − log L ( μ ) )
20
18
16
54


= 2  20 log − 20 + 18log − 18 + 16 log − 16 − 54 log + 54 
36
30
24
90



 20 
 18 
 16 
 54  
= 2 *  20 * log   + 18 * log   + 16 * log   − 54 * log   
 36 
 30 
 24 
 90  

= 0.2930949
[11⁄2]
The difference in the parameters between the models is 3 – 1 = 2, therefore we
compare this test statistics against the χ 22 which at the 5% upper level has
critical value 5.991>. 0.2931 . Therefore there is insufficient evidence to reject
the hypothesis that the three risks have a common rate.
[2]
[5]
[Total 10]
Page 6Subject CT6 (Statistical Methods Core Technical) – April 2018 – Examiners’ Report
Variants of this question have been seen many times before and so
candidates were able to score very well here, although only the
strongest candidates were able to pick up full marks.
Q6
(i)
Guess 1, then guess 2 if “Low”, then guess 3 if “Low” again. (A)
Guess 1, then guess 3 if “Low”, then guess 2 if “High” (B)
Guess 2, then guess 1 if “High”, or guess 3 if “Low” (C)
Guess 3, then guess 2 if “High”, then guess 1if “High” again (D)
Guess 3, then guess 1 if “High”, then guess 2 if “Low” (E)
[1]
[1]
[1]
[1]
(ii)
Tarik \ Liam
1
2
3
A
1
2
3
B
1
3
2
C
2
1
2
D
3
2
1
E
2
3
1
[3]
(iii)
There is no saddle point.
[1]
This is because there is no element in the matrix which is both the highest in
the row and lowest in the column, and vice versa.
[1]
[Total 9]
Candidates with a good understanding of the relevant theory were able
to score very well here, although some candidates struggled to
formulate the strategies required in part (i).
