© Institute and Faculty of Actuaries1
An actuary is simulating claim sizes, X, for a particular insurance policy. X follows
an exponential distribution with varying parameter λ, where λ can take one of three
possible values.
The table shows the distribution function of this external factor, and its impact on λ.
Probability
λ
0.3 0.3 0.4
1 2 3
Set out an algorithm to generate samples from X, using the inverse transform method.
[5]
2
(i)
State the three main components of a generalised linear model.
[3]
Consider the discrete random variable Y, with the following probability density
function
⎛ n ⎞
⎟ μ ny ( 1− μ ) n−ny
f ( y, μ ) = ⎜⎜
⎟
ny
⎝
⎠
(ii)
1 2
y = 0, , ,...,1
n n
Show that Y belongs to the exponential family of distributions, specifying each
component.[4]
(iii)
State the canonical link function in this case.
	
%%%%%%%%%%%%%%%%%%%%%%%%%%%%%%%%%%%%%%%%%%%%%%%%%%%%%%%%%%%%%%%%%%%%%%%%%%%
Solutions
Q1
For a given λ, we know that one can sample from X as
1
X = − log U
U  U (0,1)
[2]
λ
Algorithm:
1. Generate U 1  U (0,1)
2.
 1

=
λ  2
 3

[1⁄2]
U 1 ∈ ( 0, 0.3 ]
U 1 ∈ ( 0.3, 0.6 ]
U 1 ∈ ( 0.6,1 ]
[1]
3. Generate U 2  U (0,1)
[1]
4. X = −
1
λ
log U 2
[1⁄2]
[Total 5]
This question was generally well answered, but a disappointing number of candidates used
the probability density function, rather than the cumulative density function.
Q2
(i)
(ii)
The distribution of the response variable.
A linear predictor as a function of the covariates.
A link function between the response variable and the linear predictor.

 n  
=
f ( y , μ ) exp  n ( y ln μ + (1 − y ) ln ( 1 − μ ) ) + ln   
 ny  

 

 n  
 μ 
= exp  n  y ln 
 + ln ( 1 − μ )  + ln  ny  
  
 1 − μ 
   

Hence
 μ 
θ = ln 

 1 − μ 
φ = n
1
a ( φ ) =
φ
(iii)
[1]
[1]
[1]
[1]
[1]
[1⁄2]
[1⁄2]
b =
( θ ) ln ( 1 + e θ ) [1⁄2]
 φ 
c ( y , φ ) = ln  
 φ y  [1⁄2]
 μ 
Link function: g ( μ ) = ln 

 1 − μ  [1]
[Total 8]
Page 3Subject CT6 (Statistical Methods Core Technical) – September 2018 – Examiners’ Report
Well prepared candidates were able to score highly here, although many candidates gave
insufficiently detailed answers in part (i), and didn’t give expressions in part (ii) as a function
of the correct variable e.g. leaving b(theta) as a function of mu.


