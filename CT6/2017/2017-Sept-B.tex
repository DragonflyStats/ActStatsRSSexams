\documentclass[a4paper,12pt]{article}

%%%%%%%%%%%%%%%%%%%%%%%%%%%%%%%%%%%%%%%%%%%%%%%%%%%%%%%%%%%%%%%%%%%%%%%%%%%%%%%%%%%%%%%%%%%%%%%%%%%%%%%%%%%%%%%%%%%%%%%%%%%%%%%%%%%%%%%%%%%%%%%%%%%%%%%%%%%%%%%%%%%%%%%%%%%%%%%%%%%%%%%%%%%%%%%%%%%%%%%%%%%%%%%%%%%%%%%%%%%%%%%%%%%%%%%%%%%%%%%%%%%%%%%%%%%%

\usepackage{eurosym}
\usepackage{vmargin}
\usepackage{amsmath}
\usepackage{graphics}
\usepackage{epsfig}
\usepackage{enumerate}
\usepackage{multicol}
\usepackage{subfigure}
\usepackage{fancyhdr}
\usepackage{listings}
\usepackage{framed}
\usepackage{graphicx}
\usepackage{amsmath}
\usepackage{chngpage}

%\usepackage{bigints}
\usepackage{vmargin}

% left top textwidth textheight headheight

% headsep footheight footskip

\setmargins{2.0cm}{2.5cm}{16 cm}{22cm}{0.5cm}{0cm}{1cm}{1cm}

\renewcommand{\baselinestretch}{1.3}

\setcounter{MaxMatrixCols}{10}

\begin{document}

\begin{enumerate}
Q3
A is false since there cannot be a claim until time 2

B is false since the insurance company could be ruined at time 3 if there is a claim, if
U is sufficiently small.

C is false since the insurance company cannot be ruined in year 4, since by that stage
it will have sufficient premiums to cover any loss.

D is true since if it is not ruined by time 4, the insurance company cannot be
ruined.

Stronger candidates were able to apply the information given in the
question to ruin theory to score well, but a number of candidates failed
to do so.
(i)
1
2
3
4
5
6
7
1
0
1
0
0
0
0
0
2
1
0
1
0
0
0
0
3
0
1
0
1
0
0
0
Shivon
4
0
0
1
0
1
0
0
5
0
0
0
1
0
1
0
Where each element represents the payoff to Shivon.
(ii)
6
0
0
0
0
1
0
1
7
0
0
0
0
0
1
0

Clearly for Shivon, picking 1 and 7 are dominated, since she can do better by
picking 3 or 5 respectively. So now A =
Q4
1
2
3
4
5
6
7
2
1
0
1
0
0
0
0
3
0
1
0
1
0
0
0
Shivon
4
0
0
1
0
1
0
0
5
0
0
0
1
0
1
0
6
0
0
0
0
1
0
1
[11⁄2]
But now for Craig, 1 dominates 3, 7 dominates 5, and 2 (or 6) dominate 4, so
now A =
Page 4Craig
 %%%%%%%%%%%%%%%%%%%%%%%%%%%%%%%%%%%%%%%%%%%%%%% – September 2017 – Examiners’ Report
1
2
6
7
2
1
0
0
0
3
0
1
0
0
Shivon
4
0
0
0
0
5
0
0
1
0
6
0
0
0
1
[11⁄2]
But now 4 is dominated for Shivon, so A =
1
2
6
7
2
1
0
0
0
3
0
1
0
0
Shivon
5
0
0
1
0
6
0
0
0
1

(iii)
Clearly by symmetry Shivon should pick 2, 3, 5 or 6 a quarter of the time, and
the value of the game to her is 1⁄4.

Candidates who were well prepared in decision theory were typically
able to score well in all parts of this question.

%%%%%%%%%%%%%%%%%%%%%%%%%
3
On 1 January 2014 an insurance company writes a policy for a European farmer. At
the end of each year, the farmer’s crop is assessed, and if it is less than 100 tonnes, it
is deemed to have failed. If the crop fails for two years in a row the insurance policy
pays out €1m and then is immediately terminated.
Premiums are €25k per month, paid in advance, and there are no expenses. This is the
only policy the insurance company writes and it has initial surplus U > 0. Denote by
Ψ(U,t) the probability of ruin by time t, measured in years; and Ψ(U) as the ultimate
probability of ruin.
Explain whether the following statements are TRUE or FALSE:
(a) 1 

Ψ ( U ,1) < Ψ  U ,1 
2 

(b) 1 
1 


Ψ  U , 2  = Ψ  U ,3 
2 
3 


(c) Ψ ( U , 3) < Ψ ( U , 4)
(d) Ψ ( U , 4) = Ψ ( U )
[8]
CT6 S2017–24
Craig and Shivon are playing a two person zero sum game.
Craig picks an integer i from 1 to n, Shivon picks an integer j from 1 to n, and Shivon
receives from Craig:
1 if i − j = 1,
0 otherwise.
(i) Set out the payoff matrix, A, in the case that n = 7. 
(ii) Show that A can be reduced to a 4 × 4 matrix by eliminating dominated
strategies. 
Craig and Shivon adopt randomised strategies.
(iii)
\end{document}
