\documentclass[a4paper,12pt]{article}

%%%%%%%%%%%%%%%%%%%%%%%%%%%%%%%%%%%%%%%%%%%%%%%%%%%%%%%%%%%%%%%%%%%%%%%%%%%%%%%%%%%%%%%%%%%%%%%%%%%%%%%%%%%%%%%%%%%%%%%%%%%%%%%%%%%%%%%%%%%%%%%%%%%%%%%%%%%%%%%%%%%%%%%%%%%%%%%%%%%%%%%%%%%%%%%%%%%%%%%%%%%%%%%%%%%%%%%%%%%%%%%%%%%%%%%%%%%%%%%%%%%%%%%%%%%%

\usepackage{eurosym}
\usepackage{vmargin}
\usepackage{amsmath}
\usepackage{graphics}
\usepackage{epsfig}
\usepackage{enumerate}
\usepackage{multicol}
\usepackage{subfigure}
\usepackage{fancyhdr}
\usepackage{listings}
\usepackage{framed}
\usepackage{graphicx}
\usepackage{amsmath}
\usepackage{chngpage}

%\usepackage{bigints}
\usepackage{vmargin}

% left top textwidth textheight headheight

% headsep footheight footskip

\setmargins{2.0cm}{2.5cm}{16 cm}{22cm}{0.5cm}{0cm}{1cm}{1cm}

\renewcommand{\baselinestretch}{1.3}

\setcounter{MaxMatrixCols}{10}

\begin{document}

\begin{enumerate}
%%%%%%%%%%%%%%%%%%%%%%%%%
9
Consider a random variable X, with probability density:
h ( x ) = 2(1 − x )

(i)
0 < x < 1

\begin{enumerate}
    \item Construct an algorithm to sample from h(x) using the inverse transform
method.

Now consider the probability density function:
f ( x ) = 6 x (1 − x ) 0 < x < 1

\item 
(ii) Construct an algorithm to sample from f(x) using the acceptance-rejection
method and h(x).

\item (iii) Explain whether the method used in part (ii) would be more efficient when
using samples from a standard Uniform distribution U(0,1) instead of h(x). 

\end{enumerate}


Q9
(i)
The cumulative distribution function for h ( x ) is:
x
F h  x    2  1  u  du  2 x  x 2

0
the equation F h  X   U is then equivalent to 2X  X 2  U
so
 X  1  2  1  U ,

1
X  1   1  U  2
[11⁄2]
so the required inversion is then obtained. Algorithm is therefore:
1. Sample u from U(0,1)
X  1    1  U 
2. Return


Since 0 < X < 1
(ii)
In this case we need to find first:
C  max
f ( x ) 6 x (1  x )

 3 x
h ( x )
2(1  x )
i.e. C = 3
[11⁄2]
The rejection algorithm is then:
(iii)
Sample x from h  x  as in (i) 
Sample u from U(0,1) 
If u < 1/3 f(x)/h(x) = x then set y = x, otherwise go to step 1. 
If we are to use the standard uniform U(0,1) as envelope, consider
C1 = max(f(x)).
C1 = f(1/2) = 3/2. (max at 6 – 12x = 0)
[11⁄2]
Therefore, since the acceptance rate is 1 / C 1 =2/3 which is greater than
Page 9 %%%%%%%%%%%%%%%%%%%%%%%%%%%%%%%%%%%%%%%%%%%%%%% – September 2017 – Examiners’ Report
C 
1
h(x) is less efficient.
3
[11⁄2]
Most candidates were able to score well here, although a number of
students failed to convert to a cumulative distribution in part (i). There
was a markedly improved performance in answers to the question on
efficiency, relative to previous diets.
Q10
i)
Since E(Yt ) = \mu Y is constant for each t, E(Xt ) = a + bt + \mu Y . 
Since this mean depends on t then the time series Xt is not stationary. 
(ii) ΔXt = b + ΔYt (and since Yt is stationary ΔXt is) 
(iii) And the covariance function is Cov(ΔXt , ΔXt–s ) = Cov(Yt – Yt–1 ,Y t–s – Yt–s–1 ) 
= Cov(Yt ,Yt–s ) + Cov(Yt–1,Yt–1–s ) – Cov(Yt ,Yt–s–1 ) – Cov(Yt–1,Yt–s ) 
=  Y (s) +  Y (s) –  Y (s + 1) –  Y (s – 1) 
= 2 Y (s) –  Y (s + 1) –  Y (s – 1) 
Where  Y (s) represents the autocovariance of Y at s.
(iv)
(v)
Yt = \varepsilon t + \beta\varepsilon t–1 , where \varepsilon t is white noise with variance \sigma 2 then
ΔXt = b + \varepsilon t + \beta\varepsilon t–1 - \varepsilon t -1 - \beta\varepsilon t–2
or
ΔXt = b + (1 – L)(1 + \betaL) \varepsilon t .


[Max 1]
In particular Y t   t   t  1 the corresponding auto-covariance function is
\gamma Y (0)  (1   2 ) \sigma 2 and \gamma Y (1)  \sigma 2 .

So from (iii) var(Δ(X)) = 2\gamma Y (0) − 2\gamma Y (1) = 2(1 + \beta 2 )\sigma 2 − 2\beta\sigma 2 =
( 1 + \beta 2 ) \sigma 2 + ( 1 − \beta ) 2 \sigma 2 > ( 1 + \beta 2 ) \sigma 2 = \gamma Y ( 0)

Most candidates were able to score well on parts (i) and (ii), and
although only stronger candidates were able to score well on parts (iii)
and (v), this question presented few problems for those well prepared
on this topic.
