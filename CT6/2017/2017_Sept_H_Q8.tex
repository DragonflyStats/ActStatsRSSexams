\documentclass[a4paper,12pt]{article}

%%%%%%%%%%%%%%%%%%%%%%%%%%%%%%%%%%%%%%%%%%%%%%%%%%%%%%%%%%%%%%%%%%%%%%%%%%%%%%%%%%%%%%%%%%%%%%%%%%%%%%%%%%%%%%%%%%%%%%%%%%%%%%%%%%%%%%%%%%%%%%%%%%%%%%%%%%%%%%%%%%%%%%%%%%%%%%%%%%%%%%%%%%%%%%%%%%%%%%%%%%%%%%%%%%%%%%%%%%%%%%%%%%%%%%%%%%%%%%%%%%%%%%%%%%%%

\usepackage{eurosym}
\usepackage{vmargin}
\usepackage{amsmath}
\usepackage{graphics}
\usepackage{epsfig}
\usepackage{enumerate}
\usepackage{multicol}
\usepackage{subfigure}
\usepackage{fancyhdr}
\usepackage{listings}
\usepackage{framed}
\usepackage{graphicx}
\usepackage{amsmath}
\usepackage{chngpage}

%\usepackage{bigints}
\usepackage{vmargin}

% left top textwidth textheight headheight

% headsep footheight footskip

\setmargins{2.0cm}{2.5cm}{16 cm}{22cm}{0.5cm}{0cm}{1cm}{1cm}

\renewcommand{\baselinestretch}{1.3}

\setcounter{MaxMatrixCols}{10}

\begin{document}


8
\begin{enumerate}
\item (i)
Describe the key difference between excess of loss and proportional
reinsurance.

A random variable X follows an Exponential distribution.
\item (ii)
Show that $E ( X | X > M ) = E ( X ) + M $.

An insurance company writes travel insurance policies, with a premium loading factor of 15\%. There are two types of claims, and a maximum of one claim per policy.
Type I claims are for a delay. Claim amounts follow a Uniform distribution with a minimum of $500 and a maximum of $1,500.
Type II claims are for a cancellation. Claim amounts are Exponentially distributed with parameter λ = 0.001.
5% of policies result in a claim, 85% of which are Type I and 15% of which are Type II.
\item (iii)
Calculate the premium charged for each policy.

The insurer is choosing between two different reinsurance policies:
Policy A: The reinsurer covers 10% of every claim, and uses a premium loading factor of 35%.
Policy B: The reinsurer covers the maximum of {0, claim amount – \$1,500}, and uses a premium loading factor of 45%.
\item (iv)
Determine the reinsurance policy the insurance company should purchase, under:
(a)
(b)
the Bayes criterion.
the minimax criterion.
\medskip 
The insurance company’s actuary, Tom, believes the minimax criterion is more
relevant in this case.
\item (v)
CT6 S2017–5
Suggest a reason for Tom’s belief.
\end{enumerate}

%%%%%%%%%%%%%%%%%%%%%%%%%%%%%%%%%%%%%%%%%%%%%%%%%%%%%%%%%%%%%%%%%%%%%%%%%%%%5
\newpage

Q8
(i)
Under excess of loss, the reinsurer covers claims above a certain retention
amount M

Under proportional reinsurance, the reinsurer covers a proportion of claims \alpha

(ii)
1
e
 M

 x  e
 x
dx
[1]
M
Let t  x  M
Then want
1
e  M
(iii)
(iv)

e  M
  t  M 
t

M

e
dt

  
e  M
0





 t
 t 
t

e
dt

M

e
dt
 
  E  X   M

 0

0


[2]
Premium = 5%*  85%*1000  1 5%*1000  *1.15  $57.50
(a)
Expected reinsurance cost per policy for A is
35%*10%*1000*5%  $1.75
[2]

For B only need to consider Type II
E  X | X  1500   2500 from part (i)

So expected cost per policy for B is
(b)
(v)
45% *15% * e  1.5 * (2500  1500) *5%  $0.75 
So B is better under the Bayes criterion 
B is clearly better under minimax, since total claims are capped [1]
Unlikely that the assumption of independence holds, since events that lead to
travel cancellations are likely to affect more than one policy holder.
[2]
The reinsurer is a malevolent opponent so minimax is appropriate.

%%%%%%%%%%%%%%%%%%%%%%%%%%%%%%%%%%%%%%%%%%%%%%%%%%%%%

The first part aside, this question proved to be the hardest on the paper,
with only the strongest candidates able to score well, particularly on
parts (iv) and (v).
\end{document}
