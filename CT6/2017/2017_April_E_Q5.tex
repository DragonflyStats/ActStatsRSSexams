\documentclass[a4paper,12pt]{article}



%%%%%%%%%%%%%%%%%%%%%%%%%%%%%%%%%%%%%%%%%%%%%%%%%%%%%%%%%%%%%%%%%%%%%%%%%%%%%%%%%%%%%%%%%%%%%%%%%%%%%%%%%%%%%%%%%%%%%%%%%%%%%%%%%%%%%%%%%%%%%%%%%%%%%%%%%%%%%%%%%%%%%%%%%%%%%%%%%%%%%%%%%%%%%%%%%%%%%%%%%%%%%%%%%%%%%%%%%%%%%%%%%%%%%%%%%%%%%%%%%%%%%%%%%%%%
  
\usepackage{eurosym}
\usepackage{vmargin}
\usepackage{amsmath}
\usepackage{graphics}
\usepackage{epsfig}

\usepackage{enumerate}

\usepackage{multicol}

\usepackage{subfigure}

\usepackage{fancyhdr}

\usepackage{listings}

\usepackage{framed}

\usepackage{graphicx}

\usepackage{amsmath}

\usepackage{chngpage}



%\usepackage{bigints}

\usepackage{vmargin}

% left top textwidth textheight headheight

% headsep footheight footskip

\setmargins{2.0cm}{2.5cm}{16 cm}{22cm}{0.5cm}{0cm}{1cm}{1cm}

\renewcommand{\baselinestretch}{1.3}

\setcounter{MaxMatrixCols}{10}



\begin{document}

\begin{enumerate}
\item %% - Question 5 
\begin{enumerate}[(i)]
\item Show that the following discrete distribution belongs to the exponential family of distributions.
( ; ) n ny (1 )n ny 0, 1 , 2 , .,1
f y y
ny n n
  −
μ =  μ −μ = …
 
\item
(ii) Derive expressions for the mean and variance of the distribution, E(y) and Var(y), using your answer to part (i). 
\end{enumerate}

%%%%%%%%%%%%%%%%%%%%%%%%%%%%%%%%%%%%%%%%%%%%%%%%%%%%%%%%%%%5
Q5 (i) From the definition
f  y,   exp  log 1 log1  log
n
n y y
ny
  
        
  
exp log log 1  log
1
n
n y
ny
       
                 

Hence
log
1
  
      
[½]
  n [½]
a   1

[½]
b  log1 e  
c( y, ) log
y



 
  
 
[½]
(ii) From the theory we know that
     '
' log 1
1
E y b e e
e


       


%%---- Subject CT6 (Statistical Methods Core Technical) – April 2017 – Examiners’ Report
Page 6
Var     ''  1 '
1
y a b e
n e


 
        
=
1 1  1
1 1
e e
n e e n
 
 
   
       
. 
[Total 8]
Most candidates were able to make some progress, but only the better
candidates were able to obtain the precise specification. Most candidates
knew the standard results from the theory for part (ii), but then struggled to
apply it in this particular case.

%%%%%%%%%%%%%%%%%%%%%%%%%%%%%%%%%%%%%%%%%%%%%%%%%%%%%%%%%%%%%%%%%
Q6 (i) The method of moments (or method of least squares) and maximum likelihood
estimation. 
(ii) The method of moments (or method of least squares) does not make any
assumptions about the distribution of t . 
(iii) The parameter estimates are obtained from Y-W equations for the AR(1)
processes. 
Namely, 1   and 2
  0 1 where 1, 0 and 1 are estimated from
the sample quantities. 
(iv) Both models above are identical as the observations from one also satisfy the
difference equation of the other. 
(v) Answer in (iv) implies that the models are equivalent so the process is
stationary regardless of the value of c. 
[Total 11]
Candidates familiar with the theory were able to score well on parts (i) to (iii).
Unfortunately there was a typo in the introduction to part (iv), since the mu was excluded, so marks were awarded generously where there was any evidence this had caused confusion in the answers to parts (iv) and (v).
\end{document}
