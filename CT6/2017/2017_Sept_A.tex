\documentclass[a4paper,12pt]{article}

%%%%%%%%%%%%%%%%%%%%%%%%%%%%%%%%%%%%%%%%%%%%%%%%%%%%%%%%%%%%%%%%%%%%%%%%%%%%%%%%%%%%%%%%%%%%%%%%%%%%%%%%%%%%%%%%%%%%%%%%%%%%%%%%%%%%%%%%%%%%%%%%%%%%%%%%%%%%%%%%%%%%%%%%%%%%%%%%%%%%%%%%%%%%%%%%%%%%%%%%%%%%%%%%%%%%%%%%%%%%%%%%%%%%%%%%%%%%%%%%%%%%%%%%%%%%

\usepackage{eurosym}
\usepackage{vmargin}
\usepackage{amsmath}
\usepackage{graphics}
\usepackage{epsfig}
\usepackage{enumerate}
\usepackage{multicol}
\usepackage{subfigure}
\usepackage{fancyhdr}
\usepackage{listings}
\usepackage{framed}
\usepackage{graphicx}
\usepackage{amsmath}
\usepackage{chngpage}

%\usepackage{bigints}
\usepackage{vmargin}

% left top textwidth textheight headheight

% headsep footheight footskip

\setmargins{2.0cm}{2.5cm}{16 cm}{22cm}{0.5cm}{0cm}{1cm}{1cm}

\renewcommand{\baselinestretch}{1.3}

\setcounter{MaxMatrixCols}{10}

\begin{document}

\begin{enumerate}
%%%%%%%%%%%%%%%%%%%%%%%%%
\item %% -- Question 11
Claim amounts on a portfolio of insurance policies follow a Weibull distribution. The median claim amount is £1,000 and 90% of claims are less than £5,000.
Estimate the parameters of the Weibull distribution, using the method of moments. [4]
\item %%--Question 2
\begin{enumerate}[(i)]
\item (i)
Explain why a sequence of pseudo-random numbers are often preferred to truly random numbers for Monte Carlo simulation.
[2]
An actuary is generating pairs of standard Normal variates using the Polar algorithm and pairs of pseudo-random variates from a U(0,1) distribution.
(ii)
\item Determine the pairs of standard Normal variates generated by the following pairs of pseudo-random variates where possible.
(a) 0.062, 0.293
(b) 0.984, 0.794
(c) 0.008, 0.961
\end{enunerate}
\end{enunerate}
\newpage
%%%%%%%%%%%%%%%%%%%%%%%%%%%%%%%%%%%%%%%%%%%%%%%%%%%%%%%%%%%%%%%%%%%%%%%%%
Q1

1  e  c 1000  0.5

[1⁄2]
1  e  c 5000  0.9 [1⁄2]
c 1000    ln 0.5 [1⁄2]
c 5000    ln 0.1 [1⁄2]
0.2  
ln 0.5
ln 0.1
[1⁄2]
  0.74594
c 
 ln 0.5
1000 0.74594
[1⁄2]
 0.004009
[1]
%----------------------------------------------------------------------------------------%
\begin{itemize}
\item Most students scored very well on this straightforward question.
\item Unfortunately there was a typographical error in the question which referred to ‘method of moments’ rather than ‘method of percentiles’; although this did not affect many candidates, full credit was given to any reasonable attempt at applying the method of moments.
\end{itemize}
%----------------------------------------------------------------------------------------%
Q2
(i)
\begin{itemize}
\item Pseudo random numbers can be reproduced 
\item Only single routine required, rather than lengthy table/hardware 
\item Difficult to generate very large set of truly random numbers
\end{itemize}
%----------------------------------------------------------------------------------------%
(ii)
(a)
[1]
[Max 2]
v 1  2*0.062  1  .876, v 2  2*0.293  1  .414,
s  v 1 2  v 2 2  0.938772
[1]
z 1  v 1 *  2ln  s  / s   .3 2 14
z 2  v 2 *  2ln  s  / s   .1 5 19 [1]
(b)  2*0.984  1  2   2*0.794  1  2  1 , so no variate [1⁄2]
(c)  2*0.008  1  2   2*0.961  1  2  1 , so no variate [1⁄2]
%----------------------------------------------------------------------------------------%
Many candidates scored well here, although a number failed to make
the adjustment to convert a U(0,1) into a U(-1,1).


