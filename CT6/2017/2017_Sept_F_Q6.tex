\documentclass[a4paper,12pt]{article}

%%%%%%%%%%%%%%%%%%%%%%%%%%%%%%%%%%%%%%%%%%%%%%%%%%%%%%%%%%%%%%%%%%%%%%%%%%%%%%%%%%%%%%%%%%%%%%%%%%%%%%%%%%%%%%%%%%%%%%%%%%%%%%%%%%%%%%%%%%%%%%%%%%%%%%%%%%%%%%%%%%%%%%%%%%%%%%%%%%%%%%%%%%%%%%%%%%%%%%%%%%%%%%%%%%%%%%%%%%%%%%%%%%%%%%%%%%%%%%%%%%%%%%%%%%%%

\usepackage{eurosym}
\usepackage{vmargin}
\usepackage{amsmath}
\usepackage{graphics}
\usepackage{epsfig}
\usepackage{enumerate}
\usepackage{multicol}
\usepackage{subfigure}
\usepackage{fancyhdr}
\usepackage{listings}
\usepackage{framed}
\usepackage{graphicx}
\usepackage{amsmath}
\usepackage{chngpage}

%\usepackage{bigints}
\usepackage{vmargin}

% left top textwidth textheight headheight

% headsep footheight footskip

\setmargins{2.0cm}{2.5cm}{16 cm}{22cm}{0.5cm}{0cm}{1cm}{1cm}

\renewcommand{\baselinestretch}{1.3}

\setcounter{MaxMatrixCols}{10}

\begin{document}



PLEASE TURN OVER6
7
\begin{enumerate}
\item (i) State the key characteristics of the individual risk model.
\item 
(ii) State all possible values for the number of claims that may arise from a given
risk over the  period being modelled.

\item (iii) Suggest an example of an insurance contract where the individual risk model
may be suitable, and an example where it is unlikely to be.

\item (iv) Describe the differences between the individual risk model and the collective
risk model.

\item (v) State the additional assumption required such that the individual risk model
can be reduced to the collective risk model.
\end{itemize}
%%%%%%%%%%%%%%%%%%%%%%%%%%%%%%%%%%%%%%%%%%%%%%%%%%%%%%%%%%%%%%%%%%%%
%%%%%%%%%%%%%%%%%
\newpage
Q6
(i)
A fixed numbers of risks are assumed within the studied portfolio 
The number of risks does not change over the period of insurance cover 
These risks are independent 
Page 6Subject CT6 %%%%%%%%%%%%%%%%%%%%%%%%%%%%%%%%%%%%%%%%%%%%%%% – September 2017 – Examiners’ Report
Claim amounts from these risks are not (necessarily) identically distributed
random variables

(ii) The number of claims arising is either 0 or 1. 
(iii) e.g. Term assurance (yes), household contents insurance (no) 
(iv) In the individual risk model the number of risks is specified and fixed; and the
number of claims from each risk is restricted. This is not the case for the
collective risk model.

Individual risks are independent in the individual risk model, whilst the
individual claims amounts are independent under the collective risk model. 
Claims in collective risk model are identical whereas the claims in each policy
in the individual risk model are not.

[Max 3]
(v)
When claim amounts are in fact identically distributed.

Well prepared candidates were able to score highly on this bookwork
question, but a number of students were unable to demonstrate
sufficient knowledge of the individual risk model.
\end{document}
