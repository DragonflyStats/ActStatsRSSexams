\documentclass[a4paper,12pt]{article}



%%%%%%%%%%%%%%%%%%%%%%%%%%%%%%%%%%%%%%%%%%%%%%%%%%%%%%%%%%%%%%%%%%%%%%%%%%%%%%%%%%%%%%%%%%%%%%%%%%%%%%%%%%%%%%%%%%%%%%%%%%%%%%%%%%%%%%%%%%%%%%%%%%%%%%%%%%%%%%%%%%%%%%%%%%%%%%%%%%%%%%%%%%%%%%%%%%%%%%%%%%%%%%%%%%%%%%%%%%%%%%%%%%%%%%%%%%%%%%%%%%%%%%%%%%%%
  
\usepackage{eurosym}
\usepackage{vmargin}
\usepackage{amsmath}
\usepackage{graphics}
\usepackage{epsfig}

\usepackage{enumerate}

\usepackage{multicol}

\usepackage{subfigure}

\usepackage{fancyhdr}

\usepackage{listings}

\usepackage{framed}

\usepackage{graphicx}

\usepackage{amsmath}

\usepackage{chngpage}



%\usepackage{bigints}

\usepackage{vmargin}

% left top textwidth textheight headheight

% headsep footheight footskip

\setmargins{2.0cm}{2.5cm}{16 cm}{22cm}{0.5cm}{0cm}{1cm}{1cm}

\renewcommand{\baselinestretch}{1.3}

\setcounter{MaxMatrixCols}{10}



\begin{document}

\begin{enumerate}
\item 
%%%%%%%%%%%%%%%%%%%%%%%%%%%%%%%%%%%%%%%%%%%%%%%%%%%%%%%%%%%5
6 Model A is a stationary AR(1) model, which follows the equation:
  yt = μ + αyt−1 + εt
where εt is a standard white noise process.
\begin{enumerate}[(i)]
\item (i) State two approaches for estimating the parameters in Model A. 
Mary, an actuarial student, wishes to revise Model A such that the error terms εt no longer follow a Normal distribution.
\item (ii) Explain which of the approaches in part (i) she should now use for parameter estimation. 
\item (iii) Propose a method by which Mary will be able to calculate estimates of the parameters α and σ2, with reference to any relevant equations.
Mary, has now constructed Model B. She has done this by multiplying both sides of the equation above by (1 − cB), where B is the backshift operator, so that Model B follows the equation:
  \[yt (1− cB) = (αyt−1 + εt )(1− cB) .\]
\item (iv) Explain why Model A and Model B are identical.
\item (v) Explain for which values of c Model B is stationary. 
\end{itemize}
\newpage
%%%%%%%%%%%%%%%%%%%%%%%%%%%%%%%%%%%%%%%%%%%%%%%%%%%%%%%%%%%5


%%%%%%%%%%%%%%%%%%%%%%%%%%%%%%%%%%%%%%%%%%%%%%%%%%%%%%%%%%%%%%%%%
Q6 
\begin{enumerate}
    \item (i) The method of moments (or method of least squares) and maximum likelihood
estimation.  
\item (ii) The method of moments (or method of least squares) does not make any
assumptions about the distribution of t .  
\item (iii) The parameter estimates are obtained from Y-W equations for the AR(1)
processes. [1]
Namely, 1   and 2
  0 1 where 1, 0 and 1 are estimated from
the sample quantities.  
\item (iv) Both models above are identical as the observations from one also satisfy the
difference equation of the other.  
\item (v) Answer in (iv) implies that the models are equivalent so the process is
stationary regardless of the value of c.  
\item 
Candidates familiar with the theory were able to score well on parts (i) to (iii).
\item Unfortunately there was a typo in the introduction to part (iv), since the mu was excluded, so marks were awarded generously where there was any evidence this had caused confusion in the answers to parts (iv) and (v).

\end{enumerate}
\end{document}
