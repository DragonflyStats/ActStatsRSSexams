\documentclass[a4paper,12pt]{article}

%%%%%%%%%%%%%%%%%%%%%%%%%%%%%%%%%%%%%%%%%%%%%%%%%%%%%%%%%%%%%%%%%%%%%%%%%%%%%%%%%%%%%%%%%%%%%%%%%%%%%%%%%%%%%%%%%%%%%%%%%%%%%%%%%%%%%%%%%%%%%%%%%%%%%%%%%%%%%%%%%%%%%%%%%%%%%%%%%%%%%%%%%%%%%%%%%%%%%%%%%%%%%%%%%%%%%%%%%%%%%%%%%%%%%%%%%%%%%%%%%%%%%%%%%%%%

\usepackage{eurosym}
\usepackage{vmargin}
\usepackage{amsmath}
\usepackage{graphics}
\usepackage{epsfig}
\usepackage{enumerate}
\usepackage{multicol}
\usepackage{subfigure}
\usepackage{fancyhdr}
\usepackage{listings}
\usepackage{framed}
\usepackage{graphicx}
\usepackage{amsmath}
\usepackage{chngpage}

%\usepackage{bigints}
\usepackage{vmargin}

% left top textwidth textheight headheight

% headsep footheight footskip

\setmargins{2.0cm}{2.5cm}{16 cm}{22cm}{0.5cm}{0cm}{1cm}{1cm}

\renewcommand{\baselinestretch}{1.3}

\setcounter{MaxMatrixCols}{10}

\begin{document}

\begin{enumerate}
%%%%%%%%%%%%%%%%%%%%%%%%%
A random variable X follows a Poisson distribution with parameter λ.
(i)
(ii)
(iii)
CT6 S2017–4
Show that the distribution of X is a member of the Exponential family of
distributions. [4]
Show that the mean of X equals the variance of X, using your answer to
part (i). [2]
Describe the three key components required when fitting a Generalised Linear
Model (GLM).
[3]


%%%%%%%%%%%%%%%%%%%%%%%%%%%%%%%%%%%%%%%%%%%%%%%%%%%%%%%%%%%%%%%%%%%%%%%%%%%%5
\newpage

Q7
(i)
The density of the Poisson distribution is then:
f  y   e

Which to the general form
 y
 e
y !
y log 
 log y !
1
g  y   e
y  b   
 c  y ,  
a   
[1]
it corresponds to
[1]
  log  , b       e  , a       1, c  y ,     log y !
[1⁄2 per element]
 
'
(ii) From the theory E  Y   b '     e    and var  Y   b      
(iii) A GLM consists of three components:


[2]
a distribution for the data (Poisson, exponential, gamma, normal or
binomial) [1]
a linear predictor (a function of the covariates that is linear in the
parameters) [1]
Page 7Subject CT6 (Statistical Methods Core Technical) – September 2017 – Examiners’ Report

a link function (that links the mean of the response variable to the linear
predictor).
[1]
Most candidates scored very well on this straightforward GLM
question.
Q8
(i)
Under excess of loss, the reinsurer covers claims above a certain retention
amount M
[1⁄2]
Under proportional reinsurance, the reinsurer covers a proportion of claims α
[1⁄2]
(ii)
1
e
 M

 x  e
 x
dx
[1]
M
Let t  x  M
Then want
1
e  M
(iii)
(iv)

e  M
  t  M 
t

M

e
dt

  
e  M
0





 t
 t 
t

e
dt

M

e
dt
 
  E  X   M

 0

0


[2]
Premium = 5%*  85%*1000  1 5%*1000  *1.15  $57.50
(a)
Expected reinsurance cost per policy for A is
35%*10%*1000*5%  $1.75
[2]
[1⁄2]
For B only need to consider Type II
E  X | X  1500   2500 from part (i)
[1⁄2]
So expected cost per policy for B is
(b)
(v)
45% *15% * e  1.5 * (2500  1500) *5%  $0.75 [1⁄2]
So B is better under the Bayes criterion [1⁄2]
B is clearly better under minimax, since total claims are capped [1]
Unlikely that the assumption of independence holds, since events that lead to
travel cancellations are likely to affect more than one policy holder.
[2]
The reinsurer is a malevolent opponent so minimax is appropriate.
[2]
Page 8Subject CT6 (Statistical Methods Core Technical) – September 2017 – Examiners’ Report
[Max 2]
The first part aside, this question proved to be the hardest on the paper,
with only the strongest candidates able to score well, particularly on
parts (iv) and (v).
