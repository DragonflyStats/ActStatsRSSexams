\documentclass[a4paper,12pt]{article}
%%%%%%%%%%%%%%%%%%%%%%%%%%%%%%%%%%%%%%%%%%%%%%%%%%%%%%%%%%%%%%%%%%%%%%%%%%%%%%%%%%%%%%%%%%%%%%%%%%%%%%%%%%%%%%%%%%%%%%%%%%%%%%%%%%%%%%%%%%%%%%%%%%%%%%%%%%%%%%%%%%%%%%%%%%%%%%%%%%%%%%%%%%%%%%%%%%%%%%%%%%%%%%%%%%%%%%%%%%%%%%%%%%%%%%%%%%%%%%%%%%%%%%%%%%%%
 
\usepackage{eurosym}

\usepackage{vmargin}

\usepackage{amsmath}

\usepackage{graphics}

\usepackage{epsfig}

\usepackage{enumerate}

\usepackage{multicol}

\usepackage{subfigure}

\usepackage{fancyhdr}

\usepackage{listings}

\usepackage{framed}

\usepackage{graphicx}

\usepackage{amsmath}

\usepackage{chngpage}



%\usepackage{bigints}

\usepackage{vmargin}

% left top textwidth textheight headheight

% headsep footheight footskip

\setmargins{2.0cm}{2.5cm}{16 cm}{22cm}{0.5cm}{0cm}{1cm}{1cm}

\renewcommand{\baselinestretch}{1.3}
\setcounter{MaxMatrixCols}{10}

\begin{document}



4 The number of claims on a portfolio of insurance policies in a given year follows a
Poisson distribution with unknown mean λ. Prior beliefs about λ are specified by a
gamma distribution with mean 60 and variance 360. Over a period of three and onethird
years, the total number of claims is 200.
\begin{enumerate}
\item (i) Calculate the Bayesian estimate of λ under all-or-nothing loss. 
\item (ii) Comment on your result for part (i). [1]
\end{itemize}
\newpage 
Q4 (i)  ~ Γ,  where 60


 and 2 360,


 so 1
6
  and  10 . [1½]
Hence f   9e 6 ;
    
     [1]
the likelihood  
10
L 200e 3
    
     ; [1]
So the posterior  
10 7
f |x 9e 6 * 200e 3 209e 2
              
            [1]
This is gamma with parameters 210 and 3.5. [1]
ln f ˆ|x  const  209ln ˆ  3.5 ˆ
Differentiating 20ˆ9 3.5 59 7
  ˆ  .

%%%%%%%%%%%%%%%%%%%%%%%%%%%%%%%%%%%%%%%%%%%%%

Differentiating again gives 2
209
ˆ


which is clearly negative and so we have a
maximum.
(ii) The Bayesian estimate is very similar to the mean of the prior distribution,  
which is unsurprising since the average number of realised claims has been in
line with this.  
[Total 8]
Many candidates were able to score well here, especially on part (i).
\end{document}
