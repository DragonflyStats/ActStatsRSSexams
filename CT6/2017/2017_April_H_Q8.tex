\documentclass[a4paper,12pt]{article}

\usepackage{eurosym}
\usepackage{vmargin}
\usepackage{amsmath}
\usepackage{graphics}
\usepackage{epsfig}
\usepackage{enumerate}
\usepackage{multicol}
\usepackage{subfigure}
\usepackage{fancyhdr}
\usepackage{listings}
\usepackage{framed}
\usepackage{graphicx}
\usepackage{amsmath}

\usepackage{chngpage}



%\usepackage{bigints}

\usepackage{vmargin}



% left top textwidth textheight headheight

% headsep footheight footskip

\setmargins{2.0cm}{2.5cm}{16 cm}{22cm}{0.5cm}{0cm}{1cm}{1cm}

\renewcommand{\baselinestretch}{1.3}

\setcounter{MaxMatrixCols}{10}



\begin{document}

\begin{enumerate}

%%%%%%%%%%%%%%%%%%%%%%%%%%%%%%%%%%%%%%%%%%%%%%%%%%%%%%%%%%%%%%%%%%%%%%%%%%
8 (i) Write down the general form of a statistical model for a claims run-off
triangle, defining all terms used.
The table below shows the cumulative incurred claim amounts on a portfolio of
insurance policies.
Development Year
Underwriting Year 0 1 2
2014 3,215 6,847 10,078
2015 2,986 7,123
2016 4,167
Claims are assumed to fully run off after Development Year 2. The estimated loss ratio of both 2015 and 2016 is 91% and the respective premium income in each year
is:
  Premium Income
2014 11,365
2015 12,012
2016 12,867
The total of claim amounts paid to date is 21,186 from policies written in 2014 to
2016.
(ii) Calculate the outstanding claim reserve for this portfolio using the Bornheutter-Ferguson method. [9]
%%%%%%%%%%%%%%%%%%%%%%%%%%%%%%%%%%%%%%%%%%%%%%%%%%%%%
\newpage

%%%%%%%%%%%%%%%%%%%%%%%%%%%%%%%%%
\newpage
Q8 (i) The general form can be expressed as follows:
  Cij  rjsixi j  eij 
where
Cij is the cumulative or incremental entry in the run-off triangle; [½]
rj is the development factor for Development Year j; 
si represents the exposure (or number of claims / policies); 
xi+j is a parameter varying by calendar year; 
eij is an error term. 
(ii) DF from year 2 to year 3 is 10078 / 6847 = 1.471885 
DF from year 1 to year 2 is (7123 + 6847) / (3215 + 2986) = 2.252862 
For AY 2015, expected ultimate loss is 0.91 * 12012 = 10931 
Expected loss to date is 10931 / 1.471885 = 7,426 
So the adjusted ultimate loss is 10931 – (7426  7123) = 10627 
For AY 2016, expected ultimate loss is 0.91 * 12867 = 11709 
Expected loss to date is 11709 / (1.471885 * 2.252862)= 3531 
So the adjusted ultimate loss is 11709 – (3531  4167) = 12345 
So the reserve is 10078 + 10627 + 12345 – 21186 = 11864 
[Total 14]
Many candidates were unfamiliar with the bookwork required to answer part
(i), although most scored well on part (ii).
\end{document}
