\documentclass[a4paper,12pt]{article}
%%%%%%%%%%%%%%%%%%%%%%%%%%%%%%%%%%%%%%%%%%%%%%%%%%%%%%%%%%%%%%%%%%%%%%%%%%%%%%%%%%%%%%%%%%%%%%%%%%%%%%%%%%%%%%%%%%%%%%%%%%%%%%%%%%%%%%%%%%%%%%%%%%%%%%%%%%%%%%%%%%%%%%%%%%%%%%%%%%%%%%%%%%%%%%%%%%%%%%%%%%%%%%%%%%%%%%%%%%%%%%%%%%%%%%%%%%%%%%%%%%%%%%%%%%%%
 
\usepackage{eurosym}

\usepackage{vmargin}

\usepackage{amsmath}

\usepackage{graphics}

\usepackage{epsfig}

\usepackage{enumerate}

\usepackage{multicol}

\usepackage{subfigure}

\usepackage{fancyhdr}

\usepackage{listings}

\usepackage{framed}

\usepackage{graphicx}

\usepackage{amsmath}

\usepackage{chngpage}



%\usepackage{bigints}

\usepackage{vmargin}

% left top textwidth textheight headheight

% headsep footheight footskip

\setmargins{2.0cm}{2.5cm}{16 cm}{22cm}{0.5cm}{0cm}{1cm}{1cm}

\renewcommand{\baselinestretch}{1.3}
\setcounter{MaxMatrixCols}{10}

\begin{document}

\begin{enumerate}
3 (i) Explain why claim amounts from general insurance policies are typically
modelled using statistical distributions with heavy tails. [2]
Claim amounts on a portfolio of insurance policies are assumed to follow a Weibull distribution. A quarter of losses are below 15 and a quarter of losses are above 80.
(ii) Estimate the parameters c, γ of the Weibull distribution that fit this data. 
(iii) Determine whether or not this Weibull distribution has a heavier tail than that
of the exponential distribution with parameter c, by considering your estimate
of γ. [2]
[Total 7]
CT6 A2017–3 PLEASE TURN OVER
4 The number of claims on a portfolio of insurance policies in a given year follows a
Poisson distribution with unknown mean λ. Prior beliefs about λ are specified by a
gamma distribution with mean 60 and variance 360. Over a period of three and onethird
years, the total number of claims is 200.
(i) Calculate the Bayesian estimate of λ under all-or-nothing loss. 
(ii) Comment on your result for part (i). [1]
[Total 8]
Q3 (i) Insurance claims are often very positively skewed, with large claims often
being several multiples of smaller claims. This suits distributions with heavy
tails. [2]
(ii) 1 e c15 0.25     so c15  ln 0.75 [½]
Similarly c80  ln 0.25 [½]
So 15 ln 0.75 0.207 519
80 ln 0.25
       
 
[1]
So ln 0.207 519 0.9394
ln 15
80
  
 
 
 
[½]
And 0.9394
ln 0.75 0.0226
15
c    [½]
(iii) Since  is less than 1, the Weibull distribution in this case has a heavier tail
than the exponential. [2]
[Total 7]
Candidates generally scored well on parts (i) and (ii), but only the better
candidates were familiar with the bookwork for part (iii).
Q4 (i)  ~ Γ,  where 60


 and 2 360,


 so 1
6
  and  10 . [1½]
Hence f   9e 6 ;
    
     [1]
the likelihood  
10
L 200e 3
    
     ; [1]
So the posterior  
10 7
f |x 9e 6 * 200e 3 209e 2
              
            [1]
This is gamma with parameters 210 and 3.5. [1]
ln f ˆ|x  const  209ln ˆ  3.5 ˆ
Differentiating 20ˆ9 3.5 59 7
  ˆ  .

[1½]
Subject CT6 (Statistical Methods Core Technical) – April 2017 – Examiners’ Report
Page 5
Differentiating again gives 2
209
ˆ


which is clearly negative and so we have a
maximum.
(ii) The Bayesian estimate is very similar to the mean of the prior distribution, [½]
which is unsurprising since the average number of realised claims has been in
line with this. [½]
[Total 8]
Many candidates were able to score well here, especially on part (i).
