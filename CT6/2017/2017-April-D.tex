\documentclass[a4paper,12pt]{article}

\usepackage{eurosym}
\usepackage{vmargin}
\usepackage{amsmath}
\usepackage{graphics}
\usepackage{epsfig}
\usepackage{enumerate}
\usepackage{multicol}
\usepackage{subfigure}
\usepackage{fancyhdr}
\usepackage{listings}
\usepackage{framed}
\usepackage{graphicx}
\usepackage{amsmath}

\usepackage{chngpage}



%\usepackage{bigints}

\usepackage{vmargin}



% left top textwidth textheight headheight

% headsep footheight footskip

\setmargins{2.0cm}{2.5cm}{16 cm}{22cm}{0.5cm}{0cm}{1cm}{1cm}

\renewcommand{\baselinestretch}{1.3}

\setcounter{MaxMatrixCols}{10}



\begin{document}

\begin{enumerate}
7 An actuary is assessing three different insurance companies, A, B and C.
Corresponding claim amounts and number of policies are shown in the data below.
Company A Company B Company C
\$m Policies
\$m Policies \$m Policies
2013 1.16 85 0.85 68 1.48 110
2014 1.18 88 1.02 82 1.52 132
2015 1.14 85 0.96 70 1.78 143
2016 1.32 92 0.87 80 1.92 165
Total 4.8 350 3.7 300 6.7 550
Company C has 180 policies to insure in 2017.
(i) Calculate its expected claim amount, using the assumptions underlying
Empirical Bayes Credibility Theory (EBCT) Model 2. 
(ii) Discuss why it might be preferable to use EBCT Model 2 rather than EBCT
Model 1 for this purpose. 
[Total 13]
%%%%%%%%%%%%%%%%%%%%%%%%%%%%%%%%%%%%%%%%%%%%%%%%%%%%%%%%%%%%%%%%%%%%%%%%%%
8 (i) Write down the general form of a statistical model for a claims run-off
triangle, defining all terms used.
The table below shows the cumulative incurred claim amounts on a portfolio of
insurance policies.
Development Year
Underwriting Year 0 1 2
2014 3,215 6,847 10,078
2015 2,986 7,123
2016 4,167
Claims are assumed to fully run off after Development Year 2. The estimated loss ratio of both 2015 and 2016 is 91% and the respective premium income in each year
is:
  Premium Income
2014 11,365
2015 12,012
2016 12,867
The total of claim amounts paid to date is 21,186 from policies written in 2014 to
2016.
(ii) Calculate the outstanding claim reserve for this portfolio using the Bornheutter-Ferguson method. [9]
%%%%%%%%%%%%%%%%%%%%%%%%%%%%%%%%%%%%%%%%%%%%%%%%%%%%%
\newpage

Q7 (i) PA  350, PB  300, PC  550, P  1200
* 1 350* 1 350 300* 1 300 550* 1 550 70.1
11 1200 1200 1200
P
                           

4.8 0.0137, 3.7 0.0123, 6.7 0.0122
A 350 B 300 C 550 X   X   X   , 15.2 0.0127
1200
X  

%%---  %%%%%%%%%%%%%%%%%%%%%%%%%%%%%%%%%%%%%%%%%%%%%%% – April 2017 – Examiners’ Report
Page 7
Now need to calc   4 2
1
ij ij i
j
P X X

 
1.16 2 1.18 2 1.14 2 1, 85* 0.0137 88* 0.0137 85* 0.0137
85 88 85
i             
     
1.32 2 92* 0.0137 0.000 053 286
92
     
 

0.85 2 1.02 2 0.96 2 2, 68* 0.0123 82* 0.0123 70* 0.0123
68 82 70
i             
     
0.87 2 80* 0.0123 0.000 306 436
80
     
 
[½]
1.48 2 1.52 2 3,1 10* 0.0122 132* 0.0122 143
110 132
i         
   
1.78 2 1.92 2 * 0.0122 165* 0.0122 0.000 296 037
143 165
           
   
[½]
E s2    3 4 2
1 1
1 1
3 3 ij ij i
i j
P X X
 
     
 
 
1 1 0.000 005 329 0.000 030 644 0.000 029 604
3 3
      
 
 0.000 072 862 
Now need to calc   4 2
1
ij ij
j
P X X

 
1.16 2 1.18 2 1.14 2 1, 85* 0.0127 88* 0.0127 85* 0.0127
85 88 85
i            
     
1.32 2 92* 0.0127 0.00 043 741
92
      
 

%%-----  %%%%%%%%%%%%%%%%%%%%%%%%%%%%%%%%%%%%%%%%%%%%%%% – April 2017 – Examiners’ Report
Page 8
0.85 2 1.02 2 0.96 2 2, 68* 0.0127 82* 0.0127 70* 0.0127
68 82 70
i            
     
0.87 2 80* 0.0127 0.000 339 769
80
      
 
1.48 2 1.52 2 3,1 10* 0.0127 132* 0.0127 143
110 132
i         
   
1.78 2 1.92 2 * 0.0127 165* 0.0127 0.000 425 330
143 165
           
   
%-------------------%
Now
  3 4 2
1 1
1
1 ij ij
i j
P X X
Nn  

 
1 0.00043741 0.000339769 0.000425330
11
  
 0.0001 09 319 
So var   1 0.000109319 0.000072862 0.000 000 520
70.1
m      [½]
So the credibility factor
 
 
2
550
550 0.000072862
0.000000520
var
j
j
P
Z
E s
P
m

 
       
  
 0.7970 
So the expected premium per policy for company C is
0.7970*0.0122  1 0.7970*0.0127  0.012 3 
So total expected claims in 2016 is 180 * 0.012 3 = $2.21m [½]
%%----  %%%%%%%%%%%%%%%%%%%%%%%%%%%%%%%%%%%%%%%%%%%%%%% – April 2017 – Examiners’ Report
Page 9
(ii) EBCT Model 2 allows for the increasing number of policies that Company C
has seen year on year so is arguably a better estimate. 
[Total 13]
Some candidates were clearly unfamiliar with the techniques needed to apply
EBCT II from scratch, although many candidates were able to score well.

%%%%%%%%%%%%%%%%%%%%%%%%%%%%%%%%%
\newpage
Q8 (i) The general form can be expressed as follows:
  Cij  rjsixi j  eij 
where
Cij is the cumulative or incremental entry in the run-off triangle; [½]
rj is the development factor for Development Year j; 
si represents the exposure (or number of claims / policies); 
xi+j is a parameter varying by calendar year; 
eij is an error term. [½]
(ii) DF from year 2 to year 3 is 10078 / 6847 = 1.471885 
DF from year 1 to year 2 is (7123 + 6847) / (3215 + 2986) = 2.252862 
For AY 2015, expected ultimate loss is 0.91 * 12012 = 10931 
Expected loss to date is 10931 / 1.471885 = 7,426 
So the adjusted ultimate loss is 10931 – (7426  7123) = 10627 
For AY 2016, expected ultimate loss is 0.91 * 12867 = 11709 
Expected loss to date is 11709 / (1.471885 * 2.252862)= 3531 
So the adjusted ultimate loss is 11709 – (3531  4167) = 12345 
So the reserve is 10078 + 10627 + 12345 – 21186 = 11864 
[Total 14]
Many candidates were unfamiliar with the bookwork required to answer part
(i), although most scored well on part (ii).
\end{document}
