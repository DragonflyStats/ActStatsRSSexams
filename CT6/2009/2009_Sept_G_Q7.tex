\documentclass[a4paper,12pt]{article}

%%%%%%%%%%%%%%%%%%%%%%%%%%%%%%%%%%%%%%%%%%%%%%%%%%%%%%%%%%%%%%%%%%%%%%%%%%%%%%%%%%%%%%%%%%%%%%%%%%%%%%%%%%%%%%%%%%%%%%%%%%%%%%%%%%%%%%%%%%%%%%%%%%%%%%%%%%%%%%%%%%%%%%%%%%%%%%%%%%%%%%%%%%%%%%%%%%%%%%%%%%%%%%%%%%%%%%%%%%%%%%%%%%%%%%%%%%%%%%%%%%%%%%%%%%%%

\usepackage{eurosym}
\usepackage{vmargin}
\usepackage{amsmath}
\usepackage{graphics}
\usepackage{epsfig}
\usepackage{enumerate}
\usepackage{multicol}
\usepackage{subfigure}
\usepackage{fancyhdr}
\usepackage{listings}
\usepackage{framed}
\usepackage{graphicx}
\usepackage{amsmath}
\usepackage{chngpage}

%\usepackage{bigints}
\usepackage{vmargin}

% left top textwidth textheight headheight

% headsep footheight footskip

\setmargins{2.0cm}{2.5cm}{16 cm}{22cm}{0.5cm}{0cm}{1cm}{1cm}

\renewcommand{\baselinestretch}{1.3}

\setcounter{MaxMatrixCols}{10}

\begin{document}
7
The transition rules for moving between the three levels 0%, 35% and 50% of a No
Claims Discount system are as follows:
If no claim is made in a year, the policyholder moves to the next higher level of
discount, or remains at the 50% level. When at the 0% or 35% level, the policyholder
moves to (or remains at) the 0% level when one or more claims is made during the
year. When at the 50% level of discount, the policyholder moves to the 35% level if
exactly one claim is made during the year, or moves to the 0% level if two or more
claims are made during the year.
It is assumed that the number of claims X made each year has a geometric distribution
with parameter q such that
P ( X = x ) = q x (1 − q ) , x = 0,1, 2, ...
The full premium is 350.
(i)
(a) Write down the transition matrix.
(b) Verify that the equilibrium distribution (in increasing order of
discount) is of the form:
( kq 2 (2 − q ), kq (1 − q ), k (1 − q ) 2 )
for some constant k . Express k in terms of q .
CT6 S2009—4
[8](ii)
The value of the expected premium in the stationary state paid by “low risk”
policyholders (with q = 0.05) is 178.51.
(a) Calculate the corresponding figure paid by “high risk” policyholders
(with q = 0.1).
(b) Comment on the effectiveness of the No Claims Discount system.

[Total 12]

7
(i)
a. Note first that
Page 6
P ( X
P ( X 0) 1 q
1) q (1 q )
P ( X
P ( X 2) 1 (1 q ) q (1 q )
1) 1 (1 q ) q
q 2%%%%%%%%%%%%%%%%%%%%%%%%%%%%%%%%%%%%5 — %%%%%%%%%%%%%%%%%%%%%%%%%%%%%%%%%%%%5
The transition matrix is
q
q
P
q 2
1 q
0
0
1 q
q (1 q ) 1 q
b. ( kq 2 (2 q ), kq (1 q ), k (1 q ) 2 ) P ( 1 ,
2 , 3 )
Where
kq 3 (2 q ) kq 2 (1 q ) kq 2 (1 q ) 2
1
kq 2 q (2 q ) (1 q ) (1 q ) 2
kq 2 2 q q 2 1 q 1 2 q q 2
kq 2 (2 q )
kq 2 (2 q )(1 q ) k (1 q ) 2 q (1 q )
2
kq (1 q ) q (2 q ) (1 q ) 2
kq (1 q )(2 q q 2 1 2 q q 2 )
kq (1 q )
kq (1 q ) 2 k (1 q ) 3
3
k (1 q ) 2 ( q 1 q )
k (1 q ) 2
Since the proportions must sum to 1, we have
k
1 1
q 2 (1 q ) q (1 q ) (1 q ) 2 2 q 2 q 3 q 1
(ii)
a. Average premium is:
L 350
1
2
2 0.1
3
0.1
0.1 1
0.1 2 1.9 0.65 0.1 0.9 0.5 0.9 2
= 183.76
b. Policyholders are twice as likely to claim, but the premium
increases only by 3%! Suggests that the NCD system is not
effective.
A 3x3 NCD problem with generic P(claim) = q, and P(not claim) = 1-q, and then a
numerical application. Despite some fiddly algebra, this question was based on
standard NCD theory and answered very well.


\end{document}
