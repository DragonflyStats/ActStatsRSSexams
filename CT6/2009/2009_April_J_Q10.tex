\documentclass[a4paper,12pt]{article}

%%%%%%%%%%%%%%%%%%%%%%%%%%%%%%%%%%%%%%%%%%%%%%%%%%%%%%%%%%%%%%%%%%%%%%%%%%%%%%%%%%%%%%%%%%%%%%%%%%%%%%%%%%%%%%%%%%%%%%%%%%%%%%%%%%%%%%%%%%%%%%%%%%%%%%%%%%%%%%%%%%%%%%%%%%%%%%%%%%%%%%%%%%%%%%%%%%%%%%%%%%%%%%%%%%%%%%%%%%%%%%%%%%%%%%%%%%%%%%%%%%%%%%%%%%%%

\usepackage{eurosym}
\usepackage{vmargin}
\usepackage{amsmath}
\usepackage{graphics}
\usepackage{epsfig}
\usepackage{enumerate}
\usepackage{multicol}
\usepackage{subfigure}
\usepackage{fancyhdr}
\usepackage{listings}
\usepackage{framed}
\usepackage{graphicx}
\usepackage{amsmath}
\usepackage{chngpage}

%\usepackage{bigints}
\usepackage{vmargin}

% left top textwidth textheight headheight

% headsep footheight footskip

\setmargins{2.0cm}{2.5cm}{16 cm}{22cm}{0.5cm}{0cm}{1cm}{1cm}

\renewcommand{\baselinestretch}{1.3}

\setcounter{MaxMatrixCols}{10}

\begin{document}
%%%%%%%%%%%%%%%%%%%%%%%%%%%%%%%

%%-- April Question 10
Let Y t be a stationary time series with autocovariance function \gamma  Y ( s ) .
\begin{enumerate}
\item (i)  Show that the new series X t = a + bt +Y t where a and b are fixed non-zero constants, is not stationary.

\item (ii)  Express the autocovariance function of ΔX t = X t − X t−1 in terms of \gamma  Y ( s ) and
show that this new series is stationary.

\item (iii)  Show that if Y t is a moving average process of order 1, then the series ΔX t is not invertible and has variance larger than that of Y t .

\end{enumerate}

CT6 A2009—5

%%%%%%%%%%%%%%%%%%%%%%%%%%%%%%%%%%%%%%%%%%%%
\newpage


10
(i)
Since Y t is stationary, we know there is a constant \mu Y such that E ( Y t ) = \mu Y
for all values of t.
But then E ( X t ) = E ( a + bt + Y t ) = a + bt + \mu Y which depends on t since b is
non-zero.
Hence X t is not stationary.
Page 9%%%%%%%%%%%%%%%%%%%%%%%%%%%%%% — Examiners’ Report
(ii)
First note that
E ( Δ X t ) = E ( X t − X_{t − 1} )
= E ( X t ) − E ( X_{t − 1} )
= E ( a + bt + Y t ) − E ( a + b ( t − 1) + Y t − 1 )
= a + bt + \mu Y − a − b ( t − 1) − \mu Y
= b
i.e. the mean is a constant independent of t.
Secondly,
Cov ( Δ X t , Δ X t − s ) = Cov ( X t − X_{t − 1} , X t − s − X t − s − 1 )
= Cov ( b + Y t − Y t − 1 , b + Y t − s − Y t − s − 1 )
= Cov ( Y t − Y t − 1 , Y t − s − Y t − s − 1 )
= Cov ( Y t , Y t − s ) − Cov ( Y t − 1 , Y t − s ) − Cov ( Y t , Y t − s − 1 ) + Cov ( Y t − 1 , Y t − s − 1 )
= \gamma  Y ( s ) − \gamma  Y ( s − 1) − \gamma  Y ( s + 1) + \gamma  Y ( s )
= 2 \gamma  Y ( s ) − \gamma  Y ( s − 1) − \gamma  Y ( s + 1)
Since the autocovariance depends only on the lag s, and the mean is constant,
the new series is stationary.
(iii)
Suppose that Y t = e t + \beta e t − 1 where e t is a white noise process with variance
\sigma^{2}   .
Then
Δ X t = a + bt + e t + \beta e t − 1 − a − b ( t − 1) − e t − 1 − \beta e t − 2
= b + e t + ( \beta − 1) e t − 1 − \beta e t − 2
= b + (1 + ( \beta − 1) B − \beta B 2 ) e t
= b + (1 − B )(1 + \beta B ) e t
So the lag polynomial has a unit root, and hence the time series is not
invertible.
Now
Var ( Y t ) = Var ( e t + \beta e t − 1 )
= Var ( e t ) + \beta 2 Var ( e t − 1 )
= (1 + \beta 2 ) \sigma^{2}  
So \gamma  Y (0) = (1 + \beta 2 ) \sigma^{2}  
Page 10%%%%%%%%%%%%%%%%%%%%%%%%%%%%%% — Examiners’ Report
Also \gamma  Y (1) = \gamma  Y ( − 1) = Cov ( e t + \beta e t − 1 , e t − 1 + \beta e t − 2 ) = \beta\sigma^{2}  
So using the result from part (ii)
Var ( Δ X t ) = 2 \gamma  Y (0) − \gamma  Y (1) − \gamma  Y ( − 1)
= 2(1 + \beta 2 ) \sigma^{2}   − 2 \beta\sigma^{2}  
= 2(1 − \beta + \beta 2 ) \sigma^{2}  
And finally
Var ( Δ X t ) − Var ( Y t ) = (2 − 2 \beta + 2 \beta 2 ) \sigma^{2}   − (1 + \beta 2 ) \sigma^{2}  
= (1 − 2 \beta + \beta 2 ) \sigma^{2}  
= (1 − \beta ) 2 \sigma^{2}   > 0
