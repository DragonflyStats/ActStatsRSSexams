\documentclass[a4paper,12pt]{article}

%%%%%%%%%%%%%%%%%%%%%%%%%%%%%%%%%%%%%%%%%%%%%%%%%%%%%%%%%%%%%%%%%%%%%%%%%%%%%%%%%%%%%%%%%%%%%%%%%%%%%%%%%%%%%%%%%%%%%%%%%%%%%%%%%%%%%%%%%%%%%%%%%%%%%%%%%%%%%%%%%%%%%%%%%%%%%%%%%%%%%%%%%%%%%%%%%%%%%%%%%%%%%%%%%%%%%%%%%%%%%%%%%%%%%%%%%%%%%%%%%%%%%%%%%%%%

\usepackage{eurosym}
\usepackage{vmargin}
\usepackage{amsmath}
\usepackage{graphics}
\usepackage{epsfig}
\usepackage{enumerate}
\usepackage{multicol}
\usepackage{subfigure}
\usepackage{fancyhdr}
\usepackage{listings}
\usepackage{framed}
\usepackage{graphicx}
\usepackage{amsmath}
\usepackage{chngpage}

%\usepackage{bigints}
\usepackage{vmargin}

% left top textwidth textheight headheight

% headsep footheight footskip

\setmargins{2.0cm}{2.5cm}{16 cm}{22cm}{0.5cm}{0cm}{1cm}{1cm}

\renewcommand{\baselinestretch}{1.3}

\setcounter{MaxMatrixCols}{10}

\begin{document}

2 \item (i)  Express the probability density function of the gamma distribution in the form
of a member of the exponential family of distributions. Specify the natural
and scale parameters.

\item (ii)  State the corresponding canonical link function for generalised linear
modelling if the response variable has a gamma distribution.

%%%%%%%%%%%%%%%%%%%%%%%%%%%%%
We need to express the distribution function of the Gamma distribution in the
form:
⎡ ( y \theta  − b ( \theta  ))
⎤
f Y ( y ; \theta  , \phi  ) = exp ⎢
+ c ( y , \phi  ) ⎥
⎣ a ( \phi  )
⎦
Suppose Y has a Gamma distribution with parameters \alpha  and \lambda  . Then
f Y ( y ) =
And substituting \lambda  =
\lambda  \alpha  \alpha − 1 −\lambda  y
y e
\Gamma  ( \alpha  )
\alpha 
we can write the density as
\mu
f Y ( y ; \theta  , \phi  ) =
\alpha  \alpha 
\mu \alpha  \Gamma  ( \alpha  )
y
\alpha − 1
e
−
y \alpha 
\mu
⎡ ⎛ y
⎤
⎞
f Y ( y ; \theta  , \phi  ) = exp ⎢ ⎜ − − log \mu ⎟ \alpha  + ( \alpha  − 1) log y + \alpha  log \alpha  − log \Gamma  ( \alpha  ) ⎥
⎠
⎣ ⎝ \mu
⎦
1
1
; \phi  = \alpha  ; a ( \phi  ) = ; b ( \theta  ) = − log( −\theta  )
\mu
\phi 
and c ( y , \phi  ) = ( \phi  − 1) log y + \phi  log \phi  − log \Gamma  ( \phi  ) .
Which is in the right form with \theta  = −
Thus the natural parameter is
1
, ignoring the minus sign, and the scale
\mu
parameter is \alpha  .
(ii)
The corresponding link function is
1
.
\mu
Page 3%%%%%%%%%%%%%%%%%%%%%%%%%%%%%% — Examiners’ Report
