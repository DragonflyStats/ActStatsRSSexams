

%%%%%%%%%%%%%%%%%%%%%%%%%%%%%%%7
8
It is necessary to simulate samples from a distribution with density function f ( x ) = 6 x (1 − x ) 0 < x < 1 .
\item (i)  Use the acceptance-rejection technique to construct a complete algorithm for generating samples from f by first generating samples from the distribution
with density $h ( x ) = 2(1 − x )$ .

\item (ii)  Calculate how many samples from h would on average be needed to generate one realisation from f.

\item (iii)  Explain whether the acceptance-rejection method in \item (i)  would be more efficient if the uniform distribution were to be used instead.

% -------------------------- [Total 10]
An insurer has an initial surplus of U. Claims up to time t are denoted by S(t).
Annual premium income is received continuously at a rate of c per unit time.
\item (i)  Explain what is meant by the insurer’s surplus process U(t).
\item (ii)  Define carefully each of the following probabilities:
(a)
(b)

\psi  ( U , t )
\psi  h ( U , t )

\item (iii) 
Explain, for each of the following pairs of expressions, whether one of each
pair is certainly greater than the other, or whether it is not possible to reach a
conclusion.
(a)
(b)
(c)
\psi  (10, 2) and \psi  (20,1)
\psi  (10, 2) and \psi  (5,1)
\psi  0.5 (10, 2) and \psi  0.25 (10, 2)


%%%%%%%%%%%%%%%%%%%%%%%%%%%%%%%%%%%%
8
(i)
U(t) represents the insurers surplus at time t. It represents the initial surplus
plus cumulative premiums received less claims incurred:
U ( t ) = U + ct − S ( t )
Where U is the initial surplus and c is the annual premium income (assumed
payable continuously).
(ii)
(iii)
(a) \psi  ( U , t ) = P ( U ( s ) < 0 for some 0 \leq  s \leq  t given that U (0) = U )
(b) \psi  h ( U , t ) = P ( U ( s ) < 0 for some s , s = h , 2 h , ... , t − h , t
given that U (0) = U )
(a) We can say that \psi  (10, 2) > \psi  (10,1) > \psi  (20,1)
The first inequality holds because the longer the period considered
when checking, the more likely that ruin will occur. The second
inequality holds because a larger initial surplus reduces the probability
of ruin (higher claims are required to cause ruin).
(b) We can reach no definite conclusion here. The first term has a longer
term suggesting a higher probability, but a higher initial surplus
suggesting a lower probability. We can’t say anything about the size
of the two effects, so we can’t reach a definite conclusion.
(c) We can say that \psi  0.5 (10, 2) \leq  \psi  0.25 (10, 2) .
This is because the second term checks for ruin at the same times as the first
term, as well as at some additional times. So the probability of ruin when
checking over the larger set of times must be higher.
