
%%%%%%%%%%%%%%%%%%%%%%%%%%%%%%%

%%-- April Question 9

%% % -------------------------- [Total 10]
%% CT6 A2009—49

Individual claims under a certain type of insurance policy are for either 1 (with probability \alpha ) or 2 (with probability 1 − \alpha ).
The insurer is considering entering into an excess of loss reinsurance arrangement with retention 1 + k (where k < 1). Let X i denote the amount paid by the insurer (net
of reinsurance) on the ith claim.
\begin{enumerate}
\item (i) 
Calculate and simplify expressions for the mean and variance of X i .

Now assume that \alpha  = 0.2 . The number of claims in a year follows a Poisson distribution with mean 500. The insurer wishes to set the retention so that the
probability that aggregate claims in a year will exceed 700 is less than 1%.
\item (ii) 

Show that setting k = 0.334 gives the desired result for the insurer.
\end{enumerate}


Page 8%%%%%%%%%%%%%%%%%%%%%%%%%%%%%% — Examiners’ Report
9
(i)
E ( X i ) = \alpha  + (1 + k )(1 − \alpha  )
= 1 + k (1 − \alpha  )
Var ( X i ) = E ( X i 2 ) − E ( X i ) 2
= \alpha  + (1 + k ) 2 (1 − \alpha  ) − (1 + k (1 − \alpha  )) 2
= \alpha  + (1 − \alpha  ) + 2 k (1 − \alpha  ) + k 2 (1 − \alpha  ) − 1 − 2 k (1 − \alpha  ) − k 2 (1 − \alpha  ) 2
= k 2 (1 − \alpha  )(1 − (1 − \alpha  ))
= k 2 \alpha  (1 − \alpha  )
(ii)
Let Y denote the aggregate claims in a year. Then Y has a compound Poisson
distribution, and using the standard results from the tables:
E ( Y ) = 500 \times  E ( X i ) = 500 + 500 k (1 − \alpha  ) = 500 + 400 k =633.60
And
Var ( Y ) = 500 \times  E ( X i 2 )
(
)
= 500 ( \alpha  + (1 − \alpha  ) + 2 k (1 − \alpha  ) + k
= 500 \alpha  + (1 + k ) 2 (1 − \alpha  )
2
(1 − \alpha  )
)
= 500 + (1000 k + 500 k 2 )(1 − \alpha  )
= 500 + 800 k + 400 k 2
= 811.82
Using a normal approximation, we find
P ( Y > 700) = P ( Z >
700 − 633.6
)
811.82
= P ( Z > 2.33)
= 0.01
