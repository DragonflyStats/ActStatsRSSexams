\documentclass[a4paper,12pt]{article}

%%%%%%%%%%%%%%%%%%%%%%%%%%%%%%%%%%%%%%%%%%%%%%%%%%%%%%%%%%%%%%%%%%%%%%%%%%%%%%%%%%%%%%%%%%%%%%%%%%%%%%%%%%%%%%%%%%%%%%%%%%%%%%%%%%%%%%%%%%%%%%%%%%%%%%%%%%%%%%%%%%%%%%%%%%%%%%%%%%%%%%%%%%%%%%%%%%%%%%%%%%%%%%%%%%%%%%%%%%%%%%%%%%%%%%%%%%%%%%%%%%%%%%%%%%%%

\usepackage{eurosym}
\usepackage{vmargin}
\usepackage{amsmath}
\usepackage{graphics}
\usepackage{epsfig}
\usepackage{enumerate}
\usepackage{multicol}
\usepackage{subfigure}
\usepackage{fancyhdr}
\usepackage{listings}
\usepackage{framed}
\usepackage{graphicx}
\usepackage{amsmath}
\usepackage{chngpage}

%\usepackage{bigints}
\usepackage{vmargin}

% left top textwidth textheight headheight

% headsep footheight footskip

\setmargins{2.0cm}{2.5cm}{16 cm}{22cm}{0.5cm}{0cm}{1cm}{1cm}

\renewcommand{\baselinestretch}{1.3}

\setcounter{MaxMatrixCols}{10}

\begin{document}

2 An insurance company has a portfolio of two-year policies. Aggregate annual claims
from the portfolio follow an exponential distribution with mean 10 (independently
from year to year). Annual premiums of 15 are payable at the start of each year. The
insurer checks for ruin only at the end of each year. The insurer starts with no capital.
Calculate the probability that the insurer is not ruined by the end of the second year.

2. Then to avoid ruin after the first year, we require X 1 <15 and to avoid ruin after 2
years we require X 1 + X 2 <30.
15
P ( X 1 15 and X 1
X 2
30)
f X 1 ( x ) P ( X 2
30 x ) dx
0
15
0.1 e 0.1 x
0.1 e 0.1 x
(1 e
0.1(30 x )
0
15
0
Page 2
0.1 e 3 dx
) dx%%%%%%%%%%%%%%%%%%%%%%%%%%%%%%%%%%%%5 — %%%%%%%%%%%%%%%%%%%%%%%%%%%%%%%%%%%%5
e
e
0.1 x
1.5
0.1 xe
1.5 e
3
3 15
0
1
0.702189
This was a question involving the concept of ruin in discrete time and requiring
candidates to calculate a probability by integrating the pdf of the exponential
distribution. This question was not answered well. Many candidates did not recognise
the condition for ruin at t=2, ie X 1 +X 2 <30.

\end{document}
