\documentclass[a4paper,12pt]{article}


%%- Question 1
%%%%%%%%%%%%%%%%%%%%%%%%%%%%%%%%%%%%%%%%%%%%%%%%%%%%%%%%%%%%%%%%%%%%%%%%%%%%%%%%%%%%%%%%%%%%%%%%%%%%%%%%%%%%%%%%%%%%%%%%%%%%%%%%%%%%%%%%%%%%%%%%%%%%%%%%%%%%%%%%%%%%%%%%%%%%%%%%%%%%%%%%%%%%%%%%%%%%%%%%%%%%%%%%%%%%%%%%%%%%%%%%%%%%%%%%%%%%%%%%%%%%%%%%%%%%

\usepackage{eurosym}
\usepackage{vmargin}
\usepackage{amsmath}
\usepackage{graphics}
\usepackage{epsfig}
\usepackage{enumerate}
\usepackage{multicol}
\usepackage{subfigure}
\usepackage{fancyhdr}
\usepackage{listings}
\usepackage{framed}
\usepackage{graphicx}
\usepackage{amsmath}
\usepackage{chngpage}

%\usepackage{bigints}
\usepackage{vmargin}

% left top textwidth textheight headheight

% headsep footheight footskip

\setmargins{2.0cm}{2.5cm}{16 cm}{22cm}{0.5cm}{0cm}{1cm}{1cm}

\renewcommand{\baselinestretch}{1.3}

\setcounter{MaxMatrixCols}{10}

\begin{document}
© Institute of Actuaries1
Consider the stationary autoregressive process of order 1 given by
Y t = 2 \alpha Y t − 1 + Z t ,
\alpha < 0.5
where Z t denotes white noise with mean zero and variance \sigma^2  .
∞
Express Y t in the form Y t = \sum a j Z t − j and hence or otherwise find an expression for
j = 0
the variance of Y t in terms of \alpha and \sigma  .



Institute of Actuaries%%%%%%%%%%%%%%%%%%%%%%%%%%%%%%%%%%%%5 — %%%%%%%%%%%%%%%%%%%%%%%%%%%%%%%%%%%%5
1
2 Y t
Y t
Y t
2 Y t
1 Z t
(1 2 B ) Y t Z t
Z t
1
(1 2 B ) 1 Z t
Y t
(1 2 B (2 B ) 2
(2 ) j Z t
(2 B ) 3  ) Z t
j
j 0
So
Var( Y t )
(2 ) j Z t
Var
j
j 0
(4
2 j
)
2
j 0
2
(1 4
2
)
Other valid approaches to deriving the variance were given full credit.
This is a nice, short question involving time series. It requires a little knowledge about
series expansions, some algebraic manipulation and the formula for a geometric
progression. The question was answered reasonably well, with most candidates
recalling the series expansion for (1-X) -1 . Strong candidates spotted that the condition
|\alpha|<0.5 was needed to use the formula for an infinite geometric series.
2
Let X 1 denote aggregate claims in year 1, and let X 2 denote aggregate claims in year
