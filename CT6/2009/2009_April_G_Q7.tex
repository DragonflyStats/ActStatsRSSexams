\documentclass[a4paper,12pt]{article}

%%%%%%%%%%%%%%%%%%%%%%%%%%%%%%%%%%%%%%%%%%%%%%%%%%%%%%%%%%%%%%%%%%%%%%%%%%%%%%%%%%%%%%%%%%%%%%%%%%%%%%%%%%%%%%%%%%%%%%%%%%%%%%%%%%%%%%%%%%%%%%%%%%%%%%%%%%%%%%%%%%%%%%%%%%%%%%%%%%%%%%%%%%%%%%%%%%%%%%%%%%%%%%%%%%%%%%%%%%%%%%%%%%%%%%%%%%%%%%%%%%%%%%%%%%%%

\usepackage{eurosym}
\usepackage{vmargin}
\usepackage{amsmath}
\usepackage{graphics}
\usepackage{epsfig}
\usepackage{enumerate}
\usepackage{multicol}
\usepackage{subfigure}
\usepackage{fancyhdr}
\usepackage{listings}
\usepackage{framed}
\usepackage{graphicx}
\usepackage{amsmath}
\usepackage{chngpage}

%\usepackage{bigints}
\usepackage{vmargin}

% left top textwidth textheight headheight

% headsep footheight footskip

\setmargins{2.0cm}{2.5cm}{16 cm}{22cm}{0.5cm}{0cm}{1cm}{1cm}

\renewcommand{\baselinestretch}{1.3}

\setcounter{MaxMatrixCols}{10}

\begin{document}
%%%%%%%%%%%%%%%%%%%%%%%%%%%%%%%%%%%%%%%%%%%%%%%%%%%%%%%%

7
(i)
We must find C where
C = Max
f ( x )
6 x (1 − x )
= max
= max 3 x = 3
h ( x )
2(1 − x )
We need to be able to generate a random variable from the distribution with
density h(x). We can do this as follows.
First note that the cdf of h(x) is
x x
0 0
x
H ( x ) = ∫ g ( t ) dt = ∫ 2(1 − t ) dt = ⎡ 2 t − t 2 ⎤ = 2 x − x 2
⎣
⎦ 0
Given a random sample z from U(0,1) we can use the inverse transform
method to sample from h by setting:
2 x − x 2 = z
x 2 − 2 x + z = 0
x =
2 ± 4 − 4 z
= 1 ± 1 − z
2
And we can see that the solution we want is x = 1 − 1 − z
The algorithm to generate a sample y from f is then:
•
•
•
•
Sample z from U(0,1)
Generate x from h(x) via x = 1 − 1 − z
Generate u from U(0,1)
f ( x ) 6 x (1 − x )
If u <
=
= x then generate y = x otherwise begin again.
3 h ( x ) 6(1 − x )
Page 7%%%%%%%%%%%%%%%%%%%%%%%%%%%%%% — Examiners’ Report
(ii) On average, we expect to use C = 3 realisations from h to generate one sample
from f.
(iii) In this case, we must find the maximum value of f(x).
f ′ ( x ) = 6 − 12 x
And f ′ (x) = 0 when x = 0.5
Since f(0) = f(1) = 0 and f(0.5) = 6/4 = 1.5 we can see that this is the
maximum.
Since C is lower for g(x) = 1, using the constant function would be more
efficient.
