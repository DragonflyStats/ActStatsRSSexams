\documentclass[a4paper,12pt]{article}

%%%%%%%%%%%%%%%%%%%%%%%%%%%%%%%%%%%%%%%%%%%%%%%%%%%%%%%%%%%%%%%%%%%%%%%%%%%%%%%%%%%%%%%%%%%%%%%%%%%%%%%%%%%%%%%%%%%%%%%%%%%%%%%%%%%%%%%%%%%%%%%%%%%%%%%%%%%%%%%%%%%%%%%%%%%%%%%%%%%%%%%%%%%%%%%%%%%%%%%%%%%%%%%%%%%%%%%%%%%%%%%%%%%%%%%%%%%%%%%%%%%%%%%%%%%%

\usepackage{eurosym}
\usepackage{vmargin}
\usepackage{amsmath}
\usepackage{graphics}
\usepackage{epsfig}
\usepackage{enumerate}
\usepackage{multicol}
\usepackage{subfigure}
\usepackage{fancyhdr}
\usepackage{listings}
\usepackage{framed}
\usepackage{graphicx}
\usepackage{amsmath}
\usepackage{chngpage}

%\usepackage{bigints}
\usepackage{vmargin}

% left top textwidth textheight headheight

% headsep footheight footskip

\setmargins{2.0cm}{2.5cm}{16 cm}{22cm}{0.5cm}{0cm}{1cm}{1cm}

\renewcommand{\baselinestretch}{1.3}

\setcounter{MaxMatrixCols}{10}

\begin{document}
%%%%%%%%%%%%%%%%%%%%%%%%%%%%%%%

%%-- April Question 11
A motor insurance company operates a No Claims Discount scheme with discount
levels 0%, 25% and 50% of the annual premium of 1,000. The probability of having an accident during any year is 0.1 (ignore the possibility of more than one accident in
a year). The policyholder moves one level up in the discount scheme (or stays at the 50% level) in the event of a claim free year and moves one level down (or stays at the
0% level) if the claim does not involve a criminal offence. If the claim involves a criminal offence then the policyholder automatically moves to the 0% discount level.
One in ten accidents involves a criminal offence. The policyholder makes a claim only if the cost of repairs is higher than the aggregate additional premiums payable in
the next two claim-free years.
\begin{enumerate}
\item (i) 
Calculate, for each level of discount, the cost of a repair below which the policyholder will not claim, distinguishing between claims that involve a
criminal offence and claims that do not.

\item (ii)  Calculate the probability of a claim for each level of discount given that an accident has occurred, given that the repair cost following an accident has an
exponential distribution with mean 400.

\item (iii)  Calculate the stationary distribution of the proportion of policyholders at each discount level.

% -------------------------- [Total 16]
\end{enumerate}

%%%%%%%%%%%%%%%%%%%%%%%%%%%%%%%%%%%%%%%%%%%%%%%%%%%%%%%%%%%%%%%%%%%%%%%%%%%%%%%%%%%%%%%%%%%%%%%%5



11
(i)
Premiums at the three levels are £1,000, £750 and £500.
0% level
Claims involving criminal offences make no difference here.
Claim: 1000 + 750 = 1750
No Claim: 750 + 500 = 1250
No claim if the accident cost is less than 500.
25% level
Claims involving criminal offences make no difference here.
Claim: 1000 + 750 = 1750
No Claim: 500 + 500 = 1000
No claim if the accident cost is less than 750.
50% level
Criminal offence claims make a difference here so we distinguish two
scenarios:
No criminal offence
Claim: 750 + 500 = 1250
Not claim: 500 + 500 = 1000
No claim if the accident cost is less than 250.
Page 11%%%%%%%%%%%%%%%%%%%%%%%%%%%%%% — Examiners’ Report
Criminal offence
Claim: 1000 + 750 = 1750
Not claim: 500 + 500 = 1000
No claim if the accident cost is less than 750.
(ii)
⎛ 1 ⎞
Let X be the repair cost for each accident then X \sim  Exp ⎜
⎟
⎝ 400 ⎠
i.e. P(X > x) = e
−
x
400
For 0% level: P(claim|accident) = P(X > 500) = e
−
For 25% level: P(claim|accident) = P(X > 750) = e
5
4
−
= 0.2865
75
40
= 0.1534
For 50% level: P(claim|accident) = P(claim|criminal offence)
+ P(claim| no criminal offence)
= P(X > 750) \times  0.1 + P(X > 250) \times  0.9
= 0.1e
(iii)
−
75
40
+ 0.9e
−
25
40 =
0.4971
Note that P(Accident) = 0.1 and P(criminal offence| accident) = 0.1.
Hence at 0% level:
P 11 = P(current level) = P(make a claim)
= P(claim|accident) P(accident) = 0.02865
P 12 = P(move to 25% level) = P(not claim) = 1 − 0.02865 = 0.97135
P 13 = P(move to 50% level) = 0
At 25% level:
P 21 = P(move down to 0% level) = P(make a claim)
= P(claim|accident) P(accident) = 0.015335
P 22 = 0 since it can never remain at this level for more than a year at a time
P 23 = 1 − P 21 = 1 − 0.015335 = 0.984665
Page 12%%%%%%%%%%%%%%%%%%%%%%%%%%%%%% — Examiners’ Report
At the 50% levels:
P 31 = P(make a claim and criminal offence involved)
= P(accident) P(criminal offence|accident) P(claim|criminal offence)
= \times  0.1P(X > 750) = 0.01 \times  e
−
75
40
= 0.0015335
P 32 = P(make a claim and no criminal offence involved) P(accident)
P(no criminal offence|accident) P(claim|no criminal offence)
= 0.1 \times  0.9 \times  P(X > 250) = 0.09 \times  e
−
25
40
= 0.0481735
P 33 = P(no claim) = 1 − P(claim)
= 1 − 0.01 \times  e
−
75
40
− 0.09 \times  e
−
25
40
= 0.950293 (rounding errors possible here)
Therefore the transition matrix is now
0.97135
0
⎛ 0.02865
⎞
⎜
⎟
0
0.984665 ⎟
P = ⎜ 0.015335
⎜ 0.0015335 0.0481735 0.950293 ⎟
⎝
⎠
For the stationary distribution we need to find \pi  = (\pi  0 , \pi  1 , \pi  2 ) such
that \pi P = \pi :
\pi  0 0.02865 + \pi  1 0.015335 + \pi  2 0.0015335 = \pi  0
\pi  0 0.97135 + 0 + \pi  2 0.481735 = \pi  1
0 + \pi  1 0.984665 + \pi  2 0.950293 = \pi  2
\pi  0 + \pi  1 + \pi  2 = 1
Further calculations show that \pi  0 = 0.002256424, \pi  1 = 0.047946812 and
\pi  2 = 0.949796764.

\end{document}
