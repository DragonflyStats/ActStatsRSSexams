\documentclass[a4paper,12pt]{article}

%%%%%%%%%%%%%%%%%%%%%%%%%%%%%%%%%%%%%%%%%%%%%%%%%%%%%%%%%%%%%%%%%%%%%%%%%%%%%%%%%%%%%%%%%%%%%%%%%%%%%%%%%%%%%%%%%%%%%%%%%%%%%%%%%%%%%%%%%%%%%%%%%%%%%%%%%%%%%%%%%%%%%%%%%%%%%%%%%%%%%%%%%%%%%%%%%%%%%%%%%%%%%%%%%%%%%%%%%%%%%%%%%%%%%%%%%%%%%%%%%%%%%%%%%%%%

\usepackage{eurosym}
\usepackage{vmargin}
\usepackage{amsmath}
\usepackage{graphics}
\usepackage{epsfig}
\usepackage{enumerate}
\usepackage{multicol}
\usepackage{subfigure}
\usepackage{fancyhdr}
\usepackage{listings}
\usepackage{framed}
\usepackage{graphicx}
\usepackage{amsmath}
\usepackage{chngpage}

%\usepackage{bigints}
\usepackage{vmargin}

% left top textwidth textheight headheight

% headsep footheight footskip

\setmargins{2.0cm}{2.5cm}{16 cm}{22cm}{0.5cm}{0cm}{1cm}{1cm}

\renewcommand{\baselinestretch}{1.3}

\setcounter{MaxMatrixCols}{10}

\begin{document}

% -------------------------- [Total 4]
An insurer’s portfolio consists of three independent policies. Each policy can give rise to at most one claim per month, which occurs with probability \theta  independently
from month to month. The prior distribution of \theta  is beta with parameters \alpha  = 2 and
\beta = 4. A total of 9 claims are observed on this portfolio over a 12 month period.

\begin{enumerate}
\item (i)  Derive the posterior distribution of \theta .
\item (ii)  Derive the Bayesian estimate of \theta  under all or nothing loss.
\end{enumerate}


3
(i)
f ( \theta  x ) \propto  f ( x \theta  ) f ( \theta  )
\propto  \theta  9 (1 − \theta  ) 27 \theta  \alpha − 1 (1 − \theta  ) \beta− 1
\propto  \theta  9 (1 − \theta  ) 27 \theta  1 (1 − \theta  ) 3
\propto  \theta  10 (1 − \theta  ) 30
Which is the pdf of a Beta(11,31) distribution.
(ii)
Under all or nothing loss, the Bayes estimate is the value that maximises the
pdf of the posterior.
f ( \theta  ) = C \times  \theta  10 (1 − \theta  ) 30
(
)
f ′ ( \theta  ) = C 10 \theta  9 (1 − \theta  ) 30 + \theta  10 \times  30(1 − \theta  ) 29 \times  − 1
= C \theta  9 (1 − \theta  ) 29 ( 10(1 − \theta  ) − 30 \theta  )
And f ′ ( \theta  ) = 0 when
10(1 − \theta  ) − 30 \theta  = 0
40 \theta  = 10
\theta  = 1/ 4
We can check this is a maximum by observing that f(0.25) > 0 whilst
f(0) = f(1) = 0. Since the maximum on [0,1] must occur either at the end-
points of the interval, or at turning point we can see that we do have a
maximum.
