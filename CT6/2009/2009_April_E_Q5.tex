% -------------------------- [Total 6]
Individual claim amounts on a particular insurance policy can take the values 100, 150 or 200.
There is at most one claim in a year. Annual premiums are 60.
The insurer must choose between three reinsurance arrangements:
A
B
C no reinsurance
individual excess of loss with retention 150 for a premium of 10
proportional reinsurance of 25% for a premium of 20
\begin{enumerate}
\item (i)  Complete the loss table for the insurer.
Claims
0
100
150
200
A

Reinsurance
B
C
\item (ii)  Determine whether any of the reinsurance arrangements is dominated from the
viewpoint of the insurer.

\item (iii)  Determine the minimax solution for the insurer.
\end{enumerate}

CT6 A2009—2


4
(i)
The loss table is as follows:
Claims
0
100
150
200
(ii)
Impact of reinsurance
A
B
C
0
−10
−20
0
5
−10
17.5
0
−10
0
40
30
A
−60
40
90
140
Insurer’s Loss
B
C
−50
−40
50
35
100
72.5
100
110
If there are no claims, A gives the best result.
If claims are 100 C gives the best result.
If claims are 200 then B gives the best result.
So each strategy can be best under certain circumstances, and so no approach
is dominated.
Page 4%%%%%%%%%%%%%%%%%%%%%%%%%%%%%% — Examiners’ Report
(iii)
The maximum losses are:
A
B
C
140
100
110
The lowest is for B, so approach B is the minimax solution.
