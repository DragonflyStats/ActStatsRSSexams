
\documentclass[a4paper,12pt]{article}

%%%%%%%%%%%%%%%%%%%%%%%%%%%%%%%%%%%%%%%%%%%%%%%%%%%%%%%%%%%%%%%%%%%%%%%%%%%%%%%%%%%%%%%%%%%%%%%%%%%%%%%%%%%%%%%%%%%%%%%%%%%%%%%%%%%%%%%%%%%%%%%%%%%%%%%%%%%%%%%%%%%%%%%%%%%%%%%%%%%%%%%%%%%%%%%%%%%%%%%%%%%%%%%%%%%%%%%%%%%%%%%%%%%%%%%%%%%%%%%%%%%%%%%%%%%%

\usepackage{eurosym}
\usepackage{vmargin}
\usepackage{amsmath}
\usepackage{graphics}
\usepackage{epsfig}
\usepackage{enumerate}
\usepackage{multicol}
\usepackage{subfigure}
\usepackage{fancyhdr}
\usepackage{listings}
\usepackage{framed}
\usepackage{graphicx}
\usepackage{amsmath}
\usepackage{chngpage}

%\usepackage{bigints}
\usepackage{vmargin}

% left top textwidth textheight headheight

% headsep footheight footskip

\setmargins{2.0cm}{2.5cm}{16 cm}{22cm}{0.5cm}{0cm}{1cm}{1cm}

\renewcommand{\baselinestretch}{1.3}

\setcounter{MaxMatrixCols}{10}

\begin{document}
CT6 S2009—610
The total number of claims N on a portfolio of insurance policies has a Poisson
distribution with mean λ. Individual claim amounts are independent of N and each
other, and follow a distribution X with mean \mu and variance \sigma^2  . The total aggregate
claims in the year is denoted by S. The random variable S therefore has a compound
Poisson distribution.
(i) Derive an expression for the moment generating function of S in terms of the
moment generating function of X.

(ii) Derive expressions for the mean and variance of S in terms of λ , \mu and \sigma  .

For a particular type of policy, individual losses are exponentially distributed with
mean 100. For losses above 200 the insurer incurs an additional expense of 50 per
claim.
(iii)
Calculate the mean and variance of S for a portfolio of such policies with
λ = 500.
[9]
[Total 19]
END OF PAPER
CT6 S2009—7

%%%%%%%%%%%%%%%%%%%%%%%%%%%
10
E ( e tS )
(i) M S ( t )
E ( E ( e t ( X 1
X 2  X N )
N )
E ( M X ( t ) N ) since the X i are independent and identically distributed
E ( e N log M X ( t ) )
M N (log M X ( t ))
exp( (exp(log( M X ( t ) 1))))
exp( ( M X ( t ) 1))
(ii) M S ( t ) M S ( t ) M X ( t )
E ( S ) M S (0) M S (0)
M X (0)
1
M S ( t ) M S ( t )
E ( S 2 ) M S '' (0)
M X ( t ) M S ( t )
M S ' (0)
1
2 2
2
(
M X ' (0) M S (0)
2
2
)
2
And so
Var( S )
E ( S 2 ) E ( S ) 2
2 2
(
Page 10
2
2
2
)
M X ( t )
2
2 2
M X '' (0)%%%%%%%%%%%%%%%%%%%%%%%%%%%%%%%%%%%%5 — %%%%%%%%%%%%%%%%%%%%%%%%%%%%%%%%%%%%5
(iii) First, we must calculate the mean and variance of a single claim, say Y.
Let us denote by X the underlying loss. Then
200
0.01 xe
E ( Y )
0.01 x
dx
( x 50) 0.01 e
0
0.01 x dx
0.01 x dx
0.01 x dx
200
0.01 xe
0.01 x
dx 50
0
0.01 e
0.01 x
dx
200
E ( X ) 50 P ( X
200)
200 0.01
= 100 50 e
100 6.76676
106.76676
200
2
0.01 x 2 e
E ( Y )
0.01 x
( x 50) 2 0.01 e
dx
0
0.01 x 2 e
200
0.01 x
dx
0
E ( X 2 )
xe
0.01 x
dx 50 2
200
100 xe
200
0.01 x
200
100 2 100 2 20,000 e
20,000 20,000 e 2
20,000 32,500 e
24,398.39671 2
0.01 e
2
100 e
0.01 x
d 50 2 P ( X
200)
200
100 2 e
10,000 e
2
0.01 x
2,500 e
200
2
2,500 e
2
And finally, using the results from part (ii)
E ( S ) 500 E ( X ) 500 106.76676
53,383.38
and
Var( S ) 500 E ( X 2 ) 500 24,398.39671 12,199,198.36
A long question about deriving mgf of a compound Poisson distribution. Mixture of
bookwork and proof, and a numerical application to finish. This question was
answered reasonably well, in particular part i) and part ii).
END OF EXAMINERS’ REPORT
Page 11
