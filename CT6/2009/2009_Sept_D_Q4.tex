\documentclass[a4paper,12pt]{article}

%%%%%%%%%%%%%%%%%%%%%%%%%%%%%%%%%%%%%%%%%%%%%%%%%%%%%%%%%%%%%%%%%%%%%%%%%%%%%%%%%%%%%%%%%%%%%%%%%%%%%%%%%%%%%%%%%%%%%%%%%%%%%%%%%%%%%%%%%%%%%%%%%%%%%%%%%%%%%%%%%%%%%%%%%%%%%%%%%%%%%%%%%%%%%%%%%%%%%%%%%%%%%%%%%%%%%%%%%%%%%%%%%%%%%%%%%%%%%%%%%%%%%%%%%%%%

\usepackage{eurosym}
\usepackage{vmargin}
\usepackage{amsmath}
\usepackage{graphics}
\usepackage{epsfig}
\usepackage{enumerate}
\usepackage{multicol}
\usepackage{subfigure}
\usepackage{fancyhdr}
\usepackage{listings}
\usepackage{framed}
\usepackage{graphicx}
\usepackage{amsmath}
\usepackage{chngpage}

%\usepackage{bigints}
\usepackage{vmargin}

% left top textwidth textheight headheight

% headsep footheight footskip

\setmargins{2.0cm}{2.5cm}{16 cm}{22cm}{0.5cm}{0cm}{1cm}{1cm}

\renewcommand{\baselinestretch}{1.3}

\setcounter{MaxMatrixCols}{10}

\begin{document}
[Total 5]
CT6 S2009—24
A portfolio consists of k independent travel insurance policies. Each policy covers the
policyholder’s trips over one year. For policy i , the number of claims in the j th month
of the covered year, Y ij , is assumed to have a distribution given by
P ( Y ij = y ) = θ ij (1 − θ ij ) y for y = 0,1, 2, ...
where θ ij are unknown constants between 0 and 1.
(i)
(ii)
Write down the likelihood function and obtain the maximum likelihood
estimate for the parameters θ ij .
Show that P ( Y ij = y ) can be written in exponential family form and suggest its
natural parameter.
(iii)


Suppose that θ ij depends on the temperature x j recorded in the j th month.
Explain why it is not appropriate to set θ ij = \alpha + \beta x j . Suggest another
relationship between θ ij and \alpha + \beta x j that might be used.

[Total 8]
4
(i) The likelihood is
k
L
i 1
12

ij (1
ij )
y ij
j 1
Where y ij is the number of claims on the ith policy in the jth month.
Taking the logarithm of L we have
k 12
log L
log
ij
y ij log(1
ij )
i 1 j 1
and so
Page 3%%%%%%%%%%%%%%%%%%%%%%%%%%%%%%%%%%%%5 — %%%%%%%%%%%%%%%%%%%%%%%%%%%%%%%%%%%%5
log L 1
ij ij
y ij
1
ij
And setting the derivative to zero we find 1 ˆ ij
ˆ
ij
(ii)
y ij ˆ ij so that
1
1 y ij
P ( Y ij
y )
ij (1
ij )
y
e
The natural parameter is log(1
x j is (
(iii) The range of
modelling parameters
ij
log
ij
y log(1
ij )
ij ) .
,
) which means it is not suitable for
[0,1] .
A possible relationship to consider is log
ij
1
x j .
ij
Other sensible alternatives should be given credit.
A question testing derivation of the m.l.e. of a non-standard p.d.f, with an example
application of the theory in the last part. The question was not answered well,
particularly part iii).
