\documentclass[a4paper,12pt]{article}

%%%%%%%%%%%%%%%%%%%%%%%%%%%%%%%%%%%%%%%%%%%%%%%%%%%%%%%%%%%%%%%%%%%%%%%%%%%%%%%%%%%%%%%%%%%%%%%%%%%%%%%%%%%%%%%%%%%%%%%%%%%%%%%%%%%%%%%%%%%%%%%%%%%%%%%%%%%%%%%%%%%%%%%%%%%%%%%%%%%%%%%%%%%%%%%%%%%%%%%%%%%%%%%%%%%%%%%%%%%%%%%%%%%%%%%%%%%%%%%%%%%%%%%%%%%%

\usepackage{eurosym}
\usepackage{vmargin}
\usepackage{amsmath}
\usepackage{graphics}
\usepackage{epsfig}
\usepackage{enumerate}
\usepackage{multicol}
\usepackage{subfigure}
\usepackage{fancyhdr}
\usepackage{listings}
\usepackage{framed}
\usepackage{graphicx}
\usepackage{amsmath}
\usepackage{chngpage}

%\usepackage{bigints}
\usepackage{vmargin}

% left top textwidth textheight headheight

% headsep footheight footskip

\setmargins{2.0cm}{2.5cm}{16 cm}{22cm}{0.5cm}{0cm}{1cm}{1cm}

\renewcommand{\baselinestretch}{1.3}

\setcounter{MaxMatrixCols}{10}

\begin{document}
8
The cumulative incurred claims for an insurance company for the last four accident
years are given in the following table:
Accident year
2005
2006
2007
2008
0
96
100
120
136
Development year
1
2
3
136
140
168
156
160
130
It can be assumed that claims are fully run off after three years. The premiums
received for each year from 2005 to 2008 are 175, 181, 190 and 196 respectively.
Calculate the reserve at the end of year 2008 using:
(a) The basic chain ladder method.
(b) The Bornhuetter-Ferguson method.
[12]
CT6 S2009—5
8
Page 7%%%%%%%%%%%%%%%%%%%%%%%%%%%%%%%%%%%%5 — %%%%%%%%%%%%%%%%%%%%%%%%%%%%%%%%%%%%5
a. The development factors are given by
\[R 1 = (136 + 156 + 130) / (96 + 100 + 120) = 1.335443\]
\[R 2 = (140 + 160) / (136 + 156) = 1.027397]\
\[R 3 = 168 / 140 = 1.2\]
The fully developed table using the chain ladder is below:

\begin{center}
\begin{tabular}{ccccc}
Incident year & 0 & 1 & 2 & 3 \\ \hline
2005 & 96 & 136 & 140 & 168\\ \hline
2006 & 100 & 156 & 160 & 192 \\ \hline
2007 & 120 & 130 & 133.56 & 160.28 \\ \hline
2008 & 136 & 181.62 & 186.60 & 223.92\\ \hline
\end{tabular}
\end{center}

R 1.335443 1.027397 1.2 1
f 1.646436 1.232876 1.2 1
Reserve = (168 + 192 + 160.28 + 223.92)
(168 + 160 + 130 + 136) = 150.2
b. B-F method
Estimated loss ratio: 168/175 = 0.96
F
1
1/f
IUL
Emerging liab.IUL(1
1/f)
2008 2007 2006 2005
1.646436 1.232876 1.2 1
0.392627 0.188888 0.1666667 0
188.16 182.4 173.76 168
73.87678 34.45325 28.96 0
Reserve is now = 73.87678 + 34.45325 + 28.96 = 137.29
A standard chainladder / Bornhuetter-Ferguson question which candidates answered
very well.
\end{document}
