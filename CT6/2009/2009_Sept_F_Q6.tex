\documentclass[a4paper,12pt]{article}

%%%%%%%%%%%%%%%%%%%%%%%%%%%%%%%%%%%%%%%%%%%%%%%%%%%%%%%%%%%%%%%%%%%%%%%%%%%%%%%%%%%%%%%%%%%%%%%%%%%%%%%%%%%%%%%%%%%%%%%%%%%%%%%%%%%%%%%%%%%%%%%%%%%%%%%%%%%%%%%%%%%%%%%%%%%%%%%%%%%%%%%%%%%%%%%%%%%%%%%%%%%%%%%%%%%%%%%%%%%%%%%%%%%%%%%%%%%%%%%%%%%%%%%%%%%%

\usepackage{eurosym}
\usepackage{vmargin}
\usepackage{amsmath}
\usepackage{graphics}
\usepackage{epsfig}
\usepackage{enumerate}
\usepackage{multicol}
\usepackage{subfigure}
\usepackage{fancyhdr}
\usepackage{listings}
\usepackage{framed}
\usepackage{graphicx}
\usepackage{amsmath}
\usepackage{chngpage}

%\usepackage{bigints}
\usepackage{vmargin}

% left top textwidth textheight headheight

% headsep footheight footskip

\setmargins{2.0cm}{2.5cm}{16 cm}{22cm}{0.5cm}{0cm}{1cm}{1cm}

\renewcommand{\baselinestretch}{1.3}

\setcounter{MaxMatrixCols}{10}

\begin{document}
PLEASE TURN OVER6
The following data is observed from n = 500 realisations from a time series:
n n
i = 1 i = 1
\sum x i = 13153.32 , \sum ( x i − x )
(i)
2
= 3153.67 and
n − 1
\sum ( x i − x )( x i + 1 − x ) = 2176.03 .
i = 1
Estimate, using the data above, the parameters \mu , a 1 and \sigma  from the model
X t − \mu = a 1 ( X t−1 − \mu ) + \varepsilon t
where \varepsilon t is a white noise process with variance \sigma^2  .
(ii)
[7]
After fitting the model with the parameters found in (i), it was calculated that
the number of turning points of the residuals series \varepsilon ˆ t is 280.
Perform a statistical test to check whether there is evidence that \varepsilon ˆ t is not
generated from a white noise process.

[Total 10]

6
(i)
13153.32
26.30644.
500
Using the known expression of the auto covariance function for AR(1)
processes: k = a 1 k , we see that
Clearly ˆ
Page 5%%%%%%%%%%%%%%%%%%%%%%%%%%%%%%%%%%%%5 — %%%%%%%%%%%%%%%%%%%%%%%%%%%%%%%%%%%%5
a 1 k ,
499
( x x )( x i 1
i 1 i
500
( x x ) 2
i 1 i
ˆ 1
x )
2176.03
3153.67
0.6899993
Taking the variance of both sides of
Xt
= a1(Xt 1
) + t
and using the fact that 0 = var(Xt
2
0 = a 1
2
0
Hence ˆ 2
)
.
ˆ 0 (1 a ( hat ) 1 2 )
ˆ
i.e.
) = var(Xt 1
3153.67
(1 0.6899993 2 ) 3.304416,
500
3.304416 1.817805
%%%%%%%%%%%%%%%%%%%%%%%%%%%%%%%%%%%%%%%%%%%%%%%%%%%%%%%%%%%%%%%%%%%%%%%%%%%%%%%%%
(ii) Using the fact that under the white noise assumptions the mean and
variance of the number of change points are
2( N 2)
(16 N 29)
3
90
= 332 and
= 88.56667
respectively where N = 500. Therefore since the 95% confidence interval is
(332 1.96
88.56667,332 1.96
88.56667) (313.6, 350.4)
which does not contain the observed number 280, there is a strong evidence
that the errors are not close to those of a white noise.
This was a slightly more complicated parameter fitting question for a time-series
model, and with a chi-square significance test to finish. The question was answered
poorly, with many candidates finding this one tough. Candidates scoring poorly for
part i) usually did not get ii) out as well.

%%%%%%%%%%%%%%%%%%%%%%%%%%%%%%%%%%%%%%%%%%%%%%%%%%%%%%%%%%%%%%%%%%%%%%%%%%%%%%%%%%%%%%%%%%%%%%%%%%%%%%%%%
\end{document}
