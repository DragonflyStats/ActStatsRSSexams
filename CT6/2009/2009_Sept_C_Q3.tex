\documentclass[a4paper,12pt]{article}

%%%%%%%%%%%%%%%%%%%%%%%%%%%%%%%%%%%%%%%%%%%%%%%%%%%%%%%%%%%%%%%%%%%%%%%%%%%%%%%%%%%%%%%%%%%%%%%%%%%%%%%%%%%%%%%%%%%%%%%%%%%%%%%%%%%%%%%%%%%%%%%%%%%%%%%%%%%%%%%%%%%%%%%%%%%%%%%%%%%%%%%%%%%%%%%%%%%%%%%%%%%%%%%%%%%%%%%%%%%%%%%%%%%%%%%%%%%%%%%%%%%%%%%%%%%%

\usepackage{eurosym}
\usepackage{vmargin}
\usepackage{amsmath}
\usepackage{graphics}
\usepackage{epsfig}
\usepackage{enumerate}
\usepackage{multicol}
\usepackage{subfigure}
\usepackage{fancyhdr}
\usepackage{listings}
\usepackage{framed}
\usepackage{graphicx}
\usepackage{amsmath}
\usepackage{chngpage}

%\usepackage{bigints}
\usepackage{vmargin}

% left top textwidth textheight headheight

% headsep footheight footskip

\setmargins{2.0cm}{2.5cm}{16 cm}{22cm}{0.5cm}{0cm}{1cm}{1cm}

\renewcommand{\baselinestretch}{1.3}

\setcounter{MaxMatrixCols}{10}

\begin{document}
3 The loss function under a decision problem is given by:
d 1
d 2
d 3
d 4
θ 1
10
8
12
5
θ 2
15
20
15
23
θ 3
5
15
10
8
where d 1, d 2, d 3 and d 4 are the possible decisions and θ 1 , θ 2 and θ 3 are the possible
states of nature.
(i) State which decision can be discounted immediately. 
(ii) Determine the minimax solution to the problem. 
(iii) Determine the Bayes criterion solution to the problem given that P ( θ 1 ) = 0.4 ,
P ( θ 2 ) = 0.25 and P ( θ 3 ) = 0.35 .

3
(i) Decision d 3 is dominated by d 1 and can be discounted immediately.
(ii) Maximum losses are:
d 1 15
d 2 20
d 4 23
So the minimax solution is to choose d 1
(iii) Expected losses are given by:
E ( L ( d 1 )) 0.4 10 0.25 15 0.35 5 9.5
E ( L ( d 2 )) 0.4 8 0.25 20 0.35 15 13.45
E ( L ( d 4 )) 0.4 5 0.25 23 0.35 8 10.55
So the Bayes solution is also to choose d 1 .
A straightforward question involving outcomes of 3 decision functions and requiring
the candidates to derive the minimax solution and the Bayes criterion solution. This
question was answered very well by most candidates.
