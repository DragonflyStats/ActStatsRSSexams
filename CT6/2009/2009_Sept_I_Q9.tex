
\documentclass[a4paper,12pt]{article}

%%%%%%%%%%%%%%%%%%%%%%%%%%%%%%%%%%%%%%%%%%%%%%%%%%%%%%%%%%%%%%%%%%%%%%%%%%%%%%%%%%%%%%%%%%%%%%%%%%%%%%%%%%%%%%%%%%%%%%%%%%%%%%%%%%%%%%%%%%%%%%%%%%%%%%%%%%%%%%%%%%%%%%%%%%%%%%%%%%%%%%%%%%%%%%%%%%%%%%%%%%%%%%%%%%%%%%%%%%%%%%%%%%%%%%%%%%%%%%%%%%%%%%%%%%%%

\usepackage{eurosym}
\usepackage{vmargin}
\usepackage{amsmath}
\usepackage{graphics}
\usepackage{epsfig}
\usepackage{enumerate}
\usepackage{multicol}
\usepackage{subfigure}
\usepackage{fancyhdr}
\usepackage{listings}
\usepackage{framed}
\usepackage{graphicx}
\usepackage{amsmath}
\usepackage{chngpage}

%\usepackage{bigints}
\usepackage{vmargin}

% left top textwidth textheight headheight

% headsep footheight footskip

\setmargins{2.0cm}{2.5cm}{16 cm}{22cm}{0.5cm}{0cm}{1cm}{1cm}

\renewcommand{\baselinestretch}{1.3}

\setcounter{MaxMatrixCols}{10}

\begin{document}

PLEASE TURN OVER9
A certain proportion p of electrical gadgets produced by a factory is defective. Prior
beliefs about p are represented by a Beta distribution with parameters \alpha and \beta . A
sample of n gadgets is inspected, and k are found to be defective.
(i) Explain what is meant by a conjugate prior distribution. 
(ii) Derive the posterior distribution for beliefs about p . 
(iii) ⎛ 1
Show that if X ~ Beta( \alpha , \beta ) with \alpha > 1 then E ⎜
⎝ X
(iv)
⎞ \alpha + \beta − 1
.
⎟ =
\alpha − 1
⎠
It is required to make an estimate d of p. The loss function is given by
( d − p ) 2
L ( d , p ) =
.
p
Determine the Bayes estimate d* of p.
(v)


Determine a parameter Z such that d* can be written as
k
1
d* = Z × + (1 − Z ) ×
n
\mu 
where \mu is the prior expectation of 1/p.
(vi)

\alpha+ k
.
\alpha +β+ n
Comment on the difference in the two Bayes’ estimates in the specific case
where \alpha = \beta = 3 , k = 2 and n = 10.

[Total 15]
Under quadratic loss, the Bayes estimate would have been

%%%%%%%%%%%%%%%%%%%%
9
(i) A distribution is a conjugate prior for an unknown parameter if when used
as a prior distribution for that parameter it leads to a posterior distribution
which is from the same family.
(ii) f ( p k ) f ( k p ) f ( p )
p k (1 p ) n
p
Page 8
k 1
k
p
(1 p )
1
(1 p )
n k 1
1%%%%%%%%%%%%%%%%%%%%%%%%%%%%%%%%%%%%5 — %%%%%%%%%%%%%%%%%%%%%%%%%%%%%%%%%%%%5
k ,
Which is the pdf of a Beta(
(iii)
1
1
1 (
)
E
x
X
x ( ) ( )
n k ) distribution.
1
1
(1 x )
dx
0
1
(
)
x
( ) ( )
0
( 1) (
( ) (
(
(
1
(1 x )
)
1)
1) (
1) (
1
2
1
0
1
(
(
dx
1)
x
1) ( )
1) (
1) (
2
(1 x )
1
dx
1)
1
1)
.
Other derivations are acceptable.
(iv) Let
h ( d ) E ( L ( d , p ))
h ( d )
E
( d
p ) 2
p
E
d 2
2 dp
p
p 2
d 2 E (1/ p ) 2 d
E ( p )
h ( d ) 2 dE (1/ p ) 2
And
h ( d ) 0
2 d * E (1/ p )
When
2
k 1
n 1
d * 1/ E (1/ p )
Using the result from (iii) applied to the posterior distribution for p.
(v)
k 1
n 1
x
(1 Z ) Z
n
n
Where Z
1
d *
(vi)
1
1
n 1
and
k
n 1 n
n
n 1
is the prior expectation of 1/p.D
The estimates are:
Using the given loss function the estimate is
Page 9%%%%%%%%%%%%%%%%%%%%%%%%%%%%%%%%%%%%5 — %%%%%%%%%%%%%%%%%%%%%%%%%%%%%%%%%%%%5
(3 + 2
1) / (3 + 3 + 10
1) = 4/15 = 0.266666
Using Bayesian loss, we have (3 + 2) / (3 + 3 + 10) = 5/16 = 0.3125.
The mean of the prior is 0.5 and the observed sample mean is 0.2. The loss
function in (iv) penalises mis-estimates particularly when the true value of p is
lower. This means that the estimate in (iv) is lower than would result from
straight quadratic loss.
A longer Bayes question with derivation of a posterior Beta distribution, a
credibility factor Z and consideration of a non-standard loss function (given) and
the quadratic loss. There was a wide range of quality of answers for this 6 part
question. Generally i) and ii) were answered well, iii) to v) proved trickier, in
particular deriving d* in part v).
