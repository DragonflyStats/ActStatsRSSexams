\documentclass[a4paper,12pt]{article}

%%%%%%%%%%%%%%%%%%%%%%%%%%%%%%%%%%%%%%%%%%%%%%%%%%%%%%%%%%%%%%%%%%%%%%%%%%%%%%%%%%%%%%%%%%%%%%%%%%%%%%%%%%%%%%%%%%%%%%%%%%%%%%%%%%%%%%%%%%%%%%%%%%%%%%%%%%%%%%%%%%%%%%%%%%%%%%%%%%%%%%%%%%%%%%%%%%%%%%%%%%%%%%%%%%%%%%%%%%%%%%%%%%%%%%%%%%%%%%%%%%%%%%%%%%%%

\usepackage{eurosym}
\usepackage{vmargin}
\usepackage{amsmath}
\usepackage{graphics}
\usepackage{epsfig}
\usepackage{enumerate}
\usepackage{multicol}
\usepackage{subfigure}
\usepackage{fancyhdr}
\usepackage{listings}
\usepackage{framed}
\usepackage{graphicx}
\usepackage{amsmath}
\usepackage{chngpage}

%\usepackage{bigints}
\usepackage{vmargin}

% left top textwidth textheight headheight

% headsep footheight footskip

\setmargins{2.0cm}{2.5cm}{16 cm}{22cm}{0.5cm}{0cm}{1cm}{1cm}

\renewcommand{\baselinestretch}{1.3}

\setcounter{MaxMatrixCols}{10}

\begin{document}
5
The following claim amounts are believed to come from a lognormal distribution with
unknown parameters \mu and \sigma^2  :
50, 87, 103, 119, 126, 154, 183, 203
Estimate the parameters \mu and \sigma^2  using:
(i) the method of moments;
(ii) the method of percentiles, using the upper and lower quartiles.
CT6 S2009—3


%%%%%%%%%%%%%%%%%%%%%%%%%%%%%%%%%%%%%%%%%%%%%%%%%%%%%%%%%%%%%%%%%%%%%%%%%%%%%%%%%

5
(i) For the given sample
8
i 1
8
i 1
x i
8 128.125
x i 2
8 18, 641.125
From the tables:
2
2
E ( X ) e
E ( X 2 )
e
2
1 E ( X ) 2
E ( X ) 2
E ( X ) 2 e
Substituting into the second of these, we have:
18, 641.125 128.125 2 e
2
log
18, 641.125
128.125 2
2
0.12711274
And substituting back into the first expression
Page 4
2%%%%%%%%%%%%%%%%%%%%%%%%%%%%%%%%%%%%5 — %%%%%%%%%%%%%%%%%%%%%%%%%%%%%%%%%%%%5
2
2
128.125 e
2
2
log128.125
log128.125 0.5 0.12711274 4.78945
(ii) The lower and upper quartile points in the data set are 95 and 168.5
We need to solve:
0.6745
e
168.5 and e
0.6745
95
Dividing the first by the second gives:
e 2
0.6745
1.773684211
log1.773684211
0.424802711
2 0.6745
2
0.180457343
And substituting back into the first equation:
e
0.6745
168.5
log168.5 0.6745 0.424802711
4.840406
It is possible to use other definitions of upper and lower quartile. Other sensible
choices were given full credit provided the subsequent calculations followed
through correctly
A numerical question that tested the theory of fitting a distribution using two different
methods to sample data. This question was answered reasonably well, with some
candidates scoring very highly indeed. This question was a good differentiator with
strong candidates showing they had learnt the theory of distribution fitting thoroughly
and accurately calculating the answers. NB Both percentile definitions as per CT3
were given credit.
