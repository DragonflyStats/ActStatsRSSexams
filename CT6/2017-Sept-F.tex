\documentclass[a4paper,12pt]{article}

%%%%%%%%%%%%%%%%%%%%%%%%%%%%%%%%%%%%%%%%%%%%%%%%%%%%%%%%%%%%%%%%%%%%%%%%%%%%%%%%%%%%%%%%%%%%%%%%%%%%%%%%%%%%%%%%%%%%%%%%%%%%%%%%%%%%%%%%%%%%%%%%%%%%%%%%%%%%%%%%%%%%%%%%%%%%%%%%%%%%%%%%%%%%%%%%%%%%%%%%%%%%%%%%%%%%%%%%%%%%%%%%%%%%%%%%%%%%%%%%%%%%%%%%%%%%

\usepackage{eurosym}
\usepackage{vmargin}
\usepackage{amsmath}
\usepackage{graphics}
\usepackage{epsfig}
\usepackage{enumerate}
\usepackage{multicol}
\usepackage{subfigure}
\usepackage{fancyhdr}
\usepackage{listings}
\usepackage{framed}
\usepackage{graphicx}
\usepackage{amsmath}
\usepackage{chngpage}

%\usepackage{bigints}
\usepackage{vmargin}

% left top textwidth textheight headheight

% headsep footheight footskip

\setmargins{2.0cm}{2.5cm}{16 cm}{22cm}{0.5cm}{0cm}{1cm}{1cm}

\renewcommand{\baselinestretch}{1.3}

\setcounter{MaxMatrixCols}{10}

\begin{document}

\begin{enumerate}
11
(i)
Explain what is meant by a conjugate prior distribution.
[1]
The random variables X 1 , X 2 , ..., X n are independent and have density function:
P ( X = x ) = p (1 − p ) x , 0 < p < 1
(ii)
Show that the conjugate prior for p is a beta distribution.
[3]
Assume that we have an independent sample X 1 , X 2 , ..., X n from a geometric
distribution with parameter p, with the prior density function for p given by:
f ( p ) =
(iii)
(iv)
Γ( α + β ) α− 1
p ( 1 − p ) β− 1 , 0 < p < 1
Γ( α )Γ ( β )
 1 − p 
β
Show that E 
.
 =
 p  ( α − 1)
Show that the posterior mean of this distribution can be expressed as a
weighted average of the prior mean and the sample average, including
statement of the credibility factor Z.
[3]
[2]
Every day Amit and Bonnie catch the bus home from work at a bus stop next to their
office. Most buses which arrive at the bus stop do not go to their destination. Denote
the average number of buses they have to wait for as N such that the (N+1) th bus to
arrive at the bus stop goes to their destination. Amit’s prior belief is that N = 10.
Bonnie’s prior belief is that N = 5. They both use a beta prior distribution with α = 5
but with different β .
(v)
Calculate the number of bus trips home required such that the absolute
difference in Amit and Bonnie’s posterior estimates for N is less than 0.5. [5]
[Total 14]
END OF PAPER
CT6 S2017–7
%%%%%%%%%%%%%%%%%%%%%%%%%
Page 10Subject CT6 (Statistical Methods Core Technical) – September 2017 – Examiners’ Report
Q11
(i)
(ii)
A prior distribution is a conjugate prior if the resulting posterior distribution
belongs to the same family as the prior distribution.
[1]
Assume f prior  p  
Γ     
Γ    Γ   
likelihood given by L  p  
 1
p  1  1  p 
n
 p  1  p 
x i
 p n  1  p 
 x i
[1]
i  1
posterior  p n  1  1  p 
 x i  1
[1]
which is in the form of a beta distribution with parameters n   and  x i  
[1]
(iii)
1
 1  p         1  p   1
 1
E 



 p  1  p  dp

 p          0  p 

          1      1  1
       
     
[1⁄2]
     

 2
     1      1  p  1  p  dp
[11⁄2]
0
since the integrand is 1 (as it is a beta distribution), this is
          1      1 
       
(iv)
     


   1 
[1]
From part (ii), know that the posterior distribution of p is beta with parameters
n   and  x i  
from part (iii), know the posterior mean is
We can express this as
 x i  
 n    1 
 x i
n

a  1

n n  a  1    1  n  a  1
Which is in the form of a credibility estimate with Z 
(v)
[1⁄2]
[1]
n
n  a  1
[1⁄2]
 1  p 
For Amit,  A  E 
 *    1   10*4  40
 p  [1⁄2]
 1  p 
For Bonnie,  B  E 
 *    1   5*4  20
 p  [1⁄2]
Page 11Subject CT6 (Statistical Methods Core Technical) – September 2017 – Examiners’ Report
Want the difference in posterior mean to be less than 0.01, and the first term in
the credibility estimate is the same for both since it is independent of β . [1]
So want
i.e.
 A

a  1
a  1
 B
 0.5
   1  n  a  1    1  n  a  1
40
20

 0.5
n  4 n  4
[1]
[1⁄2]
i.e. 20  0.5  n  4  [1⁄2]
so n  36 i.e. n  37 [1]
Most candidates were able to score well on parts (i) and (ii), and well-
prepared candidates also scored well on part (iii).
Unfortunately, there was an ambiguity in part (iv), which should have
referred explicitly to the posterior distribution derived in part (ii). Full
credit was given to any reasonable attempt whilst using the original
distribution.
Part (v) was particularly difficult for those who had not derived the
intended credibility estimate in part (iv), and only the very strongest
candidates were able to score highly here. This has been reflected in
the lower pass mark for this diet.
END OF EXAMINERS’ REPORT
Page 12
