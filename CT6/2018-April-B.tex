3
An insurance company has collected data on the number of claims arising from
certain risks over the last n years. The number of claims from the i th risk in the j th
year is denoted by X ij for i = 1, 2, ..., N and j = 1, 2, ..., n.
The distribution of X ij depends on an unknown parameter q i . The X ij are independent
identically distributed random variables given q i .
(i) Describe briefly what is meant by each of the following: m(q), s 2 (q), E(s 2 (q)),
var(m(q)), and Z, when using Empirical Bayes Credibility Theory (EBCT)
Model 1.
[5]
(ii) Explain how the value of Z depends on the following factors: n, E(s 2 (q)),
var(m(q)).[5]
[Total 10]

CT6 A2018–24
The table below shows the cumulative incurred claims by year for a portfolio of
general insurance policies, with all figures in £m. Claims paid to date total 13.5. The
ultimate loss ratio is expected to be in line with the 2013 accident year, and claims are
assumed to be fully developed by the end of Development Year 3.
Development Year
Accident Year 0 1 2 3 Earned Premiums
2013 3.01 3.38 3.85 4.00 4.32
2014 3.30 3.67 4.15 2015 3.32 3.86 2016 3.74
4.41
4.55
4.68
Calculate the total reserve required to meet the outstanding claims, using the
Bornheutter-Ferguson method.
Q3
(i)
m ( θ ) is the average claim amount for each risk for a given value of θ i
[1]
s 2 ( θ ) is the variance of the claim amount for each risk given a value of θ i [1]
E ( s 2 ( θ ) ) is the average variability of data values from year to year for a single
risk, I [1]
var( m ( θ ) ) is the variability of the average data values for different risks [1]
Z is the credibility factor for the EBCT 1 model / weight placed on the sample
mean
[1]
(ii)
(a)
(b)
Z increases as n increases, since we place more weight on the data for
that risk
[1]
(
)
As E s 2 ( θ ) increases, Z decreases since the variance of the data
from the individual risk is high and so we place more weight on the
collective data.
[2]
(c)
(
)
As var m ( θ ) increases, Z increases since it implies that the means of
the individual risks are very different, so we place more weight on the
individual risk data compared to the collective.
[2]
[Total 10]
Most candidates scored well here, although some lost marks through a
combination of only using formulae in part (i) and giving insufficient
explanations in part (ii).
Page 4Subject CT6 (Statistical Methods Core Technical) – April 2018 – Examiners’ Report
Q4
Ultimate loss ratio = 4/4.32 = 92.5926% [1]
DF3 = 4/3.85 = 1.03896 [1]
DF2 = (3.85+4.15)/(3.38+3.67) = 1.134752 [1]
DF1 = (3.38+3.67+3.86)/(3.01+3.3+3.32) = 1.132918 [1]
Adjusted expected ultimate claim for
AY2 = 4.15+0.925926*4.41*(1 – 1/1.03896) = 4.3031 [11⁄2]
Adjusted expected ultimate claim for AY3
= 3.86+0.925926*4.55*(1 – 1/(1.03896*1.134751)) = 4.4995 [11⁄2]
Adjusted expected ultimate claim for AY4
= 3.74+0.925926*4.68*(1–1/(1.03896*1.134751*1.132918)) = 4.8290 [11⁄2]
So reserve = 4 + 4.3031 + 4.4995 + 4.8290 – 13.5 = 4.13m
[1⁄2]
[Total 9]
Most candidates scored very well on this straightforward chain-ladder
question, although some candidates did not appear to know the method
required.
