\documentclass[a4paper,12pt]{article}



%%%%%%%%%%%%%%%%%%%%%%%%%%%%%%%%%%%%%%%%%%%%%%%%%%%%%%%%%%%%%%%%%%%%%%%%%%%%%%%%%%%%%%%%%%%%%%%%%%%%%%%%%%%%%%%%%%%%%%%%%%%%%%%%%%%%%%%%%%%%%%%%%%%%%%%%%%%%%%%%%%%%%%%%%%%%%%%%%%%%%%%%%%%%%%%%%%%%%%%%%%%%%%%%%%%%%%%%%%%%%%%%%%%%%%%%%%%%%%%%%%%%%%%%%%%%
  
  
  
  \usepackage{eurosym}

\usepackage{vmargin}

\usepackage{amsmath}

\usepackage{graphics}

\usepackage{epsfig}

\usepackage{enumerate}

\usepackage{multicol}

\usepackage{subfigure}

\usepackage{fancyhdr}

\usepackage{listings}

\usepackage{framed}

\usepackage{graphicx}

\usepackage{amsmath}

\usepackage{chngpage}



%\usepackage{bigints}

\usepackage{vmargin}



% left top textwidth textheight headheight

% headsep footheight footskip

\setmargins{2.0cm}{2.5cm}{16 cm}{22cm}{0.5cm}{0cm}{1cm}{1cm}

\renewcommand{\baselinestretch}{1.3}

\setcounter{MaxMatrixCols}{10}

\begin{document}

\begin{enumerate}
1 John enjoys playing two player zero-sum games.
The matrix below shows the losses to John in a particular two player zero-sum game.
His strategies are denoted by I, II and III, whereas the strategies for his opponent are
denoted by a, b and c.
3 2 4
0 1 1
1 3 0
I II III
a
b
c
−
−
(i) Explain which of John’s strategies is dominated. [2]
The opponent now has the option of a fourth strategy, d, which results in none of
John’s strategies being dominated.
(ii) Suggest possible values for the strategy d. [1]
[Total 3]
2 Claim amounts Xi from a portfolio of insurance policies follow a gamma distribution
with parameters k and λi. Each λi also follows a gamma distribution with parameters
α and β.
(i) Show that the mixture distribution of losses is a generalised Pareto, with
parameters α, β, k. [4]
Claim amounts are now assumed to be exponentially distributed with parameter λi.
(ii) Show, using your answer to part (i), that the mixture distribution of losses is
now a standard Pareto distribution with parameters α, β. [2]
%%%%%%%%%%%%%%%%%%%%%%%%%%%%%%%%%%%%%%%%%%%%%%%%%%%%%%%%%%%%%%%%%%%%%%%%%%%%
  Solutions
Q1 (i) Strategy II is dominated since strategy I is better under all the opponent’s
strategies. [2]
(ii) e.g. d = 2, 0, 0 [1]
[Total 3]
This straightforward question was very well answered by the vast majority of
candidates.
Q2 (i)   ,  
0
fX x fX x, d

       |  
0
f fX x| d

      [½]
        1 1
0
exp
Γ Γ
k
xk exp x d
k
 
   
   
 
   
 
 
  
      1
1
0
Γ
exp
Γ Γ Γ
k k
k
k
x k x x d
k x k
   
 

    
      
      . [2]
The integral sums to 1 so we are left with
 
     
Γ 1
, 0
Γ Γ
k
k
k x x
k x
 

  

  
[1]
Which is the PDF of the Generalised Pareto distribution. [½]
(ii) The Exponential is a special case of the Gamma distribution with k = 1. [½]
So the mixture distribution is
 
       
1 1
1 1
Γ 1
Γ Γ 1
x
x x
  
 
   

    
[1]
Which is the PDF of a Pareto distribution with parameters  and  . [½]
[Total 6]
Candidates who were familiar with the necessary bookwork were able to
score very well here, although a disappointing number were not. Most
candidates spotted the first step in part (ii) – setting k equal to 1.
Subject CT6 (Statistical Methods Core Technical) – April 2017 – Examiners’ Report
Page 4
