\documentclass[a4paper,12pt]{article}

%%%%%%%%%%%%%%%%%%%%%%%%%%%%%%%%%%%%%%%%%%%%%%%%%%%%%%%%%%%%%%%%%%%%%%%%%%%%%%%%%%%%%%%%%%%%%%%%%%%%%%%%%%%%%%%%%%%%%%%%%%%%%%%%%%%%%%%%%%%%%%%%%%%%%%%%%%%%%%%%%%%%%%%%%%%%%%%%%%%%%%%%%%%%%%%%%%%%%%%%%%%%%%%%%%%%%%%%%%%%%%%%%%%%%%%%%%%%%%%%%%%%%%%%%%%%

\usepackage{eurosym}
\usepackage{vmargin}
\usepackage{amsmath}
\usepackage{graphics}
\usepackage{epsfig}
\usepackage{enumerate}
\usepackage{multicol}
\usepackage{subfigure}
\usepackage{fancyhdr}
\usepackage{listings}
\usepackage{framed}
\usepackage{graphicx}
\usepackage{amsmath}
\usepackage{chng%%-- Page}
%\usepackage{bigints}
\usepackage{vmargin}

% left top textwidth textheight headheight

% headsep footheight footskip
\setmargins{2.0cm}{2.5cm}{16 cm}{22cm}{0.5cm}{0cm}{1cm}{1cm}
\renewcommand{\baselinestretch}{1.3}
\setcounter{MaxMatrixCols}{10}
\begin{document} 

%%% - Question 3

[Total 5]
State the Markov property and explain briefly whether the following processes are
Markov:
AR(4);
ARMA (1, 1).
[5]


3
The Markov property for a process {Y t } states that the conditional distribution of
Y t |Y t−1 is the same as the conditional distribution of
Y t |Y t−1 , Y t−2 , ...
Development can be predicted from present state without any reference to past
history.
AR(4)
Y t = \alpha  + \beta  1 Y t−1 + \beta  2 Y t−2 + \beta  3 Y t−3 + \beta  4 Y t−4 + e t
This is not Markov since the distribution of Y t |Y t−1 changes when Y t−2 , Y t−3 , Y t−4 are
also given.
ARMA(1,1)
Y t = \alpha  + \beta Y t−1 + e t − \theta e t−1
Y t−1 = \alpha  + \beta Y t−2 + e t−1 − \theta e t−2
Hence e t−1 = Y t−1 − \alpha  − \beta Y t−2 + \theta e t−2 , and substituting into the expression for Y t , it
can be seen that knowledge of Y t−2 changes the distribution of Y t |Y t−1 . So this is not
Markov.
