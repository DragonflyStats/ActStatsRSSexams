\documentclass[a4paper,12pt]{article}

%%%%%%%%%%%%%%%%%%%%%%%%%%%%%%%%%%%%%%%%%%%%%%%%%%%%%%%%%%%%%%%%%%%%%%%%%%%%%%%%%%%%%%%%%%%%%%%%%%%%%%%%%%%%%%%%%%%%%%%%%%%%%%%%%%%%%%%%%%%%%%%%%%%%%%%%%%%%%%%%%%%%%%%%%%%%%%%%%%%%%%%%%%%%%%%%%%%%%%%%%%%%%%%%%%%%%%%%%%%%%%%%%%%%%%%%%%%%%%%%%%%%%%%%%%%%

\usepackage{eurosym}
\usepackage{vmargin}
\usepackage{amsmath}
\usepackage{graphics}
\usepackage{epsfig}
\usepackage{enumerate}
\usepackage{multicol}
\usepackage{subfigure}
\usepackage{fancyhdr}
\usepackage{listings}
\usepackage{framed}
\usepackage{graphicx}
\usepackage{amsmath}
\usepackage{chng%%-- Page}
%\usepackage{bigints}
\usepackage{vmargin}

% left top textwidth textheight headheight

% headsep footheight footskip
\setmargins{2.0cm}{2.5cm}{16 cm}{22cm}{0.5cm}{0cm}{1cm}{1cm}
\renewcommand{\baselinestretch}{1.3}
\setcounter{MaxMatrixCols}{10}
\begin{document} 

%%% - Question 2
A sequence of pseudo-random numbers from a uniform distribution over the interval
[0, 1] has been generated by a computer.
(i) Explain the advantage of using pseudo-random numbers rather than generating
a new set of random numbers each time.

(ii) Use examples to explain how a sequence of pseudo-random numbers can be
used to simulate observations from:
(a)
(b)
3
1
a continuous distribution
a discrete distribution


%%%%%%%%%%%%%%%%%%%%%%%%%%%%%%%%%%%%%%%%%%%%%%%%%%%%%%%%%%%
\newpage


2
(i)
Using pseudo-random numbers removes the variability of using different sets of random numbers, which is helpful for comparing different models.
Only a single routine is required for generation of pseudo-random numbers whereas in the case of truly random numbers we need either a lengthy table or
a hardware enhancement to a computer.
If we wish to use the same sequence of random numbers in 2 models we need only store the seed for the pseudo-random random numbers as opposed to a
record of potentially millions of truly random numbers.
(ii)
u is a random number from U(0, 1)
(a)
Find x from u = F(x)
so x = F −1 (u).
e.g. exponential u = e −\lambda x
x =
(b)
− log u
\lambda 
Working with integers, find x such that P(X \leq  x − 1) < u \leq  P(X \leq  x)
e.g. toss a coin, X = number of heads
X = 0 if u \leq  0.5, X = 1 otherwise
Page 4 — %%%%%%%%%%%%%%%%%%%%%%%%%%%%%%%%%%%%%%%%%%%%%%%%%%%%%%
