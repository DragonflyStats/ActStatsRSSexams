\documentclass[a4paper,12pt]{article}

%%%%%%%%%%%%%%%%%%%%%%%%%%%%%%%%%%%%%%%%%%%%%%%%%%%%%%%%%%%%%%%%%%%%%%%%%%%%%%%%%%%%%%%%%%%%%%%%%%%%%%%%%%%%%%%%%%%%%%%%%%%%%%%%%%%%%%%%%%%%%%%%%%%%%%%%%%%%%%%%%%%%%%%%%%%%%%%%%%%%%%%%%%%%%%%%%%%%%%%%%%%%%%%%%%%%%%%%%%%%%%%%%%%%%%%%%%%%%%%%%%%%%%%%%%%%

\usepackage{eurosym}
\usepackage{vmargin}
\usepackage{amsmath}
\usepackage{graphics}
\usepackage{epsfig}
\usepackage{enumerate}
\usepackage{multicol}
\usepackage{subfigure}
\usepackage{fancyhdr}
\usepackage{listings}
\usepackage{framed}
\usepackage{graphicx}
\usepackage{amsmath}
\usepackage{chngpage}
%\usepackage{bigints}
\usepackage{vmargin}

% left top textwidth textheight headheight

% headsep footheight footskip
\setmargins{2.0cm}{2.5cm}{16 cm}{22cm}{0.5cm}{0cm}{1cm}{1cm}
\renewcommand{\baselinestretch}{1.3}
\setcounter{MaxMatrixCols}{10}



\begin{document}
5
k = 5
64%
13%
3%
4%
5%
12%
9%
4%
6%
4%
in the
[4]
[Total 8]
(i) Let n be an integer and suppose that X 1 , X 2 , , X n are independent random
variables each having an exponential distribution with parameter . Show that
Z = X 1 + + X n has a Gamma distribution with parameters n and .
[2]
(ii) By using this result, generate a random sample from a Gamma distribution
with mean 30 and variance 300 using the 5 digit pseudo-random numbers.
63293
CT6 A2006 3
43937
08513
[5]
[Total 7]
PLEASE TURN OVER6
An insurance company has a set of n risks (i = 1, 2, , n) for which it has recorded
the number of claims per month, Y ij , for m months (j = 1, 2, , m).
It is assumed that the number of claims for each risk, for each month, are independent
Poisson random variables with
E[Y ij ] =
ij .
These random variables are modelled using a generalised linear model, with
log
ij
=
i
(i = 1, 2,
(i) Derive the maximum likelihood estimator of (ii) Show that the deviance for this model is
n
m
2
y ij log
i 1 j 1
where y i =
7
, n)
1
m
y ij
y i
( y ij
i .
[4]
y i )
m
y ij .
[3]
j 1
(iii) A company has data for each month over a 2 year period. For one risk, the
average number of claims per month was 17.45. In the most recent month for
this risk, there were 9 claims. Calculate the contribution that this observation
makes to the deviance.
[3]
[Total 10]


5
(i)
April 2006
Each X i has moment generating function M X i ( t ) =
Examiners Report
t
. Hence
n
M Z (t) = M X 1
... X n ( t )
n
= M X i ( t ) =
t
which is the moment generating function of a gamma distribution with parameters n and and hence Z has this distribution.
(ii)
= 30,
2
= 3 and
= 300
= 0.1
The random sample can be generated by producing three independent samples from an Exponential distribution with parameter 0.1 and adding them together.

To do this, we need to solve
F X (x) = 1 - e -0.1x = u
where u is a pseudo-random number from a U(0, 1) distribution.

Solving, we have x =
log(1 u )
0.1
So using our pseudo-random numbers to give the exponential samples we have:
u = 0.63292
u = 0.43937
u = 0.08513
x = 10.022
x = 5.787
x = 0.890
and the sample from the gamma distribution is
10.022 + 5.787 + 0.890 = 16.699.
Comments on question 5: Part (i) was well answered but most candidates failed to generate
the required random sample in part (ii).
%%-- Page 5Subject CT6 (Statistical Methods Core Technical)
y ij
6
(i)
ij
The likelihood is
e
April 2006
Examiners Report
ij
y ij !
i , j
and the loglikelihood is
n
m
( y ij log
l =
ij
ij
log y ij !)
i 1 j 1
Hence
n
m
( y ij
l =
e i
me i
i
log y ij !)
i 1 j 1
m
l
y ij
=
i
l
j 1
= 0
e
i
i
and
(ii)
i
j 1
n m
y ij
= log y i
The deviance is
n
2(l f
m
l c ) = 2
( y ij log y ij
y ij )
i 1 j 1
n
y ij log
i 1 j 1
( y ij log y i
y i )
i 1 j 1
m
= 2
(iii)
m
1
= y i , where y i =
m
y ij
y i
( y ij
y i )
The deviance is
D ij = y ij log
y ij
y i
( y ij
y i ) = 9 log
9
(9 17.45)
17.45
= 2.491
Full credit should also be given to 2 x 2.491 = 4.98
Comments on question 6: Most candidates scored well on (i). Only more able candidates
scored well on parts (ii) and (iii). There were some relatively easy marks available in (iii) for
applying data to the formula given in (ii).
%%-- Page 6
%%-- Subject CT6 (Statistical Methods Core Technical)
\end{document}
