\documentclass[a4paper,12pt]{article}

%%%%%%%%%%%%%%%%%%%%%%%%%%%%%%%%%%%%%%%%%%%%%%%%%%%%%%%%%%%%%%%%%%%%%%%%%%%%%%%%%%%%%%%%%%%%%%%%%%%%%%%%%%%%%%%%%%%%%%%%%%%%%%%%%%%%%%%%%%%%%%%%%%%%%%%%%%%%%%%%%%%%%%%%%%%%%%%%%%%%%%%%%%%%%%%%%%%%%%%%%%%%%%%%%%%%%%%%%%%%%%%%%%%%%%%%%%%%%%%%%%%%%%%%%%%%

\usepackage{eurosym}
\usepackage{vmargin}
\usepackage{amsmath}
\usepackage{graphics}
\usepackage{epsfig}
\usepackage{enumerate}
\usepackage{multicol}
\usepackage{subfigure}
\usepackage{fancyhdr}
\usepackage{listings}
\usepackage{framed}
\usepackage{graphicx}
\usepackage{amsmath}
\usepackage{chng%%-- Page}
%\usepackage{bigints}
\usepackage{vmargin}

% left top textwidth textheight headheight

% headsep footheight footskip
\setmargins{2.0cm}{2.5cm}{16 cm}{22cm}{0.5cm}{0cm}{1cm}{1cm}
\renewcommand{\baselinestretch}{1.3}
\setcounter{MaxMatrixCols}{10}



\begin{document}
6
An insurance company has a set of n risks (i = 1, 2, , n) for which it has recorded
the number of claims per month, y_{ij} , for m months (j = 1, 2, , m).
It is assumed that the number of claims for each risk, for each month, are independent
Poisson random variables with
E[y_{ij} ] =
ij .
These random variables are modelled using a generalised linear model, with
log
ij
=
i
(i = 1, 2,
\begin{enumerate}
\item (i) Derive the maximum likelihood estimator of 
\item (ii) Show that the deviance for this model is
n
m
2
y_{ij} log
i 1 j 1
where y i =
7
, n)
1
m
y_{ij}
y i
( y_{ij}
i .

y i )
m
y_{ij} .

j 1
\item (iii) A company has data for each month over a 2 year period. For one risk, the
average number of claims per month was 17.45. In the most recent month for
this risk, there were 9 claims. Calculate the contribution that this observation
makes to the deviance.
\end{enumerate}
\newpage
%%-- %%-- Page 5%%%%%%%%%%%%%%%%%%%%%%%%%%%%%%%%%%%%%%%%%%%%%
y_{ij}
6
(i)
ij
\begin{itemize}
\item The likelihood is
e
April 2006
Examiners Report
ij
y_{ij} !
i , j
and the loglikelihood is
n
m
( y_{ij} log
l =
ij
ij
log y_{ij} !)
i 1 j 1
\item Hence
n
m
( y_{ij}
l =
e i
me i
i
log y_{ij} !)
i 1 j 1
m
l
y_{ij}
=
i
l
j 1
= 0
e
i
i
and
(ii)
i
j 1
n m
y_{ij}
= log y i
The deviance is
n
2(l f
m
l c ) = 2
( y_{ij} log y_{ij}
y_{ij} )
i 1 j 1
n
y_{ij} log
i 1 j 1
( y_{ij} log y i
y i )
i 1 j 1
m
= 2
\item (iii)
m
1
= y i , where y i =
m
y_{ij}
y i
( y_{ij}
y i )
\item The deviance is
D_{ij} = y_{ij} log
y_{ij}
y i
( y_{ij}
y i ) = 9 log
9
(9 17.45)
17.45
= 2.491
\item Full credit should also be given to 2 x 2.491 = 4.98
Comments on question 6: Most candidates scored well on (i). Only more able candidates
scored well on parts (ii) and (iii). There were some relatively easy marks available in (iii) for
applying data to the formula given in (ii).
\end{itemize}
%%-- %%-- Page 6
%%-- %%%%%%%%%%%%%%%%%%%%%%%%%%%%%%%%%%%%%%%%%%%%%
\end{document}
