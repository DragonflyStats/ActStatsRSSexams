\documentclass[a4paper,12pt]{article}

%%%%%%%%%%%%%%%%%%%%%%%%%%%%%%%%%%%%%%%%%%%%%%%%%%%%%%%%%%%%%%%%%%%%%%%%%%%%%%%%%%%%%%%%%%%%%%%%%%%%%%%%%%%%%%%%%%%%%%%%%%%%%%%%%%%%%%%%%%%%%%%%%%%%%%%%%%%%%%%%%%%%%%%%%%%%%%%%%%%%%%%%%%%%%%%%%%%%%%%%%%%%%%%%%%%%%%%%%%%%%%%%%%%%%%%%%%%%%%%%%%%%%%%%%%%%

\usepackage{eurosym}
\usepackage{vmargin}
\usepackage{amsmath}
\usepackage{graphics}
\usepackage{epsfig}
\usepackage{enumerate}
\usepackage{multicol}
\usepackage{subfigure}
\usepackage{fancyhdr}
\usepackage{listings}
\usepackage{framed}
\usepackage{graphicx}
\usepackage{amsmath}
\usepackage{chng%%-- Page}
%\usepackage{bigints}
\usepackage{vmargin}

% left top textwidth textheight headheight

% headsep footheight footskip
\setmargins{2.0cm}{2.5cm}{16 cm}{22cm}{0.5cm}{0cm}{1cm}{1cm}
\renewcommand{\baselinestretch}{1.3}
\setcounter{MaxMatrixCols}{10}
\begin{document} 

%%% - Question 8

(i) Let X denote the claim amount under an insurance policy, and suppose that X
has a probability density f X (x) for x > 0. The insurer has an individual excess
of loss reinsurance arrangement with a retention of £M. Let Y be the amount
paid by the insurer net of reinsurance. Express Y in terms of X and hence
derive an expression for the probability density function of Y in terms of f X (x).


For a particular class of policy X is believed to follow a Weibull distribution with
probability density function
f X (x) = 0.75cx
0.75
0.25 e cx
(x > 0)
where c is an unknown constant. The insurer has an individual excess of loss
reinsurance arrangement with retention £500. The following claims data are
observed:
Claims below retention: 78, 104, 116, 135, 189, 243, 270, 350, 411, 491
Claims above retention: 3 in total
Total number of claims: 13
(ii) Estimate c using maximum likelihood estimation.
(iii) Apply the method of percentiles using the median claim to estimate c.

[Total 14]
CT6 S2006
4
%%%%%%%%%%%%%%%%%%%%%%%%%%%%%%%%%%%%%%%%%%%%%%%%%%%%%%%%%%%%%%%%%%%%%%%%%%%%%%%%%%

Page 9 — %%%%%%%%%%%%%%%%%%%%%%%%%%%%%%%%%%%%%%%%%%%%%%%%%%%%%%
8
(i)
⎧ X
Y = ⎨
⎩ M
if X < M
if X ≥ M
Y has a mixed distribution given by
f Y (x) = f X (x) for x < M and
P(Y = M) = 1 − F X (M)
where F X (x) =
(ii)
∫
x
0
f X ( u ) du .

The probability of an individual claim being above the retention is given by
1 − F(500) = e − c 500
0.75
= e − 105.74 \times  c
The likelihood of the observed data is then (denoting by x 1 , ..., x 10 the ten
claims below the retention)
L = k \times 
∏ cx i − 0.25 e − cx
0.75
i
\times  ( e − 105.74 c ) 3
and the log-likelihood is given by

\[l = log L = const + 10 log c − 0.25 \sum  log x i − c \sum  x i 0.75 − 3 \times  105.74 c\]

Differentiating gives
l ′ = 10 / c − \sum  x i 0.75 − 317.22
Equating this to zero gives
10 / c ˆ =
ĉ =
Page 10
\sum 
\sum  x i 0.75 + 317.22
10
x i 0.75
+ 317.22
=
10
= 0.011
589.40 + 317.22 — %%%%%%%%%%%%%%%%%%%%%%%%%%%%%%%%%%%%%%%%%%%%%%%%%%%%%%
(iii)
The median claim is for £270. We solve
F(270) = 0.5
1 − e − c 270
0.75
= 0.5
e − 66.61c = 0.5
c =
9
