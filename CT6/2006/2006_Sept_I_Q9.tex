\documentclass[a4paper,12pt]{article}

%%%%%%%%%%%%%%%%%%%%%%%%%%%%%%%%%%%%%%%%%%%%%%%%%%%%%%%%%%%%%%%%%%%%%%%%%%%%%%%%%%%%%%%%%%%%%%%%%%%%%%%%%%%%%%%%%%%%%%%%%%%%%%%%%%%%%%%%%%%%%%%%%%%%%%%%%%%%%%%%%%%%%%%%%%%%%%%%%%%%%%%%%%%%%%%%%%%%%%%%%%%%%%%%%%%%%%%%%%%%%%%%%%%%%%%%%%%%%%%%%%%%%%%%%%%%

\usepackage{eurosym}
\usepackage{vmargin}
\usepackage{amsmath}
\usepackage{graphics}
\usepackage{epsfig}
\usepackage{enumerate}
\usepackage{multicol}
\usepackage{subfigure}
\usepackage{fancyhdr}
\usepackage{listings}
\usepackage{framed}
\usepackage{graphicx}
\usepackage{amsmath}
\usepackage{chng%%-- Page}
%\usepackage{bigints}
\usepackage{vmargin}

% left top textwidth textheight headheight

% headsep footheight footskip
\setmargins{2.0cm}{2.5cm}{16 cm}{22cm}{0.5cm}{0cm}{1cm}{1cm}
\renewcommand{\baselinestretch}{1.3}
\setcounter{MaxMatrixCols}{10}
\begin{document} 

%%% - Question 9

[7]9
An insurer operates a No Claims Discount system with three levels of discount:
Discount
Level 0
Level 1
Level 2
0%
20%
50%
The annual premium in level 0 is £650.
If a policyholder makes no claims in a policy year, they move to the next high
discount level (or remain at level 2). In all other cases they move to (or remain at)
discount level 0.
For a policyholder who has not yet had an accident in a policy year, the probability of
an accident occurring is 0.1. The time at which an accident occurs in the policy year
is denoted by T, where
0 T 1;
T = 0 means that the accident occurs at the start of the policy year;
T = 1 means that the accident occurs at the end of the policy year.
It is assumed that T has a uniform distribution.
Given that a policyholder has had their first accident, the probability of them having a
second accident in the same policy year is 0.4(1 T). It is assumed that a
policyholder will not have more than two accidents in a policy year.
The cost of each accident has an exponential distribution with mean £1,000.
After each accident, the policyholder decides whether or not to make a claim by
comparing the increase in the premium they would have to pay in the next policy year
with the claim size. In doing this, they assume that they will have no further
accidents.
(i)
Show that the distribution of the number of accidents, K, that a policyholder
has in a year is:
P(K = 0) = 0.9
P(K = 1) = 0.08
P(K = 2) = 0.02

(ii) For each level of discount, calculate the probability that a policyholder makes
n claims in a policy year, where n = 0, 1, 2.
[8]
(iii) Write down the transition matrix.
(iv) Derive the steady state distribution.
CT6 S2006
5


[Total 17]
%%%%%%%%%%%%%%%%%%%%%%%%%%%%%%%%%%%%%%%%%%%%%%%%%%%%%%%%%%%%%%%%%%%%%%%%%%%%%%%%5

(i)
log 0.5
= 0.0104
− 66.61
P(K = 0) = 0.9
P(K = 1) = 0.1 \times  P(no 2 nd accident)
P(2 nd accident) =
1
∫ 0 0.4(1 − t ) f ( t ) dt
1
= 0.4 ∫ (1 − t ) dt = 0.4[ t − 1⁄2 t 2 ] 1 0
0
= 0.4 \times  1⁄2 = 0.2
∴ P(K = 1) = 0.1 \times  (1 – 0.2) = 0.08
P(K = 2) = 1 – 0.9 – 0.08 = 0.02
(ii)
Let N = number of claims a policyholder makes.
2
Then P(N = n) = \sum  P ( N = n K = k ) P ( K = k )
k = 0
Level 0:
Change in premium when first claim made = 650 − 0.8 \times 
650=130
P(X > 130) = e − 130/1,000 = 0.8781
Levels 1, 2: Change in premium when first claim made = 650 − 0.5 \times  650 =
325
P(X > 325) = e − 325/1,000 = 0.7225
P(N = 0) = P(K = 0) + P(N = 0|K=1) P(K = 1)
+ P(N = 0|K = 2) P(K = 2)
Page 11 — %%%%%%%%%%%%%%%%%%%%%%%%%%%%%%%%%%%%%%%%%%%%%%%%%%%%%%
Level 0:
P(N = 0) = 0.9 + 0.1219 \times  0.08 + 0.1219 2 \times  0.02
= 0.9100
Levels 1, 2: P(N = 0) = 0.9 + 0.2775 \times  0.08 + 0.2775 2 \times  0.02
= 0.9237
Note that if one claim has already been made then the NCD has already been
lost, and it is therefore certain that a second claim will be made, regardless of
the size of the loss. Therefore, for two accidents to result in only one claim it
must be that the first accident resulted in no claim, and the second resulted in a
claim.
P(N = 1) = P(N = 1|K = 1) P(K = 1) + P(N = 1|K = 2) P(K = 2)
Level 0:
P(N = 1) = 0.8781 \times  0.08 + 0.1219\times  0.8781 \times  0.02
= 0.0724
Levels 1, 2: P(N = 1) = 0.7225 \times  0.08 + 0.2775 \times  0.7255 \times  0.02
= 0.0618
Two accidents will result in two claims whenever the first accident results in a
claim (since in this case the second accident will certainly result in a claim).
P(N = 2) = P(N = 2|K = 2) P(K = 2)
Level 0:
0.8781 \times  0.02 = 0.0176
Levels 1, 2: P(N = 2) = 0.7225 \times  0.02 = 0.0145
(iii)
The transition matrix is
⎛ 0.0900 0.9100 0
⎞
⎜
⎟
0.9238 ⎟
⎜ 0.0762 0
⎜ 0.0762 0
0.9238 ⎟ ⎠
⎝
(iv)
\pi  = P \pi 
0.9100\pi  0 = \pi  1
0.9238(\pi  1 + \pi  2 ) = \pi  2
∴ \pi  2 = 12.123\pi  1 = 11.032\pi  0
Page 12 — %%%%%%%%%%%%%%%%%%%%%%%%%%%%%%%%%%%%%%%%%%%%%%%%%%%%%%
Since \pi  0 + \pi  1 + \pi  2 = 1
\pi  0 + 0.9100\pi  0 + 11.032\pi  0 = 1
∴ \pi  0 = 0.0773, \pi  1 = 0.0703, \pi  2 = 0.8524
