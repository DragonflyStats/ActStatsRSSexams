\documentclass[a4paper,12pt]{article}

%%%%%%%%%%%%%%%%%%%%%%%%%%%%%%%%%%%%%%%%%%%%%%%%%%%%%%%%%%%%%%%%%%%%%%%%%%%%%%%%%%%%%%%%%%%%%%%%%%%%%%%%%%%%%%%%%%%%%%%%%%%%%%%%%%%%%%%%%%%%%%%%%%%%%%%%%%%%%%%%%%%%%%%%%%%%%%%%%%%%%%%%%%%%%%%%%%%%%%%%%%%%%%%%%%%%%%%%%%%%%%%%%%%%%%%%%%%%%%%%%%%%%%%%%%%%

\usepackage{eurosym}
\usepackage{vmargin}
\usepackage{amsmath}
\usepackage{graphics}
\usepackage{epsfig}
\usepackage{enumerate}
\usepackage{multicol}
\usepackage{subfigure}
\usepackage{fancyhdr}
\usepackage{listings}
\usepackage{framed}
\usepackage{graphicx}
\usepackage{amsmath}
\usepackage{chng%%-- Page}
%\usepackage{bigints}
\usepackage{vmargin}

% left top textwidth textheight headheight

% headsep footheight footskip
\setmargins{2.0cm}{2.5cm}{16 cm}{22cm}{0.5cm}{0cm}{1cm}{1cm}
\renewcommand{\baselinestretch}{1.3}
\setcounter{MaxMatrixCols}{10}
\begin{document} 

%%% - Question 7

PLEASE TURN OVER7
The random variable W has a binomial distribution such that
P(W = w) =
w
w
(1
) n
w
(w = 0, 1, 2,
, n)
W
.
n
Let Y =
8
n
1 2
, ,..., 1 .
n n
(i) Write down an expression for P(Y = y), for y = 0,

(ii) Express the distribution of Y as an exponential family and identify the natural
parameter and the dispersion parameter.

(iii) Derive an expression for the variance function.
(iv) For a set of n independent observations of Y, derive an expression of the
scaled deviance.

[Total 10]
%%%%%%%%%%%%%%%%%%%%%%%%%%%%%%%%%%%%%%%%%%%%%%%%%%%%%%%%%%%%%%%%%%%%%%%%%%%%%%%%%%%%%
\newpage


7
(i) ⎛ n ⎞
P(Y = y) = ⎜ ⎟ \mu ny (1 − \mu) n−ny
⎝ ny ⎠
(ii) ⎡
⎛ n ⎞ ⎤
P(Y = y) = exp ⎢ ny log \mu + n (1 − y ) log(1 − \mu ) + log ⎜ ⎟ ⎥
⎝ ny ⎠ ⎦
⎣
⎡ ⎛
⎛ n ⎞ ⎤
⎞
\mu
= exp ⎢ n ⎜ y log
+ log(1 − \mu ) ⎟ + log ⎜ ⎟ ⎥
1 − \mu
⎠
⎝ ny ⎠ ⎦
⎣ ⎝
which is in the form of an exponential family.
The natural parameter is log
Page 8
\mu
.
1 − \mu — %%%%%%%%%%%%%%%%%%%%%%%%%%%%%%%%%%%%%%%%%%%%%%%%%%%%%%
The dispersion parameter is
(iii)
either \phi  = n
or \phi  =
and a(\phi ) =
1
\phi 
1
and a ( \phi  ) = \phi 
n
V ( \mu ) = b ′′ ( \theta  )
b ( \theta  ) = − log(1 − \mu ) = log
b ′ ( \theta  ) =
b ′′ ( \theta  ) =
1
= log(1 + e \theta  )
1 − \mu
e \theta 
1 + e \theta 
(1 + e \theta  ) e \theta  − e \theta  e \theta 
(1 + e \theta  ) 2
=
e \theta 
(1 + e \theta  ) 2
= \mu(1 − \mu)
(iv)
Scaled deviance is −2(l c − l f )
l c =
⎡ ⎛
i
l f =
\mu
1 ⎞ ⎛ n ⎞ ⎤
⎠ ⎦
1 ⎞ ⎛ n ⎞ ⎤
⎠ ⎦
\sum  ⎢ ⎢ n ⎜ y i log 1 − \mu i i − log 1 − \mu i ⎟ + log ⎝ ⎜ ny i ⎠ ⎟ ⎥ ⎥
⎣ ⎝
⎡ ⎛
\sum  ⎢ ⎢ n ⎜ y i log 1 − i y i − log 1 − y i ⎟ + log ⎜ ⎝ ny i ⎟ ⎠ ⎥ ⎥
i
y
⎣ ⎝
Hence the scaled deviance is
⎛
⎛ \mu 1 − y i ⎞
⎛ 1 − y i ⎞ ⎞
− log ⎜
−2(l c − l f ) = − 2 \sum  n ⎜ y i log ⎜ i
⎟
⎟ ⎟ ⎟
⎜
⎝ y i 1 − \mu i ⎠
⎝ 1 − \mu i ⎠ ⎠
⎝
i
