\documentclass[a4paper,12pt]{article}

%%%%%%%%%%%%%%%%%%%%%%%%%%%%%%%%%%%%%%%%%%%%%%%%%%%%%%%%%%%%%%%%%%%%%%%%%%%%%%%%%%%%%%%%%%%%%%%%%%%%%%%%%%%%%%%%%%%%%%%%%%%%%%%%%%%%%%%%%%%%%%%%%%%%%%%%%%%%%%%%%%%%%%%%%%%%%%%%%%%%%%%%%%%%%%%%%%%%%%%%%%%%%%%%%%%%%%%%%%%%%%%%%%%%%%%%%%%%%%%%%%%%%%%%%%%%

\usepackage{eurosym}
\usepackage{vmargin}
\usepackage{amsmath}
\usepackage{graphics}
\usepackage{epsfig}
\usepackage{enumerate}
\usepackage{multicol}
\usepackage{subfigure}
\usepackage{fancyhdr}
\usepackage{listings}
\usepackage{framed}
\usepackage{graphicx}
\usepackage{amsmath}
\usepackage{chng%%-- Page}
%\usepackage{bigints}
\usepackage{vmargin}

% left top textwidth textheight headheight

% headsep footheight footskip
\setmargins{2.0cm}{2.5cm}{16 cm}{22cm}{0.5cm}{0cm}{1cm}{1cm}
\renewcommand{\baselinestretch}{1.3}
\setcounter{MaxMatrixCols}{10}
\begin{document} 

%%% - Question 5

CT6 S2006
25
(i) Let p be an unknown parameter, and let f(p x) denote the probability density of the posterior distribution of p given information x. Show that under all-or-
nothing loss the Bayes estimate of p is the mode of f(p x).

(ii) Now suppose p is the proportion of the population carrying a particular genetic condition. Prior beliefs about p have a U(0, 1) distribution. A sample of size
N is taken from the population revealing that m individuals have the genetic condition.
(a) Suggest why the U(0, 1) distribution has been chosen as the prior, and derive the posterior distribution of p.
(b) Calculate the Bayes estimate of p under all-or-nothing loss.
[6]

%%%%%%%%%%%%%%%%%%%%%%%%%%%%%%%%%%%%%%%%%%%%%%%%%%%%%%%%%%%%%%%%%%%%%%%%%%%%%%%

5
(i)
Consider the loss function
⎧ 0 if g − \varepsilon  < p < g + \varepsilon 
L(g(x), p) = ⎨
⎩ 1 otherwise
Then the expected posterior loss is given by
g +\varepsilon 
1 −
∫
f ( p ⏐ x ) dp
g −\varepsilon 
≈ 1 − 2\varepsilon f(g|x)
for small values of \varepsilon . This is minimised by setting g to be the maximum (i.e.
the mode) of f(p|x).
Page 6 — %%%%%%%%%%%%%%%%%%%%%%%%%%%%%%%%%%%%%%%%%%%%%%%%%%%%%%
(ii)
(a)
Using U(0, 1) as the prior for p suggests that no prior information or
beliefs about p have been formed — it is equally likely to lie anywhere
in the range [0, 1].
f(p|m) ∝ f(m|p) f(p)
∝ p m (1 − p) N−m \times  1
So posterior beliefs about p have a Beta distribution with parameters
m + 1 and N − m + 1.
(b)
We must find the mode of (f(p|m).
Maximising this is the same as maximising
g(p) = log f(p|m) = m log p + (N − m) log (1 − p) + constant
g ′ ( p ) =
m N − m
−
p 1 − p
and g ′ ( p ) = 0 when
m N − m
−
= 0 i.e.
p 1 − p
m(1 − p) = (N − m) p
Np = m
p = m / N
Page 7 — %%%%%%%%%%%%%%%%%%%%%%%%%%%%%%%%%%%%%%%%%%%%%%%%%%%%%%
