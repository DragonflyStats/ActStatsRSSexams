\documentclass[a4paper,12pt]{article}

%%%%%%%%%%%%%%%%%%%%%%%%%%%%%%%%%%%%%%%%%%%%%%%%%%%%%%%%%%%%%%%%%%%%%%%%%%%%%%%%%%%%%%%%%%%%%%%%%%%%%%%%%%%%%%%%%%%%%%%%%%%%%%%%%%%%%%%%%%%%%%%%%%%%%%%%%%%%%%%%%%%%%%%%%%%%%%%%%%%%%%%%%%%%%%%%%%%%%%%%%%%%%%%%%%%%%%%%%%%%%%%%%%%%%%%%%%%%%%%%%%%%%%%%%%%%

\usepackage{eurosym}
\usepackage{vmargin}
\usepackage{amsmath}
\usepackage{graphics}
\usepackage{epsfig}
\usepackage{enumerate}
\usepackage{multicol}
\usepackage{subfigure}
\usepackage{fancyhdr}
\usepackage{listings}
\usepackage{framed}
\usepackage{graphicx}
\usepackage{amsmath}
\usepackage{chng%%-- Page}
%\usepackage{bigints}
\usepackage{vmargin}

% left top textwidth textheight headheight

% headsep footheight footskip
\setmargins{2.0cm}{2.5cm}{16 cm}{22cm}{0.5cm}{0cm}{1cm}{1cm}
\renewcommand{\baselinestretch}{1.3}
\setcounter{MaxMatrixCols}{10}



\begin{document}3
(i) Let N be a random variable representing the number of claims arising from a
portfolio of insurance policies. Let X i denote the size of the ith claim and suppose that X 1 , X 2 , are independent identically distributed random variables, all having the same distribution as X. The claim sizes are
independent of the number of claims. Let S = X 1 + X 2 +
+ X N denote the
total claim size. Show that
M S (t) = M N (logM X (t)).
(ii)

Suppose that N has a Type 2 negative binomial distribution with parameters
k > 0 and 0 < p < 1. That is
P(N = x) =
( k x )
p k q x
( x 1) ( k )
x = 0, 1, 2,
Suppose that X has an exponential distribution with mean 1/ . Derive an
expression for M s (t).
CT6 A2006 4
(iii)
Now suppose that the number of claims on another portfolio is R with the size
of the ith claim given by Y i . Let T = Y 1 +
+ Y R . Suppose that R has a
binomial distribution, with parameters k and 1 p, and that Y i has an exponential distribution with mean 1/ . Show that if is chosen appropriately then S and T have the same distribution.

You may use any standard formulae for moment generating functions of specific distributions shown in the Formulae and Tables.
[Total 11]
%%%%%%%%%%%%%%%%%%%%%%%%%%%%%%%%%%%%%%
8
An insurer has for 2 years insured a number of domestic animals against veterinary costs. In year 1 there were n 1 policies and in year 2 there were n 2 policies. The number of claims per policy per year follows a Poisson distribution with unknown parameter .
Individual claim amounts were a constant c in year 1 and a constant c(1 + r) in year 2.
The average total claim amount per policy was y 1 in year 1 and y 2 in year 2. Prior beliefs about follow a gamma distribution with mean / and variance / 2 . In
year 3 there are n 3 policies, and individual claim amounts are c(1 + r) 2 . Let Y 3 be the random variable denoting average total claim amounts per policy in year 3.
(i) State the distribution of the number of claims on the whole portfolio over the 2 year period.

(ii) Derive the posterior distribution of
(iii) Show that the posterior expectation of Y 3 given y 1 , y 2 can be written in the form of a credibility estimate
Z
k + (1
Z)

c (1 r ) 2
specifying expressions for k and Z.
(iv)
given y 1 and y 2 .

Describe k in words and comment on the impact the values of n 1 , n 2 have
on Z.

[Total 14]
CT6 A2006 5
PLEASE TURN OVER
%%%%%%%%%%%%%%%%%%%%%%%%%%%%%%%%%%%%%%%%%%%%%%%%%%%55
7
(i)
April 2006
Examiners Report
M s (t) = E(e St )
= E ( E ( e ( X 1
... X N ) t
N ))
= E ( E ( e X 1 t e X 2 t ... e X N t N ))
= E ( M X ( t ) N )
= E ( e N log M X ( t ) )
= M N (log M X ( t ))
(ii)
From the tables
M N (t) =
M X (t) =
k
p
1 qe t
t
So
p
M s (t) =
1 qM X ( t )
1 q
=
=
k
p
=
k
t
p ( t )
t q
p (
p t )
t
k
k
%%-- Page 7%%%%%%%%%%%%%%%%%%%%%%%%%%%%%%%%%%%%%%%%%%%%%
(iii)
April 2006
Examiners Report
We now have
M Y (t) =
t
M R (t) = (p + qe t ) k
and so
k
M T (t) =
=
p q
p
t
pt q
t
pt
t
=
k
k
Thus if we choose = p then M T (t) = M S (t) and by the uniqueness of
Moment Generating Functions, S and T have the same distribution.
Comments on question 7: This question was generally well answered although relatively few
managed the final step of demonstrating that S and T have the same distribution.
8
(i) The total number of claims has a Poisson distribution with parameter
(n 1 + n 2 ) .
(ii) Let Y i denote the average total claim amount per policy in year i and let X i
denote the total number of claims in year i. Then X i has a Poisson distribution
with parameter n i and
X 1 =
n 1 Y 1
Yn
and X 2 = 2 2 .
c
c (1 r )
f( y 1 , y 2 )
f(y 1 , y 2 ) f( )
e n 1 ( n 1 ) y 1 n 1 / c e
e ( n 1 n 2 )
n 2
n 1 y 1 n 2 y 2
c c (1 r )
So the posterior distribution of
and
%%-- Page 8
+ n 1 + n 2 .
( n 2 ) y 2 n 2 / c (1
r )
e
1
1
is gamma with parameters
n 1 y 1
c
n 2 y 2
c (1 r )%%%%%%%%%%%%%%%%%%%%%%%%%%%%%%%%%%%%%%%%%%%%%
(iii)
E(Y 3 y 1 , y 2 ) =
2
n 1 y 1
n 2 y 2
c
c (1 r )
n 1 n 2
c (1 r )
n 3
= c (1 r ) 2 n 1 y 1 (1 r ) 2 n 2 y 2 (1 r )
n 1 n 2
k =
Z =
n 3
%%%%%%%%%%%%%%%%%%%%%%%%%%%%%%%%%%%%%%%%%%%%%%%%%%%5555
E(X 3 y 1 , y 2 )
=
= c (1 r )
(iv)
c (1 r ) 2
n 3
April 2006
2
n 1 n 2
n 1 y 1 (1 r ) 2 n 2 y 2 (1 r )
n 1 n 2
n 1 n 2
n 1 n 2
n 1 y 1 (1 r ) 2 n 2 y 2 (1 r )
and
n 1 n 2
n 1 n 2
n 1 n 2
k is effectively a weighted average of the inflation adjusted average claim amounts for the previous 2 years, weighted by the number of policies in force. As the number of policies in force increases, Z becomes closer to 1, and so more weight is placed on the actual experience, and less on the prior expectations.

Comments on question 8: Candidates found this the most difficult question in the paper.
Only those candidates with a methodical approach and an excellent grasp of the relevant bookwork managed to progress to the later parts of the question.

\end{document}
