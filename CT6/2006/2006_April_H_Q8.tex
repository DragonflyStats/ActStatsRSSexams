\documentclass[a4paper,12pt]{article}

%%%%%%%%%%%%%%%%%%%%%%%%%%%%%%%%%%%%%%%%%%%%%%%%%%%%%%%%%%%%%%%%%%%%%%%%%%%%%%%%%%%%%%%%%%%%%%%%%%%%%%%%%%%%%%%%%%%%%%%%%%%%%%%%%%%%%%%%%%%%%%%%%%%%%%%%%%%%%%%%%%%%%%%%%%%%%%%%%%%%%%%%%%%%%%%%%%%%%%%%%%%%%%%%%%%%%%%%%%%%%%%%%%%%%%%%%%%%%%%%%%%%%%%%%%%%

\usepackage{eurosym}
\usepackage{vmargin}
\usepackage{amsmath}
\usepackage{graphics}
\usepackage{epsfig}
\usepackage{enumerate}
\usepackage{multicol}
\usepackage{subfigure}
\usepackage{fancyhdr}
\usepackage{listings}
\usepackage{framed}
\usepackage{graphicx}
\usepackage{amsmath}
\usepackage{chng%%-- Page}
%\usepackage{bigints}
\usepackage{vmargin}

% left top textwidth textheight headheight

% headsep footheight footskip
\setmargins{2.0cm}{2.5cm}{16 cm}{22cm}{0.5cm}{0cm}{1cm}{1cm}
\renewcommand{\baselinestretch}{1.3}
\setcounter{MaxMatrixCols}{10}



\begin{document}
%%%%%%%%%%%%%%%%%%%%%%%%%%%%%%%%%%%%%%
8
An insurer has for 2 years insured a number of domestic animals against veterinary costs. In year 1 there were n_{1} policies and in year 2 there were n_{2} policies. The number of claims per policy per year follows a Poisson distribution with unknown parameter .
Individual claim amounts were a constant c in year 1 and a constant c(1 + r) in year 2.
The average total claim amount per policy was y 1 in year 1 and y 2 in year 2. Prior beliefs about follow a gamma distribution with mean / and variance / 2 . In
year 3 there are n 3 policies, and individual claim amounts are c(1 + r) 2 . Let Y 3 be the random variable denoting average total claim amounts per policy in year 3.
(i) State the distribution of the number of claims on the whole portfolio over the 2 year period.

(ii) Derive the posterior distribution of
(iii) Show that the posterior expectation of Y 3 given y 1 , y 2 can be written in the form of a credibility estimate
Z
k + (1
Z)

c (1 r ) 2
specifying expressions for k and Z.
(iv)
given y 1 and y 2 .

Describe k in words and comment on the impact the values of n_{1} , n_{2} have
on Z.

[Total 14]
CT6 A2006 5
\newpage
8
(i) The total number of claims has a Poisson distribution with parameter
(n_{1} + n_{2} ) .
(ii) Let Y i denote the average total claim amount per policy in year i and let X i
denote the total number of claims in year i. Then X i has a Poisson distribution
with parameter n i and
X 1 =
n_{1} Y 1
Yn
and X 2 = 2 2 .
c
c (1 r )
f( y 1 , y 2 )
f(y 1 , y 2 ) f( )
e n_{1} ( n_{1} ) y 1 n_{1} / c e
e ( n_{1} n_{2} )
n_{2}
n_{1} y 1 n_{2} y 2
c c (1 r )
So the posterior distribution of
and
%%-- Page 8
+ n_{1} + n_{2} .
( n_{2} ) y 2 n_{2} / c (1
r )
e
1
1
is gamma with parameters
n_{1} y 1
c
n_{2} y 2
c (1 r )%%%%%%%%%%%%%%%%%%%%%%%%%%%%%%%%%%%%%%%%%%%%%
(iii)
E(Y 3 y 1 , y 2 ) =
2
n_{1} y 1
n_{2} y 2
c
c (1 r )
n_{1} n_{2}
c (1 r )
n 3
= c (1 r ) 2 n_{1} y 1 (1 r ) 2 n_{2} y 2 (1 r )
n_{1} n_{2}
k =
Z =
n 3
%%%%%%%%%%%%%%%%%%%%%%%%%%%%%%%%%%%%%%%%%%%%%%%%%%%5555
E(X 3 y 1 , y 2 )
=
= c (1 r )
(iv)
c (1 r ) 2
n 3
April 2006
2
n_{1} n_{2}
n_{1} y 1 (1 r ) 2 n_{2} y 2 (1 r )
n_{1} n_{2}
n_{1} n_{2}
n_{1} n_{2}
n_{1} y 1 (1 r ) 2 n_{2} y 2 (1 r )
and
n_{1} n_{2}
n_{1} n_{2}
n_{1} n_{2}
k is effectively a weighted average of the inflation adjusted average claim amounts for the previous 2 years, weighted by the number of policies in force. As the number of policies in force increases, Z becomes closer to 1, and so more weight is placed on the actual experience, and less on the prior expectations.

Comments on question 8: Candidates found this the most difficult question in the paper.
Only those candidates with a methodical approach and an excellent grasp of the relevant bookwork managed to progress to the later parts of the question.

\end{document}
