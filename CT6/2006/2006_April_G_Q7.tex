\documentclass[a4paper,12pt]{article}

%%%%%%%%%%%%%%%%%%%%%%%%%%%%%%%%%%%%%%%%%%%%%%%%%%%%%%%%%%%%%%%%%%%%%%%%%%%%%%%%%%%%%%%%%%%%%%%%%%%%%%%%%%%%%%%%%%%%%%%%%%%%%%%%%%%%%%%%%%%%%%%%%%%%%%%%%%%%%%%%%%%%%%%%%%%%%%%%%%%%%%%%%%%%%%%%%%%%%%%%%%%%%%%%%%%%%%%%%%%%%%%%%%%%%%%%%%%%%%%%%%%%%%%%%%%%

\usepackage{eurosym}
\usepackage{vmargin}
\usepackage{amsmath}
\usepackage{graphics}
\usepackage{epsfig}
\usepackage{enumerate}
\usepackage{multicol}
\usepackage{subfigure}
\usepackage{fancyhdr}
\usepackage{listings}
\usepackage{framed}
\usepackage{graphicx}
\usepackage{amsmath}
\usepackage{chng%%-- Page}
%\usepackage{bigints}
\usepackage{vmargin}

% left top textwidth textheight headheight

% headsep footheight footskip
\setmargins{2.0cm}{2.5cm}{16 cm}{22cm}{0.5cm}{0cm}{1cm}{1cm}
\renewcommand{\baselinestretch}{1.3}
\setcounter{MaxMatrixCols}{10}



\begin{document}3
(i) Let N be a random variable representing the number of claims arising from a
portfolio of insurance policies. Let X i denote the size of the ith claim and suppose that X 1 , X 2 , are independent identically distributed random variables, all having the same distribution as X. The claim sizes are
independent of the number of claims. Let S = X 1 + X 2 +
+ X N denote the
total claim size. Show that
M S (t) = M_{N} (logM_{X} (t)).
(ii)

Suppose that N has a Type 2 negative binomial distribution with parameters
k > 0 and 0 < p < 1. That is
P(N = x) =
( k x )
p k q x
( x 1) ( k )
x = 0, 1, 2,
Suppose that X has an exponential distribution with mean_{1}/ . Derive an
expression for M s (t).
CT6 A2006 4
(iii)
Now suppose that the number of claims on another portfolio is R with the size
of the ith claim given by Y i . Let T = Y 1 +
+ Y R . Suppose that R has a
binomial distribution, with parameters k and 1 p, and that Y i has an exponential distribution with mean_{1}/ . Show that if is chosen appropriately then S and T have the same distribution.

You may use any standard formulae for moment generating functions of specific distributions shown in the Formulae and Tables.
\newpage

%%%%%%%%%%%%%%%%%%%%%%%%%%%%%%%%%%%%%%%%%%%%%%%%%%%55
7
(i)
April 2006
Examiners Report
\begin{eqnarray*}
M s (t) &=& E(e St )\\
&=& E ( E ( e ( X 1
... X N ) t
N ))\\
&=& E ( E ( e X 1 t e X 2 t ... e X N t N ))\\
&=& E ( M_{X} ( t ) N )\\
&=& E ( e N log M_{X} ( t ) )\\
&=& M_{N} (log M_{X} ( t ))\\
\end{eqnarray*}
(ii)
From the tables
M_{N} (t) =
M_{X} (t) =
k
p
1 qe t
t
So
p
M s (t) =
1 qM_{X} ( t )
1 q
=
=
k
p
=
k
t
p ( t )
t q
p (
p t )
t
k
k
%%-- Page 7%%%%%%%%%%%%%%%%%%%%%%%%%%%%%%%%%%%%%%%%%%%%%
(iii)
April 2006
Examiners Report
We now have
M Y (t) =
t
M R (t) = (p + qe t ) k
and so
k
M T (t) =
=
p q
p
t
pt q
t
pt
t
=
k
k
Thus if we choose = p then M T (t) = M S (t) and by the uniqueness of
Moment Generating Functions, S and T have the same distribution.
Comments on question 7: This question was generally well answered although relatively few
managed the final step of demonstrating that S and T have the same distribution.
%%%%%%%%%%%%%%%%%%%%%%%%%%%%

\end{document}
