\documentclass[a4paper,12pt]{article}

%%%%%%%%%%%%%%%%%%%%%%%%%%%%%%%%%%%%%%%%%%%%%%%%%%%%%%%%%%%%%%%%%%%%%%%%%%%%%%%%%%%%%%%%%%%%%%%%%%%%%%%%%%%%%%%%%%%%%%%%%%%%%%%%%%%%%%%%%%%%%%%%%%%%%%%%%%%%%%%%%%%%%%%%%%%%%%%%%%%%%%%%%%%%%%%%%%%%%%%%%%%%%%%%%%%%%%%%%%%%%%%%%%%%%%%%%%%%%%%%%%%%%%%%%%%%

\usepackage{eurosym}
\usepackage{vmargin}
\usepackage{amsmath}
\usepackage{graphics}
\usepackage{epsfig}
\usepackage{enumerate}
\usepackage{multicol}
\usepackage{subfigure}
\usepackage{fancyhdr}
\usepackage{listings}
\usepackage{framed}
\usepackage{graphicx}
\usepackage{amsmath}
\usepackage{chng%%-- Page}
%\usepackage{bigints}
\usepackage{vmargin}

% left top textwidth textheight headheight

% headsep footheight footskip
\setmargins{2.0cm}{2.5cm}{16 cm}{22cm}{0.5cm}{0cm}{1cm}{1cm}
\renewcommand{\baselinestretch}{1.3}
\setcounter{MaxMatrixCols}{10}
\begin{document} 

%%% - Question 10
PLEASE TURN OVER10
(i)
Let I k =
m
x k e
x
dx
where k is a non-negative integer.
Show that
and
I 0 =
1
m k
I k =
m
e
e
m
k
I k
1
(k = 1, 2, 3,
)

For a certain portfolio of insurance policies the number of claims annually has a
Poisson distribution with mean 25. Claim sizes have a gamma distribution with mean
100 and variance 5,000 and the insurer includes a loading of 10% in its premium.
The insurer is considering purchasing individual excess of loss reinsurance with
retention m from a reinsurer that includes a loading of 15% in its premium.
Let X I and X R denote the amounts paid by the direct insurer and the reinsurer,
respectively, on an individual claim.
(ii) Calculate the premium, c, charged by the direct insurer for this portfolio.
(iii) Show that E[X R ] =
1
50 2
(I 2
E[X R ] = (m + 100) e

mI 1 ) and hence that
m/50 .
[7]
(iv) Use the result in (iii) to derive an expression for E[X I ]. 
(v) Derive an expression for the direct insurer s expected annual profit. 
(vi) The table below shows the direct insurer s expected annual profit (Profit) and
probability of ruin (P(ruin)), for various values of the retention level, m:
m Profit P(ruin)
36
50
100 1.8
*
148.5 0.002
0.01
0.05
Calculate the missing value in the table and discuss the issues facing the direct
insurer when deciding on the retention level to use.

[Total 19]
END OF PAPER
CT6 S2006
6

%%%%%%%%%%%%%%%%%%%%%%%%%%%%%%%%%%%%%%%%%%%%%%%%%%%%%%%%%%%%%%%%%%%%

10
(i)
I 0 =
I k =
∞
∫ ∞ −\beta  x
e dx
m ⎡ 1
⎤
1
= ⎢ − e −\beta  x ⎥ = e −\beta  m
\beta 
\beta 
⎣
⎦ m
∫ ∞ k −\beta  x
x e dx
m ∞ kx k − 1 −\beta  x
⎡ 1
⎤
= ⎢ − x k e −\beta  x ⎥ + ∫
e dx
m
\beta 
⎣ \beta 
⎦ m
∞
= m k −\beta  m k ∞ k − 1 −\beta  x
e
+ ∫ x e dx
\beta 
\beta  m
= m k −\beta  m k
+ I k − 1
e
\beta 
\beta 
(ii) c = 1.1 \times  25 \times  100 = 2,750
(iii) E [ X R ] =
∞
∫ m ( x − m ) f ( x ) dx
f ( x ) is gamma, and
∴\beta  =
\alpha 
\alpha 
= 100, 2 = 5,000
\beta 
\beta 
1
and \alpha  = 2
50
2
− x
⎛ 1 ⎞
∴ f ( x ) = ⎜ ⎟ xe 50
⎝ 50 ⎠
E [ X R ] =
=
I 0
( x > 0)
∞
1 ⎡ ∞ 2 − x / 50
x e
dx − m ∫ xe − x / 50 dx ⎤ ⎥
2 ⎢ ∫ m
m
⎦
50 ⎣
1
50 2
[ I 2 − mI 1 ]
= 50 e^{− m /50}
Page 13 — %%%%%%%%%%%%%%%%%%%%%%%%%%%%%%%%%%%%%%%%%%%%%%%%%%%%%%
= 50 me^{− m /50} + 50 2 e^{− m /50}
I 1
= 50( m + 50) e^{− m /50}
= 50 m 2 e^{− m /50} +
I 2
2
I 1
\beta 
= 50 m 2 e^{− m /50} + 5,000( m + 50) e^{− m /50}
= 50( m 2 + 100( m + 50)) e − m/50
∴ E [ X R ] =
1 ⎡
50( m 2 + 100( m + 50)) e − m / 50 − 50 m ( m + 50) e − m / 50 ⎤
2 ⎣
⎦
50
= 1 ⎡ 2
m + 100( m + 50) − m ( m + 50) ⎤ e − m / 50
⎦
50 ⎣
= 1 ⎡ 2
m + 100 m + 5, 000 − m 2 − 50 m ] e − m / 50 ⎤
⎣
⎦
50
= 1
(50 m + 5, 000) e − m / 50
50
= ( m + 100) e^{− m /50}
(iv)
E [ X I ]
= 100 − E [ X R ]
= 100 − ( m + 100) e^{− m /50}
%%%%%%%%%%%%%%%%%%%%%%%%%%%%%%%%%%%%%%%%%%%%%%%%%%%%%%%%%%%%%
(v)
Insurer’s expected profit is c − c R − 25 E [ X I ]
i.e. 2,750 − 1.15 \times  25 \times  ( m + 100) e^{− m /50}
− 25(100 − ( m + 100) e^{− m /50} )
= 250 − 0.15 \times  25( m + 100) e^{− m /50}
(vi)
Page 14
The completed table is
m Profit P(Ruin)
36
50
100 1.8
43.1
148.5 0.002
0.01
0.05 — %%%%%%%%%%%%%%%%%%%%%%%%%%%%%%%%%%%%%%%%%%%%%%%%%%%%%%
As m increases (less reinsurance)
Profit increases
P (Ruin) increases
There is a level beyond which it is not sensible to go (when Profit becomes
negative).
It is a trade-off between profit and security.
Other sensible points were given credit.
END OF EXAMINERS’ REPORT
Page 15
