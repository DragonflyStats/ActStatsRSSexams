\documentclass[a4paper,12pt]{article}

%%%%%%%%%%%%%%%%%%%%%%%%%%%%%%%%%%%%%%%%%%%%%%%%%%%%%%%%%%%%%%%%%%%%%%%%%%%%%%%%%%%%%%%%%%%%%%%%%%%%%%%%%%%%%%%%%%%%%%%%%%%%%%%%%%%%%%%%%%%%%%%%%%%%%%%%%%%%%%%%%%%%%%%%%%%%%%%%%%%%%%%%%%%%%%%%%%%%%%%%%%%%%%%%%%%%%%%%%%%%%%%%%%%%%%%%%%%%%%%%%%%%%%%%%%%%

\usepackage{eurosym}
\usepackage{vmargin}
\usepackage{amsmath}
\usepackage{graphics}
\usepackage{epsfig}
\usepackage{enumerate}
\usepackage{multicol}
\usepackage{subfigure}
\usepackage{fancyhdr}
\usepackage{listings}
\usepackage{framed}
\usepackage{graphicx}
\usepackage{amsmath}
\usepackage{chng%%-- Page}
%\usepackage{bigints}
\usepackage{vmargin}

% left top textwidth textheight headheight

% headsep footheight footskip
\setmargins{2.0cm}{2.5cm}{16 cm}{22cm}{0.5cm}{0cm}{1cm}{1cm}
\renewcommand{\baselinestretch}{1.3}
\setcounter{MaxMatrixCols}{10}



\begin{document}

\begin{enumerate}
\item Based on the proposal form, an applicant for life insurance is classified as a standard life (1), an impaired life (2) or uninsurable (3). The proposal form is not a perfect classifier and may place the applicant into the wrong category.
The decision to place the applicant in state i is denoted by d i , and the correct state for the applicant is i .
The loss function for this decision is shown below:
1
2
3
d 1 d 2 d 3
0
12
20 5
0
15 8
3
0

\begin{enumerate}
\item (i) Determine the minimax solution when assigning an applicant to a category. 
\item (ii) Based on the application form, the correct category for a new applicant appears to be as an impaired life. However, of applicants which appear to be impaired lives, 15\% are in fact standard lives and 25\% are uninsurable.

Determine the Bayes solution for this applicant.
\end{enumerate}%%--
%%%%%%%%%%%%%%%%%%%%%%%%%%%%%%%%%%%%%%%%%%%%%%%%%%%%%%%%%%%%%%%%%%% 
%%%%%%%%%%%%%%%%%%%%%%%%%%%%%%%%%%%%%%%%%%%%%%%%%%%%%%%%%%%%%%%%%%%%%%%%%%%%%%%%
3
(i)
0
12
20
Maximum loss: 20
5
0
15
15
%%April 2006
%%Examiners Report
8
3
0
8
Minimax is d 3 .
(ii)
P( 1 ) = 0.15
P( 2 ) = 0.6
P( 3 ) = 0.25
d 1 = 0.15
d 2 = 0.15
d 3 = 0.15
0 + 0.6
5 + 0.6
8 + 0.6
12 + 0.25 20 = 12.2
0 + 0.25 15 = 4.5
3 + 0.25 0 = 3
Hence the Bayes decision is d 3 .
Comments on question 3: No comments given.
4
(i)
k
= Cov(X t , X t k )
= Cov( X t 1 + e t , X t k )
= k 1
and
0
= Cov(X t , X t )
= Cov( X t 1 + e t , X t 1 + e t )
= 2 Cov(X t 1 , X t 1 ) + Cov(e t , e t )
= 2 0 + 2
and hence
2 )
0 (1
=
2
2
i.e.
0 =
1
2
So our solution is
k 2
k =
1
2
%%%%%%%%%%%%%%%%%%%%%%%%%%%%%%%%%%%%%%%%%%%%%%%%%%%%%%%%%%%%%%%%%%%%%%%%%%%%%%%%%%%%%%%%%%%%
%% - %%-- Page 3%%%%%%%%%%%%%%%%%%%%%%%%%%%%%%%%%%%%%%%%%%%%%
%% - April 2006
%% - Examiners Report
The autocorrelation function is given by
k
=
k
=
k
0
(ii)
The autocorrelation should decay exponentially as i increases. Looking at the table this behaviour occurs after differencing 2 times, suggesting the value of d = 2.
We know that the ratio of successive r s should be . We can form these
ratios as follows:
r 2 /r 1
r 3 /r 2
r 4 /r 3
r 5 /r 4
r 6 /r 5
r 7 /r 6
r 8 /r 7
r 9 /r 8
r 10 /r 9 80%
82%
83%
82%
81%
90%
89%
79%
68%
Average 81.6%
Alternatively we can take the ith root of the ith autocorrelation:
i ith root
1
2
3
4
5
6
7
8
9
10 83%
81%
81%
82%
82%
82%
83%
84%
83%
82%
Average 82%
Both approaches suggest the value of alpha is around 82%.
Full credit should be given to any reasonable approach.
Comments on question 4: This question was generally done poorly. Although most
candidates made a reasonable attempt at (i), very few correctly identified appropriate values
or sensible reasons in (ii).
\end{document}
