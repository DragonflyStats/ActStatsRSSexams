1 HC Module 5 2010
This examination paper consists of 3 printed pages, each printed on one side only.
This front cover is page 1.
Question 1 starts on page 2.
There are 4 questions altogether in the paper.
© RSS 2010
EXAMINATIONS OF THE HONG KONG STATISTICAL SOCIETY HIGHER CERTIFICATE IN STATISTICS, 2010 MODULE 5 : Further probability and inference Time allowed: One and a half hours Candidates should answer THREE questions. Each question carries 20 marks. The number of marks allotted for each part-question is shown in brackets. Graph paper and Official tables are provided. Candidates may use calculators in accordance with the regulations published in the Society's "Guide to Examinations" (document Ex1). The notation log denotes logarithm to base e. Logarithms to any other base are explicitly identified, e.g. log10. Note also that ()nris the same as nrC.
2
Turn over

1. Two tennis players, A and B, are playing a match. Let X be the number of serves faster than 125 mph served by A in one of his
service games and let Y be the number of these serves returned by B. 
The following probability model is proposed: P(X = 0) = 0.4, P(X = 1) = 0.3, P(X = 2) = 0.2 and P(X = 3) = 0.1. 


The conditional distribution of Y (given that X = x > 0) is binomial with parameters x and 0.4, and $P(Y = 0 | X = 0) = 1$. 
Assume that this model is correct when answering the following questions. (i) Find the joint probability distribution of X and Y and display it in the form of a two-way table. (7) (ii) Find the marginal distribution of Y and evaluate E(Y). (4) (iii) Find Cov(X, Y). (4) (iv) Use your joint probability distribution table to find the probability distribution of the number of serves faster than 125 mph that are not returned by B in a game. (5) 2. The joint probability density function of the random variables X and Y is ()()2211441(,)exp(1)(1),,.2fxyxyxxyπ=−−−−+−∞<<∞−∞<<∞ (i) Use integration to show that X has the Normal distribution with mean 1 and variance 2. (7) (ii) Use integration to show that the moment generating function of X is 2()exp()Xmttt=+. (7) (iii) Use the moment generating function to find 3()EX. (6)
3
3. Let 12,,,nXXX be a random sample from a distribution with probability density function 1()(1),01fxxxββ−=−<<, where β (> 0) is an unknown parameter. (i) Find the maximum likelihood estimator, ˆβ, of β. (7) (ii) Calculate the approximate variance of ˆ
βand use it to determine an approximate 95% confidence interval for β when n is large. (6) 
(iii) Show that (0.5)10.5iPXβ<=−. (3) (iv) Suppose now that the values of 12,,,nXXX are not known, but you do know Y, the number of the Xi less than 0.5. State the distribution of Y, and write down the likelihood function of β based on Y. (4) 4. (a) Define the bias, relative efficiency and efficiency of potential estimators of a population parameter, and explain briefly why these are useful when deciding between different estimators. (6) (b) The random variables 12,,,nYYY constitute a random sample from a discrete distribution, with 212(2)(1)iPYp=−=−, (0)(1)iPYpp==−, (1)iPYp==, 212(2)(1)iPYp==− and ()0iPYk== for 2,0,1or2k≠−, where p (0 < p < 1) is an unknown parameter. (i) Find the method of moments estimator, p, of p. Describe one unsatisfactory feature that this estimator possesses. (5) (ii) Find 2()iEY and hence obtain the variance of p. (5) (iii) Making use of results found in answering parts (i) and (ii), or otherwise, find an unbiased estimator of 2p based on iYΣ and 2iYΣ. 
(4)
