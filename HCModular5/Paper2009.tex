EXAMINATIONS OF THE HONG KONG STATISTICAL SOCIETY
HIGHER CERTIFICATE IN STATISTICS, 2009
MODULE 5 : Further probability and inference
Time allowed: One and a half hours
Candidates should answer THREE questions.
Each question carries 20 marks.
The number of marks allotted for each part-question is shown in brackets.
Graph paper and Official tables are provided.
Candidates may use calculators in accordance with the regulations published in
the Society's "Guide to Examinations" (document Ex1).
The notation log denotes logarithm to base e.
Logarithms to any other base are explicitly identified, e.g. log10.
Note also that is the same as ⎟⎟⎠⎞⎜⎜⎝⎛rnnCr.
1 HC Module 5 2009
This examination paper consists of 4 printed pages each printed on one side only.
This front cover is page 1.
Question 1 starts on page 2.
There are 4 questions altogether in the paper.
©RSS 2009
1. The random variables X and Y are jointly distributed with probability density function
112,1(,)3log20otherwisexyxyfxyyx⎧⎛⎞ +≤≤≤≤⎪⎜⎟=⎨⎝⎠⎪⎩ .
(i) Find the marginal probability density function of X.
(4)
(ii) Show that and ()1.4991EX≈Var()0.0847X≈.
(5)
(iii) Show that . ()2.2442EXY≈
(4)
(iv) Find the covariance of X and Y.
(3)
(v) Find the conditional probability density function (|)fyx, for , and hence evaluate 12,1xy≤≤≤≤ (1.5|1PYX<=.
(4)
2. The random variable X has the distribution (k = 1, 2, 3, …), which has moment generating function (mgf) 2χk()/2()12kmtt−=− for 12t<.
(i) Using the mgf, find the mean and variance of X.
(6)
(ii) In the case k = 4, the probability density function is given by
/214()xfxxe−= (x > 0).
Using integration, confirm that the mgf of the distribution is 24χ()2()12mtt−=− (for 12t<), as given by the above formula.
(5)
(iii) Show that if are independent, each with a distribution, then has a distribution. 12,,,nYYY…21χ1niiVY==Σ2χn
(4)
(iv) Use the previous results and the central limit theorem to find the approximate probability that when n = 300. 310V≤
(5)
2
Turn over
3. For a productive pair from a particular species of bird, the number X of eggs laid per season has the probability mass function
()!kePXkCkλλ−== (k = 1, 2, 3, … ),
where C is a constant and λ (> 0) is an unknown parameter.
(i) Show that 11Ceλ−=−.
(4)
(ii) The numbers of eggs in a random sample of n nests of productive pairs are X1, X2, …, Xn. Find ()λ􀁁, the logarithm of the likelihood of λ based on this sample, and find an equation satisfied by ˆ
λ, the maximum likelihood estimator. (Do not attempt to solve this equation.)
(5)
(iii) Find the approximate variance of the maximum likelihood estimator of λ for the random sample X1, X2, …, Xn, when n is large.
(5)
(iv) Plot the first derivative of ()λ􀁁 at λ = 2.0, 2.5, 3.0 and 3.5 for the case n = 10 and . Using your diagram, find an approximate value of the maximum likelihood estimator. 30iX=Σ
(6)
3
Turn over
4. The random variable Y has the geometric distribution, parameter p (0 < p < 1), i.e.
()(1)yPYypp==− for y = 0, 1, 2, … .
This distribution has probability generating function
()1(1)ptptπ=−− for 1(1)tp−<−.
(i) Using the probability generating function, or otherwise, show that the mean of this distribution is 1pp− and the variance is 21pp−.
(5)
(ii) The random variables Y1, Y2, …, Yn constitute a random sample from this distribution.
Define /iYY=Σ . Show that Y is a biased estimator of 1/p. Hence find an unbiased estimator of 1/p and show that it is also a consistent estimator of 1/p.
(6)
(iii) Find the method of moments estimator of p.
(4)
(iv) The random variable W is the number of the random variables Y1, Y2, …, Yn that take the value zero. (For example, if n = 5, Y1 = 1, Y2 = 0, Y3 = 3, Y4 = 0 and Y5 = 2, then W = 2.) State the distribution of W. Hence find an unbiased estimator of p based on W and give its variance. [The formulae for the mean and variance of standard distributions may be assumed.]
(5)
4
