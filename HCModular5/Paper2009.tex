\begin{enumerate}
\item The random variables X and Y are jointly distributed with probability density function
112,1(,)3log20otherwisexyxyfxyyx⎧⎛⎞ +≤≤≤≤⎪⎜⎟=⎨⎝⎠⎪⎩ .
(i) Find the marginal probability density function of X.
(4)
(ii) Show that and ()1.4991EX≈Var()0.0847X≈.
(5)
(iii) Show that . ()2.2442EXY≈
(4)
(iv) Find the covariance of X and Y.
(3)
(v) Find the conditional probability density function (|)fyx, for , and hence evaluate 12,1xy≤≤≤≤ (1.5|1PYX<=.

%%%%%%%%%%%%%%%%%%%%%%%%%%%%%%%%%%%
\item The random variable X has the distribution (k = 1, 2, 3, …), which has moment generating function (mgf) 2χk()/2()12kmtt−=− for 12t<.
\begin{enumerate}[(a)]
\item  Using the mgf, find the mean and variance of X.
\item  In the case k = 4, the probability density function is given by
/214()xfxxe−= (x > 0).
Using integration, confirm that the mgf of the distribution is 24χ()2()12mtt−=− (for 12t<), as given by the above formula.
\item Show that if are independent, each with a distribution, then has a distribution. 12,,,nYYY…21χ1niiVY==Σ2χn
\item Use the previous results and the central limit theorem to find the approximate probability that when n = 300. 310V≤
\end{enumerate}
%%%%%%%%%%%%%%%%%%%%%%%%%%%%%%%%%%%
\item For a productive pair from a particular species of bird, the number X of eggs laid per season has the probability mass function
()!kePXkCkλλ−== (k = 1, 2, 3, … ),
where C is a constant and λ (> 0) is an unknown parameter.

\begin{enumerate}[(a)]
\item Show that 11Ceλ−=−.
\item The numbers of eggs in a random sample of n nests of productive pairs are X1, X2, …, Xn. Find ()λ􀁁, the logarithm of the likelihood of λ based on this sample, and find an equation satisfied by ˆ
λ, the maximum likelihood estimator. (Do not attempt to solve this equation.)

\item Find the approximate variance of the maximum likelihood estimator of λ for the random sample X1, X2, …, Xn, when n is large.

\item Plot the first derivative of ()λ􀁁 at λ = 2.0, 2.5, 3.0 and 3.5 for the case n = 10 and . Using your diagram, find an approximate value of the maximum likelihood estimator. 30iX=Σ
\end{enumerate}
%%%%%%%%%%%%%%%%%%%%%%%%%%%%%%%%%%%
\item The random variable Y has the geometric distribution, parameter p (0 < p < 1), i.e.
()(1)yPYypp==− for y = 0, 1, 2, … .
This distribution has probability generating function
()1(1)ptptπ=−− for 1(1)tp−<−.

\begin{enumerate}[(a)]
\item Using the probability generating function, or otherwise, show that the mean of this distribution is 1pp− and the variance is 21pp−.

\item The random variables Y1, Y2, …, Yn constitute a random sample from this distribution.
Define /iYY=Σ . Show that Y is a biased estimator of 1/p. Hence find an unbiased estimator of 1/p and show that it is also a consistent estimator of 1/p.
\item Find the method of moments estimator of p.

\item The random variable W is the number of the random variables Y1, Y2, …, Yn that take the value zero. (For example, if n = 5, Y1 = 1, Y2 = 0, Y3 = 3, Y4 = 0 and Y5 = 2, then W = 2.) State the distribution of W. Hence find an unbiased estimator of p based on W and give its variance. [The formulae for the mean and variance of standard distributions may be assumed.]
\end{enumerate}
%%%%%%%%%%%%%
\end{enumerate}

\end{document}
