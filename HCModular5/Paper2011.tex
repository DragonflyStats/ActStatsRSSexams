1 HC Module 5 2011 HKSS
This examination paper consists of 5 printed pages, each printed on one side only.
This front cover is page 1.
Question 1 starts on page 2.
There are 4 questions altogether in the paper.
© RSS 2011
EXAMINATIONS OF THE HONG KONG STATISTICAL SOCIETY HIGHER CERTIFICATE IN STATISTICS, 2011 MODULE 5 : Further probability and inference Time allowed: One and a half hours Candidates should answer THREE questions. Each question carries 20 marks. The number of marks allotted for each part-question is shown in brackets. Graph paper and Official tables are provided. Candidates may use calculators in accordance with the regulations published in the Society's "Guide to Examinations" (document Ex1). The notation log denotes logarithm to base e. Logarithms to any other base are explicitly identified, e.g. log10. Note also that ()nris the same as nrC.
2
Turn over
1. The continuous random variables X and Y are jointly distributed with joint probability density function (,)(01,022)fxykxyxyx=≤≤≤≤− and zero elsewhere, where k is a constant. (i) Sketch the region where the joint density is non-zero. (2) (ii) Use integration to show that k = 6. (5) (iii) Find the marginal probability density function of X and use it to show that ()111216PX≤=. (5) (iv) Find (|)fyx, the conditional probability density function of Y given x, and use it to evaluate ()1122|PYX≤=. (4) (v) Evaluate ()1122|PYX≤≤. (4)
3
Turn over
2. (a) The random variables X and Y are jointly distributed with a bivariate Normal distribution. (i) Sketch a typical scatter plot of data from this distribution. (2) (ii) Define the five parameters usually used to specify this distribution. (2) (iii) State the names of the marginal distributions of X and Y. (2) (iv) What is the form of the conditional mean of Y given X = x, considered as a function of x? (2) (b) Suppose that the random variables V and W have 211N(,)μσ, 222N(,)μσ distributions respectively and that V and W are independent. (i) Given that the moment generating function of a 2N(,)μσ random variable is ()2212expttμσ+, find the distribution of S = V + W. (4) (ii) State the distribution of U = V – W. (1) (iii) Find E(SU) and Cov(S, U). (4) (iv) State the name of the joint distribution of S and U and give the values of the parameters of this distribution in terms of 2121,,μμσ and 22σ. (3)
4
Turn over
3. (a) Explain what is meant by the likelihood function, and why it may be useful in estimating the value of a parameter. (5) (b) The continuous random variable X has probability density function 32()(0)2xxefxxθθ−=> where 0θ> is an unknown parameter. A random sample of values X1, X2, …, Xn is available from this distribution. (i) Show that the maximum likelihood estimator of θ is 3ˆXθ=, where X is the sample mean of X1, X2, …, Xn. (5) (ii) Find the approximate distribution of ˆ
θwhen n is large and use this result to find an approximate 95% confidence interval for θ when n = 200 and6.0X=. (5) (iii) Show that ˆ
θis a biased estimator of θ when n = 1. (5)
5
4. The discrete random variable X has probability distribution given by 2()(1)(1)(0,1,2,3,)kPXkkppk==+−= where p (0 < p < 1) is an unknown parameter. (i) Show that the probability generating function of X is given by 212()(for ||(1))1(1)()pttptpπ−=<−−−. [You may use the results that, for 01||1,1kkaaa∞=<=−Σ and 20(1)kkakaa∞==−Σ.] (5) (ii) Use the probability generating function to find the mean and variance of X. (6) (iii) A random sample X1, X2, …, Xn is available from this distribution. Find the method of moments estimator of p. (3) (iv) Let Yi take the value 1 if Xi = 0, and the value 0 otherwise, for i = 1, 2, …, n. State the distribution of 1niiUY==Σ. (2) (v) Find an unbiased estimator of p2 based on U and show that this estimator is consistent. [You do not need to prove any results you use concerning the moments of standard distributions.] (4)
