\begin{enumerate}
\item Jane chooses a number X at random from the set of numbers {1, 2, 3, 4}, so that
P(X = k) = ¼ for k = 1, 2, 3, 4.
She then chooses a number Y at random from the subset of numbers {X, …, 4}; for example, if X = 3, then Y is chosen at random from {3, 4}.
\begin{enumerate}[(a)]
\item Find the joint probability distribution of X and Y and display it in the form of a two-way table.
\item Find the marginal probability distribution of Y, and hence find E(Y) and Var(Y).
\item Show that $Cov(X, Y) = 5/8$.
\item Find the probability distribution of U = X + Y.
\end{enumerate}

%%%%%%%%%%%%%%%%%%

\item  Define the probability generating function and the moment generating function of a random variable X and give the relationship between these two functions.
(3)
The random variable X has the binomial distribution with parameters n (n > 3) and p (0 < p < 1).
\begin{enumerate}[(a)]
\item  Show that the probability generating function of X is
npptt)1()(−+=π
for .∞<<∞−t
\item Use part (i) to show that $E(X) = np$ and $Var(X) = np(1 – p)$.
\item Find E(X 3).
\item Now suppose that X1, X2, …, Xm are independent random variables and Xi has the binomial distribution with parameters ni and p for i = 1, 2, …, m. Let . Find the probability generating function of Y, and hence deduce the distribution of Y. Σ==miiXY1
\end{enumerate}

%%%%%%%%%%%%%%%%%%

\item  A random sample of n independent observations X1, X2, …, Xn is taken from a population which has probability density function /2()xxefxλλ−=, x > 0,
where λ (λ > 0) is an unknown parameter. The sample mean is denoted by X.
(i) Show that ˆ/2Xλ= is the method of moments estimator of λ.
(6)
(ii) Show that ˆλ is an unbiased estimator of λ and find Var(ˆ
λ). Hence deduce that ˆλ is a consistent estimator of λ.
(9)
(iii) Suppose that n = 3 and the alternative estimator 11112848XXXλ=++􀀄
has been proposed. Find the relative efficiency of this estimator compared to ˆλ
and say, with reasons, which estimator you prefer.
(5)

%%%%%%%%%%%%%%%%%%
\item  A random sample $X1$, $X2$, $\ldots$, $Xn$ is drawn from the Normal distribution with mean 0 and variance θ.
\begin{enumerate}[(a)]
\item Obtain the likelihood function.
\item Find the maximum likelihood estimator, , of θˆ θ.
\item  Using a large sample property of maximum likelihood estimators, find the approximate distribution of when n is large. θˆ
\item Find an approximate 95\% confidence interval for θ when n = 100 and . Σ=10002iX
\end{enumerate}

\end{enumerate}
