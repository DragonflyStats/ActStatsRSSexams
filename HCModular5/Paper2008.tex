\begin{enumerate}
\item Jane chooses a number X at random from the set of numbers {1, 2, 3, 4}, so that
P(X = k) = ¼ for k = 1, 2, 3, 4.
She then chooses a number Y at random from the subset of numbers {X, …, 4}; for example, if X = 3, then Y is chosen at random from {3, 4}.
\begin{enumerate}[(a)]
\item Find the joint probability distribution of X and Y and display it in the form of a two-way table.
\item Find the marginal probability distribution of Y, and hence find E(Y) and Var(Y).
\item Show that $Cov(X, Y) = 5/8$.
\item Find the probability distribution of U = X + Y.
\end{enumerate}

%%%%%%%%%%%%%%%%%%

\item  Define the probability generating function and the moment generating function of a random variable X and give the relationship between these two functions.
(3)
The random variable X has the binomial distribution with parameters n (n > 3) and p (0 < p < 1).
\begin{enumerate}[(a)]
\item  Show that the probability generating function of X is
npptt)1()(−+=π
for .∞<<∞−t
\item Use part (i) to show that $E(X) = np$ and $Var(X) = np(1 – p)$.
\item Find E(X 3).
\item Now suppose that X1, X2, …, Xm are independent random variables and Xi has the binomial distribution with parameters ni and p for i = 1, 2, …, m. Let . Find the probability generating function of Y, and hence deduce the distribution of Y. Σ==miiXY1
\end{enumerate}

%%%%%%%%%%%%%%%%%%

\item  A random sample of n independent observations X1, X2, …, Xn is taken from a population which has probability density function /2()xxefxλλ−=, x > 0,
where λ (λ > 0) is an unknown parameter. The sample mean is denoted by X.
(i) Show that ˆ/2Xλ= is the method of moments estimator of λ.
(6)
(ii) Show that ˆλ is an unbiased estimator of λ and find Var(ˆ
λ). Hence deduce that ˆλ is a consistent estimator of λ.
(9)
(iii) Suppose that n = 3 and the alternative estimator 11112848XXXλ=++􀀄
has been proposed. Find the relative efficiency of this estimator compared to ˆλ
and say, with reasons, which estimator you prefer.
(5)

Sure, let's go through each part step by step:

### (i) Method of Moments Estimator of λ

The probability density function given is:
\[ f(x) = \frac{\lambda}{2} e^{-\lambda x/2}, \quad x > 0 \]

The first moment (mean) of the exponential distribution is given by:
\[ E(X) = \frac{1}{\lambda/2} = \frac{2}{\lambda} \]

The sample mean \( \bar{X} \) is an unbiased estimator of the population mean \( E(X) \):
\[ \bar{X} \approx E(X) = \frac{2}{\lambda} \]

Rearranging to solve for λ:
\[ \hat{\lambda} = \frac{2}{\bar{X}} \]

Thus, the method of moments estimator for λ is \( \hat{\lambda} = \frac{2}{\bar{X}} \).

### (ii) Unbiased Estimator and Variance of \( \hat{\lambda} \)

To show that \( \hat{\lambda} = \frac{2}{\bar{X}} \) is an unbiased estimator of λ:

\[ E(\hat{\lambda}) = E\left(\frac{2}{\bar{X}}\right) \]

Since \( \bar{X} \) is the mean of an exponential distribution with mean \( \frac{2}{\lambda} \):
\[ \bar{X} \sim \text{Gamma}(n, \frac{\lambda}{2}) \]

The expected value of \( \frac{2}{\bar{X}} \) is:
\[ E\left(\frac{2}{\bar{X}}\right) = \frac{2}{E(\bar{X})} = \frac{2}{\frac{2}{\lambda}} = \lambda \]

Thus, \( \hat{\lambda} = \frac{2}{\bar{X}} \) is an unbiased estimator of λ.

Now, to find \( \text{Var}(\hat{\lambda}) \):
\[ \text{Var}(\hat{\lambda}) = \text{Var}\left(\frac{2}{\bar{X}}\right) \]

Using the delta method:
\[ \text{Var}\left(\frac{2}{\bar{X}}\right) \approx \left( \frac{-2}{E(\bar{X})^2} \right)^2 \text{Var}(\bar{X}) \]

Since \( \text{Var}(\bar{X}) = \frac{\sigma^2}{n} \) for a sample mean:
\[ \text{Var}(\bar{X}) = \frac{4/\lambda^2}{n} = \frac{4}{n\lambda^2} \]

Therefore:
\[ \text{Var}(\hat{\lambda}) = \left( \frac{2}{2/\lambda} \right)^2 \frac{4}{n\lambda^2} = \frac{4\lambda^2}{n\lambda^2} = \frac{4}{n} \]

Since \( \text{Var}(\hat{\lambda}) \to 0 \) as \( n \to \infty \), \( \hat{\lambda} \) is a consistent estimator of λ.

### (iii) Relative Efficiency of Alternative Estimator

Given \( \lambda_1 = \frac{48}{X_1 + 2X_2 + 8X_3} \) for \( n = 3 \), we need to find its efficiency compared to \( \hat{\lambda} = \frac{2}{\bar{X}} \).

First, calculate the mean and variance of \( \lambda_1 \):
\[ E(\lambda_1) = E\left(\frac{48}{X_1 + 2X_2 + 8X_3}\right) \]

By linearity of expectation and properties of the exponential distribution:
\[ E(X_1 + 2X_2 + 8X_3) = E(X_1) + 2E(X_2) + 8E(X_3) = \frac{2}{\lambda} + \frac{4}{\lambda} + \frac{16}{\lambda} = \frac{22}{\lambda} \]

Thus, \( E(\lambda_1) = \frac{48\lambda}{22} = \frac{24}{11} \lambda \).

Now, find the variance:
\[ \text{Var}(\lambda_1) = \text{Var}\left(\frac{48}{X_1 + 2X_2 + 8X_3}\right) \]

Since \( X_1 + 2X_2 + 8X_3 \) is a sum of independent exponentials, its variance can be found using the Gamma distribution properties.

The relative efficiency is given by:
\[ \text{Efficiency} = \frac{\text{Var}(\hat{\lambda})}{\text{Var}(\lambda_1)} \]

Comparing variances:
\[ \text{Var}(\lambda_1) \text{ and } \text{Var}(\hat{\lambda}) \]

Finally, choose the estimator with lower variance (higher efficiency). Given the complexity of calculating exact variances here, the estimator \( \hat{\lambda} \) is generally preferred due to its simplicity and unbiased properties.

Let me know if you need any further clarification on these steps!

%%%%%%%%%%%%%%%%%%
\item  A random sample $X1$, $X2$, $\ldots$, $Xn$ is drawn from the Normal distribution with mean 0 and variance θ.
\begin{enumerate}[(a)]
\item Obtain the likelihood function.
\item Find the maximum likelihood estimator, , of θˆ θ.
\item  Using a large sample property of maximum likelihood estimators, find the approximate distribution of when n is large. θˆ
\item Find an approximate 95\% confidence interval for θ when n = 100 and . Σ=10002iX
\end{enumerate}

\end{enumerate}
