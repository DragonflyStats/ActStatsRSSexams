
\begin{enumerate}
\item The continuous random variables X and Y are jointly distributed with joint probability
density function
(1 ) 0, 0,
( , )
0 otherwise,
y x xe x y
f x y
and the discrete random variables W and V are defined by
0 if 1, 0 if 1,
1 if 1, 1 if 1.
X Y
W V
X Y
\begin{enumerate}
\item Find the marginal probability density functions of X and Y.
\item Find the conditional density function f (y | x) for x > 0 and hence evaluate
E(Y | X = x).
\item Find $P(W = 1), P(V = 1)$ and $P(W = V = 1)$.
\item Find $Cov(W, V)$.
\end{enumerate}
%%%%%%%%%%%%%%%%%%%%%%%%%%%%%%%%%%%%
\item (a) Suppose that X is a discrete random variable which can only take non-negative
integer values (i.e. 0, 1, 2, …). Define the probability generating function X(t)
of X.
(2)
(b) The discrete random variable Y has probability distribution
1
( ) for 1, 2, 3, ,
( 1)!
k e
P Y k k
k
where is a positive parameter.
(i) Show that the probability generating function of Y is ( t 1) te .
(4)
(ii) Use the probability generating function of Y to evaluate the mean and
variance of Y. Outline how you would find E(Y
3
).
(8)
(iii) Suppose that Y1, Y2, …, Yn are independent random variables, each with
the same distribution as Y, and that
1
n
i
i
W Y. Find the probability
generating function of W and hence find P(W = k) for k = 0, 1, 2, 3, … .
(6)
%%%%%%%%%%%%%%%%%%%%%%%%%%%%%%%%%%%%
\item (a) What is meant by a maximum likelihood estimator? What good properties do
maximum likelihood estimators possess, assuming that the necessary regularity
conditions hold?
(8)
(b) The probability of obtaining heads when a coin is tossed is p, where 0 < p < 1
and p is unknown. A game consists of two independent tosses of the coin, with
the player winning if there is exactly one head. In 100 independent games, the
player wins 30 times.
(i) Show that the likelihood of p is proportional to
30 30 70 p (1 p) (1 2p(1 p))
for 0 < p < 1.
(5)
(ii) The plot of the logarithm of the likelihood shown below indicates that
there are two values of p that give the same maximum value for the
likelihood. Explain why this happens and find an equation satisfied by
these values. (Do not attempt to solve this equation.)
%%%%%%%%%%%%%%%%%%%%%%%%%%%%%%%%%%%%
\item Random variables X1 and X2 are independent, each having a Normal distribution with
mean and variance both equal to > 0, an unknown parameter. It is required to use
X1 and X2 to estimate .

\begin{enumerate}
\item Find the method of moments estimator ˆ of based on the first sample
moment. Show that ˆ is unbiased and find its variance.
\item  The random variable Y = X1 – X2. State the distribution of Y.
\item Show that ˆ and Y are uncorrelated.
\item Another unbiased estimator of is 2 kY , where k is a constant. Find the
value of k.
\item Show that E(Y
4) = 12
2
.
(4)
[You may use the result that the moment generating function of a Normal
distribution with mean and variance
2 is 1 2 2
2 exp( t t ) .]
\item Use part (e) to find the variance of , and hence the efficiency of ˆ relative
to .
\end{enumerate}


\end{enumerate}
