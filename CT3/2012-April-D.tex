 88 lines (72 sloc) 2.45 KB
\documentclass[a4paper,12pt]{article}

%%%%%%%%%%%%%%%%%%%%%%%%%%%%%%%%%%%%%%%%%%%%%%%%%%%%%%%%%%%%%%%%%%%%%%%%%%%%%%%%%%%%%%%%%%%%%%%%%%%%%%%%%%%%%%%%%%%%%%%%%%%%%%%%%%%%%%%%%%%%%%%%%%%%%%%%%%%%%%%%%%%%%%%%%%%%%%%%%%%%%%%%%%%%%%%%%%%%%%%%%%%%%%%%%%%%%%%%%%%%%%%%%%%%%%%%%%%%%%%%%%%%%%%%%%%%

\usepackage{eurosym}
\usepackage{vmargin}
\usepackage{amsmath}
\usepackage{graphics}
\usepackage{epsfig}
\usepackage{enumerate}
\usepackage{multicol}
\usepackage{subfigure}
\usepackage{fancyhdr}
\usepackage{listings}
\usepackage{framed}
\usepackage{graphicx}
\usepackage{amsmath}
\usepackage{chngpage}

%\usepackage{bigints}
\usepackage{vmargin}

% left top textwidth textheight headheight

% headsep footheight footskip

\setmargins{2.0cm}{2.5cm}{16 cm}{22cm}{0.5cm}{0cm}{1cm}{1cm}

\renewcommand{\baselinestretch}{1.3}

\setcounter{MaxMatrixCols}{10}

\begin{document}
\begin{enumerate}
10
In a portfolio of car insurance policies, the number of accident-related claims, N,
made by a policyholder in a year has the following distribution:
No. of claims, n
Probability
0
1
2
0.4 0.4 0.2
The number of cars, X, involved in each accident that results in a claim is distributed
as follows:
No. of cars, x
Probability
1
2
0.7 0.3
It can be assumed that the occurrence of a claim and the number of cars involved in
the accident are independent. Furthermore, claims made by a policyholder in any year
are also independent of each other. Let S be the total number of cars involved in
accidents related to such claims by a policyholder in a year.
(i)
(a)
(b)
Determine the probability function of S.
Hence find E(S).
[4]
The expectation E(S) can also be calculated using the formula
2
E ( S ) = ∑ E ( S | N = n ) Pr ( N = n ) .
n = 0
(ii)
(a)
(b)
Find E ( S | N = n ) for n = 0,1 , 2 .
Hence calculate E(S).
[4]
[Total 8]
\end{document}
