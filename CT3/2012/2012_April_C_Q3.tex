
\documentclass[a4paper,12pt]{article}

%%%%%%%%%%%%%%%%%%%%%%%%%%%%%%%%%%%%%%%%%%%%%%%%%%%%%%%%%%%%%%%%%%%%%%%%%%%%%%%%%%%%%%%%%%%%%%%%%%%%%%%%%%%%%%%%%%%%%%%%%%%%%%%%%%%%%%%%%%%%%%%%%%%%%%%%%%%%%%%%%%%%%%%%%%%%%%%%%%%%%%%%%%%%%%%%%%%%%%%%%%%%%%%%%%%%%%%%%%%%%%%%%%%%%%%%%%%%%%%%%%%%%%%%%%%%

\usepackage{eurosym}
\usepackage{vmargin}
\usepackage{amsmath}
\usepackage{graphics}
\usepackage{epsfig}
\usepackage{enumerate}
\usepackage{multicol}
\usepackage{subfigure}
\usepackage{fancyhdr}
\usepackage{listings}
\usepackage{framed}
\usepackage{graphicx}
\usepackage{amsmath}
\usepackage{chngpage}

%\usepackage{bigints}
\usepackage{vmargin}

% left top textwidth textheight headheight

% headsep footheight footskip

\setmargins{2.0cm}{2.5cm}{16 cm}{22cm}{0.5cm}{0cm}{1cm}{1cm}

\renewcommand{\baselinestretch}{1.3}

\setcounter{MaxMatrixCols}{10}

\begin{document}

\item 
Two students are selected at random without replacement from a group of 100
students, of whom 64 are male and 36 are female.
Calculate the probability that the two selected students are of different genders.

\end{enumerate}
\newpage

%%%%%%%%%%%%%%%%%%%%%%%%%%%%%%%%%%%%%%%%
3
P(1st selected is male and 2 nd selected is female) = 64 36
.
100 99
P(1st selected is female and 2 nd selected is male) = 36 64
.
100 99
⇒ P(selected students are of different genders) = 2.
64 36 128
. =
= 0.465
100 99 275
⎛ 64 ⎞ ⎛ 36 ⎞
⎜ ⎟⎜ ⎟
1
1
64 \times 36 \times 2
= 0.465 ]
[OR P(selected students are of different genders) = ⎝ ⎠ ⎝ ⎠ =
100 \times 99
⎛ 100 ⎞
⎜
⎟
⎝ 2 ⎠
Very well managed by most candidates. Some tried to calculate the probabilities with
replacement.
% Page 3
%%%%%%%%%%%%%%%%%%%%%%%%%%%%%%%%%%%%5 – April 2012 – %%%%%%%%%%%%%%%%%%%%%%%%%%%%%%%%

\end{document}
