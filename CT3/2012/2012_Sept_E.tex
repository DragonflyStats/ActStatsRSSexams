\documentclass[a4paper,12pt]{article}

%%%%%%%%%%%%%%%%%%%%%%%%%%%%%%%%%%%%%%%%%%%%%%%%%%%%%%%%%%%%%%%%%%%%%%%%%%%%%%%%%%%%%%%%%%%%%%%%%%%%%%%%%%%%%%%%%%%%%%%%%%%%%%%%%%%%%%%%%%%%%%%%%%%%%%%%%%%%%%%%%%%%%%%%%%%%%%%%%%%%%%%%%%%%%%%%%%%%%%%%%%%%%%%%%%%%%%%%%%%%%%%%%%%%%%%%%%%%%%%%%%%%%%%%%%%%

\usepackage{eurosym}
\usepackage{vmargin}
\usepackage{amsmath}
\usepackage{graphics}
\usepackage{epsfig}
\usepackage{enumerate}
\usepackage{multicol}
\usepackage{subfigure}
\usepackage{fancyhdr}
\usepackage{listings}
\usepackage{framed}
\usepackage{graphicx}
\usepackage{amsmath}
\usepackage{chngpage}

%\usepackage{bigints}
\usepackage{vmargin}

% left top textwidth textheight headheight

% headsep footheight footskip

\setmargins{2.0cm}{2.5cm}{16 cm}{22cm}{0.5cm}{0cm}{1cm}{1cm}

\renewcommand{\baselinestretch}{1.3}

\setcounter{MaxMatrixCols}{10}

\begin{document}
\begin{enumerate}

%% - Question 8
\[E[N] = \sumn P(N = n) = 0.3 + 0.6 + 0.6 + 0.4 = 1.9\]
\[E[N 2 ] = \sumn 2 P(N = n) = 0.3 + 1.2 + 1.8 + 1.6 = 4.9\]
\[V[N] = E[N 2 ] – (E[N]) 2 = 1.29\]
Also E[Y] = exp( \mu + \sigma^2 / 2 ) = e 0.55 = 1.73325
( )
V [ Y ] = ( E [ Y ] ) 2 (exp \sigma^2 − 1 ) = 1.73325 2 * (e 0.1 – 1) = 0.31595
Using known results
%%%%%%%%%%%%%%%%%%%%%%%%%%%%%%%%%%%%5 – September 2012 – %%%%%%%%%%%%%%%%%%%%%%%%%%%%%%%%
\[E[S] = E[N] E[Y] = 1.9 * 1.73325 = 3.293\]
\[V[S] = E[N] V[Y] + V[N] (E[Y]) 2 = 0.60031 + 3.87536 = 4.476\]
Some frequent errors were due to mis-interpretation of the mean and variance of the log-
normal distribution.
9
(i)
α
= 8 = 2.8
mean =
(b) As claims are non-negative and the s.d. is quite large relative to the
mean, then the distribution will be quite positively skewed.
F ( x )
λ 2
1
x
(ii)
α
= 4 and s.d. =
λ
(a)
1 − t
= \int te 2 dt
4
0
1
x
− t
1
= − \int td ( e 2 )
2
0
1
x
1
1 − t
1 − t
= − [ te 2 ] 0 x + \int e 2 dt
2
2
0
1
1
− t
1 − x
= − xe 2 − [ e 2 ] 0 x
2
1
− x
1
= 1 − (1 + x ) e 2
2
1
(iii)
− x
1
i.e. 1 − (1 + x ) e 2 = u
2
(a) F ( x ) = u
(b) This equation would have to be solved numerically
(c) Using u = 0.66 on the vertical axis, we invert to get x = 4.5 on the
horizontal axis.
In part (ii) many candidates failed to integrate correctly. A lot of problems were caused by not using the correct limits for the integral. In part (iii) a popular answer was to use “trial-
and-error”, which is not an appropriate approach here.
Page 6%%%%%%%%%%%%%%%%%%%%%%%%%%%%%%%%%%%%5 – September 2012 – %%%%%%%%%%%%%%%%%%%%%%%%%%%%%%%%

%%%%%%%%%%%%%%%%%%%%%%%%%%%%%%%%%%%%%%%%%%%%%%%%%%%%%
10
(i)
Y ij = \mu + τ i + ε ij
with ε ij being i.i.d. N (0, \sigma^2 )
In particular, it is assumed that the variance is the same in all groups.
(ii)
Source of variation
Between regions
Residual
(iii)
d.f.
3
16
SS
4.4655
8.892
MSS
1.4885
0.55575
H_{0} : τ i = 0 for all groups i
F = 2.6784 should be from F distribution with 3,16 d.f.
From the tables we know that this gives a p-value of 0.086 (with
interpolation).
Reject at 10%, not at 5%, some but very weak evidence against H_{0}
Mainly well answered. Care is required in calculating the p-values correctly. Also, a number
of candidates had difficulties in writing down a sensible form of the ANOVA model in part (i).
%%%%%%%%%%%%%%%%%%%%%%%%%%%%%%%%%

\newpage

[7]
9
An analyst is interested in using a gamma distribution with parameters α = 2 and
1
1 − x
λ = 1⁄2, that is, with density function f ( x ) = xe 2 , 0 < x < ∞ .
4
(i)
(a) State the mean and standard deviation of this distribution.
(b) Hence comment briefly on its shape.

(ii)
Show that the cumulative distribution function is given by
1
− x
1
F ( x ) = 1 − (1 + x ) e 2 , 0 < x < ∞
2
(zero otherwise).

The analyst wishes to simulate values x from this gamma distribution and is able to
generate random numbers u from a uniform distribution on (0,1).
(iii)
(a) Specify an equation involving x and u, the solution of which will yield
the simulated value x.
(b) Comment briefly on how this equation might be solved.
(c) The graph below gives F(x) plotted against x. Use this graph to obtain
the simulated value of x corresponding to the random number u = 0.66.

[Total 8]
CT3 S2012–410
The number of hours that people watch television per day is the subject of an
empirical study that is carried out in four regions in a country. Five people are
randomly selected in each of the regions and are asked about the average number of
hours per day that they spent watching television during the last year. The results are
shown in the following table, with the last column shows the average in each region.
Region 1
Region 2
Region 3
Region 4
2.0
1.2
2.5
1.2
1.1
1.0
2.0
1.7
0.2
0.9
2.6
1.0
3.8
1.1
2.4
1.8
2.8
1.6
2.3
1.3
Average
1.98
1.16
2.36
1.40
Based on the above observations the following ANOVA table was obtained:
\begin{verbatim}
Source of variation
Between regions
Residual
d.f.
...
...
SS
4.4655
8.892
MSS
...
...
\end{verbatim}

\begin{enumerate}[(i)]
\item State the mathematical model underlying the one-way analysis of variance
together with all associated assumptions.

\item Complete the ANOVA table.
\item Carry out an analysis of variance to test the hypothesis that the region has no
effect on the average time spent watching television. You should write down
the null hypothesis, calculate the value of the test-statistic, state its distribution
including any parameters, calculate the p-value approximately and state your
conclusion.
\end{enumerate}
\end{document}
