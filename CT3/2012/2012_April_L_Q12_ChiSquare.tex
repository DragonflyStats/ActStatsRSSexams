\documentclass[a4paper,12pt]{article}

%%%%%%%%%%%%%%%%%%%%%%%%%%%%%%%%%%%%%%%%%%%%%%%%%%%%%%%%%%%%%%%%%%%%%%%%%%%%%%%%%%%%%%%%%%%%%%%%%%%%%%%%%%%%%%%%%%%%%%%%%%%%%%%%%%%%%%%%%%%%%%%%%%%%%%%%%%%%%%%%%%%%%%%%%%%%%%%%%%%%%%%%%%%%%%%%%%%%%%%%%%%%%%%%%%%%%%%%%%%%%%%%%%%%%%%%%%%%%%%%%%%%%%%%%%%%

\usepackage{eurosym}
\usepackage{vmargin}
\usepackage{amsmath}
\usepackage{graphics}
\usepackage{epsfig}
\usepackage{enumerate}
\usepackage{multicol}
\usepackage{subfigure}
\usepackage{fancyhdr}
\usepackage{listings}
\usepackage{framed}
\usepackage{graphicx}
\usepackage{amsmath}
\usepackage{chngpage}

%\usepackage{bigints}
\usepackage{vmargin}

% left top textwidth textheight headheight

% headsep footheight footskip

\setmargins{2.0cm}{2.5cm}{16 cm}{22cm}{0.5cm}{0cm}{1cm}{1cm}

\renewcommand{\baselinestretch}{1.3}

\setcounter{MaxMatrixCols}{10}

\begin{document}
\begin{enumerate}
%CT3 A2012–612

\item Consider a random sample $X_1 , ... , X_k$ of size k = 400 . Statistician A wants to use a
\chi^2-test to test the hypothesis that the distribution of $X_i$ is a binomial distribution
with parameters n = 2 and unknown p based on the following observed frequencies
of outcomes of X i :
Possible realisation of X i 0 1 2
Frequency 90 220 90
\begin{enumerate}[(i)]
\item (i) Estimate the parameter p using the method of moments.
\item (ii) Test the hypothesis that X i has a binomial distribution at the 0.05 significance
level using the data in the above table and the estimate of p obtained in
part (i).

Statistician B assumes that the data are from a binomial distribution and wants to test
the hypothesis that the true parameter is p 0 = 0.5 .
\item (iii)
Explain whether there is any evidence against this hypothesis by using the
estimate of p in part (i) and without performing any further calculations. 
Statistician C wants to test the hypothesis that the random variables X i have a
binomial distribution with known parameters n = 2 and p = 0.5 .
\item (iv) Write down the null hypothesis and the alternative hypothesis for the test in
this situation.

\item (v) Carry out the test at the significance level of 0.05 stating your decision.
\item (vi) Explain briefly the relationship between the test decisions in parts (ii), (iii) and
(v), and in particular whether there is any contradiction.
\end{enumerate}
\end{enumerate}

%%--- CT3 A2012–7


%%%%%%%%%%%%%%%%%%%%
\newpage
%% -Question 12
(i)
p ˆ =
X 220 + 2* 90
=
= 0.5
n
400* 2
We obtain the following table to test H_{0} :
Possible realisation of X i 0 1 2
Number of observations 90 220 90
expected frequency under H_{0} 100 200 100
100 400 100
1 2 1
( f j − e j ) 2
( f j − e j ) 2 / e j
2
The test-statistic is = \sum ( f j − e j ) 2 / e j . For the given data the value of C is
j = 0
c = 4 .
C is \chi^2-distributed with 3−1−1 = 1 degree of freedom.
H_{0} is rejected since the (1 − \alpha ) -quantile (\alpha = 0.05) of the \chi^2-distribution with
one degree of freedom is 3.841 < 4.
\begin{itemize}
    \item (iii) Since the estimated value is 0.5, any reasonable test will not reject that value,
since the value 0.5 will always be in the acceptance region of the test. In other
words, 0.5 will always be in any confidence interval around the estimate 0.5.
\item (iv) We now have: $H_{0} : X i ~ Bin (2, 0.5)$ and
H_{1} : X i does not follow Bin (2, 0.5) (emphasis on both Bin, p = 0.5 )
\item (v)
The value of the test-statistic is still c = 4 but the distribution of C is now a
$\chi^2$-distribution with 3−1 = 2 degrees of freedom.
\end{itemize}

Page 8%%%%%%%%%%%%%%%%%%%%%%%%%%%%%%%%%%%%5 – April 2012 – %%%%%%%%%%%%%%%%%%%%%%%%%%%%%%%%
\begin{itemize}
\item Now $H_{0}$ is NOT rejected at a 5%-level since the (1 − \alpha ) -quantile
$(\alpha = 0.05)$ of the $\chi^2$-distribution with two degrees of freedom is
5.991 > 4.
\item (vi)
The result in part (ii) states that a binomial distribution does not fit the data
well and is rejected. However, in part (iii) we found that, under the assumption
of a binomial distribution, p 0 = 0.5 cannot be rejected. A specific binomial
distribution with parameter p = 0.5 is not rejected in part (v) for the same
data. 
\item The reason is that the additional degree of freedom in part (v) allows for
a larger value of the test-statistic under the null.
Most candidates answered very well the parts of this question that concerned “knowledge”
and “application” aspects of the tests. However, there were problems with the comments and
reasoning.
\end{itemize}

\end{document}
