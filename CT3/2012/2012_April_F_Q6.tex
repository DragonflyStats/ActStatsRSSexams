\documentclass[a4paper,12pt]{article}

%%%%%%%%%%%%%%%%%%%%%%%%%%%%%%%%%%%%%%%%%%%%%%%%%%%%%%%%%%%%%%%%%%%%%%%%%%%%%%%%%%%%%%%%%%%%%%%%%%%%%%%%%%%%%%%%%%%%%%%%%%%%%%%%%%%%%%%%%%%%%%%%%%%%%%%%%%%%%%%%%%%%%%%%%%%%%%%%%%%%%%%%%%%%%%%%%%%%%%%%%%%%%%%%%%%%%%%%%%%%%%%%%%%%%%%%%%%%%%%%%%%%%%%%%%%%

\usepackage{eurosym}
\usepackage{vmargin}
\usepackage{amsmath}
\usepackage{graphics}
\usepackage{epsfig}
\usepackage{enumerate}
\usepackage{multicol}
\usepackage{subfigure}
\usepackage{fancyhdr}
\usepackage{listings}
\usepackage{framed}
\usepackage{graphicx}
\usepackage{amsmath}
\usepackage{chngpage}

%\usepackage{bigints}
\usepackage{vmargin}

% left top textwidth textheight headheight

% headsep footheight footskip

\setmargins{2.0cm}{2.5cm}{16 cm}{22cm}{0.5cm}{0cm}{1cm}{1cm}

\renewcommand{\baselinestretch}{1.3}

\setcounter{MaxMatrixCols}{10}

\begin{document}
In a random sample of 200 people taken from a large population of adults, 70 people
intend to vote for party A at the next election.
(i) Calculate an approximate equal-tailed 95% confidence interval for θ, the true
proportion of this population who intend to vote for party A at the next
election.
[3]
(ii) Give a brief interpretation of the interval calculated in part (i).

%%%%%%%%%%%%%

\begin{enumerate}


\item 
%% Question 6
A random sample of size n is taken from a gamma distribution with parameters $\alpha = 8$
and $\lambda = 1/\theta$. The sample mean is $X$ and $\theta$ is to be estimated.
\begin{enumerate}[(i)]
\item (i) Determine the method of moments estimator (MME) of $\theta$. 
\item (ii) Find the bias of the MME determined in part (i). 
\item (iii) (a) Determine the mean square error of the MME of $\theta$. (b) Comment on the efficiency of the MME of $\theta$ based on your answer in
part (iii)(a).
\end{enumerate}

%%%%%%%%%%%%%%%%%%%%%%%%%%%%%%%%%%6
(i)
⎛ \theta (1 − \theta ) ⎞
Use the normal approximation \theta ˆ ~ N ⎜ \theta ,
⎟
200 ⎠
⎝
\theta ˆ (1 − \theta ˆ )
to give the 95\% confidence interval \theta ˆ \pm 1.96
.
200
With \theta ˆ = 70 / 200 = 0.35 we obtain 0.35 \pm 0.066 , that is (0.284, 0.416).
(ii)
If we take a large number of samples from this population, we expect 95\% of
the resulting CIs to include the true value of \theta.
There were no problems with the first part. However, many candidates struggled with
providing a reasonable interpretation in part (ii).
Page 4%%%%%%%%%%%%%%%%%%%%%%%%%%%%%%%%%%%%5 – April 2012 – %%%%%%%%%%%%%%%%%%%%%%%%%%%%%%%%
