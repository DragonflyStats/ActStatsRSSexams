\documentclass[a4paper,12pt]{article}

%%%%%%%%%%%%%%%%%%%%%%%%%%%%%%%%%%%%%%%%%%%%%%%%%%%%%%%%%%%%%%%%%%%%%%%%%%%%%%%%%%%%%%%%%%%%%%%%%%%%%%%%%%%%%%%%%%%%%%%%%%%%%%%%%%%%%%%%%%%%%%%%%%%%%%%%%%%%%%%%%%%%%%%%%%%%%%%%%%%%%%%%%%%%%%%%%%%%%%%%%%%%%%%%%%%%%%%%%%%%%%%%%%%%%%%%%%%%%%%%%%%%%%%%%%%%

\usepackage{eurosym}
\usepackage{vmargin}
\usepackage{amsmath}
\usepackage{graphics}
\usepackage{epsfig}
\usepackage{enumerate}
\usepackage{multicol}
\usepackage{subfigure}
\usepackage{fancyhdr}
\usepackage{listings}
\usepackage{framed}
\usepackage{graphicx}
\usepackage{amsmath}
\usepackage{chngpage}

%\usepackage{bigints}
\usepackage{vmargin}

% left top textwidth textheight headheight

% headsep footheight footskip

\setmargins{2.0cm}{2.5cm}{16 cm}{22cm}{0.5cm}{0cm}{1cm}{1cm}

\renewcommand{\baselinestretch}{1.3}

\setcounter{MaxMatrixCols}{10}

\begin{document}
\begin{enumerate}

%%PLEASE TURN OVER8
\item In an analysis of variance investigation four treatments are compared using random
samples each of size 10. The total sum of squares is calculated as SS T = 673.5 and the
between-treatments sum of squares as SS B = 148.3.
\begin{enumerate}[(i)]
\item (i)
(a) Calculate an unbiased estimate of the error variance $\sigma^2$ .
(b) State the number of degrees of freedom associated with the estimate in
part (i)(a).

\item (ii) Suggest an unbiased estimator of $\sigma^2$ that is different from the one used in part
(i).
[1]
\item (iii) Comment on which of the two estimators should be used.
\end{enumerate}




\end{enumerate}
%%%%%%%%%%%%%%%%%%%%%%%%%%%%%%%%%%%8
(i)
(a) SS R = SS T − SS B = 673.5 − 148.3 = 525.2
σ ˆ 2 =
(b)
SS R 525.2
=
= 14.59
36
n − k
Associated d.f. 36
(ii) Alternatively, an unbiased estimator could be given using only part of the data,
( Y − Y i . ) 2
2 ∑ j ij
e.g. responses from treatment i: S i =
n i − 1
(iii) The estimator used in part (i) should be preferred as it is based on all data and
is therefore more accurate.

%%Part (i) was well answered, although the df were wrongly given in many answers. Answers in part (ii) were very poor – this question required good understanding of ANOVA concepts and critical thinking.

%%%%%%%%%%%%%%%%%%%%%%%%%%%%%%%%%5 – April 2012 – %%%%%%%%%%%%%%%%%%%%%%%%%%%%%%%%
\end{document}
