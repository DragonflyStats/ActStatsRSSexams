\documentclass[a4paper,12pt]{article}

%%%%%%%%%%%%%%%%%%%%%%%%%%%%%%%%%%%%%%%%%%%%%%%%%%%%%%%%%%%%%%%%%%%%%%%%%%%%%%%%%%%%%%%%%%%%%%%%%%%%%%%%%%%%%%%%%%%%%%%%%%%%%%%%%%%%%%%%%%%%%%%%%%%%%%%%%%%%%%%%%%%%%%%%%%%%%%%%%%%%%%%%%%%%%%%%%%%%%%%%%%%%%%%%%%%%%%%%%%%%%%%%%%%%%%%%%%%%%%%%%%%%%%%%%%%%

\usepackage{eurosym}
\usepackage{vmargin}
\usepackage{amsmath}
\usepackage{graphics}
\usepackage{epsfig}
\usepackage{enumerate}
\usepackage{multicol}
\usepackage{subfigure}
\usepackage{fancyhdr}
\usepackage{listings}
\usepackage{framed}
\usepackage{graphicx}
\usepackage{amsmath}
\usepackage{chngpage}

%\usepackage{bigints}
\usepackage{vmargin}

% left top textwidth textheight headheight

% headsep footheight footskip

\setmargins{2.0cm}{2.5cm}{16 cm}{22cm}{0.5cm}{0cm}{1cm}{1cm}

\renewcommand{\baselinestretch}{1.3}

\setcounter{MaxMatrixCols}{10}

\begin{document}
\begin{enumerate}
%CT3 A2012–612

\item Consider a random sample $X_1 , ... , X_k$ of size k = 400 . Statistician A wants to use a
\chi^2-test to test the hypothesis that the distribution of $X_i$ is a binomial distribution
with parameters n = 2 and unknown p based on the following observed frequencies
of outcomes of X i :
Possible realisation of X i 0 1 2
Frequency 90 220 90
\begin{enumerate}[(i)]
\item (i) Estimate the parameter p using the method of moments.
\item (ii) Test the hypothesis that X i has a binomial distribution at the 0.05 significance
level using the data in the above table and the estimate of p obtained in
part (i).

Statistician B assumes that the data are from a binomial distribution and wants to test
the hypothesis that the true parameter is p 0 = 0.5 .
\item (iii)
Explain whether there is any evidence against this hypothesis by using the
estimate of p in part (i) and without performing any further calculations. 
Statistician C wants to test the hypothesis that the random variables X i have a
binomial distribution with known parameters n = 2 and p = 0.5 .
\item (iv) Write down the null hypothesis and the alternative hypothesis for the test in
this situation.

\item (v) Carry out the test at the significance level of 0.05 stating your decision.
\item (vi) Explain briefly the relationship between the test decisions in parts (ii), (iii) and
(v), and in particular whether there is any contradiction.
\end{enumerate}
\end{enumerate}

%%--- CT3 A2012–7

\newpage

13
The quality of primary schools in eight regions in the UK is measured by an index
ranging from 1 (very poor) to 10 (excellent). In addition the value of a house price
index for these eight regions is observed. The results are given in the following table:
Region i
School quality index x i
House price index y i
1
2
3
4
5
6
7
8
Sum
7
8
5
8
4
9
6
9
56
195 195 170 190 150 190 200 210 1500
The last column contains the sum of all eight columns.
From these values we obtain the following results:
∑ x i y i = 10, 695; ∑ x i 2 = 416;
(i)
\sum y i 2 = 283, 750
Calculate the correlation coefficient between the index of school quality and
the house price index.

You can assume that the joint distribution of the two random variables is a bivariate
normal distribution.
(ii) Perform a statistical test for the null hypothesis that the true correlation
coefficient between the school quality index and the house price index is
equal to 0.8 against the alternative that the correlation coefficient is smaller
than 0.8, by calculating an approximate p-value.

(iii) Fit a linear regression model to the data, by considering the school quality
index as the explanatory variable. You should write down the model and
estimate all parameters.

(iv) Calculate the coefficient of determination R 2 for the regression model
obtained in part (iii).
(v)
Provide a brief interpretation of the slope of the regression model obtained in
part (iii).



CT3 A2012–8


%%%%%%%%%%%%%%%%%%%%
12
(i)
p ˆ =
X 220 + 2* 90
=
= 0.5
n
400* 2
We obtain the following table to test H_{0} :
Possible realisation of X i 0 1 2
Number of observations 90 220 90
expected frequency under H_{0} 100 200 100
100 400 100
1 2 1
( f j − e j ) 2
( f j − e j ) 2 / e j
2
The test-statistic is = ∑ ( f j − e j ) 2 / e j . For the given data the value of C is
j = 0
c = 4 .
C is \chi^2-distributed with 3−1−1 = 1 degree of freedom.
H_{0} is rejected since the (1 − \alpha ) -quantile (\alpha = 0.05) of the \chi^2-distribution with
one degree of freedom is 3.841 < 4.
(iii) Since the estimated value is 0.5, any reasonable test will not reject that value,
since the value 0.5 will always be in the acceptance region of the test. In other
words, 0.5 will always be in any confidence interval around the estimate 0.5.
(iv) We now have: H_{0} : X i ~ Bin (2, 0.5) and
H 1 : X i does not follow Bin (2, 0.5) (emphasis on both Bin, p = 0.5 )
(v)
The value of the test-statistic is still c = 4 but the distribution of C is now a
\chi^2-distribution with 3−1 = 2 degrees of freedom.
Page 8%%%%%%%%%%%%%%%%%%%%%%%%%%%%%%%%%%%%5 – April 2012 – %%%%%%%%%%%%%%%%%%%%%%%%%%%%%%%%
Now H_{0} is NOT rejected at a 5%-level since the (1 − \alpha ) -quantile
(\alpha = 0.05) of the \chi^2-distribution with two degrees of freedom is
5.991 > 4.
(vi)
The result in part (ii) states that a binomial distribution does not fit the data
well and is rejected. However, in part (iii) we found that, under the assumption
of a binomial distribution, p 0 = 0.5 cannot be rejected. A specific binomial
distribution with parameter p = 0.5 is not rejected in part (v) for the same
data. The reason is that the additional degree of freedom in part (v) allows for
a larger value of the test-statistic under the null.
Most candidates answered very well the parts of this question that concerned “knowledge”
and “application” aspects of the tests. However, there were problems with the comments and
reasoning.
13
(i)
S xx = 24, S yy = 2500, S_{xy} = 195
r =
(ii)
W =
S_{xy}
S xx S yy
= 0.796084
1
1 + ρ
1
1 + r
and
is normally distributed with mean log
log
2
1 − ρ
2
1 − r
standard deviation 1/ n − 3
observed value of W is w = 1.087828
Under $H_{0}$ the mean of $W$ is 1.098612
And the standard deviation is 0.447214
p-value is 

\begin{eqnarray*}
P [ W < 1.087828 ] &=& P [ Z < ( 1.087828 − 1.098612) / 0.447214]\\
&=& P [ Z < − 0.024113527 ] \\ &=& 1 − F (0.024113527) > 0.49\\
\end{eqnarray*}
%%%%%%%%%%%%%%%%%%%%%%%%%%%%%%%%%%%%%%%%%%%%%%%%%%%%%%%%%%
No evidence against the null hypothesis.
(iii)
Y i = a + bX i + ε i
{
For b we obtain: b ˆ = n ∑ x i y i −
∑ x i \sum y i } { n ∑ x i 2 − ( ∑ x i ) 2 }
− 1
8*10695 − 56*1500
= 8.125
And therefore: b ˆ =
8*416 − 56 2
%%%%%%%%%%%%%%%%%%%%%%%%%%%%%%%%%%%%5 – April 2012 – %%%%%%%%%%%%%%%%%%%%%%%%%%%%%%%%
And a ˆ =
1
8
( \sum y i − b ˆ ∑ x i ) = 130.625
(iv) R 2 = 0.796084 2 = 0.634
(v) Any increase in school quality by 1 index-point, leads to an increase of 8.125
in the house price index.
Mostly very well answered.
END OF %%%%%%%%%%%%%%%%%%%%%%%%%%%%%%%%
Page 10

%%%%%%%%%%%%%%%%%%%%%%%%%
12
An insurer has collected data about the body mass index of 200 males between the
age of 18 and 40. The results are shown in the following table.
Body mass index
Observed frequency
< 18.5
6
18.5–25
114
25–30
62
>30
18
A statistician suggests the following model for the distribution of the body mass index
with an unknown parameter p.
Body mass index
Relative frequency
< 18.5
p
18.5–25
20p
25−30
10p
>30
1−31p
(i) Estimate the parameter p using the method of maximum likelihood.

(ii) Perform a statistical test to decide whether the suggested distribution is appropriate for the observed data. You should state the null hypothesis for the test and your decision.

To improve the description of the distribution of the body mass index, it is suggested that the marital status of the males in this study is also recorded. The results are
shown in the following table.
Marital Status
Single
Married
Total
< 18.5
5
1
6
Body mass index
18.5–25
25–30
98
43
16
19
114
62
Total
>30
12
6
18
158
42
200
A life office has considered a sample of 10,000 men aged between 18 and 40 of which 50\% are married and the other 50\% are single.
(iii) Estimate the proportion of men with a body mass index of more than 30 in this sample, based on the data in the above table.

(iv) Determine whether the body mass index is independent of the marital status or not, using an appropriate statistical test. You should state the null hypothesis for the test, calculate the value of the test statistic and the approximate
p-value and state your decision.
[8]
%%%%%%%%%%%%%%%%%%%%%%%%%%%%%%%%%%%
13
The following data give the weight, in kilograms, of a random sample of 10 different
models of similar motorcycles and the distance, in metres, required to stop from a
speed of 20 miles per hour.
Weight x
Distance y
314
13.9
For these data:
317
14.0
320
13.9
326
14.1
331
14.0
339
14.3
346
14.1
354
14.5
361
14.5
369
14.4
∑ x = 3,377 , ∑ x 2 = 1,143,757 , \sum y = 141.7 ,
\sum y 2 = 2, 008.39 , ∑ xy = 47,888.6
Also: S xx = 3,344.1, S yy = 0.501, S_{xy} = 36.51
A scatter plot of the data is shown below.
Stopping distance against motorcycle weight
320
330
340
350
360
370
Weight (kilograms)
(i)
(a) Comment briefly on the association between weight and stopping distance, based on the scatter plot.
(b) Calculate the correlation coefficient between the two variables.

(ii) Investigate the hypothesis that there is positive correlation between the weight of the motorcycle and the stopping distance, using Fisher’s transformation of the correlation coefficient. You should state clearly the hypotheses of your
test and any assumption that you need to make for the test to be valid.

(iii) (a)
Fit a linear regression model to these data with stopping distance being the response variable and weight the explanatory variable.
(b)
Calculate the coefficient of determination for this model and give its interpretation.
CT3 S2012–8(c)
Calculate the expected change in stopping distance for every additional 10 kilograms of motorcycle weight according to the model fitted in
part (iii)(a).

\end{document}
