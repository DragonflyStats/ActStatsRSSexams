\documentclass[a4paper,12pt]{article}

%%%%%%%%%%%%%%%%%%%%%%%%%%%%%%%%%%%%%%%%%%%%%%%%%%%%%%%%%%%%%%%%%%%%%%%%%%%%%%%%%%%%%%%%%%%%%%%%%%%%%%%%%%%%%%%%%%%%%%%%%%%%%%%%%%%%%%%%%%%%%%%%%%%%%%%%%%%%%%%%%%%%%%%%%%%%%%%%%%%%%%%%%%%%%%%%%%%%%%%%%%%%%%%%%%%%%%%%%%%%%%%%%%%%%%%%%%%%%%%%%%%%%%%%%%%%

\usepackage{eurosym}
\usepackage{vmargin}
\usepackage{amsmath}
\usepackage{graphics}
\usepackage{epsfig}
\usepackage{enumerate}
\usepackage{multicol}
\usepackage{subfigure}
\usepackage{fancyhdr}
\usepackage{listings}
\usepackage{framed}
\usepackage{graphicx}
\usepackage{amsmath}
\usepackage{chngpage}

%\usepackage{bigints}
\usepackage{vmargin}

% left top textwidth textheight headheight

% headsep footheight footskip

\setmargins{2.0cm}{2.5cm}{16 cm}{22cm}{0.5cm}{0cm}{1cm}{1cm}

\renewcommand{\baselinestretch}{1.3}

\setcounter{MaxMatrixCols}{10}

\begin{document}

A coin has two sides, “heads” and “tails”. Such a coin with P(heads) = p is tossed repeatedly until it lands “heads” for the first time. Let X be the number of tosses required.
Suppose the process is repeated independently a total of n times, producing values of
the variables $X_1 , X_2 , \ldots , X_n$ , where each X i has the same distribution as X.
Let Y = min($X_1 , X_2 , \ldots , X_n$ ), so Y is the smallest number of tosses required to produce
a “heads” in the n repetitions of the experiment.
\begin{enumerate}[(a)]
\item (i)
Explain why, for each i = 1, 2, ..., n, P(X i \geq x) is given by
P(X i \geq x) = (1 – p) x−1 , x = 1, 2, ... .
(ii)

(a) Find an expression for P(Y \geq y).
(b) Hence deduce the probability function of Y.

\end{enumerate}

Page 4%%%%%%%%%%%%%%%%%%%%%%%%%%%%%%%%%%%%5 – April 2012 – %%%%%%%%%%%%%%%%%%%%%%%%%%%%%%%%
7
(i)
“X i \geq x” ≡ “no heads in first x − 1 tosses” so P(X i \geq x) = (1 − p) x−1 , x = 1,2,3, ...
[OR Recognise (as geometric) and sum the probabilities
(1 − p) x−1 p + (1 − p) x p + (1 − p) x+1 p + ... = p(1 − p) x−1 {1 – (1 − p)} −1 ]
(ii)
(a)
“Y \geq y” ≡ “all X i ’s are \geq y”
so P(Y \geq y) = P(X 1 \geq y, ..., X n \geq y) = P(X 1 \geq y) ... P(X n \geq y)
(independent)
= ((1 − p) y−1 ) n = ((1 − p) n ) y−1
(b)
The probability in part (a) implies that Y has the same distribution as X,
but with 1 – (1 − p) n in place of p
i.e. P(Y = y) = r(1 − r) y−1 , y = 1, 2, 3, ... where r = 1 – (1 − p) n .
[OR P(Y = y) = P(Y \geq y) – P(Y \geq y + 1) = ((1 − p) n ) y−1 − ((1 − p) n ) y
= (1 − p) n(y−1) {1 − (1 − p) n } as above]

% Most candidates had problems with part (ii). Carefully expressed probability statements are required here. A common error was to try to differentiate the CDF, despite this being a
% discrete distribution.
\end{document}
