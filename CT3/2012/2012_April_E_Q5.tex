\documentclass[a4paper,12pt]{article}

%%%%%%%%%%%%%%%%%%%%%%%%%%%%%%%%%%%%%%%%%%%%%%%%%%%%%%%%%%%%%%%%%%%%%%%%%%%%%%%%%%%%%%%%%%%%%%%%%%%%%%%%%%%%%%%%%%%%%%%%%%%%%%%%%%%%%%%%%%%%%%%%%%%%%%%%%%%%%%%%%%%%%%%%%%%%%%%%%%%%%%%%%%%%%%%%%%%%%%%%%%%%%%%%%%%%%%%%%%%%%%%%%%%%%%%%%%%%%%%%%%%%%%%%%%%%

\usepackage{eurosym}
\usepackage{vmargin}
\usepackage{amsmath}
\usepackage{graphics}
\usepackage{epsfig}
\usepackage{enumerate}
\usepackage{multicol}
\usepackage{subfigure}
\usepackage{fancyhdr}
\usepackage{listings}
\usepackage{framed}
\usepackage{graphicx}
\usepackage{amsmath}
\usepackage{chngpage}

%\usepackage{bigints}
\usepackage{vmargin}

% left top textwidth textheight headheight

% headsep footheight footskip

\setmargins{2.0cm}{2.5cm}{16 cm}{22cm}{0.5cm}{0cm}{1cm}{1cm}

\renewcommand{\baselinestretch}{1.3}

\setcounter{MaxMatrixCols}{10}

\begin{document}
\begin{enumerate}
\item 

%%--- Question 5
A large portfolio consists of 20\% class A policies, 50\% class B policies and 30\% class
C policies. Ten policies are selected at random from the portfolio.
(i)
(ii)
%% CT3 S2012–2
Calculate the probability that there are no policies of class A among the
randomly selected ten.

(a) Calculate the expected number of class B policies among the randomly
selected ten.
(b) Calculate the probability that there are more than five class B policies
among the randomly selected ten.

[Total 3]6
%%%%%%%%%%%%%%%%%%%%%%%%%%%%%%%%%%%%%%%%%%%%%%%%%
%%%%%%%%%%%%%%%%%%%%%%%%%%%%%%%%%%%%%%%%%%%%%%%%%%%%%%%%%%%%%%%%%%%%%%%%%%%%%%%%%%%%%%%%%%%%%%%%
5
(i)
number of claims in a month X ~ Poisson(2)
from tables: P(X=0) = 0.1353
[alternative: P(X=0) = e −2 ]
(ii)
number of claims in a year X ~ Poisson(24)
from tables: P ( X > 30) = 1 − P ( X \leq 30) = 1 − 0.9042 = 0.0958
[alternative: use normal approximation with continuity correction which
gives 0.0923]
Well answered by majority of candidates.
%%%%%%%%%%%%%%%%%%
\newpage

\end{document}
