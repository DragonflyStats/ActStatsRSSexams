\documentclass[a4paper,12pt]{article}

%%%%%%%%%%%%%%%%%%%%%%%%%%%%%%%%%%%%%%%%%%%%%%%%%%%%%%%%%%%%%%%%%%%%%%%%%%%%%%%%%%%%%%%%%%%%%%%%%%%%%%%%%%%%%%%%%%%%%%%%%%%%%%%%%%%%%%%%%%%%%%%%%%%%%%%%%%%%%%%%%%%%%%%%%%%%%%%%%%%%%%%%%%%%%%%%%%%%%%%%%%%%%%%%%%%%%%%%%%%%%%%%%%%%%%%%%%%%%%%%%%%%%%%%%%%%

\usepackage{eurosym}
\usepackage{vmargin}
\usepackage{amsmath}
\usepackage{graphics}
\usepackage{epsfig}
\usepackage{enumerate}
\usepackage{multicol}
\usepackage{subfigure}
\usepackage{fancyhdr}
\usepackage{listings}
\usepackage{framed}
\usepackage{graphicx}
\usepackage{amsmath}
\usepackage{chngpage}

%\usepackage{bigints}
\usepackage{vmargin}

% left top textwidth textheight headheight

% headsep footheight footskip

\setmargins{2.0cm}{2.5cm}{16 cm}{22cm}{0.5cm}{0cm}{1cm}{1cm}

\renewcommand{\baselinestretch}{1.3}

\setcounter{MaxMatrixCols}{10}

\begin{document}




\begin{enumerate}
% Question 10
\item In a portfolio of car insurance policies, the number of accident-related claims, N, made by a policyholder in a year has the following distribution:
No. of claims, n
Probability
0
1
2
0.4 0.4 0.2
The number of cars, X, involved in each accident that results in a claim is distributed
as follows:
\begin{verbatim}
No. of cars, x
Probability
1
2
0.7 0.3
\end{verbatim}

\item It can be assumed that the occurrence of a claim and the number of cars involved in the accident are independent. Furthermore, claims made by a policyholder in any year are also independent of each other. Let S be the total number of cars involved in
accidents related to such claims by a policyholder in a year.
(i)
(a)
(b)
Determine the probability function of S.
Hence find $E(S)$.
\newpage
%%%%%%%%%%%%%%%%%%%%%%%%%%%%%%%%%%%%%%%%%%%%%%%%%%%%%%%%%%%%%%%%%%%%%%%%
The expectation $E(S)$ can also be calculated using the formula
2
E ( S ) = \sum E ( S | N = n ) Pr ( N = n ) .
n = 0
(ii)
(a)
(b)
Find E ( S | N = n ) for n = 0,1 , 2 .
Hence calculate $E(S)$.

%%%%%%%%%%%%%%%%%%%%%%%%%%%%%%%%%%%%%%%%%%%%%

\newpage
%%%%%%%%%%%%%%%%%%%%%%%%%%%10
(i)
(a)
S takes values 0, 1, 2, 3, 4 and we have
\begin{itemize}
\item $ P ( S = 0 ) = 0.4 $
\item $ P ( S = 1 ) = 0.4 \times 0.7 = 0.28 $
\item $ P ( S = 2 ) = 0.4 \times 0.3 + 0.2 \times 0.7 2 = 0.218$
\item $ P ( S = 3 ) = 0.2 \times 2 \times 0.7 \times 0.3 = 0.084$
\item $ P ( S = 4 ) = 0.2 \times 0.3 2 = 0.018$
\end{itemize}
(b)
Hence
E ( S ) = 0.28 + 2 \times 0.218 + 3 \times 0.084 + 4 \times 0.018 = 1.04
(ii)
(a)
E ( S | N = 0 ) = 0,
E ( S | N = 1 ) = E ( X ) = 0.7 + 2 \times 0.3 = 1.3
E ( S | N = 2 ) = E ( 2 X ) = 2 \times 1.3 = 2.6
Page 6%%%%%%%%%%%%%%%%%%%%%%%%%%%%%%%%%%%%5 – April 2012 – %%%%%%%%%%%%%%%%%%%%%%%%%%%%%%%%
(b)
\begin{itemize}
\item Hence, E ( S ) = 1.3 \times 0.4 + 2.6 \times 0.2 = 1.04 as before.
\item Most candidates encountered problems here, as they failed to work out the probability
function from first principles in part (i). Also, many did not recognise this as a
compound distribution type of question.
\end{itemize}

\newpage









\end{document}
