 88 lines (72 sloc) 2.45 KB
\documentclass[a4paper,12pt]{article}

%%%%%%%%%%%%%%%%%%%%%%%%%%%%%%%%%%%%%%%%%%%%%%%%%%%%%%%%%%%%%%%%%%%%%%%%%%%%%%%%%%%%%%%%%%%%%%%%%%%%%%%%%%%%%%%%%%%%%%%%%%%%%%%%%%%%%%%%%%%%%%%%%%%%%%%%%%%%%%%%%%%%%%%%%%%%%%%%%%%%%%%%%%%%%%%%%%%%%%%%%%%%%%%%%%%%%%%%%%%%%%%%%%%%%%%%%%%%%%%%%%%%%%%%%%%%

\usepackage{eurosym}
\usepackage{vmargin}
\usepackage{amsmath}
\usepackage{graphics}
\usepackage{epsfig}
\usepackage{enumerate}
\usepackage{multicol}
\usepackage{subfigure}
\usepackage{fancyhdr}
\usepackage{listings}
\usepackage{framed}
\usepackage{graphicx}
\usepackage{amsmath}
\usepackage{chngpage}

%\usepackage{bigints}
\usepackage{vmargin}

% left top textwidth textheight headheight

% headsep footheight footskip

\setmargins{2.0cm}{2.5cm}{16 cm}{22cm}{0.5cm}{0cm}{1cm}{1cm}

\renewcommand{\baselinestretch}{1.3}

\setcounter{MaxMatrixCols}{10}

\begin{document}
\begin{enumerate}
7
A random sample of size n is taken from a gamma distribution with parameters α = 8
and λ = 1/\theta. The sample mean is X and \theta is to be estimated.
(i) Determine the method of moments estimator (MME) of \theta. 
(ii) Find the bias of the MME determined in part (i). 
(iii) (a) Determine the mean square error of the MME of \theta. (b) Comment on the efficiency of the MME of \theta based on your answer in
part (iii)(a).

[Total 7]
Analyst A collects a random sample of 30 claims from a large insurance portfolio and
calculates a 95\% confidence interval for the mean of the claim sizes in this portfolio.
She then collects a different sample of 100 claims from the same portfolio and
calculates a new 95\% confidence interval for the mean claim size.
(i)
Explain how the widths of the two confidence intervals will differ.

Analyst B obtains a 95\% confidence interval for the mean claim size of this portfolio
based on a different sample of 30 claims. She subsequently realises that one of the
claims in the sample has an extremely large value and can be considered as an outlier.
She decides to replace this claim with a new randomly selected one, whose size is not
an outlier, and obtains a new 95\% confidence interval.
(ii)
8
Explain how the two confidence intervals will differ in the case of Analyst B.

[Total 5]
The random variable S is given as S = Y 1 + Y 2 + ...+ Y N (with S = 0 if N = 0) where
the random variables Y i are identically and independently distributed according to a
lognormal distribution with parameters \mu = 0.5 and \sigma^2 = 0.1. N is also a random
variable which is independent of Y i , and its distribution given below.
N
Pr(N = n)
0
0.1
1
0.3
2
0.3
3
0.2
4
0.1
Calculate the mean and the variance of the random variable S.
CT3 S2012–3
%%%%%%%%%%%%%%%5
6
(i)
Population mean = 8\theta
So MME is solution of X = 8 \theta ⇒
(ii)
⎛ X
E ⎜
⎝ 8
MME =
X
8
⎞ 1
1
⎟ = E ( X ) = (8 \theta ) = \theta
8
⎠ 8
⎛ X ⎞
Bias = E ⎜ ⎟ − \theta = 0 (i.e. MME is unbiased for \theta).
⎝ 8 ⎠
(iii)
௑ത ௑ത ଼ఏ మ
଼ ଼ ଺ସ௡
ൌ
ఏ మ
(a) Since MME is unbiased, ‫ܧܵܯ‬ ቀ ቁ ൌ ‫ݎܽݒ‬ ቀ ቁ ൌ
(b) MME gets more efficient (MSE gets smaller) as sample size increases.
଼௡
There was a mix of quality in the answers, especially in parts (ii) and (iii). Attention to detail
is required when determining the expected value and variance of functions of sample
statistics (here the sample mean).
7
(i) With the larger sample of 100 claims the standard error of the sample mean
will be smaller, giving a narrower confidence interval.
(ii) The replacement of the extreme value will give a smaller sample mean, which
means that the interval will be shifted to the left.
The variance of the sample will also be smaller, which will again give a
narrower interval.
Many candidates recognised the correct effect on the interval, without being able to justify it
properly. Note that reasonably accurate wording is important in providing the comments and
justification required here.
