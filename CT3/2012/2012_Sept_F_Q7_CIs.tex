\documentclass[a4paper,12pt]{article}

%%%%%%%%%%%%%%%%%%%%%%%%%%%%%%%%%%%%%%%%%%%%%%%%%%%%%%%%%%%%%%%%%%%%%%%%%%%%%%%%%%%%%%%%%%%%%%%%%%%%%%%%%%%%%%%%%%%%%%%%%%%%%%%%%%%%%%%%%%%%%%%%%%%%%%%%%%%%%%%%%%%%%%%%%%%%%%%%%%%%%%%%%%%%%%%%%%%%%%%%%%%%%%%%%%%%%%%%%%%%%%%%%%%%%%%%%%%%%%%%%%%%%%%%%%%%

\usepackage{eurosym}
\usepackage{vmargin}
\usepackage{amsmath}
\usepackage{graphics}
\usepackage{epsfig}
\usepackage{enumerate}
\usepackage{multicol}
\usepackage{subfigure}
\usepackage{fancyhdr}
\usepackage{listings}
\usepackage{framed}
\usepackage{graphicx}
\usepackage{amsmath}
\usepackage{chngpage}

%\usepackage{bigints}
\usepackage{vmargin}

% left top textwidth textheight headheight

% headsep footheight footskip

\setmargins{2.0cm}{2.5cm}{16 cm}{22cm}{0.5cm}{0cm}{1cm}{1cm}

\renewcommand{\baselinestretch}{1.3}

\setcounter{MaxMatrixCols}{10}

\begin{document}

%%- Question 7


Analyst A collects a random sample of 30 claims from a large insurance portfolio and
calculates a 95\% confidence interval for the mean of the claim sizes in this portfolio.
She then collects a different sample of 100 claims from the same portfolio and
calculates a new 95\% confidence interval for the mean claim size.
\begin{enumerate}[(a)]
\item (i)
Explain how the widths of the two confidence intervals will differ.

Analyst B obtains a 95\% confidence interval for the mean claim size of this portfolio based on a different sample of 30 claims. She subsequently realises that one of the
claims in the sample has an extremely large value and can be considered as an outlier.
She decides to replace this claim with a new randomly selected one, whose size is not
an outlier, and obtains a new 95\% confidence interval.
\item (ii)
Explain how the two confidence intervals will differ in the case of Analyst B.
\end{enumerate}

\newpage


%%%%%%%%%%%%%%%%%%%%%%%%%%%%%%%%%%%%%%%%%%%%%%%%%%%%%%%%%%%%%
7
\begin{itemize}
\item (i) With the larger sample of 100 claims the standard error of the sample mean
will be smaller, giving a narrower confidence interval.
\item (ii) The replacement of the extreme value will give a smaller sample mean, which
means that the interval will be shifted to the left.
\item The variance of the sample will also be smaller, which will again give a
narrower interval.
\item Many candidates recognised the correct effect on the interval, without being able to justify it
properly. Note that reasonably accurate wording is important in providing the comments and
justification required here.
\end{enumerate}
\end{document}
