 88 lines (72 sloc) 2.45 KB
\documentclass[a4paper,12pt]{article}

%%%%%%%%%%%%%%%%%%%%%%%%%%%%%%%%%%%%%%%%%%%%%%%%%%%%%%%%%%%%%%%%%%%%%%%%%%%%%%%%%%%%%%%%%%%%%%%%%%%%%%%%%%%%%%%%%%%%%%%%%%%%%%%%%%%%%%%%%%%%%%%%%%%%%%%%%%%%%%%%%%%%%%%%%%%%%%%%%%%%%%%%%%%%%%%%%%%%%%%%%%%%%%%%%%%%%%%%%%%%%%%%%%%%%%%%%%%%%%%%%%%%%%%%%%%%

\usepackage{eurosym}
\usepackage{vmargin}
\usepackage{amsmath}
\usepackage{graphics}
\usepackage{epsfig}
\usepackage{enumerate}
\usepackage{multicol}
\usepackage{subfigure}
\usepackage{fancyhdr}
\usepackage{listings}
\usepackage{framed}
\usepackage{graphicx}
\usepackage{amsmath}
\usepackage{chngpage}

%\usepackage{bigints}
\usepackage{vmargin}

% left top textwidth textheight headheight

% headsep footheight footskip

\setmargins{2.0cm}{2.5cm}{16 cm}{22cm}{0.5cm}{0cm}{1cm}{1cm}

\renewcommand{\baselinestretch}{1.3}

\setcounter{MaxMatrixCols}{10}

\begin{document}
\begin{enumerate}

5
A large portfolio consists of 20% class A policies, 50% class B policies and 30% class
C policies. Ten policies are selected at random from the portfolio.
(i)
(ii)
CT3 S2012–2
Calculate the probability that there are no policies of class A among the
randomly selected ten.
[1]
(a) Calculate the expected number of class B policies among the randomly
selected ten.
(b) Calculate the probability that there are more than five class B policies
among the randomly selected ten.

[Total 3]6
7
A random sample of size n is taken from a gamma distribution with parameters α = 8
and λ = 1/\theta. The sample mean is X and \theta is to be estimated.
(i) Determine the method of moments estimator (MME) of \theta. 
(ii) Find the bias of the MME determined in part (i). 
(iii) (a) Determine the mean square error of the MME of \theta. (b) Comment on the efficiency of the MME of \theta based on your answer in
part (iii)(a).

[Total 7]
Analyst A collects a random sample of 30 claims from a large insurance portfolio and
calculates a 95\% confidence interval for the mean of the claim sizes in this portfolio.
She then collects a different sample of 100 claims from the same portfolio and
calculates a new 95\% confidence interval for the mean claim size.
(i)
Explain how the widths of the two confidence intervals will differ.

Analyst B obtains a 95\% confidence interval for the mean claim size of this portfolio
based on a different sample of 30 claims. She subsequently realises that one of the
claims in the sample has an extremely large value and can be considered as an outlier.
She decides to replace this claim with a new randomly selected one, whose size is not
an outlier, and obtains a new 95\% confidence interval.
(ii)

%%%%%%%%%%%%%%%%%%%%%%%%%%%%%%%%%%%%%%
5
6
7
Claims on a group of policies arise randomly and independently of each other through
time at an average rate of 2 per month.
(i) Calculate the probability that no claims arise in a particular month.

(ii) Calculate the probability that more than 30 claims arise in a period of one
year.

[Total 4]
In a random sample of 200 people taken from a large population of adults, 70 people
intend to vote for party A at the next election.
(i) Calculate an approximate equal-tailed 95\% confidence interval for \theta, the true
proportion of this population who intend to vote for party A at the next
election.

(ii) Give a brief interpretation of the interval calculated in part (i).
[1]
[Total 4]
A coin has two sides, “heads” and “tails”. Such a coin with P(heads) = p is tossed
repeatedly until it lands “heads” for the first time. Let X be the number of tosses
required.
Suppose the process is repeated independently a total of n times, producing values of
the variables $X_1 , X_2 , \ldots , X_n$ , where each X i has the same distribution as X.
Let Y = min($X_1 , X_2 , \ldots , X_n$ ), so Y is the smallest number of tosses required to produce
a “heads” in the n repetitions of the experiment.
(i)
Explain why, for each i = 1, 2, ..., n, P(X i \geq x) is given by
P(X i \geq x) = (1 – p) x−1 , x = 1, 2, ... .
(ii)

(a) Find an expression for P(Y \geq y).
(b) Hence deduce the probability function of Y.
[5]
[Total 7]

%%%%%%%%%%%%%%%%%%%%%%%%%%%%%%%%%%6
(i)
⎛ \theta (1 − \theta ) ⎞
Use the normal approximation \theta ˆ ~ N ⎜ \theta ,
⎟
200 ⎠
⎝
\theta ˆ (1 − \theta ˆ )
to give the 95\% confidence interval \theta ˆ \pm 1.96
.
200
With \theta ˆ = 70 / 200 = 0.35 we obtain 0.35 \pm 0.066 , that is (0.284, 0.416).
(ii)
If we take a large number of samples from this population, we expect 95\% of
the resulting CIs to include the true value of \theta.
There were no problems with the first part. However, many candidates struggled with
providing a reasonable interpretation in part (ii).
Page 4%%%%%%%%%%%%%%%%%%%%%%%%%%%%%%%%%%%%5 – April 2012 – %%%%%%%%%%%%%%%%%%%%%%%%%%%%%%%%
7
(i)
“X i \geq x” ≡ “no heads in first x − 1 tosses” so P(X i \geq x) = (1 − p) x−1 , x = 1,2,3, ...
[OR Recognise (as geometric) and sum the probabilities
(1 − p) x−1 p + (1 − p) x p + (1 − p) x+1 p + ... = p(1 − p) x−1 {1 – (1 − p)} −1 ]
(ii)
(a)
“Y \geq y” ≡ “all X i ’s are \geq y”
so P(Y \geq y) = P(X 1 \geq y, ..., X n \geq y) = P(X 1 \geq y) ... P(X n \geq y)
(independent)
= ((1 − p) y−1 ) n = ((1 − p) n ) y−1
(b)
The probability in part (a) implies that Y has the same distribution as X,
but with 1 – (1 − p) n in place of p
i.e. P(Y = y) = r(1 − r) y−1 , y = 1, 2, 3, ... where r = 1 – (1 − p) n .
[OR P(Y = y) = P(Y \geq y) – P(Y \geq y + 1) = ((1 − p) n ) y−1 − ((1 − p) n ) y
= (1 − p) n(y−1) {1 − (1 − p) n } as above]
Most candidates had problems with part (ii). Carefully expressed probability statements are
required here. A common error was to try to differentiate the CDF, despite this being a
discrete distribution.
\end{document}
