\documentclass[a4paper,12pt]{article}

%%%%%%%%%%%%%%%%%%%%%%%%%%%%%%%%%%%%%%%%%%%%%%%%%%%%%%%%%%%%%%%%%%%%%%%%%%%%%%%%%%%%%%%%%%%%%%%%%%%%%%%%%%%%%%%%%%%%%%%%%%%%%%%%%%%%%%%%%%%%%%%%%%%%%%%%%%%%%%%%%%%%%%%%%%%%%%%%%%%%%%%%%%%%%%%%%%%%%%%%%%%%%%%%%%%%%%%%%%%%%%%%%%%%%%%%%%%%%%%%%%%%%%%%%%%%

\usepackage{eurosym}
\usepackage{vmargin}
\usepackage{amsmath}
\usepackage{graphics}
\usepackage{epsfig}
\usepackage{enumerate}
\usepackage{multicol}
\usepackage{subfigure}
\usepackage{fancyhdr}
\usepackage{listings}
\usepackage{framed}
\usepackage{graphicx}
\usepackage{amsmath}
\usepackage{chngpage}

%\usepackage{bigints}
\usepackage{vmargin}

% left top textwidth textheight headheight

% headsep footheight footskip

\setmargins{2.0cm}{2.5cm}{16 cm}{22cm}{0.5cm}{0cm}{1cm}{1cm}

\renewcommand{\baselinestretch}{1.3}

\setcounter{MaxMatrixCols}{10}

\begin{document}
\begin{enumerate}



%%%%%%%%%%%%%%%%%%%%%%%%%%%%%%%%%%%%%%%%%%%%%5
\item % Queston 9
A random sample of 200 email messages was selected from all messages delivered through an internet provider company. Each message is monitored for the presence of
computer viruses. It is assumed that each message contains a virus with the same
probability p, independently from all other messages.
Let Y i , i = 1, ... , 200 be indicator random variables taking the value 1 if message i
contains a virus, and 0 otherwise. Also, let Y denote the total number of messages in
200
this sample found to contain viruses, i.e. $Y = \sum y i$ .
i = 1
\begin{enumerate}[(i)]
    \item (i) Derive expressions for the expected value and the variance of $Y$ in terms of the parameter $p$, using the indicator variables $Y_1 , Y_2 , ... , Y_200$ .
\item (ii) Explain why the approximate distribution of Y is N(200p, 200p(1−p)), using
the indicator variables Y 1 , Y 2 , ... , Y 200 .

It is found that 38 email messages in this sample contained viruses.
\item (iii)
%CT3 A2012–4
Calculate an equal-tailed 90% confidence interval for the probability p using
the approximate normal distribution from part (ii).
\end{enumerate}


\end{enumerate}

\newpage
%%-- Question 9
(i)
\begin{itemize}
    \item The variables Y i are independent between them and have a Bernoulli(p)
distribution with mean p and variance p(1– p).
Therefore E ( Y ) = E ( Y 1 + " + Y 200 ) = E ( Y 1 ) + " + E ( Y 200 ) = 200p
V ( Y ) = V ( Y 1 + " + Y 200 ) = V ( Y 1 ) + " + V ( Y 200 ) = 200p(1– p)
(ii)
\item Again, with Y i being iid Bernoulli(p) random variables and n being sufficiently
large,
200
the central limit theorem implies that Y = $\sum y i$ follows approximately a
i = 1
normal distribution with mean and variance given by the mean and variance of
Y as derived in (i), i.e.
Y ~ N(200p, 200p(1−p))
\end{itemize}

(iii)
From (ii) P ˆ = Y / 200 ~ N ( p , p (1 − p ) / 200) approximately, which gives a 90%
confidence interval of the form
p ˆ = 0.19 giving
%p ˆ \pm z 0.05
%p ˆ (1 − p ˆ )
200
$0.19 \pm 1.6445 \times 0.02774$
i.e. (0.144, 0.236).
Generally well tackled. Some students failed to work with the indicator variables (Bernoulli),
which was key to this question.
%%%%%%%%%%%%
