\documentclass[a4paper,12pt]{article}

%%%%%%%%%%%%%%%%%%%%%%%%%%%%%%%%%%%%%%%%%%%%%%%%%%%%%%%%%%%%%%%%%%%%%%%%%%%%%%%%%%%%%%%%%%%%%%%%%%%%%%%%%%%%%%%%%%%%%%%%%%%%%%%%%%%%%%%%%%%%%%%%%%%%%%%%%%%%%%%%%%%%%%%%%%%%%%%%%%%%%%%%%%%%%%%%%%%%%%%%%%%%%%%%%%%%%%%%%%%%%%%%%%%%%%%%%%%%%%%%%%%%%%%%%%%%

\usepackage{eurosym}
\usepackage{vmargin}
\usepackage{amsmath}
\usepackage{graphics}
\usepackage{epsfig}
\usepackage{enumerate}
\usepackage{multicol}
\usepackage{subfigure}
\usepackage{fancyhdr}
\usepackage{listings}
\usepackage{framed}
\usepackage{graphicx}
\usepackage{amsmath}
\usepackage{chngpage}

%\usepackage{bigints}
\usepackage{vmargin}

% left top textwidth textheight headheight

% headsep footheight footskip

\setmargins{2.0cm}{2.5cm}{16 cm}{22cm}{0.5cm}{0cm}{1cm}{1cm}

\renewcommand{\baselinestretch}{1.3}

\setcounter{MaxMatrixCols}{10}

\begin{document}
\begin{enumerate}

%%PLEASE TURN OVER8
\item In an analysis of variance investigation four treatments are compared using random
samples each of size 10. The total sum of squares is calculated as SS T = 673.5 and the
between-treatments sum of squares as SS B = 148.3.
\begin{enumerate}[(i)]
\item (i)
(a) Calculate an unbiased estimate of the error variance $\sigma^2$ .
(b) State the number of degrees of freedom associated with the estimate in
part (i)(a).

\item (ii) Suggest an unbiased estimator of $\sigma^2$ that is different from the one used in part
(i).
[1]
\item (iii) Comment on which of the two estimators should be used.
\end{enumerate}

%%%%%%%%%%%%%%%%%%%%%%%%%%%%%%%%%%%%%%%%%%%%%5
\item % Queston 9
A random sample of 200 email messages was selected from all messages delivered through an internet provider company. Each message is monitored for the presence of
computer viruses. It is assumed that each message contains a virus with the same
probability p, independently from all other messages.
Let Y i , i = 1, ... , 200 be indicator random variables taking the value 1 if message i
contains a virus, and 0 otherwise. Also, let Y denote the total number of messages in
200
this sample found to contain viruses, i.e. $Y = \sum y i$ .
i = 1
\begin{enumerate}[(i)]
    \item (i) Derive expressions for the expected value and the variance of Y in terms of the

parameter p, using the indicator variables $Y_1 , Y_2 , ... , Y_200$ .
\item (ii) Explain why the approximate distribution of Y is N(200p, 200p(1−p)), using
the indicator variables Y 1 , Y 2 , ... , Y 200 .

It is found that 38 email messages in this sample contained viruses.
\item (iii)
%CT3 A2012–4
Calculate an equal-tailed 90% confidence interval for the probability p using
the approximate normal distribution from part (ii).
\end{enumerate}


\end{enumerate}
%%%%%%%%%%%%%%%%%%%%%%%%%%%%%%%%%%%8
(i)
(a) SS R = SS T − SS B = 673.5 − 148.3 = 525.2
σ ˆ 2 =
(b)
SS R 525.2
=
= 14.59
36
n − k
Associated d.f. 36
(ii) Alternatively, an unbiased estimator could be given using only part of the data,
( Y − Y i . ) 2
2 ∑ j ij
e.g. responses from treatment i: S i =
n i − 1
(iii) The estimator used in part (i) should be preferred as it is based on all data and
is therefore more accurate.

%%Part (i) was well answered, although the df were wrongly given in many answers. Answers in part (ii) were very poor – this question required good understanding of ANOVA concepts and critical thinking.
Page 5%%%%%%%%%%%%%%%%%%%%%%%%%%%%%%%%%%%%5 – April 2012 – %%%%%%%%%%%%%%%%%%%%%%%%%%%%%%%%
\newpage
%%-- Question 9
(i)
\begin{itemize}
    \item The variables Y i are independent between them and have a Bernoulli(p)
distribution with mean p and variance p(1– p).
Therefore E ( Y ) = E ( Y 1 + " + Y 200 ) = E ( Y 1 ) + " + E ( Y 200 ) = 200p
V ( Y ) = V ( Y 1 + " + Y 200 ) = V ( Y 1 ) + " + V ( Y 200 ) = 200p(1– p)
(ii)
\item Again, with Y i being iid Bernoulli(p) random variables and n being sufficiently
large,
200
the central limit theorem implies that Y = $\sum y i$ follows approximately a
i = 1
normal distribution with mean and variance given by the mean and variance of
Y as derived in (i), i.e.
Y ~ N(200p, 200p(1−p))
\end{itemize}

(iii)
From (ii) P ˆ = Y / 200 ~ N ( p , p (1 − p ) / 200) approximately, which gives a 90%
confidence interval of the form
p ˆ = 0.19 giving
%p ˆ \pm z 0.05
%p ˆ (1 − p ˆ )
200
$0.19 \pm 1.6445 \times 0.02774$
i.e. (0.144, 0.236).
Generally well tackled. Some students failed to work with the indicator variables (Bernoulli),
which was key to this question.
%%%%%%%%%%%%
\newpage
%%-- Question 8
Explain how the two confidence intervals will differ in the case of Analyst B.

[Total 5]
The random variable S is given as S = Y 1 + Y 2 + ...+ Y N (with S = 0 if N = 0) where
the random variables Y i are identically and independently distributed according to a
lognormal distribution with parameters $\mu = 0.5$ and $\sigma^2 = 0.1.$ N is also a random
variable which is independent of Y i , and its distribution given below.
N
Pr(N = n)
0
0.1
1
0.3
2
0.3
3
0.2
4
0.1
Calculate the mean and the variance of the random variable S.
%%CT3 S2012–3
%%[7]
\newpage

%%%%%%%%%%%%%%%%%%%%%%%%%%%%%%%%%%%%%%%%%%%%%%%%%%%%%%%%%%%%%%%5
%%-- Question 9
An analyst is interested in using a gamma distribution with parameters α = 2 and
1
1 − x
λ = 1⁄2, that is, with density function f ( x ) = xe 2 , 0 < x < ∞ .
4
\begin{enumerate}
\tiem (i)
(a) State the mean and standard deviation of this distribution.
(b) Hence comment briefly on its shape.

\item (ii)
Show that the cumulative distribution function is given by
1
− x
1
F ( x ) = 1 − (1 + x ) e 2 , 0 < x < ∞
2
(zero otherwise).

The analyst wishes to simulate values x from this gamma distribution and is able to
generate random numbers u from a uniform distribution on (0,1).
\item (iii)
(a) Specify an equation involving x and u, the solution of which will yield
the simulated value x.
(b) Comment briefly on how this equation might be solved.
(c) The graph below gives F(x) plotted against x. Use this graph to obtain
the simulated value of x corresponding to the random number u = 0.66.
\end{enumerate}
\end{document}
