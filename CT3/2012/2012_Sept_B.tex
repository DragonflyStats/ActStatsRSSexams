\documentclass[a4paper,12pt]{article}

%%%%%%%%%%%%%%%%%%%%%%%%%%%%%%%%%%%%%%%%%%%%%%%%%%%%%%%%%%%%%%%%%%%%%%%%%%%%%%%%%%%%%%%%%%%%%%%%%%%%%%%%%%%%%%%%%%%%%%%%%%%%%%%%%%%%%%%%%%%%%%%%%%%%%%%%%%%%%%%%%%%%%%%%%%%%%%%%%%%%%%%%%%%%%%%%%%%%%%%%%%%%%%%%%%%%%%%%%%%%%%%%%%%%%%%%%%%%%%%%%%%%%%%%%%%%

\usepackage{eurosym}
\usepackage{vmargin}
\usepackage{amsmath}
\usepackage{graphics}
\usepackage{epsfig}
\usepackage{enumerate}
\usepackage{multicol}
\usepackage{subfigure}
\usepackage{fancyhdr}
\usepackage{listings}
\usepackage{framed}
\usepackage{graphicx}
\usepackage{amsmath}
\usepackage{chngpage}

%\usepackage{bigints}
\usepackage{vmargin}

% left top textwidth textheight headheight

% headsep footheight footskip

\setmargins{2.0cm}{2.5cm}{16 cm}{22cm}{0.5cm}{0cm}{1cm}{1cm}

\renewcommand{\baselinestretch}{1.3}

\setcounter{MaxMatrixCols}{10}

\begin{document}
\begin{enumerate}
%%-- 12
\item An insurer has collected data about the body mass index of 200 males between the
age of 18 and 40. The results are shown in the following table.
Body mass index
Observed frequency
< 18.5
6
18.5–25
114
25–30
62
>30
18
A statistician suggests the following model for the distribution of the body mass index
with an unknown parameter p.
Body mass index
Relative frequency
< 18.5
p
18.5–25
20p
25−30
10p
>30
1−31p
(i) Estimate the parameter p using the method of maximum likelihood.

(ii) Perform a statistical test to decide whether the suggested distribution is
appropriate for the observed data. You should state the null hypothesis for the
test and your decision.

To improve the description of the distribution of the body mass index, it is suggested
that the marital status of the males in this study is also recorded. The results are
shown in the following table.
Marital Status
Single
Married
Total
< 18.5
5
1
6
Body mass index
18.5–25
25–30
98
43
16
19
114
62
Total
>30
12
6
18
158
42
200
A life office has considered a sample of 10,000 men aged between 18 and 40 of which
50% are married and the other 50% are single.
(iii) Estimate the proportion of men with a body mass index of more than 30 in this
sample, based on the data in the above table.

(iv) Determine whether the body mass index is independent of the marital status or
not, using an appropriate statistical test. You should state the null hypothesis
for the test, calculate the value of the test statistic and the approximate
p-value and state your decision.
[8]
[Total 20]
CT3 S2012–7
\newpage
\item 
%%-- Question 13
The following data give the weight, in kilograms, of a random sample of 10 different
models of similar motorcycles and the distance, in metres, required to stop from a
speed of 20 miles per hour.
Weight x
Distance y
314
13.9
For these data:
317
14.0
320
13.9
326
14.1
331
14.0
339
14.3
346
14.1
354
14.5
361
14.5
369
14.4
∑ x = 3,377 , ∑ x 2 = 1,143,757 , \sum y = 141.7 ,
\sum y 2 = 2, 008.39 , ∑ xy = 47,888.6
Also: S xx = 3,344.1, S yy = 0.501, S_{xy} = 36.51
A scatter plot of the data is shown below.
Stopping distance against motorcycle weight
320
330
340
350
360
370
Weight (kilograms)
(i)
(a) Comment briefly on the association between weight and stopping
distance, based on the scatter plot.
(b) Calculate the correlation coefficient between the two variables.

(ii) Investigate the hypothesis that there is positive correlation between the weight of the motorcycle and the stopping distance, using Fisher’s transformation of
the correlation coefficient. You should state clearly the hypotheses of your test and any assumption that you need to make for the test to be valid.

(iii) (a)
Fit a linear regression model to these data with stopping distance being the response variable and weight the explanatory variable.
(b)
Calculate the coefficient of determination for this model and give its interpretation.
CT3 S2012–8(c)
Calculate the expected change in stopping distance for every additional 10 kilograms of motorcycle weight according to the model fitted in
part (iii)(a).
[5]
[Total 13]
\end{document}
%CT3 S2012–9
%%%%%%%%%%%%%%%%%%%%%%%%%%%%%%%%%%%%%%%%%%%%%%%%%%













12
(i)
Likelihood function
L ( p ) = p 6 ( 20 p )
114
( 10 p ) 62 (1 − 31 p ) 18 = Cp 182 (1 − 31 p ) 18
log L ( p ) = log C + 182log p + 18log(1 − 31 p )
\frac{\partial}{\partial}
182
18
+
log L ( p ) =
( − 31 ) = 0
\frac{\partial}{\partial} p
p 1 − 31 p
182
558
p 1 − 31 p
p 31 p
1
31 ⎞
⎛ 1
=
⇒
=
⇒
+
=
⇒ p ⎜
+
⎟
p 1 − 31 p
182
558
182 558 558
⎝ 182 558 ⎠
= 1/ 558
p ˆ = 0.02935
Page 8%%%%%%%%%%%%%%%%%%%%%%%%%%%%%%%%%%%%5 – September 2012 – %%%%%%%%%%%%%%%%%%%%%%%%%%%%%%%%
(ii)
H_{0} : The proposed distribution is the true distribution of the data with non-
specified parameter p (it is important to mention that the parameter itself is
not part of the null hypothesis)
Under H_{0} and using p ˆ = 0.02935 from (i)(a) we obtain the following
expected frequencies
Body-Mass-Index
Expected frequency
Test-statistic is
< 18.5
5.87
18.5–25
117.4
25–30
58.7
>30
18.03
0.286915
from a Chi-square distribution with 2 d.f.
The test statistic has a very small value, and there is no evidence against the
null.
(iii)
P [ BMI > 30 ]
= P [ BMI > 30| single ] P [ single ] + P [ BMI > 30| married ] P [ married ]
=
(iv)
12
6
*0.5 + *0.5 = 0.1094
158
42
H_{0} : Marital status is independent of BMI
Under H_{0} we have:
Marital Status
Single
Married
Total
Body-Mass-Index
Total
< 18.5 18.5–25 25–30 >30
4.74
90.06
48.98 14.22 158
1.26
23.94
13.02 3.78
42
6
114
62
18
200
Use χ 2 test.
2
4
Test-statistic: C = ∑∑
i = 1 j = 1
( f ij −
f i . * f . j
n
f i . * f . j
) 2
= 8.528399
n
C is χ 2 -distributed with (2 − 1 )(4 − 1 ) = 3 degrees of freedom.
p -value: P [ C > 8.528399 ] < 1 − 0.9616 = 0.0384
Page 9%%%%%%%%%%%%%%%%%%%%%%%%%%%%%%%%%%%%5 – September 2012 – %%%%%%%%%%%%%%%%%%%%%%%%%%%%%%%%
Therefore, we reject H_{0} at 5% level, but not at the 1% level.
There were errors in part (i) caused by failure to differentiate correctly. In part (iv)
alternative solutions involving merging of adjacent categories were given full redit where
correct. However note that merging the first and last column is not correct in this question.
13
(i)
(ii)
(a) The scatter plot suggests a positive linear association between weight
and stopping distance.
(b) r =
S_{xy}
S xx S yy
= 0.892
We want to test H_{0} : ρ = 0 against H 1 : ρ > 0 .
Need to assume that data come from a bivariate normal distribution.
Fisher’s (standardised) transformation statistic is given by
1
⎛ 1 + r ⎞
log ⎜
⎟
2
⎝ 1 − r ⎠ = 7 log ⎛ 1.892 ⎞ = 3.79
⎜
⎟
2
1 / ( n − 3)
⎝ 0.108 ⎠
and under H_{0} this should be a value from the N(0,1) distribution.
This gives P-value = Pr( Z \geq 3.79) ≈ 0.0001 , so there is very strong evidence
against H_{0} and we conclude that motorcycle weight and stopping distance are
positively correlated.
[Or by considering critical values of N(0,1) distribution.]
(iii)
(a)
β ˆ =
S_{xy}
S xx
=
36.51
= 0.01092
3344.1
α ˆ = y − β ˆ x = 14.17 − 0.01092*337.7 = 10.4823
Fitted line is
(b)
2
R =
2
S_{xy}
S xx S yy
y ˆ = 10.48 + 0.01092 x
=
36.51 2
= 0.7956
3344.1*0.501
This gives the proportion of total variation explained by the model.
(Note that R 2 can also be computed as r 2 .)
Page 10%%%%%%%%%%%%%%%%%%%%%%%%%%%%%%%%%%%%5 – September 2012 – %%%%%%%%%%%%%%%%%%%%%%%%%%%%%%%%
(c)
For every additional unit (kilogram) of weight the stopping distance is
expected to increase by β ˆ = 0.01092 metres. So, for 10 kilograms of
weight the distance is expected to increase by 0.109 meters.
Generally adequately answered. Identifying the correct hypotheses in part (ii) was
problematic in some cases, while many candidates failed to assume bivariate normality.
END OF %%%%%%%%%%%%%%%%%%%%%%%%%%%%%%%%
Page 11
