 88 lines (72 sloc) 2.45 KB
\documentclass[a4paper,12pt]{article}

%%%%%%%%%%%%%%%%%%%%%%%%%%%%%%%%%%%%%%%%%%%%%%%%%%%%%%%%%%%%%%%%%%%%%%%%%%%%%%%%%%%%%%%%%%%%%%%%%%%%%%%%%%%%%%%%%%%%%%%%%%%%%%%%%%%%%%%%%%%%%%%%%%%%%%%%%%%%%%%%%%%%%%%%%%%%%%%%%%%%%%%%%%%%%%%%%%%%%%%%%%%%%%%%%%%%%%%%%%%%%%%%%%%%%%%%%%%%%%%%%%%%%%%%%%%%

\usepackage{eurosym}
\usepackage{vmargin}
\usepackage{amsmath}
\usepackage{graphics}
\usepackage{epsfig}
\usepackage{enumerate}
\usepackage{multicol}
\usepackage{subfigure}
\usepackage{fancyhdr}
\usepackage{listings}
\usepackage{framed}
\usepackage{graphicx}
\usepackage{amsmath}
\usepackage{chngpage}

%\usepackage{bigints}
\usepackage{vmargin}

% left top textwidth textheight headheight

% headsep footheight footskip

\setmargins{2.0cm}{2.5cm}{16 cm}{22cm}{0.5cm}{0cm}{1cm}{1cm}

\renewcommand{\baselinestretch}{1.3}

\setcounter{MaxMatrixCols}{10}

\begin{document}
\begin{enumerate}
\item 
In order to compare the effectiveness of two new vaccines, A and B, for a childhood
disease, 11 infants were immunised with vaccine A and 9 infants were immunised
with vaccine B. One month after immunisation the concentration of the disease
antibodies in the blood of each infant was recorded in appropriate units. The sample
mean and variance for each group is given below.
Vaccine A: n A = 11, x A = 4.05, s 2 A = 0.692
Vaccine B: n B = 9, x B = 4.36, s B 2 = 0.813
It is assumed that the distributions of the antibody concentration levels after
immunisation with vaccine A and vaccine B are N ( \mu A , \sigma^2 A ) and N ( \mu B , \sigma^2 B )
respectively. You may assume that the samples are independent.
s 2 A / \sigma^2 A
.
(i) State the distribution of the pivotal quantity
(ii) using the
\sigma^2 B
pivotal quantity in part (i). (You are not required to show the derivation of the
interval.)

s B 2 / \sigma^2 B
Calculate an equal-tailed 95\% confidence interval for the ratio

\sigma^2 A
We now assume that \sigma^2 A = \sigma^2 B = \sigma^2 . Under this assumption, you are given that the
distribution of
18 S 2 p
2
is χ 18
, where S 2 p is the pooled variance of the two samples and
σ
is independent from x A and x B .
(iii)
2
Explain why, under the above result, the sampling distribution of
X A − X B − ( \mu A − \mu B )
1 1
S p
+
11 9
(iv)
(v)
CT3 S2012–6
is t 18 . 
Calculate an equal-tailed 95\% confidence interval for \mu A − \mu B using the
sampling distribution in part (iii). (You are not required to show the
derivation of the interval.) 
Comment on your results with regard to differences between vaccine A and
vaccine B.
\end{enumerate}
\newpage

%%%%%%%%%%%%%%%%5
11
(i) This is an F distribution with 10, 8 degrees of freedom.
(ii) ⎛ S 2 / S 2
S 2 / S 2 ⎞
The interval is given by ⎜ A B , A B ⎟
⎜ F 10,8,0.025 F 10,8,0.975 ⎟
⎝
⎠
From tables F 10,8,0.025 = 4.295 and F 10,8,0.975 = 1/ F 8,10,0.025 = 1/ 3.855
(iii)
⎛ 0.692 / 0.813
⎞
giving ⎜
, (0.692 / 0.813) *3.855 ⎟ = (0.198, 3.281)
4.295
⎝
⎠
As the two samples are independent we have that
V ( X A − X B ) =
V ( X A ) V ( X B )
+
= \sigma^2 (1/11 + 1 / 9)
11
9
Normality of the data then gives that Z =
X A − X B − ( \mu A −\mu B )
~ N (0,1)
1 1
σ
+
11 9
Page 7%%%%%%%%%%%%%%%%%%%%%%%%%%%%%%%%%%%%5 – September 2012 – 
%%%%%%%%%%%%%%%%%%%%%%%%%%%%%%%%
We are also given that Y =
σ
2
2
and with Z and Y being independent
~ χ 18
Z
X A − X B − ( \mu A −\mu B )
~ t 18 to obtain
~ t 18 .
Y / 18
1 1
S p
+
11 9
we can use that
(iv)
18 S 2 p
First compute s 2 p =
10*0.692 + 8*0.813
= 0.74577 ⇒ s p = 0.864
18
Then with t 18,0.025 = 2.101 the interval is given by (4.05 – 4.36) \pm 2.101 *
0.864 (1/11 + 1/9) 1/2 i.e. (– 1.126, 0.506).
(v)
The interval includes the value 0, suggesting that there is no difference in the
mean effectiveness of the two vaccines.
Part (iii) was problematic for many candidates. Many candidates struggled to provide a
‘proof’ that had sufficient rigour. There were errors also in determining the endpoints of the
CI in part (ii), often due to using the wrong percentiles of the F distribution.
\end{document}
