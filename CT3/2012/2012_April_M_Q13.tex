\documentclass[a4paper,12pt]{article}

%%%%%%%%%%%%%%%%%%%%%%%%%%%%%%%%%%%%%%%%%%%%%%%%%%%%%%%%%%%%%%%%%%%%%%%%%%%%%%%%%%%%%%%%%%%%%%%%%%%%%%%%%%%%%%%%%%%%%%%%%%%%%%%%%%%%%%%%%%%%%%%%%%%%%%%%%%%%%%%%%%%%%%%%%%%%%%%%%%%%%%%%%%%%%%%%%%%%%%%%%%%%%%%%%%%%%%%%%%%%%%%%%%%%%%%%%%%%%%%%%%%%%%%%%%%%

\usepackage{eurosym}
\usepackage{vmargin}
\usepackage{amsmath}
\usepackage{graphics}
\usepackage{epsfig}
\usepackage{enumerate}
\usepackage{multicol}
\usepackage{subfigure}
\usepackage{fancyhdr}
\usepackage{listings}
\usepackage{framed}
\usepackage{graphicx}
\usepackage{amsmath}
\usepackage{chngpage}

%\usepackage{bigints}
\usepackage{vmargin}

% left top textwidth textheight headheight

% headsep footheight footskip

\setmargins{2.0cm}{2.5cm}{16 cm}{22cm}{0.5cm}{0cm}{1cm}{1cm}

\renewcommand{\baselinestretch}{1.3}

\setcounter{MaxMatrixCols}{10}

\begin{document}

13
The quality of primary schools in eight regions in the UK is measured by an index
ranging from 1 (very poor) to 10 (excellent). In addition the value of a house price
index for these eight regions is observed. The results are given in the following table:
Region i
School quality index x i
House price index y i
1
2
3
4
5
6
7
8
Sum
7
8
5
8
4
9
6
9
56
195 195 170 190 150 190 200 210 1500
The last column contains the sum of all eight columns.
From these values we obtain the following results:
\sum x_i y i = 10, 695; \sum x_i 2 = 416;
(i)
\sum y i 2 = 283, 750

\begin{enumerate}
\item Calculate the correlation coefficient between the index of school quality and
the house price index.

You can assume that the joint distribution of the two random variables is a bivariate
normal distribution.
\item (ii) Perform a statistical test for the null hypothesis that the true correlation
coefficient between the school quality index and the house price index is
equal to 0.8 against the alternative that the correlation coefficient is smaller
than 0.8, by calculating an approximate p-value.
\item 
(iii) Fit a linear regression model to the data, by considering the school quality
index as the explanatory variable. You should write down the model and
estimate all parameters.
\item 
(iv) Calculate the coefficient of determination R 2 for the regression model
obtained in part (iii).
\item (v)
Provide a brief interpretation of the slope of the regression model obtained in
part (iii).
\end{enumerate}


CT3 A2012–8


%%%%%%%%%%%%%%%%%%%%%%%%%%%%%%%%%%%%
\newpage
%%-- Quesiton  13
(i)
S_{xx} = 24, S_{yy} = 2500, S_{xy} = 195
r =
(ii)
W =
S_{xy}
S_{xx} S_{yy}
= 0.796084
1
1 + ρ
1
1 + r
and
is normally distributed with mean log
log
2
1 − ρ
2
1 − r
standard deviation 1/ n − 3
observed value of W is w = 1.087828
Under $H_{0}$ the mean of $W$ is 1.098612
And the standard deviation is 0.447214
p-value is 

\begin{eqnarray*}
P [ W < 1.087828 ] &=& P [ Z < ( 1.087828 − 1.098612) / 0.447214]\\
&=& P [ Z < − 0.024113527 ] \\ &=& 1 − F (0.024113527) > 0.49\\
\end{eqnarray*}
%%%%%%%%%%%%%%%%%%%%%%%%%%%%%%%%%%%%%%%%%%%%%%%%%%%%%%%%%%
No evidence against the null hypothesis.
(iii)
Y i = a + bX i + ε i
{
For b we obtain: \hat{b} = n \sum x_i y i −
\sum x_i \sum y i } { n \sum x_i 2 − ( \sum x_i ) 2 }
− 1
8*10695 − 56*1500
= 8.125
And therefore: \hat{b} =
8*416 − 56 2
%%%%%%%%%%%%%%%%%%%%%%%%%%%%%%%%%%%%5 – April 2012 – %%%%%%%%%%%%%%%%%%%%%%%%%%%%%%%%
And \hat{a}=
1
8
( \sum y i − \hat{b} \sum x_i ) = 130.625
(iv) R^{2} \;=\; 0.796084 2 = 0.634
(v) Any increase in school quality by 1 index-point, leads to an increase of 8.125
in the house price index.
Mostly very well answered.
\end{document}
