
\documentclass[a4paper,12pt]{article}

%%%%%%%%%%%%%%%%%%%%%%%%%%%%%%%%%%%%%%%%%%%%%%%%%%%%%%%%%%%%%%%%%%%%%%%%%%%%%%%%%%%%%%%%%%%%%%%%%%%%%%%%%%%%%%%%%%%%%%%%%%%%%%%%%%%%%%%%%%%%%%%%%%%%%%%%%%%%%%%%%%%%%%%%%%%%%%%%%%%%%%%%%%%%%%%%%%%%%%%%%%%%%%%%%%%%%%%%%%%%%%%%%%%%%%%%%%%%%%%%%%%%%%%%%%%%

\usepackage{eurosym}
\usepackage{vmargin}
\usepackage{amsmath}
\usepackage{graphics}
\usepackage{epsfig}
\usepackage{enumerate}
\usepackage{multicol}
\usepackage{subfigure}
\usepackage{fancyhdr}
\usepackage{listings}
\usepackage{framed}
\usepackage{graphicx}
\usepackage{amsmath}
\usepackage{chngpage}

%\usepackage{bigints}
\usepackage{vmargin}

% left top textwidth textheight headheight

% headsep footheight footskip

\setmargins{2.0cm}{2.5cm}{16 cm}{22cm}{0.5cm}{0cm}{1cm}{1cm}

\renewcommand{\baselinestretch}{1.3}

\setcounter{MaxMatrixCols}{10}

\begin{document}
\begin{enumerate}
%%-- Question 11
\item An experiment has three possible outcomes (A, B, C) and a model states that the
probabilities of these outcomes are $\theta$, $\theta^2$ , and $1 – \theta – \theta^2$ respectively, for some suitable
value of $\theta > 0$.
Let n A , n B , and n C be the number of occurrences of outcomes A, B, and C respectively
in n (= n A + n B + n C ) repetitions of the experiment. Let A ( \theta ) represent the log-
 
likelihood function, and let \[U ( \theta ) =
\frac{\partial}{\partial} A ( \theta )\]
.
\frac{\partial}{\partial}\theta\]
(i)
(a)
Show that
U ( \theta ) =
(b)
n A + 2 n B n C ( 1 + 2 \theta )
.
−
\theta
1 − \theta − \theta^2
Hence find a quadratic equation whose solution gives the maximum
likelihood estimate of \theta.

(ii)
(a) Find an expression for
(b) Hence show that
\frac{\partial}{\partial} U ( \theta )
\frac{\partial}{\partial}\theta
(
(
.
)
)
2
⎡ \frac{\partial}{\partial} U ( \theta ) ⎤ n 1 + 4 \theta − \theta
E ⎢ −
.
⎥ =
2
\frac{\partial}{\partial}\theta
\theta
−
\theta
−
\theta
1
⎣
⎦

The results of 100 repetitions of the experiment show that outcome A occurred
51times, outcome B occurred 16 times, and outcome C occurred 33 times.
(iii)
(a) Show that the maximum likelihood estimate of $\theta$ is $\hat{\theta}= 0.4525$.
(b) Calculate an estimate of the asymptotic standard error of $\hat{\theta}$.
(c) Find an approximate 95\% confidence interval for $\theta$.



%%%%%%%%%%%%%%%%%%%%%%%%%%%10
(i)
(a)
S takes values 0, 1, 2, 3, 4 and we have
\begin{itemize}
\item $ P ( S = 0 ) = 0.4 $
\item $ P ( S = 1 ) = 0.4 \times 0.7 = 0.28 $
\item $ P ( S = 2 ) = 0.4 \times 0.3 + 0.2 \times 0.7 2 = 0.218$
\item $ P ( S = 3 ) = 0.2 \times 2 \times 0.7 \times 0.3 = 0.084$
\item $ P ( S = 4 ) = 0.2 \times 0.3 2 = 0.018$
\end{itemize}
(b)
Hence
E ( S ) = 0.28 + 2 \times 0.218 + 3 \times 0.084 + 4 \times 0.018 = 1.04
(ii)
(a)
E ( S | N = 0 ) = 0,
E ( S | N = 1 ) = E ( X ) = 0.7 + 2 \times 0.3 = 1.3
E ( S | N = 2 ) = E ( 2 X ) = 2 \times 1.3 = 2.6
Page 6%%%%%%%%%%%%%%%%%%%%%%%%%%%%%%%%%%%%5 – April 2012 – %%%%%%%%%%%%%%%%%%%%%%%%%%%%%%%%
(b)
\begin{itemize}
\item Hence, E ( S ) = 1.3 \times 0.4 + 2.6 \times 0.2 = 1.04 as before.
\item Most candidates encountered problems here, as they failed to work out the probability
function from first principles in part (i). Also, many did not recognise this as a
compound distribution type of question.
\end{itemize}

\newpage

%%----Question 11
(i)
(a)
( ) ( 1 − \theta − \theta )
n B
L ( \theta ) = k \theta n A \theta^2
2
n C
(
)
A ( \theta ) = ( n A + 2 n B ) log \theta + n C log 1 − \theta − \theta^2 + c
U ( \theta ) =
(b)
n A + 2 n B n C ( 1 + 2 \theta )
−
\theta
1 − \theta − \theta^2
Setting $U(\theta) = 0$ ⇒ (n A + 2n B )(1 – \theta – \theta^2 ) = n C \theta (1 + 2\theta)
⇒ \hat{\theta}satisfies
( n A + 2 n B + 2 n C ) \theta^2 + ( n A + 2 n B + n C ) \theta − ( n A + 2 n B ) = 0
(ii)
(a)
(b)
\frac{\partial}{\partial} U ( \theta )
\frac{\partial}{\partial}\theta
= − n A + 2 n B
= − n A + 2 n B
\theta
\theta^2
(
− n C
−
)
( 1 − \theta − \theta )
2
2
(
)
( 1 − \theta − \theta )
n C 3 + 2 \theta + 2 \theta^2
2
(
2
)(
2
2
⎡ \frac{\partial}{\partial} U ( \theta ) ⎤
n \theta + 2 n \theta^2 n 1 − \theta − \theta 3 + 2 \theta + 2 \theta
E ⎢ −
+
⎥ =
2
2
\frac{\partial}{\partial}\theta
\theta
⎣
⎦
1 − \theta − \theta^2
=
(iii)
2
2 1 − \theta − \theta^2 − ( 1 + 2 \theta )( − 1 − 2 \theta )
(
)
\theta ( 1 − \theta − \theta )
(/
)
)
n 1 + 4 \theta − \theta^2
2
(a) \hat{\theta}satisfies 149\theta^2 + 116\theta – 83 = 0 ⇒ \hat{\theta}= 0.4525
(b) Using the Cramer-Rao lower bound, estimate of asymptotic standard
error is
⎡ ˆ
ˆ ˆ 2
⎢ \theta (1 − \theta − \theta )
⎢ 100 1 + 4 \theta ˆ − \theta ˆ 2
⎣
(
1/2
)
⎤
⎥
⎥
⎦
= 0.0244
%%%%%%%%%%%%%%%%%%%%%%%%%%%%%%%%%%%5 – April 2012 – %%%%%%%%%%%%%%%%%%%%%%%%%%%%%%%%

95\% CI for $\theta$ is $0.4525 \pm (1.96 \times 0.0244)$ i.e. $0.4525 \pm 0.0478$ i.e.
(0.405, 0.500)
(c)

% Part (i) was very well answered. The differentiation in part (ii) was problematic. Also, many candidates could not identify the random variable for which expectation was required in (ii)(b).5

\end{document}
