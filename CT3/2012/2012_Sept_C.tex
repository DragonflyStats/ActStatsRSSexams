
\documentclass[a4paper,12pt]{article}

%%%%%%%%%%%%%%%%%%%%%%%%%%%%%%%%%%%%%%%%%%%%%%%%%%%%%%%%%%%%%%%%%%%%%%%%%%%%%%%%%%%%%%%%%%%%%%%%%%%%%%%%%%%%%%%%%%%%%%%%%%%%%%%%%%%%%%%%%%%%%%%%%%%%%%%%%%%%%%%%%%%%%%%%%%%%%%%%%%%%%%%%%%%%%%%%%%%%%%%%%%%%%%%%%%%%%%%%%%%%%%%%%%%%%%%%%%%%%%%%%%%%%%%%%%%%

\usepackage{eurosym}
\usepackage{vmargin}
\usepackage{amsmath}
\usepackage{graphics}
\usepackage{epsfig}
\usepackage{enumerate}
\usepackage{multicol}
\usepackage{subfigure}
\usepackage{fancyhdr}
\usepackage{listings}
\usepackage{framed}
\usepackage{graphicx}
\usepackage{amsmath}
\usepackage{chngpage}

%\usepackage{bigints}
\usepackage{vmargin}

% left top textwidth textheight headheight

% headsep footheight footskip

\setmargins{2.0cm}{2.5cm}{16 cm}{22cm}{0.5cm}{0cm}{1cm}{1cm}

\renewcommand{\baselinestretch}{1.3}

\setcounter{MaxMatrixCols}{10}

\begin{document}
\begin{enumerate}


\item 

%%-- Question 4


Consider a random variable U that has a uniform distribution on (0,1) and a random variable X that has a standard normal distribution. Assume that U and X are independent.
Determine an expression for the probability density function of the random variable Z
= U + X in terms of the cumulative distribution function oF_X.

%%-- Question 5
\item 
A large portfolio consists of 20\% class A policies, 50% class B policies and 30% class
C policies. Ten policies are selected at random from the portfolio.
(i)
(ii)


Calculate the probability that there are no policies of class A among the
randomly selected ten.

(a) Calculate the expected number of class B policies among the randomly
selected ten.
(b) Calculate the probability that there are more than five class B policies
among the randomly selected ten.

\end{enumerate}
\newpage

%%%%%%%%%%%%%%%%%%%%%

4
Let $F_Z( z )$ be the density o $F_Z= U + X$ .
1
\[F_Z( z ) = \int f U ( u ) F_X ( z − u ) du = \int F_X ( z − u ) du\]
u
0
z
=
\int F_X ( x ) dx = F_X ( z ) − F_X ( z − 1)
z − 1
where we have used the substitution u = z − x , and where $F_X$ is the distribution
function o $F_X$ .
%This question was very poorly answered. A large number of candidates did not attempt it at all, while many others did not follow any reasonable approach. Note that this is based on
% standard bookwork, viz. Unit 6, Section 3 in the Core Reading.
%%%%%%%%%%%%%%%%%%%%%%%%%%%%%%%%%%%%%%%%%%%%
5
(i) P(none of class A) = P(all 10 of class B or C) = (0.8) 10 = 0.1074
(ii) (a)
Let B = number of class B.
Note that B ~ binomial (10, 0.5), so that $E(B) = (10)(0.5) = 5$
(b)
\[P(B > 5) = 1 – P( B \leq 5 ) = 1 – 0.6230 = 0.3770\]
[0.6230 is from tables; alternatively by evaluation]
This was generally very well answered. A common error in part (ii) (b) was to calculate
P(B < or = 5) instead of P(B > 5).
\end{document} 
%%%%%%%%%%%%%%%%%%%%%%%%%%%%%%%%
\newpage
5
(i) \[P(none of class A) = P(all 10 of class B or C) = (0.8)^10 = 0.1074\]
(ii) (a)
Let B = number of class B.
Note that B ~ binomial (10, 0.5), so that E(B) = (10)(0.5) = 5
(b)
\[P(B > 5) = 1 – P( B \leq 5 ) = 1 – 0.6230 = 0.3770\]
[0.6230 is from tables; alternatively by evaluation]
This was generally very well answered. A common error in part (ii) (b) was to calculate
P(B < or = 5) instead of P(B > 5).

%%%%%%%%%%%%%%%%%%%%%%%%%%%%%%%%%%%%%%%%%%%%%%%%%%%%%%%%%%%%%%%
\newpage

6
(i)
Population mean = 8\theta
So MME is solution of X = 8 \theta ⇒
(ii)
⎛ X
E ⎜
⎝ 8
MME =
X
8
⎞ 1
1
⎟ = E ( X ) = (8 \theta ) = \theta
8
⎠ 8
⎛ X ⎞
Bias = E ⎜ ⎟ − \theta = 0 (i.e. MME is unbiased for $\theta$).
⎝ 8 ⎠
(iii)
௑ത ௑ത ଼ఏ మ
଼ ଼ ଺ସ௡
ൌ
ఏ మ
(a) Since MME is unbiased, ‫ܧܵܯ‬ ቀ ቁ ൌ ‫ݎܽݒ‬ ቀ ቁ ൌ
(b) MME gets more efficient (MSE gets smaller) as sample size increases.
%%%%%%%%%%%%%%%%%%%%%%%%%%%%%%%%%%%%%%%%%%%%%%%%%%%%%%%%%%
\newpage

%% Question 7
(i) With the larger sample of 100 claims the standard error of the sample mean
will be smaller, giving a narrower confidence interval.
(ii) The replacement of the extreme value will give a smaller sample mean, which
means that the interval will be shifted to the left.
The variance of the sample will also be smaller, which will again give a narrower interval.

%Many candidates recognised the correct effect on the interval, without being able to justify it properly. Note that reasonably accurate wording is important in providing the comments and justification required here.

\end{document}
