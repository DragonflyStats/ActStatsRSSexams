\documentclass[a4paper,12pt]{article}

%%%%%%%%%%%%%%%%%%%%%%%%%%%%%%%%%%%%%%%%%%%%%%%%%%%%%%%%%%%%%%%%%%%%%%%%%%%%%%%%%%%%%%%%%%%%%%%%%%%%%%%%%%%%%%%%%%%%%%%%%%%%%%%%%%%%%%%%%%%%%%%%%%%%%%%%%%%%%%%%%%%%%%%%%%%%%%%%%%%%%%%%%%%%%%%%%%%%%%%%%%%%%%%%%%%%%%%%%%%%%%%%%%%%%%%%%%%%%%%%%%%%%%%%%%%%

\usepackage{eurosym}
\usepackage{vmargin}
\usepackage{amsmath}
\usepackage{graphics}
\usepackage{epsfig}
\usepackage{enumerate}
\usepackage{multicol}
\usepackage{subfigure}
\usepackage{fancyhdr}
\usepackage{listings}
\usepackage{framed}
\usepackage{graphicx}
\usepackage{amsmath}
\usepackage{chngpage}

%\usepackage{bigints}
\usepackage{vmargin}

% left top textwidth textheight headheight

% headsep footheight footskip

\setmargins{2.0cm}{2.5cm}{16 cm}{22cm}{0.5cm}{0cm}{1cm}{1cm}

\renewcommand{\baselinestretch}{1.3}

\setcounter{MaxMatrixCols}{10}

\begin{document}
9
An analyst is interested in using a gamma distribution with parameters α = 2 and
1
1 − x
λ = 1⁄2, that is, with density function f ( x ) = xe 2 , 0 < x < ∞ .
4
\begin{enumerate}[(a)]
\item (i)
(a) State the mean and standard deviation of this distribution.
(b) Hence comment briefly on its shape.
\item 
(ii)
Show that the cumulative distribution function is given by
1
− x
1
F ( x ) = 1 − (1 + x ) e 2 , 0 < x < ∞
2
(zero otherwise).

The analyst wishes to simulate values x from this gamma distribution and is able to
generate random numbers u from a uniform distribution on (0,1).
\item (iii)
(a) Specify an equation involving x and u, the solution of which will yield
the simulated value x.
(b) Comment briefly on how this equation might be solved.
(c) The graph below gives F(x) plotted against x. Use this graph to obtain
the simulated value of x corresponding to the random number u = 0.66.

\end{enumerate}

\newpage
9
\begin{itemize}
\item (i)
α
= 8 = 2.8
mean =
(b) As claims are non-negative and the s.d. is quite large relative to the
mean, then the distribution will be quite positively skewed.
F ( x )
λ 2
1
x
\item (ii)
α
= 4 and s.d. =
λ
(a)
1 − t
= \int te 2 dt
4
0
1
x
− t
1
= − \int td ( e 2 )
2
0
1
x
1
1 − t
1 − t
= − [ te 2 ] 0 x + \int e 2 dt
2
2
0
1
1
− t
1 − x
= − xe 2 − [ e 2 ] 0 x
2
1
− x
1
= 1 − (1 + x ) e 2
2
1
\item 
(iii)
− x
1
i.e. 1 − (1 + x ) e 2 = u
2
(a) F ( x ) = u
(b) This equation would have to be solved numerically
(c) Using u = 0.66 on the vertical axis, we invert to get x = 4.5 on the
horizontal axis.
In part (ii) many candidates failed to integrate correctly. A lot of problems were caused by not using the correct limits for the integral. In part (iii) a popular answer was to use “trial-
and-error”, which is not an appropriate approach here.
Page 6%%%%%%%%%%%%%%%%%%%%%%%%%%%%%%%%%%%%5 – September 2012 – %%%%%%%%%%%%%%%%%%%%%%%%%%%%%%%%
\end{itemize}
\end{document}
