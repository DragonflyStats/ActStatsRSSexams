
\documentclass[a4paper,12pt]{article}

%%%%%%%%%%%%%%%%%%%%%%%%%%%%%%%%%%%%%%%%%%%%%%%%%%%%%%%%%%%%%%%%%%%%%%%%%%%%%%%%%%%%%%%%%%%%%%%%%%%%%%%%%%%%%%%%%%%%%%%%%%%%%%%%%%%%%%%%%%%%%%%%%%%%%%%%%%%%%%%%%%%%%%%%%%%%%%%%%%%%%%%%%%%%%%%%%%%%%%%%%%%%%%%%%%%%%%%%%%%%%%%%%%%%%%%%%%%%%%%%%%%%%%%%%%%%

\usepackage{eurosym}
\usepackage{vmargin}
\usepackage{amsmath}
\usepackage{graphics}
\usepackage{epsfig}
\usepackage{enumerate}
\usepackage{multicol}
\usepackage{subfigure}
\usepackage{fancyhdr}
\usepackage{listings}
\usepackage{framed}
\usepackage{graphicx}
\usepackage{amsmath}
\usepackage{chngpage}

%\usepackage{bigints}
\usepackage{vmargin}

% left top textwidth textheight headheight

% headsep footheight footskip

\setmargins{2.0cm}{2.5cm}{16 cm}{22cm}{0.5cm}{0cm}{1cm}{1cm}

\renewcommand{\baselinestretch}{1.3}

\setcounter{MaxMatrixCols}{10}

\begin{document}
\begin{enumerate}
%%-- Question 11
\item An experiment has three possible outcomes (A, B, C) and a model states that the
probabilities of these outcomes are $\theta$, $\theta^2$ , and $1 – \theta – \theta^2$ respectively, for some suitable
value of $\theta > 0$.
Let n A , n B , and n C be the number of occurrences of outcomes A, B, and C respectively
in n (= n A + n B + n C ) repetitions of the experiment. Let A ( \theta ) represent the log-
 
likelihood function, and let \[U ( \theta ) =
\frac{\partial}{\partial} A ( \theta )\]
.
\frac{\partial}{\partial}\theta\]
(i)
(a)
Show that
U ( \theta ) =
(b)
n A + 2 n B n C ( 1 + 2 \theta )
.
−
\theta
1 − \theta − \theta^2
Hence find a quadratic equation whose solution gives the maximum
likelihood estimate of \theta.

(ii)
(a) Find an expression for
(b) Hence show that
\frac{\partial}{\partial} U ( \theta )
\frac{\partial}{\partial}\theta
(
(
.
)
)
2
⎡ \frac{\partial}{\partial} U ( \theta ) ⎤ n 1 + 4 \theta − \theta
E ⎢ −
.
⎥ =
2
\frac{\partial}{\partial}\theta
\theta
−
\theta
−
\theta
1
⎣
⎦

The results of 100 repetitions of the experiment show that outcome A occurred
51times, outcome B occurred 16 times, and outcome C occurred 33 times.
(iii)
(a) Show that the maximum likelihood estimate of $\theta$ is $\hat{\theta}= 0.4525$.
(b) Calculate an estimate of the asymptotic standard error of $\hat{\theta}$.
(c) Find an approximate 95\% confidence interval for $\theta$.



%%%%%%%%%%%%%%%%%%%%%%%%%%%10
(i)
(a)
S takes values 0, 1, 2, 3, 4 and we have
\begin{itemize}
\item $ P ( S = 0 ) = 0.4 $
\item $ P ( S = 1 ) = 0.4 \times 0.7 = 0.28 $
\item $ P ( S = 2 ) = 0.4 \times 0.3 + 0.2 \times 0.7 2 = 0.218$
\item $ P ( S = 3 ) = 0.2 \times 2 \times 0.7 \times 0.3 = 0.084$
\item $ P ( S = 4 ) = 0.2 \times 0.3 2 = 0.018$
\end{itemize}
(b)
Hence
E ( S ) = 0.28 + 2 \times 0.218 + 3 \times 0.084 + 4 \times 0.018 = 1.04
(ii)
(a)
E ( S | N = 0 ) = 0,
E ( S | N = 1 ) = E ( X ) = 0.7 + 2 \times 0.3 = 1.3
E ( S | N = 2 ) = E ( 2 X ) = 2 \times 1.3 = 2.6
Page 6%%%%%%%%%%%%%%%%%%%%%%%%%%%%%%%%%%%%5 – April 2012 – %%%%%%%%%%%%%%%%%%%%%%%%%%%%%%%%
(b)
\begin{itemize}
\item Hence, E ( S ) = 1.3 \times 0.4 + 2.6 \times 0.2 = 1.04 as before.
\item Most candidates encountered problems here, as they failed to work out the probability
function from first principles in part (i). Also, many did not recognise this as a
compound distribution type of question.
\end{itemize}

\newpage

%%----Question 11
(i)
(a)
( ) ( 1 − \theta − \theta )
n B
L ( \theta ) = k \theta n A \theta^2
2
n C
(
)
A ( \theta ) = ( n A + 2 n B ) log \theta + n C log 1 − \theta − \theta^2 + c
U ( \theta ) =
(b)
n A + 2 n B n C ( 1 + 2 \theta )
−
\theta
1 − \theta − \theta^2
Setting $U(\theta) = 0$ ⇒ (n A + 2n B )(1 – \theta – \theta^2 ) = n C \theta (1 + 2\theta)
⇒ \hat{\theta}satisfies
( n A + 2 n B + 2 n C ) \theta^2 + ( n A + 2 n B + n C ) \theta − ( n A + 2 n B ) = 0
(ii)
(a)
(b)
\frac{\partial}{\partial} U ( \theta )
\frac{\partial}{\partial}\theta
= − n A + 2 n B
= − n A + 2 n B
\theta
\theta^2
(
− n C
−
)
( 1 − \theta − \theta )
2
2
(
)
( 1 − \theta − \theta )
n C 3 + 2 \theta + 2 \theta^2
2
(
2
)(
2
2
⎡ \frac{\partial}{\partial} U ( \theta ) ⎤
n \theta + 2 n \theta^2 n 1 − \theta − \theta 3 + 2 \theta + 2 \theta
E ⎢ −
+
⎥ =
2
2
\frac{\partial}{\partial}\theta
\theta
⎣
⎦
1 − \theta − \theta^2
=
(iii)
2
2 1 − \theta − \theta^2 − ( 1 + 2 \theta )( − 1 − 2 \theta )
(
)
\theta ( 1 − \theta − \theta )
(/
)
)
n 1 + 4 \theta − \theta^2
2
(a) \hat{\theta}satisfies 149\theta^2 + 116\theta – 83 = 0 ⇒ \hat{\theta}= 0.4525
(b) Using the Cramer-Rao lower bound, estimate of asymptotic standard
error is
⎡ ˆ
ˆ ˆ 2
⎢ \theta (1 − \theta − \theta )
⎢ 100 1 + 4 \theta ˆ − \theta ˆ 2
⎣
(
1/2
)
⎤
⎥
⎥
⎦
= 0.0244
%%%%%%%%%%%%%%%%%%%%%%%%%%%%%%%%%%%5 – April 2012 – %%%%%%%%%%%%%%%%%%%%%%%%%%%%%%%%

95\% CI for $\theta$ is $0.4525 \pm (1.96 \times 0.0244)$ i.e. $0.4525 \pm 0.0478$ i.e.
(0.405, 0.500)
(c)

% Part (i) was very well answered. The differentiation in part (ii) was problematic. Also, many candidates could not identify the random variable for which expectation was required in (ii)(b).5

%%%%%%%%%%%%%%%
[Total 8]
CT3 S2012–410
The number of hours that people watch television per day is the subject of an empirical study that is carried out in four regions in a country. Five people are randomly selected in each of the regions and are asked about the average number of
hours per day that they spent watching television during the last year. The results are shown in the following table, with the last column shows the average in each region.

Region 1
Region 2
Region 3
Region 4
2.0
1.2
2.5
1.2
1.1
1.0
2.0
1.7
0.2
0.9
2.6
1.0
3.8
1.1
2.4
1.8
2.8
1.6
2.3
1.3
Average
1.98
1.16
2.36
1.40
Based on the above observations the following ANOVA table was obtained:
Source of variation
Between regions
Residual
d.f.
...
...
SS
4.4655
8.892
MSS
...
...
\begin{enumerate}[(i)]
\item (i) State the mathematical model underlying the one-way analysis of variance
together with all associated assumptions.

\item (ii) Complete the ANOVA table.
\item (iii) Carry out an analysis of variance to test the hypothesis that the region has no
effect on the average time spent watching television. You should write down
the null hypothesis, calculate the value of the test-statistic, state its distribution
including any parameters, calculate the p-value approximately and state your
conclusion.
\end{enumerate}
% [Total 8]
% CT3 S2012–5
%%%%%%%%%%%%%%%%%%%%%%%%%%%%%%%%%%%%%%%%%%%%%%%%%%%%%%%%%%%%%%%%%%%%%%%%%%%%%
\newpage
11
In order to compare the effectiveness of two new vaccines, A and B, for a childhood
disease, 11 infants were immunised with vaccine A and 9 infants were immunised
with vaccine B. One month after immunisation the concentration of the disease
antibodies in the blood of each infant was recorded in appropriate units. The sample
mean and variance for each group is given below.
\begin{verbatim}
Vaccine A: n A = 11, x A = 4.05, s 2 A = 0.692
Vaccine B: n B = 9, x B = 4.36, s B 2 = 0.813
\end{verbatim}
It is assumed that the distributions of the antibody concentration levels after
immunisation with vaccine A and vaccine B are N ( \mu A , \sigma^2 A ) and N ( \mu B , \sigma^2 B )
respectively. You may assume that the samples are independent.
s 2 A / \sigma^2 A
.

(i) State the distribution of the pivotal quantity
(ii) using the
$\sigma^2_B$
pivotal quantity in part (i). (You are not required to show the derivation of the
interval.)

s B 2 / \sigma^2 B
Calculate an equal-tailed 95\% confidence interval for the ratio

\sigma^2 A
We now assume that $\sigma^2 A = \sigma^2 B = \sigma^2$ . Under this assumption, you are given that the
distribution of
18 S 2 p
2
is χ 18
, where S 2 p is the pooled variance of the two samples and
σ
is independent from x A and x B .
(iii)
2
Explain why, under the above result, the sampling distribution of
X A − X B − ( \mu A − \mu B )
1 1
S p
+
11 9
(iv)
(v)
CT3 S2012–6
is t 18 . 
\begin{itemize}
\item Calculate an equal-tailed 95\% confidence interval for \mu A − \mu B using the
sampling distribution in part (iii). 
\item You are not required to show the
derivation of the interval.
\item 
Comment on your results with regard to differences between vaccine A and
vaccine B.
\end{itemize}

\end{document}
