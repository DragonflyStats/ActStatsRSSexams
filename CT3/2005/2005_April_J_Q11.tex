\documentclass[a4paper,12pt]{article}

%%%%%%%%%%%%%%%%%%%%%%%%%%%%%%%%%%%%%%%%%%%%%%%%%%%%%%%%%%%%%%%%%%%%%%%%%%%%%%%%%%%%%%%%%%%%%%%%%%%%%%%%%%%%%%%%%%%%%%%%%%%%%%%%%%%%%%%%%%%%%%%%%%%%%%%%%%%%%%%%%%%%%%%%%%%%%%%%%%%%%%%%%%%%%%%%%%%%%%%%%%%%%%%%%%%%%%%%%%%%%%%%%%%%%%%%%%%%%%%%%%%%%%%%%%%%

\usepackage{eurosym}
\usepackage{vmargin}
\usepackage{amsmath}
\usepackage{graphics}
\usepackage{epsfig}
\usepackage{enumerate}
\usepackage{multicol}
\usepackage{subfigure}
\usepackage{fancyhdr}
\usepackage{listings}
\usepackage{framed}
\usepackage{graphicx}
\usepackage{amsmath}
\usepackage{chngpage}

%\usepackage{bigints}
\usepackage{vmargin}

% left top textwidth textheight headheight

% headsep footheight footskip

\setmargins{2.0cm}{2.5cm}{16 cm}{22cm}{0.5cm}{0cm}{1cm}{1cm}

\renewcommand{\baselinestretch}{1.3}

\setcounter{MaxMatrixCols}{10}

\begin{document}
\begin{enumerate}
%%%%%%%%%%%%%%%%%%%%%%%%%%%%%%%%%%%%%%%%%%%%%%%%%%%%%%%%%%%%%%%%%%%%%%%%%%%%%%%%%
%%- Question 10
\item A model used for claim amounts ($X$, in units of \$10,000) in certain circumstances has
the following probability density function, $f(x)$, and cumulative distribution function,
\[ F(x):
f ( x ) =
5(10 5 )
(10 x )
, x
6
0 ; F ( x ) = 1
10
10 x
5\]
.
You are given the information that the distribution of X has mean 2.5 units (\$25,000)
and standard deviation 3.23 units (\$32,300).
\begin{enumerate}
\item (i) Describe briefly the nature of a model for claim sizes for which the standard deviation can be greater than the mean.
\item 
(ii) (a)
Show that we can obtain a simulated observation of X by calculating
x = 10 (1 r )
0.2
1
where r is an observation of a random variable which is uniformly
distributed on (0,1).
(b)
Explain why we can just as well use the formula
x = 10 r
0.2
1
to obtain a simulated observation of X.
(c)
Calculate the missing values for the simulated claim amounts in the table below (which has been obtained using the method in (ii)(b)
above):
r Claim (\$)
0.7423
0.0291
0.2770
0.5895
0.1131
0.9897
0.6875
0.8525
0.0016
0.5154 6,141
10,2872
29,272
11,148
54,635
207
7,782
3,243
?
?
\end{enumerate}
%%%%%%%%%%%%%%%%%%%%%%%%%%%%%%%%%%%%%%%%%%%%%%%%%%%%%%%%%%%%%%%%%%%%%%%%%%%%%%%%%%%%%55
%% [Total 7]
%% CT3 A2005

%%%%%%%%%%%%%%%%%%%%%%%%%%%%%%%%%%%%%%%%%%%%%%%%%%%%%%%%%%%%%%%%%%%%%%%%%%%%%%
\newpage

10
\begin{itemize}
\item (i) X takes positive values only so to have such a relatively high standard deviation the distribution must be positively skewed with sizeable probability
associated with high values (i.e. the model embraces high claim sizes; the density has a long or heavy tail).
\item (ii) (a) Solving r = F(x)
\item (b) R ~ U(0,1) 1 R ~ U(0, 1) so (1 r) is also a random number from (0, 1), so we can use 1 r in place of r , giving the formula
x 10 r
(c)
Page 4
0.2
r = 0.0016
r = 0.5154
(1 + x/10) = (1 r)
1
claim = 262390
claim = 14175
0.2
x = 10[(1
r)
0.2
1] 
\end{itemize}
%%%%%%%%%%%%%%%%%%%%%%%%%%%%%%%%%%%%%%%%%%%%%%%5555
\end{document}
