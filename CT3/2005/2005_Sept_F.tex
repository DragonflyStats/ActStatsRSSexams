
\documentclass[a4paper,12pt]{article}

%%%%%%%%%%%%%%%%%%%%%%%%%%%%%%%%%%%%%%%%%%%%%%%%%%%%%%%%%%%%%%%%%%%%%%%%%%%%%%%%%%%%%%%%%%%%%%%%%%%%%%%%%%%%%%%%%%%%%%%%%%%%%%%%%%%%%%%%%%%%%%%%%%%%%%%%%%%%%%%%%%%%%%%%%%%%%%%%%%%%%%%%%%%%%%%%%%%%%%%%%%%%%%%%%%%%%%%%%%%%%%%%%%%%%%%%%%%%%%%%%%%%%%%%%%%%

\usepackage{eurosym}
\usepackage{vmargin}
\usepackage{amsmath}
\usepackage{graphics}
\usepackage{epsfig}
\usepackage{enumerate}
\usepackage{multicol}
\usepackage{subfigure}
\usepackage{fancyhdr}
\usepackage{listings}
\usepackage{framed}
\usepackage{graphicx}
\usepackage{amsmath}
\usepackage{chngpage}

%\usepackage{bigints}
\usepackage{vmargin}

% left top textwidth textheight headheight

% headsep footheight footskip

\setmargins{2.0cm}{2.5cm}{16 cm}{22cm}{0.5cm}{0cm}{1cm}{1cm}

\renewcommand{\baselinestretch}{1.3}

\setcounter{MaxMatrixCols}{10}

\begin{document}

13
A survey, carried out at a major flower and gardening show, was concerned with the association between the intention to return to the show next year and the purchase of goods at this year s show. There were 220 people interviewed and of these 101 had
made a purchase; 69 of these people said they intended to return next year. Of the 119 who had not made a purchase, 68 said they intended to return next year.
\begin{enumerate}[(i)]
\item Suppose one of the 220 people surveyed is selected at random.
Calculate the probabilities that the selected person:
(a)
(b)
(c)
(ii)
CT3 S2005
intends to return next year, given that he/she has made a purchase
intends to return next year, given that he/she has not made a purchase
has made a purchase, given that he/she intends to return next year

\item By testing the difference between the proportions of purchasers and non-purchasers who intend to return next year, examine whether there is sufficient evidence to justify concluding that the intention to return depends on whether
or not a purchase was made.

614
\item (iii) Present the data as a contingency table and perform a 2 test of association between the attributes intention to return and purchasing status .
\item
(iv) Discuss briefly the connection between the comparison of proportions carried out in part (ii) and the test of association performed in part (iii).
\end{enumerate}

%%%%%%%%%%%%%%%%%%%%%%%%%%%%%%%%%%%%%%%%%%%%%
\newpage
\begin{itemize}
\item The data given in the following table are the numbers of deaths from AIDS in
Australia for 12 consecutive quarters starting from the second quarter of 1983.
1
1
Quarter (i):
Number of deaths (n i ):
(i)
2
2
3
3
4
1
5
4
6 7 8 9
9 18 23 31
10
20
11
25
12
37
(a) Draw a scatterplot of the data.
(b) Comment on the nature of the relationship between the number of
deaths and the quarter in this early phase of the epidemic.

/=\item (ii)
A statistician has suggested that a model of the form
E N i = i 2
might be appropriate for these data, where is a parameter to be estimated
from the above data. She has proposed two methods for estimating , and
these are given in parts (a) and (b) below.
\item (a)
Show that the least squares estimate of , obtained by minimising
q =
12
( n
i 1 i
=
(b)
i 2 ) 2 , is given by
12 2
i n i
i 1
12 4
i
i 1
.
\item Show that an alternative (weighted) least squares estimate of ,
obtained by minimising q * =
(c)
= 12
n
i 1 i
12 2
i
i 1
Noting that 12 4
i
i 1
12
i 1
i 2
n i
i 2
2
is given by
.
= 60, 710 and
12 2
i
i 1
= 650 , calculate
and
for the above data.
\end{itemize}
\newpage

7
%%%%%%%%%%%%%%%%%%%%%%%%%%%%%%%%%%%%%%%%%%%%%%%%%%%%%%%%%%%%%%%%%%%%%%%%%%%%%%%%%%%%%%%%%%%%%%%%%
To assess whether the single parameter model which was used in part (ii) is appropriate for the data, a two parameter model is now considered. The model
is of the form
E N i = i
for i = 1,
(a)
, 12.
To estimate the parameters and , a simple linear regression model
E Y i =
x i
is used, where x i = log(i) and Y i = log(N i ) for i = 1, , 12. Relate the
parameters and to the regression parameters and .
(b)
The least squares estimates of and are 0.6112 and 1.6008 with standard errors 0.4586 and 0.2525 respectively (you are not asked to
verify these results). Using the value for the estimate of , conduct a formal statistical test to assess whether the form of the model suggested in (ii) is adequate.

[Total 19]
%%%%%%%%%%%%%%%%%%%%%%%%%%%%%%%%%%%%

13
return
no return
purchase
69
32
101
no purchase
68
137
51
83
119
220
(i) (a)
(b)
(c)
69/101 (= 0.6832)
68/119 (= 0.5714)
69/137 (= 0.5036)
(ii) H 0 : population proportions of those who intend to return are equal
v H 1 : not H 0
Proportion of purchasers
2
1
69 /101 ; proportion of non-purchasers
68 /119
Under H 0 , estimate of common proportion who intend to return = 137/220
Observed value of D
1
2
0.1117
137 83 1
1
Estimated standard error of D =
220 220 101 119
1/ 2
0.06558
%%-- Page 7
%%-- Subject CT3 
P-value = 2
= 2
September 2005
Examiners Report
P(D > 0.1117) = 2 P(Z > 0.1117/0.06558) = 2
(1 0.95543) = 0.08914 (i.e. 8.9%)
P(Z > 1.70)
There is not sufficient evidence (using a two-sided test) to justify rejecting H 0 i.e. there is not sufficient evidence to justify concluding that the intention to return depends on whether or not a purchase was made.
(iii)
H 0 : no association between attributes v H 1 : not H 0
Expected frequencies under H 0 in brackets:
return
no return
Test statistic = 6.1 2
P-value = P
2
1
1
62.9
2.90
purchase
69 (62.9)
32 (38.1)
101
no purchase
68 (74.1)
51 (44.9)
119
1
1
1
74.1 38.1 44.9
137
83
220
2.90
1 0.9114 0.0886 (i.e. 8.9%)
There is not sufficient evidence to justify rejecting H 0 i.e. there is not sufficient evidence to justify concluding that the intention to return is associated with purchasing status.
(iv)
The two approaches complement each other:
the P-values are the same
the conclusions are the same.
[Note: there is a formal connection: the
the z value in (ii) (1.70).]
Page 8
2
1 value
in (iii) (2.90) is the square of
%%--- Subject CT3 
(i)
(a)
Examiners Report
Points are shown on scatterplot.
40
Number
14
September 2005
30
20
10
0
0
5
10
Quarter
(b)
The mean number of deaths increases with an increasing rate with
quarter.
The variance also appears to increase with quarter.
(ii)
(a)
q
i 2 ) 2
( n i
dq
d
2 i 2 ( n i
dq
d
2 i 2 ( n i
0
i 2 n i
i 4
i 2 n i
i 4
(
d 2 q
d
2
i 2 )
i 2 ) 0
0
.
2 i 4
0
minimum.)
Page 9Subject CT3 
(b)
q * ( n i
dq *
d 2
i 2 ) 2
n i
i
i 2
dq *
d
i
0
n i
i
September 2005
2
i
i
i 2
n i
0
n i
i 2
(
d 2 q *
d
i 2
2
2
i 2 n i
(c)
i
4
n i
i
(iii)
(a)
E N i
15694
60710
174
650
2
minimum.)
0
0.259
0.268
i
Taking logs gives
log E N i
Thus
[OR
(b)
log( i ) = log
= log and
log( i ) log
= .
e and
.]
= 1.6008 s.e.( ) = 0.2525
H 0 :
t
= 2 v H 1 :
2
s.e.( )
2
1.6008 2
0.2525
1.58
Compare with a t-distribution with 10 d.f.
Page 10
x i

As the 5\% critical value of a two-tailed test is 2.228, do not reject the null hypothesis.
Therefore, the model used in (ii) with
=
= 2 seems appropriate.


\end{document}
