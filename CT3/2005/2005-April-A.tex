\documentclass[a4paper,12pt]{article}

%%%%%%%%%%%%%%%%%%%%%%%%%%%%%%%%%%%%%%%%%%%%%%%%%%%%%%%%%%%%%%%%%%%%%%%%%%%%%%%%%%%%%%%%%%%%%%%%%%%%%%%%%%%%%%%%%%%%%%%%%%%%%%%%%%%%%%%%%%%%%%%%%%%%%%%%%%%%%%%%%%%%%%%%%%%%%%%%%%%%%%%%%%%%%%%%%%%%%%%%%%%%%%%%%%%%%%%%%%%%%%%%%%%%%%%%%%%%%%%%%%%%%%%%%%%%

\usepackage{eurosym}
\usepackage{vmargin}
\usepackage{amsmath}
\usepackage{graphics}
\usepackage{epsfig}
\usepackage{enumerate}
\usepackage{multicol}
\usepackage{subfigure}
\usepackage{fancyhdr}
\usepackage{listings}
\usepackage{framed}
\usepackage{graphicx}
\usepackage{amsmath}
\usepackage{chngpage}

%\usepackage{bigints}
\usepackage{vmargin}

% left top textwidth textheight headheight

% headsep footheight footskip

\setmargins{2.0cm}{2.5cm}{16 cm}{22cm}{0.5cm}{0cm}{1cm}{1cm}

\renewcommand{\baselinestretch}{1.3}

\setcounter{MaxMatrixCols}{10}

\begin{document}
\begin{enumerate}

Institute of Actuaries1
Calculate the sample mean and standard deviation of the following claim amounts (£):
534
671
581
620
401
340
980
845
550
690
%%%%%%%%%%%%%%%%%%%%%%%%%%%%%%%%%%%%%%%%%%%%%%%%%%%%%%%%%%%%%%%%%%%%%%%%%%%%%
2
Suppose A, B and C are events with P ( A ) = 12 , P ( B ) = 12 , P ( C ) = 13 , P ( A
P ( A C ) = 16 , P ( B
(a)
(b) Determine whether or not the events A and B are independent.
Calculate the probability P ( A B C ).
C ) =
1
6
and P ( A
B
B ) = 34 ,
1 .
C ) = 12
[4]

%%%%%%%%%%%%%%%%%%%%%%%%%%%%%%%%%%%%%%%%%%%%%%%%%%%%%%%%%%%%%%%%%%%%%%%%%%%%%
3
Claim sizes in a certain insurance situation are modelled by a distribution with
moment generating function M(t) given by
M(t) = (1 10t) 2 .
Show that E[X 2 ] = 600 and find the value of E[X 3 ].

%%%%%%%%%%%%%%%%%%%%%%%%%%%%%%%%%%%%%%%%%%%%%%%%%%%%%%%%%%%%%%%%%%%%%%%%%%%%%%%

1
2
April 2005
Examiners Report
x = 6212 , x 2 = 4186784
x 6212
= £621.20
10
s 1
6212 2
4186784
9
10
(a)
P ( A
B )
327889.6
9
P ( A ) P ( B ) P ( A
P ( A ) P ( B ) ( 1 2 )( 1 2 )
(b)
P ( A B )
P ( A B
1
2
1
3
1
2
3
4
1
2
1
4
so the events A and B are independent as
P ( A ) P ( B ) P ( C ) P ( A
1 1
3 4 1
6 B C ) P ( A
1
6 1
12 5 .]
6
[OR P ( A
3
4
B )
P ( A ) P ( B ).
C )
1
2
1
4
= £190.87
1
6
1 1
6 12
B ) P ( A
C ) P ( B C )
P ( A B
C )
5
6
B ) P ( C ) P ( A
C ) P ( B
C ) P ( A
B
C )
[OR Use a Venn diagram]
3
M(t) = (1 10t) 2
M (t) = ( 2)( 10)(1 10t) 3 = 20(1 10t) 3
M (t) = ( 60)( 10)(1 10t) 4 = 600(1 10t) 4
M (t) = ( 2400)( 10)(1 10t) 5 = 24000(1 10t)
5
Putting t = 0
Putting t = 0
E[X 2 ] = 600
E[X 3 ] = 24000
[OR use the power series expansion M(t) = 1 + 20t + 600t 2 /2! + 24000t 3 /3! +
[OR use the result on E[X r ] for a gamma(2,0.1) variable in the Yellow Book]
