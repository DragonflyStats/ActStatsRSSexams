
\documentclass[a4paper,12pt]{article}

%%%%%%%%%%%%%%%%%%%%%%%%%%%%%%%%%%%%%%%%%%%%%%%%%%%%%%%%%%%%%%%%%%%%%%%%%%%%%%%%%%%%%%%%%%%%%%%%%%%%%%%%%%%%%%%%%%%%%%%%%%%%%%%%%%%%%%%%%%%%%%%%%%%%%%%%%%%%%%%%%%%%%%%%%%%%%%%%%%%%%%%%%%%%%%%%%%%%%%%%%%%%%%%%%%%%%%%%%%%%%%%%%%%%%%%%%%%%%%%%%%%%%%%%%%%%

\usepackage{eurosym}
\usepackage{vmargin}
\usepackage{amsmath}
\usepackage{graphics}
\usepackage{epsfig}
\usepackage{enumerate}
\usepackage{multicol}
\usepackage{subfigure}
\usepackage{fancyhdr}
\usepackage{listings}
\usepackage{framed}
\usepackage{graphicx}
\usepackage{amsmath}
\usepackage{chngpage}

%\usepackage{bigints}
\usepackage{vmargin}

% left top textwidth textheight headheight

% headsep footheight footskip

\setmargins{2.0cm}{2.5cm}{16 cm}{22cm}{0.5cm}{0cm}{1cm}{1cm}

\renewcommand{\baselinestretch}{1.3}

\setcounter{MaxMatrixCols}{10}

\begin{document}

%%%%%%%%%%%%%%%%%%%%%

\begin{enumerate}


12
A certain type of insurance policy has a claim rate of per year and the cover ceases and the policy expires after the first claim. Accordingly the duration of a policy is modelled by an exponential distribution with density function e x : 0 x
.
A company has data on (m + n) policies which have expired and which may beassumed to be independent. Of these, m policies had duration less than 5 years and n policies had duration greater than or equal to 5 years.
\item (i)
An investigator makes note of the actual durations, x 1 , , x n , of the latter group of n policies, but ignores the former group without even noting the value of m.
(a)
Explain why the x i s come from a truncated exponential distribution
with density function
f ( x ) = k . e
, 5
x
e 5 .
and show that k
(b)
x
Write down the likelihood for the data from the point of view of this investigator and hence show that the maximum likelihood estimate
(MLE) of is given by
n
n
.
x i 5 n
i 1
(c)
The data yield the values: n = 10 and x i = 71. Calculate this investigator s MLE of .
%%%%%%%%%%%%%%%%%%%%%%%%%%%%%%%%%%%%%%%%%%%%%%%%%%%%%%%%%%

\item %- \item (ii)
A second investigator ignores the actual policy durations and simply notes the values of m and n.
(a)
Write down the likelihood for this information and hence show that the resulting MLE of is given by
=
(b)
1
m n
log
.
5
n
The same data as in part \item (i) yield the values: $m = 120$ and $n = 10$.
Calculate this investigator s MLE of .
%%%%
\item % \item (iii)
The two investigators decide to pool their data, and so have the information that there are $m$ policies with duration less than 5 years, and n policies with actual durations $x 1 , ... , x n$ .
(a)
Explain why the likelihood for this joint information is given by
\[L ( ) = (1 e
5
) m .
n
e
x i
i 1\]
and determine an equation, the solution of which will lead to the MLE
of .
(b)
Given that this leads to an MLE of comparison of the three MLE s.
equal to 0.508, comment on the

%%%%%%%%%%%%%%%%%%%%%%%%%%%%%%%%%%%%%%%%%%%%%%%%%%%%%%%%%%%%%%%%%%%%%%%%%%%%%%%%%%%%%%%%%%%%5
\newpage

12
(b) 0.6221
0.1472
0.9862
\item (i) (a)
n = 1
n = 0
n = 3
The x i s are known to be such that x i is a scaled form of e
x
5 , therefore have density which
for 5 < x < .
The scaling constant k is such that
k . e
x
dx 1
5
k [ e
x
] 5
1
k . e
5
1
k
e 5
[Note: this can be argued in other ways; e.g. by referring to a conditional density and dividing by P(X > 5)]

n
(b)
x i
e 5 e
L ( )
n 5 n
e
September 2005
Examiners Report
x i
e
i 1
log L ( )
n log
d
log L ( )
d
5 n
n
x i
5 n
x i
n
equate to zero for MLE
n
x i 5 n
i 1
[OR It could be noted that X 5 ~ exp( ) and that the MLE is
therefore the reciprocal of the mean of the data ( x i 5) giving the required answer]
\item (ii)
(c) n = 10, x i = 71
(a) L ( ) (1 e
log L ( )
5
10
71 50
) m ( e
m log(1 e
d
5 me
log L ( )
d
1 e
5
5
5
n
m n
) n
) n log( e
5
)
5
5
equate to zero for MLE
e
0.476
5 n
e
5
1 e
5
n
m
1
m n
log(
)
5
n
[OR Reason via the MLE for a binomial p = P(X > 5) such that
n
p
and p e 5 ]
m n
(b)
Page 6
m = 120, n = 10
0.513
%%%%%%%%%%%%%%%%%%%%%%%%%%%%%%%%%%%%%%%%%%%%%%%%%%5
\item (iii)
(a)
(1 e
5
n
e

) m is the likelihood of observing m policies with duration < 5
x i
is the likelihood of observing the actual durations x 1 ,
, x n
i 1
and independence leads to the product of these
log L ( )
m log(1 e
d
5 me
log L ( )
d
1 e
5
5
) n log
n
5
x i
x i
equate to zero and the solution gives the MLE.
(b)
All three are re-assuringly close.
The pooled estimate is between the first two (as expected, but it is closer to 0.513).
\ned{document}
