\documentclass[a4paper,12pt]{article}

%%%%%%%%%%%%%%%%%%%%%%%%%%%%%%%%%%%%%%%%%%%%%%%%%%%%%%%%%%%%%%%%%%%%%%%%%%%%%%%%%%%%%%%%%%%%%%%%%%%%%%%%%%%%%%%%%%%%%%%%%%%%%%%%%%%%%%%%%%%%%%%%%%%%%%%%%%%%%%%%%%%%%%%%%%%%%%%%%%%%%%%%%%%%%%%%%%%%%%%%%%%%%%%%%%%%%%%%%%%%%%%%%%%%%%%%%%%%%%%%%%%%%%%%%%%%

\usepackage{eurosym}
\usepackage{vmargin}
\usepackage{amsmath}
\usepackage{graphics}
\usepackage{epsfig}
\usepackage{enumerate}
\usepackage{multicol}
\usepackage{subfigure}
\usepackage{fancyhdr}
\usepackage{listings}
\usepackage{framed}
\usepackage{graphicx}
\usepackage{amsmath}
\usepackage{chngpage}

%\usepackage{bigints}
\usepackage{vmargin}

% left top textwidth textheight headheight

% headsep footheight footskip

\setmargins{2.0cm}{2.5cm}{16 cm}{22cm}{0.5cm}{0cm}{1cm}{1cm}

\renewcommand{\baselinestretch}{1.3}

\setcounter{MaxMatrixCols}{10}

\begin{document}

%%-- Question 8

The distribution of claim size under a certain class of policy is modelled as a normal random variable, and previous years records indicate that the standard deviation is £120.

\begin{enumerate}[(i)]
\item Calculate the width of a 95\% confidence interval for the mean claim size if a sample of size 100 is available.

\item Determine the minimum sample size required to ensure that a 95\% confidence interval for the mean claim size is of width at most \$10 .

\item Comment briefly on the comparison of the confidence intervals in (i) and (ii) with respect to widths and sample sizes used.
\end{enumerate}

%%%%%%%%%%%%%%%%%%%%%%%%%%%%%%%%%%%%%%%%%%%%%%%%%%%%%%%%%%%%%%%%%%%%%%%%%%%%%%%%%
\newpage 
Let $X_1 , \ldots , X_n$ denote a large random sample from a distribution with unknown population mean and known standard deviation 3. The null hypothesis H 0 : = 1 is to be tested against the alternative hypothesis H 1 : > 1, using a test based on the
sample mean with a critical region of the form X
k , for a constant k.

It is required that the probability of rejecting $H_0$ when = 0.8 should be approximately 0.05, and the probability of not rejecting H 0 when = 1.2 should be approximately 0.1.

\begin{itemize}
\item (i)
Show that the test requires
k 0.8
3/ n
where
\item (ii)
0.95 and
k 1.2
3/ n
0.10
is the standard normal distribution function.

\item The values for the sample size n and the critical value k which satisfy the requirements of part(i) are n = 482 and k = 1.025 (you are not asked to verify these values).

\item Calculate the approximate level of significance of the test, and comment on the value.
\end{itemize}
%%%%%%%%%%%%%%%%%%%%%%%%%%%%%%%%%%%%%%%%%%%%%%%%%%%%%%%%%%%%%%

\begin{itemize}
\item (i)
s.d. of the total claim amount = £7,071
Width of 95\% confidence interval: 1.96
120
120
[or 2 1.96
n
n
3.92
120
]
n
120
23.52
100
120
[or 2 1.96
£47.04]
100
1.96
\item (ii)
For the width of a 95\% confidence interval to be at most
120
1.96
10
n
n
1.96 120
10
23.52 , n
10 we require
553.19
i.e., take the sample size as 554.
%%%%%%%%%%%%%%%%%%%%%%%%%%%%%%%%%%%%%%%%%%%%%%%%%%%%%%%%%%%%%%%%%%%%%%%%%%%%%%%%%%%%%%%%%%%%%%%%%%%5
9
\item (iii) The confidence interval in  part (ii) in narrower
much larger sample size.
\item (i) X approx
0.05
N
,
3 2
n
%%%%%%%%%%%%%%%%%%%%%%%%%%%%%%%%%%%%%%%%%%%%%%%%%%%%%%%%%%%%%%%%%%%%%%%%%%%%%%%%%%%%%%%%%%%%%%%%%%%%%%%%%%
to achieve this we require a
for large n by the central limit theorem.
P(reject H 0 | = 0.8) = P( X > k| = 0.8)
k 0.8
3/ n
= 1
k 0.8
3/ n
0.95
and
0.1
P(do not reject H 0 | = 1.2)
= P ( X
1.2)
k 1.2
3/ n
=
\item (ii)
k |
Significance level
1
= P(reject H 0 when H 0 is true) = P ( X
1.025 1
3 / 482
1
k |
1)
(0.18) 1 0.57 0.43.
The significance level of the test is very high (43%).
\end{itemize}
\end{document}
