
\documentclass[a4paper,12pt]{article}

%%%%%%%%%%%%%%%%%%%%%%%%%%%%%%%%%%%%%%%%%%%%%%%%%%%%%%%%%%%%%%%%%%%%%%%%%%%%%%%%%%%%%%%%%%%%%%%%%%%%%%%%%%%%%%%%%%%%%%%%%%%%%%%%%%%%%%%%%%%%%%%%%%%%%%%%%%%%%%%%%%%%%%%%%%%%%%%%%%%%%%%%%%%%%%%%%%%%%%%%%%%%%%%%%%%%%%%%%%%%%%%%%%%%%%%%%%%%%%%%%%%%%%%%%%%%

\usepackage{eurosym}
\usepackage{vmargin}
\usepackage{amsmath}
\usepackage{graphics}
\usepackage{epsfig}
\usepackage{enumerate}
\usepackage{multicol}
\usepackage{subfigure}
\usepackage{fancyhdr}
\usepackage{listings}
\usepackage{framed}
\usepackage{graphicx}
\usepackage{amsmath}
\usepackage{chngpage}

%\usepackage{bigints}
\usepackage{vmargin}

% left top textwidth textheight headheight

% headsep footheight footskip

\setmargins{2.0cm}{2.5cm}{16 cm}{22cm}{0.5cm}{0cm}{1cm}{1cm}

\renewcommand{\baselinestretch}{1.3}

\setcounter{MaxMatrixCols}{10}

\begin{document}

%%%%%%%%%%%%%%%%%%%%%

3 Claim amounts on a certain type of policy are modelled as following a gamma
distribution with parameters = 120 and = 1.2.
Calculate an approximate value for the probability that an individual claim amount
exceeds 120, giving a reason for the approach you use.

\end{enumerate}
%%%%%%%%%%%%%%%%%%%%%%%%%%%%%%%%%%%%%%%%%%%%%%%%%%%%%%%%%%%%%%%%%%%%%%%%%%%%%%%%%%%%%%%%%%%
Gamma(120, 1.2) has mean
115.5 123.75
68.0625
1
( 1)
120
120
100 and variance
1.2
1.2 2
(1) 0.841
83.333
X N (100,9.129 2 ) by the Central Limit theorem (since the gamma variable is the
sum of 120 independent gamma(1,1.2) variables)
P X
4
120
P Z
120 100
9.129
2.191
1 0.98578 0.0142


\end{document}
\begin{framed}
Mean 

\[\displaystyle \operatorname {E} [X]={\frac {\alpha }{\beta }}}  \]


Variance
\[{\displaystyle \operatorname {Var} (X)={\frac {\alpha }{\beta ^{2}}}}  \]

\end{framed}

A random variable X that is gamma-distributed with shape α and rate \beta is denoted 
\[ {\displaystyle X\sim \Gamma (\alpha ,\beta )\equiv \operatorname {Gamma} (\alpha ,\beta )}  {\displaystyle X\sim \Gamma (\alpha ,\beta )\equiv \operatorname {Gamma} (\alpha ,\beta )}\]




a shape parameter $\alpha$
a rate parameter $\beta$



For large k the gamma distribution converges to normal distribution with mean μ = kθ and variance σ2 = kθ2.

\end{document}
