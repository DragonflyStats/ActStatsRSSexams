
\documentclass[a4paper,12pt]{article}

%%%%%%%%%%%%%%%%%%%%%%%%%%%%%%%%%%%%%%%%%%%%%%%%%%%%%%%%%%%%%%%%%%%%%%%%%%%%%%%%%%%%%%%%%%%%%%%%%%%%%%%%%%%%%%%%%%%%%%%%%%%%%%%%%%%%%%%%%%%%%%%%%%%%%%%%%%%%%%%%%%%%%%%%%%%%%%%%%%%%%%%%%%%%%%%%%%%%%%%%%%%%%%%%%%%%%%%%%%%%%%%%%%%%%%%%%%%%%%%%%%%%%%%%%%%%

\usepackage{eurosym}
\usepackage{vmargin}
\usepackage{amsmath}
\usepackage{graphics}
\usepackage{epsfig}
\usepackage{enumerate}
\usepackage{multicol}
\usepackage{subfigure}
\usepackage{fancyhdr}
\usepackage{listings}
\usepackage{framed}
\usepackage{graphicx}
\usepackage{amsmath}
\usepackage{chngpage}

%\usepackage{bigints}
\usepackage{vmargin}

% left top textwidth textheight headheight

% headsep footheight footskip

\setmargins{2.0cm}{2.5cm}{16 cm}{22cm}{0.5cm}{0cm}{1cm}{1cm}

\renewcommand{\baselinestretch}{1.3}

\setcounter{MaxMatrixCols}{10}

\begin{document}

27
A sample of 20 claim amounts (\$) on a group of household policies gave the
following data summaries:
x = 3,256 and
8
x 2 = 866,600.
\begin{enumerate}
\item (a) Calculate the sample mean and standard deviation for these claim amounts.
\item (b) Comment on the skewness of the distribution of these claim amounts, giving reasons for your answer.
\end{enumerate}

%%%%%%%%%%%%%%%%%%%%%%%%%%%%%%%%%%%%%%%%%%%%%%%%%%%%%%%%%%%%
7
(a)
x
s
8
2
1
(3256) 162.8
20
1
3256 2
866600
19
20
17711.7
s 133.1
(b) Distribution must have strong positive skewness
since the s.d. is large relative to the mean and the amounts must be positive.
(i) Using C for a claim, N for no claim, then
\begin{itemize}
\item P(premium = 400) = P(C in year 3, regardless of the first 2 years) = p
\item P(premium = 400k) = P(CN in years 2/3, regardless of the first year) = p(1 p)
\item P(premium = 400k 2 ) = P(NN in years 2/3, regardless of the first year) = (1 p) 2
\end{itemize}

[These probabilities may be derived in other ways, such as via a tree diagram]

%%%%%%%%%%%%%%%%%%%%%%%%%%%%%%%%%%%%%%%%%%%%%%%%%%%%%%%%%%%%%%%%%%%%%%%%%%%%%%%%%%%%%%%%%%%%%


(ii) E(premium) = 400.p + 400k.p(1 p) + 400k 2 .(1
= 400{p + kp(1 p) + k 2 (1 p) 2 }
(iii) For E(premium) = 300 when p = 0.1
then 0.1 + 0.09k + 0.81k 2 = 0.75
0.81k 2 + 0.09k
0.09 2 4(0.81)(0.65)
1.62
0.09
k
k
9

%%%%%%%%%%%%%%%%%%%%%%%%%%%%%%%%%%%%%%%%%%%%%%%%%%%%%%%%%%%%%%%%%%%%%%%%%%%%%%%%%%%%%%%%%%%%%
p) 2
0.09 1.454
1.62
0.65 = 0
for 0 k 1
0.84
Pooled estimate of common population variance =
10 59
14 42
10 14
49.0833
t 24 (0.025) = 2.064
95\%  CI for
1
124 105
10
2
is given by
2.064
49.0833
1 1
11 15
1/ 2
i.e. 19 5.74 i.e. 13.3 , 24.7
\end{document}
