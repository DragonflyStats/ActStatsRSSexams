\documentclass[a4paper,12pt]{article}



%%%%%%%%%%%%%%%%%%%%%%%%%%%%%%%%%%%%%%%%%%%%%%%%%%%%%%%%%%%%%%%%%%%%%%%%%%%%%%%%%%%%%%%%%%%%%%%%%%%%%%%%%%%%%%%%%%%%%%%%%%%%%%%%%%%%%%%%%%%%%%%%%%%%%%%%%%%%%%%%%%%%%%%%%%%%%%%%%%%%%%%%%%%%%%%%%%%%%%%%%%%%%%%%%%%%%%%%%%%%%%%%%%%%%%%%%%%%%%%%%%%%%%%%%%%%



\usepackage{eurosym}

\usepackage{vmargin}

\usepackage{amsmath}
\usepackage{graphics}
\usepackage{epsfig}
\usepackage{enumerate}
\usepackage{multicol}
\usepackage{subfigure}
\usepackage{fancyhdr}
\usepackage{listings}
\usepackage{framed}
\usepackage{graphicx}
\usepackage{amsmath}
\usepackage{chngpage}

%\usepackage{bigints}

\usepackage{vmargin}

% left top textwidth textheight headheight

% headsep footheight footskip

\setmargins{2.0cm}{2.5cm}{16 cm}{22cm}{0.5cm}{0cm}{1cm}{1cm}
\renewcommand{\baselinestretch}{1.3}
\setcounter{MaxMatrixCols}{10}
\begin{document}


%%%%%%%%%%%%%%%%%%%%%%%%%%%%%%%%%%%%%%%%%%%%%%%%%%%%%%%%%%%%%%%%%%%%%%%%%%%%%%%%%

A simple procedure for incorporating a no claims discount into an annual insurance policy is as follows:
a premium of \$400 is payable for the first year;
\begin{itemize}
\item if no claim is made in the first year, the premium for the second year is \$400k,
where k is a constant such that 0 < k < 1;
\item if no claim is made in the first and second years, the premium for the third
year is \$400k 2 ;
\item if no subsequent claims are made in future years, the premium remains as
\$400k 2 ;
\item if a claim is made, the premium the following year reverts to \$400 and the
procedure starts again as above.
\end{itemize}
%%%%%%%%%%%%%%%%%%%%%%%%%

\begin{enumerate}
    \item (i)
Show that the probability distribution of the premium for the fourth year, that is, for the year following the third year, is given by
\$400
with probability
\$400k with probability
\$400k 2 with probability
p
p(1 p)
(1 p) 2
where p is the probability of a claim being made in any year.

\item (ii) Obtain an expression for the expected premium for the fourth year under this procedure.
\item (iii) If it is desired that this expected premium should equal \$300, determine the required value of $k$ for the case where $p = 0.1$.

\end{enumerate}


%%%%%%%%%%%%%%%%%%%%%%%%%%%%%%%%%%%%%%%%%%%%%%%%%%%%%%%%%%%%%%%%%%%%%%%%%%%%%%%%%%%%%%%%%%%5
(i) Using C for a claim, N for no claim, then
\begin{itemize}
\item P(premium = 400) = P(C in year 3, regardless of the first 2 years) = p
\item P(premium = 400k) = P(CN in years 2/3, regardless of the first year) = p(1 p)
\item P(premium = 400k 2 ) = P(NN in years 2/3, regardless of the first year) = (1 p) 2
\end{itemize}
[These probabilities may be derived in other ways, such as via a tree diagram]
%%%%%%%%%%%%%%%%%%%%%%%%%%%%%%%%%%%%%%%%%%%%%%%%%%%%%%%%%%%%%%%%%%%%%%%%%%%%%%%%%%%%%

(ii) E(premium) = 400.p + 400k.p(1 p) + 400k 2 .(1
= 400{p + kp(1 p) + k 2 (1 p) 2 }
(iii) For E(premium) = 300 when p = 0.1
then 0.1 + 0.09k + 0.81k 2 = 0.75
0.81k 2 + 0.09k
0.09 2 4(0.81)(0.65)
1.62
0.09
k
k
9
%%-- September 2005-- Examiners Report
p) 2
0.09 1.454
1.62
0.65 = 0
for 0 k 1
0.84

%%%%%%%%%%%%%%%%%%%%%%%%%%%%%%%%%%%%%%%%%%%%%%%%%%%%%%%%%%%%%%%%%%%%%%%%%%%%%
\end{document}
