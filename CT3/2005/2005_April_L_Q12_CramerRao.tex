\documentclass[a4paper,12pt]{article}

%%%%%%%%%%%%%%%%%%%%%%%%%%%%%%%%%%%%%%%%%%%%%%%%%%%%%%%%%%%%%%%%%%%%%%%%%%%%%%%%%%%%%%%%%%%%%%%%%%%%%%%%%%%%%%%%%%%%%%%%%%%%%%%%%%%%%%%%%%%%%%%%%%%%%%%%%%%%%%%%%%%%%%%%%%%%%%%%%%%%%%%%%%%%%%%%%%%%%%%%%%%%%%%%%%%%%%%%%%%%%%%%%%%%%%%%%%%%%%%%%%%%%%%%%%%%

\usepackage{eurosym}
\usepackage{vmargin}
\usepackage{amsmath}
\usepackage{graphics}
\usepackage{epsfig}
\usepackage{enumerate}
\usepackage{multicol}
\usepackage{subfigure}
\usepackage{fancyhdr}
\usepackage{listings}
\usepackage{framed}
\usepackage{graphicx}
\usepackage{amsmath}
\usepackage{chngpage}

%\usepackage{bigints}
\usepackage{vmargin}

% left top textwidth textheight headheight

% headsep footheight footskip

\setmargins{2.0cm}{2.5cm}{16 cm}{22cm}{0.5cm}{0cm}{1cm}{1cm}

\renewcommand{\baselinestretch}{1.3}

\setcounter{MaxMatrixCols}{10}

\begin{document}

%%%%%%%%%%%%%%%%%%%%%%%%%%%%%%%%%%%%%%%%%%%%%%%%%%%%%%%%%%%%%%%%%%%%%%%%%%%%%%%%%
%%--- Question 12
(i)
 A random variable $Y$ has a Poisson distribution with parameter but there is a restriction that zero counts cannot occur. The distribution of $Y$ in this case is referred to as the zero-truncated Poisson distribution.
\begin{enumerate}[(a)]
\item
Show that the probability function of Y is given by
y
p ( y ) =
(ii)
e
y !(1 e )
( y = 1, 2, ).
(b) Show that E [ Y ] = /(1 e ).

(a) Let y 1 , , y n denote a random sample from the zero-truncated Poisson distribution.
Show that the maximum likelihood estimate of by the solution to the following equation:
y
e
may be determined
= 0,
1 e
and deduce that the maximum likelihood estimate is the same as the method of moments estimate.
(b)
\item (iii)
Obtain an expression for the Cramer-Rao lower bound (CRlb) for the variance of an unbiased estimator of .

The following table gives the numbers of occupants in 2,423 cars observed on a road junction during a certain time period on a weekday morning.

Number of occupants
Frequency of cars
1
1,486
2
694
3
195
4
37
5
10
6
1
The above data were modelled by a zero-truncated Poisson distribution as given in (i).
The maximum likelihood estimate of is = 0.8925 and the Cramer-Rao lower bound on variance at = 0.8925 is 5.711574 10 4 (you do not need to verify these results.)
%%%%%%%%%%%%%%%%%%%%%%%%%%%%%%%%%%%%%%%%%%%%%%%%%%%%%%%%%%%%%%%%%%
\item (a) Obtain the expected frequencies for the fitted model, and use a $\chi^2$ goodness-of-fit test to show that the model is appropriate for the data.
\item (b) Calculate an approximate 95\% confidence interval for and hence calculate a 95\% confidence interval for the mean of the zero-truncated Poisson distribution.
\end{enumerate}
\newpage
%%%%%%%%%%%%%%%%%%%%%%%%%%%%%%%%%%%%%%%%%%%%%%%%%%%%%%%%%%%%%%%%%%%

12
\begin{itemize}
\item (i)
(a)
The probability function for the zero-truncated Poisson distribution is given by
P ( Y
y | Y
P ( Y
0)
y and Y
P ( Y 0)
y
e
y !(1 P ( Y
y
0))
e
( y 1, 2, ).
y !(1 e )
\item (b)
Expectation of Y:
y
E [ Y ]
y
y 1
e
y !(1 e )
z
(1 e )
(1 e
Page 6
)
z 0
[1]
e
z !
( z
.
(1 e
)
y 1)
0)%%%%%%%%%%%%%%%%%%%%%%%%%%%%%%%%%%%5
\item (ii)
(a)
The log likelihood function for
%%--- April 2005
%%%%%%%%%%%%%%%%%%%%%%%%%%%%%%%%%%%%
is:
n
log L ( )
y i log
n
n log(1 e )
constant
i 1
d log L ( )
d
ny
n n
e
1 e
the ML estimate is determined by the solution of the equation
d log L ( )
d
0
e
y
0
1 e
As this equation may be rewritten as
y
and E [ Y ]
1 e
1 e
the ML estimate is the same as the method of moments estimate.
(b)
d 2
d
2
ny
log L ( )
n
2
and since E [ Y ] E [ Y ]
e
(1 e ) 2
(1 e )
, the Cramer-Rao lower bound is
given by,
1
CR lb
E
or
d
d
2
log L ( )
(1 e ) 2
n (1 e
1
2
e )
n
1
(1 e )
e
(1 e ) 2
.
%%%%%%%%%%%%%%%%%%%%%%%%%%%%%%%%%%%%%%%%%%%%%%%%%%%%%%%%%%%%%%%%%%%%%%%%%%%%%%%%%%%%%%%%%%%%%%%%5
\item (iii)
(a)
April 2005
%%%%%%%%%%%%%%%%%%%%%%%%%%%%%%%%%%%%
The expected frequencies for the fitted zero-truncated Poisson model are given by
y
n
e
( y 1, 2, ) where
0.8925 and n
2423
y !(1 e )
y
e i
f i
1
1500.48
1486
2
669.59
694
3
199.20
195
4
44.45
37
( f i e i ) 2
(1486 1500.48) 2
=
e i
1500.48
(on 4 df).
2
5
7.93
10
6
1.35
1
Total
2423.00
2423
(1 1.35) 2
= 2.99
1.35
\item 
The Yellow Book gives that the probability value is greater than 50\%, therefore there is no evidence to reject the null hypothesis, i.e. the model seems appropriate for the data.
%%%%%%%%%%%%%%%%%%%%%%%%%%%%%%%%%%%%%%%%%%%%%%%%%%%%%%%%%%%%%%%%%%%%%%%
[OR
(b)
2
= 2.68 on 3 df if
5 combined rather than 6.]
As approx. ~ N ( , CR lb) for large n, a 95\%  confidence interval for
is given by
1.96 CR lb
0.8925 1.96 5.711574 10
at
4
, since CR lb
5.711574
= 0.8925,
= 0.8925 1.96(0.0238989) = 0.8925
= (0.84566, 0.93934)
0.046847
\item 
Then the 95\%  confidence interval for the mean of Y ,
by
0.84566
1 e
Page 8
10 -4
0.84566
,
0.93934
1 e
0.93934
= (1.48, 1.54).
1
, is given
1 e
\end{itemize}
\end{document}
