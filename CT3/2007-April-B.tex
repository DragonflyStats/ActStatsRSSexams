4
The sample correlation coefficient for the set of data consisting of the three pairs of
values
(−1,−2) , (0,0) , (1,1)
is 0.982. After the x and y values have been transformed by particular linear functions,
the data become:
(2,2) , (6,−4) , (10,−7).
State (or calculate) the correlation coefficient for the transformed data.
CT3 A2007—2
[2]5
The number of claims arising in one year from a group of policies follows a Poisson
distribution with mean 12. The claim sizes independently follow an exponential
distribution with mean £80 and they are independent of the number of claims.
The current financial year has six months remaining.
Calculate the mean and the standard deviation of the total claim amount which arises
during this remaining six months.
[4]
6
Consider the discrete random variable X with probability function
f ( x ) =
(i)
4
5 x + 1
,
x = 0, 1, 2, ...
Show that the moment generating function of the distribution of X is given by
M X ( t ) = 4(5 − e t ) − 1 ,
for e t < 5.
(ii)
7
[3]
Determine E[X] using the moment generating function given in part (i).
[3]


%%%%%%%%%%%%%%%%%%%%%%%%%%%%
6
(i)
M X ( t ) = E [ e tx ]
∞
4 ⎛ 1 ⎞
4 ∞
= ∑ e
=
∑
⎜ ⎟
5 ⎝ 5 ⎠
5 x = 0
x = 0
x
tx
⎛ e t
⎜ ⎜
⎝ 5
x
⎞
⎟ ⎟ ,
⎠
and for e t < 5 ,
M X ( t ) =
(ii)
4 1
5 1 − e t
(
(
= 4 5 − e t
)
− 1
.
5
M '( t ) = 4 e t 5 − e t
)
− 2
Mean is given by E ( X ) = M '(0)
(
∴ E [ X ] = 4 e 0 5 − e 0
)
− 2
=
1
.
4
[OR, by expansion as a power series.]
Page 3Subject CT3 (Probability and Mathematical Statistics Core Technical) — April 2007 — Examiners’ Report
7
(i)
Let X = the sum repaid for a single certificate.
E ( X ) = 10(0.99) + 20(0.01) = 10.1
E ( X 2 ) = 10 2 (0.99) + 20 2 (0.01) = 103
∴ V ( X ) = 103 − 10.1 2 = 0.99 ∴ sd ( X ) = 0.9950
(ii)
Let S = the sum repaid for 200 certificates.
∴ E ( S ) = 200(10.1) = 2020, V ( S ) = 200(0.99) = 198 ∴ sd ( X ) = 14.07
P ( S > 2040) = P ( Z >
2040 − 2020
= 1.42)
14.07
= 1 − 0.9222 = 0.0778
(iii)
N ~ binomial(200, 0.01) ≈ Poisson(2)
P ( S > 2040) = P ( N > 4)
= 1 − P ( N ≤ 4) = 1 − 0.94735 = 0.0527
(iv)
Clearly the Poisson approximation to the binomial is better than the Central
Limit Theorem approximation.
OR:
Since S is discrete and increases in steps of 10, one can argue for the use of a
continuity correction in (ii) above:
2045 − 2020 ⎞
⎛
P ( S > 2040 ) = P ⎜ Z ≥
⎟ = P ( Z > 1.78 )
14.07
⎝
⎠
= 1 − 0.96246 = 0.0375
(Either approach is acceptable for the marks.)
Page 4Subject CT3 (Probability and Mathematical Statistics Core Technical) — April 2007 — Examiners’ Report
∞
