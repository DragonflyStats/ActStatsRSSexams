\documentclass[a4paper,12pt]{article}

%%%%%%%%%%%%%%%%%%%%%%%%%%%%%%%%%%%%%%%%%%%%%%%%%%%%%%%%%%%%%%%%%%%%%%%%%%%%%%%%%%%%%%%%%%%%%%%%%%%%%%%%%%%%%%%%%%%%%%%%%%%%%%%%%%%%%%%%%%%%%%%%%%%%%%%%%%%%%%%%%%%%%%%%%%%%%%%%%%%%%%%%%%%%%%%%%%%%%%%%%%%%%%%%%%%%%%%%%%%%%%%%%%%%%%%%%%%%%%%%%%%%%%%%%%%%

\usepackage{eurosym}
\usepackage{vmargin}
\usepackage{amsmath}
\usepackage{graphics}
\usepackage{epsfig}
\usepackage{enumerate}
\usepackage{multicol}
\usepackage{subfigure}
\usepackage{fancyhdr}
\usepackage{listings}
\usepackage{framed}
\usepackage{graphicx}
\usepackage{amsmath}
\usepackage{chngpage}

%\usepackage{bigints}
\usepackage{vmargin}

% left top textwidth textheight headheight

% headsep footheight footskip

\setmargins{2.0cm}{2.5cm}{16 cm}{22cm}{0.5cm}{0cm}{1cm}{1cm}

\renewcommand{\baselinestretch}{1.3}

\setcounter{MaxMatrixCols}{10}

\begin{document}
\begin{enumerate}
%%%%%%%%%%%%%%%%%%%%%%%%%%%%%%%%%%%%%%%%%%%%%%%%%%%%%%%%%%%%%%%%%%%%%%%%%%%%%%%%%
PLEASE TURN OVER10
A model used for claim amounts (X, in units of £10,000) in certain circumstances has
the following probability density function, f(x), and cumulative distribution function,
F(x):
f ( x ) =
5(10 5 )
(10 x )
, x
6
0 ; F ( x ) = 1
10
10 x
5
.
You are given the information that the distribution of X has mean 2.5 units (£25,000)
and standard deviation 3.23 units (£32,300).
(i) Describe briefly the nature of a model for claim sizes for which the standard
deviation can be greater than the mean.
[2]
(ii) (a)
Show that we can obtain a simulated observation of X by calculating
x = 10 (1 r )
0.2
1
where r is an observation of a random variable which is uniformly
distributed on (0,1).
(b)
Explain why we can just as well use the formula
x = 10 r
0.2
1
to obtain a simulated observation of X.
(c)
Calculate the missing values for the simulated claim amounts in the
table below (which has been obtained using the method in (ii)(b)
above):
r Claim (£)
0.7423
0.0291
0.2770
0.5895
0.1131
0.9897
0.6875
0.8525
0.0016
0.5154 6,141
10,2872
29,272
11,148
54,635
207
7,782
3,243
?
?
[5]
[Total 7]
CT3 A2005
411
Twenty insects were used in an experiment to examine the effect on their activity
level, y, of 3 standard preparations of a chemical. The insects were randomly
assigned, 4 to receive each of the preparations and 8 to remain untreated as controls.
Their activity levels were metered from vibrations in a test chamber and were as
follows:
Activity levels (y)
Totals
Control
Preparation A
Preparation B
Preparation C 43
73
84
46
40
55
63
91
For these data y = 1, 203 ,
65
61
51
84
51
65
72
71
33
39
54
62
387
254
270
292
y 2 = 77, 249 .
(i) Conduct an analysis of variance test to establish whether the data indicate
significant differences amongst the results for the four treatments.
[7]
(ii) (a)
Complete the following table of residuals for the data and analysis in
part (i) above:
Control
Preparation A
Preparation B
Preparation C
(iii)
?
9.5
16.5
?
?
8.5
?
?
16.6
2.5
16.5
?
2.6
1.5
4.5
2
15.4
9.4
5.6 13.6
(b) Make a rough plot of the residuals against the treatment means.
(c) State the assumptions underlying the analysis of variance test
conducted in part (i).
(d) Comment on how well the data conform to these assumptions in the
light of the residual plot.
[8]
It is suggested that any differences can be explained in terms of a difference
between controls on the one hand and treated groups on the other.
Comment on any evidence for this and state how you would formally test for
this effect (but do not carry out the test).
[4]
[Total 19]
CT3 A2005
5



%%%%%%%%%%%%%%%%%%%%%%%%%%%%%%%%%%%%%%%%%%%%%%%%%%%%%%%%%%%%%%%%%%%%%%%%%%%%%%

10
(i) X takes positive values only so to have such a relatively high standard
deviation the distribution must be positively skewed with sizeable probability
associated with high values (i.e. the model embraces high claim sizes; the
density has a long or heavy tail).
(ii) (a) Solving r = F(x)
(b) R ~ U(0,1) 1 R ~ U(0, 1) so (1 r) is also a random number from
(0, 1), so we can use 1 r in place of r , giving the formula
x 10 r
(c)
Page 4
0.2
r = 0.0016
r = 0.5154
(1 + x/10) = (1 r)
1
claim = 262390
claim = 14175
0.2
x = 10[(1
r)
0.2
1]Subject CT3 (Probability and Mathematical Statistics Core Technical)
%%%%%%%%%%%%%%%%%%%%%%%%%%%%%%%%%%%%%%%%%%%%%%%5555
(i)
SS T = 77249 1203 2 /20 = 4888.55
SS B = 387 2 /8 + 254 2 /4 + 270 2 /4 + 292 2 /4
SS R = 2857.875
Examiners Report
1203 2 /20 = 2030.675
H 0 : no treatment effects (i.e. population means are equal) v H 1 : not H 0
Analysis of Variance
Source
DF
SS
Factor
3
2031
Error
16
2858
Total
19
4889
MS
677
179
F
3.79
F 3,16 (0.05) = 3.239, F 3,16 (0.01) = 5.292
P-value is lower than 0.05 (but higher than 0.01), so we can reject H 0 at least
at the 5% level of testing (Note: actually P-value is 0.032). The data do
indicate significant differences amongst the treatment means.
(ii)
(a)
Residual = observed value treatment mean
Treatment means are: Control 48.375, A 63.5, B 67.5, C 73.0
Missing values are:
Control
Preparation A
Preparation B
Preparation C
5.4
9.5
16.5
27
8.4
8.5
4.5
18
16.6
2.5
16.5
11
2.6
1.5
4.5
2
15.4
9.4
5.6
13.6
(b)
20
10
11
April 2005
0
-10
-20
-30
means
treatment
50
Control
60
70
A
B
C
Page 5Subject CT3 (Probability and Mathematical Statistics Core Technical)
(iii)
April 2005
Examiners Report
(c) Observations Y ij (j th value for treatment i) are independent and
normally distributed with variance 2 which is constant across
treatments.
(d) The assumptions seem reasonable
with the exception of the constant
variance assumption, which is questionable
the data for preparation
A appear to be less variable than the data for the other treatments.
The control mean is lower than all three treatment means
(48.4 v 63.5, 67.5, 73.0) so there is prima facie evidence to support the
suggestion.
One could perform a two-sample t-test of control mean = treatment mean
by combining the data for the 3 preparations (and using samples of sizes 8
and 12).
