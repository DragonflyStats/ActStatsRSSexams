An employment survey is carried out in order to determine the percentage, p, of
unemployed people in a certain population in a way such that the estimation has a
margin of error less than 0.5% with probability at least 0.95. In a similar study
conducted a year ago it was found that the percentage of unemployed people in the
population was 6%.
Calculate the sample size, n, that is required to achieve this margin of error, by
constructing an appropriate confidence interval (or otherwise).
8

%%%%%%%%%%%%%%%%%%%%%%%%%%%%%%%%%%%%%%%%%%%%%%%%%%%%%%%%%%%%%%%%%%%%%%%%%%
For a sample of 100 insurance policies the following frequency distribution gives the
number of policies, f, which resulted in x claims during the last year:
x:
f:
0
76
1
22
2
1
3
1
\begin{enumerate}{(a)]
\item 
Calculate the sample mean, standard deviation and coefficient of skewness for
these data on the number of claims per policy.
[4]
A Poisson model has been suggested as appropriate for the number of claims per
policy.
\item 
(a) State the value of the estimated parameter when a Poisson distribution
is fitted to these data using the method of maximum likelihood.
(b) Verify that the coefficient of skewness for the fitted model is 1.92, and
hence comment on the shape of the frequency distribution relative to
that of the corresponding fitted Poisson distribution.
\end{enumerate}

%%%%%%%%%%%%%%%%%%%%%%%%%%%%%%%%%%%%%%%%%%%%%%%%%%%%%%%%%%%%%%%%%%%%%%%%%%

7
The 95% CI for the population percentage p is
giving | p − p ˆ | ≤ 1.96
p ˆ ± 1.96
p ˆ (1 − p ˆ )
n
p ˆ (1 − p ˆ )
n
For the margin of error to be less than 0.5% we need to solve
0.005 = 1.96
1.96 2 p ˆ (1 − p ˆ )
p ˆ (1 − p ˆ )
⇒ n =
.
n
0.005 2
Using the percentage from the previous study as the value for p̂ , i.e. p ˆ = 0.06 , we
obtain n = 8,666.6.
So we need a sample of (at least) 8667 people.
⎛ p ( 1 − p ) ⎞
(OR, solution can be based on p ˆ ~ N ⎜ p ,
⎟ and
n
⎝
⎠
P ( − 0.005 < p ˆ − p < 0.005 ) > 0.95 without referring to the CI.)


%%%%%%%%%%%%%%%%%%%%%%%%%%%%%%%%%%%%%%%%%%%%%%%%%%%%%%%%%%%%%%%%%%%%%%%%%%%%%%%%%%%%%%
8
(i)
Σ f = 100, Σ fx = 27, Σ fx 2 = 35
x =
27
= 0.27
100
1
27 2
s = {35 −
} = 0.2799 ∴ s = 0.529
99
100
2
Third moment about mean is
m 3 =
1
{76(0 − 0.27) 3 + 22(1 − 0.27) 3 + (2 − 0.27) 3 + (3 − 0.27) 3 } = 0.3259
100
[OR: using Σ fx 3 = 57, m 3 =
1
{57 − 3(0.27)(35) + 2(100)(0.27) 3 } ]
100
So coefficient of skewness is
0.3259
(0.2799) 3/2
= 2.20
[OR: can use m 2 = 0.2771 in denominator to give 2.23 ]
(ii)
(a) μ ˆ = x = 0.27
(b) Coefficient of skewness is
1
= 1.92 (from book of formulae, p7)
0.27
so, the data distribution is slightly more positively skewed than the
fitted Poisson.

%%%%%%%%%%%%%%%%%%%%%%%%%%%%%%%%%%%%%%%%%%%%%%%%%%%%%%%%%%%%%%%%%%%%%%%%%%
\end{document}
