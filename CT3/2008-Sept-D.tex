[Total 6]
9
A random sample of four insurance policies of a certain type was examined for each
of three insurance companies and the sums insured were recorded. An analysis of
variance was then conducted to test the hypothesis that there are no differences in the
means of the sums insured under such policies by the three companies.
The total sum of squares was found to be SS T = 420.05 and the between-companies
sum of squares was found to be SS B = 337.32.
Perform the analysis of variance to test the above hypothesis and state your
conclusion.
[4]
(ii) State clearly any assumptions that you made in performing the analysis in (i).
[2]
(iii) The plot of the residuals of this analysis of variance against the associated
fitted values, is given below.
(i)
18
20
22
24
26
28
30
32
Fitted
Comment briefly on the validity of the test performed in (i), basing your
answer on the above plot.
[2]
[Total 8]
CT3 S2007—410
When a new claim comes into an office it is screened at a first stage and has a
probability  of being cleared for progress, otherwise it is rejected. If it clears the first
stage, it is then independently screened at a second stage and has the same probability
 of being cleared for progress, otherwise it is rejected.
(i) Explain clearly why the probability of a claim being rejected at the first stage
is 1 - , of being rejected at the second stage is  (1 - ) and of progressing
after the two stages is  2 .
[3]
(ii) For a sample of n independent claims which came into the office x 1 were
rejected at the first stage, x 2 were rejected at the second stage and x 3
progressed after the two stages (x 1 + x 2 + x 3 = n).
(a)
Write down the likelihood L() for this sample and hence show that the
derivative of the log-likelihood is given by
x  2 x 3 x 1  x 2

.
log L (  )  2

1  


(b)
Show that the maximum likelihood estimator (MLE) is given by
 ˆ 
x 2  2 x 3
.
x 1  2 x 2  2 x 3
[7]
(iii)
(iv)

2
(a) log L (  ) of the log-likelihood in
 2
part (ii) above and hence show that the Cramer-Rao lower bound
 (1   )
(CRlb) is given by
.
n (1   )
(b) Use the asymptotic distribution for the MLE ̂ with the CRlb
evaluated at ̂ to obtain an approximate large-sample 95% confidence
interval for  expressing it simply in terms of ̂ and n.
[7]
Determine the second derivative
For a sample of 1,000 independent claims, 110 were rejected at the first stage,
96 were rejected at the second stage and 794 progressed after the two stages.
Calculate the MLE ̂ together with an approximate 95% confidence interval
for .
[3]
[Total 20]
CT3 S2008—5
%%%%%%%%%%%%%%%%%%%%%%%%%%%%%%%%%%%%%%%%%%%%%%%%%%%%%%%%%%%%%%%%%%%%%%%%%%%%%%%%%%%%%%%%%
9
(i)
SS R = SS T – SS B = 420.05 – 337.32 = 82.73.
The degrees of freedom are 3 – 1 = 2 for the treatment (company) SS, and
12 – 1 – 2 = 9 for the residual SS.
Page 4Subject CT3 (Probability and Mathematical Statistics Core Technical) — September2008 — Examiners’ Report
These give F =
337.32 2
= 18.348.
82.73 9
From tables, F 0.01,2,9 = 8.022, and therefore we have strong evidence against
the hypothesis that the means of the insured sums are equal for the 3
companies.
10
(ii) To perform the ANOVA we assume that the data follow normal distributions
and that their variance is constant.
(iii) The variance of the residuals seems to depend on the company from which the
data come. This violates the assumption of constant variance in the response
variable, and therefore the analysis may not be valid.
(i) P(rejected at 1 st ) = 1 – P(cleared at 1 st ) = 1 – θ
P(rejected at 2 nd ) = P(cleared at 1 st )P(rejected at 2 nd | cleared at 1 st )
= θ (1 – θ)
P(progressing after two) = P(cleared at 1 st ) P(cleared at 2 nd ) = θ 2
(ii)
(a)
L ( θ ) = [(1 − θ )] x 1 [ θ (1 − θ )] x 2 [ θ 2 ] x 3
= θ x 2 + 2 x 3 (1 − θ ) x 1 + x 2
∴ log L ( θ ) = ( x 2 + 2 x 3 ) log θ + ( x 1 + x 2 ) log(1 − θ )
∴
(b)
x + 2 x 3 x 1 + x 2
∂
log L ( θ ) = 2
−
1 − θ
∂θ
θ
equate to zero for MLE
∴θ ( x 1 + x 2 ) = (1 − θ )( x 2 + 2 x 3 )
∴θ ( x 1 + 2 x 2 + 2 x 3 ) = x 2 + 2 x 3
x 2 + 2 x 3
x 1 + 2 x 2 + 2 x 3
∴θ ˆ =
(iii)
(a)
∂ 2 x 2 + 2 x 3
∂θ θ 2
log L ( θ ) = −
2
E {
−
x 1 + x 2
(1 − θ ) 2
∂ 2 n θ (1 − θ ) + 2 n θ 2
∂θ θ 2
log L ( θ )} = −
2
−
n (1 − θ ) + n θ (1 − θ )
(1 − θ ) 2
Page 5Subject CT3 (Probability and Mathematical Statistics Core Technical) — September 2008 — Examiners’ Report
n
n
n (1 + θ )
1
1
= − (1 + θ ) −
(1 + θ ) = − n (1 + θ )( +
) =−
θ
θ 1 − θ )
θ (1 − θ )
(1 − θ )
1
CRlb =
− E [
∂
log L ( θ )]
∂θ 2
2
=
θ (1 − θ )
n (1 + θ )
θ ˆ ≈ N ( θ , CRlb ) for large n
(b)
using CRlb =
θ ˆ (1 − θ ˆ )
θ ˆ (1 − θ ˆ )
, then θ ˆ ≈ N ( θ ,
)
n (1 + θ ˆ )
n (1 + θ ˆ )
θ ˆ (1 − θ ˆ )
95% CI is θ ˆ ± 1.96
n (1 + θ ˆ )
(iv)
θ ˆ =
96 + 2(794)
1684
=
= 0.8910
110 + 2(96) + 2(794) 1890
CRlb ≈
0.8910(1 − 0.8910)
= 0.0000514 ∴ CRlb = 0.00717
1000(1 + 0.8910)
∴95% CI is 0.8910 ± 1.96(0.00717)
⇒ 0.891 ± 0.014 or (0.877, 0.905)
\end{document}
