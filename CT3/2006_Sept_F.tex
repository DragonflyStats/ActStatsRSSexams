
\documentclass[a4paper,12pt]{article}

%%%%%%%%%%%%%%%%%%%%%%%%%%%%%%%%%%%%%%%%%%%%%%%%%%%%%%%%%%%%%%%%%%%%%%%%%%%%%%%%%%%%%%%%%%%%%%%%%%%%%%%%%%%%%%%%%%%%%%%%%%%%%%%%%%%%%%%%%%%%%%%%%%%%%%%%%%%%%%%%%%%%%%%%%%%%%%%%%%%%%%%%%%%%%%%%%%%%%%%%%%%%%%%%%%%%%%%%%%%%%%%%%%%%%%%%%%%%%%%%%%%%%%%%%%%%

\usepackage{eurosym}
\usepackage{vmargin}
\usepackage{amsmath}
\usepackage{graphics}
\usepackage{epsfig}
\usepackage{enumerate}
\usepackage{multicol}
\usepackage{subfigure}
\usepackage{fancyhdr}
\usepackage{listings}
\usepackage{framed}
\usepackage{graphicx}
\usepackage{amsmath}
\usepackage{chngpage}

%\usepackage{bigints}
\usepackage{vmargin}

% left top textwidth textheight headheight

% headsep footheight footskip

\setmargins{2.0cm}{2.5cm}{16 cm}{22cm}{0.5cm}{0cm}{1cm}{1cm}

\renewcommand{\baselinestretch}{1.3}

\setcounter{MaxMatrixCols}{10}

\begin{document}
\begin{enumerate}

CT S2006
612
A development engineer examined the relationship between the speed a vehicle is travelling (in miles per hour, mph), and the stopping distance (in metres, m) for a new
braking system fitted to the vehicle. The following data were obtained in a series of
independent tests conducted on a particular type of vehicle under identical road conditions.
Speed of vehicle (x):
Stopping distance (y):
x = 280
(i)
y = 241
10
5
20
10
x 2 = 14,000
0
2
40
4
50
40
y 2 = 11,951
60
54
70
75
xy = 12,790
Construct a scatterplot of the data, and comment on whether a linear
regression is appropriate to model the relationship between the stopping
distance and speed. [4]
(ii) Calculate the equation of the least-squares fitted regression line. [5]
(iii) Calculate a 95\% confidence interval for the slope of the underlying regression
line, and use this confidence interval to test the hypothesis that the slope of the
underlying regression line is equal to 1.
[5]
(iv) Use the fitted line obtained in part (ii) to calculate estimates of the stopping
distance for a vehicle travelling at 50 mph and for a vehicle travelling at 100
mph.
Comment briefly on the reliability of these estimates.
END OF PAPER
CT S2006
7
[4]
[Total 18]

%%%%%%%%%%%%%%%%%%%%%%%%%%%%%%%%%%%%%%%%%%%%%%%%%%%%%%%%%%%
12
(i)
Plot of stopping distance against speed
80
distance
70
60
50
40
0
20
10
0
10
20
0
40
50
60
70
Speed
There is a suggestion of a curve but linear regression might still be reasonable.
(ii)
n = 7
S xx = \sigma x 2 − (\sigma x) 2 /n = 14000 − (280) 2 /7 = 2800
S yy = \sigma_y 2 − (\sigma_y) 2 /n = 11951 − (241) 2 /7 = 65.714
S xy = \sigma xy − (\sigma x)( \sigma_y)/n = 12790 − (280)(241)/7 = 150
Model: E[Y] = \alpha + \betax
Slope: \hat{\beta} =
S xy
S xx
=
150
= 1.125
2800
Intercept: \alpha ˆ = y − \hat{\beta} x = 241/7 − (1.125)(280/7) = −10.571
The equation of the least-squares fitted regression line is:
Distance = −10.571 + 1.125 Speed
Page 12Subject CT  — September 2006 — 
%%%%%%%%%%%%%%%%%%%%%%%%%%%%%%%%%

(iii)
\sigma ˆ 2 =
S 2
1 ⎛
⎜ S yy − xy
n − 2 ⎜
S xx
⎝
s.e.( \betâ ) =
\sigma ˆ 2
=
S xx
⎞ 1 ⎛
(150) 2 ⎞
⎟ = ⎜ 65.714 −
⎟ = 21.99
⎟ 5 ⎜ ⎝
2800 ⎟ ⎠
⎠
21.99
= 0.0886
2800
95\% confidence interval for slope:
\hat{\beta} \pm t n − 2 (0.025) s.e.( \betâ )
(df = n − 2 = 5)
= 1.125 \pm (2.571)(0.0886) = 1.125 \pm 0.228 or (0.897, 1.5)
\beta = 1 is within this 95\% confidence interval, therefore we would not reject the
null hypothesis \beta = 1 at the 5% significance level.
(iv)
When x = 50: y = −10.571 + 1.125(50) = 45.7 m
When x = 100: y = −10.571 + 1.125(100) = 101.9 m
The stopping distance of 45.7 m when the speed is 50 mph can be regarded as
a reliable estimate as x = 50 is well within the range of the x data values.
However, the stopping distance for a speed of 100 mph may be unreliable as
x = 100 is outside the range of the data and involves extrapolation.
END OF 
%%%%%%%%%%%%%%%%%%%%%%%%%%%%%%%%%

Page 1
