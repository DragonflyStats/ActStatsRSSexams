\documentclass[a4paper,12pt]{article}

%%%%%%%%%%%%%%%%%%%%%%%%%%%%%%%%%%%%%%%%%%%%%%%%%%%%%%%%%%%%%%%%%%%%%%%%%%%%%%%%%%%%%%%%%%%%%%%%%%%%%%%%%%%%%%%%%%%%%%%%%%%%%%%%%%%%%%%%%%%%%%%%%%%%%%%%%%%%%%%%%%%%%%%%%%%%%%%%%%%%%%%%%%%%%%%%%%%%%%%%%%%%%%%%%%%%%%%%%%%%%%%%%%%%%%%%%%%%%%%%%%%%%%%%%%%%
  \usepackage{eurosym}
\usepackage{vmargin}
\usepackage{amsmath}
\usepackage{graphics}
\usepackage{epsfig}
\usepackage{enumerate}
\usepackage{multicol}
\usepackage{subfigure}
\usepackage{fancyhdr}
\usepackage{listings}
\usepackage{framed}
\usepackage{graphicx}
\usepackage{amsmath}
\usepackage{chngpage}
%\usepackage{bigints}
\usepackage{vmargin}

% left top textwidth textheight headheight

% headsep footheight footskip

\setmargins{2.0cm}{2.5cm}{16 cm}{22cm}{0.5cm}{0cm}{1cm}{1cm}

\renewcommand{\baselinestretch}{1.3}

\setcounter{MaxMatrixCols}{10}

\begin{document}

\begin{enumerate}

CT3 A2016–6
10 Consider a large portfolio of insurance policies and denote the claim size (in £) per
claim by X. A random sample of policies with a total of 20 claims is taken from this
portfolio and the claims made for these policies are reported in the following table:
  Claim i 1 2 3 4 5 6 7 8 9 10
Claim size xi 130 164 170 173 173 175 177 183 183 184
Claim i 11 12 13 14 15 16 17 18 19 20
Claim size xi 185 186 197 202 208 213 215 229 233 272
For these data: xi = 3,852 and 2
xi = 759,348.
(i) Calculate the mean, the median and the standard deviation of the claim size
per claim in this sample. [3]
(ii) Determine a 95% confidence interval for the expected value E[X] based on the
above random sample, stating any assumptions you make. [3]
(iii) Determine a 95% confidence interval for the standard deviation of X based on
the above random sample. [2]
(iv) Explain briefly why the confidence interval in part (iii) is not symmetric
around the estimated value of the standard deviation. [1]
An actuary assumes that the number of claims from each policy has a Poisson
distribution with an unknown parameter . In a new sample of 50 policies the actuary
has observed a total of 80 claims yielding an estimated value of ˆ 1.6 for the
parameter .
(v) Determine a 95% confidence interval for  using a normal approximation. [2]
(vi) Determine the smallest required sample size n for which a 95% confidence
interval for  has a width of less than 0.5. You should use the same normal
approximation as in part (v), and assume that the estimated value of  will not
change. [3]
Now assume that the true value of λ is 1.6 and the values calculated in part (i) are the
true values. Also assume that all claims in the portfolio are independent and the claim
sizes are independent of the number of claims.
(vii) Determine the expected value and the standard deviation of the total amount of
all claims from a portfolio of 5,000 insurance policies. [3]
[Total 17]
Page 9
Q10 (i) Mean: x = 3,852 / 20 =1 92.6 [1]
Median: (184+185)/2 = 184.5 [1]
SD: s = (759,348− 20*192.62 ) /19 = 30.31 [1]
(ii) 0.025,1 9
192.6 2.093 30.31, 1 92.6 2.093 30.31 [178.4, 206.8]
20 20 20
x t s  
± =  − +  =
   
[2]
We need to assume that the claim size is normally distributed. [1]
(iii)
2 2
2 2
0.025,19 0.975,19
19 , 19 30.31 19 ,30.31 19 [23.05, 44.27]
32.85 8.907
 s s   
  =   =
   χ χ   
[2]
(iv) Theχ2 distribution is not symmetric. [1]
(v) 1.6 1.96 1.6 ,1.6 1.96 1.6 [1.25,1 .95]
50 50
 
 − +  =
   
[2]
(vi) The length of the confidence interval is 2*1.96 1.6 1/ 2
n
< . Therefore,
1.6 1 2
n 4*1.96
<    
 
giving n >1.6*(4*1.96)2 = 98.3 [2]
So, n ≥ 99. [1]
(vii) We are now having a compound Poisson distribution for total S.
Expected number of claims from 5,000 policies:
  E(N) = 1.6 * 5,000 = 8,000
Expected value of total claim size:
  E(S) = E(N)*E(X) =8,000 * 192.6 = 1,540,800 [1]
V(S) = E(N)V(X) + V(N)[E(X)]2= 1.6 * 5,000 * (30.312 + 192.62)
= 304,107,649 [1]
Subject CT3 (Probability and Mathematical Statistics Core Technical) – April 2016 – Examiners’ Report
Page 10
SD = 17,438.68 [1]
[TOTAL 17]
Parts (i), (ii), (iii) and (v) were very well answered. Note that in part (ii) the
assumptions must be stated for full marks.
In part (iv) a common error was not mentioning the asymmetry of the
distribution, and part (vi) was tackled with mixed success. Part (vii) was not
particularly well answered – many candidates failed to treat this as a
compound distribution.

