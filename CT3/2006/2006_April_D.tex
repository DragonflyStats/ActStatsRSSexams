\documentclass[a4paper,12pt]{article}

%%%%%%%%%%%%%%%%%%%%%%%%%%%%%%%%%%%%%%%%%%%%%%%%%%%%%%%%%%%%%%%%%%%%%%%%%%%%%%%%%%%%%%%%%%%%%%%%%%%%%%%%%%%%%%%%%%%%%%%%%%%%%%%%%%%%%%%%%%%%%%%%%%%%%%%%%%%%%%%%%%%%%%%%%%%%%%%%%%%%%%%%%%%%%%%%%%%%%%%%%%%%%%%%%%%%%%%%%%%%%%%%%%%%%%%%%%%%%%%%%%%%%%%%%%%%

\usepackage{eurosym}
\usepackage{vmargin}
\usepackage{amsmath}
\usepackage{graphics}
\usepackage{epsfig}
\usepackage{enumerate}
\usepackage{multicol}
\usepackage{subfigure}
\usepackage{fancyhdr}
\usepackage{listings}
\usepackage{framed}
\usepackage{graphicx}
\usepackage{amsmath}
\usepackage{chngpage}

%\usepackage{bigints}
\usepackage{vmargin}

% left top textwidth textheight headheight

% headsep footheight footskip

\setmargins{2.0cm}{2.5cm}{16 cm}{22cm}{0.5cm}{0cm}{1cm}{1cm}

\renewcommand{\baselinestretch}{1.3}

\setcounter{MaxMatrixCols}{10}

\begin{document}
\begin{enumerate}
%%%%%%%%%%%%%%%%%%%%%%%%%%%%%%%%%%%%%%%%%%%%%%%%%%%%%%%%%%%%%%%%%%%%%%%%%%%%%%%%%
%[Total 6]10
\item A marketing consultant was commissioned to conduct a questionnaire survey of the clients of a financial company. The total number of respondents was 650, of whom 220 had investments above a specified threshold.
\begin{enumerate}[(i)]
\item (i)
Each respondent who had investments above the threshold was asked about the percentage of these investments that was held in the form of a certain type of trust. The respondents answered by ticking appropriate boxes and the results led to the following frequency distribution.

percentage
frequency
(ii)
< 10
22
10 25
76
25 50
73
> 50
49
(a) Present these data graphically using a carefully drawn histogram.
(b) Calculate the mean percentage for the full set of 220 such respondents, assuming that the frequencies in each category are uniformly spread over the corresponding range.

\item Calculate a 95\% confidence interval for the percentage of such investors who would have investments above the threshold.

The same respondents with investments referred to in part (i) were also asked to specify their satisfaction with the current return received from their full portfolio of investment. This was in the form of a four-point qualitative scale: very satisfied, quite satisfied, a little disappointed, very disappointed. The following two-way table of frequencies was obtained.

<10
percentage in type of trust
10 25
25 50
>50
1
8
10
3
very satisfied
quite satisfied
a little disappointed
very disappointed
6
29
37
4
7
36
28
2
6
27
15
1
In order to investigate whether there is any relationship between the percentage in such trusts and satisfaction with current return, a 2 test is to be performed. 
\item (iii) Calculate the expected frequencies for the above table under an appropriate hypothesis (which should be stated) and comment on why it would be inappropriate to carry out a 2 test directly with these data.

\item (iv) Combining the very satisfied and quite satisfied categories together and the a little disappointed and very disappointed categories together results in the following reduced two-way table.
<10
satisfied
disappointed
9
13
percentage in type of trust
10 25
25 50
>50
35
41
43
30
33
16
Perform the required 2 test at the 5% level using this reduced table and
comment on your conclusion.
\end{enumrate}
%% CT3 A2006 5


%%%%%%%%%%%%%%%%%%%%%%%%%%%%%%%%%%%%%%%%%%%%%%%%%%%%%%%%%%%%%%%%%%%%%%%%%%%%%%%%%%%%%%%%%5
%%%%%%%%%%%%%%%%%%%%%%%%%%%%%%%%%%%%%%%%%%%%%%%%%%%%%%%%%%%%%%%%%%%%%%%%%%%%%
\newpage
10
(i)
(a)
April 2006
 
The key feature of the histogram is that the areas of the four rectangles should be proportional to the frequencies.
See histogram below.
(b)
Mean is calculated from the following frequency distribution:
5
22
x
f
f = 220,
(ii)
17.5
76
fx = 7852.5
\begin{itemize}
\item Estimated proportion is p
220
650
37.5
73
7852.5
220
x
0.338
75
49
35.7%
(or 33.8%)
\item 95\%  confidence interval for underlying proportion is
p 1.96
p (1 p )
650
0.338 1.96
0.338(0.662)
650
as a percentage: 33.8%
0.338 0.036
3.6% or (30.2%, 37.4%)
%%Page 7 
%%(iii)
%% 
\item Under the null hypothesis of no association between percentage in type of trust
and satisfaction with current return, expected frequencies are
2.0
10.0
9.0
1.0
22
6.9
34.5
31.1
3.5
76
6.6
33.2
29.9
3.3
73
six are less than 5 which would invalidate a
(iv)
April 2006
4.5
22.3
20.0
2.2
49
2
20
100
90
10
220
test
expected frequencies (e) are
12.000
10.000
41.455
34.545 39.818
33.182 26.727
22.273
6.455
+6.455 +3.182
3.182 +6.273
6.273
1.005
1.206 0.254
0.305 1.472
1.767
table of residuals (o e) is
3.000
+3.000
table of contributions to
2
is
0.750
0.900
giving
2
= 7.659 on 3 d.f.
2
3 (5%)
7.815

must accept the null hypothesis that there is no relationship between percentage in type of trust and satisfaction with current return.
\item However this decision to accept is marginal at the 5\% level and there is some evidence, but not strong, to suggest that satisfaction improves as the percentage increases.
\end{itemize}
\end{document}
