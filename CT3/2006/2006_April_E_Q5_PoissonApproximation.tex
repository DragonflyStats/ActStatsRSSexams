\documentclass[a4paper,12pt]{article}

%%%%%%%%%%%%%%%%%%%%%%%%%%%%%%%%%%%%%%%%%%%%%%%%%%%%%%%%%%%%%%%%%%%%%%%%%%%%%%%%%%%%%%%%%%%%%%%%%%%%%%%%%%%%%%%%%%%%%%%%%%%%%%%%%%%%%%%%%%%%%%%%%%%%%%%%%%%%%%%%%%%%%%%%%%%%%%%%%%%%%%%%%%%%%%%%%%%%%%%%%%%%%%%%%%%%%%%%%%%%%%%%%%%%%%%%%%%%%%%%%%%%%%%%%%%%

\usepackage{eurosym}
\usepackage{vmargin}
\usepackage{amsmath}
\usepackage{graphics}
\usepackage{epsfig}
\usepackage{enumerate}
\usepackage{multicol}
\usepackage{subfigure}
\usepackage{fancyhdr}
\usepackage{listings}
\usepackage{framed}
\usepackage{graphicx}
\usepackage{amsmath}
\usepackage{chngpage}

%\usepackage{bigints}
\usepackage{vmargin}

% left top textwidth textheight headheight

% headsep footheight footskip

\setmargins{2.0cm}{2.5cm}{16 cm}{22cm}{0.5cm}{0cm}{1cm}{1cm}

\renewcommand{\baselinestretch}{1.3}

\setcounter{MaxMatrixCols}{10}

\begin{document}

A large portfolio of policies is such that a proportion p (0 < p < 1) incurred claims
during the last calendar year. An investigator examines a randomly selected group of
25 policies from the portfolio.
\begin{enumerate}
\item (i) Use a Poisson approximation to the binomial distribution to calculate an
approximate value for the probability that there are at most 4 policies with
claims in the two cases where (a) p = 0.1 and (b) p = 0.2.
\item 
(ii) Comment briefly on the above approximations, given that the exact values of
the probabilities in part (i), using the binomial distribution, are 0.9020 and
0.4207 respectively.
\end{enumerate}
%%%%%%%%%%%%%%%%%%%%%%%%%%%%%%%%%%%%%%%%%%%%%%%%%%%%%%%%%%%%%%%%%%%%%%%%%%%%

Let X = number of policies with claims
So X ~ binomial(25, p).
Poisson approximation is X Poisson(25p).
(a) using Poisson(2.5)
P(X 4) = 0.89118 from tables [or evaluation]
(b) using Poisson(5)
P(X 4) = 0.44049 again from tables

\end{document}
