
\documentclass[a4paper,12pt]{article}

%%%%%%%%%%%%%%%%%%%%%%%%%%%%%%%%%%%%%%%%%%%%%%%%%%%%%%%%%%%%%%%%%%%%%%%%%%%%%%%%%%%%%%%%%%%%%%%%%%%%%%%%%%%%%%%%%%%%%%%%%%%%%%%%%%%%%%%%%%%%%%%%%%%%%%%%%%%%%%%%%%%%%%%%%%%%%%%%%%%%%%%%%%%%%%%%%%%%%%%%%%%%%%%%%%%%%%%%%%%%%%%%%%%%%%%%%%%%%%%%%%%%%%%%%%%%

\usepackage{eurosym}
\usepackage{vmargin}
\usepackage{amsmath}
\usepackage{graphics}
\usepackage{epsfig}
\usepackage{enumerate}
\usepackage{multicol}
\usepackage{subfigure}
\usepackage{fancyhdr}
\usepackage{listings}
\usepackage{framed}
\usepackage{graphicx}
\usepackage{amsmath}
\usepackage{chngpage}

%\usepackage{bigints}
\usepackage{vmargin}

% left top textwidth textheight headheight

% headsep footheight footskip

\setmargins{2.0cm}{2.5cm}{16 cm}{22cm}{0.5cm}{0cm}{1cm}{1cm}

\renewcommand{\baselinestretch}{1.3}

\setcounter{MaxMatrixCols}{10}

\begin{document}
\begin{enumerate}
%%%%%%%%%%%%%%%%%%%%%
%-Question 3
\item 
Consider 12 independent insurance policies, numbered 1, 2, 3, , 12, for each of which a maximum of 1 claim can occur. For each policy, the probability of a claim
occurring is 0.1.
Find the probability that no claims arise on the group of policies numbered 1, 2, 3, 4, 5 and 6, and exactly 1 claim arises in total on the group of policies numbered 7, 8, 9,
10, 11, and 12.
%%%%%%%%%%%%%%%%%%%%%
%-Question 4
\item In a large portfolio 65\% of the policies have been in force for more than five years. An investigation considers a random sample of 500 policies from the portfolio.
Calculate an approximate value for the probability that fewer than 300 of the policies in the sample have been in force for more than five years.
%%%%%%%%%%%%%%%%%%%%%
%- Question 5
\item 
In a random sample of 200 policies from a company s private motor business, there are 68 female policyholders and 132 male policyholders.
Let the proportion of policyholders who are female in the corresponding population of all policyholders be denoted .
Test the hypotheses
H 0 :
0.4 v H 1 :
< 0.4
stating clearly the approximate probability value of the observed statistic and your conclusion.
\end{enumerate}
\newpage
%%%%%%%%%%%%%%%%%%%%%
3
P(no claims on 6 policies) = 0.5314 (from tables p186 — or using 0.9 6 )
P(1 claim on 6 policies) = 0.8857 − 0.5314 = 0.3543 (or using 6(0.1)(0.9 5 ))
So required probability = 0.5314 × 0.3543 = 0.188.
%%%%%%%%%%%%%%%%%%%%%
4
Let X be the number in force for more than five years
then X ~ binomial(500,0.65)
Using a normal approximation, X ≈ N(325, 10.665 2 )
P(X < 300) becomes P(X < 299.5) using continuity correction
 P ( Z <
299.5 − 325
) where Z ~ N(0,1)
10.665
= P ( Z < − 2.39) = 1 − 0.99158 = 0.0084
Page 4  — September 2006 — 
%%%%%%%%%%%%%%%%%%%%%%%%%%%%%%%%%%%%%%%%%%%%%%%%%%%%%%%%%%%%%%%%%%%%%%%%%%%%%%%
\end{document}
