
\documentclass[a4paper,12pt]{article}

%%%%%%%%%%%%%%%%%%%%%%%%%%%%%%%%%%%%%%%%%%%%%%%%%%%%%%%%%%%%%%%%%%%%%%%%%%%%%%%%%%%%%%%%%%%%%%%%%%%%%%%%%%%%%%%%%%%%%%%%%%%%%%%%%%%%%%%%%%%%%%%%%%%%%%%%%%%%%%%%%%%%%%%%%%%%%%%%%%%%%%%%%%%%%%%%%%%%%%%%%%%%%%%%%%%%%%%%%%%%%%%%%%%%%%%%%%%%%%%%%%%%%%%%%%%%

\usepackage{eurosym}
\usepackage{vmargin}
\usepackage{amsmath}
\usepackage{graphics}
\usepackage{epsfig}
\usepackage{enumerate}
\usepackage{multicol}
\usepackage{subfigure}
\usepackage{fancyhdr}
\usepackage{listings}
\usepackage{framed}
\usepackage{graphicx}
\usepackage{amsmath}
\usepackage{chngpage}

%\usepackage{bigints}
\usepackage{vmargin}

% left top textwidth textheight headheight

% headsep footheight footskip

\setmargins{2.0cm}{2.5cm}{16 cm}{22cm}{0.5cm}{0cm}{1cm}{1cm}

\renewcommand{\baselinestretch}{1.3}

\setcounter{MaxMatrixCols}{10}

\begin{document}
\begin{enumerate}

PLEASE TURN OVER8
The events that lead to potential claims on a policy arise as a Poisson process at a rate
of 0.8 per year. However the policy is limited such that only the first three claims in
any one year are paid.
(i) Determine the probabilities of 0, 1, 2 and 3 claims being paid in a particular
year.
[2]
(ii) The amounts (in units of £100) for the claims paid follow a gamma
distribution with parameters = 2 and = 1.
Calculate the expectation of the sum of the amounts for the claims paid in a
particular year.
[3]
(iii)
9
Calculate the expectation of the sum of the amounts for the claims paid in a
particular year, given that there is at least one claim paid in the year.
[2]
[Total 7]
The total claim amount on a portfolio, S, is modelled as having a compound
distribution
S = X 1 + X 2 +
+ X N
where N is the number of claims and has a Poisson distribution with mean , X i is the
amount of the i th claim, and the X i s are independent and identically distributed and
independent of N. Let M X (t) denote the moment generating function of X i .
(i)
Show, using a conditional expectation argument, that the cumulant generating
function of S, C S (t), is given by
C S (t) =
M X (t) 1}.
Note: You may quote the moment generating function of a Poisson random
variable from the book of Formulae and Tables.
[4]
(ii)
Calculate the variance of S in the case where
variance 10.
CT3 A2006 4
= 20 and X has mean 20 and
[2]



%%%%%%%%%%%%%%%%%%%%%%%%%%%%%%%%%%%%%%%%%%%%%%%%%%%%%%%%%%%%%%%%%%%%%%%%%%%%%%%%%%%%% Solutions

8
\begin{enumerate}
\item (i)
April 2006
Examiners Report
By subtraction using entries in tables for Poisson(0.8), the probabilities for the
Poisson distribution for 0, 1, 2 and 3 are: [or by evaluation]
0.44933, 0.35946, 0.14379 and (1 0.95258) = 0.04742
\item (ii)
Let N = number of claims paid and let X 1 ,
S = X i is the sum of the amounts.
, X n be the claim amounts then
E[S] = E[N]E[X]
Here E[N] = 1(0.35946) + 2(0.14379) + 3(0.04742) = 0.7893
and E[X] = 2/1 = 2 from gamma(2,1)
So E[S] = (0.7893)(2) = 1.5786 = £157.86
\item (iii)
Given that N > 0, divide the probabilities in part (i) by (1 0.44933) =
0.55067 to give the probabilities for 1, 2 and 3 claims paid as:
0.6528, 0.2611 and 0.0861
E[N] = 1(0.6528) + 2(0.2611) + 3(0.0861) = 1.4333
So E[S] = (1.4333)(2) = 2.8666 = £286.66
\end{enumerate}
%%%%%%%%%%%%%%%%%%%%%%%%%%%%%%%%%%%%%%%%%%%%%%%%%%%%%%%%%%%%%%%%%%%%%%%%%%%%5
9
\begin{enumerate}
\item (i)
M S (t) = E[e tS ] = E[E[e tS |N]]
Now E[e tS |N = n] = E[exp(tX 1 +
+ tX n )] = E[exp(tX i )] = {M X (t)} n
M S (t) = E[{M X (t)} N ] = E[exp{NlogM X (t)}] = M N {logM X (t)}
= exp[ M X (t) 1}] since N ~ Poisson( )
C S (t) = logM S (t) = M X (t) 1}
\item (ii)
V[S] = C S (0) =
M X (0)} = E[X 2 ] = 20(10 + 20 2 ) = 8200
OR V[S] = E[N]V[X] + V[N]{E[X]} 2 = 20 10 + 20 20 2 = 8200
\end{enumerate}

\end{document}
