\documentclass[a4paper,12pt]{article}

%%%%%%%%%%%%%%%%%%%%%%%%%%%%%%%%%%%%%%%%%%%%%%%%%%%%%%%%%%%%%%%%%%%%%%%%%%%%%%%%%%%%%%%%%%%%%%%%%%%%%%%%%%%%%%%%%%%%%%%%%%%%%%%%%%%%%%%%%%%%%%%%%%%%%%%%%%%%%%%%%%%%%%%%%%%%%%%%%%%%%%%%%%%%%%%%%%%%%%%%%%%%%%%%%%%%%%%%%%%%%%%%%%%%%%%%%%%%%%%%%%%%%%%%%%%%

\usepackage{eurosym}
\usepackage{vmargin}
\usepackage{amsmath}
\usepackage{graphics}
\usepackage{epsfig}
\usepackage{enumerate}
\usepackage{multicol}
\usepackage{subfigure}
\usepackage{fancyhdr}
\usepackage{listings}
\usepackage{framed}
\usepackage{graphicx}
\usepackage{amsmath}
\usepackage{chngpage}

%\usepackage{bigints}
\usepackage{vmargin}

% left top textwidth textheight headheight

% headsep footheight footskip

\setmargins{2.0cm}{2.5cm}{16 cm}{22cm}{0.5cm}{0cm}{1cm}{1cm}

\renewcommand{\baselinestretch}{1.3}

\setcounter{MaxMatrixCols}{10}

\begin{document}
\begin{enumerate}

\item %% Question 1
The stem and leaf plot below gives the surrender values (to the nearest 1,000) of 40
endowment policies issued in France and recently purchased by a dealer in such
policies in Paris. The stem unit is 10,000 and the leaf unit is 1,000.
\begin{verbatim}
5
5
6
6
7
7
8
8
9
9
3
6
02
5779
122344
556677899
1123444
567778
024
\end{verbatim}

6
Determine the median surrender value for this batch of policies.
2
%%%%%%%%%%%%%%%%%%%%%%%%%%%%%%%%%%%%%%%%%%%%%%%
\item In a certain large population 45\% of people have blood group A. A random sample of 300 individuals is chosen from this population.
Calculate an approximate value for the probability that more than 115 of the sample have blood group A.


%%%%%%%%%%%%%%%%%%%%%%%%%%%%%%%%%%%%%%%%%%%%%%%
\item
%%- Question 3
A random sample of size 10 is taken from a normal distribution with mean variance 2 = 1.
= 20 and
Find the probability that the sample variance exceeds 1, that is find $P(S^2 > 1)$.

\end{enumerate}
%-----------------------------------------------------------%

In a one-way analysis of variance, in which samples of 10 claim amounts (\$) from each of three different policy types are being compared, the following means were
calculated:
y 1
276.7 ,
y 2
254.6 ,
y 3
296.3
with residual sum of squares SS R given by
3 10
SS R
( y ij
y i ) 2 15,508.6
i 1 j 1
Calculate estimates for each of the parameters in the usual mathematical model, that
is, calculate , 1 , 2 , 3 , and 2 .


\newpage
%%%%%%%%%%%%%%%%%%%%%%%%%%%%%%%%%%%%%%%%%%%%%%%%%%%%%%%%%%%%%%%%%%%%%%
1 n = 40, so the median is the 20.5 th observation, which is 1⁄2(7.7+7.8) = 7.75.
This represents 77,500.
%%%%%%%%%%%%%%%%%%%%%%%%%%%%%%%%%%%%%%%%%%%%%%%%%%%%%%%%%%%%%%%%%%%%%%%
2 If X is the number in the sample with group A, then X has a binomial (300, 0.45)
distribution, so
E[X] = 300
0.45 = 135 and Var[X] = 300
0.45
0.55 = 74.25.
Then, using the continuity correction,
P(X > 115) = P(X > 115.5) 1
3
n 1 S 2
2
~
P S 2 1
1 0.5627
4
1
2
3
2
5
%%%%%%%%%%%%%%%%%%%%%%%%%%%%%%%%%%%%%%%%%%%%%%%%%%%%%%%%%%%%Page 4
(i)
so here 9 S 2 ~
2
n 1
P
2
9
0.437
115.5 135
= 1
74.25
( 2.26) =
2
9
9
(tables p165)
1
(276.7 254.6 296.3) 275.87
3
276.7 275.87 0.83
254.6 275.87
21.27
296.3 275.87 20.43
SS R
27
15508.6
27
574.4
\begin{itemize}
\item Let X = number of policies with claims
\item So X ~ binomial(25, p).
\item Poisson approximation is X Poisson(25p).
(a) using Poisson(2.5)
\item P(X 4) = 0.89118 from tables [or evaluation]
(b) using Poisson(5)
\item P(X 4) = 0.44049 again from tables
(2.26) = 0.99.
\end{itemize}
%%%%%%%%%%%%%%%%%%%%%%%%%%%%%%%%%%%5
(ii)

in (a) error is 0.8912 0.9020 = 0.0108
in (b) error is 0.4405 0.4207 = 0.0198
The approximation is valid for small p , and, as p is smaller in (a), this gives a better approximation as noted with the smaller error.
%[Note: candidates may also comment on the fact that the sample size 25 is not large and so we would not expect the Poisson approximations to be very good anyway. In fact the key to the approximations is the small p and here the given approximations are quite good]


\end{document}
