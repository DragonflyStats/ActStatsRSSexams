\documentclass[a4paper,12pt]{article}

%%%%%%%%%%%%%%%%%%%%%%%%%%%%%%%%%%%%%%%%%%%%%%%%%%%%%%%%%%%%%%%%%%%%%%%%%%%%%%%%%%%%%%%%%%%%%%%%%%%%%%%%%%%%%%%%%%%%%%%%%%%%%%%%%%%%%%%%%%%%%%%%%%%%%%%%%%%%%%%%%%%%%%%%%%%%%%%%%%%%%%%%%%%%%%%%%%%%%%%%%%%%%%%%%%%%%%%%%%%%%%%%%%%%%%%%%%%%%%%%%%%%%%%%%%%%

\usepackage{eurosym}
\usepackage{vmargin}
\usepackage{amsmath}
\usepackage{graphics}
\usepackage{epsfig}
\usepackage{enumerate}
\usepackage{multicol}
\usepackage{subfigure}
\usepackage{fancyhdr}
\usepackage{listings}
\usepackage{framed}
\usepackage{graphicx}
\usepackage{amsmath}
\usepackage{chngpage}

%\usepackage{bigints}
\usepackage{vmargin}

% left top textwidth textheight headheight

% headsep footheight footskip

\setmargins{2.0cm}{2.5cm}{16 cm}{22cm}{0.5cm}{0cm}{1cm}{1cm}

\renewcommand{\baselinestretch}{1.3}

\setcounter{MaxMatrixCols}{10}

\begin{document}


In a one-way analysis of variance, in which samples of 10 claim amounts (£) from
each of three different policy types are being compared, the following means were
calculated:
y 1
276.7 ,
y 2
254.6 ,
y 3
296.3
with residual sum of squares SS R given by
3 10
SS R
( y ij
y i ) 2 15,508.6
i 1 j 1
Calculate estimates for each of the parameters in the usual mathematical model, that
is, calculate , 1 , 2 , 3 , and 2 .


Page 4
(i)
so here 9 S 2 ~
2
n 1
P
2
9
0.437
115.5 135
= 1
74.25
( 2.26) =
2
9
9
(tables p165)
1
(276.7 254.6 296.3) 275.87
3
276.7 275.87 0.83
254.6 275.87
21.27
296.3 275.87 20.43
SS R
27
15508.6
27
574.4
