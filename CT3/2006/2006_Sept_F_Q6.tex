
\documentclass[a4paper,12pt]{article}

%%%%%%%%%%%%%%%%%%%%%%%%%%%%%%%%%%%%%%%%%%%%%%%%%%%%%%%%%%%%%%%%%%%%%%%%%%%%%%%%%%%%%%%%%%%%%%%%%%%%%%%%%%%%%%%%%%%%%%%%%%%%%%%%%%%%%%%%%%%%%%%%%%%%%%%%%%%%%%%%%%%%%%%%%%%%%%%%%%%%%%%%%%%%%%%%%%%%%%%%%%%%%%%%%%%%%%%%%%%%%%%%%%%%%%%%%%%%%%%%%%%%%%%%%%%%

\usepackage{eurosym}
\usepackage{vmargin}
\usepackage{amsmath}
\usepackage{graphics}
\usepackage{epsfig}
\usepackage{enumerate}
\usepackage{multicol}
\usepackage{subfigure}
\usepackage{fancyhdr}
\usepackage{listings}
\usepackage{framed}
\usepackage{graphicx}
\usepackage{amsmath}
\usepackage{chngpage}

%\usepackage{bigints}
\usepackage{vmargin}

% left top textwidth textheight headheight

% headsep footheight footskip

\setmargins{2.0cm}{2.5cm}{16 cm}{22cm}{0.5cm}{0cm}{1cm}{1cm}

\renewcommand{\baselinestretch}{1.3}

\setcounter{MaxMatrixCols}{10}

\begin{document}
\large 
\noindent

It is assumed that claims arising on an industrial policy can be modelled as a Poisson process at a rate of 0.5 per year.

\begin{enumerate}[(i)]
\item Determine the probability that no claims arise in a single year.

\item Determine the probability that, in three consecutive years, there is one or more
claims in one of the years and no claims in each of the other two years.

\item Suppose a claim has just occurred. Determine the probability that more than
two years will elapse before the next claim occurs.
\end{enumerate}



%%%%%%%%%%%%%%%%%%%%%%%%%%%%%%%%%%%%%%%%%%%%%%%%%%%%%%%%%%%%%%%%%%%%%%%%%%%%%%%%%%%%%%
\noindent \textbf{Question 6}
\begin{itemize}
\item (i) $P(no claims) = P(X = 0)$ where X ~ Poisson(0.5)
= 0.6065 from tables [or evaluation]
\item (ii) Let $Y$ = number of years with a claim
then $Y ~ binomial(,0.95)$ [or just directly as below]
\[P(Y = 1) = (0.95)(0.6065)^2 = 0.44\]
\item (iii)
Let T = time until next claim
then T ~ exp(0.5)
\[P(T > 2) = e^{ –0.5(2)} 
 = e^{ –1} = 0.68\]
 [or by integration]
% [OR: answer = {P(no claim)} 2 = 0.6065 2 = 0.68]
% [OR: claim rate for period of 2 years = 1, so P(no claim in 2 years)
% = e -1 = 0.68]
\end{itemize}
%%%%%%%%%%%%%%%%%%%%%%%%%%%%%%%%%%%%%%%%%%%%%%%%%%%%%%%%%%%%%%%%%%%%%%%%%%%%%%%%%%%%%%%%

\end{document}
