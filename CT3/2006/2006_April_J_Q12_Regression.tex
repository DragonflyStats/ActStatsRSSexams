
\documentclass[a4paper,12pt]{article}

%%%%%%%%%%%%%%%%%%%%%%%%%%%%%%%%%%%%%%%%%%%%%%%%%%%%%%%%%%%%%%%%%%%%%%%%%%%%%%%%%%%%%%%%%%%%%%%%%%%%%%%%%%%%%%%%%%%%%%%%%%%%%%%%%%%%%%%%%%%%%%%%%%%%%%%%%%%%%%%%%%%%%%%%%%%%%%%%%%%%%%%%%%%%%%%%%%%%%%%%%%%%%%%%%%%%%%%%%%%%%%%%%%%%%%%%%%%%%%%%%%%%%%%%%%%%

\usepackage{eurosym}
\usepackage{vmargin}
\usepackage{amsmath}
\usepackage{graphics}
\usepackage{epsfig}
\usepackage{enumerate}
\usepackage{multicol}
\usepackage{subfigure}
\usepackage{fancyhdr}
\usepackage{listings}
\usepackage{framed}
\usepackage{graphicx}
\usepackage{amsmath}
\usepackage{chngpage}

%\usepackage{bigints}
\usepackage{vmargin}

% left top textwidth textheight headheight

% headsep footheight footskip

\setmargins{2.0cm}{2.5cm}{16 cm}{22cm}{0.5cm}{0cm}{1cm}{1cm}

\renewcommand{\baselinestretch}{1.3}

\setcounter{MaxMatrixCols}{10}

\begin{document}
\begin{enumerate}
In an experiment to compare the effects of vaccines of differing strengths intended to give protection to children against a particular condition, twelve batches of vaccine
were tested in twelve equal-sized groups of children. The percentages of children who subsequently remained healthy after exposure to the condition, named the PRH
values, were recorded. The strength of each batch of vaccine was measured by an independent test and recorded as the SV value.

%%--- CT3 A2006 6

The recorded values are:
Batch:
1
PRH (y): 16
SV (x):
0.9
x 38.4 ;
\item (i)
2
3
4
68 23 35
1.6 2.3 2.7
y
528 ;
5
6
7
8
9
42 41 46 48 52
3.0 3.3 3.7 3.8 4.1
x 2 137.16 ;
y 2
10 11 12
50 54 53
4.2 4.3 4.5
25, 428 ;
xy 1, 778.4
Draw a rough plot of the data to show the relationship between the SV and PRH values.

It is evident that one of the observations is out of line and so may have an undue effect on any regression analysis. You are asked to investigate this as follows.
\item (ii)
(a) Calculate the total, regression, and error sums of squares for a least- squares linear regression analysis for predicting PRH values from SV
values using all 12 data observations.
(b) Determine the coefficient of determination $R^2$ .
(c) Determine the equation of the fitted regression line.
(d) Examine whether or not there is evidence, at the 5\% level of testing, to enable one to conclude that the slope of the underlying regression equation is non-zero.
%%%%%%%%%%%%%%%%%%%%%%%%%%%%%%%%%%%%%%%%%%%%%%%%%%%%%%%%%%%%%
The details of the regression analysis after removing the data for batch 2 are given in
the box below.
Regression equation: y = 3.76 + 11.4 x
Coef
Stdev
t-ratio p-val
Intercept
x 0.255
0.000
3.757
11.377

Analysis of Variance
Source
df
SS
Regression 1
Error
9
Total
10
(iii)
1486.9
80.8
1567.6
3.092
0.8838
MS
1486.9
8.98
1.22
12.9
F p-val
165.69 0.000
(a) Comment on the main differences in the results of the regression analysis resulting from removing the data for batch 2.
(b) Calculate a 95\% confidence interval for the expected (mean) PRH value for a batch of vaccine with SV value 3.5.

\end{enumerate}

%%%%%%%%%%%%%%%%%%%%%%%%%%%%%%%%%%%%%%%%%%%%%%%%%%%%%%%%%%%%%%%%%%%%%%%%%%%%%%%%%%%%%%%%%%%%%%%%%%%%
\newpage

12
\begin{itemize}
\item (i)
70
60
50
40
30
20
1
2
3
4
SV
\item (ii)
Page 10
(a)
SSTOT = S_{yy} = 25428
528 2 /12 = 2196
S_{xx} = 137.16 38.4 2 /12 = 14.28 , S xy = 1778.4 (38.4 528)/12 = 88.8
SSREG = 88.8 2 /14.28 = 552.20, SSRES = 2196 552.20 = 1643.80 
(b) R 2 = 552.2/2196 = 0.251 (25.1%)
(c) y = a + bx:
April 2006
b 88.8 /14.28 6.2185
a 528 /12 88.8 /14.28 (38.4 /12)
%%%%%%%%%%%%%%%%%%%%%%%%%%%%%%%%%%%%
24.101
Fitted line is $y = 24.101 + 6.2185x$
(d)
s . e . b
1643.8 /10
14.28
1/ 2
3.3928
Observed t = (6.2185 0)/3.3928 = 1.833 < t 10 (0.025) = 2.228
so we do not have evidence at the 5\% level of testing to justify rejecting b = 0 and concluding that the underlying slope is non-zero.
\end{itemize}

(iii)
\begin{itemize}
\item (a)
Large change in slope (and intercept) of fitted line.
The total and error sums of squares are much reduced.
The fit of the linear regression model is much improved ($R^2$ is much increased from 25.1\% to 94.9\%).
We have overwhelming evidence to justify concluding that the slope is non-zero.
\item (b)
Fitted PRH value at SV = 3.5 is 3.757 + (11.377 3.5) = 43.577
n 11,
x 38.4 1.6 36.8,
x 2 137.16 1.6 2 134.6
S_{xx} = 11.4873
s.e. of estimation =
1
11
3.5 36.8 /11
11.4873
1/ 2
2
8.98
0.9138
t 9 (0.025) = 2.262
\item 95\% CI for expected PRH is 43.577
i.e. 43.577
2.067
(2.262
0.9138)
or (41.51, 45.64)
\end{itemize}
%%%%%%%%%%%%%%%%%%%%%%%%%%%%%%%%%%%%%%%%%%%%%%%%%%%%%%%%%%%%%%%%%%%%%%%%%%%%%%%%%%%%%%%%%%%%%%5555

\end{document}
