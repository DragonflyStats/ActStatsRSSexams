
\documentclass[a4paper,12pt]{article}

%%%%%%%%%%%%%%%%%%%%%%%%%%%%%%%%%%%%%%%%%%%%%%%%%%%%%%%%%%%%%%%%%%%%%%%%%%%%%%%%%%%%%%%%%%%%%%%%%%%%%%%%%%%%%%%%%%%%%%%%%%%%%%%%%%%%%%%%%%%%%%%%%%%%%%%%%%%%%%%%%%%%%%%%%%%%%%%%%%%%%%%%%%%%%%%%%%%%%%%%%%%%%%%%%%%%%%%%%%%%%%%%%%%%%%%%%%%%%%%%%%%%%%%%%%%%

\usepackage{eurosym}
\usepackage{vmargin}
\usepackage{amsmath}
\usepackage{graphics}
\usepackage{epsfig}
\usepackage{enumerate}
\usepackage{multicol}
\usepackage{subfigure}
\usepackage{fancyhdr}
\usepackage{listings}
\usepackage{framed}
\usepackage{graphicx}
\usepackage{amsmath}
\usepackage{chngpage}

%\usepackage{bigints}
\usepackage{vmargin}

% left top textwidth textheight headheight

% headsep footheight footskip

\setmargins{2.0cm}{2.5cm}{16 cm}{22cm}{0.5cm}{0cm}{1cm}{1cm}

\renewcommand{\baselinestretch}{1.3}

\setcounter{MaxMatrixCols}{10}

\begin{document}
\begin{enumerate}
\item 

It is assumed that claims arising on an industrial policy can be modelled as a Poisson process at a rate of 0.5 per year.

\begin{enumerate}[(i)]
\item Determine the probability that no claims arise in a single year.

\item Determine the probability that, in three consecutive years, there is one or more
claims in one of the years and no claims in each of the other two years.

\item Suppose a claim has just occurred. Determine the probability that more than
two years will elapse before the next claim occurs.
\end{enumerate}
%%%%%%%%%%%%%%%%%%%%%%%%%%%%%%%%%%%%%%%%%%%%%%%%%%%%%%%%%%%%%%%
\item 
A commuter catches a bus each morning for 100 days. The buses arrive at the stop
according to a Poisson process, at an average rate of one per 15 minutes, so if $X_{i}$ is the
waiting time on day i, then $X_{i}$ has an exponential distribution with parameter
1
15
so
$E[X_{i} ] = 15, Var[X_{i} ] = 15 2 = 225.$

\begin{enumerate}[(i)]
\item Calculate (approximately) the probability that the total time the commuter spends waiting for buses over the 100 days exceeds 27 hours.

\item  At the end of the 100 days the bus frequency is increased, so that buses arrive at one per 10 minutes on average (still behaving as a Poisson process). The commuter then catches a bus each day for a further 99 days. Calculate
(approximately) the probability that the total time spent waiting over the
whole 199 days exceeds 40 hours.
\end{enumerate}

\end{enumerate}
%%%%%%%%%%%%%%%%%%%%%%%%%%%%%%%%%%%%%%%%%%%%%%%%%%%%%%%%%%%%%%%%%%%%%%%%%%%
\newpage


%%%%%%%%%%%%%%%%%%%%%%%%%%%%%%%%%%%%%%%%%%%%%%%%%%%%%%%%%%%%%%%%%%%%%%%%%%%%%%%%%%%%%%
\noindent \textbf{Question 5}
\begin{itemize}
\item Under H 0 : sample proportion P is approximately normally distributed with mean 0.4 and standard error (0.4×0.6/200) 1/2 = 0.0464
\item Therefore P-value of observed proportion (68/200 = 0.4)
0.4 − 0.4 ⎞
⎛
= P ⎜ Z <
⎟ = P ( Z < − 1.72 ) = 0.042
0.0464 ⎠
⎝
\item We reject H 0 at the 5% level of testing and conclude that the proportion of
policyholders who are female is less than 0.4.
\item [OR This is actually better - working with the number of female policyholders
(observed = 68), the P-value is
⎛
⎞
68.5 − 80
P ⎜ Z <
= − 1.660 ⎟ = 0.048
⎜
⎟
200(0.4)(0.6)
⎝
⎠
]
\item Note: We can word the conclusion: we reject H 0 at levels of testing down to 4.2% (or
4.8%) and conclude ...
\end{itemize}
%%%%%%%%%%%%%%%%%%%%%%%%%%%%%%%%%%%%%%%%%%%%%%%%%%%%%%%%%%%%%%%%%%%%%%%%%%%%%%%%%%%%%%
\noindent \textbf{Question 6}
\begin{itemize}
\item (i) P(no claims) = P(X = 0) where X ~ Poisson(0.5)
= 0.6065 from tables [or evaluation]
\item (ii) Let Y = number of years with a claim
then Y ~ binomial(,0.95) [or just directly as below]
P(Y = 1) = (0.95)(0.6065) 2 = 0.44
\item (iii)
Let T = time until next claim
then T ~ exp(0.5)
P(T > 2) = e –0.5(2) [or by integration]
= e –1 = 0.68
[OR: answer = {P(no claim)} 2 = 0.6065 2 = 0.68]
[OR: claim rate for period of 2 years = 1, so P(no claim in 2 years)
= e -1 = 0.68]
\end{itemize}
%%%%%%%%%%%%%%%%%%%%%%%%%%%%%%%%%%%%%%%%%%%%%%%%%%%%%%%%%%%%%%%%%%%%%%%%%%%%%%%%%%%%%%%%
\noindent \textbf{Question 7}
\begin{itemize}
\item (i)
As stated in the question, if X_{i} is the waiting time on day i, then X_{i} has an
exponential distribution with parameter
1
15
so E(X_{i} ) = 15, Var(X_{i} ) = 15 2 = 225.
If X_{i}s the total waiting time over the 100 days, X = \sum i = 1 X_{i} ,
100
% Page 5


so $E [ X ] = 1500$ and $Var [ X ] = 22500$ and by the CLT
X has approximately an N (1500, 22500) distribution,
⎛ 1620 − 1500 ⎞
so P ( X > 1620) ≈ 1 − \Phi ⎜
⎟ = 1 − \Phi(0.8) = 0.2119.
150
⎝
⎠
\item (ii)
If Y j is the waiting time on day j of the extra 99 days, then $E ( Y_j ) = 10$ and
Var ( Y j ) = 100 so that if $Y =
\sum^{99}_{j = 1} Y_j$

is the total waiting time over the 99 days,
then Y is approximately N (990,9900) by CLT.
If Z = X + Y (so that Z is the total waiting time over the whole 199 days), then
since X and Y are independent, Z is approximately N (1500+990, 22500+9900),
i.e. N (2490, 2400).
⎛ 2400 − 2490 ⎞
Hence P ( Z > 2400) ≈ 1 − \Phi ⎜
⎟ = 1 − \Phi(−0.5) = \Phi(0.5) = 0.6915.
180
⎝
⎠
\end{itemize}
\end{document}
