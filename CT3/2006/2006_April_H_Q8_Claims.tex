
\documentclass[a4paper,12pt]{article}

%%%%%%%%%%%%%%%%%%%%%%%%%%%%%%%%%%%%%%%%%%%%%%%%%%%%%%%%%%%%%%%%%%%%%%%%%%%%%%%%%%%%%%%%%%%%%%%%%%%%%%%%%%%%%%%%%%%%%%%%%%%%%%%%%%%%%%%%%%%%%%%%%%%%%%%%%%%%%%%%%%%%%%%%%%%%%%%%%%%%%%%%%%%%%%%%%%%%%%%%%%%%%%%%%%%%%%%%%%%%%%%%%%%%%%%%%%%%%%%%%%%%%%%%%%%%

\usepackage{eurosym}
\usepackage{vmargin}
\usepackage{amsmath}
\usepackage{graphics}
\usepackage{epsfig}
\usepackage{enumerate}
\usepackage{multicol}
\usepackage{subfigure}
\usepackage{fancyhdr}
\usepackage{listings}
\usepackage{framed}
\usepackage{graphicx}
\usepackage{amsmath}
\usepackage{chngpage}

%\usepackage{bigints}
\usepackage{vmargin}

% left top textwidth textheight headheight

% headsep footheight footskip

\setmargins{2.0cm}{2.5cm}{16 cm}{22cm}{0.5cm}{0cm}{1cm}{1cm}

\renewcommand{\baselinestretch}{1.3}

\setcounter{MaxMatrixCols}{10}

\begin{document}

The events that lead to potential claims on a policy arise as a Poisson process at a rate
of 0.8 per year. However the policy is limited such that only the first three claims in
any one year are paid.

\begin{enumerate}[(a)]
\item 
(i) Determine the probabilities of 0, 1, 2 and 3 claims being paid in a particular
year.

\item 
(ii) The amounts (in units of \$100) for the claims paid follow a gamma
distribution with parameters = 2 and = 1.
Calculate the expectation of the sum of the amounts for the claims paid in a
particular year.

\item (iii)
9
Calculate the expectation of the sum of the amounts for the claims paid in a
particular year, given that there is at least one claim paid in the year.


\end{enumerate}
\newpage



%%%%%%%%%%%%%%%%%%%%%%%%%%%%%%%%%%%%%%%%%%%%%%%%%%%%%%%%%%%%%%%%%%%%%%%%%%%%%%%%%%%%% Solutions

8
\begin{itemize}
\item (i)
April 2006
 
By subtraction using entries in tables for Poisson(0.8), the probabilities for the Poisson distribution for 0, 1, 2 and 3 are: [or by evaluation]
0.44933, 0.35946, 0.14379 and (1 0.95258) = 0.04742
%-------------------%
\item (ii)
Let N = number of claims paid and let X 1 ,
S = X i is the sum of the amounts.
, X n be the claim amounts then
\[E[S] = E[N]E[X]\]
Here E[N] = 1(0.35946) + 2(0.14379) + 3(0.04742) = 0.7893
and E[X] = 2/1 = 2 from gamma(2,1)
So E[S] = (0.7893)(2) = 1.5786 = \$157.86
\item (iii)
Given that N > 0, divide the probabilities in part (i) by (1 0.44933) =
0.55067 to give the probabilities for 1, 2 and 3 claims paid as:
0.6528, 0.2611 and 0.0861

\[E[N] = 1(0.6528) + 2(0.2611) + 3(0.0861) = 1.4333\]
So \[E[S] = (1.4333)(2) = 2.8666 = \$286.66\]
\end{itemize}
%%%%%%%%%%%%%%%%%%%%%%%%%%%%%%%%%%%%%%%%%%%%%%%%%%%%%%%%%%%%%%%%%%%%%%%%%%%%5

\end{document}
