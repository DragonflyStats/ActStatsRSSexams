
\documentclass[a4paper,12pt]{article}

%%%%%%%%%%%%%%%%%%%%%%%%%%%%%%%%%%%%%%%%%%%%%%%%%%%%%%%%%%%%%%%%%%%%%%%%%%%%%%%%%%%%%%%%%%%%%%%%%%%%%%%%%%%%%%%%%%%%%%%%%%%%%%%%%%%%%%%%%%%%%%%%%%%%%%%%%%%%%%%%%%%%%%%%%%%%%%%%%%%%%%%%%%%%%%%%%%%%%%%%%%%%%%%%%%%%%%%%%%%%%%%%%%%%%%%%%%%%%%%%%%%%%%%%%%%%

\usepackage{eurosym}
\usepackage{vmargin}
\usepackage{amsmath}
\usepackage{graphics}
\usepackage{epsfig}
\usepackage{enumerate}
\usepackage{multicol}
\usepackage{subfigure}
\usepackage{fancyhdr}
\usepackage{listings}
\usepackage{framed}
\usepackage{graphicx}
\usepackage{amsmath}
\usepackage{chngpage}

%\usepackage{bigints}
\usepackage{vmargin}

% left top textwidth textheight headheight

% headsep footheight footskip

\setmargins{2.0cm}{2.5cm}{16 cm}{22cm}{0.5cm}{0cm}{1cm}{1cm}

\renewcommand{\baselinestretch}{1.3}

\setcounter{MaxMatrixCols}{10}

\begin{document}

9

The total claim amount on a portfolio, S, is modelled as having a compound
distribution
S = X 1 + X 2 +
+ X N
where N is the number of claims and has a Poisson distribution with mean , X_{i} is the amount of the i th claim, and the X_{i} s are independent and identically distributed and
independent of N. Let M X (t) denote the moment generating function of X_{i} .
\begin{enumerate}
\item 

Show, using a conditional expectation argument, that the cumulant generating
function of S, C S (t), is given by
C S (t) =
M X (t) 1}.
Note: You may quote the moment generating function of a Poisson random variable from the book of Formulae and Tables.

\item 
Calculate the variance of S in the case where
variance 10.

= 20 and X has mean 20 and
\end{enumerate}
\newpage



%%%%%%%%%%%%%%%%%%%%%%%%%%%%%%%%%%%%%%%%%%%%%%%%%%%%%%%%%%%%%%%%%%%%%%%%%%%%%%%%%%%%% Solutions

%%%%%%%%%%%%%%%%%%%%%%%%%%%%%%%%%%%%%%%%%%%%%%%%%%%%%%%%%%%%%%%%%%%%%%%%%%%%5
9
\begin{enumerate}
\item (i)
M S (t) = E[e tS ] = E[E[e tS |N]]
Now E[e tS |N = n] = E[exp(tX 1 +
+ tX n )] = E[exp(tX_{i} )] = {M X (t)} n
M S (t) = E[{M X (t)} N ] = E[exp{NlogM X (t)}] = M N {logM X (t)}
= exp[ M X (t) 1}] since N ~ Poisson( )
C S (t) = logM S (t) = M X (t) 1}
\item (ii)
V[S] = C S (0) =
M X (0)} = E[X 2 ] = 20(10 + 20 2 ) = 8200
OR V[S] = E[N]V[X] + V[N]{E[X]} 2 = 20 10 + 20 20 2 = 8200
\end{enumerate}

\end{document}
