
\documentclass[a4paper,12pt]{article}

%%%%%%%%%%%%%%%%%%%%%%%%%%%%%%%%%%%%%%%%%%%%%%%%%%%%%%%%%%%%%%%%%%%%%%%%%%%%%%%%%%%%%%%%%%%%%%%%%%%%%%%%%%%%%%%%%%%%%%%%%%%%%%%%%%%%%%%%%%%%%%%%%%%%%%%%%%%%%%%%%%%%%%%%%%%%%%%%%%%%%%%%%%%%%%%%%%%%%%%%%%%%%%%%%%%%%%%%%%%%%%%%%%%%%%%%%%%%%%%%%%%%%%%%%%%%

\usepackage{eurosym}
\usepackage{vmargin}
\usepackage{amsmath}
\usepackage{graphics}
\usepackage{epsfig}
\usepackage{enumerate}
\usepackage{multicol}
\usepackage{subfigure}
\usepackage{fancyhdr}
\usepackage{listings}
\usepackage{framed}
\usepackage{graphicx}
\usepackage{amsmath}
\usepackage{chngpage}

%\usepackage{bigints}
\usepackage{vmargin}

% left top textwidth textheight headheight

% headsep footheight footskip

\setmargins{2.0cm}{2.5cm}{16 cm}{22cm}{0.5cm}{0cm}{1cm}{1cm}

\renewcommand{\baselinestretch}{1.3}

\setcounter{MaxMatrixCols}{10}

\begin{document}
\begin{enumerate}
\item 

It is assumed that claims arising on an industrial policy can be modelled as a Poisson process at a rate of 0.5 per year.

\begin{enumerate}[(i)]
\item Determine the probability that no claims arise in a single year.

\item Determine the probability that, in three consecutive years, there is one or more
claims in one of the years and no claims in each of the other two years.

\item Suppose a claim has just occurred. Determine the probability that more than
two years will elapse before the next claim occurs.
\end{enumerate}

\end{enumerate}
%%%%%%%%%%%%%%%%%%%%%%%%%%%%%%%%%%%%%%%%%%%%%%%%%%%%%%%%%%%%%%%%%%%%%%%%%%%
\newpage


%%%%%%%%%%%%%%%%%%%%%%%%%%%%%%%%%%%%%%%%%%%%%%%%%%%%%%%%%%%%%%%%%%%%%%%%%%%%%%%%%%%%%%
\noindent \textbf{Question 5}
\begin{itemize}
\item Under $H_0$ : sample proportion P is approximately normally distributed with mean 0.4 and standard error (0.4×0.6/200) 1/2 = 0.0464
\item Therefore P-value of observed proportion (68/200 = 0.4)
0.4 − 0.4 ⎞
⎛
= P ⎜ Z <
⎟ = P ( Z < − 1.72 ) = 0.042
0.0464 ⎠
⎝
\item We reject $H_0$ at the 5\% level of testing and conclude that the proportion of
policyholders who are female is less than 0.4.
\item [OR This is actually better - working with the number of female policyholders
(observed = 68), the P-value is
⎛
⎞
68.5 − 80
P ⎜ Z <
= − 1.660 ⎟ = 0.048
⎜
⎟
200(0.4)(0.6)
⎝
⎠
]
\item Note: We can word the conclusion: we reject H 0 at levels of testing down to 4.2\% (or
4.8\%) and conclude ...
\end{itemize}

%%%%%%%%%%%%%%%%%%%%%%%%%%%%%%%%%%%%%%%%%%%%%%%%%%%%%%%%%%%%%%%%%%%%%%%%%%%%%%%%%%%%%%%%
\end{document}
