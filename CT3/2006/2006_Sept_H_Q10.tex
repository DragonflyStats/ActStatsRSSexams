
\documentclass[a4paper,12pt]{article}

%%%%%%%%%%%%%%%%%%%%%%%%%%%%%%%%%%%%%%%%%%%%%%%%%%%%%%%%%%%%%%%%%%%%%%%%%%%%%%%%%%%%%%%%%%%%%%%%%%%%%%%%%%%%%%%%%%%%%%%%%%%%%%%%%%%%%%%%%%%%%%%%%%%%%%%%%%%%%%%%%%%%%%%%%%%%%%%%%%%%%%%%%%%%%%%%%%%%%%%%%%%%%%%%%%%%%%%%%%%%%%%%%%%%%%%%%%%%%%%%%%%%%%%%%%%%

\usepackage{eurosym}
\usepackage{vmargin}
\usepackage{amsmath}
\usepackage{graphics}
\usepackage{epsfig}
\usepackage{enumerate}
\usepackage{multicol}
\usepackage{subfigure}
\usepackage{fancyhdr}
\usepackage{listings}
\usepackage{framed}
\usepackage{graphicx}
\usepackage{amsmath}
\usepackage{chngpage}

%\usepackage{bigints}
\usepackage{vmargin}

% left top textwidth textheight headheight

% headsep footheight footskip

\setmargins{2.0cm}{2.5cm}{16 cm}{22cm}{0.5cm}{0cm}{1cm}{1cm}

\renewcommand{\baselinestretch}{1.3}

\setcounter{MaxMatrixCols}{10}

\begin{document}
\begin{enumerate}

\item %%-- Queation 10
Let (X 1 , X 2 ,
, X n ) be a random sample of a gamma(4.5, ) random variable, with
sample mean X .
(i)
(ii)
2
9 n .
\begin{enumerate}[(i)]
\item (a) Using moment generating functions, show that 2 nX ~
\item (b) Construct a 95% confidence interval for , based on X and the result
in (i)(a) above.
\item (c) Evaluate the interval in (i)(b) above in the case for which a random
sample of 10 observations gave a value
x i 21.47
\end{enumerate}
%%%%%%%%%%%%%%
(a) Show that the maximum likelihood estimator of
is given by
4.5
.
X
(b)
Show that the asymptotic standard error of
4.5n
(c)
1/ 2
is estimated by
.
Construct a 95% confidence interval for based on the asymptotic
distribution of , and evaluate this interval in the case for which a
random sample of 100 observations gave a value
x i 225.3.
[9]
[Total 18]

%%%%%%%%%%%%%%%%%%%%%%%%%%%%%%%%%%%%%%%%%%%%%%%%%%%%%%%%%%%%%%%%%%%%%%%%%%%%%%%%%%%%%%%%%%%%%%%%%%%%%%%%%%%8
\newpage
%%--- Question10
(i)
(a)
Mgf of X i is (1 – t / \lambda ) − 4.5 so mgf of
n
∏ ( 1 − t / \lambda )
− 4.5
= ( 1 − t / \lambda )
n
∑ X i
is
i = 1
− 4.5 n
i = 1
n
Hence mgf of 2 \lambda ∑ X i = 2 \lambda nX is ( 1 − 2 \lambda t / \lambda )
− 4.5 n
= ( 1 − 2 t )
− 4.5 n
i = 1
This is the mgf of a χ 2 variable — with 9n degrees of freedom.
(b)
b ⎞
⎛ a
<\lambda<
P ( a < 2 \lambda nX < b ) = 0.95 ⇒ P ⎜
⎟ = 0.95
2 nX ⎠
⎝ 2 nX
where a and b are such that
(
)
(
)
P χ 9 2 n < a = 0.025 and P χ 9 2 n > b = 0.025.
b
⎛ a
so a 95% CI for \lambda is given by ⎜
,
⎝ 2 nX 2 nX
(c)
⎞
⎟ .
⎠
9 n = 90, and from tables of χ 2 with 90df we have a = 65.65, b = 118.1
118.1 ⎞
⎛ 65.65
CI is ⎜
,
⎟ = (1.53 , 2.75).
⎝ 2 × 21.47 2 × 21.47 ⎠
(ii)
(a)
(
L ( \lambda ) ∝ \lambda 4.5 n exp −\lambda ∑ x i
)
so
A ( \lambda ) = ( 4.5 n ) log \lambda − \lambda ∑ x i + constant
⇒
d A
= 4.5 n / \lambda − ∑ x i
d \lambda
d 2 A
Setting
= 4.5 n / \lambda 2 so s . e .( \hat{\lambda}) ≅
(b) −
(c) 95% CI is \hat{\lambda}\pm 1.96 × s . e . \lambda ˆ
d \lambda
2
{
\lambda ˆ
( 4.5 n ) 1/ 2
( ) }
In the case n = 100, \sigmax = 225.3,
Page 8
4.5 n 4.5
d A
=
= 0 ⇒ \hat{\lambda}=
d \lambda
∑ X i X
%%---- Subject CT3 (Probability and Mathematical Statistics Core Technical) — September 2006 — Examiners’ Report
4.5 / 2.253
\hat{\lambda}= 4.5 / 2.253 = 1.9973 and s . e .( \hat{\lambda}) ≅
= 0.0942
( 450 ) 1/ 2
so CI is 1.9973 \pm(1.96 × 0.0942) i.e. (1.81 , 2.18).
