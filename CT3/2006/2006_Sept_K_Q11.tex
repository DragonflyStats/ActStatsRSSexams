
\documentclass[a4paper,12pt]{article}

%%%%%%%%%%%%%%%%%%%%%%%%%%%%%%%%%%%%%%%%%%%%%%%%%%%%%%%%%%%%%%%%%%%%%%%%%%%%%%%%%%%%%%%%%%%%%%%%%%%%%%%%%%%%%%%%%%%%%%%%%%%%%%%%%%%%%%%%%%%%%%%%%%%%%%%%%%%%%%%%%%%%%%%%%%%%%%%%%%%%%%%%%%%%%%%%%%%%%%%%%%%%%%%%%%%%%%%%%%%%%%%%%%%%%%%%%%%%%%%%%%%%%%%%%%%%

\usepackage{eurosym}
\usepackage{vmargin}
\usepackage{amsmath}
\usepackage{graphics}
\usepackage{epsfig}
\usepackage{enumerate}
\usepackage{multicol}
\usepackage{subfigure}
\usepackage{fancyhdr}
\usepackage{listings}
\usepackage{framed}
\usepackage{graphicx}
\usepackage{amsmath}
\usepackage{chngpage}

%\usepackage{bigints}
\usepackage{vmargin}

% left top textwidth textheight headheight

% headsep footheight footskip

\setmargins{2.0cm}{2.5cm}{16 cm}{22cm}{0.5cm}{0cm}{1cm}{1cm}

\renewcommand{\baselinestretch}{1.3}

\setcounter{MaxMatrixCols}{10}

\begin{document}

%%%%%%%%%%%%%%%%%%%%%%%%%

%%-- Question 11
A survey of financial institutions which offered tax-efficient savings accounts was conducted. These accounts had limits on the amounts of money that could be invested each year. The study was interested in comparing the maturity values achieved by investing the maximum possible amount each year over a certain time period.

The following values (in units of \$1,000 and rounded to 2 decimal places) are the
maturity values for such investments for 8 high street banks and 12 other banks (i.e.
those without high street branches).
High street banks (x 1 ): 11.91, 11.87, 11.8, 11.66, 11.5, 11.49, 11.49, 11.42
( x 1 = 9.20, x 1 2 = 1,086.0470)
Other banks (x 2 ): 12.2, 12.17, 12.16, 11.90, 11.82, 11.77, 11.74, 11.70, 11.64, 11.60,
11.55, 11.50
( x 2 = 141.78, x 2 2 = 1,675.8224)
\begin{enumerate}[(a)]
 \item (i) Draw a diagram in which the maturity values for high street and other banks
may be compared.
\item 
\item (ii) Calculate a 95\% confidence interval for the difference between the means of the maturity values for high street and other banks, and comment on any
implications suggested by the interval.
\item (iii) (a)
Show that a test of the equality of variances of maturity values for high street banks and other banks is not significant at the 5% level.
\item (b)
Comment briefly on the validity of the assumptions required for the interval in (ii).

The following values (in units of \$1,000 and rounded to 2 decimal places) are the
maturity values for the maximum possible investment for a random sample of 12
building societies (a different kind of financial institution).
Building societies (x  ): 12.40, 12.19, 12.06, 12.01, 12.00, 11.97, 11.94, 11.92, 11.88,
11.86, 11.81, 11.79
( x  = 14.8, x  2 = 1,724.2449)
\item (iv) Add further points to your diagram in part(i) such that the maturity values for
all three types of financial institution may be compared.
\item (v) Use one-way analysis of variance to compare the maturity values for the three different types of financial institution, and comment briefly on the validity of
the assumptions required for analysis of variance.
\item (vi) Interpret the results of the statistical analyses conducted in (ii) and (v).
\end{enumerate}

\newpage
%%%%%%%%%%%%%%%%%%%%%%%%%%%%%%%%%%%%%%%%%%%%%%%%%%%%%%%%%%%%%%%%%%%%%%%%%%%%%%%%%%%%%%%%%%%%%%%%%%%%%%%%%%%%%%%%%%%%%%%%%%%%%%%
11
\begin{itemize}
\item (i)
Maturity values for high street banks and other banks
High street
Other banks
11.4
\item (ii)
11.5
11.6
11.7
11.8
11.9
12.0
12.1
12.2
x 1 : maturity value for high street bank
x 2 : maturity value for other bank
x 1 = 9.20
= 11.650
8
x 2 = 141.78
= 11.815
12
1 ⎛
9.20 2 ⎞
s 1 2 = ⎜ 1086.0470 −
⎟ = 0.0814
7 ⎜ ⎝
8 ⎟ ⎠
s 2 2 =
1 ⎛
141.78 2 ⎞
1675.8224
−
⎜
⎟ = 0.062882
11 ⎜ ⎝
12 ⎟ ⎠
Pooled estimate of $\sigma$:
s 2 p
( n 1 − 1) s 1 2 + ( n 2 − 1) s 2 2 7(0.0814) + 11(0.062882)
=
=
= 0.05261
n 1 + n 2 − 2
18
%%%%%%%%%%%%%%%%%%%%%%%%%%%%%%%%%%%%%%%%%%%%%%%%%%%%%%%%%%%%%%%%5
∴ s p = 0.208
95\% confidence interval for $\mu  1 − \mu  2$ is
11.650 − 11.815 \pm t 18 (21⁄2%) s p
= −0.165 \pm (2.101)(0.208)
1 1
+
8 12
1 1
+
8 12
%%--- Page 9Subject CT  — September 2006 — 
%%%%%%%%%%%%%%%%%%%%%%%%%%%%%%%%%

i.e. −0.165 \pm 0.221
i.e. (−0.86, 0.056)
i.e. the confidence interval for the difference between the means for high street banks and other banks (\mu  1 − \mu  2 ) is −\$86 to \$56.

\medskip

As zero is within the confidence interval, there is insufficient evidence, at 5\% level, to reject the null hypothesis that the mean maturity values do not differ
for the accounts offered by high street banks and other banks.
%%%%%%%%%%%%%%%%%%%%%%%%%%%%%%%%%%%%%%%%%%%%%%%%%%%%
\item (iii)
(a)
\[ \frac{ s^{2}_{2} }{ s^{2}_{1}}  \sim F 11,7 \]

under the assumption that the variances are equal for high street and
other banks,
i.e. \[H 0 : \sigma^{2}_{1} = \sigma^{2}_{2}\]

\[ \frac{ s^{2}_{2} }{ s^{2}_{1}} = \frac{0.0814}{0.062882} = 1.65 \]

%%%%%%%%%%%%%%%%%%%%%%%%%%%%%%%%%%



We cannot reject the null hypothesis at the 5\% level as the two-sided critical value of a 5\% level test is approximately 4.71 (by interpolation using 21⁄2\% one-sided F table in Yellow Book).
[OR probability value is p > 0.20 as a two-sided 20\% level test has a critical value of approximately 2.69.]
(b)
(iv)
\begin{itemize}
\item The plot in (i) indicates that the assumption of a normal distribution for maturity values is reasonable (but small samples) for both high street
and other banks. 
\item The assumption of equal variance also seems valid as the test in (iii)(a) is not significant (and the plot above supports this).
\item Adding points for building societies to previous plot in (i).
\end{itemize}
%%%%%%%%%%%%%%%%%%%%%%%%%%%%%%%%%%%%%%%%%%%%
Maturity values
High street
Other banks
Building soc
11.4
Page 10
11.9
12.4
%%-- Subject CT  — September 2006 — 
%%%%%%%%%%%%%%%%%%%%%%%%%%%%%%%%%

(v)
$\sigma x = 9.20 + 141.78 + 14.8 = 78.81$
$\sigma x^2 = 1086.0470 + 1675.8224 + 1724.2449 = 4486.114$
SS_{T}= 4486.114 −
78.81 2
= 1.82
2
9.20 2 141.78 2 14.8 2 78.81 2
+
+
−
= 0.551
8
12
12
2
∴ SS R = SS_{T}− SS B = 1.82 − 0.551 = 1.281
SS B =
\begin{verbatim}
Analysis of variance table
Source of variation
Financial institution types
Residual
Total
F =
df
2
29
1
SS
0.551
1.281
1.82
MSS
0.276
0.044
0.276
= 6.27 on (2, 29) degrees of freedom
0.044
F 2,29 (5%) = .28 and F 2,29 (1%) = 5.42    
\end{verbatim}

\begin{itemize}
\item Reject H 0 : $\mu_1 = \mu_2 = \mu$   (population means are equal) at 1\% level.
\item Strong evidence of differences between the  financial institutions.
\item The plot shows nothing strong enough to invalidate the assumptions of normality and equal variances, even though the variability for the building societies is a bit smaller than for the banks.
\end{itemize}
%%%%%%%%%%%%%%%%%%%%%%%%%%%%%%%%%%%%%%%%
(vi)
\begin{itemize}
\item part(ii) indicates that there are no differences between the mean maturity values of the two types of bank, but (v) indicates that there are differences between the mean maturity values of the  types of financial institution.
\item Therefore, in conclusion, it seems that the mean maturity value for building societies is not equal to the mean maturity values of the banks. 
\item Also, the plot in (iv) suggests that the maturity value for building societies is higher than the
mean maturity values for the banks.
\end{itemize}
\end{document}
%%%%%%%%%%%%%%%%%%%%%%%%%%%%%%%%%

