4
An actuary is considering statistical models for the observed number of claims, X,
which occur in a year on a certain class of non-life policies. The actuary only
considers policies o\documentclass[a4paper,12pt]{article}

%%%%%%%%%%%%%%%%%%%%%%%%%%%%%%%%%%%%%%%%%%%%%%%%%%%%%%%%%%%%%%%%%%%%%%%%%%%%%%%%%%%%%%%%%%%%%%%%%%%%%%%%%%%%%%%%%%%%%%%%%%%%%%%%%%%%%%%%%%%%%%%%%%%%%%%%%%%%%%%%%%%%%%%%%%%%%%%%%%%%%%%%%%%%%%%%%%%%%%%%%%%%%%%%%%%%%%%%%%%%%%%%%%%%%%%%%%%%%%%%%%%%%%%%%%%%

\usepackage{eurosym}
\usepackage{vmargin}
\usepackage{amsmath}
\usepackage{graphics}
\usepackage{epsfig}
\usepackage{enumerate}
\usepackage{multicol}
\usepackage{subfigure}
\usepackage{fancyhdr}
\usepackage{listings}
\usepackage{framed}
\usepackage{graphicx}
\usepackage{amsmath}
\usepackage{chngpage}

%\usepackage{bigints}
\usepackage{vmargin}

% left top textwidth textheight headheight

% headsep footheight footskip

\setmargins{2.0cm}{2.5cm}{16 cm}{22cm}{0.5cm}{0cm}{1cm}{1cm}

\renewcommand{\baselinestretch}{1.3}

\setcounter{MaxMatrixCols}{10}

\begin{document}
n which claims do actually arise. Among the considered models is
a model for which
P ( X = x ) = −
1
\theta  x
, x=1, 2, 3, ...
log(1 − \theta  ) x
where \theta  is a parameter such that 0 < \theta  < 1.
Suppose that the actuary has available a random sample X 1 , X 2 , ..., X n with sample
mean X .
(i)
Show that the method of moments estimator (MME),  \theta  , satisfies the equation
(
) (
)
X 1 −  \theta  log 1 −  \theta  +  \theta  = 0 .
(ii)
(a)
[3]
Show that the log likelihood of the data is given by
n
l ( \theta  ) ∝ − n log { − log ( 1 − \theta  ) } + ∑ x i log( \theta  ) .
i = 1
(b)
(iii)
5
Hence verify that the maximum likelihood estimator (MLE) of \theta  is the
same as the MME.
[4]
Suggest two ways in which the MLE of \theta  can be computed when a particular
data set is given.


%%%%%%%%%%%%%%%%%%%%%%%%%%%%%%%%%%%%%%%%%%%%%%%%%%%%%%%%%%%%%%%%%%%%%%%
4
(i)
First derive expected value:

E  X  =   x
x  =  1
1
 x
log(1   ) x

=


 x
1

   log(1   )  \theta  log(1   )

x  =  0 
= 

(1   ) log  1   
X  =  E  X   X  =  
 
(1   ) log(1    )

 

 X 1  
 log 1   \theta     =  0
(ii)
(a)
L     = 
  i
x i
(  log(1   )) n  x i
i
And l     =   n log   log  1    \theta   x i log   \theta  C
i
(b)
MLE given by:
dl   
d 
= 0 

n
 x i
 i = 0
 ˆ
log 1   ˆ 1   ˆ

 



 X 1   ˆ log 1   ˆ\theta   ˆ  =  0
(iii)
The equation above needs to be solved numerically. Alternatively, the
likelihood (or log-likelihood) function can be plotted and the maximum can be
identified from the graph.
In part (ii)(a) of the question the log-likelihood was shown as being equal, rather than
proportional, to the given expression plus a constant (as given in the solution above).
Candidates did not seem to be confused by this, but marking was adjusted in relevant cases.
In general the question was not particularly well answered, mainly due to difficulties in the
mathematical operations involved in obtaining the log-likelihood function of non-standard
densities. Candidates are advised to practise their calculus skills to deal with such questions.
Page 4Subject CT3 – September 2013 – Examiners’ Report
