\documentclass[a4paper,12pt]{article}

%%%%%%%%%%%%%%%%%%%%%%%%%%%%%%%%%%%%%%%%%%%%%%%%%%%%%%%%%%%%%%%%%%%%%%%%%%%%%%%%%%%%%%%%%%%%%%%%%%%%%%%%%%%%%%%%%%%%%%%%%%%%%%%%%%%%%%%%%%%%%%%%%%%%%%%%%%%%%%%%%%%%%%%%%%%%%%%%%%%%%%%%%%%%%%%%%%%%%%%%%%%%%%%%%%%%%%%%%%%%%%%%%%%%%%%%%%%%%%%%%%%%%%%%%%%%

\usepackage{eurosym}
\usepackage{vmargin}
\usepackage{amsmath}
\usepackage{graphics}
\usepackage{epsfig}
\usepackage{enumerate}
\usepackage{multicol}
\usepackage{subfigure}
\usepackage{fancyhdr}
\usepackage{listings}
\usepackage{framed}
\usepackage{graphicx}
\usepackage{amsmath}
\usepackage{chngpage}

%\usepackage{bigints}
\usepackage{vmargin}

% left top textwidth textheight headheight

% headsep footheight footskip

\setmargins{2.0cm}{2.5cm}{16 cm}{22cm}{0.5cm}{0cm}{1cm}{1cm}

\renewcommand{\baselinestretch}{1.3}

\setcounter{MaxMatrixCols}{10}

\begin{document}
\begin{enumerate}

%%8
\item A random sample of 10 independent claim amounts was taken from each of three
different regions and an analysis of variance was performed to compare the mean
level of claims in these regions. The resulting ANOVA table is given below.
\begin{verbatim}
Source
d.f.
SS
MSS
2
4,439.7
2,219.9
Between regions
27 10,713.5 396.8
Residual
29 15,153.2
Total
\end{verbatim}
\begin{enumerate}[(a)]
\item (i)
Perform the appropriate F test to determine whether there are significant
differences between the mean claim amounts for the three regions. You
should state clearly the hypotheses of the test.

The three sample means were:
Region
A
Sample mean 147.47
B
C
154.56 125.95
It was of particular interest to compare regions A and B.
\item (ii)
9
(a) Calculate a 95\% confidence interval for the difference between the
means for regions A and B.
(b) Comment on your answer in part (ii)(a) given the result of the F test
performed in part (i).

\end{enumerate}
%%%%%%%%%%%%%%%%%%%%%%%%%%%%%%%%%%%%%%%%%%%%%%%%%%%%%%%%%%%%%%%%%%%%%%%%%%%%%%%%%%%%%%%%%%%%%%%%%%%%%%%%%%%%%%%%%%%5
\newpage
8
(i)
\begin{itemize}
\item H 0 : The means of the claims in the 3 regions are all equal; H 1 : means are
different for at least one pair.
F = 5.59 on 2 and 27 d.f. From tables the 1\% critical point is 5.488.
Therefore, we have (strong) evidence against the null hypothesis, and conclude that there are differences in the means for the 3 regions.
\item (ii)
(a)
95\% CI for $\mu_A − \mu_B$ is given by
$( y A − y B ) \pm t 0.025, 27 \hat{\sigma}$
1 1
1 1
+
+
giving ( 147.47 − 154.56 ) \pm 2.052 396.8
10 10
10 10
i.e $− 7.09 \pm 18.28$ or $(–25.37, 11.19)$
(b)
\item The CI comfortably contains zero, suggesting no difference between
the true means for regions A and B.
\item The significant result of the F test clearly comes from region C mean being much lower than the means for regions A and B.
Generally well answered. Some candidates failed to identify the connection between the conclusion of the ANOVA and that of the CI for regions A and B.
\end{itemize}
Page 5 – %%%%%%%%%%%%%%%%%%%%%%%%%%%%%%%%%
\end{document}
