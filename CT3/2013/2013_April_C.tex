\documentclass[a4paper,12pt]{article}

%%%%%%%%%%%%%%%%%%%%%%%%%%%%%%%%%%%%%%%%%%%%%%%%%%%%%%%%%%%%%%%%%%%%%%%%%%%%%%%%%%%%%%%%%%%%%%%%%%%%%%%%%%%%%%%%%%%%%%%%%%%%%%%%%%%%%%%%%%%%%%%%%%%%%%%%%%%%%%%%%%%%%%%%%%%%%%%%%%%%%%%%%%%%%%%%%%%%%%%%%%%%%%%%%%%%%%%%%%%%%%%%%%%%%%%%%%%%%%%%%%%%%%%%%%%%

\usepackage{eurosym}
\usepackage{vmargin}
\usepackage{amsmath}
\usepackage{graphics}
\usepackage{epsfig}
\usepackage{enumerate}
\usepackage{multicol}
\usepackage{subfigure}
\usepackage{fancyhdr}
\usepackage{listings}
\usepackage{framed}
\usepackage{graphicx}
\usepackage{amsmath}
\usepackage{chngpage}

%\usepackage{bigints}
\usepackage{vmargin}

% left top textwidth textheight headheight

% headsep footheight footskip

\setmargins{2.0cm}{2.5cm}{16 cm}{22cm}{0.5cm}{0cm}{1cm}{1cm}

\renewcommand{\baselinestretch}{1.3}

\setcounter{MaxMatrixCols}{10}

\begin{document}
\begin{enumerate}

6
A survey is undertaken to investigate the proportion p of an adult population that
support a certain government policy. A random sample of 100 adults is taken and
contains 30 who support the policy.
(i) Calculate an approximate 95% confidence interval for p. [2]
(ii) Comment on the validity of the interval obtained in part (i). [1]
A different sample of 1,000 adults is taken and it contains 300 who support the policy.
(iii)
7
Explain how the width of a 95% confidence interval for p in this case will
compare to the width of the interval in part (i), without performing any
calculations.
[1]
[Total 4]
A regulator wishes to inspect a sample of an insurer’s claims. The insurer estimates
that 10% of policies have had one claim in the last year and no policies had more than
one claim. All policies are assumed to be independent.
(i)
Determine the number of policies that the regulator would expect to examine
before finding 5 claims.
[1]
On inspecting the sample claims, the regulator finds that actual payments exceeded
initial estimates by the following amounts:
£35
(ii)
£120
£48
£200
£76
Find the mean and variance of these extra amounts.
[3]
It is assumed that these amounts follow a gamma distribution with parameters α
and λ.
(iii)
CT3 A2013–3
Estimate these parameters using the method of moments.
[3]
[Total 7]
6
(i)
Using approximate normality, and with p ˆ = 0.3 we can calculate the interval a
⎛
p ˆ (1 − p ˆ )
p ˆ (1 − p ˆ )
, p ˆ + 1.96
⎜ ⎜ p ˆ − 1.96
n
n
⎝
⎞
⎟ ⎟ = (0.21, 0.39)
⎠
(ii) Sample size is large (or np, or np(1−p)), so normal approximation is valid.
(iii) With larger sample size the standard error will be smaller, and therefore the
interval will be narrower.
This was straightforward for most candidates. However, the explanation was often not clear
or convincing.
Page 4Subject CT3 (Probability and Mathematical Statistics) – %%%%%%%%%%%%%%%%%%%%%%%%%%%%%%%%%
7
(i) No of inspected policies ~ Negative binomial(5, 0.1).
Expected no of inspected policies = 5/0.1 =50
(ii) ∑ x = 479, ∑ x 2 = 63705
Mean = 479/5 = 95.8
Variance = (63705−5*95.8 2 )/4 = 4454.2
(iii)
α
α
= 95.8, V [ X ] = 2 = 4454.2
λ
λ
E [ X ]
95.8
⇒λ=
=
= 0.0215
V [ X ] 4454.2
E [ X ] =
⇒ α = λ E [ X ] = 0.0215*95.8 = 2.06
Generally very well answered with no particular issues.
