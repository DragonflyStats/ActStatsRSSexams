\documentclass[a4paper,12pt]{article}

%%%%%%%%%%%%%%%%%%%%%%%%%%%%%%%%%%%%%%%%%%%%%%%%%%%%%%%%%%%%%%%%%%%%%%%%%%%%%%%%%%%%%%%%%%%%%%%%%%%%%%%%%%%%%%%%%%%%%%%%%%%%%%%%%%%%%%%%%%%%%%%%%%%%%%%%%%%%%%%%%%%%%%%%%%%%%%%%%%%%%%%%%%%%%%%%%%%%%%%%%%%%%%%%%%%%%%%%%%%%%%%%%%%%%%%%%%%%%%%%%%%%%%%%%%%%

\usepackage{eurosym}
\usepackage{vmargin}
\usepackage{amsmath}
\usepackage{graphics}
\usepackage{epsfig}
\usepackage{enumerate}
\usepackage{multicol}
\usepackage{subfigure}
\usepackage{fancyhdr}
\usepackage{listings}
\usepackage{framed}
\usepackage{graphicx}
\usepackage{amsmath}
\usepackage{chngpage}

%\usepackage{bigints}
\usepackage{vmargin}

% left top textwidth textheight headheight

% headsep footheight footskip

\setmargins{2.0cm}{2.5cm}{16 cm}{22cm}{0.5cm}{0cm}{1cm}{1cm}

\renewcommand{\baselinestretch}{1.3}

\setcounter{MaxMatrixCols}{10}

\begin{document}
\begin{enumerate}

PLEASE TURN OVER11
The table below gives the frequency of a critical illness disease by age group in a
certain study. The table also gives the age midpoint (x), the number of people in each
⎛ \theta ˆ ⎞
, where \thetâ denotes the proportion in an age group with
group (n), and y = log ⎜ ⎜
ˆ ⎟ ⎟
⎝ 1 − \theta ⎠
the disease.
Age group
20–29
30–34
35–39
40–44
45–49
50–54
55–59
60–69
x
25
32.5
37.5
42.5
47.5
52.5
57.5
65
Contracted
disease?
Yes No
1
9
2
13
3
9
5
10
6
7
5
3
13
4
8
2
n
10
15
12
15
13
8
17
10
y
−2.19722
−1.87180
−1.09861
−0.69315
−0.15415
0.51083
1.17865
1.38629
∑x = 360; ∑x 2 = 17437.5; ∑y = −2.9392; ∑y 2 = 13.615; ∑xy= −9.0429
(i)
Calculate an estimate of the probability of having the disease under the assumption that the probability is the same for all age groups.
[1]
Consider the hypothesis that there are no differences in the probability of having the
disease for the different age groups.
(ii)
(a) Construct an 8 × 2 contingency table which includes the expected
frequencies under this hypothesis.
(b) Conduct a χ 2 test to investigate the hypothesis.
[6]
Consider the linear regression model y = α + \betax + ε, where the error terms (ε) are
independent and identically distributed following a N (0, \sigma 2 ) distribution.
(iii)
(a) Draw a scatterplot of y against x and comment on the appropriateness
of the considered model.
(b) Calculate the fitted regression line of y on x.
(c) Calculate a 99% confidence interval for the slope parameter.
(d) Interpret the result obtained in part (ii) with reference to the confidence
interval obtained in part (iii)(c).
[14]

%%%%%%%%%%%%%%%%%%%%%%%%%%%%%%%%%%%%%%%%%%%%%%%%%%%%%
11
(i)
Notation: n i = number in group i; r i = number with disease in group i.
θ ˆ =
(ii)
(a)
∑ r i = 43 = 0.43
∑ n i 100
Expected frequencies (in brackets) are given assuming constant
probability of disease for all groups, independently of age:
Disease
Age group Yes
No
Total
1
9
10
20−29
(4.30) (5.70)
2
13
15
30−34
(6.45) (8.55)
3
9
12
35−39
(5.16) (6.84)
5
10
15
40−44
(6.45) (8.55)
6
7
13
45−49
(5.59) (7.41)
5
3
8
50−54
(3.44) (4.56)
13
4
17
55−59
(7.31) (9.69)
8
2
10
60−69
(4.30) (5.70)
Total
43
57
100
Page 8Subject CT3 (Probability and Mathematical Statistics) – April 2013 – Examiners’ Report
χ = ∑
2
(b)
( f i − e i ) 2 = ( 1 − 4.3 ) 2 + ...+ ( 2 − 5.7 ) 2
4.3
e i
5.7
= 26.6 on 7 d.f.
2
From tables, χ 7,
0.01 = 18.48
We have (strong) evidence against the hypothesis of no differences in
probability of disease among age groups.
Plot given below. Linear model seems appropriate for middle ages,
but perhaps not for younger and older ages.
(a)
(iii)
30
40
50
60
x
(b)
360 2
= 1237.5
8
S xx = 17437.5 −
S yy = 13.615 −
( − 2.9392) 2
= 12.535
8
S xy = − 9.0429 −
( 360 ) ( − 2.9392) = 123.22
8
Least squares estimates:
β ˆ =
S xy
S xx
=
123.22
= 0.09957
1237.5
α ˆ = y − β ˆ x = − 0.3674 − 0.09957 ( 45 ) = − 4.85
Fitted line: y ˆ = − 4.85 + 0.09957 x
Page 9Subject CT3 (Probability and Mathematical Statistics) – April 2013 – Examiners’ Report
(c)
⎛
2 ⎞
S xy 2 ⎞ ⎛
⎜ S yy −
⎟ ⎜ 12.535 − 123.22 ⎟
⎜
S xx ⎟ ⎜
1237.5 ⎟ ⎠
⎠ = ⎝
σ ˆ 2 = ⎝
= 0.04430
n − 2
6
()
se β ˆ =
σ ˆ 2
0.210
=
= 0.0060
S xx
1237. 5
and t 6, 0.005 = 3.707
99% CI for β̂ is given by 0.09957 ± 3.707(0.0060)
i.e. 0.09957 ± 0.0222 or (0.0774, 0.1218)
(d)
In (ii) it was found that the probability of having the disease is
different for different age groups. In part (iii)(c) it was also found that
the probability of disease depends on age, as zero was not included in
the interval for the slope parameter.
The quality of the answers was mixed, with some common errors appearing in part (ii)
where many candidates failed to produce an appropariate 8 × 2 table (both “yes” and “no”
columns) and perform the correct chi-square test (with 7 d.f.).
It is noted that in part (ii)(b) the chi-square test can alternatively be performed by combining
some of the age groups (both columns) to achieve expected frequencies greater than 5, with
no change in the conclusion of the test. Although this is not strictly required in this case, full
credit was given to candidates that combined groups sensibly and completed the question
correctly.
END OF EXAMINERS’ REPORT
Page 10
\end{document}
