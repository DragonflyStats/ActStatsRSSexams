\documentclass[a4paper,12pt]{article}

%%%%%%%%%%%%%%%%%%%%%%%%%%%%%%%%%%%%%%%%%%%%%%%%%%%%%%%%%%%%%%%%%%%%%%%%%%%%%%%%%%%%%%%%%%%%%%%%%%%%%%%%%%%%%%%%%%%%%%%%%%%%%%%%%%%%%%%%%%%%%%%%%%%%%%%%%%%%%%%%%%%%%%%%%%%%%%%%%%%%%%%%%%%%%%%%%%%%%%%%%%%%%%%%%%%%%%%%%%%%%%%%%%%%%%%%%%%%%%%%%%%%%%%%%%%%

\usepackage{eurosym}
\usepackage{vmargin}
\usepackage{amsmath}
\usepackage{graphics}
\usepackage{epsfig}
\usepackage{enumerate}
\usepackage{multicol}
\usepackage{subfigure}
\usepackage{fancyhdr}
\usepackage{listings}
\usepackage{framed}
\usepackage{graphicx}
\usepackage{amsmath}
\usepackage{chngpage}

%\usepackage{bigints}
\usepackage{vmargin}

% left top textwidth textheight headheight

% headsep footheight footskip

\setmargins{2.0cm}{2.5cm}{16 cm}{22cm}{0.5cm}{0cm}{1cm}{1cm}

\renewcommand{\baselinestretch}{1.3}

\setcounter{MaxMatrixCols}{10}

\begin{document}
\begin{enumerate}

[Total 17]
CT3 A2013–5
PLEASE TURN OVER10
The random variable S represents the annual aggregate claims for an insurer from
policies covering damage due to windstorms. S is modelled as follows:
M
S = ∑ Y i
i = 1
where:
M denotes the number of windstorms each year and has a Poisson distribution
with mean κ
Y i denotes the aggregate claims from the ith windstorm and is modelled as
N i
Y i = ∑ X ij
j = 1
where:
N i denotes the number of claims from the ith
windstorm.
N 1 , N 2 , ... , N M are independent and identically distributed
random variables, each with a Poisson
distribution with rate λ.
X ij denotes the amount of the jth claim from the
ith windstorm.
X ij , i = 1, ..., M, j = 1, ..., N i
is a sequence of independent and identically
distributed random variables, each with mean
μ and variance \sigma 2 .
It is assumed that the random variables M, N i and X ij are independent of each other.
(i) Derive expressions for the mean and the variance of Y i in terms of λ, μ and \sigma.
[2]
(ii) Derive expressions for the mean and the variance of S in terms of κ, λ, μ
and \sigma.
[3]
Now suppose that X ij has an exponential distribution with mean 1.
(iii)
Show that for any positive numbers x and C
P ( X ij ≤ x + C | X ij > C ) = P ( X ij ≤ x ) .
CT3 A2013–6
[3]Consider the new random variable S R given as:
M N i
S R = ∑∑ X ij *
i = 1 j = 1
⎪ X ij − 2
where: X ij * = ⎨
⎩ 0
if X ij ≥ 2
otherwise
.
Let N i * be the number of non-zero X ij * amounts, i.e. the number of claim amounts
from the ith windstorm that are greater than 2.
*
are independent and identically distributed Poisson
Also assume that N 1 * , N 2 * , ... , N M
random variables, with parameter λ * .
Let κ = 4, λ = 1,000.
(iv)
CT3 A2013–7
(a) Show that λ * = 135.3.
(b) Explain why the distribution of X ij * is exponential with mean 1.
(c) Calculate the mean and variance of S R .
[7]
[Total15]
