\documentclass[a4paper,12pt]{article}

%%%%%%%%%%%%%%%%%%%%%%%%%%%%%%%%%%%%%%%%%%%%%%%%%%%%%%%%%%%%%%%%%%%%%%%%%%%%%%%%%%%%%%%%%%%%%%%%%%%%%%%%%%%%%%%%%%%%%%%%%%%%%%%%%%%%%%%%%%%%%%%%%%%%%%%%%%%%%%%%%%%%%%%%%%%%%%%%%%%%%%%%%%%%%%%%%%%%%%%%%%%%%%%%%%%%%%%%%%%%%%%%%%%%%%%%%%%%%%%%%%%%%%%%%%%%

\usepackage{eurosym}
\usepackage{vmargin}
\usepackage{amsmath}
\usepackage{graphics}
\usepackage{epsfig}
\usepackage{enumerate}
\usepackage{multicol}
\usepackage{subfigure}
\usepackage{fancyhdr}
\usepackage{listings}
\usepackage{framed}
\usepackage{graphicx}
\usepackage{amsmath}
\usepackage{chngpage}

%\usepackage{bigints}
\usepackage{vmargin}

% left top textwidth textheight headheight

% headsep footheight footskip

\setmargins{2.0cm}{2.5cm}{16 cm}{22cm}{0.5cm}{0cm}{1cm}{1cm}

\renewcommand{\baselinestretch}{1.3}

\setcounter{MaxMatrixCols}{10}

\begin{document}


%%- Question 6
A survey is undertaken to investigate the proportion p of an adult population that
support a certain government policy. A random sample of 100 adults is taken and
contains 30 who support the policy.
\begin{enumerate}[(a)]
\item (i) Calculate an approximate 95\% confidence interval for p. 
\item (ii) Comment on the validity of the interval obtained in part (i). 
A different sample of 1,000 adults is taken and it contains 300 who support the policy.
\item (iii)

Explain how the width of a 95\% confidence interval for p in this case will
compare to the width of the interval in part (i), without performing any
calculations.
\end{enumerate}



%%%%%%%%%%%%%%%%%%%%%%%%%%%%%%%%%%%%
6
(i)
Using approximate normality, and with \hat{p} = 0.3 we can calculate the interval a
⎛
\hat{p} (1 − \hat{p} )
\hat{p} (1 − \hat{p} )
, \hat{p} + 1.96
⎜ ⎜ \hat{p} − 1.96
n
n
⎝
⎞
⎟ ⎟ = (0.21, 0.39)
⎠
\begin{itemize}
    \item (ii) Sample size is large (or np, or np(1−p)), so normal approximation is valid.
\item (iii) With larger sample size the standard error will be smaller, and therefore the
interval will be narrower.
\item This was straightforward for most candidates. However, the explanation was often not clear
or convincing.
\end{itemize}

\end{document}
