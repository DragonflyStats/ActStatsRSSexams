5 A large portfolio consists of 20% class A policies, 50% class B policies and 30% class
C policies. Ten policies are selected at random from the portfolio.
(i) Calculate the probability that there are no policies of class A among the
randomly selected ten. [1]
(ii) (a) Calculate the expected number of class B policies among the randomly
selected ten.
(b) Calculate the probability that there are more than five class B policies
among the randomly selected ten.
[2]
[Total 3]
CT3 S2012–3 PLEASE TURN OVER
6 A random sample of size n is taken from a gamma distribution with parameters α = 8
and λ = 1/θ. The sample mean is X and θ is to be estimated.
(i) Determine the method of moments estimator (MME) of θ. [2]
(ii) Find the bias of the MME determined in part (i). [2]
(iii) (a) Determine the mean square error of the MME of θ.
(b) Comment on the efficiency of the MME of θ based on your answer in
part (iii)(a).
[3]


5 (i) P(none of class A) = P(all 10 of class B or C) = (0.8)10 = 0.1074
(ii) (a) Let B = number of class B.
Note that B ~ binomial (10, 0.5), so that E(B) = (10)(0.5) = 5
(b) P(B > 5) = 1 – P( B ≤ 5 ) = 1 – 0.6230 = 0.3770
[0.6230 is from tables; alternatively by evaluation]
This was generally very well answered. A common error in part (ii) (b) was to calculate
P(B < or = 5) instead of P(B > 5).
Subject CT3 (Probability and Mathematical Statistics) – September 2012 – Examiners’ Report
Page 5
6 (i) Population mean = 8θ
So MME is solution of X = 8θ ⇒ MME
8
= X
(ii) 1 ( ) 1 (8 )
8 8 8
E X E X
⎛ ⎞
⎜ ⎟ = = θ = θ
⎝ ⎠
Bias = 0
8
E X
⎛ ⎞
⎜ ⎟ − θ =
  ⎝ ⎠
(i.e. MME is unbiased for θ).
(iii) (a) Since MME is unbiased, 􀜯􀜵􀜧 􁉀􀯑􀴤
􀬼􁉁 􀵌 􀝒􀜽􀝎 􁉀􀯑􀴤
􀬼􁉁 􀵌 􀬼􀰏􀰮
􀬺􀬸􀯡 􀵌 􀰏􀰮
􀬼􀯡
(b) MME gets more efficient (MSE gets smaller) as sample size increases.
There was a mix of quality in the answers, especially in parts (ii) and (iii). Attention to detail
is required when determining the expected value and variance of functions of sample
statistics (here the sample mean).
