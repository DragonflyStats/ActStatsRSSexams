\documentclass[a4paper,12pt]{article}

%%%%%%%%%%%%%%%%%%%%%%%%%%%%%%%%%%%%%%%%%%%%%%%%%%%%%%%%%%%%%%%%%%%%%%%%%%%%%%%%%%%%%%%%%%%%%%%%%%%%%%%%%%%%%%%%%%%%%%%%%%%%%%%%%%%%%%%%%%%%%%%%%%%%%%%%%%%%%%%%%%%%%%%%%%%%%%%%%%%%%%%%%%%%%%%%%%%%%%%%%%%%%%%%%%%%%%%%%%%%%%%%%%%%%%%%%%%%%%%%%%%%%%%%%%%%

\usepackage{eurosym}
\usepackage{vmargin}
\usepackage{amsmath}
\usepackage{graphics}
\usepackage{epsfig}
\usepackage{enumerate}
\usepackage{multicol}
\usepackage{subfigure}
\usepackage{fancyhdr}
\usepackage{listings}
\usepackage{framed}
\usepackage{graphicx}
\usepackage{amsmath}
\usepackage{chngpage}

%\usepackage{bigints}
\usepackage{vmargin}

% left top textwidth textheight headheight

% headsep footheight footskip

\setmargins{2.0cm}{2.5cm}{16 cm}{22cm}{0.5cm}{0cm}{1cm}{1cm}

\renewcommand{\baselinestretch}{1.3}

\setcounter{MaxMatrixCols}{10}

\begin{document}
\begin{enumerate}


CT3 S2013–24
An actuary is considering statistical models for the observed number of claims, X,
which occur in a year on a certain class of non-life policies. The actuary only
considers policies on which claims do actually arise. Among the considered models is
a model for which
P ( X = x ) = −
1
\theta x
, x=1, 2, 3, ...
log(1 − \theta ) x
where \theta is a parameter such that 0 < \theta < 1.
Suppose that the actuary has available a random sample X 1 , X 2 , ..., X n with sample
mean X .
(i)
Show that the method of moments estimator (MME),  \theta , satisfies the equation
(
) (
)
X 1 −  \theta log 1 −  \theta +  \theta = 0 .
(ii)
(a)
[3]
Show that the log likelihood of the data is given by
n
l ( \theta ) ∝ − n log { − log ( 1 − \theta ) } + \sum  x i log( \theta ) .
i = 1
(b)
(iii)
5
Hence verify that the maximum likelihood estimator (MLE) of \theta is the
same as the MME.

Suggest two ways in which the MLE of \theta can be computed when a particular
data set is given.

[Total 8]
%%%%%%%%%%%%%%%%%%%%%%%%%%%%%%5
Consider a random sample consisting of the random variables X 1 , X 2 ,..., X n with
mean \mu  and variance σ 2 . The variables are independent of each other.
(i)
Show that the sample variance, S 2 , is an unbiased estimator of the true
variance σ 2 .
[3]
Now consider in addition that the random sample comes from a normal distribution,
( n − 1) S 2
in which case it is known that
~ χ n 2 − 1 .
2
σ
(ii)
CT3 S2013–3
(a) Derive the variance of S 2 in terms of σ and n.
(b) Comment on the quality of the estimator S 2 with respect to the sample
size n.

[Total 7]
%%%%%%%%%%%%%%%%%%%%%%%%%%%%%%%%%%%
6
(i)
H 0 = variances are the same, H 1 = variances are different
S 2 / S 1 ~ F 24,24
Test statistic = 9.21/2.86 = 3.22.
F 24,24,0.995 = 0.337 and F 24,24,0.005 = 2.967
i.e. reject H 0 at 1% significance level.
(ii)
  n  1  S 2  n  1  S 2
Confidence interval is given by  2
, 2
 X 0.025, n  1 X 0.975,
n  1





2
2
X 0.975,24
 =  12.40, X 0.025,24
 =  39.36
Confidence interval 1 = (1.74, 5.54)
Confidence interval 2 = (5.61, 17.83)
Page 5Subject CT3 – September 2013 – Examiners’ Report
(iii)
Confidence intervals don’t overlap i.e. agree with result in (i) that variances
are different.
Generally well answered. In part (i) some candidates worked with the S 1 /S 2 ratio, which of
course gives the same conclusion. Part (ii) requires the calculation of two CIs, but some
candidates attempted to provide a CI for the ratio.
7
(i)
P (claim ) = P (claim| A ) P (A )  P (claim| B ) P (B )  P (claim| C ) P (C 
= 0.15 \times 0.2  0.1 \times 0.2  0.05 \times 0.6  =  0.08
(ii)
P (claim \cup A )  =  P (claim| A ) P [ A ]  =  0.15 \times 0.2  =  0.03
P (claim \cup B )  =  P (claim| B ) P [ B ]  =  0.1 \times 0.2  =  0.02
P ( claim \cup \bar{C} )  =  0.03  0.02  =  0.05
P ( claim | \bar{C} )  = 
P ( claim \cup \bar{C} )
P ( \bar{C} )
(iii) P (A |claim last year )  = 
(iv) P (B |claim last year )  = 
 = 
0.05
 =  0.125
0.4
P (claim \cup A 
P (claim 
P (claim \cup B 
P (claim 
 =  0.03
 =  0.375
0.08
 =  0.02
 =  0.25
0.08
P (C |claim last year )  =  1  0.375  0.25  =  0.375
(CLY means “claim last year”)
P (claim|claim last year ) = P (claim| A ) P (A | CLY )  P (claim| C ) P (C | CLY 
 =  0.15 \times 0.375  0.1 \times 0.25  0.05 \times 0.375  =  0.1
(v)
Let Y be the event that a claim is submitted in two consecutive years
P (Y )  =  P (claim in second year|claim in first year ) P (claim in first year 
 =  0.1 \times 0.08  =  0.008
alternatively:
P (Y ) = P (Y | A ) P (A )  P (Y | B ) P (B )  P (Y | C ) P (C 
= 0.15 \times 0.15 \times 0.2  0.1 \times 0.1 \times 0.2  0.05 \times 0.05 \times 0.6  =  0.008
This turned out to be the most challenging question for the majority of candidates, with only
a small number of “full mark” answers. Many candidates did not attempt parts (iv) and (v) at
Page 6Subject CT3 – September 2013 – Examiners’ Report
all. The question deals with conditional probability concepts, starting with straightforward
parts but building up to more complex calculations .
