\documentclass[a4paper,12pt]{article}

%%%%%%%%%%%%%%%%%%%%%%%%%%%%%%%%%%%%%%%%%%%%%%%%%%%%%%%%%%%%%%%%%%%%%%%%%%%%%%%%%%%%%%%%%%%%%%%%%%%%%%%%%%%%%%%%%%%%%%%%%%%%%%%%%%%%%%%%%%%%%%%%%%%%%%%%%%%%%%%%%%%%%%%%%%%%%%%%%%%%%%%%%%%%%%%%%%%%%%%%%%%%%%%%%%%%%%%%%%%%%%%%%%%%%%%%%%%%%%%%%%%%%%%%%%%%

\usepackage{eurosym}
\usepackage{vmargin}
\usepackage{amsmath}
\usepackage{graphics}
\usepackage{epsfig}
\usepackage{enumerate}
\usepackage{multicol}
\usepackage{subfigure}
\usepackage{fancyhdr}
\usepackage{listings}
\usepackage{framed}
\usepackage{graphicx}
\usepackage{amsmath}
\usepackage{chngpage}

%\usepackage{bigints}
\usepackage{vmargin}

% left top textwidth textheight headheight

% headsep footheight footskip

\setmargins{2.0cm}{2.5cm}{16 cm}{22cm}{0.5cm}{0cm}{1cm}{1cm}

\renewcommand{\baselinestretch}{1.3}

\setcounter{MaxMatrixCols}{10}

\begin{document}
\begin{enumerate}

%%--- Question 10
The random variable $S$ represents the annual aggregate claims for an insurer from
policies covering damage due to windstorms. S is modelled as follows:
M
S = ∑ Y i
i = 1
where:
M denotes the number of windstorms each year and has a Poisson distribution
with mean \kappa 
Y i denotes the aggregate claims from the ith windstorm and is modelled as
N i
Y i = ∑ X_{ij}
j = 1
where:
N i denotes the number of claims from the ith
windstorm.
N 1 , N 2 , ... , N M are independent and identically distributed
random variables, each with a Poisson
distribution with rate λ.
X_{ij} denotes the amount of the jth claim from the
ith windstorm.
X_{ij} , i = 1, ..., M, j = 1, ..., N i
is a sequence of independent and identically
distributed random variables, each with mean
\mu  and variance \sigma 2 .
It is assumed that the random variables M, N i and X_{ij} are independent of each other.
\begin{enumerate}[(i)]
\item  Derive expressions for the mean and the variance of Y i in terms of λ, \mu  and \sigma.

\item  Derive expressions for the mean and the variance of S in terms of \kappa , λ, \mu 
and \sigma.

Now suppose that X_{ij} has an exponential distribution with mean 1.
\item 
Show that for any positive numbers x and C
P ( X_{ij} \leq x + C | X_{ij} > C ) = P ( X_{ij} \leq  x ) .

%%--- CT3 A2013–6
Consider the new random variable S R given as:
M N i
S R = ∑∑ X_{ij} *
i = 1 j = 1
⎪ X_{ij} − 2
where: X_{ij} * = ⎨
⎩ 0
if X_{ij} ≥ 2
otherwise
.
Let N i * be the number of non-zero X_{ij} * amounts, i.e. the number of claim amounts
from the ith windstorm that are greater than 2.
*
are independent and identically distributed Poisson
Also assume that N 1 * , N 2 * , ... , N M
random variables, with parameter \lambda^{\ast} .
Let \kappa  = 4, \lambda = 1,000.
\item %%--- (iv)
CT3 A2013–7
(a) Show that \lambda^{\ast} = 135.3.
(b) Explain why the distribution of X_{ij} * is exponential with mean 1.
(c) Calculate the mean and variance of S R .

[Total15]
\end{enumerate}

10
(i)
Y i has a compound distribution, so
( )
E ( Y i ) = E ( N i ) E X_{ij} = λ\mu 
( )
V ( Y i ) = E ( N i ) V X_{ij} + V ( N i ) E ( X_{ij} ) 2 = λ\sigma^2 + λ\mu  2
(ii)
S also has a compound distribution.
E ( S ) = E ( M ) E ( Y i ) = \kappa λ\mu 
V ( S ) = E ( M ) V ( Y i ) + V ( M ) E ( Y i ) 2 = \kappa \lambda ( \sigma^2 + \mu  2 ) + \kappa \lambda 2 \mu  2 = \kappa \lambda ( \sigma^2 + \mu  2 + λ\mu  2 )
(iii)
P ( X_{ij} \leq  x + C | X_{ij} > C ) =
( 1 − e
=
− x − C
e
(
P C < X_{ij} \leq  x + C
− 1 + e − C
− C
(
P X_{ij} > C
) = 1 − e
)
)
− x
= P ( X_{ij} \leq  x )
%% Page 7Subject CT3 (Probability and Mathematical Statistics) – April 2013 – Examiners’ Report
(iv)
(
)
(a) \lambda^{\ast} = 1000 × P X_{ij} > 2 = 1000 e − 2 = 135.3
(b) From definition of new variable and part (iii) we have that
(
)
P X_{ij} * \leq  x = P ( X_{ij} − 2 \leq  x | X_{ij} > 2) = P ( X_{ij} \leq  x + 2 | X_{ij} > 2) = P ( X_{ij} \leq  x )
meaning that X_{ij} * has the same distribution as X_{ij} , i.e. Exp(1).
(c)
E ( S R ) = \kappa \lambda^{\ast} \mu  = 4 × 135.3 × 1 = 541.2
V ( S R ) = \kappa \lambda^{\ast} ( \sigma^2 + \mu  2 + \lambda^{\ast} \mu  2 ) = 4 × 135.3 × ( 1 + 1 + 135.3 ) = 74306.8

%-------------------------------------------------------%
Most candidates found this question challenging. Answers to the memoryless property of the
exponential distribution (amply discussed in the CR) in part (iii) were often disappointing,
and the relevant application in part (iv) was poorly attempted. These shortcomings highlight
the issue of being prepared to tackle questions that deviate from the form that appears in
past papers.
\end{document}
\end{document}
