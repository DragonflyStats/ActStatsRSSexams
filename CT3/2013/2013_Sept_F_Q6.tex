\documentclass[a4paper,12pt]{article}

%%%%%%%%%%%%%%%%%%%%%%%%%%%%%%%%%%%%%%%%%%%%%%%%%%%%%%%%%%%%%%%%%%%%%%%%%%%%%%%%%%%%%%%%%%%%%%%%%%%%%%%%%%%%%%%%%%%%%%%%%%%%%%%%%%%%%%%%%%%%%%%%%%%%%%%%%%%%%%%%%%%%%%%%%%%%%%%%%%%%%%%%%%%%%%%%%%%%%%%%%%%%%%%%%%%%%%%%%%%%%%%%%%%%%%%%%%%%%%%%%%%%%%%%%%%%

\usepackage{eurosym}
\usepackage{vmargin}
\usepackage{amsmath}
\usepackage{graphics}
\usepackage{epsfig}
\usepackage{enumerate}
\usepackage{multicol}
\usepackage{subfigure}
\usepackage{fancyhdr}
\usepackage{listings}
\usepackage{framed}
\usepackage{graphicx}
\usepackage{amsmath}
\usepackage{chngpage}

%\usepackage{bigints}
\usepackage{vmargin}

% left top textwidth textheight headheight

% headsep footheight footskip

\setmargins{2.0cm}{2.5cm}{16 cm}{22cm}{0.5cm}{0cm}{1cm}{1cm}

\renewcommand{\baselinestretch}{1.3}

\setcounter{MaxMatrixCols}{10}

\begin{document}
\begin{enumerate}

%%-- PLEASE TURN OVER
%%-- Question 6
\item A researcher obtains samples of 25 items from normally distributed measurements
from each of two factories. The sample variances are 2.86 and 9.21 respectively.
\begin{enumerate}[(i)]
\item Perform a test to determine if the true variances are the same. 
\item  For each factory calculate central 95\% confidence intervals for the true
variances of the measurements. 
\item 
Comment on how your answers in parts (i) and (ii) relate to each other.
\end{enumerate}
%%%%%%%%%%%%%%%%%%%%%%%%%%%%%%%%%%%%%%%%%%%%%%%%%%
\newpage
6
\begin{itemize}
\item (i)
H 0 = variances are the same, H 1 = variances are different
S 2 / S 1 ~ F 24,24
Test statistic = 9.21/2.86 = 3.22.
F 24,24,0.995 = 0.337 and F 24,24,0.005 = 2.967
i.e. reject H 0 at 1\% significance level.
\item (ii)
  n  1  S 2  n  1  S 2
Confidence interval is given by  2
, 2
 X 0.025, n  1 X 0.975,
n  1





2
2
X 0.975,24
 =  12.40, X 0.025,24
 =  39.36
Confidence interval 1 = (1.74, 5.54)
Confidence interval 2 = (5.61, 17.83)
%%-- Page 5Subject CT3 – September 2013 – Examiners’ Report
\item (iii)
Confidence intervals don’t overlap i.e. agree with result in (i) that variances
are different.

Generally well answered. In part (i) some candidates worked with the S 1 /S 2 ratio, which of course gives the same conclusion. Part (ii) requires the calculation of two CIs, but some
candidates attempted to provide a CI for the ratio.
\end{itemize}
\end{document}
