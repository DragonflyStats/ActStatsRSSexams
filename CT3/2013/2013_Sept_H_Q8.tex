\documentclass[a4paper,12pt]{article}

%%%%%%%%%%%%%%%%%%%%%%%%%%%%%%%%%%%%%%%%%%%%%%%%%%%%%%%%%%%%%%%%%%%%%%%%%%%%%%%%%%%%%%%%%%%%%%%%%%%%%%%%%%%%%%%%%%%%%%%%%%%%%%%%%%%%%%%%%%%%%%%%%%%%%%%%%%%%%%%%%%%%%%%%%%%%%%%%%%%%%%%%%%%%%%%%%%%%%%%%%%%%%%%%%%%%%%%%%%%%%%%%%%%%%%%%%%%%%%%%%%%%%%%%%%%%

\usepackage{eurosym}
\usepackage{vmargin}
\usepackage{amsmath}
\usepackage{graphics}
\usepackage{epsfig}
\usepackage{enumerate}
\usepackage{multicol}
\usepackage{subfigure}
\usepackage{fancyhdr}
\usepackage{listings}
\usepackage{framed}
\usepackage{graphicx}
\usepackage{amsmath}
\usepackage{chngpage}

%\usepackage{bigints}
\usepackage{vmargin}

% left top textwidth textheight headheight

% headsep footheight footskip

\setmargins{2.0cm}{2.5cm}{16 cm}{22cm}{0.5cm}{0cm}{1cm}{1cm}

\renewcommand{\baselinestretch}{1.3}

\setcounter{MaxMatrixCols}{10}

\begin{document}
\begin{enumerate}
CT3 S2013–48
The follow ing graph sh hows the number of policyholders s who made e 0, 1, 2, 3 or
o 4
claims during the last year
y in a group of 100 policyholde ers.
(i)
Calculate the s ample mean n, median, mode
m
and standard deviation of th e
num
mber of claim
ms per polic cyholder.
[5]
Assume that the numb ber of claims X per policyholder per
p annum f from this gr roup of
λ .
policyholders has a Po oisson distribution with
h unknown parameter
p
(ii)
Calculate an ap pproximate 95\% confid
dence interval for the u unknown parameter
λ using
u
the data in the abo ove graph, justifying
j
th he validity o of your appr roach.

m size for eac ch group of f number of f claims
The follow ing table sh hows the ave erage claim
that a polic yholder ma ade during th he last year .
Number of f claims per policyholde er
Average cl aim size (£) )
0
--
1
2
1000 1100
3
930 0
4
980
Assume tha at the claim size is inde ependent of the number of claims, and that
policyholders make cla aims indepe endently. Also assume that the size e of each cl laim is
normally distributed with
w estimated standard
ddeviation s = £120.
(iii) Estimate the expected size e of a single e claim.

(iv) State the type of the distrib bution of the total amo ount claimed in the group of the 100 policyholders.

Now assum
me that the number
n
of claims
c
per policyholder has a Poiss son distribution
with true parameter λ = 1.15 and that the true expected value of the size of a single claim is £1,010 and its true standaard deviation is £120.
(v)
CT3 S2 2013–5
Calculate the expected
e
value of the total amount claimed in the group of the
100 0 policyholders and its standard deviation.


%%%%%%%%%%%%%%%%%%%%
8
(i)
Mean = (0 \times 40 + 1 \times 25 +2 \times 20+3 \times 10+4 \times 5)/100 = (25+40+30+20)/100 = 1.15
Median = 1
Mode = 0
VAR = [ (−1.15) 2  \times 40 + (−0.15) 2  \times 25 + 0.85 2  \times 20+ 1.85 2  \times 10 + 2.85 2  \times 5]/99
= 1.4419
STD = 1.2
(ii)
The estimate for the expectation of X is ̂ =1.15
n ˆ =115 is rather large and we can therefore, use a normal approximation to
calculate the confidence interval.
1.15  1.96
(iii)
1.15
 =   0.9398,1 .3602 
100
Total amount of claims = 25 \times 1 \times 1000 + 20 \times 2 \times 1100 +10 \times 3 \times 930 +5 \times  4 \times 980
= 116,500
Average claim size = 116500/115 = 1,013.043
(iv) Compound Poisson
(v) Using standard results on compound distributions:
Expected value: E  =  100 \times    \times  1010 = 115  \times  1010 = 116,150.00
Var = 115 \times 120 2 + 115 \times 1010 2 = 118,967,500
STD = 10,907.22
Generally well done, but some mixed performance in parts (ii) and (v). Note that calculations
refer to a group of 100 policyholders – some candidates failed to take this into account.
\end{document}
