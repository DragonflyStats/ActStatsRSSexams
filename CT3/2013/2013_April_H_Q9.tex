\documentclass[a4paper,12pt]{article}

%%%%%%%%%%%%%%%%%%%%%%%%%%%%%%%%%%%%%%%%%%%%%%%%%%%%%%%%%%%%%%%%%%%%%%%%%%%%%%%%%%%%%%%%%%%%%%%%%%%%%%%%%%%%%%%%%%%%%%%%%%%%%%%%%%%%%%%%%%%%%%%%%%%%%%%%%%%%%%%%%%%%%%%%%%%%%%%%%%%%%%%%%%%%%%%%%%%%%%%%%%%%%%%%%%%%%%%%%%%%%%%%%%%%%%%%%%%%%%%%%%%%%%%%%%%%

\usepackage{eurosym}
\usepackage{vmargin}
\usepackage{amsmath}
\usepackage{graphics}
\usepackage{epsfig}
\usepackage{enumerate}
\usepackage{multicol}
\usepackage{subfigure}
\usepackage{fancyhdr}
\usepackage{listings}
\usepackage{framed}
\usepackage{graphicx}
\usepackage{amsmath}
\usepackage{chngpage}

%\usepackage{bigints}
\usepackage{vmargin}

% left top textwidth textheight headheight

% headsep footheight footskip

\setmargins{2.0cm}{2.5cm}{16 cm}{22cm}{0.5cm}{0cm}{1cm}{1cm}

\renewcommand{\baselinestretch}{1.3}

\setcounter{MaxMatrixCols}{10}

\begin{document}


[Total 8]
A behavioural scientist is observing a troop of monkeys and is investigating whether
social status affects the amount of food that an individual takes. The monkeys are
divided into two groups of different social rank and the scientist counts the number of
bananas each individual takes. Each monkey can take a maximum of 7 bananas.
Social rank
Number of monkeys
Total bananas taken
(i)
A
6
33
B
11
37
It is first suggested that the number of bananas taken by each individual of each group follows the same binomial distribution with common parameter p
and n=7.
(a) Use the method of moments to estimate the parameter p.
(b) The scientist is unsure whether a common parameter is appropriate and wishes to compare p A and p B , the probability that a banana is taken
by an individual in groups A and B respectively.
Test the hypothesis that p A = p B .
[7]
%%%%%%%%%%%%%%%%%%%%%%%%%%%%%%%%%%%%%%%%%%%%%%%%%%%%%%%%%%%%%%%%%%%%%%%%%%%%%%%%%%%%%%%%%%%%%%%%%%%%%%%%%%%%%%%%%%%%%%%%%%%%%%%%%%%%%%%%%%%%%
\item 
A statistician suggests an alternative model. The number of bananas taken by
an individual still follows a binomial distribution with n=7, but for group A
the parameter is 2\theta and for group B the parameter is $\theta$, where $\theta < 0.5$.
(a)
Show that the log likelihood for \theta is given by:
33
\[\ln ( 2 \theta ) + 9 \ln ( 1 − 2 \theta ) + 37 \ln ( \theta ) + 40 \ln ( 1 − \theta ) + constant\]
(b)
Hence calculate the maximum likelihood estimate of0 $\theta$.

(iii)
(a) Compare the fit of the two suggested models in parts (i) (with common
parameter p) and (ii) by considering the expected number of bananas
taken in groups A and B under the two models. You are not required
to perform a formal test.
(b) Comment on the above comparison in relation to your answer in
part (i)(b).

%%%%%%%%%%%%%%%%%%%%%%%%%%%%%%%%%%%%%%%%%%%%%%%%%%%%%%%%%%%%%%%%%%%%%%%%%%%%%%%%%%%%%%%%%%%%%%%%%%%%%%%%%%%%%%%%%%%5

\newpage
9
\begin{itemize}
\item (i)
(a)
If X i is the number of bananas for each monkey then X i ~ Bin(7, p)
E ( X i ) = x ⇒ 7 \hat{p} =
\item (b)
p̂ A =
33 + 37
⇒ \hat{p} = 0.588
( 6 + 11 )
33
37
= 0.786, p̂ B =
= 0.481
6*7
11*7
\sigma 2 = Variance of test statistic = 0.588 * (1 − .588) * (1/42 + 1/77)
= 0.00891
Test statistic =
\hat{p} A − \hat{p} B 0.786 − 0.481
=
= 3.23
\sigma
0.00891
\item Test statistic has N(0,1) distribution so p-value is 0.00124
i.e. reject H 0 : p A = p B
\item (ii)
(a)
Let n i be the number of monkeys in group i and B i be the total number
of bananas taken by group i.
L ( b ; \theta ) = (2 \theta ) B A (1 − 2 \theta ) 7 n A − B A ( \theta ) B B (1 − \theta ) 7 n B − B B × constant
l ( b ; \theta ) = ln L ( b ; \theta )
= 33ln ( 2 \theta ) + (42 − 33) ln ( 1 − 2 \theta ) + 37 ln ( \theta ) + (77 − 37) ln ( 1 − \theta ) + constant
= 33ln ( 2 \theta ) + 9ln ( 1 − 2 \theta ) + 37 ln ( \theta ) + 40ln ( 1 − \theta ) + constant
(b)
18
37 40 70 18
40
dl 66
= −
+ −
= −
−
d\theta^2\theta 1− 2\theta \theta 1− \theta \theta 1− 2\theta 1− \theta
Set equal to zero and solve
(
) (
)
(
)
70 1− 2\theta 1− \theta −18\theta 1− \theta − 40\theta(1− 2\theta)
\theta(1− 2\theta)(1− \theta)
= 0
⇒ 70 − 210\theta +140\theta^2 −18\theta +18\theta^2 − 40\theta + 80\theta^2 = 0
⇒ 238\theta^2 − 268\theta + 70 = 0
⇒ \theta = 0.412 or 0.714
As \theta<0.5, \thetâ = 0.412.
Page 6 – %%%%%%%%%%%%%%%%%%%%%%%%%%%%%%%%%
\item (iii)
(a)
Expected values under 2 models are:
A
42 * 0.588 = 24.7
42 * 2 * 0.412 = 34.6
33
Model in (i)
Model in (ii)
Observed
B
77 * 0.588 = 45.3
77 * 0.412 = 31.7
37
\item Model in (ii) seems to provide a better fit as expected values are
closer to observed.
\item (b)
In part (i)(b) we rejected p A = p B which suggests a model with a
common value of p would not be appropriate. The comparison above
suggests that an improved model can be used.
\item There were some common errors here, mainly involving part (i)(b) where many candidates
failed to identify an appropriate test to perform. There were also basic errors with algebraic
and calculus operations.
\item We note that in part (i)(b)an alternative solution can be given, using a chi-square test with
1 d.f. in a 2x2 table (4 cells). This is exactly equivalent to the test presented here and full
credit was given when completed correctly.
\end{itemize}

\end{document}
