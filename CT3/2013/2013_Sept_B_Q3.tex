\documentclass[a4paper,12pt]{article}

%%%%%%%%%%%%%%%%%%%%%%%%%%%%%%%%%%%%%%%%%%%%%%%%%%%%%%%%%%%%%%%%%%%%%%%%%%%%%%%%%%%%%%%%%%%%%%%%%%%%%%%%%%%%%%%%%%%%%%%%%%%%%%%%%%%%%%%%%%%%%%%%%%%%%%%%%%%%%%%%%%%%%%%%%%%%%%%%%%%%%%%%%%%%%%%%%%%%%%%%%%%%%%%%%%%%%%%%%%%%%%%%%%%%%%%%%%%%%%%%%%%%%%%%%%%%

\usepackage{eurosym}
\usepackage{vmargin}
\usepackage{amsmath}
\usepackage{graphics}
\usepackage{epsfig}
\usepackage{enumerate}
\usepackage{multicol}
\usepackage{subfigure}
\usepackage{fancyhdr}
\usepackage{listings}
\usepackage{framed}
\usepackage{graphicx}
\usepackage{amsmath}
\usepackage{chngpage}

%\usepackage{bigints}
\usepackage{vmargin}

% left top textwidth textheight headheight

% headsep footheight footskip

\setmargins{2.0cm}{2.5cm}{16 cm}{22cm}{0.5cm}{0cm}{1cm}{1cm}

\renewcommand{\baselinestretch}{1.3}

\setcounter{MaxMatrixCols}{10}

\begin{document}
3
The random variable X has a distribution with probability density function given by
⎧ 2 x
⎪
f ( x ) = ⎨ \theta 2
⎪ 0
⎩
; 0 \leq  x \leq \theta
; x < 0 or x > \theta
where \theta is the parameter of the distribution.
\begin{enumerate}[(a)]
\item (i)
Derive expressions in terms of \theta for the expected value and the variance of X.
[3]
\item Suppose that X 1 , X 2 ,..., X n is a random sample, with mean X , from the distribution
of X.
(ii)
3 X
is an unbiased estimator of \theta.
Show that the estimator \theta ˆ =
2
\end{enumerate}

[Total 5]
%%%%%%%%%%%%%%%%%%%%
\newpage
3
(i)
E ( X )  =  

0
E ( X )  =  
2

 2 x 3 
2 
x 2 dx  =   2   = 

  3  )  0 3
2 x
 2
x
0

 2 x 4 
 2
dx
 = 
 = 
 2 
 2
  4   ) 0 2
2 x
So, var( X )  =  E ( X 2 )  [ E ( X )] 2  = 
(ii)
 3 X
E (  ˆ )  =  E 
 2
 2 4  2  2

 = 
2
9
18
 3
3
  =  E ( X )  =  E ( X )  =   , so estimator is unbiased.
2
 2
%%Generally well answered. A few problems were encountered when deriving the variance.
%%Page 3Subject CT3 – September 2013 – Examiners’ Report

\end{document}
