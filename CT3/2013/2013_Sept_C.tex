\documentclass[a4paper,12pt]{article}

%%%%%%%%%%%%%%%%%%%%%%%%%%%%%%%%%%%%%%%%%%%%%%%%%%%%%%%%%%%%%%%%%%%%%%%%%%%%%%%%%%%%%%%%%%%%%%%%%%%%%%%%%%%%%%%%%%%%%%%%%%%%%%%%%%%%%%%%%%%%%%%%%%%%%%%%%%%%%%%%%%%%%%%%%%%%%%%%%%%%%%%%%%%%%%%%%%%%%%%%%%%%%%%%%%%%%%%%%%%%%%%%%%%%%%%%%%%%%%%%%%%%%%%%%%%%

\usepackage{eurosym}
\usepackage{vmargin}
\usepackage{amsmath}
\usepackage{graphics}
\usepackage{epsfig}
\usepackage{enumerate}
\usepackage{multicol}
\usepackage{subfigure}
\usepackage{fancyhdr}
\usepackage{listings}
\usepackage{framed}
\usepackage{graphicx}
\usepackage{amsmath}
\usepackage{chngpage}

%\usepackage{bigints}
\usepackage{vmargin}

% left top textwidth textheight headheight

% headsep footheight footskip

\setmargins{2.0cm}{2.5cm}{16 cm}{22cm}{0.5cm}{0cm}{1cm}{1cm}

\renewcommand{\baselinestretch}{1.3}

\setcounter{MaxMatrixCols}{10}

\begin{document}
\begin{enumerate}

%%-- PLEASE TURN OVER
%%-- Question 6
\item A researcher obtains samples of 25 items from normally distributed measurements
from each of two factories. The sample variances are 2.86 and 9.21 respectively.
\begin{enumerate}[(i)]
\item Perform a test to determine if the true variances are the same. 
\item  For each factory calculate central 95\% confidence intervals for the true
variances of the measurements. 
\item 
Comment on how your answers in parts (i) and (ii) relate to each other.
\end{enumerate}
%%%%%%%%%%%%%%%%%%%%%%%%%%%%%%%
\item %%[Total 8]
A motor insurance company has a portfolio of 100,000 policies. It distinguishes
between three groups of policyholders depending on the geographical region in which they live. The probability p of a policyholder submitting at least one claim during a year is given in the following table together with the number, n, of policyholders belonging to each group.. Each policyholder belongs to exactly one group and it is
assumed that they do not move from one group to another over time.
Group
p
A
B
C
0.15 0.1 0.05
20 20 60
n (in 1000s)
It is assumed that any individual policyholder submits a claim during any year
independently of claims submitted by other policyholders. It is also assumed that
whether a policyholder submits any claims in a year is independent of claims in
previous years conditional on belonging to a particular group.

\begin{enumerate}[(i)]
\item Show that the probability that a randomly selected policyholder will submit a claim in a particular year is 0.08.

\item Calculate the probability that a randomly selected policyholder will submit a claim in a particular year given that the policyholder is not in group C.

\item Calculate the probability for a randomly selected policyholder to belong to
group A given that the policyholder submitted a claim last year.

\item Calculate the probability that a randomly selected policyholder will submit a
claim in a particular year given that the policyholder submitted a claim in the
previous year. It is assumed that the insurance company does not know to
which group the policyholder belongs.

\item Calculate the probability that a randomly selected policyholder will submit a
claim in two consecutive years.
\end{enumerate}

%%%%%%%%%%%%%%%%%%%%%%%%%%%%%%%%%%%%%%%5
\end{enumerate}
\end{document}
