\documentclass[a4paper,12pt]{article}

%%%%%%%%%%%%%%%%%%%%%%%%%%%%%%%%%%%%%%%%%%%%%%%%%%%%%%%%%%%%%%%%%%%%%%%%%%%%%%%%%%%%%%%%%%%%%%%%%%%%%%%%%%%%%%%%%%%%%%%%%%%%%%%%%%%%%%%%%%%%%%%%%%%%%%%%%%%%%%%%%%%%%%%%%%%%%%%%%%%%%%%%%%%%%%%%%%%%%%%%%%%%%%%%%%%%%%%%%%%%%%%%%%%%%%%%%%%%%%%%%%%%%%%%%%%%

\usepackage{eurosym}
\usepackage{vmargin}
\usepackage{amsmath}
\usepackage{graphics}
\usepackage{epsfig}
\usepackage{enumerate}
\usepackage{multicol}
\usepackage{subfigure}
\usepackage{fancyhdr}
\usepackage{listings}
\usepackage{framed}
\usepackage{graphicx}
\usepackage{amsmath}
\usepackage{chngpage}

%\usepackage{bigints}
\usepackage{vmargin}

% left top textwidth textheight headheight

% headsep footheight footskip

\setmargins{2.0cm}{2.5cm}{16 cm}{22cm}{0.5cm}{0cm}{1cm}{1cm}

\renewcommand{\baselinestretch}{1.3}

\setcounter{MaxMatrixCols}{10}

\begin{document}
\begin{enumerate}
9 An analyst is interested in using a gamma distribution with parameters \alpha = 2 and
\lambda = ½, that is, with density function
1
( ) 1 2 , 0
4
x
f x xe x
−
= < < ∞.
(i) (a) State the mean and standard deviation of this distribution.
(b) Hence comment briefly on its shape.

(ii) Show that the cumulative distribution function is given by
1
( ) 1 (1 1 ) 2 , 0
2
x
F x x e x
−
= − + < < ∞ (zero otherwise). 
The analyst wishes to simulate values x from this gamma distribution and is able to
generate random numbers u from a uniform distribution on (0,1).
(iii) (a) Specify an equation involving x and u, the solution of which will yield
the simulated value x.
(b) Comment briefly on how this equation might be solved.
(c) The graph below gives F(x) plotted against x. Use this graph to obtain
the simulated value of x corresponding to the random number u = 0.66.

[Total 8]
CT3 S2012–5 \newpage
10 The number of hours that people watch television per day is the subject of an
empirical study that is carried out in four regions in a country. Five people are
randomly selected in each of the regions and are asked about the average number of
hours per day that they spent watching television during the last year. The results are
shown in the following table, with the last column shows the average in each region.
Average
Region 1 2.0 1.1 0.2 3.8 2.8 1.98
Region 2 1.2 1.0 0.9 1.1 1.6 1.16
Region 3 2.5 2.0 2.6 2.4 2.3 2.36
Region 4 1.2 1.7 1.0 1.8 1.3 1.40
Based on the above observations the following ANOVA table was obtained:
  Source of variation d.f. SS MSS
Between regions ... 4.4655 ...
Residual ... 8.892 ...
(i) State the mathematical model underlying the one-way analysis of variance
together with all associated assumptions. 
(ii) Complete the ANOVA table. 
(iii) Carry out an analysis of variance to test the hypothesis that the region has no
effect on the average time spent watching television. You should write down
the null hypothesis, calculate the value of the test-statistic, state its distribution
including any parameters, calculate the p-value approximately and state your
conclusion. 
[Total 8]
%%%%%%%%%%%%%%%%%%%%%%%%%%%%%%%%%%%%%%%%
  
  10 (i) Yij = μ + τi + εij
with εij being i.i.d. N(0, \sigma2 )
In particular, it is assumed that the variance is the same in all groups.
(ii)
Source of variation d.f. SS MSS
Between regions 3 4.4655 1.4885
Residual 16 8.892 0.55575
(iii) H0 : τi = 0 for all groups i
F = 2.6784 should be from F distribution with 3,16 d.f.
From the tables we know that this gives a p-value of 0.086 (with
                                                            interpolation).
Reject at 10%, not at 5%, some but very weak evidence against H0
Mainly well answered. Care is required in calculating the p-values correctly. Also, a number
of candidates had difficulties in writing down a sensible form of the ANOVA model in part (i).
11 (i) This is an F distribution with 10, 8 degrees of freedom.
(ii) The interval is given by
2 2 2 2
10,8,0.025 10,8,0.975
SA / SB , SA / SB
F F
⎛ ⎞
⎜⎜ ⎟⎟
⎝ ⎠
From tables F10,8,0.025 = 4.295 and F10,8,0.975 =1/ F8,10,0.025 =1/ 3.855
giving 0.692 / 0.813 , (0.692 / 0.813)*3.855
4.295
⎛ ⎞
⎜ ⎟
⎝ ⎠
= (0.198, 3.281)
(iii) As the two samples are independent we have that
( ) ( ) ( ) 2 (1/11 1 / 9)
11 9
A B
A B
V X V X
V X − X = + =\sigma +
  Normality of the data then gives that ( ) ~ (0,1)
1 1
11 9
Z X A XB A B N
− − μ −μ
=
  \sigma +
   – September 2012 – %%%%%%%%%%%%%%%%%%%%%%%%%%%%%%%%%%%%%%%%%
Page 8
We are also given that
2
2
2 18
18
p ~ S
Y = χ
\sigma
and with Z and Y being independent
we can use that ~ 18
/18
Z t
Y
to obtain ( )
1 1
11 9
A B A B
p
X X
S
− − μ −μ
+
  ~ t18.
(iv) First compute 2 10*0.692 8*0.813 0.74577 0.864
p 18 p s s +
  = = ⇒ =
  Then with t18,0.025 = 2.101 the interval is given by (4.05 – 4.36) ± 2.101 *
  0.864 (1/11 + 1/9)1/2 i.e. (– 1.126, 0.506).
(v) The interval includes the value 0, suggesting that there is no difference in the
mean effectiveness of the two vaccines.
Part (iii) was problematic for many candidates. Many candidates struggled to provide a
‘proof’ that had sufficient rigour. There were errors also in determining the endpoints of the
CI in part (ii), often due to using the wrong percentiles of the F distribution.
12 (i) Likelihood function
L( p) = p6 (20 p)114 (10 p)62 (1−31p)18 = Cp182 (1− 31p)18
log L( p) = logC +182log p +18log(1−31p)
log ( ) 182 18 ( 31) 0
1 31
L p
p p p
∂
= + − =
  ∂ −
182 558 1 31 31 1 1 31
1 31 182 558 182 558 558 182 558
p p p p p
p p
− ⎛ ⎞ = ⇒ = ⇒ + = ⇒ ⎜ + ⎟ − ⎝ ⎠
=1/ 558
pˆ = 0.02935
 – September 2012 – %%%%%%%%%%%%%%%%%%%%%%%%%%%%%%%%%%%%%%%%%
Page 9
(ii) H0 : The proposed distribution is the true distribution of the data with nonspecified
parameter p (it is important to mention that the parameter itself is
             not part of the null hypothesis)
Under H0 and using pˆ = 0.02935 from (i)(a) we obtain the following
expected frequencies
Body-Mass-Index < 18.5 18.5–25 25–30 >30
Expected frequency 5.87 117.4 58.7 18.03
Test-statistic is 0.286915
from a Chi-square distribution with 2 d.f.
The test statistic has a very small value, and there is no evidence against the
null.
(iii) P[BMI > 30]
= P[BMI > 30|single]P[single]+ P[BMI > 30|married]P[married]
12 *0.5 6 *0.5 0.1094
158 42
= + =
  (iv) H0 : Marital status is independent of BMI
Under H0 we have:
  Marital Status Body-Mass-Index Total
< 18.5 18.5–25 25–30 >30
Single 4.74 90.06 48.98 14.22 158
Married 1.26 23.94 13.02 3.78 42
Total 6 114 62 18 200
Use χ2 test.
Test-statistic:
  . . 2 2 4
1 1 . .
*
  ( )
* 8.528399
i j
ij
i j i j
f f
f
C n f f
n
= =
  −
=\sigma\sigma =
  C is χ2 -distributed with (2 −1 )(4 −1 ) = 3 degrees of freedom.
p -value: P[C > 8.528399] < 1− 0.9616 = 0.0384
 – September 2012 – %%%%%%%%%%%%%%%%%%%%%%%%%%%%%%%%%%%%%%%%%
Page 10
Therefore, we reject H0 at 5% level, but not at the 1% level.
There were errors in part (i) caused by failure to differentiate correctly. In part (iv)
alternative solutions involving merging of adjacent categories were given full redit where
correct. However note that merging the first and last column is not correct in this question.
