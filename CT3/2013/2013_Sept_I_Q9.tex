\documentclass[a4paper,12pt]{article}

%%%%%%%%%%%%%%%%%%%%%%%%%%%%%%%%%%%%%%%%%%%%%%%%%%%%%%%%%%%%%%%%%%%%%%%%%%%%%%%%%%%%%%%%%%%%%%%%%%%%%%%%%%%%%%%%%%%%%%%%%%%%%%%%%%%%%%%%%%%%%%%%%%%%%%%%%%%%%%%%%%%%%%%%%%%%%%%%%%%%%%%%%%%%%%%%%%%%%%%%%%%%%%%%%%%%%%%%%%%%%%%%%%%%%%%%%%%%%%%%%%%%%%%%%%%%

\usepackage{eurosym}
\usepackage{vmargin}
\usepackage{amsmath}
\usepackage{graphics}
\usepackage{epsfig}
\usepackage{enumerate}
\usepackage{multicol}
\usepackage{subfigure}
\usepackage{fancyhdr}
\usepackage{listings}
\usepackage{framed}
\usepackage{graphicx}
\usepackage{amsmath}
\usepackage{chngpage}

%\usepackage{bigints}
\usepackage{vmargin}

% left top textwidth textheight headheight

% headsep footheight footskip

\setmargins{2.0cm}{2.5cm}{16 cm}{22cm}{0.5cm}{0cm}{1cm}{1cm}

\renewcommand{\baselinestretch}{1.3}

\setcounter{MaxMatrixCols}{10}

\begin{document}The random variables Y A and Y B describe the number of hours per month that a
randomly selected household in Cities A and B, respectively, uses its car. Both cities
recently decided to introduce measures to reduce road congestion. To investigate the
effect of these measures ten households in each city were randomly selected and
asked about the hours per month that they use their car before and after the measures
were introduced. The random variables Z A and Z B describe the hours of car usage
after the measures have been introduced, and X A = Y A − Z A and X B = Y B − Z B
denote the reduction in car usage. The following table shows the summary statistics
for the ten households in the two cities.
Sample size n
City A
10
City B
10
y
33
29
s Y
7.5
8
z
28.5
28
s Z
7
7
s X
2
2.5
Here, y and z denote the sample means of Y and Z in the two cities, and s Y , s Z
and s X denote the sample standard deviations for Y, Z and X respectively.
You can assume that the random variables Y A and Y B are independent and
approximately normally distributed

\begin{enumerate}[(a)]
\item (i)
Perform a statistical test at a 5% significance level to test the null hypothesis that expected car usage in City A was the same as expected car usage in City B before the measures were introduced. State all other assumptions that you
make and justify them.
\medskip
An actuary wishes to investigate whether the measures to reduce road congestion have been effective.
\item (ii) Perform a statistical test at the 5% significance level, where the alternative hypothesis is that car usage in City A has been reduced as a result of the
measures.

\item (iii) Calculate a 95\% confidence interval for the expected reduction in car usage
for City B.
\end{enumerate}

To investigate further the impact of measures to reduce road congestion, a third city, City C, is included in the study. The following table contains the data for 10 randomly
selected households in City C:
Sample size n
City C
10
y
37
s Y
9
z
33
s Z
8
s X
3
Let x ij denote the observed reduction in car usage in city i for household j .
10
(iv)
Confirm that
\sum  x Aj = 45 and
j=1
CT3 S2013–6
10
\sum  x Aj 2 = 238.5 .
j = 1
10
You are also given
\sum  x Bj = 10 ,
j = 1
(v)
CT3 S2013–7
10
\sum  x Cj = 40 ,
j = 1
10
\sum  x Bj 2 = 66.25 and
j = 1
10
\sum  x Cj 2 = 241 .
j = 1
Perform an analysis of variance to test at a 5% significance level the null
hypothesis that there is no difference in the mean reduction in car usage
between the three cities.
\end{enumerate}
\newpage
%%%%%%%%%%%%%%%%%%%%%%%%%%%%%%%%%%%%%%%%%%%%%%%%%%%%%%%%%%%%%%%%%%%%%%%%%%%%%%%%%%%
9
\begin{itemize}
\item(i)
Sample sizes are small, therefore, we need a t -test. We need to assume that
the variances are equal, although the sample standard deviations are different.
Since the sample size is small we can argue that equal variances is a
reasonable assumption.

1
1 
Test statistic t  =  ( Y A  Y B ) /  S P

  ~ t n A  n B  2 under the null

n
n
A
B 

hypothesis that expected car usage is equal in both cities.
Page 7Subject CT3 – September 2013 – Examiners’ Report
S P 2  = 
t  = 
9 S A 2  9 S B 2 1
 =  7.5 2  8 2  =  60.125
18
2


33  29
 =  1.1535
12.025
\item Two sided test, critical values are −2.101 and 2.101 from t 18 .
The null hypothesis of equal car usage is not rejected.
\item (ii)

X A  =  Y A  Z A ~ N  A ,  A 2

H 0 :  A  0 and H 1 :  A  0 (also full marks for H 0 :  A  =  0 vs. H 1 :  A  0 )
t  = 
33  28.5
4.5
 = 
 =  7.115
0.6325
2 / 10
Critical values from t 9 at 5%: 1.833
\item This is clear evidence that the null hypothesis is rejected, and therefore, car
usage has been reduced significantly in City A.
\item (iii)
X B  =  Y B  Z B ~ N   B ,  B 2 
CI: 29  28 
t 9,0.025 2.5
10
 =  (1  2.262  0.79, 1  2.262  0.79 )  =  ( 0.788, 2.788 
(marking: test statistic 1mark, critical value 1mark, correct answer 1mark)
\item (iv)
Let x ij be the difference in city i household j .
10
 x Aj  =  10  y  z   =  45,
j  =  1
10
 x 2 Aj  =   10  1  2 2  10(33  28.5) 2  =  238.5
j  =  1
(v)

10
10 10 10
j  =  1 j  =  1 j  =  1
 x ij 2  =   x 2 Aj   x Bj 2   x Cj 2
i  =  A , B , C j  =  1
SS T  =   238.5  66.25  241  
Page 8
 45  10  40  2
10  10  10
 =  545.75 
95 2
 =  244.92
30
%%---- Subject CT3 – September 2013 – Examiners’ Report
 45 2 10 2 40 2  95 2 3  3725  9025
SS B  =  


 = 
 =  71.67
  
 10 10
10
30
30


SS R  =  244.92  71.67  =  173.25
F  = 
71.67 / 2
35.835
 = 
 =  5.58 on (2, 27) degrees of freedom
173.25 / 27 6.417
Critical value at 5%: 3.354
\item The null hypothesis that reduction in car usage is equal in the three cities is
rejected.
\item There were no particular problems with this question. However a number of candidates
failed to justify the assumptions in part (i), while some seemed not to understand fully the
different test (or CI) requirements in different parts of the question.
\end{itemize}
\end{document}
