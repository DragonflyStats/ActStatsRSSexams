[Total 7]
7 Analyst A collects a random sample of 30 claims from a large insurance portfolio and
calculates a 95% confidence interval for the mean of the claim sizes in this portfolio.
She then collects a different sample of 100 claims from the same portfolio and
calculates a new 95% confidence interval for the mean claim size.
(i) Explain how the widths of the two confidence intervals will differ. [2]
Analyst B obtains a 95% confidence interval for the mean claim size of this portfolio
based on a different sample of 30 claims. She subsequently realises that one of the
claims in the sample has an extremely large value and can be considered as an outlier.
She decides to replace this claim with a new randomly selected one, whose size is not
an outlier, and obtains a new 95% confidence interval.
(ii) Explain how the two confidence intervals will differ in the case of Analyst B.
[3]
[Total 5]
8 The random variable S is given as S = Y1 + Y2 + …+ YN (with S = 0 if N = 0) where
the random variables Yi are identically and independently distributed according to a
lognormal distribution with parameters μ = 0.5 and σ2 = 0.1. N is also a random
variable which is independent of Yi , and its distribution given below.
N 0 1 2 3 4
Pr(N = n) 0.1 0.3 0.3 0.2 0.1
Calculate the mean and the variance of the random variable S. [7]
CT3 S2012–4

%%%%%%%%%%%%%%%%%%%%%%%%%%%%%
7 (i) With the larger sample of 100 claims the standard error of the sample mean
will be smaller, giving a narrower confidence interval.
(ii) The replacement of the extreme value will give a smaller sample mean, which
means that the interval will be shifted to the left.
The variance of the sample will also be smaller, which will again give a
narrower interval.
Many candidates recognised the correct effect on the interval, without being able to justify it
properly. Note that reasonably accurate wording is important in providing the comments and
justification required here.
8 E[N] = Σn P(N = n) = 0.3 + 0.6 + 0.6 + 0.4 = 1.9
E[N 2] = Σn2 P(N = n) = 0.3 + 1.2 + 1.8 + 1.6 = 4.9
V[N] = E[N 2] – (E[N])2 = 1.29
Also E[Y] = exp(μ + σ2 / 2 ) = e 0.55 = 1.73325
V [Y ] = (E[Y ])2 (exp(σ2 )−1 ) = 1.733252 * (e 0.1 – 1) = 0.31595
Using known results
Subject CT3 (Probability and Mathematical Statistics) – September 2012 – Examiners’ Report
Page 6
E[S] = E[N] E[Y] = 1.9 * 1.73325 = 3.293
V[S] = E[N] V[Y] + V[N] (E[Y])2 = 0.60031 + 3.87536 = 4.476
Some frequent errors were due to mis-interpretation of the mean and variance of the lognormal
distribution.
9 (i) (a) mean = 4
α
=
  λ
and s.d. = 2 8 2.8
α
= =
  λ
(b) As claims are non-negative and the s.d. is quite large relative to the
mean, then the distribution will be quite positively skewed.
(ii) F(x)
1
2
0
1
4
x t
te dt
−
= ∫
1
2
0
1 ( )
2
x t
td e
−
= − ∫
1 1
2 2
0
0
1 [ ] 1
2 2
t x t te x e dt
− −
= − + ∫
1 1
2 2
0
1 [ ]
2
x t x xe e
− −
= − −
1
1 (1 1 ) 2
2
x
x e
−
= − +
  (iii) (a) F(x) = u i.e.
1
1 (1 1 ) 2
2
x
xe u
−
− + =
  (b) This equation would have to be solved numerically
(c) Using u = 0.66 on the vertical axis, we invert to get x = 4.5 on the
horizontal axis.
In part (ii) many candidates failed to integrate correctly. A lot of problems were caused by
not using the correct limits for the integral. In part (iii) a popular answer was to use “trialand-
  error”, which is not an appropriate approach here.
Subject CT3 (Probability and Mathematical Statistics) – September 2012 – Examiners’ Report
Page 7
