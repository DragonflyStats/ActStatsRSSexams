\documentclass[a4paper,12pt]{article}

%%%%%%%%%%%%%%%%%%%%%%%%%%%%%%%%%%%%%%%%%%%%%%%%%%%%%%%%%%%%%%%%%%%%%%%%%%%%%%%%%%%%%%%%%%%%%%%%%%%%%%%%%%%%%%%%%%%%%%%%%%%%%%%%%%%%%%%%%%%%%%%%%%%%%%%%%%%%%%%%%%%%%%%%%%%%%%%%%%%%%%%%%%%%%%%%%%%%%%%%%%%%%%%%%%%%%%%%%%%%%%%%%%%%%%%%%%%%%%%%%%%%%%%%%%%%

\usepackage{eurosym}
\usepackage{vmargin}
\usepackage{amsmath}
\usepackage{graphics}
\usepackage{epsfig}
\usepackage{enumerate}
\usepackage{multicol}
\usepackage{subfigure}
\usepackage{fancyhdr}
\usepackage{listings}
\usepackage{framed}
\usepackage{graphicx}
\usepackage{amsmath}
\usepackage{chngpage}

%\usepackage{bigints}
\usepackage{vmargin}

% left top textwidth textheight headheight

% headsep footheight footskip

\setmargins{2.0cm}{2.5cm}{16 cm}{22cm}{0.5cm}{0cm}{1cm}{1cm}

\renewcommand{\baselinestretch}{1.3}

\setcounter{MaxMatrixCols}{10}

\begin{document}
\begin{enumerate}

PLEASE TURN OVER11
The table below gives the frequency of a critical illness disease by age group in a
certain study. The table also gives the age midpoint (x), the number of people in each
⎛ \theta ˆ ⎞
, where \thetâ denotes the proportion in an age group with
group (n), and y = log ⎜ ⎜
ˆ ⎟ ⎟
⎝ 1 − \theta ⎠
the disease.
Age group
20–29
30–34
35–39
40–44
45–49
50–54
55–59
60–69
x
25
32.5
37.5
42.5
47.5
52.5
57.5
65
Contracted
disease?
Yes No
1
9
2
13
3
9
5
10
6
7
5
3
13
4
8
2
n
10
15
12
15
13
8
17
10
y
−2.19722
−1.87180
−1.09861
−0.69315
−0.15415
0.51083
1.17865
1.38629
∑x = 360; ∑x 2 = 17437.5; ∑y = −2.9392; ∑y 2 = 13.615; ∑xy= −9.0429
(i)
Calculate an estimate of the probability of having the disease under the assumption that the probability is the same for all age groups.
[1]
Consider the hypothesis that there are no differences in the probability of having the
disease for the different age groups.
(ii)
(a) Construct an 8 × 2 contingency table which includes the expected
frequencies under this hypothesis.
(b) Conduct a χ 2 test to investigate the hypothesis.
[6]
Consider the linear regression model y = α + \betax + ε, where the error terms (ε) are
independent and identically distributed following a N (0, \sigma 2 ) distribution.
(iii)
(a) Draw a scatterplot of y against x and comment on the appropriateness
of the considered model.
(b) Calculate the fitted regression line of y on x.
(c) Calculate a 99% confidence interval for the slope parameter.
(d) Interpret the result obtained in part (ii) with reference to the confidence
interval obtained in part (iii)(c).
[14]
\end{document}
