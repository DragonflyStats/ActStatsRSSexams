CT3 S2012–2
1 Calculate the mean, the median and the mode for the data in the following frequency
table.
Observation 0 1 2 3 4
Frequency 20 54 58 28 0
[3]
2 The following data are sizes of claims (ordered) for a random sample of 20 recent
claims submitted to an insurance company:
  174 214 264 298 335 368 381 395 402 442
487 490 564 644 686 807 1092 1328 1655 2272
(i) Calculate the interquartile range for this sample of claim sizes. [3]
(ii) Give a brief interpretation of the interquartile range calculated in part (i). [1]
[Total 4]
3 Let X be a discrete random variable with the following probability distribution:
  X 0 1 2 3
P(X = x) 0.4 0.3 0.2 0.1
Calculate the variance of Y, where Y = 2X + 10. [3]
4 Consider a random variable U that has a uniform distribution on (0,1) and a random
variable X that has a standard normal distribution. Assume that U and X are
independent.
Determine an expression for the probability density function of the random variable Z
= U + X in terms of the cumulative distribution function of X. [4]



1 1 (54 2*58 3*28) 254 1.5875
160 160
mean= + + = = 1
Median = value between 80th and 81st observation = 2 1
Mode = 2 1
Generally well answered. Note that the median is NOT the 80th observation, as some
candidates quoted.
2 (i) 1
2 th observation counting from below 5.5th observation
4
Q n
⎛ + ⎞ = ⎜ ⎟ =
  ⎝ ⎠
335 368 351.5
2
+
  = = 1
3
2 th observation counting from above 5.5th observation from above
4
Q n
⎛ + ⎞ = ⎜ ⎟ =
  ⎝ ⎠
807 686 746.5
2
+
  = = 1
IQR = Q3 −Q1 = 395 1
[With alternative definition:
    1
  1 th observation counting from below 343.25
  4
  Q n
  ⎛ + ⎞ = ⎜ ⎟ =
    ⎝ ⎠
  ,
  3
  1 th observation counting from above 776.75
  4
  Q n
  ⎛ + ⎞ = ⎜ ⎟ =
    ⎝ ⎠
  , IQR = 433.5 .]
(ii) The length of the interval containing the central half of the claim sizes is 395.
1
The vast majority of candidates calculated the quartiles correctly, although some were
confused with their definition. Part (ii) was not very well answered.
Subject CT3 (Probability and Mathematical Statistics) – September 2012 – Examiners’ Report
Page 4
3 E[X] = 1×0.3 + 2×0.2 + 3×0.1 = 1
⇒ V[X] = (0 –1)2×0.4 + (1 – 1)2×0.3 + (2 – 1)2×0.2 + (3 – 1)2×0.1
= 0.4 + 0.2 + 0.4 = 1
(OR via E[X2] = 2)
V[Y] = 4V[X] = 4
[OR: Directly from the distribution of Y, which is Y = 10, 12, 14, 16 with
  probabilities 0.4, 0.3, 0.2, 0.1 respectively.]
No particular problems encountered here. There are a variety of different methods for
obtaining the correct answer.
4 Let fZ (z) be the density of Z =U + X .
( ) ( ) ( ) ( )
1
0
Z U X X
u
f z =∫f u f z −u du = ∫f z − u du
( ) ( )
1
( 1)
z
X X X
z
f x dx F z F z
−
= ∫ = − −
where we have used the substitution u = z − x , and where FX is the distribution
function of X .
This question was very poorly answered. A large number of candidates did not attempt it at
all, while many others did not follow any reasonable approach. Note that this is based on
standard bookwork, viz. Unit 6, Section 3 in the Core Reading.
