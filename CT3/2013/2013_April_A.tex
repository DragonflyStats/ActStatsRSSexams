\documentclass[a4paper,12pt]{article}

%%%%%%%%%%%%%%%%%%%%%%%%%%%%%%%%%%%%%%%%%%%%%%%%%%%%%%%%%%%%%%%%%%%%%%%%%%%%%%%%%%%%%%%%%%%%%%%%%%%%%%%%%%%%%%%%%%%%%%%%%%%%%%%%%%%%%%%%%%%%%%%%%%%%%%%%%%%%%%%%%%%%%%%%%%%%%%%%%%%%%%%%%%%%%%%%%%%%%%%%%%%%%%%%%%%%%%%%%%%%%%%%%%%%%%%%%%%%%%%%%%%%%%%%%%%%

\usepackage{eurosym}
\usepackage{vmargin}
\usepackage{amsmath}
\usepackage{graphics}
\usepackage{epsfig}
\usepackage{enumerate}
\usepackage{multicol}
\usepackage{subfigure}
\usepackage{fancyhdr}
\usepackage{listings}
\usepackage{framed}
\usepackage{graphicx}
\usepackage{amsmath}
\usepackage{chngpage}

%\usepackage{bigints}
\usepackage{vmargin}

% left top textwidth textheight headheight

% headsep footheight footskip

\setmargins{2.0cm}{2.5cm}{16 cm}{22cm}{0.5cm}{0cm}{1cm}{1cm}

\renewcommand{\baselinestretch}{1.3}

\setcounter{MaxMatrixCols}{10}

\begin{document}
\begin{enumerate}

1
The following data represent the number of claims for twenty policyholders made
during a year.
0 0 0 0 0 0 0 1 1 1
1 1 1 1 1 1 2 2 2 3
Determine the sample mean, median, mode and standard deviation of these data.
2
Consider a random variable U that has a uniform distribution on [0,1] and let F be
the cumulative distribution function of the standard normal distribution.
Show that the random variable X = F − 1 ( U ) has a standard normal distribution.
3
%%%%%%%%%%%%%%%%%%%%%%%%%%%%%%%%%%%%%%%%%%%%%%%%%%%55
[3]
A discrete random variable X has a cumulative distribution function (CDF) with the
following values:
Observation 10 20 30
40 50
CDF
0.5 0.7 0.85 0.95 1
Calculate the probability that X takes a value:
(i)
(ii)
(iii)
(iv)
(v)
4
larger than 10.
less than 30.
exactly 40.
larger than 20 but less than 50.
exactly 20 or exactly 40.
[1]
[1]
[1]
[2]
[2]
[Total 7]



1
Mean = (7 × 0 + 9 × 1 + 3 × 2 + 3) / 20 = (9 + 6 + 3) / 20 = 18 /20 = 0.9
Median = 1 (observation with rank 10.5)
Mode = 1
7 × 0.9 2 + 9 × 0.1 2 + 3 × 1.1 2 + 2.1 2 5.67 + 0.09 + 3.63 + 2.1 2 13.80
VAR =
=
=
= 0.7263
19
19
19
STD = 0.8522
Well answered. Some working needs to be shown for full marks.
2
For any number x we get
P [ X ≤ x ] = P ⎡ F − 1 ( U ) ≤ x ⎤ = P ⎡ ⎣ U ≤ F ( x ) ⎤ ⎦ = F ( x )
⎣
⎦
which shows that F is the distribution function of the random variable X , which
proves the result.
Very poorly answered. Most candidates did not attempt this question and very few completed
it correctly.
3
P [ X > 10 ] = 1 − P [ X ≤ 10 ] = 1 − F ( 10 ) = 0.5
P [ X < 30 ] = P [ X ≤ 20 ] = F ( 20 ) = 0.7
P [ X = 40 ] = F ( 40 ) − F ( 30 ) = 0.1
P [ 20 < X < 50 ] = F ( 40 ) − F ( 20 ) = 0.25
P ⎡ ⎣ { X = 20 } ∪ { X = 40 } ⎤ ⎦ = P [ X = 20 ] + P [ X = 40 ]
= ⎡ ⎣ F ( 20 ) − F ( 10 ) ⎤ ⎦ + ⎡ ⎣ F ( 40 ) − F ( 30 ) ⎤ ⎦ = 0.2 + 0.1 = 0.3
Some problems were encountered here involving understanding and distinguishing the need
(or not) for strict inequalities for discrete variables, e.g. P[X<30] = P[X≤20].

\end{document}
