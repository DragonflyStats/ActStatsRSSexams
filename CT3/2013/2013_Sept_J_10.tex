\documentclass[a4paper,12pt]{article}

%%%%%%%%%%%%%%%%%%%%%%%%%%%%%%%%%%%%%%%%%%%%%%%%%%%%%%%%%%%%%%%%%%%%%%%%%%%%%%%%%%%%%%%%%%%%%%%%%%%%%%%%%%%%%%%%%%%%%%%%%%%%%%%%%%%%%%%%%%%%%%%%%%%%%%%%%%%%%%%%%%%%%%%%%%%%%%%%%%%%%%%%%%%%%%%%%%%%%%%%%%%%%%%%%%%%%%%%%%%%%%%%%%%%%%%%%%%%%%%%%%%%%%%%%%%%

\usepackage{eurosym}
\usepackage{vmargin}
\usepackage{amsmath}
\usepackage{graphics}
\usepackage{epsfig}
\usepackage{enumerate}
\usepackage{multicol}
\usepackage{subfigure}
\usepackage{fancyhdr}
\usepackage{listings}
\usepackage{framed}
\usepackage{graphicx}
\usepackage{amsmath}
\usepackage{chngpage}

%\usepackage{bigints}
\usepackage{vmargin}

% left top textwidth textheight headheight

% headsep footheight footskip

\setmargins{2.0cm}{2.5cm}{16 cm}{22cm}{0.5cm}{0cm}{1cm}{1cm}

\renewcommand{\baselinestretch}{1.3}

\setcounter{MaxMatrixCols}{10}

\begin{document}
\begin{enumerate}
PLEASE TURN OVER10
c
the e results from
m investing g in a certain n category of
o hedge
An analyst wishes to compare
funds, f, wi ith those fro om the stock k market, x. She uses an n appropria ate index for r each,
which over r 12 years ea ach produce ed the follow
wing return ns (in percen ntages to on
ne
decimal pla ace).
Year
Market (x)
Funds (f)
2000
5.0
2.1
2001 1
15.4 4
3.7 7
0.10 1, ¦ x 2
¦ x
2002
25.0
1.6
20 003
1 6.6
1 7.3
0. .3612, ¦ f
2004
9.2
11.6
2005
18.1
9.7
0.622, ¦ f 2
2006
13.2
14.4
2007
2.0
13.7
20 08
32 2.8
19 9.8
0.1710 0, ¦ xf
2009
25.0
19.5
2010
2
10.9
1.2
2011
6.7
0.3
0. .1989
It is assume ed that obse ervations fro om differen t years are independen
i
nt of each ot her.
Below is a scatter plot of market returns
r
agai inst fund ret turns for eac ch year.
25%
20%
15%
10%
5%
0%
Ͳ40%
Ͳ30%
Ͳ20%
Ͳ
Ͳ10%
%
Ͳ5% 0%
10%
Ͳ10%
20%
30%
M
Market
Ͳ20%
Ͳ25%
(i)
Ͳ15%
mment on th he relationsh hip between
n the two se eries.
Com

The hedge fund indust try often cla aims that he edge funds have
h
low co orrelation wi ith the
stock mark et.
(ii)
(iii)
(a) Calcula ate the corre elation coeff
fficient betw
ween the two o series. (b) Test wh hether the c orrelation coefficient
c
is s significan ntly differen
nt
from 0. [7]
Calc culate the parameters
p
f a linear regression
for
r
o the fund i index on the e
of
mar rket index. 
(iv) Calc culate a 95%
% confidenc ce interval for
f the unde erlying slop pe coefficien
nt for
the linear mode el in part (ii ii).

(v) Com
mment on your
y
answer s to parts (ii i)(b) and (iv v).
END
D OF PA PER
CT3 S2 2013–8

[T
Total 16]
%%%%%%%%%%%%%%%%%%%%%%%%%%%%%%10
(i) There is a positive linear relationship between the two.
(ii) (a)
S xx  =  0.3612  0.101 2 /12  =  0.360
S ff  =  0.1710  0.622 2 /12  =  0.139
S xf  =  0.1989  0.101 \times 0.622 /12  =  0.194
r  = 
(b)
S xf
S xx S ff
Statistic  = 
 = 
0.194
 =  0.867
0.360 \times 0.139
r n  2
1  r 2
 = 
0.867  \times  10
1  0.867 2
 =  5.50
t 10,0.995 = 3.169
So reject H 0 that correlation coefficient = 0 at 1% level (2-sided test)
(iii)
 ˆ  =  S xf / S xx  =  0.194 / 0.360  =  0.539
 ˆ  =  f   ˆ x  = 
0.622
0.101
 0.539 \times 
 =  0.0473
12
1 2
Page 9Subject CT3 – September 2013 – Examiners’ Report
(iv)
S xf 2
1 
 S ff 
 ˆ  = 
n  2 
S xx

2
 1 
0.194 2 
  =   0.139 
  =  0.0034
 10  
0.360  

t 10,0.975 = 2.228
C.I.  =   ˆ  t 10;0.975  ˆ 2 / S xx  =  0.539  2.228 0.0034 / 0.36 = (0.321,0.757)
(v)
C.I. does not contain zero. Consistent with correlation coefficient not equal to
zero as the test is actually the same. Both suggest that the hedge industry’s
claim that correlation is low may not be correct.
Very well answered in general. This is a typical regression/correlation question and the only
(minor) problems concerned errors with calculators. Note that part (ii)(b) can also be
answered using Fisher’s transformation, which results in the same conclusion.
END OF EXAMINERS’ REPORT
Page 10
