CT3 S2012–7 
12 An insurer has collected data about the body mass index of 200 males between the
age of 18 and 40. The results are shown in the following table.
Body mass index < 18.5 18.5–25 25–30 >30
Observed frequency 6 114 62 18
A statistician suggests the following model for the distribution of the body mass index
with an unknown parameter p.
Body mass index < 18.5 18.5–25 25−30 >30
Relative frequency p 20p 10p 1−31p
(i) Estimate the parameter p using the method of maximum likelihood. 
(ii) Perform a statistical test to decide whether the suggested distribution is
appropriate for the observed data. You should state the null hypothesis for the
test and your decision. [6]
To improve the description of the distribution of the body mass index, it is suggested
that the marital status of the males in this study is also recorded. The results are
shown in the following table.
Marital Status Body mass index Total
< 18.5 18.5–25 25–30 >30
Single 5 98 43 12 158
Married 1 16 19 6 42
Total 6 114 62 18 200
A life office has considered a sample of 10,000 men aged between 18 and 40 of which
50% are married and the other 50% are single.
(iii) Estimate the proportion of men with a body mass index of more than 30 in this
sample, based on the data in the above table. 
(iv) Determine whether the body mass index is independent of the marital status or
not, using an appropriate statistical test. You should state the null hypothesis
for the test, calculate the value of the test statistic and the approximate
p-value and state your decision. [8]
[Total 20]

%%
  CT3 S2012–6
11 In order to compare the effectiveness of two new vaccines, A and B, for a childhood
disease, 11 infants were immunised with vaccine A and 9 infants were immunised
with vaccine B. One month after immunisation the concentration of the disease
antibodies in the blood of each infant was recorded in appropriate units. The sample
mean and variance for each group is given below.
Vaccine A: nA =11, xA = 4.05, s2A = 0.692
Vaccine B: nB = 9, xB = 4.36, sB2 = 0.813
It is assumed that the distributions of the antibody concentration levels after
immunisation with vaccine A and vaccine B are N(\mu_A,\sigma2A) and N(μB,\sigma2B )
respectively. You may assume that the samples are independent.
(i) State the distribution of the pivotal quantity
2 2
2 2
/ .
/
  A A
B B
s
s
\sigma
\sigma

(ii) Calculate an equal-tailed 95\%  confidence interval for the ratio
2
2
A
B
\sigma
\sigma
using the
pivotal quantity in part (i). (You are not required to show the derivation of the
                               interval.) 
We now assume that 2 2 2. $\sigma_A = \sigma_B = \sigma$ Under this assumption, you are given that the
distribution of
2
2
18Sp
\sigma
is 2
χ18, where 2
Sp is the pooled variance of the two samples and
is independent from xA and xB.
(iii) Explain why, under the above result, the sampling distribution of
( )
1 1
11 9
A B A B
p
X X
S
− − μ −μ
+
  is t18. 
(iv) Calculate an equal-tailed 95\%  confidence interval for \mu_A − μB using the
sampling distribution in part (iii). (You are not required to show the
                                      derivation of the interval.) 
(v) Comment on your results with regard to differences between vaccine A and
vaccine B. 
[Total 15]
