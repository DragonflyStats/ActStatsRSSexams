\documentclass[a4paper,12pt]{article}

%%%%%%%%%%%%%%%%%%%%%%%%%%%%%%%%%%%%%%%%%%%%%%%%%%%%%%%%%%%%%%%%%%%%%%%%%%%%%%%%%%%%%%%%%%%%%%%%%%%%%%%%%%%%%%%%%%%%%%%%%%%%%%%%%%%%%%%%%%%%%%%%%%%%%%%%%%%%%%%%%%%%%%%%%%%%%%%%%%%%%%%%%%%%%%%%%%%%%%%%%%%%%%%%%%%%%%%%%%%%%%%%%%%%%%%%%%%%%%%%%%%%%%%%%%%%

\usepackage{eurosym}
\usepackage{vmargin}
\usepackage{amsmath}
\usepackage{graphics}
\usepackage{epsfig}
\usepackage{enumerate}
\usepackage{multicol}
\usepackage{subfigure}
\usepackage{fancyhdr}
\usepackage{listings}
\usepackage{framed}
\usepackage{graphicx}
\usepackage{amsmath}
\usepackage{chngpage}

%\usepackage{bigints}
\usepackage{vmargin}

% left top textwidth textheight headheight

% headsep footheight footskip

\setmargins{2.0cm}{2.5cm}{16 cm}{22cm}{0.5cm}{0cm}{1cm}{1cm}

\renewcommand{\baselinestretch}{1.3}

\setcounter{MaxMatrixCols}{10}

\begin{document}
\begin{enumerate}

%%8
\item A random sample of 10 independent claim amounts was taken from each of three
different regions and an analysis of variance was performed to compare the mean
level of claims in these regions. The resulting ANOVA table is given below.
Source
d.f.
SS
MSS
2
4,439.7
2,219.9
Between regions
27 10,713.5 396.8
Residual
29 15,153.2
Total

(i)
Perform the appropriate F test to determine whether there are significant
differences between the mean claim amounts for the three regions. You
should state clearly the hypotheses of the test.

The three sample means were:
Region
A
Sample mean 147.47
B
C
154.56 125.95
It was of particular interest to compare regions A and B.
(ii)
9
(a) Calculate a 95\% confidence interval for the difference between the
means for regions A and B.
(b) Comment on your answer in part (ii)(a) given the result of the F test
performed in part (i).

[Total 8]
A behavioural scientist is observing a troop of monkeys and is investigating whether
social status affects the amount of food that an individual takes. The monkeys are
divided into two groups of different social rank and the scientist counts the number of
bananas each individual takes. Each monkey can take a maximum of 7 bananas.
Social rank
Number of monkeys
Total bananas taken
(i)
A
6
33
B
11
37
It is first suggested that the number of bananas taken by each individual of each group follows the same binomial distribution with common parameter p
and n=7.
(a) Use the method of moments to estimate the parameter p.
(b) The scientist is unsure whether a common parameter is appropriate and wishes to compare p A and p B , the probability that a banana is taken
by an individual in groups A and B respectively.
Test the hypothesis that p A = p B .
[7]
%%%%%%%%%%%%%%%%%%%%%%%%%%%%%%%%%%%%%%%%%%%%%%%%%%%%%%%%%%%%%%%%%%%%%%%%%%%%%%%%%%%%%%%%%%%%%%%%%%%%%%%%%%%%%%%%%%%%%%%%%%%%%%%%%%%%%%%%%%%%%
\item 
A statistician suggests an alternative model. The number of bananas taken by
an individual still follows a binomial distribution with n=7, but for group A
the parameter is 2\theta and for group B the parameter is $\theta$, where $\theta < 0.5$.
(a)
Show that the log likelihood for \theta is given by:
33
\[\ln ( 2 \theta ) + 9 \ln ( 1 − 2 \theta ) + 37 \ln ( \theta ) + 40 \ln ( 1 − \theta ) + constant\]
(b)
Hence calculate the maximum likelihood estimate of0 $\theta$.

(iii)
(a) Compare the fit of the two suggested models in parts (i) (with common
parameter p) and (ii) by considering the expected number of bananas
taken in groups A and B under the two models. You are not required
to perform a formal test.
(b) Comment on the above comparison in relation to your answer in
part (i)(b).

%%%%%%%%%%%%%%%%%%%%%%%%%%%%%%%%%%%%%%%%%%%%%%%%%%%%%%%%%%%%%%%%%%%%%%%%%%%%%%%%%%%%%%%%%%%%%%%%%%%%%%%%%%%%%%%%%%%5
8
(i)
H 0 : The means of the claims in the 3 regions are all equal; H 1 : means are
different for at least one pair.
F = 5.59 on 2 and 27 d.f. From tables the 1\% critical point is 5.488.
Therefore, we have (strong) evidence against the null hypothesis, and
conclude that there are differences in the means for the 3 regions.
(ii)
(a)
95\% CI for \mu A − \mu B is given by
( y A − y B ) \pm t 0.025, 27 \sigma ˆ
1 1
1 1
+
+
giving ( 147.47 − 154.56 ) \pm 2.052 396.8
10 10
10 10
i.e − 7.09 \pm 18.28 or (–25.37, 11.19)
(b)
The CI comfortably contains zero, suggesting no difference between
the true means for regions A and B.
The significant result of the F test clearly comes from region C mean being much lower than the means for regions A and B.
Generally well answered. Some candidates failed to identify the connection between the conclusion of the ANOVA and that of the CI for regions A and B.
Page 5 – %%%%%%%%%%%%%%%%%%%%%%%%%%%%%%%%%
\newpage
9
(i)
(a)
If X i is the number of bananas for each monkey then X i ~ Bin(7, p)
E ( X i ) = x ⇒ 7 \hat{p} =
(b)
p̂ A =
33 + 37
⇒ \hat{p} = 0.588
( 6 + 11 )
33
37
= 0.786, p̂ B =
= 0.481
6*7
11*7
\sigma 2 = Variance of test statistic = 0.588 * (1 − .588) * (1/42 + 1/77)
= 0.00891
Test statistic =
\hat{p} A − \hat{p} B 0.786 − 0.481
=
= 3.23
\sigma
0.00891
Test statistic has N(0,1) distribution so p-value is 0.00124
i.e. reject H 0 : p A = p B
(ii)
(a)
Let n i be the number of monkeys in group i and B i be the total number
of bananas taken by group i.
L ( b ; \theta ) = (2 \theta ) B A (1 − 2 \theta ) 7 n A − B A ( \theta ) B B (1 − \theta ) 7 n B − B B × constant
l ( b ; \theta ) = ln L ( b ; \theta )
= 33ln ( 2 \theta ) + (42 − 33) ln ( 1 − 2 \theta ) + 37 ln ( \theta ) + (77 − 37) ln ( 1 − \theta ) + constant
= 33ln ( 2 \theta ) + 9ln ( 1 − 2 \theta ) + 37 ln ( \theta ) + 40ln ( 1 − \theta ) + constant
(b)
18
37 40 70 18
40
dl 66
= −
+ −
= −
−
d\theta^2\theta 1− 2\theta \theta 1− \theta \theta 1− 2\theta 1− \theta
Set equal to zero and solve
(
) (
)
(
)
70 1− 2\theta 1− \theta −18\theta 1− \theta − 40\theta(1− 2\theta)
\theta(1− 2\theta)(1− \theta)
= 0
⇒ 70 − 210\theta +140\theta^2 −18\theta +18\theta^2 − 40\theta + 80\theta^2 = 0
⇒ 238\theta^2 − 268\theta + 70 = 0
⇒ \theta = 0.412 or 0.714
As \theta<0.5, \thetâ = 0.412.
Page 6 – %%%%%%%%%%%%%%%%%%%%%%%%%%%%%%%%%
(iii)
(a)
Expected values under 2 models are:
A
42 * 0.588 = 24.7
42 * 2 * 0.412 = 34.6
33
Model in (i)
Model in (ii)
Observed
B
77 * 0.588 = 45.3
77 * 0.412 = 31.7
37
Model in (ii) seems to provide a better fit as expected values are
closer to observed.
(b)
In part (i)(b) we rejected p A = p B which suggests a model with a
common value of p would not be appropriate. The comparison above
suggests that an improved model can be used.
There were some common errors here, mainly involving part (i)(b) where many candidates
failed to identify an appropriate test to perform. There were also basic errors with algebraic
and calculus operations.
We note that in part (i)(b)an alternative solution can be given, using a chi-square test with
1 d.f. in a 2x2 table (4 cells). This is exactly equivalent to the test presented here and full
credit was given when completed correctly.
10
(i)
Y i has a compound distribution, so
( )
E ( Y i ) = E ( N i ) E X ij = \lambda\mu
( )
V ( Y i ) = E ( N i ) V X ij + V ( N i ) E ( X ij ) 2 = \lambda\sigma 2 + \lambda\mu 2
(ii)
S also has a compound distribution.
E ( S ) = E ( M ) E ( Y i ) = κ\lambda\mu
V ( S ) = E ( M ) V ( Y i ) + V ( M ) E ( Y i ) 2 = κ\lambda ( \sigma 2 + \mu 2 ) + κ\lambda 2 \mu 2 = κ\lambda ( \sigma 2 + \mu 2 + \lambda\mu 2 )
(iii)
P ( X ij ≤ x + C | X ij > C ) =
( 1 − e
=
− x − C
e
(
P C < X ij ≤ x + C
− 1 + e − C
− C
(
P X ij > C
) = 1 − e
)
)
− x
= P ( X ij ≤ x )
Page 7 – %%%%%%%%%%%%%%%%%%%%%%%%%%%%%%%%%
(iv)
(
)
(a) \lambda * = 1000 × P X ij > 2 = 1000 e − 2 = 135.3
(b) From definition of new variable and part (iii) we have that
(
)
P X ij * ≤ x = P ( X ij − 2 ≤ x | X ij > 2) = P ( X ij ≤ x + 2 | X ij > 2) = P ( X ij ≤ x )
meaning that X ij * has the same distribution as X ij , i.e. Exp(1).
(c)
E ( S R ) = κ\lambda * \mu = 4 × 135.3 × 1 = 541.2
V ( S R ) = κ\lambda * ( \sigma 2 + \mu 2 + \lambda * \mu 2 ) = 4 × 135.3 × ( 1 + 1 + 135.3 ) = 74306.8
Most candidates found this question challenging. Answers to the memoryless property of the
exponential distribution (amply discussed in the CR) in part (iii) were often disappointing, and the relevant application in part (iv) was poorly attempted. These shortcomings highlight the issue of being prepared to tackle questions that deviate from the form that appears in
past papers.
11
(i)
Notation: n i = number in group i; r i = number with disease in group i.
\theta ˆ =
(ii)
(a)
∑ r i = 43 = 0.43
∑ n i 100
Expected frequencies (in brackets) are given assuming constant
probability of disease for all groups, independently of age:
Disease
Age group Yes
No
Total
1
9
10
20−29
(4.30) (5.70)
2
13
15
30−34
(6.45) (8.55)
3
9
12
35−39
(5.16) (6.84)
5
10
15
40−44
(6.45) (8.55)
6
7
13
45−49
(5.59) (7.41)
5
3
8
50−54
(3.44) (4.56)
13
4
17
55−59
(7.31) (9.69)
8
2
10
60−69
(4.30) (5.70)
Total
43
57
100
Page 8 – %%%%%%%%%%%%%%%%%%%%%%%%%%%%%%%%%
χ = ∑
2
(b)
( f i − e i ) 2 = ( 1 − 4.3 ) 2 + ...+ ( 2 − 5.7 ) 2
4.3
e i
5.7
= 26.6 on 7 d.f.
2
From tables, χ 7,
0.01 = 18.48
We have (strong) evidence against the hypothesis of no differences in probability of disease among age groups.
Plot given below. Linear model seems appropriate for middle ages,
but perhaps not for younger and older ages.
(a)
(iii)
30
40
50
60
x
%%%%%%%%%%
(b)
360 2
= 1237.5
8
S_{xx} = 17437.5 −
S_{yy} = 13.615 −
( − 2.9392) 2
= 12.535
8
S_{xy} = − 9.0429 −
( 360 ) ( − 2.9392) = 123.22
8
Least squares estimates:
\hat{\beta} =
S_{xy}
S_{xx}
=
123.22
= 0.09957
1237.5
\hat{\alpha} = y − \hat{\beta} x = − 0.3674 − 0.09957 ( 45 ) = − 4.85
Fitted line:
\[\hat{y} = − 4.85 + 0.09957 x\]
Page 9 – 
%%%%%%%%%%%%%%%%%%%%%%%%%%%%%%%%%
(c)
⎛
2 ⎞
S_{xy} 2 ⎞ ⎛
⎜ S_{yy} −
⎟ ⎜ 12.535 − 123.22 ⎟
⎜
S_{xx} ⎟ ⎜
1237.5 ⎟ ⎠
⎠ = ⎝
\sigma ˆ 2 = ⎝
= 0.04430
n − 2
6
()
se \hat{\beta} =
\sigma ˆ 2
0.210
=
= 0.0060
S_{xx}
1237. 5
and t 6, 0.005 = 3.707

%%%%%%%%%%%%%%%%%%%%%%%%%%%%%%%%%%%%%%%%%%%%%%%%%%%%%%%%%%%%%%%%%%%%%%%%%%%%%%%%%%%%%%%%%%%%%%%%%

99\% CI for \betâ is given by 0.09957 \pm 3.707(0.0060)
i.e. 0.09957 \pm 0.0222 or (0.0774, 0.1218)
(d)
In (ii) it was found that the probability of having the disease is
different for different age groups. In part (iii)(c) it was also found that
the probability of disease depends on age, as zero was not included in
the interval for the slope parameter.
The quality of the answers was mixed, with some common errors appearing in part (ii)
where many candidates failed to produce an appropariate 8 × 2 table (both “yes” and “no”
columns) and perform the correct chi-square test (with 7 d.f.).
It is noted that in part (ii)(b) the chi-square test can alternatively be performed by combining
some of the age groups (both columns) to achieve expected frequencies greater than 5, with
no change in the conclusion of the test. Although this is not strictly required in this case, full
credit was given to candidates that combined groups sensibly and completed the question
correctly.
\end{document}
