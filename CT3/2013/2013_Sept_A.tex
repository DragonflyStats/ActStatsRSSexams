1
The stem and leaf plot below shows 40 observations of an exchange rate.
1.21
1.22
1.23
1.24
1.25
1.26
1.27
1.28
9
4569
2679
3467889
011222345677778
00346688
1
For these data,
2
\sum  x = 50.000.
(i) Find the mean, median and mode.
[3]
(ii) State, with reasons, which measure of those considered in part (i) you would
prefer to use to estimate the central point of the observations.

[Total 4]
An insurance company experiences claims at a constant rate of 150 per year.
Find the approximate probability that the company receives more than 90 claims in a
period of six months.

3
The random variable X has a distribution with probability density function given by
⎧ 2 x
⎪
f ( x ) = ⎨ \theta 2
⎪ 0
⎩
; 0 \leq  x \leq \theta
; x < 0 or x > \theta
where \theta is the parameter of the distribution.
(i)
Derive expressions in terms of \theta for the expected value and the variance of X.
[3]
Suppose that X 1 , X 2 ,..., X n is a random sample, with mean X , from the distribution
of X.
(ii)
3 X
is an unbiased estimator of \theta.
Show that the estimator \theta ˆ =
2

[Total 5]


1
(i)
Mean  =  50
 =  1.25
40
Median = 1.252
Mode = 1.257
(ii)
Mean. Distribution is roughly symmetrical with no outliers so no reason to use
anything else.
Generally well answered. In part (ii), answers claiming that the median is preferred due to
some skewness in the distribution were not penalised.
2
Annual claims ~ Poisson(150) so six-month claims, X ~ Poisson(75)
CLT gives approximate distribution N(75,75)
 X  75 90.5  75 
P  X  90   =  P  X  90.5   =  P 

  =  1  Φ  1.790   =  1  0.963  =  0.037
75 
 75
[Without continuity correction 1  Φ  1.732   =  0.042 ]
There were no particular problems with this question. Note that the continuity correction
must be applied for full marks.
3
(i)
E ( X )  =  

0
E ( X )  =  
2

 2 x 3 
2 
x 2 dx  =   2   = 

  3  )  0 3
2 x
 2
x
0

 2 x 4 
 2
dx
 = 
 = 
 2 
 2
  4   ) 0 2
2 x
So, var( X )  =  E ( X 2 )  [ E ( X )] 2  = 
(ii)
 3 X
E (  ˆ )  =  E 
 2
 2 4  2  2

 = 
2
9
18
 3
3
  =  E ( X )  =  E ( X )  =   , so estimator is unbiased.
2
 2
Generally well answered. A few problems were encountered when deriving the variance.
Page 3Subject CT3 – September 2013 – Examiners’ Report


%%%%%%%%%%%%%%%%%%%%%%%%%%%%%%%%%%%%%%%%%%%%%%%%%%%%%%%%%%%%%%%%%%%%%%%
4
(i)
First derive expected value:

E  X  =   x
x  =  1
1
 x
log(1   ) x

=


 x
1

   log(1   )    log(1   )

x  =  0 
= 

(1   ) log  1   
X  =  E  X   X  =  
 
(1   ) log(1    )

 

 X 1  
 log 1        =  0
(ii)
(a)
L     = 
  i
x i
(  log(1   )) n  x i
i
And l     =   n log   log  1       x i log     C
i
(b)
MLE given by:
dl   
d 
= 0 

n
 x i
 i = 0
 ˆ
log 1   ˆ 1   ˆ

 



 X 1   ˆ log 1   ˆ   ˆ  =  0
(iii)
The equation above needs to be solved numerically. Alternatively, the
likelihood (or log-likelihood) function can be plotted and the maximum can be
identified from the graph.
In part (ii)(a) of the question the log-likelihood was shown as being equal, rather than
proportional, to the given expression plus a constant (as given in the solution above).
Candidates did not seem to be confused by this, but marking was adjusted in relevant cases.
In general the question was not particularly well answered, mainly due to difficulties in the
mathematical operations involved in obtaining the log-likelihood function of non-standard
densities. Candidates are advised to practise their calculus skills to deal with such questions.
Page 4Subject CT3 – September 2013 – Examiners’ Report
5
(i)
 
E S
2


 =  E 
 
  X

 nX 2  
1
n
E X i 2 
E X 2
  = 

n  1
n  1
  n  1
2
i
 
 
 
and using E X 2  =  var  X    E ( X ) 
 
E S 2  = 
(ii)
(a)




n
n
σ 2  \mu  2 
σ 2 / n  \mu  2  =  σ 2
n  1
n  1
  ( n  1) S 2  
var 
  =  2( n  1)
2
  σ
 
 
 var S 2  = 
(b)
2
2( n  1)σ 4
( n  1) 2
 = 
2σ 4
( n  1)
Estimator gets better (more accurate) as n increases, as its variance
reduces.
(MSE also gets smaller)
This question was generally well answered. There were a few problems with determining the
expectation of the sample mean in part (i).
