\documentclass[a4paper,12pt]{article}

%%%%%%%%%%%%%%%%%%%%%%%%%%%%%%%%%%%%%%%%%%%%%%%%%%%%%%%%%%%%%%%%%%%%%%%%%%%%%%%%%%%%%%%%%%%%%%%%%%%%%%%%%%%%%%%%%%%%%%%%%%%%%%%%%%%%%%%%%%%%%%%%%%%%%%%%%%%%%%%%%%%%%%%%%%%%%%%%%%%%%%%%%%%%%%%%%%%%%%%%%%%%%%%%%%%%%%%%%%%%%%%%%%%%%%%%%%%%%%%%%%%%%%%%%%%%

\usepackage{eurosym}
\usepackage{vmargin}
\usepackage{amsmath}
\usepackage{graphics}
\usepackage{epsfig}
\usepackage{enumerate}
\usepackage{multicol}
\usepackage{subfigure}
\usepackage{fancyhdr}
\usepackage{listings}
\usepackage{framed}
\usepackage{graphicx}
\usepackage{amsmath}
\usepackage{chngpage}

%\usepackage{bigints}
\usepackage{vmargin}

% left top textwidth textheight headheight

% headsep footheight footskip

\setmargins{2.0cm}{2.5cm}{16 cm}{22cm}{0.5cm}{0cm}{1cm}{1cm}

\renewcommand{\baselinestretch}{1.3}

\setcounter{MaxMatrixCols}{10}

\begin{document}
\begin{enumerate}

%%%%%%%%%%%%%%%%%%%%%%%%%%%%%%%
\item %%[Total 8]
A motor insurance company has a portfolio of 100,000 policies. It distinguishes
between three groups of policyholders depending on the geographical region in which they live. The probability p of a policyholder submitting at least one claim during a year is given in the following table together with the number, n, of policyholders belonging to each group.. Each policyholder belongs to exactly one group and it is
assumed that they do not move from one group to another over time.
Group
p
A
B
C
0.15 0.1 0.05
20 20 60
n (in 1000s)
It is assumed that any individual policyholder submits a claim during any year
independently of claims submitted by other policyholders. It is also assumed that
whether a policyholder submits any claims in a year is independent of claims in
previous years conditional on belonging to a particular group.

\begin{enumerate}[(i)]
\item Show that the probability that a randomly selected policyholder will submit a claim in a particular year is 0.08.

\item Calculate the probability that a randomly selected policyholder will submit a claim in a particular year given that the policyholder is not in group C.

\item Calculate the probability for a randomly selected policyholder to belong to
group A given that the policyholder submitted a claim last year.

\item Calculate the probability that a randomly selected policyholder will submit a
claim in a particular year given that the policyholder submitted a claim in the
previous year. It is assumed that the insurance company does not know to
which group the policyholder belongs.

\item Calculate the probability that a randomly selected policyholder will submit a
claim in two consecutive years.
\end{enumerate}

%%%%%%%%%%%%%%%%%%%%%%%%%%%%%%%%%%%%%%%5
7
(i)
P (claim ) = P (claim| A ) P (A )  P (claim| B ) P (B )  P (claim| C ) P (C 
= 0.15 \times 0.2  0.1 \times 0.2  0.05 \times 0.6  =  0.08
(ii)
P (claim \cup A )  =  P (claim| A ) P [ A ]  =  0.15 \times 0.2  =  0.03
P (claim \cup B )  =  P (claim| B ) P [ B ]  =  0.1 \times 0.2  =  0.02
P ( claim \cup \bar{C} )  =  0.03  0.02  =  0.05
P ( claim | \bar{C} )  = 
P ( claim \cup \bar{C} )
P ( \bar{C} )
(iii) P (A |claim last year )  = 
(iv) P (B |claim last year )  = 
 = 
0.05
 =  0.125
0.4
P (claim \cup A 
P (claim 
P (claim \cup B 
P (claim 
 =  0.03
 =  0.375
0.08
 =  0.02
 =  0.25
0.08
P (C |claim last year )  =  1  0.375  0.25  =  0.375
(CLY means “claim last year”)
P (claim|claim last year ) = P (claim| A ) P (A | CLY )  P (claim| C ) P (C | CLY 
 =  0.15 \times 0.375  0.1 \times 0.25  0.05 \times 0.375  =  0.1
(v)
Let Y be the event that a claim is submitted in two consecutive years
P (Y )  =  P (claim in second year|claim in first year ) P (claim in first year 
 =  0.1 \times 0.08  =  0.008
alternatively:
P (Y ) = P (Y | A ) P (A )  P (Y | B ) P (B )  P (Y | C ) P (C 
= 0.15 \times 0.15 \times 0.2  0.1 \times 0.1 \times 0.2  0.05 \times 0.05 \times 0.6  =  0.008
This turned out to be the most challenging question for the majority of candidates, with only
a small number of “full mark” answers. Many candidates did not attempt parts (iv) and (v) at
Page 6Subject CT3 – September 2013 – Examiners’ Report
all. The question deals with conditional probability concepts, starting with straightforward
parts but building up to more complex calculations .
\end{enumerate}
\end{document}
