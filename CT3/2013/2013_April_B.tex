\documentclass[a4paper,12pt]{article}

%%%%%%%%%%%%%%%%%%%%%%%%%%%%%%%%%%%%%%%%%%%%%%%%%%%%%%%%%%%%%%%%%%%%%%%%%%%%%%%%%%%%%%%%%%%%%%%%%%%%%%%%%%%%%%%%%%%%%%%%%%%%%%%%%%%%%%%%%%%%%%%%%%%%%%%%%%%%%%%%%%%%%%%%%%%%%%%%%%%%%%%%%%%%%%%%%%%%%%%%%%%%%%%%%%%%%%%%%%%%%%%%%%%%%%%%%%%%%%%%%%%%%%%%%%%%

\usepackage{eurosym}
\usepackage{vmargin}
\usepackage{amsmath}
\usepackage{graphics}
\usepackage{epsfig}
\usepackage{enumerate}
\usepackage{multicol}
\usepackage{subfigure}
\usepackage{fancyhdr}
\usepackage{listings}
\usepackage{framed}
\usepackage{graphicx}
\usepackage{amsmath}
\usepackage{chngpage}

%\usepackage{bigints}
\usepackage{vmargin}

% left top textwidth textheight headheight

% headsep footheight footskip

\setmargins{2.0cm}{2.5cm}{16 cm}{22cm}{0.5cm}{0cm}{1cm}{1cm}

\renewcommand{\baselinestretch}{1.3}

\setcounter{MaxMatrixCols}{10}

\begin{document}
\begin{enumerate}
\item 
Consider a random sample, X 1 , ... , X n , from a normal $N(\mu, \sigma^2 )$ distribution, with
sample mean X and sample variance S 2 .
(i) Define carefully what it means to say that X 1 , ... , X n is a random sample from
a normal distribution.

(ii) State what is known about the distributions of X and S 2 in this case,
including the dependencies between the two statistics.
(iii)
CT3 A2013–2
Define the t -distribution and explain its relationship with X and S 2 .

%%%%%%%%%%%%%%%%%%%%%
\item 5
Bank robberies in various countries are assumed to occur according to Poisson
processes with rates that vary from year to year. It was reported that the number of robberies in a particular country in a specific year was 123. The number of robberies
in a different country in the same year was 111. It can be assumed that each robbery is an independent event and that robberies occur independently in the two countries.
Determine an approximate 90\% confidence interval for the difference between the true yearly robbery rates in the two countries.

%%%%%%%%%%%%%%%%%%5

4
(i)
The random variables X 1 , ... , X n are independent
and identically distributed with X i ~ N ( \mu , \sigma 2 )
(ii)
X and S 2 are independent
X ~ N ( \mu , \sigma 2 / n )
( n − 1 ) S 2
\sigma
2
~ χ n 2 − 1
Page 3 – 
%%%%%%%%%%%%%%%%%%%%%%%%%%%%%%%%%
(iii)
t k = N (0,1) / χ 2 k / k where N (0,1) and χ 2 k are independent
This result can be applied here, and we get
X −\mu
~ t n − 1
S / n

%%Mixed quality in the answers. Some candidates answered part (iii) in the process of answering part (ii) – this did not always show clear understanding, but full marks were given.
%%%%%%%%%%%%%%%%%%%%%%%%%%%%%%%%%%%%%%%%%%%%%%%%%%%%%%%%%%%%%%%%%%%%%%%%%%%%%%%%%%%%%%
5
Under given assumptions X 1 ~ Poisson(\lambda 1 ), X 2 ~ Poisson(\lambda 2 )
and approximately
X 1 ~ N(\lambda 1 , \lambda 1 ), X 2 ~ N(\lambda 2 , \lambda 2 )
giving X 1 − X 2 ~ N(\lambda 1 − \lambda 2 , \lambda 1 + \lambda 2 ), or
X 1 − X 2 − ( \lambda 1 − \lambda 2 )
∼ N (0,1)
\lambda 1 + \lambda 2
Approximate 90% interval given as
X 1 − X 2 \pm z 0.05 \lambda ˆ 1 + \lambda ˆ 2 = X 1 − X 2 \pm z 0.05 X 1 + X 2
= 12 \pm 1.6449 × (234) 1/2 = 12 \pm 25.1621 i.e. (−13.162, 37.162)
A common error here involved the normal approximation of the difference of the two
variables – especially its variance.
\end{document}
