\documentclass[a4paper,12pt]{article}

%%%%%%%%%%%%%%%%%%%%%%%%%%%%%%%%%%%%%%%%%%%%%%%%%%%%%%%%%%%%%%%%%%%%%%%%%%%%%%%%%%%%%%%%%%%%%%%%%%%%%%%%%%%%%%%%%%%%%%%%%%%%%%%%%%%%%%%%%%%%%%%%%%%%%%%%%%%%%%%%%%%%%%%%%%%%%%%%%%%%%%%%%%%%%%%%%%%%%%%%%%%%%%%%%%%%%%%%%%%%%%%%%%%%%%%%%%%%%%%%%%%%%%%%%%%%

\usepackage{eurosym}
\usepackage{vmargin}
\usepackage{amsmath}
\usepackage{graphics}
\usepackage{epsfig}
\usepackage{enumerate}
\usepackage{multicol}
\usepackage{subfigure}
\usepackage{fancyhdr}
\usepackage{listings}
\usepackage{framed}
\usepackage{graphicx}
\usepackage{amsmath}
\usepackage{chngpage}

%\usepackage{bigints}
\usepackage{vmargin}

% left top textwidth textheight headheight

% headsep footheight footskip

\setmargins{2.0cm}{2.5cm}{16 cm}{22cm}{0.5cm}{0cm}{1cm}{1cm}

\renewcommand{\baselinestretch}{1.3}

\setcounter{MaxMatrixCols}{10}

\begin{document}
\begin{enumerate}


%%%%%%%%%%%%%%%%%%%%%%%%%%%%%%%%%%%%%%%%%%%%%%%%%%%%%%%%%%%%%%%%%%%%%%%%%%%%%%%%%
%%CT3 S2012–8
%% Question 13 
\item The following data give the weight, in kilograms, of a random sample of 10 different
models of similar motorcycles and the distance, in metres, required to stop from a
speed of 20 miles per hour.
Weight x 314 317 320 326 331 339 346 354 361 369
Distance y 13.9 14.0 13.9 14.1 14.0 14.3 14.1 14.5 14.5 14.4
For these data: Σx = 3,377 , Σx2 =1,143,757 , Σy =141.7 ,
Σy2 =2,008.39 , Σxy =47,888.6
Also: Sxx = 3,344.1, Syy = 0.501, Sxy = 36.51
A scatter plot of the data is shown below.
(i) (a) Comment briefly on the association between weight and stopping
distance, based on the scatter plot.
(b) Calculate the correlation coefficient between the two variables.

(ii) Investigate the hypothesis that there is positive correlation between the weight
of the motorcycle and the stopping distance, using Fisher’s transformation of
the correlation coefficient. You should state clearly the hypotheses of your
test and any assumption that you need to make for the test to be valid. [6]
(iii) (a) Fit a linear regression model to these data with stopping distance being
the response variable and weight the explanatory variable.
(b) Calculate the coefficient of determination for this model and give its
interpretation.
320 330 340 350 360 370
13.9 14.0 14.1 14.2 14.3 14.4 14.5
Weight (kilograms)
Stopping distance (meters)
Stopping distance against motorcycle weight
CT3 S2012–9
(c) Calculate the expected change in stopping distance for every additional
10 kilograms of motorcycle weight according to the model fitted in
part (iii)(a).

[Total 13]
END OF PAPER
%%%%%%%%%%%%%%%%%%%
13 (i) (a) The scatter plot suggests a positive linear association between weight
and stopping distance.
(b) xy 0.892
xx yy
S
r
S S
= =
  (ii) We want to test H0: ρ = 0 against H1: ρ > 0 .
Need to assume that data come from a bivariate normal distribution.
Fisher’s (standardised) transformation statistic is given by
1 log 1
2 1 7 log 1.892 3.79
1/ ( 3) 2 0.108
r
r
n
⎛ + ⎞
⎜ − ⎟ ⎛ ⎞ ⎝ ⎠ = = − ⎜ ⎟ ⎝ ⎠
and under H0 this should be a value from the N(0,1) distribution.
This gives P-value = Pr(Z ≥ 3.79) ≈ 0.0001, so there is very strong evidence
against H0 and we conclude that motorcycle weight and stopping distance are
positively correlated.
[Or by considering critical values of N(0,1) distribution.]
(iii) (a) ˆ 36.51 0.01092
3344.1
xy
xx
S
S
β = = =
  \alphaˆ = y −βˆ x =14.17 − 0.01092*337.7 =10.4823
Fitted line is yˆ = 10.48 + 0.01092x
(b) R2 =
  2 36.512 0.7956
3344.1*0.501
xy
xx yy
S
S S
= =
  This gives the proportion of total variation explained by the model.
(Note that R2 can also be computed as r2.)
 – September 2012 – %%%%%%%%%%%%%%%%%%%%%%%%%%%%%%%%%%
Page 11
(c) For every additional unit (kilogram) of weight the stopping distance is
expected to increase by βˆ = 0.01092 metres. So, for 10 kilograms of
weight the distance is expected to increase by 0.109 meters.
Generally adequately answered. Identifying the correct hypotheses in part (ii) was
problematic in some cases, while many candidates failed to assume bivariate normality.
END OF %%%%%%%%%%%%%%%%%%%%%%%%%%%%%%%%%%
