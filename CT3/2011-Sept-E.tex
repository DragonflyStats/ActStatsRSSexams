8
Consider a random sample X 1 , ..., X n from a Poisson distribution with expectation
E[X i ] = λ. An estimator λ̂ for the parameter λ is given by the observed mean of the
sample, that is:
1 n
λ ˆ = ∑ X i .
n i = 1
(i)
Derive formulae for the expected value and the variance of λ̂ in terms of λ
and n.
[3]
Assume in parts (ii) to (v) that the true parameter value is λ = 0.25.
(ii) Calculate the exact probability that 0.2 ≤ λ̂ ≤ 0.3 if the sample size is n = 10.
[3]
(iii) Calculate the approximate probability that 0.2 ≤ λ̂ ≤ 0.3 if the sample size is
n = 10 using the following:
(a) the normal approximation to ∑ i = 1 X i with continuity correction
(b) the normal approximation to ∑ i = 1 X i without continuity correction.
n
n
[6]
(iv) Comment on the differences in your answers in parts (ii) and (iii).
[2]
(v) Calculate the minimal required sample size n for which the probability that
0.2 ≤ λ̂ ≤ 0.3 is at least 0.95, using the normal approximation without
continuity correction
[4]
Suppose a random sample of size n = 400 gives the estimate λ̂ = 0.27.
(vi)
Calculate a 95% confidence interval for λ.
CT3 S2011—5
[3]
[Total 21]
PLEASE TURN OVER9
In a recent study of attitudes to a proposed new piece of consumer legislation
(“proposal X”) independent random samples of 200 men and 200 women were asked
to state simply whether they were “for” (in favour of) , or “against”, the proposal.
The resulting frequencies, as reported by the consultants who carried out the survey,
are given in the following table:
For
Against
(i)
Men
138
62
Women
130
70
Carry out a formal chi-squared test to investigate whether or not an association
exists between gender and attitude to proposal X.
Note: in this and any later such tests in this question you should state the P-
value of the data and your conclusion clearly.
[6]
At a subsequent meeting to discuss these and other results, the consultants revealed
that they had in fact stratified the survey, sampling 100 men and 100 women in
England and 100 men and 100 women in Wales. The resulting frequencies were as
follows:
For
Against
England
Men
Women
82
66
18
34
Wales
Men
56
44
Women
64
36
A chi-squared test to investigate whether or not an association exists between gender
and attitude to proposal X in England gives χ 2 = 6.653, while an equivalent test for
Wales gives χ 2 = 1.333.
(ii)
(iii)
(a) Find the P-value for each of the chi-squared tests mentioned above and
state your conclusions regarding possible association between gender
and attitude to proposal X in England and in Wales.
(b) Discuss the results of the survey for England and Wales separately and
together, quoting relevant percentages to support your comments.
[9]
A different survey of 200 people conducted in each of England, Wales, and
Scotland gave the following percentages in favour of another proposal:
% in favour of proposal
England
62%
Wales
53%
Scotland
58%
A chi-squared test of association between country and attitude to the proposal
gives χ 2 = 3.332 on 2 degrees of freedom, with P-value 0.189.
Suppose a second survey of the same size is conducted in the three countries
and results in the same percentages in favour of the proposal as in the first
survey. The results of the two surveys are now combined, giving a survey
based on the attitudes of 1,200 people.
CT3 S2011—6State (or find) the results of a second chi-squared test for an association
between country and attitude to the proposal, based on the overall
survey of 1,200 people.
[3]
(b) Comment briefly on the results.
[1]
[Total 19]
%%%%%%%%%%%%%%%%%%%%%%%%%%%%%%%%%%%%%%%%%%%%%%%%
9
(i)
H 0 : no association exists v. H 1 : association exists
for
against
men
138
62
200
Under H 0 : expected frequencies:
O – E:
4
–4
women
130
70
200
134
66
268
132
400
134
66
–4
4
1
1
1 ⎞
⎛ 1
χ 2 = 4 2 ⎜
+
+
+
⎟ = 0.724
66 ⎠
⎝ 134 134 66
P -value = P (χ 21 > 0.724) = 0.395
No evidence against H 0 – we conclude that no association exists between
gender and attitude to proposal X.
[ Note: using Yates’ correction (not in the Core Reading)
P -value = P (χ 21 > 0.554) = 0.457]
(ii)
(a)
For England:
P -value = P ( χ 1 2 > 6.653) = 0.010
Evidence against H 0 – we reject it at the 1% level of testing and
conclude that an association exists between gender and attitude to
proposal X in England.
For Wales:
P -value = P ( χ 1 2 > 1.333) = 0.248
Page 7Subject CT3 (Probability and Mathematical Statistics) — September 2011 — Examiners’ Report
No evidence against H 0 – we conclude that there is no association
between gender and attitude to proposal X in Wales.
(b)
England: there is evidence of an association – 82% of men and only
66% of women support proposal X – these proportions are
significantly different.
Wales: there is no evidence of an association – 56% of men and 64%
of women support proposal X – these proportions are not significantly
different.
The effects are in different directions and cancel out to some extent
when the data are combined: now there is no evidence of an
association – overall 69% of men and 65% of women support proposal
X – these proportions are not significantly different.
The combined data give a misleading message – they hide the effect of
the factor “country” and fail to reveal that there is an association in
England.
(iii)
(a)
The χ 2 value doubles to 6.664
P -value = P (χ 22 > 6.664) = 0.0357
Conclusion: reject “no association” at the 3.6% level of testing and
conclude that an association does exist.
(b)
Comment: having more data with the same proportions provides strong
enough evidence to justify claiming that an association exists.
Caution required with the null and alternative hypotheses in (i) – some candidates got these
wrong. Also, the associated degrees of freedom were wrongly given in some cases. Part
(ii)(b) required comments on the results, but very few candidates did this. Part (iii) was not
well answered either.
