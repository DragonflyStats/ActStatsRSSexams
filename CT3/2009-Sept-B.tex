\documentclass[a4paper,12pt]{article}

%%%%%%%%%%%%%%%%%%%%%%%%%%%%%%%%%%%%%%%%%%%%%%%%%%%%%%%%%%%%%%%%%%%%%%%%%%%%%%%%%%%%%%%%%%%%%%%%%%%%%%%%%%%%%%%%%%%%%%%%%%%%%%%%%%%%%%%%%%%%%%%%%%%%%%%%%%%%%%%%%%%%%%%%%%%%%%%%%%%%%%%%%%%%%%%%%%%%%%%%%%%%%%%%%%%%%%%%%%%%%%%%%%%%%%%%%%%%%%%%%%%%%%%%%%%%

\usepackage{eurosym}
\usepackage{vmargin}
\usepackage{amsmath}
\usepackage{graphics}
\usepackage{epsfig}
\usepackage{enumerate}
\usepackage{multicol}
\usepackage{subfigure}
\usepackage{fancyhdr}
\usepackage{listings}
\usepackage{framed}
\usepackage{graphicx}
\usepackage{amsmath}
\usepackage{chngpage}

%\usepackage{bigints}
\usepackage{vmargin}

% left top textwidth textheight headheight

% headsep footheight footskip

\setmargins{2.0cm}{2.5cm}{16 cm}{22cm}{0.5cm}{0cm}{1cm}{1cm}

\renewcommand{\baselinestretch}{1.3}

\setcounter{MaxMatrixCols}{10}

\begin{document}
\begin{enumerate}
\item 
%% - 5
Let X be a random variable with probability density function given by
\[f ( x ) = 2 x θ − 2 ,
0 < x < θ .\]
Find an unbiased estimator of $theta$ , based on a single observation of X.
6
[4]
A random sample of size n is taken from an exponential distribution with parameter
$\lambda$, that is, with probability density function
f ( x ) = λ e −λ x ,
(i)
0 < x <∞ .
Determine the maximum likelihood estimator (MLE) of $\lambda$ .
[3]
Claim sizes for certain policies are modelled using an exponential distribution with
parameter $\lambda$  . A random sample of such claims results in the value of the MLE of $\lambda$  as
λ ˆ = 0.00124 .
A large claim is defined as one greater than £4,000 and the claims manager is
particularly interested in p, the probability that a claim is a large claim.
(ii)
Determine p̂ , the MLE of p, explaining why it is the MLE.
%%%%%%%%%%%%%%%%%%%%%%%%%%%%%%%%%%%%%%%%%%%%%%%%%%%%%%%%%%%%%%%%%%%%%%%%%%%%%%%%%%%%%%%%%%%%%%%%%%%%%%5
[3]

5
E [ X ]
0
xf ( x ) dx
2
3
Consider Z
2
x 3
3
X
2
0
2 x 2
2
1
dx
2
.
3
0
1
3
E [ X ]
2
E [ Z ]
Z is an unbiased estimator of
.
.
2
6
(i)
L ( )
Π e
log L ( )
x i
n log
n
x i
e
x i
1
d
log L ( )
d
and
Equate to zero for the MLE
ˆ
n
X i
n
x i
1
1
X
(second derivative is clearly negative, so maximum)
1
%%%%%%%%%%%%%%%%%%%%%%%%%%%%%%%%%%%%%%%%%%%%%%%%%%%%%%%%%%%%%%%%%%%%%%%%%%%%%%%%%%%%%555
(ii)
p = P(X > 4000) = exp(–4000λ) for the exponential distribution
1
p ˆ exp( 4000 ˆ ) using the invariance property of MLE’s 1
p ˆ exp( 4000(0.00124)) 0.0070 1
\end{document}
