\documentclass[a4paper,12pt]{article}

%%%%%%%%%%%%%%%%%%%%%%%%%%%%%%%%%%%%%%%%%%%%%%%%%%%%%%%%%%%%%%%%%%%%%%%%%%%%%%%%%%%%%%%%%%%%%%%%%%%%%%%%%%%%%%%%%%%%%%%%%%%%%%%%%%%%%%%%%%%%%%%%%%%%%%%%%%%%%%%%%%%%%%%%%%%%%%%%%%%%%%%%%%%%%%%%%%%%%%%%%%%%%%%%%%%%%%%%%%%%%%%%%%%%%%%%%%%%%%%%%%%%%%%%%%%%

\usepackage{eurosym}
\usepackage{vmargin}
\usepackage{amsmath}
\usepackage{graphics}
\usepackage{epsfig}
\usepackage{enumerate}
\usepackage{multicol}
\usepackage{subfigure}
\usepackage{fancyhdr}
\usepackage{listings}
\usepackage{framed}
\usepackage{graphicx}
\usepackage{amsmath}
\usepackage{chngpage}

%\usepackage{bigints}
\usepackage{vmargin}

% left top textwidth textheight headheight

% headsep footheight footskip

\setmargins{2.0cm}{2.5cm}{16 cm}{22cm}{0.5cm}{0cm}{1cm}{1cm}

\renewcommand{\baselinestretch}{1.3}

\setcounter{MaxMatrixCols}{10}

\begin{document}
\begin{enumerate}
3
In a large population, 35% of voters intend to vote for party A at the next election. A
random sample of 200 voters is selected from this population and asked which party
they will vote for.
Calculate, approximately, the probability that 80 or more of the people in this sample
intend to vote for party A.
[4]
%%%%%%%%%%%%%%%%%%%%%%%%%%%%%%%%%%%%%%%%%%%%%
4
Let N be the random variable that describes the number of claims that an insurer
receives per month for one of its claim portfolios. We assume that N has a Poisson
distribution with E[N] = 50. The amount X i of each claim in the portfolio is normally
distributed with mean μ = 1,000 and variance σ 2 = 200 2 . The total amount of all
claims received during one month is
N
S = ∑ X i
i = 1
with S = 0 for N = 0. We assume that N, X 1 , X 2 , ... are all independent of each other.
(i) Specify the type of the distribution of S.
(ii) Calculate the mean and standard deviation of S.
CT3 A2011—2
[1]
[3]
[Total 4]

3
If X is the number of voters in the sample voting for party A, we have
X ~ Binomial(200, 0.35) and using the CLT X ~ N(70, 45.5) approximately.
Using continuity correction
79.5 − 70 ⎞
⎛
P(X ≥ 80) = P ⎜ Z >
⎟ = P(Z > 1.408)
45.5 ⎠
⎝
= 1 − P(Z < 1.408) = 1 − 0.920 = 0.08.
4
(i) Compound Poisson distribution
(ii)
E[S] = 50 * 1000 = 50,000
V[S] = 50 * E[S 2 ] = 50 * {V[X] + (E[X]) 2 } = 50 * {200 2 + 1000 2 } =
52,000,000
SD[S] = 7,211.10
5
