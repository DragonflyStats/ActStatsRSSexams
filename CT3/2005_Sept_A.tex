
\documentclass[a4paper,12pt]{article}

%%%%%%%%%%%%%%%%%%%%%%%%%%%%%%%%%%%%%%%%%%%%%%%%%%%%%%%%%%%%%%%%%%%%%%%%%%%%%%%%%%%%%%%%%%%%%%%%%%%%%%%%%%%%%%%%%%%%%%%%%%%%%%%%%%%%%%%%%%%%%%%%%%%%%%%%%%%%%%%%%%%%%%%%%%%%%%%%%%%%%%%%%%%%%%%%%%%%%%%%%%%%%%%%%%%%%%%%%%%%%%%%%%%%%%%%%%%%%%%%%%%%%%%%%%%%

\usepackage{eurosym}
\usepackage{vmargin}
\usepackage{amsmath}
\usepackage{graphics}
\usepackage{epsfig}
\usepackage{enumerate}
\usepackage{multicol}
\usepackage{subfigure}
\usepackage{fancyhdr}
\usepackage{listings}
\usepackage{framed}
\usepackage{graphicx}
\usepackage{amsmath}
\usepackage{chngpage}

%\usepackage{bigints}
\usepackage{vmargin}

% left top textwidth textheight headheight

% headsep footheight footskip

\setmargins{2.0cm}{2.5cm}{16 cm}{22cm}{0.5cm}{0cm}{1cm}{1cm}

\renewcommand{\baselinestretch}{1.3}

\setcounter{MaxMatrixCols}{10}

\begin{document}

%%%%%%%%%%%%%%%%%%%%%

\begin{enumerate}
\item 1
The table below gives the number of thunderstorms reported in a particular summer
month by 100 meteorological stations.
Number of thunderstorms:
Number of stations:
0
22
1
37
2
20
3
13
4
6
5
2
(a) Calculate the sample mean number of thunderstorms.
(b) Calculate the sample median number of thunderstorms.
(c) Comment briefly on the comparison of the mean and the median.
\item 
2 In an opinion poll, each individual in a random sample of 275 individuals from a large population is asked which political party he/she supports. If 45\% of the population
support party A, calculate (approximately) the probability that at least 116 of the sample support A.
\item 
3 Claim amounts on a certain type of policy are modelled as following a gamma distribution with parameters = 120 and = 1.2.
Calculate an approximate value for the probability that an individual claim amount exceeds 120, giving a reason for the approach you use.
\item 
4 Calculate a 99\% confidence interval for the percentage of claims for household accidental damage which are fully settled within six months of being submitted, given
that in a random sample of 100 submitted claims of this type, exactly 83 were fully settled within six months of being submitted.
\item 
5 A random sample of 500 claim amounts resulted in a mean of £237 and a standard deviation of £137.
Calculate an approximate 95\% confidence interval for the true underlying mean claim amount for such claims, explaining why the normal distribution can be used.

\end{enumerate}
%%%%%%%%%%%%%%%%%%%%%%%%%%%%%%%%%%%%%%%%%%%%%%%%%%%%%%%%%%%%%%%%%%%%%%%%%%%%%%%%%%%%%%%%%%%
September 2005
Examiners Report
150
1.5
100
(a) x
(b) Median is (101/2) th observation i.e. the mean of the 50 th and 51 st observations
so median = 1
(c) Mean is higher than the median as the data are positively skewed (skewed to
the right).
Let X denote the number in the sample who support party A.
\[X \sim Binomial(275, 0.45)\]
E[X] = 275 0.45 = 123.75
V[X] = 275 0.45 0.55 = 68.0625
The normal approximation to the binomial gives, using a continuity correction,
P ( X
3
116)
P ( X
115.5) 1
Gamma(120, 1.2) has mean
115.5 123.75
68.0625
1
( 1)
120
120
100 and variance
1.2
1.2 2
(1) 0.841
83.333
X N (100,9.129 2 ) by the Central Limit theorem (since the gamma variable is the
sum of 120 independent gamma(1,1.2) variables)
P X
4
120
P Z
120 100
9.129
2.191
1 0.98578 0.0142
Sample proportion = 0.83
99\% CI for the population proportion is 0.83
i.e. 0.83 0.0968 i.e. (0.733, 0.927)
99\% CI for percentage is thus (73.3% , 92.7%)
Page 2
[2.5758
(0.83 0.17/100) 1/2 ]Subject CT3 (Probability and Mathematical Statistics Core Technical)
5
September 2005
Examiners Report
n = 500 is very large, so the Central Limit Theorem justifies normality.
95\% CI is x 1.96
237 1.96
s
n
137
500
237 12.0 or (£225.0, £249.0)

\end{document}
