5
[3]
The number of claims, X, arising on each policy in a certain portfolio depends on
another random variable Y. X is considered to follow a Poisson distribution with
mean Y. The variable Y itself is assumed to have a gamma distribution with
parameters (a,b).
Find expressions for the unconditional moments E[X] and E[X 2 ] using appropriate
conditional moments.
[4]
CT3 S2007—26
Suppose that the time T, measured in days, until the next claim arises under a
portfolio of non-life insurance policies, follows an exponential distribution with
mean 2.
\begin{enumerate}[(i)]
\item Find the probability that no claim is made in the next one day period.
\item The median of a random variable is defined as the value for which the
cumulative distribution function of the variable is equal to 0.5.
Find the median time until the next claim arises.
\item 
Now let T 1 , T 2 , ..., T 30 be the times (in days) until the next claim arises under
each one of 30 similar portfolios of non-life insurance policies, and assume
that each T i , i = 1,...,30, follows an exponential distribution with mean 2,
independently of all others.
Calculate, approximately, the probability that the total of all 30 times which
elapse until a claim arises on each of the portfolios exceeds 45 days.
\end{enumerate}

%%%%%%%%%%%%%%%
5
E ( X ) = E Y { E X ( X Y ) } = E ( Y ) =
a
.
b
E ( X 2 ) = var( X ) + E 2 ( X )
with
var( X ) = E Y { var X ( X Y ) } + var Y { E X ( X Y ) }
= E ( Y ) + var( Y ) =
a a
+
b b 2
giving
a a a 2
E ( X ) = + 2 + 2 .
b b
b
2
{
{ var
}
[ OR E ( X 2 ) = E Y E X ( X 2 Y )
= E Y
X
}
( X Y ) + E X 2 ( X Y ) = E ( Y ) + E ( Y 2 )
= E ( Y ) + var( Y ) + E 2 ( Y )
=
6
a a a 2
.]
+ +
b b 2 b 2
%---------------------------------------------------------------%
(i) $T ~ Exp(0.5) and therefore P ( T > 1) = e − 0.5 × 1 = 0.6065$ .
(ii) The median, $M$, is such that
M
∫
0
M
f ( t ) dt = 0.5 ⇒ ∫ 0.5 e − 0.5 t dt = 0.5
0
which gives
\[1 − e − 0.5 M = 0.5 ⇒ M = − 2 log(0.5) , or M = 2 log(2) = 1.386 .\]
(Note: the cdf is available from the Yellow Book, p11.)
30
(iii)
From CLT, Y = ∑ T i ~ N (30 × 2, 30 × 4), i.e. N (60,120) , approximately.
i = 1
Then,
45 − 60 ⎞
⎛
P ( Y > 45) = P ⎜ Z >
⎟ = P ( Z > − 1.3693) = P ( Z < 1.3693) = 0.915.
120 ⎠
⎝
%%%%%%%%%%%%%%%%%%%%%%%%%%%%%%%%%%%%%%%%%%%%%%%%%%%%%%%%%%%%%%%%%%%%%%%%%%%%%%%
2
, from which we can then use the
[OR Y ~ gamma(30,1/2), that is Y ~ χ 60
normal approximation as above, or get P(Y > 45) = 0.922 (approximately) by
2
(Yellow Book p168).]
interpolating in tables of percentage points of χ 60
