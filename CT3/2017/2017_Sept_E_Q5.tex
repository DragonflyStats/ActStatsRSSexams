\documentclass[a4paper,12pt]{article}
%%%%%%%%%%%%%%%%%%%%%%%%%%%%%%%%%%%%%%%%%%%%%%%%%%%%%%%%%%%%%%%%%%%%%%%%%%%%%%%%%%%%%%%%%%%%%%%%%%%%%%%%%%%%%%%%%%%%%%%%%%%%%%%%%%%%%%%%%%%%%%%%%%%%%%%%%%%%%%%%%%%%%%%%%%%%%%%%%%%%%%%%%%%%%%%%%%%%%%%%%%%%%%%%%%%%%%%%%%%%%%%%%%%%%%%%%%%%%%%%%%%%%%%%%%%%
\usepackage{eurosym}
\usepackage{vmargin}
\usepackage{amsmath}
\usepackage{graphics}
\usepackage{epsfig}
\usepackage{enumerate}
\usepackage{multicol}
\usepackage{subfigure}
\usepackage{fancyhdr}
\usepackage{listings}
\usepackage{framed}
\usepackage{graphicx}
\usepackage{amsmath}
\usepackage{chngpage}
%\usepackage{bigints}

\usepackage{vmargin}
% left top textwidth textheight headheight
% headsep footheight footskip
\setmargins{2.0cm}{2.5cm}{16 cm}{22cm}{0.5cm}{0cm}{1cm}{1cm}
\renewcommand{\baselinestretch}{1.3}

\setcounter{MaxMatrixCols}{10}
\begin{document}
In an election between two candidates A and B in a large district, a sample poll of 100
voters chosen at random, indicated that 55% were in favour of candidate A.


%%%%%%%%%%%%%%%%%%%%%%%%%%%%%%%%%%%%%%%%%%%%%%%%%%%%%%
\newpage

\begin{table}[ht!]
 \centering
 \begin{tabular}{|p{15cm}|}
 \hline  
(i)
Calculate a 95% confidence interval for the proportion of all voters in favour
of candidate A based on the above sample. 
 \\ \hline
  \end{tabular}
\end{table}




Q5
(i)
Confidence interval (CI) is given by
\[ \hat{p} \pm 1.96 \sqrt{ \frac{\hat{p} (1 - \hat{p}}{n} }  = 

0.55 \pm 1.96 \sqrt{ \frac{0.55 \times 0.45 }{100} } = 0.55 \pm 0.0975 \]

The confidence interval is $(0.4525, 0.6475)$

%%%%%%%%%%%%%%%%%%%%%%%%%%%%%%%%%%%%%%%%%%%%%%%%%%%%%%
\newpage

\begin{table}[ht!]
 \centering
 \begin{tabular}{|p{15cm}|}
 \hline  
 A candidate is elected if they win more than 50\% of the votes. We want a test in which
the alternative hypothesis is that support for candidate A is such that she will win the
election.
(ii)
(a) Write down the hypotheses for this test in terms of a suitable
parameter.
(b) Explain whether or not the confidence interval in part (i) can be used to
test the hypothesis in part (ii)(a) at the 5\% level of significance.

 \\ \hline
  \end{tabular}
\end{table}


%%%%%%%%%%%%%%%%%%%%%%%%%%%%%%%%%%%%%%%%%%%%%%%%%%%%%%
\newpage

\begin{table}[ht!]
 \centering
 \begin{tabular}{|p{15cm}|}
 \hline  
 
It has been reported in the news that a new poll estimates support for candidate A at
52\%, with a margin of error of no more than $\pm 2\%$ with confidence 95\%.
(iii)

%%--- CT3 S2017–3 
Determine the minimum size of the sample of voters that was taken in this
new poll.
 \\ \hline
  \end{tabular}
\end{table}





%%%%%%%%%%%%%%%%%%%%%%%%%%%%%%%%%%%%%
\newpage


(ii)
(a)
(b)
(iii)

If p is proportion voting for candidate A, we want
H 0 : p = 0.5 (or p ≤ 0.5) v. H 1 : p> 0.5 
CI in part (i) is 2-sided, so cannot be used here. 
With p̂ = 0.52 we want the endpoints of the 95\% CI to be at most $0.52 \pm 0.02$.
This implies that
%-------%


\[  1.96 \sqrt{ \frac{ 0.52 \times 0.48 }{n} }  = 0.02 \]

\[  \frac{0.52 \times 0.48 }{(0.02 / 1.96)^2
}  = 2397.16  \]

The sample size must be at least 2398.

%%Most parts of the question were very well answered. In part (ii)(b),
%%many answers were not adequately explained, with candidates failing
%%to demonstrate understanding of the relation between one or two sided
%%--- Page5Subject CT3 (Probability and Mathematical Statistics Core Technical) – %%%%%%%%%%%%%%%%%%%%%%%%%%%%%
%%-- tests and CIs.
\end{document}
