\documentclass[a4paper,12pt]{article}

%%%%%%%%%%%%%%%%%%%%%%%%%%%%%%%%%%%%%%%%%%%%%%%%%%%%%%%%%%%%%%%%%%%%%%%%%%%%%%%%%%%%%%%%%%%%%%%%%%%%%%%%%%%%%%%%%%%%%%%%%%%%%%%%%%%%%%%%%%%%%%%%%%%%%%%%%%%%%%%%%%%%%%%%%%%%%%%%%%%%%%%%%%%%%%%%%%%%%%%%%%%%%%%%%%%%%%%%%%%%%%%%%%%%%%%%%%%%%%%%%%%%%%%%%%%%

\usepackage{eurosym}
\usepackage{vmargin}
\usepackage{amsmath}
\usepackage{graphics}
\usepackage{epsfig}
\usepackage{enumerate}
\usepackage{multicol}
\usepackage{subfigure}
\usepackage{fancyhdr}
\usepackage{listings}
\usepackage{framed}
\usepackage{graphicx}
\usepackage{amsmath}
\usepackage{chngpage}

%\usepackage{bigints}
\usepackage{vmargin}

% left top textwidth textheight headheight

% headsep footheight footskip

\setmargins{2.0cm}{2.5cm}{16 cm}{22cm}{0.5cm}{0cm}{1cm}{1cm}

\renewcommand{\baselinestretch}{1.3}

\setcounter{MaxMatrixCols}{10}

\begin{document}
In an election between two candidates A and B in a large district, a sample poll of 100
voters chosen at random, indicated that 55% were in favour of candidate A.
(i)
Calculate a 95% confidence interval for the proportion of all voters in favour
of candidate A based on the above sample.

A candidate is elected if they win more than 50% of the votes. We want a test in which
the alternative hypothesis is that support for candidate A is such that she will win the
election.
(ii)
(a) Write down the hypotheses for this test in terms of a suitable
parameter.
(b) Explain whether or not the confidence interval in part (i) can be used to
test the hypothesis in part (ii)(a) at the 5% level of significance.

It has been reported in the news that a new poll estimates support for candidate A at
52%, with a margin of error of no more than \pm2% with confidence 95%.
(iii)

CT3 S2017–3
Determine the minimum size of the sample of voters that was taken in this
new poll.



Q5
(i)
Confidence interval (CI) is given by
p ˆ  1 .96
p ˆ ( 1  p ˆ )
n
\;=\; 0.55  1.96
0.55*0.45
\;=\; 0.55  0.0975
100
i.e. (0.4525, 0.6475)
(ii)
(a)
(b)
(iii)

If p is proportion voting for candidate A, we want
H 0 : p = 0.5 (or p ≤ 0.5) v. H 1 : p> 0.5 
CI in part (i) is 2-sided, so cannot be used here. 
With p̂ = 0.52 we want the endpoints of the 95% CI to be at most 0.52 \pm0.02.
This implies that
1.96
0.52*0.48
0.52*0.48
\;=\; 0.02  n \;=\;
\;=\; 2397.16
n
( 0.02 /1.96 ) 2
The sample size must be at least 2398.

Most parts of the question were very well answered. In part (ii)(b),
many answers were not adequately explained, with candidates failing
to demonstrate understanding of the relation between one or two sided
Page 5Subject CT3 %%%%%%%%%%%%%%%%%%%%%%%%%%%%%%%%%%%%%%%%%%%%%%%%%%%%%%% %%%%%%%%%%%%%%%%%%%%%%%%%%%%%%%%%%%%%%%%%%%%%%%%%%%%%%%
tests and CIs.
