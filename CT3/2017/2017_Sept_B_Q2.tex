\documentclass[a4paper,12pt]{article}

%%%%%%%%%%%%%%%%%%%%%%%%%%%%%%%%%%%%%%%%%%%%%%%%%%%%%%%%%%%%%%%%%%%%%%%%%%%%%%%%%%%%%%%%%%%%%%%%%%%%%%%%%%%%%%%%%%%%%%%%%%%%%%%%%%%%%%%%%%%%%%%%%%%%%%%%%%%%%%%%%%%%%%%%%%%%%%%%%%%%%%%%%%%%%%%%%%%%%%%%%%%%%%%%%%%%%%%%%%%%%%%%%%%%%%%%%%%%%%%%%%%%%%%%%%%%

\usepackage{eurosym}
\usepackage{vmargin}
\usepackage{amsmath}
\usepackage{graphics}
\usepackage{epsfig}
\usepackage{enumerate}
\usepackage{multicol}
\usepackage{subfigure}
\usepackage{fancyhdr}
\usepackage{listings}
\usepackage{framed}
\usepackage{graphicx}
\usepackage{amsmath}
\usepackage{chngpage}

%\usepackage{bigints}
\usepackage{vmargin}

% left top textwidth textheight headheight

% headsep footheight footskip

\setmargins{2.0cm}{2.5cm}{16 cm}{22cm}{0.5cm}{0cm}{1cm}{1cm}

\renewcommand{\baselinestretch}{1.3}

\setcounter{MaxMatrixCols}{10}

\begin{document}
2
Assume that n independent random variables X 1 , X 2 , ..., X n are observed, which all
have the discrete distribution given by
x 1 2 3
P [ X i = x ] 0.6 0.3 0.1
and define the mean as X =
1 n
\sum X i .
n i = 1
(i) Determine the exact distribution of X for n = 2.
(ii) Determine the approximate distribution of X for n = 50 including all relevant
parameters, stating any assumptions you make.

[Total 8]

Q2
(i)
The only possible outcomes for X when n \;=\; 2 are 1, 1.5, 2, 2.5 and 3,
[0.5]
and the probabilities are given by:
P \left[  X \;=\; 1 \right] \;=\; P [ X 1 \;=\; 1 and X 2 \;=\; 1  \;=\; 0.6 2 \;=\; 0.36
P \left[  X \;=\; 1.5 \right] \;=\; 2 \times 0.6 \times 0.3 \;=\; 0.36
P \left[  X \;=\; 2 \right] \;=\; 2 \times 0.6 \times 0.1  0.3 \times 0.3 \;=\; 0.21
P \left[  X \;=\; 2.5 \right] \;=\; 2 \times 0.3 \times 0.1 \;=\; 0.06
P \left[  X \;=\; 3 \right] \;=\; 0.1 2 \;=\; 0.01
Marking: 0.5 for each correct probability
(ii)
[2.5]
For n \;=\; 50 we can use the CLT to obtain an approximate distribution of X .

We first find E \left[  X \right] \;=\; E [ X 1  \;=\; 0.6  2 \times 0.3  3 \times 0.1 \;=\; 1.5
And V ( X 1 ) \;=\; 0.5 2 \times 0.6  0.5 2 \times 0.3  1.5 2 \times 0.1 \;=\;
0.9
 0.225 \;=\; 0.45
4


Page 3Subject CT3 %%%%%%%%%%%%%%%%%%%%%%%%%%%%%%%%%%%%%%%%%%%%%%%%%%%%%%% %%%%%%%%%%%%%%%%%%%%%%%%%%%%%%%%%%%%%%%%%%%%%%%%%%%%%%%
The variance of X is then given by V ( X ) \;=\;
0.45
\;=\; 0.009
50

And the approximate distribution of X is N ( 1.5, 0.009 ) .

Part (i) was not well answered. Note that the question asks for the
exact distribution and involves a discrete random variable. Most
candidates answered in terms of an approximation to this distribution,
which was essentially part (ii) of the question.
