\documentclass[a4paper,12pt]{article}
%%%%%%%%%%%%%%%%%%%%%%%%%%%%%%%%%%%%%%%%%%%%%%%%%%%%%%%%%%%%%%%%%%%%%%%%%%%%%%%%%%%%%%%%%%%%%%%%%%%%%%%%%%%%%%%%%%%%%%%%%%%%%%%%%%%%%%%%%%%%%%%%%%%%%%%%%%%%%%%%%%%%%%%%%%%%%%%%%%%%%%%%%%%%%%%%%%%%%%%%%%%%%%%%%%%%%%%%%%%%%%%%%%%%%%%%%%%%%%%%%%%%%%%%%%%%
\usepackage{eurosym}
\usepackage{vmargin}
\usepackage{amsmath}
\usepackage{graphics}
\usepackage{epsfig}
\usepackage{enumerate}
\usepackage{multicol}
\usepackage{subfigure}
\usepackage{fancyhdr}
\usepackage{listings}
\usepackage{framed}
\usepackage{graphicx}
\usepackage{amsmath}
\usepackage{chngpage}
%\usepackage{bigints}

\usepackage{vmargin}
% left top textwidth textheight headheight
% headsep footheight footskip
\setmargins{2.0cm}{2.5cm}{16 cm}{22cm}{0.5cm}{0cm}{1cm}{1cm}
\renewcommand{\baselinestretch}{1.3}

\setcounter{MaxMatrixCols}{10}
\begin{document}
\large
%%-Question 2
\noindent Assume that $n$ independent random variables $\{X_1 , X_2 , \ldots , X_n\}$ are observed, which all
have the discrete distribution given by
\begin{center}
\begin{tabular}{|c|c|c|c|} \hline
x & 1 & 2 & 3 \\ \hline
P ( $X_i = x $) & 0.6 & 0.3&  0.1\\ \hline
\end{tabular}
\end{center}
and define the mean as 
\[ \bar{X} = \frac{1}{n} \sum ^{n}_{i=1} X_i\]

\begin{enumerate}[(a)]
\item %(i) 
Determine the exact distribution of $\bar{X}$ for $n = 2$.
\item %(ii) 
Determine the approximate distribution o/f $\bar{X}$ for $n = 50$ including all relevant
parameters, stating any assumptions you make.
\end{enumerate}
%%-- [Total 8]
%%%%%%%%%%%%%%%%%%%%%%%%%%%%%%%%%%%%%%%%%%%%%%%%%%%%%%%%%%%%%%%%%%%%%%%%%%%%%%%%%%%%%%%%%%%%%%
\large
\begin{framed}
For $n=4$, two possible outcomes are

\begin{itemize}
    \item  $\{3,1,1,2\}$ which has a mean of $\bar{X} = 1.75$
    \item $\{2,1,1,1\}$ which has a mean $\bar{X} = 1.25$
\end{itemize}

\end{framed}


\newpage
\begin{table}[ht!]
 \centering
 \begin{tabular}{|p{15cm}|}
 \hline  \large
 \noindent \textbf{Part (b)}\\ \large
Determine the exact distribution of $\bar{X}$ for $n = 2$.
 \\ \hline
  \end{tabular}
\end{table}

\noindent Possible outcomes $\{ X_1,X_2\}$ \\

\noindent \textit{Helpful to considering ordering, i.e First number is ${X_1}$ and second number is ${X_2}$ }
\begin{center}
\begin{tabular}{|l|c|}
\hline
 $\{X_1, X_2\}$ & Mean $\bar{X}$\\ \hline
$\{1 , 1\}$ & 1\\ \hline
$\{1 , 2\}$, $\{2 , 1\}$  & 1.5 \\ \hline
$\{1 , 3\}$, $\{3 , 1\}$  & 2 \\ \hline
$\{2 , 2\}$ & 2 \\ \hline
$\{1 , 2\}$, $\{2 , 1\}$ & 2.5 \\ \hline
$\{3 , 3\}$ & 3 \\ \hline
\end{tabular}
\end{center}

\noindent 
The only possible outcomes for $\bar{X}$ when $n = 2$ are 1, 1.5, 2, 2.5 and 3,
\medskip
The probabilities are given by:
\begin{itemize}
\item ${ \displaystyle P ( \bar{X} =  1 )\; = \; P ( X_1 = 1 \mbox{ and } X_2 = 1 ) = 0.6^2 \; = \; 0.36 }$
\item ${ \displaystyle P ( \bar{X} =  1.5 )\; = \; 2 \times ( 0.6 \times 0.3 ) \; = \; 0.36 }$
\item ${ \displaystyle P ( \bar{X} =  2 )\; = \; 2 ( 0.6 \times 0.1 ) +  (0.3^2 )\     ; 0.21 }$
\item ${ \displaystyle P ( \bar{X} =  2.5 )\; = \; 2 ( 0.3 ( 0.1 \; = \; 0.06 }$
\item ${ \displaystyle P ( \bar{X} =  3 )\; = \; 0.1^2 \; = \; 0.01 }$
\end{itemize}
%%-- Marking: 0.5 for each correct probability

\begin{table}[ht!]
 \centering
 \begin{tabular}{|p{15cm}|}
 \hline  \large 
\noindnt Determine the approximate distribution of $X$ for n = 50 including all relevant
parameters, stating any assumptions you make.

 \\ \hline
  \end{tabular}
\end{table}
(ii)
\begin{itemize}
    \item For n = 50 we can use the CLT to obtain an approximate distribution of X .

\item We first find \[E (\bar{X } )\; = \; E ( X_1 ) = 0.6 + 2 (\times 0.3 )+ 3 (\times0.1 = 1.5\]
\item 
\begin{eqnarray*}V ( X_1 ) &=& 0.5 2 ( 0.6  0.5 2 ( 0.3  1.5 2 ( 0.1 \\&=&
0.9
 0.225 \\ &=& 0.45
4


%%--- Page3Subject CT3 (Probability and Mathematical Statistics Core Technical) – %%%%%%%%%%%%%%%%%%%%%%%%%%%%%
\item The variance of X is then given by 

\[V ( \bar{X }) = \frac{0.45}{50} = 0.009 \]
\end{itemize}



And the approximate distribution of X is $N ( 1.5, 0.009 )$ .

% Part (i) was not well answered. Note that the question asks for the exact distribution and involves a discrete random variable. Most
% candidates answered in terms of an approximation to this distribution, which was essentially part (ii) of the question.
\end{document}
