\documentclass[a4paper,12pt]{article}
%%%%%%%%%%%%%%%%%%%%%%%%%%%%%%%%%%%%%%%%%%%%%%%%%%%%%%%%%%%%%%%%%%%%%%%%%%%%%%%%%%%%%%%%%%%%%%%%%%%%%%%%%%%%%%%%%%%%%%%%%%%%%%%%%%%%%%%%%%%%%%%%%%%%%%%%%%%%%%%%%%%%%%%%%%%%%%%%%%%%%%%%%%%%%%%%%%%%%%%%%%%%%%%%%%%%%%%%%%%%%%%%%%%%%%%%%%%%%%%%%%%%%%%%%%%%
\usepackage{eurosym}
\usepackage{vmargin}
\usepackage{amsmath}
\usepackage{graphics}
\usepackage{epsfig}
\usepackage{enumerate}
\usepackage{multicol}
\usepackage{subfigure}
\usepackage{fancyhdr}
\usepackage{listings}
\usepackage{framed}
\usepackage{graphicx}
\usepackage{amsmath}
\usepackage{chngpage}
%\usepackage{bigints}

\usepackage{vmargin}
% left top textwidth textheight headheight
% headsep footheight footskip
\setmargins{2.0cm}{2.5cm}{16 cm}{22cm}{0.5cm}{0cm}{1cm}{1cm}
\renewcommand{\baselinestretch}{1.3}

\setcounter{MaxMatrixCols}{10}
\begin{document}
\large
%%-- Question 8
The two random variables $X_1$ and $X_2$ are independent from each other and follow a
uniform $U(– \theta, \theta)$ distribution, where $\theta > 0$ is a parameter.
Let $\hat{\theta}_1 = 3Z$ denote a possible estimator of $\theta$, where $Z = \operatorname{max}(X_1,X_2)$.


\begin{enumerate}[(i)]
\item Show that the probability density function of Z is given by 
${ \displaystyle f_Z(z) = \frac{z+ \theta}{2\theta^2} }$
by first deriving its cumulative distribution function.
\item Show that ${ \displaystyle E ( Z ) = \frac{\theta}{3} }$. 
\item 
Derive the bias of $\hat{\theta}_1$ .
Derive an expression for the mean squared error (MSE) of $\hat{\theta}_1$ in terms
of the unknown parameter $\theta$.
\medskip
Let $\hat{\theta}_2 = 2 Z$ denote a different estimator of $\theta$, where again $Z = max(X_1 , X_2 )$.
\smallskip
\item Show that bias of$\hat{\theta}_2 = − \frac{\theta}{3}$. Also show that Mean Square Error of $\hat{\theta} 2 = \theta^2$ . 
\item 
Comment on how good the two estimators are, based on your answers in two parts of the question
\end{enumerate}


%%%%%%%%%%%%%%%%%%%%%%%%%%%%%%%%%%%%%%%%%%%%%%%%%%%%%%
\newpage

\begin{table}[ht!]
 \centering
 \begin{tabular}{|p{15cm}|}
 \hline  
 
 \\ \hline
  \end{tabular}
\end{table}



Q8
(i)

Q8
(i)
%%%%%%%%%%%%%%%%%%%%%%%%%%%%
\begin{eqnarray*}
F_{Z} ( z )  &=&  P \left[  z \leq  z \right]  \\ 
&=&  P ( max ( X 1 , X 2 ) \leq  z )  \\ 
&=&   P \left[  X 1 \leq  z , X 2 \leq  z \right] [1]
\end{eqnarray*}
and because of X 1 and X 2 being iid [1]

%%%%%%%%%%%%%%%%%%%%%%%%%%%%
\begin{eqnarray*}
F_{Z} ( z )  &=&  P \left[  X 1 \leq  z \right]\\
 &=& \left(  \frac{z +  \theta}{2\theta} \right)^2\\
\end{eqnarray*}
So, 
%%%%%%%%%%%%%%%%%%%%%%%%%%%%
\begin{eqnarray*}f_{Z} ( z )  &=&  \frac{d}{dz} \left(  \frac{z +  \theta}{2\theta} \right)^2 \\
&=& \frac{z +  \theta}{2\theta^2}\\
\end{eqnarray*}


%%%%%%%%%%%%%%%%%%%%%%%%%%%%
\begin{eqnarray*}
E(Z)
&=& \int^{\theta}{-\theta} z \frac{z +  \theta}{2\theta^2} dx \\
&=& \left[ \frac{z^3}{6 \theta^2}  \;+\; \frac{z^3}{4\theta} \right]^{\theta}{-\theta} \\
&=& \frac{\theta}{3}\\
\end{eqnarray*}


%%%%%%%%%%%%%%%%%%%%%%%%%%%%
\begin{eqnarray*}
bias( \hat{\theta}_{1} )  &=&  E \hat{\theta}_{1} \;-\; \theta  \\ 
&=&  3 E ( Z ) \;-\; \theta   \\ 
&=&  0\\
\end{eqnarray*}
%%%%%%%%%%%%%%%%%%%%%%%%%%%%


\begin{eqnarray*}
E(Z^2)
&=& \int^{\theta}{-\theta} z^2 \frac{z +  \theta}{2\theta^2} dx \\
&=& \left[ \frac{z^4}{8 \theta^2}  \;+\; \frac{z^3}{6\theta} \right]^{\theta}{-\theta} \\
&=& \frac{\theta^2}{3}\\
\end{eqnarray*}

%%%%%%%%%%%%%%%%%%%%%%%%%%%%%%%%%%%%%%%%

\[MSE( \hat{\theta}_{1} )  =  V \hat{\theta}_{1}\]
%%%%%%%%%%%%%%%%%%%%%%%%%%%%%%%%%%%%%%%%%%%%%%%%%%%%%%
\newpage

\begin{table}[ht!]
 \centering
 \begin{tabular}{|p{15cm}|}
 \hline  
 
 \\ \hline
  \end{tabular}
\end{table}



%%%%%%%%%%%%%%%%%%%%%%%%%%%%%%%%%%%%%%%
(ii)  z 3 z 2 

E \lambda Z  =  z 2 dz =  2   =
4   
3
2 
  6 


(iii) (a)
z 
\lambda 
bias(  ˆ 1 ) = E  ˆ 1   = 3 E \lambda Z    = 0

(b)

\lambda  =  z
E Z
2

2
z 
2  2


 z 4 z 3 
 2
dz =  2 
=

6   
3
  8 

\lambda 
MSE(  ˆ 1 ) = V  ˆ 1


  2  2 
= V \lambda 3 Z  = 9 V \lambda Z  = 9  E Z 2  E 2 \lambda Z   = 9    = 2  2


9  
  3

\lambda 
(iv)
(a)


bias(  ˆ 2 ) = E  ˆ 2   = 2 E \lambda Z    = 2   = 
3
3
\lambda 
%%%%%%%%%%%%%%%%%%%%%%%%%%%%%%%%%%%%%%%%%%%%%%%%%%%%%%
\newpage

\begin{table}[ht!]
 \centering
 \begin{tabular}{|p{15cm}|}
 \hline  
 
 \\ \hline
  \end{tabular}
\end{table}

%%%%%%%%%%%%%%%%%%%%%%%%%%%%%%%%%%%%%%%%%%%%%%%%%%%
%%--- Page7Subject CT3 (Probability and Mathematical Statistics Core Technical) – %%%%%%%%%%%%%%%%%%%%%%%%%%%%%
(b)
 2
 2
MSE(  ˆ 2 ) = bias 2  ˆ 2  V  ˆ 2 =
 V \lambda 2 Z  =
 4 V \lambda Z 
9
9
\lambda  \lambda 
=
 2
2  2
 4
=  2
9
9

(v)
̂ 1 is unbiased, but has much larger MSE compared to ̂ 2 (by factor of 2). 
On the other hand ̂ 2 has considerable bias, equal to a thirD_{O}f the true value
of the parameter.

Part (i) was not well answered with a number of candidates not
attempting it at all. Note that similar questions have appeared in past
examination papers in different contexts. Part (ii) was generally well
answered, although there were issues with the simple integration (e.g.
limits). Parts (iii), (iv) and (v) were well answered.
\end{document}
