\documentclass[a4paper,12pt]{article}

%%%%%%%%%%%%%%%%%%%%%%%%%%%%%%%%%%%%%%%%%%%%%%%%%%%%%%%%%%%%%%%%%%%%%%%%%%%%%%%%%%%%%%%%%%%%%%%%%%%%%%%%%%%%%%%%%%%%%%%%%%%%%%%%%%%%%%%%%%%%%%%%%%%%%%%%%%%%%%%%%%%%%%%%%%%%%%%%%%%%%%%%%%%%%%%%%%%%%%%%%%%%%%%%%%%%%%%%%%%%%%%%%%%%%%%%%%%%%%%%%%%%%%%%%%%%

\usepackage{eurosym}
\usepackage{vmargin}
\usepackage{amsmath}
\usepackage{graphics}
\usepackage{epsfig}
\usepackage{enumerate}
\usepackage{multicol}
\usepackage{subfigure}
\usepackage{fancyhdr}
\usepackage{listings}
\usepackage{framed}
\usepackage{graphicx}
\usepackage{amsmath}
\usepackage{chngpage}

%\usepackage{bigints}
\usepackage{vmargin}

% left top textwidth textheight headheight

% headsep footheight footskip

\setmargins{2.0cm}{2.5cm}{16 cm}{22cm}{0.5cm}{0cm}{1cm}{1cm}

\renewcommand{\baselinestretch}{1.3}

\setcounter{MaxMatrixCols}{10}

\begin{document}

8
The two random variables X 1 and X 2 are independent from each other and follow a
uniform U(– θ, θ) distribution, where θ > 0 is a parameter.
Let θ ˆ 1 = 3 Z denote a possible estimator of θ, where Z = max(X 1 , X 2 ).
(i)
(ii)
(iii)
z +θ
Show that the probability density function of Z is given by f Z ( z ) =
,
2 θ 2
by first deriving its cumulative distribution function.

θ
Show that E ( Z ) = .
3
(a)
Derive the bias of θ ˆ .
1
(b)
Derive an expression for the mean squared error (MSE) of θ ˆ 1 in terms
of the unknown parameter θ.
Let θ ˆ 2 = 2 Z denote a different estimator of θ, where again Z = max(X 1 , X 2 ).
(iv)
(v)

CT3 S2017–5
(a) θ
Show that bias θ ˆ 2 = − .
3
(b) Show that MSE θ ˆ 2 = θ 2 .
( )
( )
Comment on how good the two estimators are, based on your answers in parts
(iii) and (iv).



%%%%%%%%%%%%%%%%%%%%%%%%%%%%%%%%%%%%%%%%%%%%%%%%%%%%%%%%%%%%%%%%%%%%%%%%%%%%%%%%%%%%%%%%%%%%%%%%%

Q8
(i)
F_Z(z) \;=\; P [ z \leq z ] \;=\; P \left[  max ( X 1 , X 2 ) \leq z \right] \;=\; P [ X 1 \leq z , X 2 \leq z ] 
and because of X 1 and X 2 being iid 
F_Z(z) \;=\; P [ X 1 \leq z ]
2
 z \theta 
\;=\;

 2 \theta 
2

2
d  z \theta 
z \theta
So, F_Z(z) \;=\; 
 \;=\;
dz  2 \theta 
2 \theta 2
\theta

\theta
(ii)  z 3 z 2 
\theta
E ( Z ) \;=\;  z 2 dz \;=\;  2   \;=\;
4 \theta  
3
2 \theta
  6 \theta
\theta
\theta

%%%%%%%%%%%%%%%%%%%%%%%%%%%%%%%%%%%%%%%%%%%%%%
(iii) (a)
z \theta
( )
bias( \hat{\theta}1 ) \;=\; E \hat{\theta}1  \theta \;=\; 3 E ( Z )  \theta \;=\; 0
\theta
(b)

( ) \;=\;  z
E Z
2
\theta
2
z \theta
2 \theta 2

\theta
 z 4 z 3 
\theta 2
dz \;=\;  2 
\;=\;

6 \theta  
3
  8 \theta
\theta
( )
MSE( \hat{\theta}1 ) \;=\; V \hat{\theta}1


 \theta 2 \theta 2 
\;=\; V ( 3 Z ) \;=\; 9 V ( Z ) \;=\; 9  E Z 2  E 2 ( Z )  \;=\; 9    \;=\; 2 \theta 2


9  
  3


( )
%%%%%%%%%%%%%%%%%%%%%%%%%%%%%%%%%%%%%%%%%%%%%%%%%%%%%%%%%%%%%%%%%%%%%%%%%%%%%%5
(iv)
(a)
\theta
\theta
bias( \hat{\theta}2 ) \;=\; E \hat{\theta}2  \theta \;=\; 2 E ( Z )  \theta \;=\; 2  \theta \;=\; 
3
3
( )

%%--Page 7
%%-- Subject CT3 %%%%%%%%%%%%%%%%%%%%%%%%%%%%%%%%%%%%%%%%%%%%%%%%%%%%%%% %%%%%%%%%%%%%%%%%%%%%%%%%%%%%%%%%%%%%%%%%%%%%%%%%%%%%%%
September 2017 - Question 8
Part (b)

\begin{eqnarray*}
MSE( \hat{\theta}^{2} ) 
&=& bias 2 \hat{\theta}^{2}  V \hat{\theta}^{2} \\
&=& \frac{\theta^2}{9} \; + \; V ( 2 Z ) \\
&=& \frac{\theta^2}{9} \; + \; 4V ( Z ) \\
&=& \frac{\theta^2}{9} \; + \; 4 \times \frac{2\theta^2}{9} \\
&=& \frac{\theta^2}{9} \; + \;  \frac{8\theta^2}{9} \\
&=& \theta^2 \\
\end{eqnarray*}


(v)
\hat{\theta} 1 is unbiased, but has much larger MSE compared to \hat{\theta} 2 (by factor of 2). 
On the other hand \hat{\theta} 2 has considerable bias, equal to a third of the true value
of the parameter.
h
Part (i) was not well answered with a number of candidates not
attempting it at all. Note that similar questions have appeared in past
examination papers in different contexts. Part (ii) was generally well
answered, although there were issues with the simple integration (e.g.
limits). Parts (iii), (iv) and (v) were well answered.
\end{document}
