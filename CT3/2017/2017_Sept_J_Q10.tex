\documentclass[a4paper,12pt]{article}

%%%%%%%%%%%%%%%%%%%%%%%%%%%%%%%%%%%%%%%%%%%%%%%%%%%%%%%%%%%%%%%%%%%%%%%%%%%%%%%%%%%%%%%%%%%%%%%%%%%%%%%%%%%%%%%%%%%%%%%%%%%%%%%%%%%%%%%%%%%%%%%%%%%%%%%%%%%%%%%%%%%%%%%%%%%%%%%%%%%%%%%%%%%%%%%%%%%%%%%%%%%%%%%%%%%%%%%%%%%%%%%%%%%%%%%%%%%%%%%%%%%%%%%%%%%%

\usepackage{eurosym}
\usepackage{vmargin}
\usepackage{amsmath}
\usepackage{graphics}
\usepackage{epsfig}
\usepackage{enumerate}
\usepackage{multicol}
\usepackage{subfigure}
\usepackage{fancyhdr}
\usepackage{listings}
\usepackage{framed}
\usepackage{graphicx}
\usepackage{amsmath}
\usepackage{chngpage}

%\usepackage{bigints}
\usepackage{vmargin}

% left top textwidth textheight headheight

% headsep footheight footskip

\setmargins{2.0cm}{2.5cm}{16 cm}{22cm}{0.5cm}{0cm}{1cm}{1cm}

\renewcommand{\baselinestretch}{1.3}

\setcounter{MaxMatrixCols}{10}

\begin{document}
CT3 S2017–610
A company leases animals, which have been trained to perform certain tasks, for
use in the movie industry. The table below gives the number of tasks that each of
nine monkeys in a random sample can perform, along with the number of years the
monkeys have been working with the company.
Name Hellion Freeway SuSu Henri Jo Peepers Cleo Jeep Maggie
Years 10 8 6.5 6 5 1.5 0.5 0.5 0.4
Tasks 28 24 28 28 27 23 15 6 23
The random variable Y i denotes the number of years and T i the number of tasks for
each monkey i = 1, ..., 9.
\sum y i = 38.4, \sum y i 2 = 270.16, \sum y i t i = 1011.2, \sum t i = 202, \sum t i 2 = 4976
(i)
Explain the roles of response and explanatory variables in a linear regression.

(ii) Determine the correlation coefficient between Y and T.
(iii) Perform a statistical test using Fisher’s transformation to determine whether
the population correlation coefficient is significantly different from zero. [6]
(iv) Determine the parameters of a linear regression, including writing down the
equation.
[Total 15]

END OF PAPER
CT3 S2017–7
PLEASE TURN OVER

%%%%%%%%%%%%%%%%%%%%%%%%%%%%%%%%%%%%%%%%%%%%%%%%%%%%%%%%%%%%%%%%%%%%%%%%%%%%%%%%%%%%%%%%%%%%5

Q10
(i)
In bivariate data, the response variable is a random variable whose value may
be influenced by the value of the explanatory variable.

NOTE: This is not defined precisely in the core reading and should be marked
on the basis of understanding rather than precision.
(
(ii)
(iii)
)
S yy \;=\; 270.16  38.4 2 / 9 \;=\; 106.32 
S tt \;=\; 4976  202 2 / 9 \;=\; 442.22 
S yt \;=\; 1011.2  38.4 \times 202 / 9 \;=\; 149.33 
r \;=\; 149.33 / 106.32 \times 442.22 \;=\; 0.6887 
H 0 :  \;=\; 0 vs H 1 :   0 
1
1 + r
z r \;=\; log
\;=\; 0.8455
2
1  r 
1 

z r ~ N  z 0 ,
 \;=\; N ( 0,1/ 6 )
n  3 
 
 1 
Test statistic \;=\; 0.8455 /  
 6 
0. 5
\;=\; 2.071
Compare with Z 0.975 = 1.96


Therefore reject H 0 at 5% level (but note that we do not reject H 0 at 1%
level).

(iv)
 \;=\; S yt / S yy \;=\; 149.33 /106.32 \;=\; 1.405 
 \;=\; 202 / 9  1.405 \times ( 38.4 / 9 ) \;=\; 16.45 
t \;=\; 16.45 + 1.405 y 
The question was very well answered. In part (iii) some candidates
performed a t test instead of the test based on Fisher’s transformation.
Also, a few candidates were not clear in their answers about the
response and explanatory variables.
Page 10Subject CT3 %%%%%%%%%%%%%%%%%%%%%%%%%%%%%%%%%%%%%%%%%%%%%%%%%%%%%%% %%%%%%%%%%%%%%%%%%%%%%%%%%%%%%%%%%%%%%%%%%%%%%%%%%%%%%%
END OF EXAMINERS’ REPORT
Page 11
