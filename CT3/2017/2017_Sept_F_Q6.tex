\documentclass[a4paper,12pt]{article}

%%%%%%%%%%%%%%%%%%%%%%%%%%%%%%%%%%%%%%%%%%%%%%%%%%%%%%%%%%%%%%%%%%%%%%%%%%%%%%%%%%%%%%%%%%%%%%%%%%%%%%%%%%%%%%%%%%%%%%%%%%%%%%%%%%%%%%%%%%%%%%%%%%%%%%%%%%%%%%%%%%%%%%%%%%%%%%%%%%%%%%%%%%%%%%%%%%%%%%%%%%%%%%%%%%%%%%%%%%%%%%%%%%%%%%%%%%%%%%%%%%%%%%%%%%%%

\usepackage{eurosym}
\usepackage{vmargin}
\usepackage{amsmath}
\usepackage{graphics}
\usepackage{epsfig}
\usepackage{enumerate}
\usepackage{multicol}
\usepackage{subfigure}
\usepackage{fancyhdr}
\usepackage{listings}
\usepackage{framed}
\usepackage{graphicx}
\usepackage{amsmath}
\usepackage{chngpage}

%\usepackage{bigints}
\usepackage{vmargin}

% left top textwidth textheight headheight

% headsep footheight footskip

\setmargins{2.0cm}{2.5cm}{16 cm}{22cm}{0.5cm}{0cm}{1cm}{1cm}

\renewcommand{\baselinestretch}{1.3}

\setcounter{MaxMatrixCols}{10}

\begin{document}
[Total 6]
PLEASE TURN OVER6
Some tea experts claim that the taste of a cup of tea does not change according to
whether the tea or the milk is added first to the cup. To test the hypothesis that
people cannot tell the difference, an actuary organises a tasting experiment where an
individual is asked to taste 10 randomly presented cups of tea; five of these cups were
prepared with tea added first, and the other five with milk added first. The individual
does not know how many cups of tea where prepared in either way.
Under the null hypothesis that people cannot tell the difference, the probability of
correctly recognising the preparation method purely by chance is considered to be 0.5.
The actuary adopts the following decision rule for testing the hypothesis:
Reject the null hypothesis if seven or more cups are correctly identified; conclude
that the individual can tell the difference.
Do not reject the null hypothesis if less than seven cups are correctly identified;
conclude that the individual cannot tell the difference.
(i)
Determine the probability of a type I error for this test.

Based on past experience we can quantify the probability p that an individual
recognises the tea preparation method correctly.
(ii)
(iii)

CT3 S2017–4
Determine the probability of a type II error assuming that the true value of
p is:
(a) p = 0.7. 
(b) p = 0.8. 
Comment on the power of the test, based on your answers in part (ii).

Q6
(i)
The null hypothesis here is that the individual is guessing, i.e. H 0 : p = 0.5
where p is the probability of success in a binomial(10, p) experiment.

P(type I error) = P(reject null hypothesis when it is correct, i.e. when p = 0.5)

= P(correct answers ≥ 7 | p = 0.5)
(ii)
10
 10 
\;=\;    0.5 x (1  0.5) 10  x \;=\; 1  0.8281 \;=\; 0.1719 (using tables)
x 
x \;=\; 7  
(a) P(type II error) = P(accept H 0 when it is false, i.e. when p = 0.7) 
= P(correct answers < 7 | p = 0.7) = 0.3504 
(b)
P(type II error) = P(accept H 0 when it is false, i.e. when p = 0.8)
= P(correct answers < 7 | p = 0.8) = 0.1209
(iii)

The power of the test is given by 1−P(type II error), and therefore increases as
p increases. In other words, the power is higher (i.e. it is more likely to
correctly reject the hypothesis of guessing) when the individual is more skilled
in identifying the preparation method correctly.

The question was generally well answered, with most candidates
showing understanding of the concepts involved. There were some
problems however with calculating the probabilities – note that these
are given in the statistical tables. Credit was given to answers using a
normal approximation to the binomial distribution in parts (i) and (ii)
despite the sample size being small. In part (iii) all valid comments
were given credit.
