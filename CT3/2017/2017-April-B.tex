\documentclass[a4paper,12pt]{article}

%%%%%%%%%%%%%%%%%%%%%%%%%%%%%%%%%%%%%%%%%%%%%%%%%%%%%%%%%%%%%%%%%%%%%%%%%%%%%%%%%%%%%%%%%%%%%%%%%%%%%%%%%%%%%%%%%%%%%%%%%%%%%%%%%%%%%%%%%%%%%%%%%%%%%%%%%%%%%%%%%%%%%%%%%%%%%%%%%%%%%%%%%%%%%%%%%%%%%%%%%%%%%%%%%%%%%%%%%%%%%%%%%%%%%%%%%%%%%%%%%%%%%%%%%%%%

\usepackage{eurosym}
\usepackage{vmargin}
\usepackage{amsmath}
\usepackage{graphics}
\usepackage{epsfig}
\usepackage{enumerate}
\usepackage{multicol}
\usepackage{subfigure}
\usepackage{fancyhdr}
\usepackage{listings}
\usepackage{framed}
\usepackage{graphicx}
\usepackage{amsmath}
\usepackage{chngpage}

%\usepackage{bigints}
\usepackage{vmargin}

% left top textwidth textheight headheight

% headsep footheight footskip

\setmargins{2.0cm}{2.5cm}{16 cm}{22cm}{0.5cm}{0cm}{1cm}{1cm}

\renewcommand{\baselinestretch}{1.3}

\setcounter{MaxMatrixCols}{10}

\begin{document}
\begin{enumerate}
[ Total 4]3
Consider two random variables X and Y and assume that X and Y both follow a
standard normal distribution but are not independent. Define the random variables:
Z − = X – Y and Z + = X + Y.
4
(i) Determine the covariance between Z − and Z + .
(ii) Determine whether Z − and Z + are uncorrelated based on your answer in
part (i).
[1]
[Total 3]
[2]
An insurance company calculates car insurance premiums based on the age of the
policyholder according to three age groups: Group A consists of drivers younger than
22 years old; Group B consists of drivers 22–33 years old; and, Group C consists of
drivers older than 33 years.
Its portfolio consists of 10% Group A policyholders, 38% Group B policyholders and
52% Group C policyholders.
The probability of a claim in any 12-month period for a policyholder belonging to
Group A, B or C is 13%, 3% and 2%, respectively.
(i)
Calculate the probability that a randomly chosen policyholder from this
portfolio will make a claim during a 12-month period.
[3]
One of the company’s policyholders has just made a claim.
(ii)
CT3 A2017–3
Calculate the probability that the policyholder is younger than 22 years.
[2]
[Total 5]
%%%%%%%%%%%%%%%%%%%%%5
Q4
Denote by A, B, C the event that policyholder belongs to the corresponding group.
Also let F be the event that a policyholder makes a claim.
(i)
P(F) = P(F|A)P(A) + P(F|B)P(B) + P(F|C)P(C)
= 0.13*0.1 + 0.03*0.38 + 0.02*0.52 = 0.0348
P  A | F  
(ii)
P (F | A) P  A 
P  F 

0.13*0.1
 0.374
0.0348
[3]
[2]
[Total 5]
Generally very well answered, with no particular issues.
n
%%%%%%%%%%%%%%%%%%%%%%%%%%%%%%%%
Q5
(i)
The CLT states that as n   , approximately,
 X i approaches the
i  1
2
N ( n μ, n ) distribution.
(ii)
[2]
The mean of X i is 0.5 and its variance is 0.25.
Therefore, from CLT, Y 
50
 X i ~ N(25, 12.5) approximately.
[2]
i  1
(iii)
Exact distribution is Y 
50
 X i ~ Gamma(50, 2).
i  1
Page 4
[2]Subject CT3 (Probability and Mathematical Statistics Core Technical) – April 2017 – Examiners’ Report
(iv)
Generally the gamma distribution is an asymmetric distribution. Here, as n is
large, the CLT suggests that the distribution of Y is approximately normal, and
therefore symmetric.
[2]
[Total 8]
The performance in parts (i)–(iii) was generally good. In part (i) the
approximation needs to be clearly indicated in the answer. In part (iv) a
typical issue was not giving a direct conclusion on the shape based on the
approximation.
