\documentclass[a4paper,12pt]{article}

%%%%%%%%%%%%%%%%%%%%%%%%%%%%%%%%%%%%%%%%%%%%%%%%%%%%%%%%%%%%%%%%%%%%%%%%%%%%%%%%%%%%%%%%%%%%%%%%%%%%%%%%%%%%%%%%%%%%%%%%%%%%%%%%%%%%%%%%%%%%%%%%%%%%%%%%%%%%%%%%%%%%%%%%%%%%%%%%%%%%%%%%%%%%%%%%%%%%%%%%%%%%%%%%%%%%%%%%%%%%%%%%%%%%%%%%%%%%%%%%%%%%%%%%%%%%

\usepackage{eurosym}
\usepackage{vmargin}
\usepackage{amsmath}
\usepackage{graphics}
\usepackage{epsfig}
\usepackage{enumerate}
\usepackage{multicol}
\usepackage{subfigure}
\usepackage{fancyhdr}
\usepackage{listings}
\usepackage{framed}
\usepackage{graphicx}
\usepackage{amsmath}
\usepackage{chngpage}

%\usepackage{bigints}
\usepackage{vmargin}

% left top textwidth textheight headheight

% headsep footheight footskip

\setmargins{2.0cm}{2.5cm}{16 cm}{22cm}{0.5cm}{0cm}{1cm}{1cm}

\renewcommand{\baselinestretch}{1.3}

\setcounter{MaxMatrixCols}{10}

\begin{document}
\begin{enumerate}
%% [ Total 4]3
\item Consider two random variables X and Y and assume that X and Y both follow a standard normal distribution but are not independent. Define the random variables:
Z − = X – Y and Z + = X + Y.
4
\begin{enumerate}[(i)]
\item (i) Determine the covariance between Z − and Z + .
\item (ii) Determine whether Z − and Z + are uncorrelated based on your answer in part (i).
\end{enumerate}

Q3
(i)
Cov  X  Y , X  Y   Var  X   Cov  Y , X   Cov  X , Y   Var  Y   1  1  0
[2]
Alternative solution:
 
E  Z   E  X  Y   E  X   E  Y   0  0  0
E Z   E  X  Y   E  X   E  Y   0

Page 3Subject CT3 (Probability and Mathematical Statistics Core Technical) – April 2017 – Examiners’ Report



  
E Z  Z   E   X  Y  X  Y    E X 2  Y 2  E X 2  E ( Y 2 )  0

 
    
So: Cov Z  , Z   E Z  Z   E Z  E Z   0
Since cor( Z  , Z  ) = cov( Z  , Z  )/{sd( Z  ) sd( Z  )} it follows that Z  and
Z  are uncorrelated.
[1]
[Total 3]
(ii)
Generally well answered. A typical mistake in part (i) was not substituting for
Var(X) and Var(Y). Note that in part (ii) the answer needs to be justified,
e.g. by connecting correlation and covariance using the definition.



%%The performance in parts (i)–(iii) was generally good. In part (i) the approximation needs to be clearly indicated in the answer. In part (iv) a typical issue was not giving a direct conclusion on the shape based on the approximation.
\end{document}

%%%%%%%%%%%%%%%%%%%%%%%%%%%%%%%
