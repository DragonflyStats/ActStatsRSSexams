\documentclass[a4paper,12pt]{article}

%%%%%%%%%%%%%%%%%%%%%%%%%%%%%%%%%%%%%%%%%%%%%%%%%%%%%%%%%%%%%%%%%%%%%%%%%%%%%%%%%%%%%%%%%%%%%%%%%%%%%%%%%%%%%%%%%%%%%%%%%%%%%%%%%%%%%%%%%%%%%%%%%%%%%%%%%%%%%%%%%%%%%%%%%%%%%%%%%%%%%%%%%%%%%%%%%%%%%%%%%%%%%%%%%%%%%%%%%%%%%%%%%%%%%%%%%%%%%%%%%%%%%%%%%%%%

\usepackage{eurosym}
\usepackage{vmargin}
\usepackage{amsmath}
\usepackage{graphics}
\usepackage{epsfig}
\usepackage{enumerate}
\usepackage{multicol}
\usepackage{subfigure}
\usepackage{fancyhdr}
\usepackage{listings}
\usepackage{framed}
\usepackage{graphicx}
\usepackage{amsmath}
\usepackage{chngpage}

%\usepackage{bigints}
\usepackage{vmargin}

% left top textwidth textheight headheight

% headsep footheight footskip

\setmargins{2.0cm}{2.5cm}{16 cm}{22cm}{0.5cm}{0cm}{1cm}{1cm}

\renewcommand{\baselinestretch}{1.3}

\setcounter{MaxMatrixCols}{10}

\begin{document}
\begin{enumerate}
 Institute and Faculty of Actuaries1
A company y is collectin
ng data on i its sales. It has 39 sale s employee s and record
ds the
number of sales
s
that ea ach one mad de in a mon
nth. The dat ta are summ
marised by
grouping th he sales emp
ployees acc ording to th
he number of
o sales they y made and
presented in n the follow
wing chart:
8
7
6
5
4
3
2
1
0
75-80
81-101
102-126 6
127-15
59
160-18
86
187-2 200
Number of sales per pers son
made by a sales
Determine the mean an
nd standard d deviation of
o the numb
ber of sales m
s
employee.
[4]
2
v
X h having a dist tribution wi ith probabil lity density
Consider th he random variable
function:
f ( x ) = νλ x ν ν− 1 exp( −λ x ν ) , 0 < x < ∞
where v > 0 and λ > 0 are the para ameters of the
t distribut tion.
(i)
Sho ow that the cumulative
c
distribution
n function o f X is given n by:
 1 − exp( − λ x ν ) , x > 0
F ( x ) = 
0
, x ≤ 0

[2]
0,1) distribu
ution.
You are giv ven a value u = 0.671 f from the U(0
(ii)
CT3 A2 2017–2
Det termine by simulation
s
a value of th
he random variable
v
X w
when v = 1.1
1 and
λ = 0.2.
[2]

%%%%%%%%%%%%%%%%%%%%%%%%%%%%%%%%%%%%%%%%%%%%%%%%%%%%%%%%%%%%%%%%%%%%%%%%%%%%%%%%
Q1
Mid point
Count
Mean =
77.5
1
91
8
114
8
143
8
173
8
193.5
6
 fx 5406.5

 138.63
 f
39
Variance =
[2]
 fx 2  nx 2 803899.8  39  138.63 2

 1431.24
n  1
38
Standard dev = 37.83
[11⁄2]
[1⁄2]
[Total 4]
This was generally very well answered. There were some minor mistakes in
the calculations, typically taking the wrong midpoints for each range of values.
x
x
Q2
(i)
F ( x )    u  1 exp(  u  ) du    exp(  u  )   1  exp(  x  )


0
(ii)
[2]
0
Using the inverse CDF method
1
 log(1  u )  
u  1  exp(  x  )  x   




and with   1.1 ,   0.2 and u = 0.671 we have x = 4.756.
[1]
[1]
[Total 4]
The question was reasonably well answered, with some errors in part (i)
where many candidates used infinity as the upper limit in the integration.

%%%%%%%%%%%%%%%%%%%%%%%%%%%%%%%%%%%%%%%%%%%%%%%%%%%%%%%%%%%%%%%%%%%%%%%%%%%%%%%%%%%%%


Q2
(i)
F ( x )    u  1 exp(  u  ) du    exp(  u  )   1  exp(  x  )


0
(ii)
[2]
0
Using the inverse CDF method
1
 log(1  u )  
u  1  exp(  x  )  x   




and with   1.1 ,   0.2 and u = 0.671 we have x = 4.756.

%%%%%%%%%%%%%%%%%%%%%%%%%%%%%%%%%%%%%

[Total 4]
The question was reasonably well answered, with some errors in part (i)
where many candidates used infinity as the upper limit in the integration.
\end{document}
