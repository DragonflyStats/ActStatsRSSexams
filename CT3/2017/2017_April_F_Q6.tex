\documentclass[a4paper,12pt]{article}

%%%%%%%%%%%%%%%%%%%%%%%%%%%%%%%%%%%%%%%%%%%%%%%%%%%%%%%%%%%%%%%%%%%%%%%%%%%%%%%%%%%%%%%%%%%%%%%%%%%%%%%%%%%%%%%%%%%%%%%%%%%%%%%%%%%%%%%%%%%%%%%%%%%%%%%%%%%%%%%%%%%%%%%%%%%%%%%%%%%%%%%%%%%%%%%%%%%%%%%%%%%%%%%%%%%%%%%%%%%%%%%%%%%%%%%%%%%%%%%%%%%%%%%%%%%%

\usepackage{eurosym}
\usepackage{vmargin}
\usepackage{amsmath}
\usepackage{graphics}
\usepackage{epsfig}
\usepackage{enumerate}
\usepackage{multicol}
\usepackage{subfigure}
\usepackage{fancyhdr}
\usepackage{listings}
\usepackage{framed}
\usepackage{graphicx}
\usepackage{amsmath}
\usepackage{chngpage}

%\usepackage{bigints}
\usepackage{vmargin}

% left top textwidth textheight headheight

% headsep footheight footskip

\setmargins{2.0cm}{2.5cm}{16 cm}{22cm}{0.5cm}{0cm}{1cm}{1cm}

\renewcommand{\baselinestretch}{1.3}

\setcounter{MaxMatrixCols}{10}

\begin{document}

\item We consider the impact that different types of cars have on the amount spent on fuel
per month. Three different types of cars are considered: small, medium and large.
For each type of car a group of 15 drivers are asked about the amount of money (in \$)
spent on fuel per month. The results are summarised in the following table
Type of car
Sample mean
Sample standard deviation
Small Medium Large
70
16
75
19
83
16
For example, the 15 drivers of medium sized cars spent on average \$75 per month
with a sample standard deviation of \$19.

\begin{enumerate}[(a)]
\item %(i)
Perform a one-way analysis of variance to test the hypothesis that the type of
car has no impact on the monthly amount spent on fuel.
[6]
For some further investigation, only the difference between small and large cars is
considered.
\item (ii) Determine a 95\% confidence interval for the difference between the average
amount spent on fuel for small cars and large cars, stating any assumptions
you make.
\item 
(iii) Test the null hypothesis that the average fuel costs for small and large cars are
the same at a 5\% significance level against the alternative that the fuel costs
for small and large cars are different.
\end{enumerate}
%%%%%%%%%%%%%%%%%%%%%%%
\newpage
Q6


SS R  14 16 2  19 2  16 2  12, 222
(i)
Y 
70  75  83
 76
3


SS B  15  70  76    75  76    83  76 
F 2,42

2
2
2
  1, 290

SS B
1290 42
 2 
 2.216
SS R
2 12222
42

This is clearly a rather small value compared to the 5\% point from a F 2,42 distribution which is 3.22 (from Tables, using interpolation), so the null hypothesis is not rejected. We conclude that there is no evidence that the type
of car has an impact on the monthly amount of money spent on petrol.

(ii)
We need to assume equal variances and also that observations are independent
and normally distributed.

Pooled variance:
14  16 2  14 *16 2
 16 2  256
28
x L  x S  t 0.025,28  16
(iii)

2
2
 13  2.048  16 
  1.035, 24.965 
15
15

0   1.035, 24.965  , therefore, we would reject the null hypothesis of equal amounts spent on petrol for large cars and small cars.


Generally well answered. In part (ii) some candidates did not give the assumptions of the model. In part (iii) a number of candidates performed a

full t-test – this was not required but was given full credit where performed correctly. Also note that there are alternative ways for calculating the sums in part (i) and these also received full credit when performed correctly.

%%%--- Page 5 – %%%%%%%%%%%%%%%%%%%%%%%%%%%%%%%%%%

\end{document}
