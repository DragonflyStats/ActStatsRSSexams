\documentclass[a4paper,12pt]{article}

%%%%%%%%%%%%%%%%%%%%%%%%%%%%%%%%%%%%%%%%%%%%%%%%%%%%%%%%%%%%%%%%%%%%%%%%%%%%%%%%%%%%%%%%%%%%%%%%%%%%%%%%%%%%%%%%%%%%%%%%%%%%%%%%%%%%%%%%%%%%%%%%%%%%%%%%%%%%%%%%%%%%%%%%%%%%%%%%%%%%%%%%%%%%%%%%%%%%%%%%%%%%%%%%%%%%%%%%%%%%%%%%%%%%%%%%%%%%%%%%%%%%%%%%%%%%

\usepackage{eurosym}
\usepackage{vmargin}
\usepackage{amsmath}
\usepackage{graphics}
\usepackage{epsfig}
\usepackage{enumerate}
\usepackage{multicol}
\usepackage{subfigure}
\usepackage{fancyhdr}
\usepackage{listings}
\usepackage{framed}
\usepackage{graphicx}
\usepackage{amsmath}
\usepackage{chngpage}

%\usepackage{bigints}
\usepackage{vmargin}

% left top textwidth textheight headheight

% headsep footheight footskip

\setmargins{2.0cm}{2.5cm}{16 cm}{22cm}{0.5cm}{0cm}{1cm}{1cm}

\renewcommand{\baselinestretch}{1.3}

\setcounter{MaxMatrixCols}{10}

\begin{document}
%%- Question 1
The number of cans of fizzy drinks consumed by teenagers each day is the subject of
an empirical study. The following data have been recorded.
cans per day 0 1 2 3 4 5
number of teenagers 25 30 26 20 14 10
Assume that no teenager drinks more than five cans per day.
\begin{enumerate}
\item (i) Calculate the mean, median and mode for this sample.
\item (ii) Comment on the symmetry of the observed data, using your answer to part (i)
and without making any further calculations.
\end{enumerate}
%%%%%%%%%%%%%%%%%%%%%%%%%%%%%%%%%%%%%%%5
%% [Total 5]
\noindent \textbf{Solutions}
Q1
(i)
(ii)
Mean:
1
248
\;=\; 1.984
( 30  52  60  56  50 ) \;=\;
125
125

\begin{itemize}
    \item The median is the 63 rd largest observation, that is 2. 
\item The mode is the observation with the highest frequency, that is, 1 
\item The mean is almost identical to the median which is an indication for
symmetrical data as 50\% of data are below the mean. 
\item On the other hand, the mode is different. However, the frequency of the mode,
1, and the median, 2, are almost identical. Again, this is no strong sign for
asymmetry.
\end{itemize}


%% Generally well answered. Many candidates ignored the fact that the mean and median are almost identical, but credit was given if the answer was given in terms of the distribution being skewed, based on a graph or on values of the summary statistics.
\end{document}



