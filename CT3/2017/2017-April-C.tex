\documentclass[a4paper,12pt]{article}

%%%%%%%%%%%%%%%%%%%%%%%%%%%%%%%%%%%%%%%%%%%%%%%%%%%%%%%%%%%%%%%%%%%%%%%%%%%%%%%%%%%%%%%%%%%%%%%%%%%%%%%%%%%%%%%%%%%%%%%%%%%%%%%%%%%%%%%%%%%%%%%%%%%%%%%%%%%%%%%%%%%%%%%%%%%%%%%%%%%%%%%%%%%%%%%%%%%%%%%%%%%%%%%%%%%%%%%%%%%%%%%%%%%%%%%%%%%%%%%%%%%%%%%%%%%%

\usepackage{eurosym}
\usepackage{vmargin}
\usepackage{amsmath}
\usepackage{graphics}
\usepackage{epsfig}
\usepackage{enumerate}
\usepackage{multicol}
\usepackage{subfigure}
\usepackage{fancyhdr}
\usepackage{listings}
\usepackage{framed}
\usepackage{graphicx}
\usepackage{amsmath}
\usepackage{chngpage}

%\usepackage{bigints}
\usepackage{vmargin}

% left top textwidth textheight headheight

% headsep footheight footskip

\setmargins{2.0cm}{2.5cm}{16 cm}{22cm}{0.5cm}{0cm}{1cm}{1cm}

\renewcommand{\baselinestretch}{1.3}

\setcounter{MaxMatrixCols}{10}

\begin{document}
\begin{enumerate}
PLEASE TURN OVER5
Let X 1 , X 2 , ..., X n be a sequence of independent, identically distributed random
variables with finite mean \mu and finite (non-zero) variance σ 2 .
n
(i)
State the central limit theorem (CLT) in terms of the sum
 X i .

i = 1
Assume now that each X i , i = 1, 2, ..., 50, follows an exponential distribution with
50
parameter \lambda = 2 and let Y =  X i .
i = 1
6
(ii) Determine the approximate distribution of Y together with its parameters using
the CLT.

(iii) State the exact distribution of Y together with its parameters.
(iv) Comment on the shape of the distribution of Y based on your answers to parts
(ii) and (iii).

[Total 8]

We consider the impact that different types of cars have on the amount spent on fuel
per month. Three different types of cars are considered: small, medium and large.
For each type of car a group of 15 drivers are asked about the amount of money (in \$)
spent on fuel per month. The results are summarised in the following table
Type of car
Sample mean
Sample standard deviation
Small Medium Large
70
16
75
19
83
16
For example, the 15 drivers of medium sized cars spent on average \$75 per month
with a sample standard deviation of \$19.
(i)
Perform a one-way analysis of variance to test the hypothesis that the type of
car has no impact on the monthly amount spent on fuel.
[6]
For some further investigation, only the difference between small and large cars is
considered.
(ii) Determine a 95\% confidence interval for the difference between the average
amount spent on fuel for small cars and large cars, stating any assumptions
you make.

(iii) Test the null hypothesis that the average fuel costs for small and large cars are
the same at a 5\% significance level against the alternative that the fuel costs
for small and large cars are different.

[Total 11]
CT3 A2017–47
An investigation at a large airport focuses on the delay with which flights arrive. The
delay time X, in minutes, is the difference between the actual time of arrival and the
scheduled arrival time of delayed flights. Assume that X has an exponential
distribution with parameter \lambda > 0.
(i)
Derive the estimator \lambdâ for \lambda using the method of moments.

The following table shows the observed values of X for a random sample of ten
delayed flights.
45 20 120 90 60 30 45 90 60 150
(ii)
Estimate the value of \lambda for this sample using the method of moments.

To gain further insight into the distribution of flight delays, it is suggested that the
time at which a flight is scheduled to arrive during a day has an impact on the delay.
Therefore, assume now that X i has an exponential distribution with a parameter \lambda that
depends on the scheduled arrival time as follows:
X i ~ Exp( \lambda i ) with \lambda i = θ Z i
where the random variable Z i describes the scheduled arrival time (in minutes) after
midnight on the day of arrival for the i th randomly selected delayed flight and θ > 0 is
a parameter in this model.
(iii)
CT3 A2017–5
Derive the maximum likelihood estimator θ̂ for the parameter θ. You should
show that your solution is indeed a maximum.

%%%%%%%%%%%%%%%%%%%%%%%
Q6


SS R  14 16 2  19 2  16 2  12, 222
(i)
Y 
70  75  83
 76
3


SS B  15  70  76    75  76    83  76 
F 2,42

2
2
2
  1, 290

SS B
1290 42
 2 
 2.216
SS R
2 12222
42

This is clearly a rather small value compared to the 5\% point from a F 2,42
distribution which is 3.22 (from Tables, using interpolation), so the null
hypothesis is not rejected. We conclude that there is no evidence that the type
of car has an impact on the monthly amount of money spent on petrol.

(ii)
We need to assume equal variances and also that observations are independent
and normally distributed.

Pooled variance:
14  16 2  14 *16 2
 16 2  256
28
x L  x S  t 0.025,28  16
(iii)

2
2
 13  2.048  16 
  1.035, 24.965 
15
15

0   1.035, 24.965  , therefore, we would reject the null hypothesis of equal
amounts spent on petrol for large cars and small cars.

[Total 11]
Generally well answered. In part (ii) some candidates did not give the
assumptions of the model. In part (iii) a number of candidates performed a
Page 5Subject CT3 (Probability and Mathematical Statistics Core Technical) – April 2017 – %%%%%%%%%%%%%%%%%%%%%%%%%%%%%%%%%%
full t-test – this was not required but was given full credit where performed
correctly. Also note that there are alternative ways for calculating the sums in
part (i) and these also received full credit when performed correctly.
Q7

X  E
 X   1/  ˆ 
 ˆ  1/ X 
(ii) 1
1
 ˆ    0.014085
X 71 
(iii) L     L     n  Z i exp(-  Z i X i )
(i)
n

i  1
n n
i  1 i  1
l     n log    log Z i    Z i X i
l '    
 ˆ 

n n
 Z i X i  0
 
i  1 
n 
 i  1
n
Z i X i
This is indeed a maximum since the second derivative of l    is  n   2 <0 for
   ˆ  0.

[Total 8]
Parts (i) and (ii) were very well answered. Performance in part (iii) was
mixed. Most candidates exhibited a soundunderstanding of how to approach
the question; many encountered problems related to writing the sums and the
products and doing the required maths.
Page 6Subject CT3 (Probability and Mathematical Statistics Core Technical) – April 2017 – %%%%%%%%%%%%%%%%%%%%%%%%%%%%%%%%%%
Q8
p p p
8  4  2  1
15
8  15 p
   1 
p  1  p 
2 4 8
8
8
