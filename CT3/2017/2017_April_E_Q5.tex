\documentclass[a4paper,12pt]{article}

%%%%%%%%%%%%%%%%%%%%%%%%%%%%%%%%%%%%%%%%%%%%%%%%%%%%%%%%%%%%%%%%%%%%%%%%%%%%%%%%%%%%%%%%%%%%%%%%%%%%%%%%%%%%%%%%%%%%%%%%%%%%%%%%%%%%%%%%%%%%%%%%%%%%%%%%%%%%%%%%%%%%%%%%%%%%%%%%%%%%%%%%%%%%%%%%%%%%%%%%%%%%%%%%%%%%%%%%%%%%%%%%%%%%%%%%%%%%%%%%%%%%%%%%%%%%

\usepackage{eurosym}
\usepackage{vmargin}
\usepackage{amsmath}
\usepackage{graphics}
\usepackage{epsfig}
\usepackage{enumerate}
\usepackage{multicol}
\usepackage{subfigure}
\usepackage{fancyhdr}
\usepackage{listings}
\usepackage{framed}
\usepackage{graphicx}
\usepackage{amsmath}
\usepackage{chngpage}

%\usepackage{bigints}
\usepackage{vmargin}

% left top textwidth textheight headheight

% headsep footheight footskip

\setmargins{2.0cm}{2.5cm}{16 cm}{22cm}{0.5cm}{0cm}{1cm}{1cm}

\renewcommand{\baselinestretch}{1.3}

\setcounter{MaxMatrixCols}{10}

\begin{document}
\begin{enumerate}
PLEASE TURN OVER5
Let X 1 , X 2 , ..., X n be a sequence of independent, identically distributed random
variables with finite mean \mu and finite (non-zero) variance σ 2 .
n

\begin{enumerate}[(i)]
\item State the central limit theorem (CLT) in terms of the sum
 X i .

i = 1
Assume now that each X i , i = 1, 2, ..., 50, follows an exponential distribution with
50
parameter \lambda = 2 and let Y =  X i .
i = 1
6
\item (ii) Determine the approximate distribution of Y together with its parameters using
the CLT.

\item (iii) State the exact distribution of Y together with its parameters.
\item (iv) Comment on the shape of the distribution of Y based on your answers to parts
(ii) and (iii).

\end{enumerate}
%%%%%%%%%%%%%%%%%%%%%%%%%%%%%%%%%%%%%%
%%%%%%%%%%%%%%%%%%%%%%%%%%%%%%%%
Q5
(i)
\begin{itemize}
\item The CLT states that as n   , approximately,
 X i approaches the
i  1
2
N ( n \mu, n ) distribution.
(ii)
[2]
\item The mean of X i is 0.5 and its variance is 0.25.
Therefore, from CLT, Y 
50
 X i ~ N(25, 12.5) approximately.
[2]
i  1
(iii)
\item Exact distribution is Y 
50
 X i ~ Gamma(50, 2).
i  1
Page 4
%%[2]Subject CT3 (Probability and Mathematical Statistics Core Technical) – April 2017 – Examiners’ Report
(iv)
\item Generally the gamma distribution is an asymmetric distribution. Here, as $n$ is large, the CLT suggests that the distribution of Y is approximately normal, and therefore symmetric.
\end{itemize}
%%[2]
%%[Total 8]

%%%%%%%%%%%%%%%%%%%%%%%%%%%%%%%%%%%%%%%%%%%%%%%%%%%%%%%%%%%%%%%%%%%%%%%%%%%%%%%%%%%%%%%%%%%%%

\end{document}
