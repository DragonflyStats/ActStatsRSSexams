\documentclass[a4paper,12pt]{article}

%%%%%%%%%%%%%%%%%%%%%%%%%%%%%%%%%%%%%%%%%%%%%%%%%%%%%%%%%%%%%%%%%%%%%%%%%%%%%%%%%%%%%%%%%%%%%%%%%%%%%%%%%%%%%%%%%%%%%%%%%%%%%%%%%%%%%%%%%%%%%%%%%%%%%%%%%%%%%%%%%%%%%%%%%%%%%%%%%%%%%%%%%%%%%%%%%%%%%%%%%%%%%%%%%%%%%%%%%%%%%%%%%%%%%%%%%%%%%%%%%%%%%%%%%%%%

\usepackage{eurosym}
\usepackage{vmargin}
\usepackage{amsmath}
\usepackage{graphics}
\usepackage{epsfig}
\usepackage{enumerate}
\usepackage{multicol}
\usepackage{subfigure}
\usepackage{fancyhdr}
\usepackage{listings}
\usepackage{framed}
\usepackage{graphicx}
\usepackage{amsmath}
\usepackage{chngpage}

%\usepackage{bigints}
\usepackage{vmargin}

% left top textwidth textheight headheight

% headsep footheight footskip

\setmargins{2.0cm}{2.5cm}{16 cm}{22cm}{0.5cm}{0cm}{1cm}{1cm}

\renewcommand{\baselinestretch}{1.3}

\setcounter{MaxMatrixCols}{10}

\begin{document}[Total 16]
PLEASE TURN OVER9
The reaction time of drivers under the influence of alcohol is the subject of an
empirical study with an emphasis on the relationship between regular physical
exercise and reaction time. For this study drivers are categorised into four groups
according to the time spent exercising per week: group A (no regular exercising),
group B (0–2 hours), group C (2–4 hours) and group D (more than 4 hours per week).
Five drivers are randomly chosen from each group and their reaction time in seconds
is measured during a simulated driving test 60 minutes after ingesting a specific
amount of alcohol. The measured reaction times are in the following table.
\sum x i \sum x i 2
Reaction time, x i

%%- September 2017 Question 9

\begin{center}
\begin{tabular}{ccccccc} \hline
Group A& 0.95& 1.40& 1.00& 1.35& 1.30& 6.00& 7.375 \\ \hline
Group B& 1.25& 1.35& 0.90& 1.20& 1.35& 6.05& 7.458 \\ \hline
Group C& 1.10& 1.30& 1.10& 0.95& 1.30& 5.75& 6.703 \\ \hline
Group D& 1.05& 0.80& 1.25& 1.05& 1.05& 5.20& 5.510 \\ \hline
\end{tabular}
\end{center}

In a first approach the mean reaction time is studied without considering the time
spent on exercising.
\begin{enumerate}
\item (i)
Calculate a 95% confidence interval for the mean reaction time based on the
sample of all twenty drivers, stating any assumptions you make.
[6]
It is estimated that the average reaction time of drivers not under the influence of
alcohol is 0.9 seconds.
\item (ii)
Test the null hypothesis that the mean reaction time under the influence of
alcohol is equal to 0.9 seconds against the alternative hypothesis that the
reaction time is different from 0.9 seconds. The test decision should be based
on the sample of all twenty drivers. Use a significance level of 5%.

To investigate the impact of regular exercising on the reaction time under the
influence of alcohol an analyst suggests carrying out an ANOVA test based on the
grouping of drivers and the data in the table above.
\item (iii) State the mathematical model underlying the one-way analysis of variance
together with all assumptions.
\item 
(iv) Carry out an analysis of variance to test the null hypothesis that the time spent
on exercising has no effect on the average reaction time under the influence of
alcohol.[9]
\end{enumerate}

%%%%%%%%%%%%%%%%%%%%%%%%%%%%%%%%%%%%%%%%%%%%%%%%%%%%%%%%%%%%%%%%%%%%%%%%%%
Q9
(i)
Overall mean: X \;=\;
1
23
( 6 + 6.05 + 5.75 + 5.2 ) \;=\; \;=\; 1.15
20
20

Variance:
S 2 \;=\;
\;=\;
1 
( 7.375 + 7.458 + 6.703 + 5.510 )  20 \times 1.15 2 \right]

19
1
[ 27.046  26.45  \;=\; 0.031368
19

Standard deviation
S= 0.031316 \;=\; 0.177

With the sample not being large we assume that the sample comes from a
normal distribution.
[0.5]
t 0.025,19 \;=\; 2.093
[0.5]
95% Confidence interval using t result
0.177
0.177 

,1 .15 + t 0.025,19
 1.15  t 0.025,19

20
20 

[ 1.15  0.0828,1 .15 + 0.0828  \;=\; [ 1.067, 1 .233 
Page 8
Subject CT3 %%%%%%%%%%%%%%%%%%%%%%%%%%%%%%%%%%%%%%%%%%%%%%%%%%%%%%% %%%%%%%%%%%%%%%%%%%%%%%%%%%%%%%%%%%%%%%%%%%%%%%%%%%%%%%
(ii) The confidence interval in part (i) does not contain 0.9. Therefore, there is a
significant difference between the reaction time under alcohol intoxication and
0.9.

(iii) Y ij \;=\; \mu + \tau i + \varepsilon ij
[0.5]
where \mu is the overall mean and \tau i is the treatment effect of treatment i [0.5]
(
and \varepsilon ij being i.i.d. N 0, \sigma 2
)
[0.5]
In particular, it is assumed that the variance is the same in all groups.
(iv)
[0.5]
2
2
2
2
SS B \;=\; 5 \times   ( 1.2  1.15 ) + ( 1.21  1.15 ) + ( 1.15  1.15 ) + ( 1.04  1.15 )  


\;=\; 5 \times  0.05 2 + 0.06 2 + 0.11 2  \;=\; 0.091


SS R \;=\; 7.375 
\;=\; 27.046 

6.00 2
6.05 2
5.75 2
5.20 2
+ 7.458 
+ 6.703 
+ 5.510 
5
5
5
5
(
)
1
36 + 6.05 2 + 5.75 2 + 5.2 2 \;=\; 27.046  26.541 \;=\; 0.505
5
%%%%%%%%%%%%%%%%%%%%%%%%%%%%%%%%%%%%%%%%%%%%%%%


%%- September 2017 Question 9

\begin{center}
\begin{tabular}{|c|c|c|c|c|c|}
Source of variation	&	d.f. 	&	SS 	&	MSS	& F \\  \hline 
Between treatments	&	3 & 0.091 	& 0.030 & 	0.961 & \\ \hline 
Residual	&	16 & 	0.505 & 	0.032 & & \\ \hline

Total 	&	19	&	0.596	&		& \\ \hline 
\end{tabular}
\end{center}

The value of the test statistic is very close to 1, which needs to be compared to
an F 3,16 distribution (3.239 at 5% level).

We can therefore conclude that regular exercising has no impact on the reaction time under alcohol intoxication.

Alternatively, the critical values (at 10\%) is 2.462 > 0.961, and we come to the same conclusion.
Answers were very satisfactory overall. In part (ii) many candidates performed a full test, and received full credit, but note that this was not required. There were some numerical errors in calculating the
relevant sums in part (iv).
\end{document}
