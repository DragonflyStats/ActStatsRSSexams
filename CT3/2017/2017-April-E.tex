\documentclass[a4paper,12pt]{article}

%%%%%%%%%%%%%%%%%%%%%%%%%%%%%%%%%%%%%%%%%%%%%%%%%%%%%%%%%%%%%%%%%%%%%%%%%%%%%%%%%%%%%%%%%%%%%%%%%%%%%%%%%%%%%%%%%%%%%%%%%%%%%%%%%%%%%%%%%%%%%%%%%%%%%%%%%%%%%%%%%%%%%%%%%%%%%%%%%%%%%%%%%%%%%%%%%%%%%%%%%%%%%%%%%%%%%%%%%%%%%%%%%%%%%%%%%%%%%%%%%%%%%%%%%%%%

\usepackage{eurosym}
\usepackage{vmargin}
\usepackage{amsmath}
\usepackage{graphics}
\usepackage{epsfig}
\usepackage{enumerate}
\usepackage{multicol}
\usepackage{subfigure}
\usepackage{fancyhdr}
\usepackage{listings}
\usepackage{framed}
\usepackage{graphicx}
\usepackage{amsmath}
\usepackage{chngpage}

%\usepackage{bigints}
\usepackage{vmargin}

% left top textwidth textheight headheight

% headsep footheight footskip

\setmargins{2.0cm}{2.5cm}{16 cm}{22cm}{0.5cm}{0cm}{1cm}{1cm}

\renewcommand{\baselinestretch}{1.3}

\setcounter{MaxMatrixCols}{10}

\begin{document}
\begin{enumerate}
[Total 20]
9
A statistician is examining the survey methodology of a country’s national statistics department. It conducts much of its data collection by telephoning individuals
selected at random and asking them questions.
\item (i) Comment on whether this methodology will give a random sample.

\item (ii) Comment on whether this methodology will give a representative random
sample of the population.

The department has been experimenting with surveying in person by visiting randomly selected individuals in their homes. To make this economical the
department will only conduct surveys in a limited number of areas. It has asked the
statistician to validate the effectiveness of its process.

For its first trial it conducts a small survey in two locations on the daily time spent accessing social media and gets the following results (in minutes).
Area 1
Area 2
Number of
interviews
25
13
Mean
50.0
61.6
Standard
deviation
20.2
15.6
The statistician assumes that the underlying population is normally distributed.
(iii)
(a) Determine a 95\% confidence interval for the ratio of sample variances.
(b) Determine whether it is reasonable to assume that the variances are
equal.

(iv)
Perform a test at a 5\% significance level to investigate whether the means are
the same against a two sided alternative.

The statistician then learns that there is an expectation that the mean of Area 2 is
larger than the mean of Area 1.
(v)
(vi)
CT3 A2017–7
Perform a test to investigate whether the means are the same against an
appropriate alternative at the same significance level as in part (iv).
Comment on the results of parts (iv) and (v).


[Total 19]

%%%%%%%%%%%%%%%%%%
Q9
\item (i) A random sample should be independent and identically distributed. As
people are chosen at random the methodology should give a random sample.

\item (ii) While the sample chosen will be independent, they will not necessarily be
representative of the population as a whole. In many places phone ownership
may be restricted by economic, cultural or geographic limitations so some
parts of the population may be excluded.

(iii) (a)
 s 2

s 2
1
C . I .   1 2
, 1 2 F 12,24;0.975 
 s F 24,12;0.975 s

2
 2
 
 20.2 2 1 20.2 2

 
,
2.541
    0.555, 4.260 
 15.6 2 3.019 15.6 2

 
Alternative solution:
CI for
(b)
(iv)
is (0.234, 1.802)
As 1 lies in the confidence interval it is reasonable to assume the
standard deviations are the same.
H 0 :  1   2 vs . H 1 :  1   2




Pooled variance s P 2   n 1  1  s 1 2   n 2  1  s 2 2 /  n 1  n 2  2 


 24  20.2 2  12  15.6 2 /  25  13  2   353.15

Test statistic
   1   2  / s P
(v)
1 1
 1 1 

  61.6  50  / 353.15     1.805
n 1 n 2
 25 13 

t 36;0.975  2.028 >test statistic 
So do not reject H 0 at a 5\% significance level 
Now H 1 :  1   2 
Test statistic is the same, but now use t 36;0.95  1.688 
This time we reject H 0 
Page 9  – April 2017 – %%%%%%%%%%%%%%%%%%%%%%%%%%%%%%%%%%
(vi)
The results of the tests in parts (iv) and (v) were different. The additional
information allowed us to choose a more appropriate alternative hypothesis.

[Total 19]
The performance in this question was mixed. In parts \item (i) and \item (ii) many
candidates failed to demonstrate that they can distinguish between a sample
being random and being representative. A typical error in part ( iii) was the
use of wrong critical values. Note that a 2-sided test is required in part (iv) –
some candidates used a 1-sided test instead. Parts (v) and (vi) were
generally well answered.
