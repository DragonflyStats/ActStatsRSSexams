\documentclass[a4paper,12pt]{article}
%%%%%%%%%%%%%%%%%%%%%%%%%%%%%%%%%%%%%%%%%%%%%%%%%%%%%%%%%%%%%%%%%%%%%%%%%%%%%%%%%%%%%%%%%%%%%%%%%%%%%%%%%%%%%%%%%%%%%%%%%%%%%%%%%%%%%%%%%%%%%%%%%%%%%%%%%%%%%%%%%%%%%%%%%%%%%%%%%%%%%%%%%%%%%%%%%%%%%%%%%%%%%%%%%%%%%%%%%%%%%%%%%%%%%%%%%%%%%%%%%%%%%%%%%%%%
\usepackage{eurosym}
\usepackage{vmargin}
\usepackage{amsmath}
\usepackage{graphics}
\usepackage{epsfig}
\usepackage{enumerate}
\usepackage{multicol}
\usepackage{subfigure}
\usepackage{fancyhdr}
\usepackage{listings}
\usepackage{framed}
\usepackage{graphicx}
\usepackage{amsmath}
\usepackage{chngpage}
%\usepackage{bigints}

\usepackage{vmargin}
% left top textwidth textheight headheight
% headsep footheight footskip
\setmargins{2.0cm}{2.5cm}{16 cm}{22cm}{0.5cm}{0cm}{1cm}{1cm}
\renewcommand{\baselinestretch}{1.3}

\setcounter{MaxMatrixCols}{10}
\begin{document}

\large
\noindent Let $\{X_1 , X_2 , ..., X_n\}$ be a random sample of independent random variables from a
$N(\mu, \sigma^2 )$ distribution.


\begin{enumerate}[(a)]
    \item % (i) 
Comment on the shape of the sampling distribution of the sample variance $S^2$
with respect to the sample size n.
\item %(ii)
 Determine the variance of $S^2$ based on its sampling distribution.
\end{enumerate}

%%%%%%%%%%%%%%%%%%%%%%%%%%%%%%%%%%%%%%%%%%%%%%%%%%%%%%
\newpage

\begin{table}[ht!]
 \centering
 \begin{tabular}{|p{15cm}|}
 \hline  \large
\noindent \textbf{Part (a)}\\
\large
Comment on the shape of the sampling distribution of the sample variance $S^2$
with respect to the sample size n.
\smallskip
 \\ \hline
  \end{tabular}
\end{table}

$ \displaystyle{  \frac{(n-1)S^2}{\sigma^2}   \sim \chi^2_{n-1}  }$


\begin{itemize}
\item For small n the distribution is heavily skewed to the right.
\item As n gets larger the distribution becomes symmetric.
\end{itemize}
%%%%%%%%%%%%%%%%%%%%%%%%%%%%%%%%%%%%%%%%%%%%%%%%%%%%%%
\newpage

\begin{table}[ht!]
 \centering
 \begin{tabular}{|p{15cm}|}
 \hline  \large
\noindent \textbf{Part (b)}\\\large
 Determine the variance of $S^2$ based on its sampling distribution.

\\ \hline
\end{tabular}
\end{table}




\[ \displaystyle{  V \left( \frac{(n-1)S^2}{\sigma^2}  \right) = 2(n-1) }\]

\smallskip
\begin{framed}
\noindent \textbf{Recall:}
 \[ \operatorname{Var}(k X)  = k^2 \operatorname{Var}(X) \]
where $k$ is a constant.
\end{framed}

$ \displaystyle{  \operatorname{Var} \left( S^2 \right) = \frac{\sigma^4}{(n-1)^2} 2(n-1) }$

$ \displaystyle{  \operatorname{Var} \left( S^2 \right) = \frac{2\sigma^4}{(n-1)} }$



%In part (i) the quality of the answers was mixed. Many candidates
%failed to comment with respect to the sample size n. Part (ii) was better answered.
\end{document}
