\documentclass[a4paper,12pt]{article}

%%%%%%%%%%%%%%%%%%%%%%%%%%%%%%%%%%%%%%%%%%%%%%%%%%%%%%%%%%%%%%%%%%%%%%%%%%%%%%%%%%%%%%%%%%%%%%%%%%%%%%%%%%%%%%%%%%%%%%%%%%%%%%%%%%%%%%%%%%%%%%%%%%%%%%%%%%%%%%%%%%%%%%%%%%%%%%%%%%%%%%%%%%%%%%%%%%%%%%%%%%%%%%%%%%%%%%%%%%%%%%%%%%%%%%%%%%%%%%%%%%%%%%%%%%%%

\usepackage{eurosym}
\usepackage{vmargin}
\usepackage{amsmath}
\usepackage{graphics}
\usepackage{epsfig}
\usepackage{enumerate}
\usepackage{multicol}
\usepackage{subfigure}
\usepackage{fancyhdr}
\usepackage{listings}
\usepackage{framed}
\usepackage{graphicx}
\usepackage{amsmath}
\usepackage{chngpage}

%\usepackage{bigints}
\usepackage{vmargin}

% left top textwidth textheight headheight

% headsep footheight footskip

\setmargins{2.0cm}{2.5cm}{16 cm}{22cm}{0.5cm}{0cm}{1cm}{1cm}

\renewcommand{\baselinestretch}{1.3}

\setcounter{MaxMatrixCols}{10}

\begin{document}
3


Let X 1 , X 2 , ..., X n be a random sample of independent random variables from a
N(m, s 2 ) distribution.
(i) Comment on the shape of the sampling distribution of the sample variance S 2
with respect to the sample size n.
(ii)
 Determine the variance of S 2 based on its sampling distribution.
CT3 S2017–2

[Total 4]4

Q3
(i)
( n  1 ) S 2

2
~ X n 2  1 . For small n the distribution is heavily skewed to the right.

As n gets larger the distribution becomes symmetric.
(ii)

 ( n  1 ) S 2 
 4
2  4
2
2
V 
\;=\;
2
n

1

V
S
\;=\;
2
n

1

V
S
\;=\;

(
)
(
)


( n  1 )
 2
( n  1 ) 2


( )
( )

In part (i) the quality of the answers was mixed. Many candidates
failed to comment with respect to the sample size n. Part (ii) was better
answered.
