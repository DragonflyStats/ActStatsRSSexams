\documentclass[a4paper,12pt]{article}

%%%%%%%%%%%%%%%%%%%%%%%%%%%%%%%%%%%%%%%%%%%%%%%%%%%%%%%%%%%%%%%%%%%%%%%%%%%%%%%%%%%%%%%%%%%%%%%%%%%%%%%%%%%%%%%%%%%%%%%%%%%%%%%%%%%%%%%%%%%%%%%%%%%%%%%%%%%%%%%%%%%%%%%%%%%%%%%%%%%%%%%%%%%%%%%%%%%%%%%%%%%%%%%%%%%%%%%%%%%%%%%%%%%%%%%%%%%%%%%%%%%%%%%%%%%%

\usepackage{eurosym}
\usepackage{vmargin}
\usepackage{amsmath}
\usepackage{graphics}
\usepackage{epsfig}
\usepackage{enumerate}
\usepackage{multicol}
\usepackage{subfigure}
\usepackage{fancyhdr}
\usepackage{listings}
\usepackage{framed}
\usepackage{graphicx}
\usepackage{amsmath}
\usepackage{chngpage}

%\usepackage{bigints}
\usepackage{vmargin}

% left top textwidth textheight headheight

% headsep footheight footskip

\setmargins{2.0cm}{2.5cm}{16 cm}{22cm}{0.5cm}{0cm}{1cm}{1cm}

\renewcommand{\baselinestretch}{1.3}

\setcounter{MaxMatrixCols}{10}

\begin{document}

[Total 10]7
The annual number of claims an insurance company incurs, N, is believed to follow
a Poisson distribution with mean l. The value of each claim X i , i = 1, 2, ... follows
a known distribution with mean m and variance s 2 . The value of each claim is
independent of the value of any other claim and of the number of claims. Let
S = X 1 + X 2 + ... + X N denote the total claims in any given year.
(i)
Write down an expression for the moment generating function of S in terms of
the moment generating function of X i .
(ii)
Derive formulae for the mean and variance of S using your answer to part (i).


[Total 6]
Q7


(i) M S ( y ) \;=\; exp  ( M X ( y )  1 ) where M X ( y ) is the MGF of X
(ii) M S  ( y ) \;=\;  M X ' ( y ) exp  ( M X ( y )  1 )






 E ( S ) \;=\; M S  ( 0 ) \;=\;  M  X ( 0 ) exp  ( M X ( 0 )  1 ) \;=\; 





M S '' ( y ) \;=\;  M  X ( y ) exp  ( M X ( y )  1 )  (  M  X ( y ) ) exp  ( M X ( y )  1 ) 
Page 6
2Subject CT3 %%%%%%%%%%%%%%%%%%%%%%%%%%%%%%%%%%%%%%%%%%%%%%%%%%%%%%% %%%%%%%%%%%%%%%%%%%%%%%%%%%%%%%%%%%%%%%%%%%%%%%%%%%%%%%
()
() { (
( ) ) } (
( ) )
Þ M S ¢¢ 0 = l M ¢¢ X 0 exp l M X 0 -1 + l M ¢ X 0
(
)
2
{ ( ( ) ) }
exp l M X 0 -1
\;=\;   2   2  (  )
2

(
)
(
V ( S ) \;=\; M S  ( 0 )  ( M S  ( 0 ) ) \;=\;   2   2  (  )  (  ) \;=\;   2   2
2
2
2
)

The answer in part (i) was given in a variety of forms, and credit was
given as appropriate. There were various errors or incomplete answers
in part (ii), with many candidates showing knowledge of the required
steps but failing to derive the correct answer mainly due to algebra and
differentiation problems.
