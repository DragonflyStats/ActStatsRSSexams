\documentclass[a4paper,12pt]{article}

%%%%%%%%%%%%%%%%%%%%%%%%%%%%%%%%%%%%%%%%%%%%%%%%%%%%%%%%%%%%%%%%%%%%%%%%%%%%%%%%%%%%%%%%%%%%%%%%%%%%%%%%%%%%%%%%%%%%%%%%%%%%%%%%%%%%%%%%%%%%%%%%%%%%%%%%%%%%%%%%%%%%%%%%%%%%%%%%%%%%%%%%%%%%%%%%%%%%%%%%%%%%%%%%%%%%%%%%%%%%%%%%%%%%%%%%%%%%%%%%%%%%%%%%%%%%

\usepackage{eurosym}
\usepackage{vmargin}
\usepackage{amsmath}
\usepackage{graphics}
\usepackage{epsfig}
\usepackage{enumerate}
\usepackage{multicol}
\usepackage{subfigure}
\usepackage{fancyhdr}
\usepackage{listings}
\usepackage{framed}
\usepackage{graphicx}
\usepackage{amsmath}
\usepackage{chngpage}

%\usepackage{bigints}
\usepackage{vmargin}

% left top textwidth textheight headheight

% headsep footheight footskip

\setmargins{2.0cm}{2.5cm}{16 cm}{22cm}{0.5cm}{0cm}{1cm}{1cm}

\renewcommand{\baselinestretch}{1.3}

\setcounter{MaxMatrixCols}{10}

\begin{document}
\begin{enumerate}
PLEASE TURN OVER10
A geologist is trying to determine what causes sand granules to have different sizes.
She measures the gradient of nine different beaches in degrees, g, and the diameter in mm of the granules of sand on each beach, d.
0.63
0.17
g
d
0.70
0.19
0.82
0.22
0.88
0.235
1.15
0.235
1.50
0.30
4.40
0.35
7.30
0.42
11.30
0.85
 g = 28.68,  g 2 = 206.2462,  d = 2.97,  d 2 = 1.33525,  gd = 15.55855
\begin{enumerate}[(i)]
\item (i)
Determine the linear regression equation of d on g.

The geologist assumes that the error terms in the linear regression are normally distributed.
\item (ii)
Perform a test to determine whether the slope coefficient is significantly different from zero.

(iii) Determine a 95\% confidence interval for the mean estimate of d on a beach with a slope of exactly 3 degrees.

(iv) (a)
Plot the data from the table above.
(b)
Comment on the plot suggesting what the geologist might do to
improve her analysis.

\end{enumerate}

%%-- CT3 A2017–8
%%%%%%%%%%%%%%%%%%%%%%%%%
Q10
\item (i)
S gg

28.68 2 
  206.2462 
  114.8526

9  

S gd  15.55855 
 ˆ 
S gd
S gg

2.97 * 28.68
 6.09415
9
6.09415
 0.05306
114. 8526
 ˆ  d   g 
2.97  0.05306* 28.68
 0.1609
9
So d  0.1609  0.05306 g
\item (ii)
S dd  1.33525 
2.97 2
 0.35515
9





2
S dg
 2 1 
   S dd 
7 
S gg
  1 
6.09415 2 
   0.35515 
  0.004542
 7  
114.8526  
 
test statistic   ˆ / 
 2
0.004542
 0.05306 /
 8.438
S gg
114. 8526 
t 7;0.975  2.365 (two sided) so reject H 0 :   0
Page 10

Subject CT3  – April 2017 – %%%%%%%%%%%%%%%%%%%%%%%%%%%%%%%%%%
(iii)
If g 0 =3 then d ˆ 0  0.1609  0.05306 *3  0.320 0 8

2 
  1  g  g  2  

0
2  1  3  3.187  
ˆ
Var d 0   
  ˆ   
 *0.004542
n
S
9
114.8526
gg
 
 
 
 
 
= 5.060  10  4
 
C.I.= d ˆ 0  t 7;0.975 * Var d ˆ 0

1
2

 0.32008  2.365* 5.060  1 0  4

1
2
 (0.267, 0.373 )
(a)
0.9
0.8
(iv)

0.7
0.6
0.5
0.4
0.3
0.2
0.1
0
0
2
4
6
8
10
12
gradient of beach (degrees)

(b)
With only three observations for g>1.5, the slope is determined by a small amount of data. Getting more observations in that range would give a better analysis.

%%% Page 11Subject CT3  – April 2017 – %%%%%%%%%%%%%%%%%%%%%%%%%%%%%%%%%%
Alternatives (up to 2 marks total): could try a data transformation (e.g. logarithmic); other non-linear regression; more data.
[Total 18]

% Parts  (i) and (ii) were very well answered. However, a small number of candidates performed the regression using the wrong response and explanatory variables (g on d). Also in part(ii) some candidates attempted a test using the correlation coefficient. For full marks the equivalence of the two tests should be explicitly mentioned. In part (iii) there were some computational errors, while there were a few problems with the plot in part (iv) with inappropriate scales, missing axes labels etc.
\end{document}
