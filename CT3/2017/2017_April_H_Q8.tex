\documentclass[a4paper,12pt]{article}

%%%%%%%%%%%%%%%%%%%%%%%%%%%%%%%%%%%%%%%%%%%%%%%%%%%%%%%%%%%%%%%%%%%%%%%%%%%%%%%%%%%%%%%%%%%%%%%%%%%%%%%%%%%%%%%%%%%%%%%%%%%%%%%%%%%%%%%%%%%%%%%%%%%%%%%%%%%%%%%%%%%%%%%%%%%%%%%%%%%%%%%%%%%%%%%%%%%%%%%%%%%%%%%%%%%%%%%%%%%%%%%%%%%%%%%%%%%%%%%%%%%%%%%%%%%%

\usepackage{eurosym}
\usepackage{vmargin}
\usepackage{amsmath}
\usepackage{graphics}
\usepackage{epsfig}
\usepackage{enumerate}
\usepackage{multicol}
\usepackage{subfigure}
\usepackage{fancyhdr}
\usepackage{listings}
\usepackage{framed}
\usepackage{graphicx}
\usepackage{amsmath}
\usepackage{chngpage}

%\usepackage{bigints}
\usepackage{vmargin}

% left top textwidth textheight headheight

% headsep footheight footskip

\setmargins{2.0cm}{2.5cm}{16 cm}{22cm}{0.5cm}{0cm}{1cm}{1cm}

\renewcommand{\baselinestretch}{1.3}

\setcounter{MaxMatrixCols}{10}

\begin{document}
\begin{enumerate}
%%-- [Total 8]
%%-- PLEASE TURN OVER8
An actuary models the number of claims X per year per policy as a discrete random variable with the following distribution
Number of claims
Probability
0
*
1
p
2
3 More than 3
p/2 p/4
p/8
where p is an unknown parameter.
8 − 15 p
.

8
\item (i) Show that P [ X = 0] =
\item (ii) Determine the range of possible values of p.


In a sample of n independent policies there are N 0 policies with no claims during a year, N 1 policies with one claim, N 2 policies with two claims and N 3 policies with three claims. There are also some policies with more than three claims.

\item (iii)
Show that the maximum likelihood estimator p̂ for p based on observations of
N 0 , ..., N 3 in a sample of n independent claims is given by:
p ˆ =
8 n − N 0
.
15 n
You do not need to check that your solution is a maximum.

\item (iv) Explain why the distribution of N 0 is a Binomial distribution specifying its parameters.
\item
(v) Verify that p̂ is an unbiased estimator for p.

Assume that in a sample of size n = 300 there were 100 policies with no claims during
the previous year.

\item (vi)
Determine the value of the variance of the estimator p̂ .
\end{itemize}
\newpage
%%%%%%%%%%%%%%%%%
The insurance company has now decided to limit the maximum number of claims per year to four per policy, but otherwise continue to use the distribution above. The claim amount of any individual claim is assumed to have a normal distribution with
expectation 100 and standard deviation 20. Let S denote the total amount claimed in a
portfolio of 300 independent policies during a year. We assume that claim amounts
are independent of each other and independent of the number of claims.
Let X be the number of claims per policy per year and Y be the total number of claims
per year.
CT3 A2017–6(vii)
(a)
Show that E(X) = 3.25p and Var(X) = 7.25p − 10.5625p 2 .
Assume now that p = 0.2.
(b) Determine E(Y) and Var(Y).
(c) Determine the expected value and the standard deviation of S.
\newpage 

%%%%%%%%%%%%%%%%%%%%%%%%%%%%%%%%%%%%%%%%%%%%%%%%%%%%%%5
8
\begin{itemize}
\item (i) P  X  0   1  p 
\item (ii) To ensure that 0  P  X  k   1 for all k we only need to check this condition

 8 
for k  0,1 , and we need that p   0,  . All other probabilities will then
 15 
also be between 0 and 1.

(iii)
Let N 4  n  N 0  N 1  N 2  N 3 , the number of policies with more than
three claims.
MLE:
 8  15 p 
L  p   

 8 
N 0
p
N 1 
p 
 
 2 
N 2
 p 
 
 4 
N 3
 p 
 
 8 
N 4
log L  p   N 0 log  8  15 p    N 1  N 2  N 3  N 4  log p  C
where C is a constant which does not depend on p .

First derivative:
l '  p  
 15 N 0 N 1  N 2  N 3  N 4  15 N 0 n  N 0



 0
8  15 p
p
8  15 p
p

Solving this equation:
n  N 0
15 N 0
p
8  15 p
8 n  N 0



 p ˆ 
p
n  N 0
8  15 p
15 N 0
1 5 n
\item (iv)
N 0 has a binomial distribution since it counts the outcome “no claim” in
independent trials
 8  15 p 
The distribution of N 0 is B  n ,
 .
8 

(v)
E  p ˆ  
8  E  N 0  
 1 

1 5 
n 
And therefore with




we obtain E  p ˆ   p , so p̂ is unbiased.

Page 7Subject CT3  – April 2017 
2
\item (vi)
2
 8 
 8  1  15  15
Var  p ˆ   
 Var  N 0      1  p  p
8  8
 15 n 
 15  n 
p̂ 

8 300  100 8 2 16
  
 0.35555
15 300
15 3 45
\item 
Estimated variance of p̂ :
2
2
 8  1  15 16  15 16  8  1 1 2
   0.0002107
   1       
8 45  8 45  15  300 3 3
 15  n 
\item (vii)
(a)
E  X   p  2
[1⁄2]
p
p
p
 3  4  3.25 p
2
4
8

4
9
16
E  X 2   p  p  p  p  7.25 p
 
2
4
8 
Var  X   E  X 2   E  X   7.25 p  10.5625 p 2
  
2
(b)
Let Y 
300
 X i be the total number of claims.
i  1
Expected total number of claims: E  Y   300  3.25 p  975 p  195
[1⁄2]


Var( Y )  300 7.25 p  10.5625 p 2  308.25
(c)
E  S   195  100  19, 500
Var  S   195  20 2  308.25  100 2  3,160,500
Std  S   3,160,500  1, 777.78
[1⁄2]

[11⁄2]
[1⁄2]
[Total 20]
Parts \item (i) and \item (ii) were very well answered. In part (iii) most candidates followed the correct approach, but started with a wrong likelihood function.
Answers in part (iv) often did not provide a reasonable justification of why this was a binomial distribution. In part (v) most candidates gave answers that exhibited understanding of what bias is but failed to arrive at the final result. Performance in parts (vi) and (vii) was mixed, with many candidates using wrong formulas and missing the variance of the linear combination.
%%--- Page 8Subject CT3  – April 2017 
\end{itemize}
\end{document}
