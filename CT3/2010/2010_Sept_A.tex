\documentclass[a4paper,12pt]{article}

%%%%%%%%%%%%%%%%%%%%%%%%%%%%%%%%%%%%%%%%%%%%%%%%%%%%%%%%%%%%%%%%%%%%%%%%%%%%%%%%%%%%%%%%%%%%%%%%%%%%%%%%%%%%%%%%%%%%%%%%%%%%%%%%%%%%%%%%%%%%%%%%%%%%%%%%%%%%%%%%%%%%%%%%%%%%%%%%%%%%%%%%%%%%%%%%%%%%%%%%%%%%%%%%%%%%%%%%%%%%%%%%%%%%%%%%%%%%%%%%%%%%%%%%%%%%
  \usepackage{eurosym}
\usepackage{vmargin}
\usepackage{amsmath}
\usepackage{graphics}
\usepackage{epsfig}
\usepackage{enumerate}
\usepackage{multicol}
\usepackage{subfigure}
\usepackage{fancyhdr}
\usepackage{listings}
\usepackage{framed}
\usepackage{graphicx}
\usepackage{amsmath}
\usepackage{chngpage}
%\usepackage{bigints}
\usepackage{vmargin}

% left top textwidth textheight headheight

% headsep footheight footskip

\setmargins{2.0cm}{2.5cm}{16 cm}{22cm}{0.5cm}{0cm}{1cm}{1cm}

\renewcommand{\baselinestretch}{1.3}

\setcounter{MaxMatrixCols}{10}

\begin{document}

\begin{enumerate}

\item %%-Qeustion 1
The marks of a sample of 25 students from a large class in a recent test have sample mean 57.2 and standard deviation 7.3. The marks are subsequently adjusted: each mark is multiplied by 1.1 and the result is then increased by 8.
Calculate the sample mean and standard deviation of the adjusted marks.
2

\item In a survey, a sample of 10 policies is selected from the records of an insurance
company. The following data give, in ascending order, the time (in days) from the
start date of the policy until a claim has arisen from each of the policies in the sample.
\[297 301 312 317 355 379 404 419 432+ 463+\]
Some of the policies have not yet resulted in any claims at the time of the survey, so
the times until they each give rise to a claim are said to be censored. These values are
represented with a plus sign in the above data.
3
4
(i) Calculate the median of this sample.

(ii) State what you can conclude about the mean time until claims arise from the
policies in this sample.

%%%%%%%%%%%%%%%%%%%%%%5
\newpage
1 Sample mean = (1.1 × 57.2) + 8 = 70.92
Sample standard deviation = 1.1 × 7.3 = 8.03
2 (i)
Sample median is not affected by the fact that the last two observations are
censored.
It is therefore given by the 5.5 th ranked observation, i.e. (355 + 379) / 2 = 367
days.
(ii)
We know that the last two observations have minimum values 432 and 463.
Using these two values the sample mean would be equal to 3679/10 = 367.9.
So, the sample mean is at least equal to 367.9 days.
\newpage
%%%%%%%%%%%%%%%%%%%%%%%%%%%%%
3
(i)
(ii)
4
Using the negative binomial distribution, or from first principles,
⎛ 5 − 1 ⎞
2
3
P (5 policies required) = ⎜
⎟ (0.2) (0.8) = 0.0819
⎝ 2 − 1 ⎠
2
= 10
\end{document}
