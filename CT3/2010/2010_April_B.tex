\documentclass[a4paper,12pt]{article}

%%%%%%%%%%%%%%%%%%%%%%%%%%%%%%%%%%%%%%%%%%%%%%%%%%%%%%%%%%%%%%%%%%%%%%%%%%%%%%%%%%%%%%%%%%%%%%%%%%%%%%%%%%%%%%%%%%%%%%%%%%%%%%%%%%%%%%%%%%%%%%%%%%%%%%%%%%%%%%%%%%%%%%%%%%%%%%%%%%%%%%%%%%%%%%%%%%%%%%%%%%%%%%%%%%%%%%%%%%%%%%%%%%%%%%%%%%%%%%%%%%%%%%%%%%%%

\usepackage{eurosym}
\usepackage{vmargin}
\usepackage{amsmath}
\usepackage{graphics}
\usepackage{epsfig}
\usepackage{enumerate}
\usepackage{multicol}
\usepackage{subfigure}
\usepackage{fancyhdr}
\usepackage{listings}
\usepackage{framed}
\usepackage{graphicx}
\usepackage{amsmath}
\usepackage{chngpage}

%\usepackage{bigints}
\usepackage{vmargin}

% left top textwidth textheight headheight

% headsep footheight footskip

\setmargins{2.0cm}{2.5cm}{16 cm}{22cm}{0.5cm}{0cm}{1cm}{1cm}

\renewcommand{\baselinestretch}{1.3}

\setcounter{MaxMatrixCols}{10}

\begin{document}
\begin{enumerate}
%%%%%%% Question 4
4
[3]
It is assumed that the numbers of claims arising in one year from motor insurance policies for young male drivers and young female drivers are distributed as Poisson random variables with parameters λ m and λ f respectively.
Independent random samples of 120 policies for young male drivers and 80 policies for young female drivers were examined and yielded the following mean number of claims per policy in the last calendar year: x m = 0.24 and x f = 0.15 .
Calculate an approximate 95% confidence interval for λ m − λ f , the difference between
the respective Poisson parameters.
[3]
5
A computer routine selects one of the integers 1, 2, 3, 4, 5 at random and replicates the process a total of 100 times. Let S denote the sum of the 100 numbers selected.
Calculate the approximate probability that S assumes a value between 280 and 320
inclusive.
[5]

%%%%%%%%%%%%%%%%%%%%%%%%%%%%%%%%%%%%%%%%%%%%%%%%%%%%%%%%%%%%%%%%%%%%%%%%%%%%%
Let X 1 , X 2 , ... , X n be a random sample of claim amounts which are modelled using a
gamma distribution with known parameter α = 4 and unknown parameter λ .
n
\item 
(a)
Specify the distribution of
∑ X i .
i = 1
(b)
Justify the fact that 2n λ X has a χ 2 k distribution, where X is the mean of the sample, by using a suitable relationship between the gamma and the χ 2 distribution, and specify the degrees of freedom k.



%%%%%%%%%%%%%%%%%%%%%%%%%%%%%%%%%%%%%%%%%%%%%%%%%%%%%%%%%%%%%%%%%%%%%%%%%%%%%5
4
1
E ⎡ S 2 ⎤ =
{ Σ E ( X i 2 ) − nE ( X 2 )}
⎣ ⎦
n − 1
σ 2
1
=
{ Σ ( σ 2 + μ 2 ) − n ( + μ 2 )}
n − 1
n
1
{ n ( σ 2 + μ 2 ) − σ 2 − n μ 2 }
=
n − 1
1
{( n − 1) σ 2 } = σ 2
=
n − 1
Approximate 95% CI for λ m − λ f is ( x m − x f ) ± 1.96
⇒ (0.24 − 0.15) ± 1.96
x m x f
+
120 80
0.24 0.15
+
120
80
⇒ 0.09 ± 1.96(0.062) ⇒ 0.09 ± 0.122 or ⇒ ( − 0.032, 0.212)

%%%%%%%%%%%%%%%%%%%%%%%%%%%%%%%%%%%%%%%%%%%%%%%%%%%%%%%%%%%%%%%%%%%%%%%%%%%%%5
5
S = Σ X i where X i has a uniform distribution on 1, 2, 3, 4, 5, with mean 3 and variance
(25 – 1)/12 = 2 (result known, or calculated via E [ X 2 ] = 11, or from book of formulae,
p10, with a = 1, b = 5, h = 1).
So S ~ N (300, 200) approximately
320.5 − 300 ⎞
⎛ 279.5 − 300
P (280 ≤ S ≤ 320) = P ⎜
< Z <
⎟
200
200 ⎠
⎝
= P ( − 1.450 < Z < 1.450 ) = 0.853

%%%%%%%%%%%%%%%%%%%%%%%%%%%%%%%%%%%%%%%%%%%%%%%%%%%%%%%%%%%%%%%%%%%%%%%%%%%%%5
6
(i)
(a) Σ X i ~ gamma(4 n , λ)
(b) If Y ~ gamma(α, λ) and 2α is an integer, then 2 λ Y ~ χ 22 α (from book
of formulae, p12)
So 2 λ nX ~ χ 2 with df 8 n .
(ii)
P ( χ 2 40 (97.5) < 10 λ X < χ 2 40 (2.5)) = 0.95
⎛ χ 2 (97.5) χ 2 40 (2.5) ⎞
giving the 95% CI as ⎜ 40
,
⎟ ⎟
⎜ 10 X
10
X
⎝
⎠
⎛ 24.43
59.34 ⎞
Data ⇒ ⎜
,
⎟ = (0.140, 0.339)
⎝ 10(17.5) 10(17.5) ⎠

%%%%%%%%%%%%%%%%%%%%%%%%%%%%%%%%%%%%%%%%%%%%%%%%%%%%%%%%%%%%%%%%%%%%%%%%%%%%%5
\end{document}
