\documentclass[a4paper,12pt]{article}

%%%%%%%%%%%%%%%%%%%%%%%%%%%%%%%%%%%%%%%%%%%%%%%%%%%%%%%%%%%%%%%%%%%%%%%%%%%%%%%%%%%%%%%%%%%%%%%%%%%%%%%%%%%%%%%%%%%%%%%%%%%%%%%%%%%%%%%%%%%%%%%%%%%%%%%%%%%%%%%%%%%%%%%%%%%%%%%%%%%%%%%%%%%%%%%%%%%%%%%%%%%%%%%%%%%%%%%%%%%%%%%%%%%%%%%%%%%%%%%%%%%%%%%%%%%%
  \usepackage{eurosym}
\usepackage{vmargin}
\usepackage{amsmath}
\usepackage{graphics}
\usepackage{epsfig}
\usepackage{enumerate}
\usepackage{multicol}
\usepackage{subfigure}
\usepackage{fancyhdr}
\usepackage{listings}
\usepackage{framed}
\usepackage{graphicx}
\usepackage{amsmath}
\usepackage{chngpage}
%\usepackage{bigints}
\usepackage{vmargin}

% left top textwidth textheight headheight

% headsep footheight footskip

\setmargins{2.0cm}{2.5cm}{16 cm}{22cm}{0.5cm}{0cm}{1cm}{1cm}

\renewcommand{\baselinestretch}{1.3}

\setcounter{MaxMatrixCols}{10}

\begin{document}


A certain type of claim amount (in units of £1,000) is modelled as an exponential
random variable with parameter λ = 1.25. An analyst is interested in S, the total of 10
such independent claim amounts. In particular he wishes to calculate the probability
that S exceeds £10,000.
\begin{enumerate}
\item 
(i)
(a)
Show, using moment generating functions, that:
(1)
(2)
(b)
S has a gamma distribution, and
2.5S has a χ 220 distribution.
Use tables to calculate the required probability.
%%%%%%%%%%%%%%%%%%%%%%%%%%%%%%%%%%%%5
\item (ii)
(a) Specify an approximate normal distribution for S by applying the
central limit theorem, and use this to calculate an approximate value
for the required probability.
(b) Comment briefly on the use of this approximation and on the result.
\end{enumerate}
%%%%%%%%%%%%%%%%%%%%%%%%%%%%%%%%%%%%%%%%%%%%%%
\newpage
8
(i)
(a)
(1)
\begin{itemize}
\item Let X i be a claim amount.
t ⎞
⎛
Mgf of X i is $M_X ( t )$ = ⎜ 1 −
⎟
⎝ 1.25 ⎠
− 1
10
t ⎞
⎛
Mgf of S = ∑ X i is M S ( t ) = [ M X ( t )] 10 = ⎜ 1 −
⎟
⎝ 1.25 ⎠
i = 1
− 10
,
which is the mgf of a gamma(10, 1.25) variable.
(2)
\item Mgf of 2.5 S is \begin{eqnarray*}E [ e t (2.5 S ) ] &=& E [ e (2.5 t ) S ] \\ &=& M S (2.5 t ) \\ &=& (1 − 2 t ) − 10\\ 
\end{eqnarray*},
which is the mgf of a gamma(10, 1⁄2) variable, i.e. χ 220 .
\item (ii)
(b) P (total > £10, 000) = P ( S > 10) = P ( χ 220 > 25) = 1 − 0.7986 = 0.2014
(a) S has mean
10
10
= 8 and variance
= 6.4 . So S ≈ N (8, 6.4)
1.25
1.25 2
10 − 8
P ( S > 10) ≅ P ( Z >
= 0.791) = 1 − 0.786 = 0.214
6.4
\end{itemize}
\end{document}
