\documentclass[a4paper,12pt]{article}

%%%%%%%%%%%%%%%%%%%%%%%%%%%%%%%%%%%%%%%%%%%%%%%%%%%%%%%%%%%%%%%%%%%%%%%%%%%%%%%%%%%%%%%%%%%%%%%%%%%%%%%%%%%%%%%%%%%%%%%%%%%%%%%%%%%%%%%%%%%%%%%%%%%%%%%%%%%%%%%%%%%%%%%%%%%%%%%%%%%%%%%%%%%%%%%%%%%%%%%%%%%%%%%%%%%%%%%%%%%%%%%%%%%%%%%%%%%%%%%%%%%%%%%%%%%%
  \usepackage{eurosym}
\usepackage{vmargin}
\usepackage{amsmath}
\usepackage{graphics}
\usepackage{epsfig}
\usepackage{enumerate}
\usepackage{multicol}
\usepackage{subfigure}
\usepackage{fancyhdr}
\usepackage{listings}
\usepackage{framed}
\usepackage{graphicx}
\usepackage{amsmath}
\usepackage{chngpage}
%\usepackage{bigints}
\usepackage{vmargin}

% left top textwidth textheight headheight

% headsep footheight footskip

\setmargins{2.0cm}{2.5cm}{16 cm}{22cm}{0.5cm}{0cm}{1cm}{1cm}

\renewcommand{\baselinestretch}{1.3}

\setcounter{MaxMatrixCols}{10}

\begin{document}

\begin{enumerate}

[Total 4]
Suppose that in a group of insurance policies (which are independent as regards
occurrence of claims), 20% of the policies have incurred claims during the last year.
An auditor is examining the policies in the group one by one in random order until
two policies with claims are found.
(i) Determine the probability that exactly five policies have to be examined until
two policies with claims are found.

(ii) Find the expected number of policies that have to be examined until two
policies with claims are found.
[1]
[Total 3]
For a certain class of business, claim amounts are independent of one another and are
distributed about a mean of μ = £4,000 and with standard deviation \sigma = £500.
Calculate an approximate value for the probability that the sum of 100 such claim
amounts is less than £407,500.





5
A random sample of 200 travel insurance policies contains 29 on which the
policyholders made claims in their most recent year of cover.
Calculate a 99% confidence interval for the proportion of policyholders who make
claims in a given year of cover.

CT3 S2010—26
The random variable X has a Poisson distribution with mean Y, where Y itself is
considered to be a random variable. The distribution of Y is lognormal with
parameters μ and \sigma 2 .
Derive the unconditional mean E[X] and variance V[X] using appropriate conditional
moments. (You may use any standard results without proof, including results from
the book of Formulae and Tables.)



%%%%%%%%%%%%%%%%%%%%%%%5
Expected number = mean of negative binomial distribution =
0.2
Working in units of £1,000, sum of 100 claim amounts S has E [ S ] = 100×4 = 400 and
V [ S ] = 100 × 0.5 2 = 25, and so S ~ N (400, 5 2 ) approximately.
P ( S < 407.5) = P ( Z < 1.5) = 0.933
5
Sample proportion = 29/200 = 0.145
99% CI is given by 0.145 ± 2.5758
i.e. (0.081, 0.209).
Page 2
0.145 × 0.855
i.e. 0.145 ± 0.064
200Subject CT3 (Probability and Mathematical Statistics Core Technical) — %%%%%%%%%%%%%%%%%%%%%%%%%%%%%%%%%%%%5
6
E [ X ] = E [ E ( X | Y )] = E [ Y ] = e μ+\sigma
2
/2
V [ X ] = E [ V ( X | Y )] + V [ E ( X | Y )] = E [ Y ] + V [ Y ] = e μ+\sigma
