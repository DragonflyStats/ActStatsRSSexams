\documentclass[a4paper,12pt]{article}

%%%%%%%%%%%%%%%%%%%%%%%%%%%%%%%%%%%%%%%%%%%%%%%%%%%%%%%%%%%%%%%%%%%%%%%%%%%%%%%%%%%%%%%%%%%%%%%%%%%%%%%%%%%%%%%%%%%%%%%%%%%%%%%%%%%%%%%%%%%%%%%%%%%%%%%%%%%%%%%%%%%%%%%%%%%%%%%%%%%%%%%%%%%%%%%%%%%%%%%%%%%%%%%%%%%%%%%%%%%%%%%%%%%%%%%%%%%%%%%%%%%%%%%%%%%%

\usepackage{eurosym}
\usepackage{vmargin}
\usepackage{amsmath}
\usepackage{graphics}
\usepackage{epsfig}
\usepackage{enumerate}
\usepackage{multicol}
\usepackage{subfigure}
\usepackage{fancyhdr}
\usepackage{listings}
\usepackage{framed}
\usepackage{graphicx}
\usepackage{amsmath}
\usepackage{chngpage}

%\usepackage{bigints}
\usepackage{vmargin}

% left top textwidth textheight headheight

% headsep footheight footskip

\setmargins{2.0cm}{2.5cm}{16 cm}{22cm}{0.5cm}{0cm}{1cm}{1cm}

\renewcommand{\baselinestretch}{1.3}

\setcounter{MaxMatrixCols}{10}

\begin{document}
\begin{enumerate}
%%--- Question 9
\item The number of claims, N, arising over a period of five years for a particular policy is assumed to follow a “Type 2” negative binomial distribution (as in the book of
k (1 − p )
Formulae and Tables page 9) with mean E [ N ] =
and variance
\[p^k (1 − p )\]
.
\[V [ N ] =p^2\]
Each claim amount, X (in units of £1,000), is assumed to follow an exponential distribution with parameter $\lambda$ independently of each other claim amount and of the number of claims.

Let S be the total of the claim amounts for the period of five years, in the case
k = 2, p = 0.8 and $\lambda=2$.
\item 
Calculate the mean and the standard deviation of $S$ based on the above assumptions.

Now assume that:
\begin{itemize}
\item  N follows a Poisson distribution with parameter $\mu = 0.5$, that is, with the same mean as N above;
\item  X follows a gamma distribution with parameters $\alpha = 2$ and $\lambda = 4$, that is, with the same mean as $X$ above.
\end{itemize}

10
Calculate the mean and the standard deviation of S based on these
assumptions.
Compare the two sets of answers  above.
%%%%%%%%%%%%%%%%%%%%%%%%%%%%%%%%%%%%%%%%%%%%%%%%%%%%%%%%%%%%%%%%%%%%%%%%%%%%%%%%%%%%%%%%%%%%%


%%%%%%%%%%%%%%%%%%%%%%%%%%%%%%%%%%%%%%%%%%%%%%%%%%%%%%%%%%%%%%%%%%%%%%%%
\newpage

%%%%%%%%%%%%%%%%%%%%%%%%%%%%%%%%%%%%%%%%%%%%%%%%%%%%%%%%%%%%%%%%%%%%%%%%%%%%%%


9

\begin{itemize}
    \item 
(i)
E [ N ] = k (1 − p ) 2(0.2)
k (1 − p ) 2(0.2)
=
= 0.5 and V [ N ] =
=
= 0.625
p
0.8
p 2
0.8 2
E [ X ] = 1 1
1
1
= = 0.5 and V [ X ] = 2 = 2 = 0.25
λ 2
λ
2
%---------------------------------------%
\item E [ S ] = E [ N ] E [ X ] = 0.5 × 0.5 = 0.25 , i.e. £250
%---------------------------------------%
V [ S ] = E [ N ] V [ X ] + V [ N ] { E [ X ] } = 0.5 × 0.25 + 0.625 × 0.5 2 = 0.28125
%---------------------------------------%
2
∴ SD [ S ] = 0.530 , i.e. £530
%-- Page 4
%-- Subject CT3 (Probability and Mathematical Statistics Core Technical) — April 2010 — Examiners’ Report
\item (ii)
E [ N ] = V [ N ] = μ = 0.5
E [ X ] =
α 2
α
2
= = 0.5 and V [ X ] = 2 = 2 = 0.125
λ 4
4
λ
E [ S ] = E [ N ] E [ X ] = 0.5 × 0.5 = 0.25 , i.e. £250
V [ S ] = E [ N ] V [ X ] + V [ N ] { E [ X ] } = 0.5 × 0.125 + 0.5 × 0.5 2 = 0.1875
2
∴ SD [ S ] = 0.433 , i.e. £433

\end{itemize}



%%%%%%%%%%%%%%%%%%%%%%%%%%%%%%%%%%%%%%%%%%%%%%%%%%%%%%%%%%%%%%%%%%%%%%%%%
\end{document}
