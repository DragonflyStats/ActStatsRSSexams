\documentclass[a4paper,12pt]{article}

%%%%%%%%%%%%%%%%%%%%%%%%%%%%%%%%%%%%%%%%%%%%%%%%%%%%%%%%%%%%%%%%%%%%%%%%%%%%%%%%%%%%%%%%%%%%%%%%%%%%%%%%%%%%%%%%%%%%%%%%%%%%%%%%%%%%%%%%%%%%%%%%%%%%%%%%%%%%%%%%%%%%%%%%%%%%%%%%%%%%%%%%%%%%%%%%%%%%%%%%%%%%%%%%%%%%%%%%%%%%%%%%%%%%%%%%%%%%%%%%%%%%%%%%%%%%

\usepackage{eurosym}
\usepackage{vmargin}
\usepackage{amsmath}
\usepackage{graphics}
\usepackage{epsfig}
\usepackage{enumerate}
\usepackage{multicol}
\usepackage{subfigure}
\usepackage{fancyhdr}
\usepackage{listings}
\usepackage{framed}
\usepackage{graphicx}
\usepackage{amsmath}
\usepackage{chngpage}

%\usepackage{bigints}
\usepackage{vmargin}

% left top textwidth textheight headheight

% headsep footheight footskip

\setmargins{2.0cm}{2.5cm}{16 cm}{22cm}{0.5cm}{0cm}{1cm}{1cm}

\renewcommand{\baselinestretch}{1.3}

\setcounter{MaxMatrixCols}{10}

\begin{document}
\begin{enumerate}
\item An employment survey is carried out in order to determine the percentage, $p$, of
unemployed people in a certain population in a way such that the estimation has a margin of error less than 0.5\% with probability at least 0.95. In a similar study conducted a year ago it was found that the percentage of unemployed people in the
population was 6\%. Calculate the sample size, n, that is required to achieve this margin of error, by
constructing an appropriate confidence interval (or otherwise).
8



\end{enumerate}
%%%%%%%%%%%%%%%%%%%%%%%%%%%%%%%%%%%%%%%%%%%%%%%%%%%%%%%%%%%%%%%%%%%%%%%%%%
\newpage

%%--- Question 7
\begin{itemize}
\item The 95\% CI for the population percentage p is
giving $| p - \hat{p} | \leq 1.96$
\[ \hat{p} \pm 1.96
\hat{p} (1 - \hat{p} )
n
\hat{p} (1 - \hat{p} )
n\]
\item For the margin of error to be less than 0.5\% we need to solve
\[0.005 = 1.96
1.96 2 \hat{p} (1 - \hat{p} )\]
$\hat{p} (1 - \hat{p} )$
⇒ n =
.
n
0.005 2
\item Using the percentage from the previous study as the value for $\hat{p}$, i.e. $\hat{p} = 0.06$ , we obtain $n = 8,666.6$.
\item So we need a sample of (at least) 8667 people.
⎛ p ( 1 - p ) ⎞
\item (OR, solution can be based on \hat{p} ~ N ⎜ p ,
⎟ and
n
⎝
⎠
\end{itemize}
\[P ( - 0.005 < \hat{p} - p < 0.005 ) > 0.95 \]without referring to the CI.)


%%%%%%%%%%%%%%%%%%%%%%%%%%%%%%%%%%%%%%%%%%%%%%%%%%%%%%%%%%%%%%%%%%%%%%%%%%%%%%%%%%%%%%

%%%%%%%%%%%%%%%%%%%%%%%%%%%%%%%%%%%%%%%%%%%%%%%%%%%%%%%%%%%%%%%%%%%%%%%%%%
\end{document}
