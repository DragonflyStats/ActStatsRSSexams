\documentclass[a4paper,12pt]{article}

%%%%%%%%%%%%%%%%%%%%%%%%%%%%%%%%%%%%%%%%%%%%%%%%%%%%%%%%%%%%%%%%%%%%%%%%%%%%%%%%%%%%%%%%%%%%%%%%%%%%%%%%%%%%%%%%%%%%%%%%%%%%%%%%%%%%%%%%%%%%%%%%%%%%%%%%%%%%%%%%%%%%%%%%%%%%%%%%%%%%%%%%%%%%%%%%%%%%%%%%%%%%%%%%%%%%%%%%%%%%%%%%%%%%%%%%%%%%%%%%%%%%%%%%%%%%

\usepackage{eurosym}
\usepackage{vmargin}
\usepackage{amsmath}
\usepackage{graphics}
\usepackage{epsfig}
\usepackage{enumerate}
\usepackage{multicol}
\usepackage{subfigure}
\usepackage{fancyhdr}
\usepackage{listings}
\usepackage{framed}
\usepackage{graphicx}
\usepackage{amsmath}
\usepackage{chngpage}

%\usepackage{bigints}
\usepackage{vmargin}

% left top textwidth textheight headheight

% headsep footheight footskip

\setmargins{2.0cm}{2.5cm}{16 cm}{22cm}{0.5cm}{0cm}{1cm}{1cm}

\renewcommand{\baselinestretch}{1.3}

\setcounter{MaxMatrixCols}{10}

\begin{document}
\begin{enumerate}
\item %11
%%%%%%%%%%%%%%%%%%%%%%%%%%%%%%%%%%%%%%%%%%
Consider the following three independent random samples from a normally distributed population with unknown mean $\mu$:
\\
\noindent \textbf{Sample 1:}
19.9 20.4 20.3 22.3 16.7 18.7 20.5 19.0 20.1 16.4 21.5 21.4 17.8 22.5 15.2
For these data: n = 15,
∑ x i = 292.7,
∑ x i 2 = 5, 778.69
\\
\noindent \textbf{Sample 2:}
20.8 25.9 22.1 21.7 16.0 12.1 27.6 16.1 16.8 17.1 21.3 18.6 24.9 14.8 22.2
For these data: n = 15,
∑ x i = 298.0,
sample mean = 19.867,
∑ x i 2 = 6,192.32
sample variance = 19.432
\\
\noindent \textbf{Sample 3:}
20.6 18.5 21.5 16.9 21.5 21.2 20.9 22.4 14.5 22.0 20.2 17.0 20.3 23.0 19.3
18.9 20.6 20.9 15.3 21.5 16.8 18.5 21.6 16.8 20.4
For these data: n = 25, ∑ x i = 491.1,
sample mean = 19.644,
∑ x i 2 = 9, 773.77
sample variance = 5.275
Consider t-tests of the hypotheses H 0 : μ = 18 v H 1 : μ > 18.
\item 

(a) Calculate the sample mean and variance for Sample 1.
(b) Carry out a t-test of the stated hypotheses using the Sample 1 data (stating the approximate P–value) and show that H 0 can be rejected at the 1\% level of testing.
\item 
(a) Carry out a t-test of the stated hypotheses using the Sample 2 data (stating the approximate P–value and the conclusion clearly).
(b) Discuss the comparison of the results with those based on Sample 1 (include reasons for any difference or similarity in the test
conclusions).
[6]
\item 
(a) Carry out a t-test of the stated hypotheses using the Sample 3 data (stating the approximate P–value and your conclusion clearly).
(b) Discuss the comparison of the results with those based on Sample 1 (include reasons for any difference or similarity in the test
conclusions).

%%%%%%%%%%%%%%%%%%%%%%%%%%%%%%%%%%%%%%%%%%%%%%%%%%%%%%%%%%%%%%%%%%%%%%%%
\newpage

11
(i)
(a) x = 19.513, s 2 =
(b) Test statistic is
1 ⎛
292.7 2 ⎞
⎜ ⎜ 5778.69 −
⎟ = 4.7955
14 ⎝
15 ⎟ ⎠
X −μ
S 2 / n
~ t n − 1
\begin{itemize}
\item Here t = (19.513 – 18)/(4.7955/15) 1/2 = 2.68
\item P-value = P(t 14 > 2.68), which is just less than 0.01 (1\%)
\item We reject H 0 and accept “$\mu > 18$” at the 1\% level of testing.
\end{itemize}
(ii)
(a)
Here t = (19.867 – 18)/(19.432/15) 1/2 = 1.64
P-value = P(t 14 > 1.64), which is between 0.05 and 0.1.
P-value exceeds 5\% and so we cannot reject H 0 , so “$\mu = 18$” can stand.
(b)
Sample 2 does not provide enough evidence to justify rejecting H 0 , despite having the same size and a similar mean to Sample 1.
The reason for the loss of significance is the much greater variation in the data in Sample 2 – the variance is four times bigger than in Sample 1 ($19.432 v 4.7955$)

– this greatly increases the standard error of estimation and reduces the value of the t-statistic (1.64 v 2.68).
(iii)
(a)
Here t = (19.644 – 18)/(5.275/25) 1/2 = 3.58
P-value = P(t 24 > 3.58), which is less than 0.001 (0.1\%)
We reject H 0 and accept “μ > 18” at a level lower than 0.1\%.
(b)
Sample 3 provides even stronger evidence against $H_0$ , despite having a similar mean and variance to Sample 1.
The main reason for the much greater level of significance is the increased sample size (25 v 15)
– this decreases the standard error of estimation and increases the value of the t-statistic considerably (3.58 v 2.68).
\end{document}
