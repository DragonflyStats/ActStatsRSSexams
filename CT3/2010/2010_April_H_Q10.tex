\documentclass[a4paper,12pt]{article}

%%%%%%%%%%%%%%%%%%%%%%%%%%%%%%%%%%%%%%%%%%%%%%%%%%%%%%%%%%%%%%%%%%%%%%%%%%%%%%%%%%%%%%%%%%%%%%%%%%%%%%%%%%%%%%%%%%%%%%%%%%%%%%%%%%%%%%%%%%%%%%%%%%%%%%%%%%%%%%%%%%%%%%%%%%%%%%%%%%%%%%%%%%%%%%%%%%%%%%%%%%%%%%%%%%%%%%%%%%%%%%%%%%%%%%%%%%%%%%%%%%%%%%%%%%%%

\usepackage{eurosym}
\usepackage{vmargin}
\usepackage{amsmath}
\usepackage{graphics}
\usepackage{epsfig}
\usepackage{enumerate}
\usepackage{multicol}
\usepackage{subfigure}
\usepackage{fancyhdr}
\usepackage{listings}
\usepackage{framed}
\usepackage{graphicx}
\usepackage{amsmath}
\usepackage{chngpage}

%\usepackage{bigints}
\usepackage{vmargin}

% left top textwidth textheight headheight

% headsep footheight footskip

\setmargins{2.0cm}{2.5cm}{16 cm}{22cm}{0.5cm}{0cm}{1cm}{1cm}

\renewcommand{\baselinestretch}{1.3}

\setcounter{MaxMatrixCols}{10}

\begin{document}

10
Calculate the mean and the standard deviation of S based on these
assumptions.
Compare the two sets of answers  above.
%%%%%%%%%%%%%%%%%%%%%%%%%%%%%%%%%%%%%%%%%%%%%%%%%%%%%%%%%%%%%%%%%%%%%%%%%%%%%%%%%%%%%%%%%%%%%
\begin{itemize}
    \item  The size of claims (in units of £1,000) arising from a portfolio of house contents
insurance policies can be modelled using a random variable $X$ with probability density
function (pdf) given by:
ac a
f X ( x ) = a + 1 , x ≥ c
x
where $a > 0$ and $c > 0$ are the parameters of the distribution.
ac
, for a > 1 .
a − 1
\item  Show that the expected value of X is E [ X ] =
\item  Verify that the cumulative distribution function of X is given by
[2]
a
⎛ c ⎞
F X ( x ) = 1 − ⎜ ⎟ ,
⎝ x ⎠
%%--- CT3 A2010—4
x ≥ c
(and = 0 for x < c).
\item Suppose that for the distribution of claim sizes X it is known that $c = 2.5$, but a is
unknown and needs to be estimated given a random sample x 1 , x 2 , ..., x n .
\item 
(iv)
Show that the maximum likelihood estimate (MLE) of a is given by:
n
a ˆ = n
.
⎛ x i ⎞
∑ log ⎜ ⎝ 2.5 ⎟ ⎠
i = 1
\item Derive the asymptotic variance of the MLE â , and hence determine its
approximate asymptotic distribution.


\item Consider a sample of 30 observations from this distribution, for which:
30
∑ log( x i ) = 32.9 .
i = 1
(v)
\item Calculate the MLE â in this case, together with an approximate 95%
confidence interval for a.

In the current year, claim sizes are assumed to follow the distribution of X with a = 6,
c = 2.5. Inflation for the following year is expected to be 5%.
(vi)
\item Calculate the probability that the size of a claim arising from this portfolio in
the following year will exceed £4,000.
\end{enumerate}
%%%%%%%%%%%%%%%%%%%%%%%%%%%%%%%%%%%%%%%%%%%%%%%%%%%%%%%%%%%%%%%%%%%%%%%%%%%%%%



%%%%%%%%%%%%%%%%%%%%%%%%%%%%%%%%%%%%%%%%%%%%%%%%%%%%%%%%%%%%%%%%%%%%%%%%
\newpage

10
\begin{itemize}
\item (i) We have:
∞
∞
E [ X ] = ∫ xf X ( x ) dx = ∫ x
c
c
ac a
x
dx = ac
a + 1
∞
a
− a
∫ x dx = −
c
ac a ⎡ − ( a − 1) ⎤ ∞
x
⎦ c
a − 1 ⎣
and for a > 1
E [ X ] = −
\ite, (ii)
ac a
ac
.
(0 − c − a + 1 ) =
a − 1
a − 1
x x
c t a + 1
c
F X ( x ) = ∫ f X ( t ) dt = ∫
ac a
dt
which gives
a
x
⎛ c ⎞
F X ( x ) = − c ⎡ t − a ⎤ = − c a ( x − a − c − a ) = 1 − ⎜ ⎟ ,
⎣ ⎦ c
⎝ x ⎠
a
x ≥ c
[OR differentiate F X ( x ) to obtain f X ( x ) ]
Page 5Subject CT3 (Probability and Mathematical Statistics Core Technical) — April 2010 — Examiners’ Report
\item (iii)
The likelihood function is given by:
n n ac a
i = 1 i = 1 x i a + 1
L ( a ) = ∏ f X ( x i ) = ∏
n
= a n c na ∏ x i − ( a + 1)
i = 1
and
n
l ( a ) = n log( a ) + na log( c ) − ( a + 1) ∑ log( x i )
i = 1
For the MLE:
l ′ ( a ) = 0 ⇒
⇒ a ˆ =
n
n
+ n log( c ) − ∑ log( x i ) = 0
a
i = 1
n
=
n
∑ log( x i ) − n log( c )
i = 1
and for c = 2.5, a ˆ =
n
n
⎛ x ⎞
∑ log ⎜ ⎝ c i ⎟ ⎠
i = 1
,
n
n
⎛ x ⎞
∑ log ⎜ ⎝ 2.5 i ⎟ ⎠
i = 1
\item (iv)
For the asymptotic variance we use the Cramer-Rao lower bound:
l ′′ ( a ) = −
n
a
2
, and E ⎡ ⎣ l ′′ ( a ) ⎤ ⎦ = −
n
a 2
giving
{
}
V [ a ˆ ] = − E ⎡ ⎣ l ′′ ( a ) ⎤ ⎦
− 1
=
a 2
.
n
Hence, asymptotically, a ˆ ~ N ( a , a 2 n ) .
\item (v)
MLE is
a ˆ =
n
n
n
n
∑ log ⎜ ⎝ c i ⎟ ⎠ ∑ log( x i ) − n log( c )
i = 1
Page 6
⎛ x ⎞
=
i = 1
=
30
= 5.544 .
32.9 − 30 × log(2.5)
%--------------------------%
\item Using the asymptotic normal distribution given above, an approximate 95% CI
is given by
a ˆ \pm 1.96
a 2
a ˆ
= a ˆ \pm 1.96
n
n
i.e. 5.544 \pm 1.96
\item (vi)
5.544
, giving $(3.560, 7.528)$.
30
\item Size of claim in the following year will be given by 1.05X
4 ⎞
⎛
⎛ 4 ⎞
So we want $P (1.05 X > 4)$ = P ⎜ X >
⎟ = 1 − F X ⎜
⎟
1.05 ⎠
⎝
⎝ 1.05 ⎠
and using F X given in the question
6
⎛ 1.05 × 2.5 ⎞
P (1.05 X > 4) = ⎜
⎟ = 0.0799 .
4
⎝
⎠
\end{itemize}
%%%%%%%%%%%%%%%%%%%%%%%%%%%%%%%%%%%%%%%%%%%%%%%%%%%%%%%%%%%%%%%%%%%%%%%%%
\end{document}
