\documentclass[a4paper,12pt]{article}

%%%%%%%%%%%%%%%%%%%%%%%%%%%%%%%%%%%%%%%%%%%%%%%%%%%%%%%%%%%%%%%%%%%%%%%%%%%%%%%%%%%%%%%%%%%%%%%%%%%%%%%%%%%%%%%%%%%%%%%%%%%%%%%%%%%%%%%%%%%%%%%%%%%%%%%%%%%%%%%%%%%%%%%%%%%%%%%%%%%%%%%%%%%%%%%%%%%%%%%%%%%%%%%%%%%%%%%%%%%%%%%%%%%%%%%%%%%%%%%%%%%%%%%%%%%%
  \usepackage{eurosym}
\usepackage{vmargin}
\usepackage{amsmath}
\usepackage{graphics}
\usepackage{epsfig}
\usepackage{enumerate}
\usepackage{multicol}
\usepackage{subfigure}
\usepackage{fancyhdr}
\usepackage{listings}
\usepackage{framed}
\usepackage{graphicx}
\usepackage{amsmath}
\usepackage{chngpage}
%\usepackage{bigints}
\usepackage{vmargin}

% left top textwidth textheight headheight

% headsep footheight footskip

\setmargins{2.0cm}{2.5cm}{16 cm}{22cm}{0.5cm}{0cm}{1cm}{1cm}

\renewcommand{\baselinestretch}{1.3}

\setcounter{MaxMatrixCols}{10}

\begin{document}

\begin{itemize}
\item In the collection of questionnaire data, randomised response sampling is a method
which is used to obtain answers to sensitive questions. For example a company is
interested in estimating the proportion, p, of its employees who falsely take days off
sick. Employees are unlikely to answer a direct question truthfully and so the
company uses the following approach.
\item Each employee selected in the survey is given a fair six-sided die and asked to throw
it. If it comes up as a 5 or 6, then the employee answers yes or no to the question
“have you falsely taken any days off sick during the last year?”. If it comes up as a 1,
2, 3 or 4, then the employee is instructed to toss a coin and answer yes or no to the
question “did you obtain heads?”. So an individual’s answer is either yes or no, but it
is not known which question the individual has answered.
\item 
For the purpose of the following analysis you should assume that each employee
answers the question truthfully.
\item 
(i)
Show that the probability that an individual answers yes is
1
( p + 1) .
3
\item Suppose that 100 employees are surveyed and that this results in 56 yes answers.
CT3 S2010—4
(ii)
\item (a)
Show that the likelihood function L(p) can be expressed in the form:
L ( p ) ∝ ( p + 1) 56 (2 − p ) 44 .
(b)
\item Hence show that the maximum likelihood estimate (MLE) of p is
\hat{p}= 0.68 .

\end{itemize}

%%%%%%%%%%%%%%%%%%%%%%%%%%%%%%%%%%%%5
1
56
Let θ = ( p + 1) and note that, using binomial results, the MLE of θ is θ ˆ =
.
3
100
(iii)
Explain why p̂ can be obtained as the solution of
verify that \hat{p}= 0.68 .
(iv)
1
( \hat{p}+ 1) = θ ˆ , and hence
3

(a) Determine the second derivative of the log likelihood for p and
evaluate it at \hat{p}= 0.68 .
(b) State an approximate large-sample distribution for the MLE p̂ .
(c) Hence calculate approximate 95\% confidence limits for p.
%%%%%%%%%%%%%%%%%%%%%%%%%%%%%%%%%%%%5
Now suppose that the same numerical estimate, that is \hat{p}= 0.68 , had been obtained from a sample of the same size, that is 100, without using the randomised response method but relying on truthful answers. So the number of yes answers was 68 and
68
using binomial results.
\hat{p}=
100
(v)
(a) Calculate approximate 95\% confidence limits for p for this situation.
(b) Suggest why the confidence limits in part (iv)(c) are wider than these
limits.



%%---- Page 4  — %%%%%%%%%%%%%%%%%%%%%%%%%%%%%%%%%%%%5
\begin{itemize}
\item 10
(i)
P (yes) = P (5, 6) P (yes | main question) + P (1, 2,3, 4) P (yes | coin question)
=
\item (ii)
(a)
(b)
2
4 1 1
p + ⋅ = ( p + 1)
6
6 2 3
1
1
L ( p ) ∝ [ ( p + 1)] 56 [1 − ( p + 1)] 100 − 56
3
3
56
44
∝ ( p + 1) (2 − p )
\item log L = 56 log( p + 1) + 44 log(2 − p ) + const .
d
56
44
log L =
−
dp
p + 1 2 − p
=
56(2 − p ) − 44( p + 1)
68 − 100 p
=
( p + 1)(2 − p )
( p + 1)(2 − p )
Equate to zero => 68 − 100 p = 0 ∴ \hat{p}=
\item (iii)
68
= 0.68
100
1
( \hat{p}+ 1) = θ ˆ
3
Due to the invariance property of MLEs
1
56
168
68
∴ ( \hat{p}+ 1) =
∴ \hat{p}=
− 1 =
= 0.68
3
100
100
100
(iv)
(a)
d 2
dp
2
log L = −
at \hat{p}= 0.68 ,
(v)
56
( p + 1)
d 2
dp
2
2
−
44
(2 − p ) 2
log L = −
56
1.68
2
−
− 1
= 0.02218 and
− 45.0938
44
1.32 2
(b) CRlb =
(c) Approximate 95\% CI for p is
giving 0.68 \pm 0.292
(a) Approximate 95\% CI for p is \hat{p}\pm 1.96
= − 45.0938
\hat{p}≈ N ( p , 0.02218)
\hat{p}\pm 1.96 0.02218
\hat{p}(1 − \hat{p})
100
giving
0.68 \pm 1.96
0.68(1 − 0.68)
100
⇒ 0.68 \pm 1.96(0.0466) ⇒ 0.68 \pm 0.091
Page 5  — %%%%%%%%%%%%%%%%%%%%%%%%%%%%%%%%%%%%5
(b)

Less accurate estimation is the penalty paid for using the randomised
response method.
\end{document}
