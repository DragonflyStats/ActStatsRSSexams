\documentclass[a4paper,12pt]{article}

%%%%%%%%%%%%%%%%%%%%%%%%%%%%%%%%%%%%%%%%%%%%%%%%%%%%%%%%%%%%%%%%%%%%%%%%%%%%%%%%%%%%%%%%%%%%%%%%%%%%%%%%%%%%%%%%%%%%%%%%%%%%%%%%%%%%%%%%%%%%%%%%%%%%%%%%%%%%%%%%%%%%%%%%%%%%%%%%%%%%%%%%%%%%%%%%%%%%%%%%%%%%%%%%%%%%%%%%%%%%%%%%%%%%%%%%%%%%%%%%%%%%%%%%%%%%
  \usepackage{eurosym}
\usepackage{vmargin}
\usepackage{amsmath}
\usepackage{graphics}
\usepackage{epsfig}
\usepackage{enumerate}
\usepackage{multicol}
\usepackage{subfigure}
\usepackage{fancyhdr}
\usepackage{listings}
\usepackage{framed}
\usepackage{graphicx}
\usepackage{amsmath}
\usepackage{chngpage}
%\usepackage{bigints}
\usepackage{vmargin}

% left top textwidth textheight headheight

% headsep footheight footskip

\setmargins{2.0cm}{2.5cm}{16 cm}{22cm}{0.5cm}{0cm}{1cm}{1cm}

\renewcommand{\baselinestretch}{1.3}

\setcounter{MaxMatrixCols}{10}

\begin{document}

\begin{enumerate}
7
Let X be a discrete random variable with the following probability distribution:
X:
P(X = x):
(i)
0
0.4
1
0.3
2
0.2
3
0.1
Simulate three observations of X using the following three random numbers
from a uniform distribution on (0,1) (you should explain your method briefly
and clearly).
Random numbers: 0.4936, 0.7269, 0.1652

Let X be a random variable with cumulative distribution function:
F X ( x ) =
(ii)
8
1
1 − e − 1
( 1 − e ) ,
− x 2
0 < x < 1 (F X (x) = 0 for x ≤ 0 and F X (x) = 1 for x \geq 1).
Derive an expression for a simulated value of X using a random number u
from a uniform distribution on (0,1) and hence simulate an observation of X
using the random number u = 0.8149.

[Total 6]
A certain type of claim amount (in units of £1,000) is modelled as an exponential
random variable with parameter λ = 1.25. An analyst is interested in S, the total of 10
such independent claim amounts. In particular he wishes to calculate the probability
that S exceeds £10,000.
(i)
(a)
Show, using moment generating functions, that:
(1)
(2)
(b)
S has a gamma distribution, and
2.5S has a χ 220 distribution.
Use tables to calculate the required probability.
%%%%%%%%%%%%%%%%%%%%%%%%%%%%%%%%%%%%5
(ii)
(a) Specify an approximate normal distribution for S by applying the
central limit theorem, and use this to calculate an approximate value
for the required probability.
(b) Comment briefly on the use of this approximation and on the result.

[Total 8]

%%%%%%%%%%%%%%%%%%%%%%%%%%%%%%%%%%%%%%%%%%%%%%
7
(i)
2
/2
2
2
+ e 2 μ+\sigma ( e \sigma − 1)
Method
0 < u ≤ 0.4
0.4 < u ≤ 0.7
0.7 < u ≤ 0.9
0.9 < u ≤ 1
⇒ x = 0
⇒ x = 1
⇒ x = 2
⇒ x = 3
We get x = 1, 2, 0
(ii)
Setting u =
1
1 − e
− 1
( 1 − e ) ⇒ e
− x 2
(
− x 2
)
(
)
= 1 − 1 − e − 1 u
1/2
⇒ x = ⎡ − log ⎡ 1 − 1 − e − 1 u ⎤ ⎤
⎢ ⎣
⎣
⎦ ⎥ ⎦
u = 0.8149 ⇒ x = 0.851
8
(i)
(a)
(1)
Let X i be a claim amount.
t ⎞
⎛
Mgf of X i is M X ( t ) = ⎜ 1 −
⎟
⎝ 1.25 ⎠
− 1
10
t ⎞
⎛
Mgf of S = ∑ X i is M S ( t ) = [ M X ( t )] 10 = ⎜ 1 −
⎟
⎝ 1.25 ⎠
i = 1
− 10
,
which is the mgf of a gamma(10, 1.25) variable.
(2)
Mgf of 2.5 S is E [ e t (2.5 S ) ] = E [ e (2.5 t ) S ] = M S (2.5 t ) = (1 − 2 t ) − 10 ,
which is the mgf of a gamma(10, 1⁄2) variable, i.e. χ 220 .
(ii)
(b) P (total > £10, 000) = P ( S > 10) = P ( χ 220 > 25) = 1 − 0.7986 = 0.2014
(a) S has mean
10
10
= 8 and variance
= 6.4 . So S ≈ N (8, 6.4)
1.25
1.25 2
10 − 8
P ( S > 10) ≅ P ( Z >
= 0.791) = 1 − 0.786 = 0.214
6.4
