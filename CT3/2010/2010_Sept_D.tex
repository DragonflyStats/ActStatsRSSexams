\documentclass[a4paper,12pt]{article}

%%%%%%%%%%%%%%%%%%%%%%%%%%%%%%%%%%%%%%%%%%%%%%%%%%%%%%%%%%%%%%%%%%%%%%%%%%%%%%%%%%%%%%%%%%%%%%%%%%%%%%%%%%%%%%%%%%%%%%%%%%%%%%%%%%%%%%%%%%%%%%%%%%%%%%%%%%%%%%%%%%%%%%%%%%%%%%%%%%%%%%%%%%%%%%%%%%%%%%%%%%%%%%%%%%%%%%%%%%%%%%%%%%%%%%%%%%%%%%%%%%%%%%%%%%%%
  \usepackage{eurosym}
\usepackage{vmargin}
\usepackage{amsmath}
\usepackage{graphics}
\usepackage{epsfig}
\usepackage{enumerate}
\usepackage{multicol}
\usepackage{subfigure}
\usepackage{fancyhdr}
\usepackage{listings}
\usepackage{framed}
\usepackage{graphicx}
\usepackage{amsmath}
\usepackage{chngpage}
%\usepackage{bigints}
\usepackage{vmargin}

% left top textwidth textheight headheight

% headsep footheight footskip

\setmargins{2.0cm}{2.5cm}{16 cm}{22cm}{0.5cm}{0cm}{1cm}{1cm}

\renewcommand{\baselinestretch}{1.3}

\setcounter{MaxMatrixCols}{10}

\begin{document}

\begin{enumerate}
CT3 S2010—3
PLEASE TURN OVER9
Let the random variable X have the Poisson distribution with probability function:
f ( x ) =
(i)
e −λ λ x
,
x !
x = 0,1, 2,...
Show that P ( X = k + 1) =
λ
P ( X = k ), k = 0,1, 2,...
k + 1

It is believed that the distribution of the number of claims which arise on insurance
policies of a certain class is Poisson. A random sample of 1,000 policies is taken
from all the policies in this class which have been in force throughout the past year.
The table below gives the number of claims per policy in this sample.
No. of claims, k: 0
1
2
3 4 5 6 7 8 or more
No. of policies, f k : 310 365 202 88 26 6 2 1
0
For these data the maximum likelihood estimate (MLE) of the Poisson parameter λ is
λ ˆ = 1.186.
10
(ii) Calculate the frequencies expected under the Poisson model with parameter
given by the MLE above, using the recurrence formula of part (i) (or
otherwise).

(iii) Perform an appropriate statistical test to investigate the assumption that the
numbers of claims arising from this particular class of policies follow a
Poisson distribution.
%%%%%%%%%%%%%%%%%%%%%%%%%%%%%%%%%%%%5
[Total 10]
In the collection of questionnaire data, randomised response sampling is a method
which is used to obtain answers to sensitive questions. For example a company is
interested in estimating the proportion, p, of its employees who falsely take days off
sick. Employees are unlikely to answer a direct question truthfully and so the
company uses the following approach.
Each employee selected in the survey is given a fair six-sided die and asked to throw
it. If it comes up as a 5 or 6, then the employee answers yes or no to the question
“have you falsely taken any days off sick during the last year?”. If it comes up as a 1,
2, 3 or 4, then the employee is instructed to toss a coin and answer yes or no to the
question “did you obtain heads?”. So an individual’s answer is either yes or no, but it
is not known which question the individual has answered.
For the purpose of the following analysis you should assume that each employee
answers the question truthfully.
(i)
Show that the probability that an individual answers yes is
1
( p + 1) .
3
Suppose that 100 employees are surveyed and that this results in 56 yes answers.
CT3 S2010—4
(ii)
(a)
Show that the likelihood function L(p) can be expressed in the form:
L ( p ) ∝ ( p + 1) 56 (2 − p ) 44 .
(b)
Hence show that the maximum likelihood estimate (MLE) of p is
p ˆ = 0.68 .
%%%%%%%%%%%%%%%%%%%%%%%%%%%%%%%%%%%%5
1
56
Let θ = ( p + 1) and note that, using binomial results, the MLE of θ is θ ˆ =
.
3
100
(iii)
Explain why p̂ can be obtained as the solution of
verify that p ˆ = 0.68 .
(iv)
1
( p ˆ + 1) = θ ˆ , and hence
3

(a) Determine the second derivative of the log likelihood for p and
evaluate it at p ˆ = 0.68 .
(b) State an approximate large-sample distribution for the MLE p̂ .
(c) Hence calculate approximate 95% confidence limits for p.
%%%%%%%%%%%%%%%%%%%%%%%%%%%%%%%%%%%%5
Now suppose that the same numerical estimate, that is p ˆ = 0.68 , had been obtained
from a sample of the same size, that is 100, without using the randomised response
method but relying on truthful answers. So the number of yes answers was 68 and
68
using binomial results.
p ˆ =
100
(v)
(a) Calculate approximate 95% confidence limits for p for this situation.
(b) Suggest why the confidence limits in part (iv)(c) are wider than these
limits.

[Total 17]
CT3 S2010—5
%%%%%%%%%%%%%%%
Page 3Subject CT3 (Probability and Mathematical Statistics Core Technical) — %%%%%%%%%%%%%%%%%%%%%%%%%%%%%%%%%%%%5
(b)
9
(i)
(ii)
n is not particularly large for the use of the CLT, but the approximation
is still quite close to the true probability.
P ( X = k + 1) = e −λ
λ k + 1
λ k λ
λ
= e −λ
=
P ( X = k ) .
k ! k + 1 k + 1
( k + 1 ) !
Using P ( X = 0) = e
− 1.186
7
, P ( X = 8 or more) = 1 − ∑ P ( X = i ) , and the
i = 0
recurrent formula, we obtain:
K
P ( X = k )
Expected, e k
(iii)
0
1
2
3
4
5
6
7
0.3054 0.3623 0.2148 0.0849 0.0252 0.0060 0.0012 0.0002
305.4 362.3 214.8
84.9
25.2
6.0
1.2
0.2
8 or more
4 × 10 −5
0.0
Combining the last 4 categories to obtain expected frequencies greater than 5,
we have:
k
No. of policies, f k
Expected, e k
0
310
305.4
1
365
362.3
2
202
214.8
3
88
84.9
4
26
25.2
5 or more
9
7.4
This gives
χ = ∑
2
( f k − e k ) 2
e k
= 0.0693 + 0.0201 + 0.7628 + 0.1132 + 0.0254 + 0.3459 = 1.3367
2
DF = 6 − 1 − 1 = 4, and from statistical tables, χ 0.05,4
= 9.488 .
Therefore, we do not have evidence against the hypothesis that the number of
claims comes from a Poisson(1.186) distribution.
(Alternatively if we only combine the last 3 categories, the expected
frequencies for 5 and 6 or more policies are 6 and 1.4, with observed
frequencies 6 and 3 respectively. These give χ 2 = 2.819 on 5 DF, and with
2
χ 0.05,5
= 11.071 the conclusion is the same as before.)
Page 4Subject CT3 (Probability and Mathematical Statistics Core Technical) — %%%%%%%%%%%%%%%%%%%%%%%%%%%%%%%%%%%%5
10
(i)
P (yes) = P (5, 6) P (yes | main question) + P (1, 2,3, 4) P (yes | coin question)
=
(ii)
(a)
(b)
2
4 1 1
p + ⋅ = ( p + 1)
6
6 2 3
1
1
L ( p ) ∝ [ ( p + 1)] 56 [1 − ( p + 1)] 100 − 56
3
3
56
44
∝ ( p + 1) (2 − p )
log L = 56 log( p + 1) + 44 log(2 − p ) + const .
d
56
44
log L =
−
dp
p + 1 2 − p
=
56(2 − p ) − 44( p + 1)
68 − 100 p
=
( p + 1)(2 − p )
( p + 1)(2 − p )
Equate to zero => 68 − 100 p = 0 ∴ p ˆ =
(iii)
68
= 0.68
100
1
( p ˆ + 1) = θ ˆ
3
Due to the invariance property of MLEs
1
56
168
68
∴ ( p ˆ + 1) =
∴ p ˆ =
− 1 =
= 0.68
3
100
100
100
(iv)
(a)
d 2
dp
2
log L = −
at p ˆ = 0.68 ,
(v)
56
( p + 1)
d 2
dp
2
2
−
44
(2 − p ) 2
log L = −
56
1.68
2
−
− 1
= 0.02218 and
− 45.0938
44
1.32 2
(b) CRlb =
(c) Approximate 95% CI for p is
giving 0.68 ± 0.292
(a) Approximate 95% CI for p is p ˆ ± 1.96
= − 45.0938
p ˆ ≈ N ( p , 0.02218)
p ˆ ± 1.96 0.02218
p ˆ (1 − p ˆ )
100
giving
0.68 ± 1.96
0.68(1 − 0.68)
100
⇒ 0.68 ± 1.96(0.0466) ⇒ 0.68 ± 0.091
Page 5Subject CT3 (Probability and Mathematical Statistics Core Technical) — %%%%%%%%%%%%%%%%%%%%%%%%%%%%%%%%%%%%5
(b)
