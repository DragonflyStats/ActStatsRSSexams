\documentclass[a4paper,12pt]{article}

%%%%%%%%%%%%%%%%%%%%%%%%%%%%%%%%%%%%%%%%%%%%%%%%%%%%%%%%%%%%%%%%%%%%%%%%%%%%%%%%%%%%%%%%%%%%%%%%%%%%%%%%%%%%%%%%%%%%%%%%%%%%%%%%%%%%%%%%%%%%%%%%%%%%%%%%%%%%%%%%%%%%%%%%%%%%%%%%%%%%%%%%%%%%%%%%%%%%%%%%%%%%%%%%%%%%%%%%%%%%%%%%%%%%%%%%%%%%%%%%%%%%%%%%%%%%
  \usepackage{eurosym}
\usepackage{vmargin}
\usepackage{amsmath}
\usepackage{graphics}
\usepackage{epsfig}
\usepackage{enumerate}
\usepackage{multicol}
\usepackage{subfigure}
\usepackage{fancyhdr}
\usepackage{listings}
\usepackage{framed}
\usepackage{graphicx}
\usepackage{amsmath}
\usepackage{chngpage}
%\usepackage{bigints}
\usepackage{vmargin}

% left top textwidth textheight headheight

% headsep footheight footskip

\setmargins{2.0cm}{2.5cm}{16 cm}{22cm}{0.5cm}{0cm}{1cm}{1cm}

\renewcommand{\baselinestretch}{1.3}

\setcounter{MaxMatrixCols}{10}

\begin{document}


Let the random variable X have the Poisson distribution with probability function:
\[f ( x ) =
(i)
e −\lambda \lambda x
,
x !
x = 0,1, 2,...\]
Show that \[P ( X = k + 1) =
\lambda
P ( X = k ), k = 0,1, 2,...
k + 1\]

\item It is believed that the distribution of the number of claims which arise on insurance
policies of a certain class is Poisson. A random sample of 1,000 policies is taken
from all the policies in this class which have been in force throughout the past year.
\item The table below gives the number of claims per policy in this sample.
No. of claims, k: 0
1
2
3 4 5 6 7 8 or more
No. of policies, f k : 310 365 202 88 26 6 2 1

\begin{enumerate}

\item For these data the maximum likelihood estimate (MLE) of the Poisson parameter \lambda is
\lambda ˆ = 1.186.
\item 
10
(ii) Calculate the frequencies expected under the Poisson model with parameter
given by the MLE above, using the recurrence formula of part (i) (or
otherwise).
\item 
(iii) Perform an appropriate statistical test to investigate the assumption that the
numbers of claims arising from this particular class of policies follow a
Poisson distribution.
\end{e3numerate}


%%%%%%%%%%%%%%%%%%%%%%%%%%%%%%%%%%%%5
\newpage


\begin{itemize}
\item 9
(i)
(ii)
n is not particularly large for the use of the CLT, but the approximation
is still quite close to the true probability.
\item %%- September 2010 Question 9

\begin{eqnarray*}
P ( X = k + 1) 
&=& e^{−\lambda} \frac{\lambda^{k+1}}{(k + 1)!} \\
&=& e^{−\lambda} \frac{\lambda \times \lambda^{k}}{(k + 1) \;\times \;k!} \\
&=& e^{−\lambda} \frac{\lambda}{k + 1} \frac{\lambda^{k}}{k!} \\
&=& e^{−\lambda} \frac{\lambda}{k + 1} P(X=k) \\
\end{eqnarray*}

Using P ( X = 0) = e
− 1.186

Using P ( X = 0) = e
− 1.186
7
, P ( X = 8 or more) = 1 − \sum P ( X = i ) , and the
i = 0
recurrent formula, we obtain:

\begin{center}
\begin{tabular}{c|c|c|c|c|c|c|c|c|c}
K & 0 & 1 & 2 & 3 & 4 & 
5 & 6 & 7 &  8 or more \\ \hiine
P ( X = k ) & 0.3054 0.3623 0.2148 0.0849 0.0252 0.0060 0.0012 0.0002 4 × 10 −5
Expected, e k & 305.4 & 362.3 & 214.8 & 84.9 & 25.2 & 6.0 & 1.2 & 
0.2 & 
0.0
\end{tabular}
\end{center}

\item 
Combining the last 4 categories to obtain expected frequencies greater than 5,
we have:
k
No. of policies, f k
Expected, e k
0
310
305.4
1
365
362.3
2
202
214.8
3
88
84.9
4
26
25.2
5 or more
9
7.4
\item This gives
χ = \sum
2
( f k − e k ) 2
e k
= 0.0693 + 0.0201 + 0.7628 + 0.1132 + 0.0254 + 0.3459 = 1.3367
2
DF = 6 − 1 − 1 = 4, and from statistical tables, χ 0.05,4
= 9.488 .

\item Therefore, we do not have evidence against the hypothesis that the number of
claims comes from a Poisson(1.186) distribution.
\item (Alternatively if we only combine the last 3 categories, the expected
frequencies for 5 and 6 or more policies are 6 and 1.4, with observed
frequencies 6 and 3 respectively. These give χ 2 = 2.819 on 5 DF, and with
2
χ 0.05,5
= 11.071 the conclusion is the same as before.)
\end{itemize}
\end{document}
