\documentclass[a4paper,12pt]{article}

%%%%%%%%%%%%%%%%%%%%%%%%%%%%%%%%%%%%%%%%%%%%%%%%%%%%%%%%%%%%%%%%%%%%%%%%%%%%%%%%%%%%%%%%%%%%%%%%%%%%%%%%%%%%%%%%%%%%%%%%%%%%%%%%%%%%%%%%%%%%%%%%%%%%%%%%%%%%%%%%%%%%%%%%%%%%%%%%%%%%%%%%%%%%%%%%%%%%%%%%%%%%%%%%%%%%%%%%%%%%%%%%%%%%%%%%%%%%%%%%%%%%%%%%%%%%

\usepackage{eurosym}
\usepackage{vmargin}
\usepackage{amsmath}
\usepackage{graphics}
\usepackage{epsfig}
\usepackage{enumerate}
\usepackage{multicol}
\usepackage{subfigure}
\usepackage{fancyhdr}
\usepackage{listings}
\usepackage{framed}
\usepackage{graphicx}
\usepackage{amsmath}
\usepackage{chngpage}

%\usepackage{bigints}
\usepackage{vmargin}

% left top textwidth textheight headheight

% headsep footheight footskip

\setmargins{2.0cm}{2.5cm}{16 cm}{22cm}{0.5cm}{0cm}{1cm}{1cm}

\renewcommand{\baselinestretch}{1.3}

\setcounter{MaxMatrixCols}{10}

\begin{document}
5
A computer routine selects one of the integers 1, 2, 3, 4, 5 at random and replicates
the process a total of 100 times. Let S denote the sum of the 100 numbers selected.
Calculate the approximate probability that S assumes a value between 280 and 320
inclusive.
[5]

%%%%%%%%%%%%%%%%%%%%%%%%%%%%%%%%%%%%%%%%%%%%%%%%%%%%%%%%%%%%%%%%%%%%%%%%%%%%%5
5
S = Σ X i where X i has a uniform distribution on 1, 2, 3, 4, 5, with mean 3 and variance
(25 – 1)/12 = 2 (result known, or calculated via E [ X 2 ] = 11, or from book of formulae,
p10, with a = 1, b = 5, h = 1).
So S ~ N (300, 200) approximately
320.5 − 300 ⎞
⎛ 279.5 − 300
P (280 ≤ S ≤ 320) = P ⎜
< Z <
⎟
200
200 ⎠
⎝
= P ( − 1.450 < Z < 1.450 ) = 0.853
