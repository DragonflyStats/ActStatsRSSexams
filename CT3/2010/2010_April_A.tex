\documentclass[a4paper,12pt]{article}

%%%%%%%%%%%%%%%%%%%%%%%%%%%%%%%%%%%%%%%%%%%%%%%%%%%%%%%%%%%%%%%%%%%%%%%%%%%%%%%%%%%%%%%%%%%%%%%%%%%%%%%%%%%%%%%%%%%%%%%%%%%%%%%%%%%%%%%%%%%%%%%%%%%%%%%%%%%%%%%%%%%%%%%%%%%%%%%%%%%%%%%%%%%%%%%%%%%%%%%%%%%%%%%%%%%%%%%%%%%%%%%%%%%%%%%%%%%%%%%%%%%%%%%%%%%%

\usepackage{eurosym}
\usepackage{vmargin}
\usepackage{amsmath}
\usepackage{graphics}
\usepackage{epsfig}
\usepackage{enumerate}
\usepackage{multicol}
\usepackage{subfigure}
\usepackage{fancyhdr}
\usepackage{listings}
\usepackage{framed}
\usepackage{graphicx}
\usepackage{amsmath}
\usepackage{chngpage}

%\usepackage{bigints}
\usepackage{vmargin}

% left top textwidth textheight headheight

% headsep footheight footskip

\setmargins{2.0cm}{2.5cm}{16 cm}{22cm}{0.5cm}{0cm}{1cm}{1cm}

\renewcommand{\baselinestretch}{1.3}

\setcounter{MaxMatrixCols}{10}

\begin{document}
\begin{enumerate}

%%% Question 1
\item The mean height of the women in a large population is 1.671m while the mean height of the men in the population is 1.758m. The mean height of all the members of the population is 1.712m.
Calculate the percentage of the population who are women.

%%% Question 2
\item 
Consider a group of 10 life insurance policies, seven of which are on male lives and three of which are on female lives. Three of the 10 policies are chosen at random (one after the other, without replacement).
Find the probability that the three selected policies are all on male lives.

%%% Question 3
\item 
Let X 1 , X 2 , ... , X n be a random sample of size n from a population with mean μ and
variance σ 2 .
Let the sample mean be X and the sample variance be S 2 =
1
{ Σ X i 2 − nX 2 } .
n − 1
σ 2
.
You may assume that E ⎣ ⎡ X ⎦ ⎤ = μ and V ⎡ ⎣ X ⎤ ⎦ =
n
Show that $E [ S^2 ] = \sigma^2$ .
%%% Question 4
\item 
A random sample of five such claim amounts yields a mean of x = 17.5 .
\item 
7
Use the pivotal method with the $\chi^2$ result from part \item (b) to obtain a 95%
confidence interval for $\lambda$
\end{enumerate}
%%%%%%%%%%%%%%%%%%%%%%%%%%%%%%%%%%%%%%%%%%%%%%%%%%%%%%%%%%%%%%%%%%%%%%%%%%%%%%%%%%%%%%%%%%%%%%%%%%%%%%%%%%%%%%%%%%%%%%%%5
\newpage



%%%%%%%%%%%%%%%
1
Let p be the proportion of women.
Then, using a weighted average, 1.671p + 1.758(1− p) = 1.712
⇒ 0.087p = 0.046 ⇒ p = 0.529 so percentage is 52.9%
%%%%%%%%%%%%%%%%%%%%%
2
P(all 3 on male lives) =
7 6 5
× ×
10 9 8
=
7
24
= 0.292
⎛ 7 ⎞ ⎛ 10 ⎞
[OR ⎜ ⎟ / ⎜ ⎟ = 35 /120 = 7 / 24 ]
⎝ 3 ⎠ ⎝ 3 ⎠
3
%%%%%%%%%%%%%%%%%%%%%%%%%%%%%%%%
\end{document}
