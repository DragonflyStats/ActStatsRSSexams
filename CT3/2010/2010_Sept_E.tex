\documentclass[a4paper,12pt]{article}

%%%%%%%%%%%%%%%%%%%%%%%%%%%%%%%%%%%%%%%%%%%%%%%%%%%%%%%%%%%%%%%%%%%%%%%%%%%%%%%%%%%%%%%%%%%%%%%%%%%%%%%%%%%%%%%%%%%%%%%%%%%%%%%%%%%%%%%%%%%%%%%%%%%%%%%%%%%%%%%%%%%%%%%%%%%%%%%%%%%%%%%%%%%%%%%%%%%%%%%%%%%%%%%%%%%%%%%%%%%%%%%%%%%%%%%%%%%%%%%%%%%%%%%%%%%%
  \usepackage{eurosym}
\usepackage{vmargin}
\usepackage{amsmath}
\usepackage{graphics}
\usepackage{epsfig}
\usepackage{enumerate}
\usepackage{multicol}
\usepackage{subfigure}
\usepackage{fancyhdr}
\usepackage{listings}
\usepackage{framed}
\usepackage{graphicx}
\usepackage{amsmath}
\usepackage{chngpage}
%\usepackage{bigints}
\usepackage{vmargin}

% left top textwidth textheight headheight

% headsep footheight footskip

\setmargins{2.0cm}{2.5cm}{16 cm}{22cm}{0.5cm}{0cm}{1cm}{1cm}

\renewcommand{\baselinestretch}{1.3}

\setcounter{MaxMatrixCols}{10}

\begin{document}

\begin{enumerate}
11
\item A life insurance company issuing critical illness insurance wants to compare the delay times from the date when a claim is made until it is settled, for different causes of
illness covered. Random samples of 12 claims associated with two types of illness (A and B) related to heart disease have been collected. The logarithms of the delay times
are given below (where the original times were measured in days):
Cause A, y A : 4.0 5.4 4.6 3.5 4.2 4.5 4.2 4.9 5.1 5.2 5.1 5.4
Cause B, y B : 5.7 5.6 4.2 5.1 4.4 5.9 5.4 3.9 5.7 4.5 4.8 3.9
For these data:
\sum y A = 56.1, \sum y 2 A = 266.33,
\sum y B = 59.1, \sum y B 2 = 297.03
\begin{enumerate}
\item 
(i) Use a suitable t-test to investigate the hypothesis that the mean delay time is the same for claims related to the two causes of illness and state clearly your
conclusion.
\item 
(ii) Give a possible reason why the logarithms of the original delay time observations are used in this analysis.
\item (iii)

(a) Calculate an equal-tailed 95\% confidence interval for \sigma 2 A \sigma 2 B , the
ratio of the variances of the delay times for the two causes of illness.
(b) Comment on the validity of the test in part (i) based on this confidence
interval.
\end{enumerate}

%%%%%%%%%%%%%%%%%%%%%%%%%%%%%

\item The company collects a third sample of 12 claims associated with an illness (C) related to brain disease, and the logarithms of the delay times are given below:

Cause C, y c : 5.6 6.2 6.0 5.6 7.1 5.0 4.5 6.4 4.6 6.0 5.5 5.3

For these data:
$\sum y C = 67.8$, $\sum y C 2 = 389.28$
For data in all three samples:
$\sum\sum y = 183.0$, $\sum\sum y 2 = 952.64$

(iv) Use analysis of variance to test the hypothesis that the mean delay times are the same for all three causes of illness.
(v) State the assumptions made for this analysis of variance.

Comment briefly on the validity of the test in (iv), using the plot of the
residuals of the analysis given below.

(vi)
-1.5
-1.0
-0.5
0.0
0.5
1.0
1.5
Residuals
\end{enumerate}
\newpage
%%%%%%%%%%%%%%%%%%%%%%%%%%%%%%%
11
(i)
Less accurate estimation is the penalty paid for using the randomised
response method.
We want to test H 0 : \mu A = \mu B against H 1 : \mu A \neq\mu B .
Data give: y A = 56.1/12 = 4.675 , y B = 59.1/12 = 4.925
s 2 A = (266.33 − 56.1 2 /12) /11 = 0.36932 ,
s B 2 = (297.03 − 59.1 2 /12) /11 = 0.54205
Assuming that the two samples come from normal distributions with the same
variance,
11 s 2 A + 11 s B 2
2
= 0.455685
we first compute the pooled variance as s p =
22
y − y B
which gives t = A
= − 0.907 .
s p 2 /12
Critical values at 5\% level are t 22 (0.025) = −2.074 and t 22 (0.975) = 2.074
so we don’t have evidence against H 0 and conclude that the mean delay time is the same for claims associated with the two causes of illness.
(ii)
(iii)
Distribution of times can be skewed to the right, and we need a log
transformation to normalise the data (for test to be valid).
(a)
CI is given by
⎛ s 2 A / s B 2
⎞
, ( s A 2 / s B 2 ) * F 11,11 (0.025) ⎟
⎜ ⎜
⎟
⎝ F 11,11 (0.025)
⎠
F 11,11 (0.025) = 3.478 (using interpolation in the tables)
giving CI as (0.68134/3.478, 0.68134*3.478) = (0.196, 2.370).
(b)
(iv)
Page 6
The value “1” is included in the 95\% CI, meaning that the assumption
of common variance made for the test is valid.
SS T = 952.64 – 183 2 /36 = 22.39
SS B = (56.1 2 + 59.1 2 + 67.8 2 )/12 − 183 2 /36 = 6.155
⇒ SS R = 22.39 − 6.155 = 16.235
Source of variation d.f. SS MSS
Between
Residual
Total 2
33
35 6.155
16.235
22.390 3.078
0.492  — %%%%%%%%%%%%%%%%%%%%%%%%%%%%%%%%%%%%5
F = 3.078/0.492 = 6.256 on (2,33) degrees of freedom.
From tables, F 2,33 (0.05) is between 3.276 and 3.295, and F 2,33 (0.01) is
between 5.289 and 5.336.
We have strong evidence against the null hypothesis and conclude that the three mean delay times are not equal.
12
(v) The assumptions are that the data come from normal populations with constant variance.
(vi) The plot suggests that the normality assumption is reasonable and that variance does not depend on cause. Test seems valid.
\end{document}
