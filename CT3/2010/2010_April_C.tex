\documentclass[a4paper,12pt]{article}

%%%%%%%%%%%%%%%%%%%%%%%%%%%%%%%%%%%%%%%%%%%%%%%%%%%%%%%%%%%%%%%%%%%%%%%%%%%%%%%%%%%%%%%%%%%%%%%%%%%%%%%%%%%%%%%%%%%%%%%%%%%%%%%%%%%%%%%%%%%%%%%%%%%%%%%%%%%%%%%%%%%%%%%%%%%%%%%%%%%%%%%%%%%%%%%%%%%%%%%%%%%%%%%%%%%%%%%%%%%%%%%%%%%%%%%%%%%%%%%%%%%%%%%%%%%%

\usepackage{eurosym}
\usepackage{vmargin}
\usepackage{amsmath}
\usepackage{graphics}
\usepackage{epsfig}
\usepackage{enumerate}
\usepackage{multicol}
\usepackage{subfigure}
\usepackage{fancyhdr}
\usepackage{listings}
\usepackage{framed}
\usepackage{graphicx}
\usepackage{amsmath}
\usepackage{chngpage}

%\usepackage{bigints}
\usepackage{vmargin}

% left top textwidth textheight headheight

% headsep footheight footskip

\setmargins{2.0cm}{2.5cm}{16 cm}{22cm}{0.5cm}{0cm}{1cm}{1cm}

\renewcommand{\baselinestretch}{1.3}

\setcounter{MaxMatrixCols}{10}

\begin{document}
\begin{enumerate}
\item An employment survey is carried out in order to determine the percentage, p, of
unemployed people in a certain population in a way such that the estimation has a margin of error less than 0.5\% with probability at least 0.95. In a similar study conducted a year ago it was found that the percentage of unemployed people in the
population was 6\%. Calculate the sample size, n, that is required to achieve this margin of error, by
constructing an appropriate confidence interval (or otherwise).
8

%%%%%%%%%%%%%%%%%%%%%%%%%%%%%%%%%%%%%%%%%%%%%%%%%%%%%%%%%%%%%%%%%%%%%%%%%%
\item For a sample of 100 insurance policies the following frequency distribution gives the
number of policies, f, which resulted in x claims during the last year:
\begin{center}
\begin{tabular}
x:   & f:
0    & 76
1    & 22
2     & 1
3    & 1
\end{tabular}
\end{center}
\begin{enumerate}[(a)]
\item 
Calculate the sample mean, standard deviation and coefficient of skewness for these data on the number of claims per policy.
A Poisson model has been suggested as appropriate for the number of claims per policy.
\item 
(a) State the value of the estimated parameter when a Poisson distribution is fitted to these data using the method of maximum likelihood.
\item (b) Verify that the coefficient of skewness for the fitted model is 1.92, and hence comment on the shape of the frequency distribution relative to that of the corresponding fitted Poisson distribution.
\end{enumerate}

\end{enumerate}
%%%%%%%%%%%%%%%%%%%%%%%%%%%%%%%%%%%%%%%%%%%%%%%%%%%%%%%%%%%%%%%%%%%%%%%%%%
\newpage

%%--- Question 7
\begin{itemize}
\item The 95\% CI for the population percentage p is
giving $| p - \hat{p} | \leq 1.96$
\[ \hat{p} \pm 1.96
\hat{p} (1 - \hat{p} )
n
\hat{p} (1 - \hat{p} )
n\]
\item For the margin of error to be less than 0.5% we need to solve
\[0.005 = 1.96
1.96 2 \hat{p} (1 - \hat{p} )\]
$$\hat{p} (1 - \hat{p} )$
⇒ n =
.
n
0.005 2
\item Using the percentage from the previous study as the value for p̂ , i.e. \hat{p} = 0.06 , we obtain n = 8,666.6.
\item So we need a sample of (at least) 8667 people.
⎛ p ( 1 - p ) ⎞
\item (OR, solution can be based on \hat{p} ~ N ⎜ p ,
⎟ and
n
⎝
⎠
\end{itemize}
\[P ( - 0.005 < \hat{p} - p < 0.005 ) > 0.95 \]without referring to the CI.)


%%%%%%%%%%%%%%%%%%%%%%%%%%%%%%%%%%%%%%%%%%%%%%%%%%%%%%%%%%%%%%%%%%%%%%%%%%%%%%%%%%%%%%
\newpage
8
(i)
$\sum f = 100$, $\sum  fx = 27$, $\sum fx^2 = 35$
\[x =
27
= 0.27
100
1
27 2\]
\[
s = {35 -
} = 0.2799 ∴ s = 0.529
99
100
2\]
\begin{itemize}
\item Third moment about mean is
\[m 3 =
1
{76(0 - 0.27) 3 + 22(1 - 0.27) 3 + (2 - 0.27) 3 + (3 - 0.27) 3 } = 0.3259\]
100\]
\item [OR: using Σ fx 3 = 57, m 3 =
1
\[{57 - 3(0.27)(35) + 2(100)(0.27) 3 } \]
100
\item So coefficient of skewness is
\[0.3259
(0.2799) 3/2
= 2.20\]
\item [OR: can use m 2 = 0.2771 in denominator to give 2.23 ]

\end{itemize}
%%%%%%%%%%%%%%%%%%%5
(ii)
(a) $\hat{\mu}  = \bar{x} = 0.27$
(b) Coefficient of skewness is
1
= 1.92 (from book of formulae, p7)
0.27
so, the data distribution is slightly more positively skewed than the fitted Poisson.

%%%%%%%%%%%%%%%%%%%%%%%%%%%%%%%%%%%%%%%%%%%%%%%%%%%%%%%%%%%%%%%%%%%%%%%%%%
\end{document}
