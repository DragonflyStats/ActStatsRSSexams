\documentclass[a4paper,12pt]{article}

%%%%%%%%%%%%%%%%%%%%%%%%%%%%%%%%%%%%%%%%%%%%%%%%%%%%%%%%%%%%%%%%%%%%%%%%%%%%%%%%%%%%%%%%%%%%%%%%%%%%%%%%%%%%%%%%%%%%%%%%%%%%%%%%%%%%%%%%%%%%%%%%%%%%%%%%%%%%%%%%%%%%%%%%%%%%%%%%%%%%%%%%%%%%%%%%%%%%%%%%%%%%%%%%%%%%%%%%%%%%%%%%%%%%%%%%%%%%%%%%%%%%%%%%%%%%
  \usepackage{eurosym}
\usepackage{vmargin}
\usepackage{amsmath}
\usepackage{graphics}
\usepackage{epsfig}
\usepackage{enumerate}
\usepackage{multicol}
\usepackage{subfigure}
\usepackage{fancyhdr}
\usepackage{listings}
\usepackage{framed}
\usepackage{graphicx}
\usepackage{amsmath}
\usepackage{chngpage}
%\usepackage{bigints}
\usepackage{vmargin}

% left top textwidth textheight headheight

% headsep footheight footskip

\setmargins{2.0cm}{2.5cm}{16 cm}{22cm}{0.5cm}{0cm}{1cm}{1cm}

\renewcommand{\baselinestretch}{1.3}

\setcounter{MaxMatrixCols}{10}

\begin{document}

\begin{enumerate}

12
An investigation concerning the improvement in the average performance of female
track athletes relative to male track athletes was conducted using data from various
international athletics meetings over a period of 16 years in the 1950s and 1960s. For
each year and each selected track distance the observation y was recorded as the
average of the ratios of the twenty best male times to the corresponding twenty best
female times.
The data for the 100 metres event are given below together with some summaries.
year t:
ratio y: 1
0.882 2
0.879 3
0.876 4
0.888 5
0.890 6
0.882 7
0.885 8
0.886
year t:
ratio y: 9
0.885 10
0.887 11
0.882 12
0.893 13
0.878 14
0.889 15
0.888 16
0.890
\sigma t = 136, \sigma t 2 = 1496, \sigma y = 14.160, \sigma y 2 = 12.531946, \sigma ty = 120.518
(i)
(ii)
Draw a scatterplot of these data and comment briefly on any relationship
between ratio and year.
Verify that the equation of the least squares fitted regression line of ratio on
year is given by:
y = 0.88105 + 0.000465t.
(iii)


(a) Calculate the standard error of the estimated slope coefficient in part
(ii).
(b) Determine whether the null hypothesis of “no linear relationship”
would be accepted or rejected at the 5% level.
(c) Calculate a 95\% confidence interval for the underlying slope
coefficient for the linear model.
%%%%%%%%%%%%%%%%%%%%%%%%%%%%%%%%%%%%5
Corresponding data for the 200 metres event resulted in an estimated slope coefficient
of:
\beta ˆ = 0.000487 with standard error 0.000220.
(iv)
(a) Determine whether the “no linear relationship” hypothesis would be
accepted or rejected at the 5% level.
(b) Calculate a 95\% confidence interval for the underlying slope
coefficient for the linear model and comment on whether or not the
underlying slope coefficients for the two events, 100m and 200m, can
be regarded as being equal.
(c) Discuss why the results of the tests in parts (iii)(b) and (iv)(a) seem to
contradict the conclusion in part (iv)(b).

\newpage



%%%%%%%%%%%%%%%%%%%%%%%%%%%%%%%%%%%%%%%%%%%%%%%%%%%%%%%%%%%%%%%%%%%%%%%%%%%%%%%%%%%%%%%%%%%%%%%%%%%%%%%%%%%%

\begin{itemize}
\item 

(i) Scatterplot with suitable axes and clearly labelled:
There does not appear to be much of a relationship, perhaps a slight increasing
linear relationship but it is weak with quite a bit of scatter.
\item 
(ii)
n = 16
S tt = 1496 −
136 2
= 340
16
S_{yy} = 12.531946 −
S ty = 120.518 −
14.160 2
= 0.000346
16
(136)(14.160)
= 0.158
16
Page 7  — %%%%%%%%%%%%%%%%%%%%%%%%%%%%%%%%%%%%5
\beta ˆ =
S ty
S tt
=
0.158
= 0.0004647
340
14.160
136
\hat{\alpha}= y − \beta ˆ t =
− (0.0004647)
= 0.88105
16
16
Fitted line is y = 0.88105 + 0.000465t
\item 
(iii)
(a)
s.e. ( \betâ ) =
\hat{\sigma} 2 =
\hat{\sigma} 2
S tt
2
S ty
1
where \hat{\sigma} =
( S_{yy} −
)
n − 2
S tt
2
1
0.158 2
(0.000346 −
) = 0.0000195
14
340
0.0000195
∴ s . e .( \beta ˆ ) =
= 0.000239
340
\item 
(b)
Null hypothesis of “no linear relationship” is equivalent to H 0 : \beta = 0
We use t =
\beta ˆ
~ t 14 under H 0 : \beta = 0
s . e .( \beta ˆ )
Observed t =
0.000465
= 1.95 and
0.000239
t 0.025,14 = 2.145
So we must accept H 0 : no linear relationship at the 5% level.
\item 
(iv)
(c) 95\% CI is 0.000465 \pm 2.145 × 0.000239
giving 0.000465 \pm 0.000513 or (−0.000048, 0.000978)
(a) Observed t =
0.000487
= 2.21 – this is greater than t 0.025,14 = 2.145
0.000220
So we reject H 0 : no linear relationship at the 5% level.
\item 
(b)
95\% CI is 0.000487 \pm 2.145 × 0.000220
giving 0.000487 \pm 0.000472 or (0.000015, 0.000959)
The two CIs overlap substantially, so there is no evidence to suggest
that the slopes are different.
Page 8  — %%%%%%%%%%%%%%%%%%%%%%%%%%%%%%%%%%%%5
\item 
(c)
Although the tests have different conclusions at the 5% level, the 100m
observed t is only just inside the critical value of 2.145 and the 200m
one is just outside. This in fact agrees with, rather than contradicts, the
conclusion that the slopes are not different.
\end{itemize}
\end{document}
