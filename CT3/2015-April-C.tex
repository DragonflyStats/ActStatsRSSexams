\documentclass[a4paper,12pt]{article}



%%%%%%%%%%%%%%%%%%%%%%%%%%%%%%%%%%%%%%%%%%%%%%%%%%%%%%%%%%%%%%%%%%%%%%%%%%%%%%%%%%%%%%%%%%%%%%%%%%%%%%%%%%%%%%%%%%%%%%%%%%%%%%%%%%%%%%%%%%%%%%%%%%%%%%%%%%%%%%%%%%%%%%%%%%%%%%%%%%%%%%%%%%%%%%%%%%%%%%%%%%%%%%%%%%%%%%%%%%%%%%%%%%%%%%%%%%%%%%%%%%%%%%%%%%%%
  
  
  
  \usepackage{eurosym}

\usepackage{vmargin}

\usepackage{amsmath}

\usepackage{graphics}

\usepackage{epsfig}

\usepackage{enumerate}

\usepackage{multicol}

\usepackage{subfigure}

\usepackage{fancyhdr}

\usepackage{listings}

\usepackage{framed}

\usepackage{graphicx}

\usepackage{amsmath}

\usepackage{chngpage}



%\usepackage{bigints}

\usepackage{vmargin}



% left top textwidth textheight headheight



% headsep footheight footskip



\setmargins{2.0cm}{2.5cm}{16 cm}{22cm}{0.5cm}{0cm}{1cm}{1cm}



\renewcommand{\baselinestretch}{1.3}



\setcounter{MaxMatrixCols}{10}



\begin{document}

\begin{enumerate}

%%%%%%%%%%%%%%%%%%%%%%%%%%%%%%%%%%%%%%%%%%%%%%%%%%%%%%%%%%%%%%%%%%%%%%%%%%%%%%%%%
7 A continuous random variable X has the cumulative distribution function FX(x) given
by:
  3
0, 0
( ) 1 , 0 2
8
1, 2
X
x
F x x x
x
 
    

 
(i) Determine the probability density function of X. [2]
(ii) Calculate P(0.5 <X< 1). [2]
Let Y  X .
(iii) Determine the cumulative distribution function and the probability density
function of Y. [4]
(iv) Calculate the expected values of X and Y. [4]
[Total 12]
CT3 A2015–4
8 The random variables X and Y have a joint probability distribution with density
function:
  3 , 0 1
, )
0, otherwise
xy (
  x y x
  f x y
     
  
  
  (i) Determine the marginal densities of X and Y. [4]
  (ii) State, with reasons, whether X and Y are independent. [2]
  (iii) Determine E[X] and E[Y]. [2]
  [Total 8]
  
%%%%%%%%%%%%%%%%%%%%%%%%%%%%%%%%%%%%%%%%%%%5
  7 (i)     3 2
  8
  f x  F x  x for 0 ≤ x ≤2
  (ii) 0.5 1 1 0.5 1 1 0.53  7 0.109375
  8 64
  P  X   F  F    
  (iii) Distribution function for 0  y  2 :
      2  2  1 6
  X 8 P Y  y  P X  y   F y  y
  Density function 0  y  2 :
      6 5
  Y 8 f y  y
  (iv)  
  2 2
  3 4 4
  0 0
  3 3 1 3 2 3 16 3
  8 84 32 32 2
  E X  x dx   x       
   
  2 2
  6 7 3.5
  0 0
  6 6 1 6 2 1.212183
  8 8 7 56
  E Y  y dy   y      
  Most candidates did very well. However, candidates that were not very competent with
  differentiation and integration of functions made basic errors.
  8 (i)     2
  0
  0
  3 3 3
  x
  y x
  X y f x xdy xy x 
       for 0 <x< 1
      1 1
  3 3 2 3 1 2
  2 2
  x
  Y
  y x y
  f y xdx x y
  
  
           for 0 <y< 1
  (ii) Not independent because fX x fY  y  fXY x, y
  (iii)    
  1 1
  4
  0 0
  3 0.75
  X 4 E X  xf x dx   x     
     
  1 2 4 1
  0 0
  3 3
  Y 2 2 4 8
  E Y yf y dy y y
    
           
  
  Subject CT3 (Probability and Mathematical Statistics) – April 2015 – Examiners’ Report
  Page 7
  In part (i) many candidates could not identify correctly the range of integration, but part (ii)
  was well answered. In part (iii) again the wrong range of the variables was often used.
  
