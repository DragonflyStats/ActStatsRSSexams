CT3 A2014–49
The weekly amount spent on childcare for one child is believed to depend on the age
of the child. We denote by X the random variable describing the cost per child for a
randomly selected child of age one year, Y being the cost for a three year old child,
and Z the cost for a five year old child. It is assumed that X, Y, and Z are normally
distributed and that childcare costs are independent between children. Random
samples of children of different ages are taken and the weekly childcare costs are
recorded during the year 2012. A summary of the data is given in the following table:
Random variable
X
Y
Z
Age of child
1
3
5
Average cost per week per child 200 170 155
Sample standard deviation
30 30 20
Sample size
25 25 25
(i)
Calculate the overall average weekly cost of childcare per child for the
children in these samples.
[1]
(ii) Calculate a 95% confidence interval for the expected childcare cost for a child
aged one year.
[2]
(iii) Calculate a 95% confidence interval for the expected childcare cost for a child
aged five years.
[2]
(iv) Calculate a 95% confidence interval for the ratio of the variances of X and Z .
[3]
(v) Perform a test at 5% significance level for the null hypothesis that the
variances of X and Z are equal based on your answer to part (iv).
[1]
(vi) Calculate an approximate 95% confidence interval for the difference between
the average weekly childcare cost per child for children aged one and for
children aged five. Justify any assumptions that you make and explain any
approximate values you use.
[5]
(vii) Perform a test to decide if there is a difference between the expected weekly
childcare cost per child spent for children aged one and for children aged five
based on your answer to part (vi).
[1]
(viii) Perform an analysis of variance to decide if the age of a child has an impact on
the weekly amount spent on childcare.
[7]
[Total 22]
CT3 A2014–5

%%%%%%%%%%%%%%%%%%%%%%%%%%%%%%%%%%%%%%%%%%%%%%%%%%%%%%%%%%%%%%%%%%%%%%%%%%%%%%%%%%%%%%%%%%%%%%%%%%%%%%%%%%%%%
9
(i) Overall average is (200 + 170 + 155)/3 = 175 since sample sizes are all equal.
(ii) X  t 0.025, 24
30
  200  2.064*6, 200  2.064*6    187.62, 212.38 
5
(iii) Z  t 0.025, 24
20
  155  2.064 * 4,1 55  2.064 * 4    146.74, 1 63.26 
5
(iv)  S X 2
  2.25
S X 2
1



,
, 2.25  2.269    0.992, 5.105 
F
 2
  
24,24
2

  S Z F 24,24 S Z
   2.269
(v)
(vi)
The ratio 1 is contained in the confidence interval, therefore the null
hypothesis  2 X   2 Z cannot be rejected.
Pooled variance:
s 2 p


24  30 2  20 2
48
  650 .
Difference: 200  155 = 45

2
2 
, 45  t 0.025,48 650
 45  t 0.025,48 650

25
25 

 45  2.01  7.21, 45  2.01  7.21    30.51, 59.49 
where we have used the approximation t 0.025,48  2.01 (see tables, value for
t 0.025,50  2.009 )
We made the assumption  2 X   2 Z which is justified by the result in parts (iv)
and (v).
(vii) The confidence interval does not contain 0, so there is a difference.
(viii) SS R  24  30 2  30 2  20 2  52800


Alternative solution possible
2
2
2
SS B  25     200  175    170  175    155  175     26250


Page 8Subject CT3 (Probability and Mathematical Statistics) – April 2014 – Examiners’ Report
SS B / 2 13125

 17.9
SS R / 72 733.33
This is clearly a very large value compared to F 2,72  F 2,60  4.977 at the 1%
level, so the age of the child has an impact on childcare cost.
Generally well answered. In part (iv) calculation of the ratio of the variance of Z over the
variance of X was given full credit. In part (viii) many candidates attempted to calculate the
SS values using the original data, rather than the “quick” formulae given in the answer. This
was given full marks where appropriate, but was not the best use of time in the exam.
