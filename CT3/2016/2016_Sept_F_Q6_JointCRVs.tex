\documentclass[a4paper,12pt]{article}

%%%%%%%%%%%%%%%%%%%%%%%%%%%%%%%%%%%%%%%%%%%%%%%%%%%%%%%%%%%%%%%%%%%%%%%%%%%%%%%%%%%%%%%%%%%%%%%%%%%%%%%%%%%%%%%%%%%%%%%%%%%%%%%%%%%%%%%%%%%%%%%%%%%%%%%%%%%%%%%%%%%%%%%%%%%%%%%%%%%%%%%%%%%%%%%%%%%%%%%%%%%%%%%%%%%%%%%%%%%%%%%%%%%%%%%%%%%%%%%%%%%%%%%%%%%%
  \usepackage{eurosym}
\usepackage{vmargin}
\usepackage{amsmath}
\usepackage{graphics}
\usepackage{epsfig}
\usepackage{enumerate}
\usepackage{multicol}
\usepackage{subfigure}
\usepackage{fancyhdr}
\usepackage{listings}
\usepackage{framed}
\usepackage{graphicx}
\usepackage{amsmath}
\usepackage{chngpage}
%\usepackage{bigints}
\usepackage{vmargin}

% left top textwidth textheight headheight

% headsep footheight footskip

\setmargins{2.0cm}{2.5cm}{16 cm}{22cm}{0.5cm}{0cm}{1cm}{1cm}

\renewcommand{\baselinestretch}{1.3}

\setcounter{MaxMatrixCols}{10}

\begin{document}



6
Let X and Y be random variables with joint probability distribution:
  kx 2 y 2 , 0  x  y  1
f XY ( x , y )  
otherwise
 0,
where k is a constant.
\begin{enumerate}
\item (i) Show that k = 18. [4]
\item (ii) Determine f Y (y), the marginal density function of Y. 
\item (iii) Determine P(X > 0.5| =Y  0.75).
\end{enumerate}
%%%%%%%%%%%%%%%%%%%%%%%%%%%%%%%%%%%%%%%%%%%%%%%
\newpage
Q6
 f XY
(i)
xy
11
y = 1
1
k
( x , y ) dydx =  kx y dydx =    x 2 y 3   dx
 3
 y = x
0 x
0
2 2

1
1
k 2 5
k  x 3 x 6 
k  1 1  k
=  x − x dx =  −  =  −  =
3
3   3
6  
3  3 6  18
0
0 
Want integral equal to 1  k = 18 
y
x = y
f Y ( y ) =  f XY ( x , y ) dx =  18 x 2 y 2 dx =  6 x 3 y 2 
= 6 y 5

 x = 0
(ii)
x
(iii)
0
P ( X > 0.5| Y = 0.75 ) =
0.75

0.5
0.75

f ( x | Y = 0.75 ) ( x ) dx =
0.75

f XY ( x , 0.75 ) / f Y ( 0.75 ) dx 
0.5
0.75
3
3
 4   x 
=  18 x 0.75 / (6 × 0.75 ) dx = 3 ×    
= 0.7037
3    3  

0.5
0.5
2
2
5

[Total 9]
Parts (i) and (ii) were very well answered. Dealing with the conditional
distribution in part (iii) was problematic.
\end{document}
