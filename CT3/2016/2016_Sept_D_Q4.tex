\documentclass[a4paper,12pt]{article}

%%%%%%%%%%%%%%%%%%%%%%%%%%%%%%%%%%%%%%%%%%%%%%%%%%%%%%%%%%%%%%%%%%%%%%%%%%%%%%%%%%%%%%%%%%%%%%%%%%%%%%%%%%%%%%%%%%%%%%%%%%%%%%%%%%%%%%%%%%%%%%%%%%%%%%%%%%%%%%%%%%%%%%%%%%%%%%%%%%%%%%%%%%%%%%%%%%%%%%%%%%%%%%%%%%%%%%%%%%%%%%%%%%%%%%%%%%%%%%%%%%%%%%%%%%%%
  \usepackage{eurosym}
\usepackage{vmargin}
\usepackage{amsmath}
\usepackage{graphics}
\usepackage{epsfig}
\usepackage{enumerate}
\usepackage{multicol}
\usepackage{subfigure}
\usepackage{fancyhdr}
\usepackage{listings}
\usepackage{framed}
\usepackage{graphicx}
\usepackage{amsmath}
\usepackage{chngpage}
%\usepackage{bigints}
\usepackage{vmargin}

% left top textwidth textheight headheight

% headsep footheight footskip

\setmargins{2.0cm}{2.5cm}{16 cm}{22cm}{0.5cm}{0cm}{1cm}{1cm}

\renewcommand{\baselinestretch}{1.3}

\setcounter{MaxMatrixCols}{10}

\begin{document}



[Total 8]4
Consider two portfolios, A and B, of insurance policies and denote by X A the number of claims received in portfolio A and by X B the number of claims received in portfolio B during a calendar year. The observed numbers of claims received during the last calendar year are 134 for portfolio A and 91 for portfolio B. X A and X B are assumed to be independent and to have Poisson distributions with 
unknown parameters  A and  B .
Determine an approximate 99\% confidence interval for the difference  A   B . You
may use an appropriate normal distribution.
[4]
%%%%%%%%%%%%%%%%%%%%%%%%%

Q4
Since the sample size for each portfolio is one, we have that $\hat{\beta} A = 134$ and $\hat{\beta} B = 91$ .

Using the normal approximation we find:
\[\hat{\beta} A − \hat{\beta} B = X A − X B ~ N ( \beta A − \beta B , \beta A + \beta B )\]

The confidence interval is then given by
\[134 − 91 \pm 2.5758 \sqrt{134 + 91} = 43 \pm 2.5758 \times 15 = [ 4.4, 81.6 ]\]

[Total 4]
Performance in this question was mixed. Many candidates seemed unsure
as to what is the correct size of each sample (portfolio).

\end{document}
