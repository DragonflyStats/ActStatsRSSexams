\documentclass[a4paper,12pt]{article}

%%%%%%%%%%%%%%%%%%%%%%%%%%%%%%%%%%%%%%%%%%%%%%%%%%%%%%%%%%%%%%%%%%%%%%%%%%%%%%%%%%%%%%%%%%%%%%%%%%%%%%%%%%%%%%%%%%%%%%%%%%%%%%%%%%%%%%%%%%%%%%%%%%%%%%%%%%%%%%%%%%%%%%%%%%%%%%%%%%%%%%%%%%%%%%%%%%%%%%%%%%%%%%%%%%%%%%%%%%%%%%%%%%%%%%%%%%%%%%%%%%%%%%%%%%%%
  \usepackage{eurosym}
\usepackage{vmargin}
\usepackage{amsmath}
\usepackage{graphics}
\usepackage{epsfig}
\usepackage{enumerate}
\usepackage{multicol}
\usepackage{subfigure}
\usepackage{fancyhdr}
\usepackage{listings}
\usepackage{framed}
\usepackage{graphicx}
\usepackage{amsmath}
\usepackage{chngpage}
%\usepackage{bigints}
\usepackage{vmargin}

% left top textwidth textheight headheight

% headsep footheight footskip

\setmargins{2.0cm}{2.5cm}{16 cm}{22cm}{0.5cm}{0cm}{1cm}{1cm}

\renewcommand{\baselinestretch}{1.3}

\setcounter{MaxMatrixCols}{10}

\begin{document}


An insurance company has a portfolio of 10,000 policies. Based on past data the
company estimates that the probability of a claim on any one policy in a year is 0.003.
It assumes no policy will generate more than one claim in a year.
(i)
(ii)

\begin{enumerate}
\item Determine the approximate probability of more than 40 claims from the
portfolio of 10,000 policies in a year. 
\item Determine an approximate equal-tailed interval into which the number of
claims per year will fall with probability 0.95. 
In practice 42 claims were received in a particular year. A Director of the company
complains about the range of estimates in part (ii) being wrong.
(iii)
\item Comment on the Director’s complaint.
\end{itemize}
\newpage
%%%%%%%%%%%%%%%%%%%%%%%%%
Q3
(i)
Number of claims, N ~ Bin(10000,0.003)
By CLT number of claims approximately N(30, 29.91)

Continuity correction applies
 N − 30 40.5 − 30 
P ( N > 40 ) = P ( N > 40.5 ) = P 
>

29.91 
 29.91
= 1 − Φ ( 1.920 ) = 1 − 0.973 = 0.027
(ii)
95% interval is 30 ± Z 0.975 σ = 30 ± 1.96*5.469 = ( 19.28, 40.72 )


Page 3Subject CT3 (Probability and Mathematical Statistics Core Technical) – September 2016 – Examiners’ Report
(iii)
The probability that the result in any year will lie in the interval is 0.95 so
there is a 5\% probability that the company will see a result outside that range.

% [Total 8]
% Performance here was generally good, but there were some common errors. In part (i) reference to the CLT was often omitted, as was the % continuity correction (or it was applied in the wrong direction). In part (iii) many
% candidates did not identify the main point, about the 95% coverage of the interval.

\end{document}
