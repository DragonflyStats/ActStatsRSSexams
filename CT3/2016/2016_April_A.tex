\documentclass[a4paper,12pt]{article}

%%%%%%%%%%%%%%%%%%%%%%%%%%%%%%%%%%%%%%%%%%%%%%%%%%%%%%%%%%%%%%%%%%%%%%%%%%%%%%%%%%%%%%%%%%%%%%%%%%%%%%%%%%%%%%%%%%%%%%%%%%%%%%%%%%%%%%%%%%%%%%%%%%%%%%%%%%%%%%%%%%%%%%%%%%%%%%%%%%%%%%%%%%%%%%%%%%%%%%%%%%%%%%%%%%%%%%%%%%%%%%%%%%%%%%%%%%%%%%%%%%%%%%%%%%%%
  \usepackage{eurosym}
\usepackage{vmargin}
\usepackage{amsmath}
\usepackage{graphics}
\usepackage{epsfig}
\usepackage{enumerate}
\usepackage{multicol}
\usepackage{subfigure}
\usepackage{fancyhdr}
\usepackage{listings}
\usepackage{framed}
\usepackage{graphicx}
\usepackage{amsmath}
\usepackage{chngpage}
%\usepackage{bigints}
\usepackage{vmargin}

% left top textwidth textheight headheight

% headsep footheight footskip

\setmargins{2.0cm}{2.5cm}{16 cm}{22cm}{0.5cm}{0cm}{1cm}{1cm}

\renewcommand{\baselinestretch}{1.3}

\setcounter{MaxMatrixCols}{10}

\begin{document}

\begin{enumerate}
%%CT3 A2016–2
\item 1 An university director of studies records the number of students failing the
examinations of several courses. The data are presented in the following stem-andleaf
plot where the stems are with units 10 and the leaves with units 1:
  0 | 224
0 | 59
1 | 03
1 | 57889
2 |
  2 |
  3 | 144
3 | 5
\begin{enumerate}[(i)]
\item Determine the range of the data. 
\item Determine the median number of students failing the examinations of these courses. 
\item Determine the mean number of students failing the examinations of these courses. 
\end{enumerate}
%%%%%%%%%%%%%%%%%%%%%%%%%%%%%%%%%%%%%%%%%%%%%%%%%%%%%%%%%%%%%%%%%%%%%%%%%%%%
\item 2 Consider two random variables X and Y.
\begin{enumerate}[(i)]
\item Write down the precise mathematical definition for the correlation coefficient
(X, Y) between X and Y. 
Assume now that $Y = aX + b$ where $a < 0$ and   b  .
\item Determine the value of the correlation coefficient (X, Y). 
\end{enumerate}
%%%%%%%%%%%%%%%%%%%%%%%%%%%%%%%%%%%%%%%%%%%%%%%%%%%%%%%%%%%%%%%%%%%%%%%%%
\item  % Question 3 

A random variable Y has probability density function
f(y) = 1 ,
y

y > 1
where  > 0 is a parameter.
\begin{enumerate}[(i)]
\item  Show that the probability density function of Z = ln(Y) is given by ez and determine its range.
\item State the distribution of Z identifying any parameters involved. 
\end{enumerate}

\end{enumerate}
%%%%%%%%%%%%%%%%%%%%%%%%%%%%%%%%%%%%%%%%%%%%%%%%%%%%%%%%%%%%%%%%%%%%%%%%%%%
\newpage

Solutions
Q1 (i) Range is [2, 35], or 35 −2 = 33. [1]
(ii) Median given as the observation with rank 8.5, i.e. 16. [1]
(iii) Mean is 266/16 = 16.625. [1]
[TOTAL 3]
Well answered. In part (i) both answers shown above were given full credit.
%%%%%%%%%%%%%%%%%%%%%%%%%%%%%%%%%%%%%%%%%%%%%%%%%%%%%%%%%%%%%%%%%%%%%%%%%%%%%%%
Q2 
\begin{itemize}
\item (i) ( , ) Cov( , )
( ) ( )
X Y X Y
V X V Y
ρ = [1]

\item (ii) Cov(X, aX + b) = aV(X) [1]
V(Y) = V(aX + b) = a2V(X) [1]
For a < 0 we obtain
2
( , ) Cov( , ) ( ) 1
( ) ( ) ( ) ( )
X Y X Y aV X
V X V Y V X a V X
ρ = = = − [1]
\end{itemize}
%Part (i) was generally well answered. Some candidates gave the sample correlation coefficient, which is not what the question required. Performance in part (ii) was not very competent, with many candidates
%failing to recognise the negative correlation.
%%%%%%%%%%%%%%%%%%%%%%%%%%%%%%%%%%%%%%%%%%%%%%%%%%%%%%%%%%%%%%%%%%%%%%%%%%%%%%%%%%%%%%%%%%%%%%%5
Q3 (i) z = log yy = ez∴ dy
dz
= ez
f(y) = yθ+1
θ
∴f(z) = θ(ez)−(θ+1) ez = θe−θz [1½]
and since y> 1 z> 0. 

(ii) This is the probability density function of an exponential distribution with
parameter θ. 
[OR: Obtain using the distribution function:
    FZ(z) = Pr(Z≤z) = Pr(Y≤ez) = FY(ez)
  Also, 1 1
  1
  ( ) 1
  y y
  FY y du u y
  u
  −θ −θ
  θ+
    = θ =−   = −   
  Thus, FZ(z) = 1−e−θz implying an Exp(θ) distribution.]

Part (i) required some work involving probabilistic arguments and was not well
answered. There were no problems with part (ii).
\end{document}

