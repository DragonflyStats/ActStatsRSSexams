\documentclass[a4paper,12pt]{article}

%%%%%%%%%%%%%%%%%%%%%%%%%%%%%%%%%%%%%%%%%%%%%%%%%%%%%%%%%%%%%%%%%%%%%%%%%%%%%%%%%%%%%%%%%%%%%%%%%%%%%%%%%%%%%%%%%%%%%%%%%%%%%%%%%%%%%%%%%%%%%%%%%%%%%%%%%%%%%%%%%%%%%%%%%%%%%%%%%%%%%%%%%%%%%%%%%%%%%%%%%%%%%%%%%%%%%%%%%%%%%%%%%%%%%%%%%%%%%%%%%%%%%%%%%%%%
  \usepackage{eurosym}
\usepackage{vmargin}
\usepackage{amsmath}
\usepackage{graphics}
\usepackage{epsfig}
\usepackage{enumerate}
\usepackage{multicol}
\usepackage{subfigure}
\usepackage{fancyhdr}
\usepackage{listings}
\usepackage{framed}
\usepackage{graphicx}
\usepackage{amsmath}
\usepackage{chngpage}
%\usepackage{bigints}
\usepackage{vmargin}

% left top textwidth textheight headheight

% headsep footheight footskip

\setmargins{2.0cm}{2.5cm}{16 cm}{22cm}{0.5cm}{0cm}{1cm}{1cm}

\renewcommand{\baselinestretch}{1.3}

\setcounter{MaxMatrixCols}{10}

\begin{document}

\begin{enumerate}

%% CT3 A2016–7
%%- Question 11 
\item A car magazine published an article exploring the relationship between the mileage
(in units of 1,000 miles) and the selling price (in units of \$1,000) of used cars. The
following data were collected on 10 four year old cars of the same make.
\begin{verbatim}
Car 1 2 3 4 5 6 7 8 9 10
Mileage, x 42 29 51 46 38 59 18 32 22 39
Price, y 5.3 6.1 4.7 4.5 5.5 5.0 6.9 5.7 5.8 5.9
\end{verbatim}
%%%%%%%%%%%%%%%%%%%%%%%%%%%%%%%
x= 376; x2 = 15,600; y = 55.4; y2 = 311.44; xy = 2,014.5
\begin{enumerate}[(i)]
\item ((a) Determine the correlation coefficient between x and y.
(b) Comment on its value.

A linear model of the form y = α + βx + ε is fitted to the data , where the error terms (ε) independently follow a N(0, 2) distribution, with 2 being an unknown
parameter.
\item  Determine the fitted line of the regression model. 
\item (a) Determine a 95\% confidence interval for β.
The article suggests that there is a “clear relationship” between mileage and
selling price of the car.
(b) Comment on this suggestion based on the confidence interval obtained in part (iii)(a).
\item Calculate the estimated difference in the selling prices for cars that differ in
mileage by 5,000 miles. 
\end{enumerate}


%%%%%%%%%%%%%%%%%%%%%%%%%%%%%%%%%%%%%%%%%%%%%%%%%5
%%-- Question 11

\begin{itemize}
\item (i) (a)
3762 15600 1462.4
xx 10 S = − =
  (55.4)2 311.44 4.524
yy 10 S = − =
  Sxy = 2014.5 − (376)(55.4)
10
= −68.54 
68.54 0.843
1462.4 4.524
r = − = −
×

\item (b) There appears to be strong negative linear correlation between mileage and price. 
\item (ii) Least squares estimates:
  68.54 0.0469
1462.4
ˆ xy
xx
s
s
β = = − = −
55.4 0.0469 376 7.303
10 10
αˆ = y −βˆ x = +   =
   

Line given as: yˆ = 7.30−0.0469x 
\item (iii) (a)
2 ( )2
2
68.54
4.524
1462.4
0.163957
2 8
ˆ
xy
yy
xx
s
s
n
 s   −   −  −    
σ =   =   =
  −

2 0.163957 ( ) 0.01059
1462.
ˆ ˆ
xx 4
se
S
β = σ = = 
and t8, 0.025 = 2.306 
%%Subject CT3  – April 2016 55
%%Page 11
95\% CI for ˆβ is given by −0.0469 ±2.306(0.01059) 
i.e. −0.0469 ± 0.02442 or (−0.071, −0.022) 
\item (b) Since the value zero is not included in the interval, the suggestion in the article seems valid. 
\item (iv) βˆ ×5 = −0.2345
\item Estimated difference in price will be \$234.34. 
\end{itemize}

%Parts \item (i), \item (ii) and (iii) are typical questions on correlation and regression and were tackled without problems by most candidates. Note that in part (iii)(b) clear reasoning must be provided for full marks. Performance in part (iv) was
% mixed, with some answers suggesting inadequate understanding of the interpretation of regression coefficients.

%%END OF EXAMINERS’ REPORT
\end{document}
