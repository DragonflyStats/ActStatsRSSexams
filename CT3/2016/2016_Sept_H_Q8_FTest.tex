\documentclass[a4paper,12pt]{article}

%%%%%%%%%%%%%%%%%%%%%%%%%%%%%%%%%%%%%%%%%%%%%%%%%%%%%%%%%%%%%%%%%%%%%%%%%%%%%%%%%%%%%%%%%%%%%%%%%%%%%%%%%%%%%%%%%%%%%%%%%%%%%%%%%%%%%%%%%%%%%%%%%%%%%%%%%%%%%%%%%%%%%%%%%%%%%%%%%%%%%%%%%%%%%%%%%%%%%%%%%%%%%%%%%%%%%%%%%%%%%%%%%%%%%%%%%%%%%%%%%%%%%%%%%%%%
  \usepackage{eurosym}
\usepackage{vmargin}
\usepackage{amsmath}
\usepackage{graphics}
\usepackage{epsfig}
\usepackage{enumerate}
\usepackage{multicol}
\usepackage{subfigure}
\usepackage{fancyhdr}
\usepackage{listings}
\usepackage{framed}
\usepackage{graphicx}
\usepackage{amsmath}
\usepackage{chngpage}
%\usepackage{bigints}
\usepackage{vmargin}

% left top textwidth textheight headheight

% headsep footheight footskip

\setmargins{2.0cm}{2.5cm}{16 cm}{22cm}{0.5cm}{0cm}{1cm}{1cm}

\renewcommand{\baselinestretch}{1.3}

\setcounter{MaxMatrixCols}{10}

\begin{document}


8
Ten pairs of data on a predictor variable (x) and a response variable (y) are available
with the following summary statistics:
x  5.93 y  7.15
10
 ( x i  x ) 2  81.15
i  1
10
 ( x i  x )( y i  y )  89.91 .
i  1
A linear model of the form y = α + \betax + ε is fitted to the data, where the error
terms (ε) independently follow a N (0,  2 ) distribution with  2 being an unknown
parameter.
\begin{enumerate}
\item (i)
Determine the fitted line of the regression model.

A partially completed ANOVA table for this regression analysis is given below.
Source of
variation Degrees of
freedom Sums of
squares Mean
squares
Regression
Residual
Total A
B
9 99.61
21.63
121.24 C
D
\item (ii) Determine the missing values A, B, C and D in the table.

(iii) Determine an estimate of the variance  2 based on the above table.

(iv) (a) Give the interpretation of the coefficient of determination, R 2 , in a
linear regression model.
(b) Determine the value of R 2 for the regression model fitted here, using
the above table.

(v)
CT3 S2016–5
Perform an F test to test the null hypothesis that there is no linear relationship
between x and y, based on the above table.

\end{enumerate}
%%%%%%%%%%%%%%%%%%%%%%%%%%
\newpage
%%%%%%%%%%%%%%%%%%%%%%%%%%%%%%%%%%%%%%%%%%%%%%%%%

Q8
\item (i)
\beta ˆ =
S xy
S_{xx}
=
89.91
= 1.108
81.15

a ˆ = y − \beta ˆ x = 7.15 − 1.108 × 5.93 = 0 .58 
Fitted model is: y ˆ = 0.58 + 1.108 x 
\item (ii) A = 1, B = 8, C = 99.61/1 = 99.61, D = 21.63/8 = 2.704 
(iii) \hat{\sigma} 2 = 21.63 / 8 = 2.7 0 4 
(iv) (a) R 2 gives the proportion of the total variation of y that is explained by x.

(b) R 2 = 99.61/121.24 = 0.822
(v)
We want to test H 0 : \beta= 0 v. H 1 : \beta ≠ 0


Under H 0 the value MS REG / MS RES = 99.61/2.704 = 36.84 should be a value

from the F 1,8 distribution.
The 0.99 quantile of F 1,8 is 11.26

We have strong evidence to reject H 0 and we conclude that there is linear relationship between x and y.

[Total 13]
Generally very well answered. However, many candidates struggled with the
interpretation of the coefficient of determination in part (iv)(a) – this is an
important (and very widely used) concept in regression analysis.
Page 7Subject CT3  – September 2016 

\end{document}
