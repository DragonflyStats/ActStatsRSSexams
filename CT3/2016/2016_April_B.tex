\documentclass[a4paper,12pt]{article}

%%%%%%%%%%%%%%%%%%%%%%%%%%%%%%%%%%%%%%%%%%%%%%%%%%%%%%%%%%%%%%%%%%%%%%%%%%%%%%%%%%%%%%%%%%%%%%%%%%%%%%%%%%%%%%%%%%%%%%%%%%%%%%%%%%%%%%%%%%%%%%%%%%%%%%%%%%%%%%%%%%%%%%%%%%%%%%%%%%%%%%%%%%%%%%%%%%%%%%%%%%%%%%%%%%%%%%%%%%%%%%%%%%%%%%%%%%%%%%%%%%%%%%%%%%%%
  \usepackage{eurosym}
\usepackage{vmargin}
\usepackage{amsmath}
\usepackage{graphics}
\usepackage{epsfig}
\usepackage{enumerate}
\usepackage{multicol}
\usepackage{subfigure}
\usepackage{fancyhdr}
\usepackage{listings}
\usepackage{framed}
\usepackage{graphicx}
\usepackage{amsmath}
\usepackage{chngpage}
%\usepackage{bigints}
\usepackage{vmargin}

% left top textwidth textheight headheight

% headsep footheight footskip

\setmargins{2.0cm}{2.5cm}{16 cm}{22cm}{0.5cm}{0cm}{1cm}{1cm}

\renewcommand{\baselinestretch}{1.3}

\setcounter{MaxMatrixCols}{10}

\begin{document}

\begin{enumerate}

CT3 A2016–3 PLEASE TURN OVER
4 A manufacturing company is analysing its accident record. The accidents fall into two categories:
   Minor – dealt with by first aider. Average cost £50.
 Major – hospital visit required. Average cost £1,000.
The company has 1,000 employees, of which 180 are office staff and the rest work in the factory.

The analysis shows that 10\% of employees have an accident each year and 20\% of  accidents are major. It is assumed that an employee has no more than one accident in
a year.
\begin{enuerate}[(i)]
\item (i) Determine the expected total cost of accidents in a year. [2]
On further analysis it is discovered that a member of office staff has half the probability of having an accident relative to those in the factory.
\item (ii) Show that the probability that a given member of office staff has an accident in a year is 0.0549. [3]
\item (iii) Determine the probability that a randomly chosen employee who has had an accident is office staff. [2]
\end{enumerate}
%%%%%%%%%%%%%%%%%%%%%%%%%%%%%%%%%%%%%%%%%%%%%%%%%%
  5 Players A and B play a game of “heads or tails”, each throwing 50 fair coins. Player
A will win the game if she throws 5 or more heads than B; otherwise, B wins. Let the
random variables XA and XB denote the numbers of heads scored by each player and
D = XA  XB.
\begin{enuerate}[(i)]
\item (i) Explain why the approximate asymptotic distribution of D is normal with mean 0 and variance 25. [3]
\item (ii) Determine the approximate probability that player A wins any particular game, based on your answer in part (i). [2]
\end{enumerate}
%%%%%%%%%%%%%%%%%%%%%%%%%%%%%%%%%%%%%%%%%%%%%%%
6 A statistician is sent a summary of some data. She is told that the sample mean is
9.46 and the sample variance is 25.05. She decides to fit a continuous uniform
distribution to the data.
\begin{enuerate}[(i)]
\item (i) Estimate the parameters of the distribution using the method of moments. [4]
The full data are sent later and are given below:
  3.5 5.4 7.3 8.5 9.2 10.3 11.4 20.1
\item (ii) Comment on the results in part (i) in the light of the full data. [2]
\end{enumerate}
%%%%%%%%%%%%%%%%%%%%%%%%%%%%%%%%%%%%%%%%%%%%%%%%%%%%%%%%5
Q4 (i) Let A = Accident
Expected cost of an accident= Average Cost(Major)*P(Major|A)
+ Average Cost(Minor)*P(Minor|A)
= 0.2*1000 + 0.8*50 =£240 [1]
Expected total cost of accidents = £240 *1000 * 0.1 = £24,000 [1]
(ii) Let O = Office Staff &F = Factory Staff. Then P(A|O) = 0.5 P(A|F)
P(A) = P(A|O)P(O)+P(A|F)P(F) = P(A|O)*0.18 + 2*P(A|O)*0.82 [2]
P(A|O) = 0.1 / (0.18+.82*2) = 0.0549 [1]
(iii) P(O|A) ( | ) ( ) 0.0549*0.18 0.099.
( ) 0.1
P A O P O
P A
= = = [2]
[TOTAL 7]
Part (i) was well answered with most candidates giving a fully or partially correct answer. Answers in parts (ii) and (iii) were generally poor; some candidates were confused with the conditional probabilities and considered conditioning on the wrong event.
%%%%%%%%%%%%%%%%%%%%%%%%%%%%%%%%%%%%%%%%%%%%%%%%%%%%%%%%%%%%%%%%%%%%%%%%5
%%Subject CT3 (Probability and Mathematical Statistics Core Technical) – April 2016 – Examiners’ Report
%%bPage 5
Q5 (i) XA~ Bin(50, 0.5), XB ~ Bin(50, 0.5). [1]
E(D) = 0, V(D) = V(XA) + V(XB) = 2*50*0.5*0.5 = 25 [1]
From CLT: D ~ N(0, 25) [1]
[Alternatively, using CLT XA,XB~ N(25,12.5), and therefore D ~ N(0, 25).]
(ii) Z = (XA−XB − 0) / 5 ~ N(0,1)
P(A wins) = P(D ≥ 5) = P(D> 4.5) with continuity correction [1]
= P(Z> 4.5/5) = P(Z>0.9) = 1 – 0.8159 = 0.1841 [1]
[TOTAL 5]
Mixed performance in both parts. The use of the CLT is important in part (i),
so reference to it must be made for full credit. A common error in part (ii) was
not using or wrongly applying the continuity correction.
Q6 (i) E[X] = (a + b) / 2 b = 2E[X] −a
Var(X ) = (b − a)2 /12 = (2E[X ]− 2a)2 /12 = (E[X ]− a)2 / 3. [2]
aˆ = x − 3s = 0.791bˆ = 2* x − aˆ =18.129. [2]
(ii) The largest observation is greater than our estimate of b in part (i). This would
suggest the uniform distribution is not a good fit to this data, or the largest
observation is a mistaken observation. This also highlights a potential
weakness of the method of moments. [2]
[TOTAL 6]
Part (i) was very well answered. In part (ii) some candidates failed to
recognise that there is zero probability of having a sample value outside the
range given by the parameters.
%%Subject CT3 (Probability and Mathematical Statistics Core Technical) – April 2016 – Examiners’ Report
%%Page 6
\end{document}

