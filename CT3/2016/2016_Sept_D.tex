\documentclass[a4paper,12pt]{article}

%%%%%%%%%%%%%%%%%%%%%%%%%%%%%%%%%%%%%%%%%%%%%%%%%%%%%%%%%%%%%%%%%%%%%%%%%%%%%%%%%%%%%%%%%%%%%%%%%%%%%%%%%%%%%%%%%%%%%%%%%%%%%%%%%%%%%%%%%%%%%%%%%%%%%%%%%%%%%%%%%%%%%%%%%%%%%%%%%%%%%%%%%%%%%%%%%%%%%%%%%%%%%%%%%%%%%%%%%%%%%%%%%%%%%%%%%%%%%%%%%%%%%%%%%%%%
  \usepackage{eurosym}
\usepackage{vmargin}
\usepackage{amsmath}
\usepackage{graphics}
\usepackage{epsfig}
\usepackage{enumerate}
\usepackage{multicol}
\usepackage{subfigure}
\usepackage{fancyhdr}
\usepackage{listings}
\usepackage{framed}
\usepackage{graphicx}
\usepackage{amsmath}
\usepackage{chngpage}
%\usepackage{bigints}
\usepackage{vmargin}

% left top textwidth textheight headheight

% headsep footheight footskip

\setmargins{2.0cm}{2.5cm}{16 cm}{22cm}{0.5cm}{0cm}{1cm}{1cm}

\renewcommand{\baselinestretch}{1.3}

\setcounter{MaxMatrixCols}{10}

\begin{document}

\begin{enumerate}
PLEASE TURN OVER7
An analyst is investigating the number of car insurance claims made by policyholders
living in different parts of the country. Denote by X the number of claims made by
policyholders living in large cities, Y the number of claims made by policyholders
living in small cities, and Z the number of claims made by policyholders living in the
countryside. For each of the three groups of policyholders consider a random sample
of size 500 and count the number of claims made during the last calendar year.
The following table shows the results for the three groups of policyholders:
No claim
One claim
More than one claim
Total
Large City Small City Countryside
370
93
37
500 390
99
11
500 410
87
3
500
Total
1,170
279
51
1,500
For example, 390 of the 500 policyholders living in small cities had no claim during
the last year, and 3 of the 500 policyholders living in the countryside had more than
one claim during the same year.
\item (i)
Perform a  2 -test to test the null hypothesis that the number of claims per
policy is independent of the place of living.
[7]
After some further analysis, an actuary has estimated that the joint distribution of the
number of claims last year and the place of living is given by the following table:
Large City Small City Countryside
No claim
One claim
More than one claim
0.23
0.06
0.04
0.25
0.06
0.02
0.27
0.06
0.01
For example, the probability that a randomly selected policyholder lives in the
countryside and made no claim last year is 0.27.
\item (ii)
CT3 S2016–4
Determine the probability that a randomly selected policyholder:
(a) lives in a small city.
(b) has submitted more than one claim last year.
(c) has submitted more than one claim last year given that the policyholder
lives in a large city.
(d) lives in the countryside given the policyholder submitted fewer than
two claims last year.
(e) lives in a city (small or large) given the policyholder made at least one
claim last year.
[8]
[Total 15]8
Ten pairs of data on a predictor variable (x) and a response variable (y) are available
with the following summary statistics:
x  5.93 y  7.15
10
 ( x i  x ) 2  81.15
i  1
10
 ( x i  x )( y i  y )  89.91 .
i  1
A linear model of the form y = α + \betax + ε is fitted to the data, where the error
terms (ε) independently follow a N (0,  2 ) distribution with  2 being an unknown
parameter.
\item (i)
Determine the fitted line of the regression model.

A partially completed ANOVA table for this regression analysis is given below.
Source of
variation Degrees of
freedom Sums of
squares Mean
squares
Regression
Residual
Total A
B
9 99.61
21.63
121.24 C
D
\item (ii) Determine the missing values A, B, C and D in the table.

(iii) Determine an estimate of the variance  2 based on the above table.

(iv) (a) Give the interpretation of the coefficient of determination, R 2 , in a
linear regression model.
(b) Determine the value of R 2 for the regression model fitted here, using
the above table.

(v)
CT3 S2016–5
Perform an F test to test the null hypothesis that there is no linear relationship
between x and y, based on the above table.

[Total 13]
%%%%%%%%%%%%%%%%%%%%%%%%%%
Q7
\item (i)
Observed frequencies:
No claim
One claim
More than one claim
Total
Large City Small City Countryside Total
370
93
37
500 390
99
11
500 410
87
3
500 1,170
279
51
1,500
Page 5Subject CT3  – September 2016 
Expected frequencies (under independence)
No claim
One claim
More than one claim
Total
Large City Small City Countryside Total
390
93
17
500 390
93
17
500 390
93
17
500 1,170
279
51
1,500

2
e − f )
(
Values of
e
No claim
One claim
More than one claim
:
Large City Small City Countryside
1.025641
0
23.52941 0
0.387097
2.117647 1.025641
0.387097
11.52941

Test statistic:
2
e − f )
(

e
= 2*1.026 + 2*0.387 + 23.53 + 2.118 + 11.53 = 40.004

This compares to aχ 2 -distribution with (3−1)*(3−1) = 4 degrees of freedom.

The value of the test statistic is clearly very high and the null hypothesis is
rejected at all reasonable significance levels. We conclude that the number of
claims depends on the place of living.

\item (ii)
(a) P [small city] = 0.25+0.06+0.02=0.33 
(b) P [more than one claim]=0.04+0.02+0.01=0.07 
(c) P [more than one claim | large city]
0.04
0.04
=
= 0.1212
=
0.23 + 0.06 + 0.04 0.33 
P [countryside | no claim or one claim]
0.27 + 0.06
0.33
=
=
= 0.3548
0.23 + 0.25 + 0.27 + 3*0.06 0.93 
(d)
Page 6Subject CT3  – September 2016 
(e)
P [small city or large city | one claim or more than one claim]
2*0.06 + 0.04 + 0.02
0.18
=
= 0.72
=

3*0.06 + 0.04 + 0.02 + 0.01 0.25
[Total 15]
Generally well answered. There were some calculation errors in part(i), while
in part(ii) not all steps were always shown clearly.
%%%%%%%%%%%%%%%%%%%%%%%%%%%%%%%%%%%%%%%%%%%%%%%%%

Q8
\item (i)
\beta ˆ =
S xy
S_{xx}
=
89.91
= 1.108
81.15

a ˆ = y − \beta ˆ x = 7.15 − 1.108 × 5.93 = 0 .58 
Fitted model is: y ˆ = 0.58 + 1.108 x 
\item (ii) A = 1, B = 8, C = 99.61/1 = 99.61, D = 21.63/8 = 2.704 
(iii) \hat{\sigma} 2 = 21.63 / 8 = 2.7 0 4 
(iv) (a) R 2 gives the proportion of the total variation of y that is explained by x.

(b) R 2 = 99.61/121.24 = 0.822
(v)
We want to test H 0 : \beta= 0 v. H 1 : \beta ≠ 0


Under H 0 the value MS REG / MS RES = 99.61/2.704 = 36.84 should be a value

from the F 1,8 distribution.
The 0.99 quantile of F 1,8 is 11.26

We have strong evidence to reject H 0 and we conclude that there is linear relationship between x and y.

[Total 13]
Generally very well answered. However, many candidates struggled with the
interpretation of the coefficient of determination in part (iv)(a) – this is an
important (and very widely used) concept in regression analysis.
Page 7Subject CT3  – September 2016 

\end{document}
