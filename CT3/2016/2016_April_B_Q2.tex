
\documentclass[a4paper,12pt]{article}

%%%%%%%%%%%%%%%%%%%%%%%%%%%%%%%%%%%%%%%%%%%%%%%%%%%%%%%%%%%%%%%%%%%%%%%%%%%%%%%%%%%%%%%%%%%%%%%%%%%%%%%%%%%%%%%%%%%%%%%%%%%%%%%%%%%%%%%%%%%%%%%%%%%%%%%%%%%%%%%%%%%%%%%%%%%%%%%%%%%%%%%%%%%%%%%%%%%%%%%%%%%%%%%%%%%%%%%%%%%%%%%%%%%%%%%%%%%%%%%%%%%%%%%%%%%%

\usepackage{eurosym}
\usepackage{vmargin}
\usepackage{amsmath}
\usepackage{graphics}
\usepackage{epsfig}
\usepackage{enumerate}
\usepackage{multicol}
\usepackage{subfigure}
\usepackage{fancyhdr}
\usepackage{listings}
\usepackage{framed}
\usepackage{graphicx}
\usepackage{amsmath}
\usepackage{chngpage}

%\usepackage{bigints}
\usepackage{vmargin}

% left top textwidth textheight headheight

% headsep footheight footskip

\setmargins{2.0cm}{2.5cm}{16 cm}{22cm}{0.5cm}{0cm}{1cm}{1cm}

\renewcommand{\baselinestretch}{1.3}

\setcounter{MaxMatrixCols}{10}

\begin{document}

%%%%%%%%%%%%%%%%%%%%%%%%%%%%%%%%%%%%%%%%%%%%%%%%%%%%%%%%%%%%%%%%%%%%%%%%%%%%%%%%%%%


Consider two random variables X and Y.
(i)
Write down the precise mathematical definition for the correlation coefficient
(X, Y) between X and Y.
[1]
Assume now that Y = aX + b where a < 0 and   b   .
(ii)
3
Determine the value of the correlation coefficient (X, Y).
[3]
[Total 4]

%%%%%%%%%%%%%%%%%%%%%%%%%%%%%%%%%%%%%%%%%%%%%%%%%%%%%%%%%%%%%%%%%%%%%%%%%%5

Q2
Cov( X , Y )
V ( X ) V ( Y )
(i) ρ ( X , Y ) = (ii) Cov(X, aX + b) = aV(X) [1]
V(Y) = V(aX + b) = a 2 V(X) [1]
For a < 0 we obtain ρ ( X , Y ) =
[1]
Cov( X , Y )
aV ( X )
=
= − 1
[1]
V ( X ) V ( Y )
V ( X ) a 2 V ( X )
[TOTAL 4]
Part (i) was generally well answered. Some candidates gave the sample
correlation coefficient, which is not what the question required.
Performance in part (ii) was not very competent, with many candidates
failing to recognise the negative correlation.

\end{document}
