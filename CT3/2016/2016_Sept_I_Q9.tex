\documentclass[a4paper,12pt]{article}

%%%%%%%%%%%%%%%%%%%%%%%%%%%%%%%%%%%%%%%%%%%%%%%%%%%%%%%%%%%%%%%%%%%%%%%%%%%%%%%%%%%%%%%%%%%%%%%%%%%%%%%%%%%%%%%%%%%%%%%%%%%%%%%%%%%%%%%%%%%%%%%%%%%%%%%%%%%%%%%%%%%%%%%%%%%%%%%%%%%%%%%%%%%%%%%%%%%%%%%%%%%%%%%%%%%%%%%%%%%%%%%%%%%%%%%%%%%%%%%%%%%%%%%%%%%%
  \usepackage{eurosym}
\usepackage{vmargin}
\usepackage{amsmath}
\usepackage{graphics}
\usepackage{epsfig}
\usepackage{enumerate}
\usepackage{multicol}
\usepackage{subfigure}
\usepackage{fancyhdr}
\usepackage{listings}
\usepackage{framed}
\usepackage{graphicx}
\usepackage{amsmath}
\usepackage{chngpage}
%\usepackage{bigints}
\usepackage{vmargin}

% left top textwidth textheight headheight

% headsep footheight footskip

\setmargins{2.0cm}{2.5cm}{16 cm}{22cm}{0.5cm}{0cm}{1cm}{1cm}

\renewcommand{\baselinestretch}{1.3}

\setcounter{MaxMatrixCols}{10}

\begin{document}


PLEASE TURN OVER9
A statistical model is used to describe the total loss, S (in pounds), experienced in a
certain portfolio of an insurance company over a period of one year. The total loss is
given by:
S  X 1  X 2    X N
where X i gives the size of the loss from claim i =1,..., N. N is a random variable
representing the number of claims per year and follows a Poisson distribution. The
Xi s are independent, identically distributed according to a gamma distribution with
parameters α and λ, and are also independent of N.
Data from previous years show that the average number of claims per year was 14,
while the average size of claims was £500 and their standard deviation was £150.
\begin{enumerate}
\item (i) Estimate the parameters α and λ using the method of moments.
[4]
\item
(ii) Estimate the mean and the variance of the total loss S using the information from the data above.

Now suppose that the value of parameter α is known to be equal to α* and
n = 5 claims have been made in a particular year with average size again £500.

\item (iii)
(a) Derive an expression for the maximum likelihood (ML) estimate of the parameter λ in terms of α*. You should verify that your answer corresponds to a maximum.
(b) Derive the asymptotic distribution of the ML estimator of the parameter  in terms of α*.
(c) Comment on the validity of the distribution in part (iii)(b).
[9]
Now suppose that the values of both parameters α and λ are unknown and n claims
have been made in a particular year.
\item (iv)
(a)
Show that the ML estimate, ̂ of the parameter α needs to satisfy the
equation:
n
Γ  (  ˆ )
 log( x i )
log(  ˆ ) 
 log( x )  i  1
Γ(  ˆ )
n
where Γ  (  ˆ ) denotes the first derivative of Γ(  ˆ ) with respect to ̂ .
(b)
CT3 S2016–6
Comment on how the ML estimates of the parameters α and λ can be
obtained in this case.
\end{enumerate}
\newpage
%%%%%%%%%%%%%%%%%%%%%%%%%%%%%%%%%%%%%%%%%%%%%%%%%%%%%%%%%%%%%%%
Q9
\begin{itemize}
\item (i)
For MME we want:


α
α
E ( X ) = x   = x and V ( X ) = s 2  2 = s 2

λ
λ

These give
\item (ii)
2
 = x =500/150 2 = 0.022 and α
 = x = 500 2 /150 2 = 11.111
 = x λ
λ
s 2
s 2 
E(S) = E(N)E(X) = 14*500 = 7,000 
V(S) = E(N)V(X) + V(N)E(X) 2 = 14*(150 2 + 500 2 ) = 3,815,000 
n
*
 α *

λ
λ n α
α * − 1 −λ x i 

L ( λ ; x ) = ∏
x
e
=
* i

 
*
i = 1  Γ α

 Γ α
n
(iii)
(a)
( )
{ ( ) }
n
e
−λ  x i n
1
*
∏ x i α − 1

1
n
l ( λ ) = n α log ( λ ) − λ  x i + constant 
n
d
n α *
α * α *
l ( λ ) = 0 
=  x i  λ ˆ =
=
5 00
d λ
x
λ ˆ
1 
*
1
We can confirm that this is a maximum as
(b)
(
Page 8
n α *
d λ λ 2
l λ =−
2 ( )
< 0 .
 d 2

CRLB = − 1/ E  2 l ( λ )  = λ 2 / ( n α * )
 d λ



So, approximately, λ ˆ ~ N λ , λ 2 / 5 α *
\item (iv)
d 2


)

(c) The approximation of the distribution relies on a very small sample
(n = 5) and therefore may not be valid.

(a) We now have
n n
i = 1 i = 1
l ( α , λ ) = n α log ( λ ) − n log ( Γ ( α ) ) + ( α − 1 )  log ( x i ) − λ  x i 
α
From part (iii): λ ˆ = , and
x 

%%---- Subject CT3 (Probability and Mathematical Statistics Core Technical) – September 2016 – Examiners’ Report
Γ ′ ( α ˆ ) n
d
l ( α , λ ) = 0  n log λ ˆ − n
+ log ( x i ) = 0
d α
Γ ( α ˆ ) 
i = 1
()

Γ ′ ( α ˆ ) n
 α ˆ 
 n log   − n
+ log ( x i ) = 0
Γ ( α ˆ ) 
 x 
i = 1
Γ ′ ( α ˆ )
 log ( x i )
 log ( α ˆ ) −
= log ( x ) − i = 1
n
Γ ( α ˆ )
n
(b)

The equation for α̂ cannot be solved analytically, so a numerical solution is required. Then α̂ can be substituted in the equation for λ̂ .

[Total 21]
Parts (i), (ii) and (iii)(a) were very well answered, while performance in part (iii)(b) was mixed. In part (iii)(c) many candidates failed to comment on the effect of the small sample size on the approximation. Many answers to
part (iv) were weaker. This is a maximum likelihood derivation, requiring moderate calculus and algebra skills.
\end{itemize}
\end{document}
