\documentclass[a4paper,12pt]{article}

%%%%%%%%%%%%%%%%%%%%%%%%%%%%%%%%%%%%%%%%%%%%%%%%%%%%%%%%%%%%%%%%%%%%%%%%%%%%%%%%%%%%%%%%%%%%%%%%%%%%%%%%%%%%%%%%%%%%%%%%%%%%%%%%%%%%%%%%%%%%%%%%%%%%%%%%%%%%%%%%%%%%%%%%%%%%%%%%%%%%%%%%%%%%%%%%%%%%%%%%%%%%%%%%%%%%%%%%%%%%%%%%%%%%%%%%%%%%%%%%%%%%%%%%%%%%
  \usepackage{eurosym}
\usepackage{vmargin}
\usepackage{amsmath}
\usepackage{graphics}
\usepackage{epsfig}
\usepackage{enumerate}
\usepackage{multicol}
\usepackage{subfigure}
\usepackage{fancyhdr}
\usepackage{listings}
\usepackage{framed}
\usepackage{graphicx}
\usepackage{amsmath}
\usepackage{chngpage}
%\usepackage{bigints}
\usepackage{vmargin}

% left top textwidth textheight headheight

% headsep footheight footskip

\setmargins{2.0cm}{2.5cm}{16 cm}{22cm}{0.5cm}{0cm}{1cm}{1cm}

\renewcommand{\baselinestretch}{1.3}

\setcounter{MaxMatrixCols}{10}

\begin{document}

\begin{enumerate}
[Total 21]10
A randomised clinical trial was conducted with the aim of investigating the
effectiveness of two drugs (A and B) for tackling stage-fright in musicians before a
performance. Ten musicians were allocated to each of two groups: group A (taking
drug A) and group B (taking drug B). A further 10 musicians were allocated to a third
group C, which served as a control group with the musicians not taking any drug. All
group allocations were random and the musicians did not know which treatment they
received.
At the end of the performance all 30 musicians were asked to give a score indicating
their stage-fright, on a scale 1–5, with a score of 1 implying “not nervous at all” and a
score of 5 implying “extremely nervous”. The scores were then transformed by
taking their logarithm (values denoted by y), and the results are shown below:
Group
A
B
C
\sumy
0.693 1.099
0
1.099 0.693 0.693
0
1.099 1.099
0
0
0.693 0.693 1.386
0
0.693 1.099 0.693
0
1.099
0
1.609 1.099 1.386 1.099 1.609 0.693 1.099 1.609 1.386
6.475
6.356
11.589
For these data: SS_{T}= 8.364, SS_{B} = 1.785, SS R = 6.579 (as defined in page 26 of the
Formulae and Tables).
\begin{enumerate}
\item (i) Perform an analysis of variance to test the null hypothesis that the mean level
of stage-fright is the same among the three groups.

\item (ii) Determine the residuals for the first musician in groups A and B using the
fitted model in part (i).

\item
(iii) Determine a 95\% confidence interval for the variance of the scores (on the
logarithmic scale), based on:
(a)
(b)
the residual sum of squares.
the sum of squares between treatments.
\item 
(iv)
Comment on the interval obtained in part (iii)(b).

It is suggested that any differences in the scores could be explained by the difference
between the scores of the control group and the groups receiving a stage-fright drug.
\item (v)
Suggest how this effect can be formally tested. You should not carry out any
test.

[Total 17]
\end{enumerate}
CT3 S2016–7
%%%%%%%%%%%%%%%%%%%%%%%%%%%%%%%%%%%%%%%%%%%%%%%%%%%%%%%%%%%%%%%%%%%%%%%%%%%%
Q10
\item (i)
ANOVA table:
\begin{verbatim}
Source of variation df SS MSS
Between groups
Residual
Total 2
27
29 1.785
6.579
8.364 0.8925
0.2437

F =
0.8925
= 3.662 on 2, 27 df
0.2437
\end{verbatim
F 2,27 (5%) = 3.354, F 2,27 (1%) = 5.488, so reject the null hypothesis at the 5%
level.

There is evidence against the null hypothesis. We conclude that there are
differences in the mean level of nervousness scores among the three groups.

\item (ii)
Residuals given as:
r A 1 = y A 1 − y A = 0.693 −
6.475
= 0.0455
10
%%%%%%%%%%%%%%%%%%%%%%%
r B 1 = y B 1 − y B = 0 −
(iii)
(a)
6.356
= − 0.6356
10
SSR
Interval will be based on
σ
2

~\chi^2 227

and therefore a 95% CI is given as
SSR
 2

2
\chi^2 27 ( 0.025 ) ≤ 2 ≤\chi^2 27 ( 0.975 ) 
σ


(b)


SSR
SSR
i.e.  2
≤ \sigma2 ≤ 2

\chi^2 ( 0.975 )

χ
0.025
(
)
27
 27
 
6.579 
 6.579
≤ \sigma2 ≤
which gives 
 , i.e. (0.1523, 0.4515)
14.57 
 43.19 
Interval now based on
SSB
σ
2
~\chi^2 22

Working similarly as above we obtain a 95% CI as:
1.785 
 1.785
≤ \sigma2 ≤

 , i.e. (0.2419, 35.2488)
0.05064 
 7.378

(iv) This interval is too wide. Notice that its validity depends on H_0 being true,
which is rejected at 5% level.

(v) We could perform a two-sample t-test of control mean = treatment mean by
combining the data for the two treatment groups (and using samples of sizes
10 and 20).

[Total 17]
Part \item (i) was well answered. However, performance in the remaining part of the
question was not strong. Part \item (ii) involves basic understanding of the concept
of residuals, and is not examined often. Parts (iii) and (iv) require deeper
understanding of the construction and assumptions behind confidence
intervals, while part (v) examines a more rounded knowledge and
understanding of statistical testing.
END OF EXAMINERS’ REPORT
Page 10
