\documentclass[a4paper,12pt]{article}

%%%%%%%%%%%%%%%%%%%%%%%%%%%%%%%%%%%%%%%%%%%%%%%%%%%%%%%%%%%%%%%%%%%%%%%%%%%%%%%%%%%%%%%%%%%%%%%%%%%%%%%%%%%%%%%%%%%%%%%%%%%%%%%%%%%%%%%%%%%%%%%%%%%%%%%%%%%%%%%%%%%%%%%%%%%%%%%%%%%%%%%%%%%%%%%%%%%%%%%%%%%%%%%%%%%%%%%%%%%%%%%%%%%%%%%%%%%%%%%%%%%%%%%%%%%%
  \usepackage{eurosym}
\usepackage{vmargin}
\usepackage{amsmath}
\usepackage{graphics}
\usepackage{epsfig}
\usepackage{enumerate}
\usepackage{multicol}
\usepackage{subfigure}
\usepackage{fancyhdr}
\usepackage{listings}
\usepackage{framed}
\usepackage{graphicx}
\usepackage{amsmath}
\usepackage{chngpage}
%\usepackage{bigints}
\usepackage{vmargin}

% left top textwidth textheight headheight

% headsep footheight footskip

\setmargins{2.0cm}{2.5cm}{16 cm}{22cm}{0.5cm}{0cm}{1cm}{1cm}

\renewcommand{\baselinestretch}{1.3}

\setcounter{MaxMatrixCols}{10}

\begin{document}

\begin{enumerate}
5
For each of two life insurance companies, A and B, a random sample of 150 policies
is examined. The number of policies which have given rise to claims in the past year
is 45 for company A and 33 for company B.
Test the null hypothesis that the underlying proportions of policies which have given
rise to claims in the past year are equal for the two companies.
[4]
6
Let X and Y be random variables with joint probability distribution:
  kx 2 y 2 , 0  x  y  1
f XY ( x , y )  
otherwise
 0,
where k is a constant.
(i) Show that k = 18. [4]
(ii) Determine f Y (y), the marginal density function of Y. 
(iii) Determine P(X > 0.5Y = 0.75).
CT3 S2016–3

[Total 9]
%%%%%%%%%%%%%%%%%%%%%%%%%%%%%%%%%%%%%%%%%%%%%%%
Q5
45 = 0.3
Company A: n 1 = 150, and θ ˆ 1 = 150
33 = 0.22
Company B: n 2 = 150, and θ ˆ 2 = 150
Combined:
n = 300, and θ ˆ =
The test statistic is:
45 + 33
300

= 0.26
θ ˆ 1 − θ ˆ 2
0.3 − 0.22
=
= 1.58
1 
 1
θ ˆ (1 − θ ˆ ) θ ˆ (1 − θ ˆ )
+
0.26(1 − 0.26) 
+

 150 150 
n 1
n 2

The significance probability of the test of H 0 : θ 1 = θ 2 against a two sided alternative is
2*P(Z> 1.58) = 2*(1 − 0.94295) = 0.114.

(Or, compare with the 97.5 percentile (1.96) of the normal distribution.)
Page 4Subject CT3 (Probability and Mathematical Statistics Core Technical) – September 2016 – Examiners’ Report
Therefore there is insufficient evidence to reject the null hypothesis, and we conclude
that there is no difference between the underlying proportions.

[Total 4]
Mixed performance. The question was answered competently by well
prepared candidates, but there were errors mainly concerning the use of a
common θ estimate in the denominator. Also, note that this is a 2-sided test.
Q6
 f XY
(i)
xy
11
y = 1
1
k
( x , y ) dydx =  kx y dydx =    x 2 y 3   dx
 3
 y = x
0 x
0
2 2

1
1
k 2 5
k  x 3 x 6 
k  1 1  k
=  x − x dx =  −  =  −  =
3
3   3
6  
3  3 6  18
0
0 
Want integral equal to 1  k = 18 
y
x = y
f Y ( y ) =  f XY ( x , y ) dx =  18 x 2 y 2 dx =  6 x 3 y 2 
= 6 y 5

 x = 0
(ii)
x
(iii)
0
P ( X > 0.5| Y = 0.75 ) =
0.75

0.5
0.75

f ( x | Y = 0.75 ) ( x ) dx =
0.75

f XY ( x , 0.75 ) / f Y ( 0.75 ) dx 
0.5
0.75
3
3
 4   x 
=  18 x 0.75 / (6 × 0.75 ) dx = 3 ×    
= 0.7037
3    3  

0.5
0.5
2
2
5

[Total 9]
Parts (i) and (ii) were very well answered. Dealing with the conditional
distribution in part (iii) was problematic.
