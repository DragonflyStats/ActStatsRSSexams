
\documentclass[a4paper,12pt]{article}

%%%%%%%%%%%%%%%%%%%%%%%%%%%%%%%%%%%%%%%%%%%%%%%%%%%%%%%%%%%%%%%%%%%%%%%%%%%%%%%%%%%%%%%%%%%%%%%%%%%%%%%%%%%%%%%%%%%%%%%%%%%%%%%%%%%%%%%%%%%%%%%%%%%%%%%%%%%%%%%%%%%%%%%%%%%%%%%%%%%%%%%%%%%%%%%%%%%%%%%%%%%%%%%%%%%%%%%%%%%%%%%%%%%%%%%%%%%%%%%%%%%%%%%%%%%%
  \usepackage{eurosym}
\usepackage{vmargin}
\usepackage{amsmath}
\usepackage{graphics}
\usepackage{epsfig}
\usepackage{enumerate}
\usepackage{multicol}
\usepackage{subfigure}
\usepackage{fancyhdr}
\usepackage{listings}
\usepackage{framed}
\usepackage{graphicx}
\usepackage{amsmath}
\usepackage{chngpage}
%\usepackage{bigints}
\usepackage{vmargin}

% left top textwidth textheight headheight

% headsep footheight footskip

\setmargins{2.0cm}{2.5cm}{16 cm}{22cm}{0.5cm}{0cm}{1cm}{1cm}

\renewcommand{\baselinestretch}{1.3}

\setcounter{MaxMatrixCols}{10}

\begin{document}

\begin{enumerate}

%%-- CT3 A2016–5 PLEASE TURN OVER
\item 9 An insurance company is examining its claims and reserving history over the last ten years. Ten years ago the data in one year gave the following number of policies with a given
number of claims:
  Number of Claims 0 1 2 3 4
Number of Policies 871 117 5 5 2
The company assumes that claims occur independently of each other and at a constant rate.
\begin{enumerate}
\item (i) Estimate the rate of claims for an individual policy. [2]
In the most recent year the following data were obtained:
  Number of Claims 0 1 2 3 4
Number of Policies 1850 140 5 3 2
\item (ii) Perform a goodness-of-fit test to investigate if the number of claims in the new data follow a Poisson distribution with the same rate as the rate estimated from the old data. [6]
To date the company has believed that the size of claims, xi for claim i, was independent of the policyholders’ claims history. It now wishes to investigate that belief. It splits the policyholders who have made claims in the most recent year into
those with no claims in the preceding five years and those with at least one claim.
The total amount of claims in the most recent year is given below:
  Number of policies xi 2
xi
No claims in previous 5 years 70 6.42m 8.76  1011
Claim in previous 5 years 80 9.22m 1.52  1012
\item (iii) Perform a test of the null hypothesis that the mean claim amount per policy in each group is equal against the alternative that the mean claim amount per
policy is not the same. Use a 5\% significance level. [8]
The company wishes to budget for next year. It estimates it will write 2,200 policies, of which half will be for policies with no claims in the last 5 years. It assumes that claims occur at a rate of 0.1 per policy per annum. It also assumes that claim amounts
will average £94,000 for policies with no claims in the last 5 years and £120,000 otherwise. Assume that the standard deviation of a claim from a randomly chosen policy is £70,000.
\item (iv) Determine the estimated mean and variance of the total amount of claims next year. [5]
\end{enumerate}

%%%%%%%%%%%%%%%%%%%%%%%%%%%%%%%%%%%%%%%%%%%%%%%%%%%%%5
Q9 (i) Total no. of claims 871 fini = ×0+117×1+ 5×2+ 5×3+ 2×4 =150 [1]
Rate = 150/1000 = 0.15 [1]
(ii)
x 0 1 2 3 4 Total
1850 140 5 3 2 2000
P(X≤x) 0.86071 0.98981 0.9995 0.99998 1
P(X=x) 0.86071 0.1291 0.00969 0.00048 2E−05
ex 1721.42 258.2 19.38 0.96 0.04
[2]
So sums >5
x 0 1 2 Total
1850 140 10 2000
P(X≤x) 0.86071 0.98981 1
P(X=x) 0.86071 0.1291 0.01019
ex 1721.42 258.2 20.38
(ax−ex)2/ex 9.60 54.11 5.29 69.00
[1]
So test statistic is 69.00 on 2d.f.Compared to 22
χ clearly significant at any
reasonable level, so reject H0 that number of claims follows Poisson(0.15). [3]
(iii) Let Y=claims from group with no claims in last 5 years, Z= claims from group
with claim in last 5 years
%%--- Subject CT3 (Probability and Mathematical Statistics Core Technical) – April 2016 – Examiners’ Report
Page 8
11 6420000)2
2 70 9
8.76 10 ( 6, 420,000 91714, 4.162 10
          70 y 69 y s
          × −
          = = = = × [2]
          12 9220000)2
2 80 9
1.52 10 ( 9, 220,000 115250, 5.790 10
          80 z 79 z s
          × −
          = = = = × [2]
          Under assumption of equal variances, pooled variance is
          2 (69*4.162 109 79*5.790 109 ) / (69 79) 5.031 109 sp = × + × + = × [1]
          Test statistic = 9 1 1
          70 80
          (91714 115250) / 5.031 10 *  2.027  + 
           
          − × = − [1]
          t148;0.975 is between 1.96 and 1.98 so reject H0 that means are the same [2]
          (iv) Overall number of claims for each category = 2200*0.1*0.5 =110 [
          Mean of total amount:
            E(S) = E(S1) + E(S2 ) = E(N1)× E(X1) + E(N2 )× E(X2 )
          =110 ×94,000 +110×120,000 = £23.54m. [2]
          Variance of total amount:
            V (S) =V (S1) +V (S2 )
          2
          = E(N1)×V (X1) +V (N1)×[E(X1)]
          2
          + E(N2 )×V (X2 ) +V (N2 )×[E(X2 )]
          =110(70,0002 + 94,0002 ) +110(70,0002 +120,0002 ) = 3.63396×1012 [2]
          [TOTAL 21]

Parts (i) and (iii) were well answered. In part (ii) there was a wide range of  answers, some involving errors in calculations. For full marks, categories with
          small expected frequencies must be combined. Some candidates performed
          a test of equality between the two rates, which is not what the question
          requires here.
          Part (iv) was not answered competently. Most candidates gave a correct
          answer for the mean, but many did not attempt to calculate the variance at all
          – or gave a wrong answer. Note here that if a common variance is assumed,
          then V(S) = 2*110*70,0002 = £21.078×1012 .
          A small number of candidates worked along these lines, and full credit was
          given if this answer was provided.
  %%-        Subject CT3 (Probability and Mathematical Statistics Core Technical) – April 2016 – Examiners’ Report
  \end{document}
