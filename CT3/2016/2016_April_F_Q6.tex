\documentclass[a4paper,12pt]{article}

%%%%%%%%%%%%%%%%%%%%%%%%%%%%%%%%%%%%%%%%%%%%%%%%%%%%%%%%%%%%%%%%%%%%%%%%%%%%%%%%%%%%%%%%%%%%%%%%%%%%%%%%%%%%%%%%%%%%%%%%%%%%%%%%%%%%%%%%%%%%%%%%%%%%%%%%%%%%%%%%%%%%%%%%%%%%%%%%%%%%%%%%%%%%%%%%%%%%%%%%%%%%%%%%%%%%%%%%%%%%%%%%%%%%%%%%%%%%%%%%%%%%%%%%%%%%

\usepackage{eurosym}
\usepackage{vmargin}
\usepackage{amsmath}
\usepackage{graphics}
\usepackage{epsfig}
\usepackage{enumerate}
\usepackage{multicol}
\usepackage{subfigure}
\usepackage{fancyhdr}
\usepackage{listings}
\usepackage{framed}
\usepackage{graphicx}
\usepackage{amsmath}
\usepackage{chngpage}

%\usepackage{bigints}
\usepackage{vmargin}

% left top textwidth textheight headheight

% headsep footheight footskip

\setmargins{2.0cm}{2.5cm}{16 cm}{22cm}{0.5cm}{0cm}{1cm}{1cm}

\renewcommand{\baselinestretch}{1.3}

\setcounter{MaxMatrixCols}{10}

\begin{document}


A statistician is sent a summary of some data. She is told that the sample mean is
9.46 and the sample variance is 25.05. She decides to fit a continuous uniform
distribution to the data.
(i)
Estimate the parameters of the distribution using the method of moments. [4]
The full data are sent later and are given below:
3.5 5.4 7.3 8.5 9.2 10.3 11.4 20.1
(ii)
CT3 A2016–3
Comment on the results in part (i) in the light of the full data.
[2]
[Total 6]



%%%%%%%%%%%%%%%%%%%%%%%%%%%%%%%%%%%%
6
Q6
(i)
(ii)
E[X] = (a + b) / 2 b = 2E[X] −a
Var( X ) = ( b − a ) 2 /12 = (2 E [ X ] − 2 a ) 2 /12 = ( E [ X ] − a ) 2 / 3. [2]
a ˆ = x − 3 s = 0.791  b ˆ = 2* x − a ˆ = 18.12 9 . [2]
The largest observation is greater than our estimate of b in part (i). This would
suggest the uniform distribution is not a good fit to this data, or the largest
observation is a mistaken observation. This also highlights a potential
weakness of the method of moments.
[2]
[TOTAL 6]
Part (i) was very well answered. In part (ii) some candidates failed to
recognise that there is zero probability of having a sample value outside the
range given by the parameters.

\end{document}
