\documentclass[a4paper,12pt]{article}

%%%%%%%%%%%%%%%%%%%%%%%%%%%%%%%%%%%%%%%%%%%%%%%%%%%%%%%%%%%%%%%%%%%%%%%%%%%%%%%%%%%%%%%%%%%%%%%%%%%%%%%%%%%%%%%%%%%%%%%%%%%%%%%%%%%%%%%%%%%%%%%%%%%%%%%%%%%%%%%%%%%%%%%%%%%%%%%%%%%%%%%%%%%%%%%%%%%%%%%%%%%%%%%%%%%%%%%%%%%%%%%%%%%%%%%%%%%%%%%%%%%%%%%%%%%%
  \usepackage{eurosym}
\usepackage{vmargin}
\usepackage{amsmath}
\usepackage{graphics}
\usepackage{epsfig}
\usepackage{enumerate}
\usepackage{multicol}
\usepackage{subfigure}
\usepackage{fancyhdr}
\usepackage{listings}
\usepackage{framed}
\usepackage{graphicx}
\usepackage{amsmath}
\usepackage{chngpage}
%\usepackage{bigints}
\usepackage{vmargin}

% left top textwidth textheight headheight

% headsep footheight footskip

\setmargins{2.0cm}{2.5cm}{16 cm}{22cm}{0.5cm}{0cm}{1cm}{1cm}

\renewcommand{\baselinestretch}{1.3}

\setcounter{MaxMatrixCols}{10}

\begin{document}

%%  Institute and Faculty of Actuaries1
Consider the following sample with 20 observations x i :
1 1 5 7 9 11 11 14 14 19
20 21 23 28 28 31 39 41 43 47
20
 x i  413 and
i  1
2
3
20
 x i 2  12,311
i  1

\begin{enumerate}
\item (i) Calculate the mean of this sample. 
\item (ii) Calculate the standard deviation of this sample. 
\item (iii) Calculate the median of this sample. 
\item (iv) Calculate the interquartile range of this sample.
\end{enumerate}


Solutions
Q1
%%%%%%%%%%%%%%%%%%%%%%%%%%%%%%%%%%%%%%%%%%%%%%%%%%%%%%%%%%%%%%%%
Solutions
Q1
Mean = 413 / 20 = 20.65
(i)
1 
 413 
 12,311 − 20 * 

19  
 20 
(ii) SD =
(iii) Median = 19.5
(iv) IQR = 29.5−10 = 19.5

2 
 = 14.11
 



[Total 6]
Very well answered. Note that the Core Reading mentions two different ways
for calculating the quartiles in part (iv). Both ways were given full credit when
applied correctly.
