\documentclass[a4paper,12pt]{article}

%%%%%%%%%%%%%%%%%%%%%%%%%%%%%%%%%%%%%%%%%%%%%%%%%%%%%%%%%%%%%%%%%%%%%%%%%%%%%%%%%%%%%%%%%%%%%%%%%%%%%%%%%%%%%%%%%%%%%%%%%%%%%%%%%%%%%%%%%%%%%%%%%%%%%%%%%%%%%%%%%%%%%%%%%%%%%%%%%%%%%%%%%%%%%%%%%%%%%%%%%%%%%%%%%%%%%%%%%%%%%%%%%%%%%%%%%%%%%%%%%%%%%%%%%%%%

\usepackage{eurosym}
\usepackage{vmargin}
\usepackage{amsmath}
\usepackage{graphics}
\usepackage{epsfig}
\usepackage{enumerate}
\usepackage{multicol}
\usepackage{subfigure}
\usepackage{fancyhdr}
\usepackage{listings}
\usepackage{framed}
\usepackage{graphicx}
\usepackage{amsmath}
\usepackage{chngpage}

%\usepackage{bigints}
\usepackage{vmargin}

% left top textwidth textheight headheight

% headsep footheight footskip

\setmargins{2.0cm}{2.5cm}{16 cm}{22cm}{0.5cm}{0cm}{1cm}{1cm}

\renewcommand{\baselinestretch}{1.3}

\setcounter{MaxMatrixCols}{10}

\begin{document}
%%%%%%%%%%%%%%%%%%%%%%%%%%%%%%%%%%%%%%%%%%%%%%%%%%%%%%%%%%%%%%%%%%%%%%%%%
\item  % Question 3 

A random variable Y has probability density function
f(y) = 1 ,
y

y > 1
where  > 0 is a parameter.
\begin{enumerate}[(i)]
\item  Show that the probability density function of $Z = \ln(Y)$ is given by ez and determine its range.
\item State the distribution of Z identifying any parameters involved. 
\end{enumerate}

\end{enumerate}
%%%%%%%%%%%%%%%%%%%%%%%%%%%%%%%%%%%%%%%%%%%%%%%%%%%%%%%%%%%%%%%%%%%%%%%%%%%%%%%%%%%%%%%%%%%%%%%5
Q3 (i) z = log yy = ez∴ dy
dz
= ez
f(y) = y\theta+1
\theta
∴f(z) = \theta(ez)−(\theta+1) ez = \thetae−\thetaz [1½]
and since y> 1 z> 0. 

(ii) This is the probability density function of an exponential distribution with
parameter \theta. 
[OR: Obtain using the distribution function:
\[    FZ(z) = Pr(Z≤z) = Pr(Y≤ez) = FY(ez)\]
  Also, 1 1
  1
  ( ) 1
  y y
  FY y du u y
  u
  −\theta −\theta
  \theta+
    = \theta =−   = −   
  Thus, FZ(z) = 1−e−\thetaz implying an Exp(\theta) distribution.]

Part (i) required some work involving probabilistic arguments and was not well
answered. There were no problems with part (ii).
\end{document}
