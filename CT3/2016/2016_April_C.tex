\documentclass[a4paper,12pt]{article}

%%%%%%%%%%%%%%%%%%%%%%%%%%%%%%%%%%%%%%%%%%%%%%%%%%%%%%%%%%%%%%%%%%%%%%%%%%%%%%%%%%%%%%%%%%%%%%%%%%%%%%%%%%%%%%%%%%%%%%%%%%%%%%%%%%%%%%%%%%%%%%%%%%%%%%%%%%%%%%%%%%%%%%%%%%%%%%%%%%%%%%%%%%%%%%%%%%%%%%%%%%%%%%%%%%%%%%%%%%%%%%%%%%%%%%%%%%%%%%%%%%%%%%%%%%%%
  \usepackage{eurosym}
\usepackage{vmargin}
\usepackage{amsmath}
\usepackage{graphics}
\usepackage{epsfig}
\usepackage{enumerate}
\usepackage{multicol}
\usepackage{subfigure}
\usepackage{fancyhdr}
\usepackage{listings}
\usepackage{framed}
\usepackage{graphicx}
\usepackage{amsmath}
\usepackage{chngpage}
%\usepackage{bigints}
\usepackage{vmargin}

% left top textwidth textheight headheight

% headsep footheight footskip
\setmargins{2.0cm}{2.5cm}{16 cm}{22cm}{0.5cm}{0cm}{1cm}{1cm}

\renewcommand{\baselinestretch}{1.3}

\setcounter{MaxMatrixCols}{10}

\begin{document}

\begin{enumerate}
%%-- Question 7 
\item A random sample is taken from an exponential distribution with parameter $\lambda$ . The
sample contains some censored observations for which we only know that the value is
greater than 3. The observed values are given in the following table:
  i 1 2 3 4 5 6 7 8 9 10
xi 1.3 1.8 2.1 2.2 2.2 2.4 >3 >3 >3 >3

Estimate the parameter λ using the method of maximum likelihood. You are not required to verify that your answer corresponds to the maximum. [4]
%%%%%%%%%%%%%%%%%%%%%%%%%%%%%%%%%%%%%%%%%%%%%%%%%%%%%%%%%%%%%%%%%%%%%%%%%%%%%%%%%%%%%%
%%----Question 8 
\item A scientist is comparing how productive three new strains of wheat are. Thirty widely spread plots of equal size are chosen randomly. Ten plots are planted with each strain and the weight of wheat produced in each plot is measured. The scientist
wishes to compare the strains using an analysis of variance and produces the
following calculations:
  Sum of squares between strains 55.672
Sum of squares within strains 13.332
\begin{enumerate}[(i)]
\item Perform an analysis of variance on the data. 
\item  (a) Determine the width of a 95\% confidence interval for the difference
between any two of the mean weights produced for each strain. 
\end{enumerate}
%%%%%%%%%%%%%%%%%%%%%%%%%%%%%%%%%%%%%%%%%%%%%%%%%%%
The mean weight produced for each sample is:
  Strain 1 2 3
Mean 2.03 4.42 5.24
(b) State which means are significantly different. 
The scientist discovers that in practice only ten plots in total were chosen, with a third of each planted with each strain.
(iii) Comment on what effect this discovery might have on the scientist’s original analysis above. 

%%%%%%%%%%%%%%%%%%%%%%%%%%%%%%%%%%%
Q7
6
1
6 10
6 3 4
1 7
( ) [ 3] ( )
i
i i
x
x
i
i i
L e P X e = e
−λ
−λ − λ
= =
      
λ =  λ   >  = λ
  
Π Π [1]
6 6
1 1
( ) 6log( ) i 12 6log( ) i 12 6log( ) 24
i i
l x x
= =
   
λ = λ − λ − λ = λ − λ +  = λ − λ
 
  [1]
l (λ) = 6 − 24
λ
′ [1]
6 0.25
24
ˆλ = = . [1]
[TOTAL 4]
%Mixed performance. Most candidates worked through the maximisation steps, but many could not figure out the part of the likelihood corresponding to the censored information.
%%%%%%%%%%%%%%%%%%%%%%%%%%%%%%%%%%%%%%%%%%%%%%5
\newpage

Q8 (i) Test H0 that means are the same, against H1 that at least one pair different.
SS df MS F
SSB 55.672 2 27.836 56.35
SSR 13.332 27 0.494
SST 69.004 29

[3]
F2,27 = 5.488 at 1% so reject H0 that means are the same. [2]
(ii) (a) ˆ σ2 = 0.494, t27;0.975 = 2.052 [2]
CI width =
  1
2
27,0.975 1
2
2* * * 1 1 2*0.703* 2.052 1.290
10 10
5
σˆ t  +  = =
   
[2]
(b) From part (ii)(a), LSD = 1.290/2 = 0.645. Therefore, all strain means are significantly different from each other. 
%%%%%%%%%%%%%%%%%%%%%%%%%%%%%%%%%%%%%%%%%%%%%%%%%%%%%%%%%%%%%%%%%%%%%%%%%%%%%%%%%%%%%%%%%%
Page 7
(iii) The samples may not be independent of each other and, so, the ANOVA may not be valid. 


%Part (i) was well answered. In part (ii)(a) there were mixed efforts with many partial answers providing intervals rather than the required width. Part (ii)(b) was poorly answered with some answers contradicting the findings in part (a).
%There seemed to be some misinterpretation in part (iii), where a number of candidates referred to a “smaller sample, affecting accuracy of test”, rather than addressing the independence issue. Sensible comments along these lines were given credit.

\end{document}


