\documentclass[a4paper,12pt]{article}

%%%%%%%%%%%%%%%%%%%%%%%%%%%%%%%%%%%%%%%%%%%%%%%%%%%%%%%%%%%%%%%%%%%%%%%%%%%%%%%%%%%%%%%%%%%%%%%%%%%%%%%%%%%%%%%%%%%%%%%%%%%%%%%%%%%%%%%%%%%%%%%%%%%%%%%%%%%%%%%%%%%%%%%%%%%%%%%%%%%%%%%%%%%%%%%%%%%%%%%%%%%%%%%%%%%%%%%%%%%%%%%%%%%%%%%%%%%%%%%%%%%%%%%%%%%%

\usepackage{eurosym}
\usepackage{vmargin}
\usepackage{amsmath}
\usepackage{graphics}
\usepackage{epsfig}
\usepackage{enumerate}
\usepackage{multicol}
\usepackage{subfigure}
\usepackage{fancyhdr}
\usepackage{listings}
\usepackage{framed}
\usepackage{graphicx}
\usepackage{amsmath}
\usepackage{chngpage}

%\usepackage{bigints}
\usepackage{vmargin}

% left top textwidth textheight headheight

% headsep footheight footskip

\setmargins{2.0cm}{2.5cm}{16 cm}{22cm}{0.5cm}{0cm}{1cm}{1cm}

\renewcommand{\baselinestretch}{1.3}

\setcounter{MaxMatrixCols}{10}

\begin{document}
\begin{enumerate}
\item 1
In a sample of 100 households in a specific city, the following distribution of number
of people per household was observed:
Number of people x
Number of households f x
1
7
2
f 2
3
20
4
f 4
5
18
6
10
7
5
The mean number of people per household was found to be 4.0. However, the
frequencies for two and four members per household (f 2 and f 4 respectively) are
missing.
2
3
(i) Calculate the missing frequencies f 2 and f 4 .
[2]
(ii) Find the median of these data, and hence comment on the symmetry of the
data.
[2]
[Total 4]
\item Two tickets are selected at random, one after the other and without replacement, from
a group of six tickets, numbered 1, 2, 3, 4, 5, and 6.
(i) Calculate the probability that the numbers on the selected tickets add up to 8.
[2]
(ii) Calculate the probability that the numbers on the selected tickets differ by
3 or more.
[2]
[Total 4]
\item Let X be a random variable with moment generating function M X (t) and cumulant
generating function C X (t), and let Y = aX + b, where a and b are constants. Let Y have
moment generating function M Y (t) and cumulant generating function C Y (t).
(i) Show that C Y (t) = bt + C X (at).
(ii) Find the coefficient of skewness of Y in the case that M X (t) = (1 – t) –2 and
Y = 3X + 2 (you may use the fact that C Y ′′′ (0) = E[(Y − μ Y ) 3 ]).
[5]
[Total 7]
CT3 S2009—2
[2]4
Let the random variables (X,Y) have the joint probability density function
f X , Y ( x , y ) = exp{ − ( x + y )}, x > 0, y > 0.
(i) Derive the marginal probability density functions of X and Y, and hence
determine (giving reasons) whether or not the two variables are independent.
[3]
(ii) Derive the joint cumulative distribution function F X , Y ( x , y ).
\end{enumerate}
%%%%%%%%%%%%%%%%%%%%%%%%%%%%%%%%%%%%%%%%%%%%%%%%%%%%%%%%%%%%%%%%%%%%%%%%%%%%%%%%%%%%
\newpage
1
(i)
We have 60
These give f 2
f 4 100 and
f 2
40
4 .
1
f 4 and 2 f 2 4 f 4 148
from which we obtain f 2
(ii)
7 2 f 2 60 4 f 4 90 60 35
100
6 and f 4
34 .
1
Median is equal to the midpoint between the 50 th and 51 st ordered
observations, i.e. median = 4. 1
We have mean = median, suggesting that the distribution of these data is
roughly symmetric. 1
2
(i)
With sample space {(i,j), i = 1, ..., 6, j = 1, ..., 6, j
i}
(that is, i is the number on the first ticket selected, j that on the second
selected) there are 30 equally likely outcomes.
Favourable outcomes are (2,6), (3,5), (5,3), (6,2)
so probability = 4/30 = 2/15 = 0.133
(ii)
2
Favourable outcomes are
(1,4), (4,1), (1,5), (5,1), (1,6), (6,1), (2,5), (5,2), (2,6), (6,2), (3,6), (6,3)
so probability = 12/30 = 0.4
2
OR: Use a sample space of size 15: {(i,j)} where i is smaller number selected,
j is larger.
Then event (i) has 2 favourable outcomes and event (ii) has 6.
3
(i)
MY (t) = E[etY ] = E[et(aX+b)] = etbE[eatX ] = ebtMX (at)
CY (t) = log MY (t) = bt + log MX (at) = bt + CX (at)
(ii)
CY (t) = 2t + log(1 – 3t) –2 = 2t – 2log(1 – 3t) 1
CY (t) = 2 + 6(1 – 3t)-1 , CY (t) = 18(1 – 3t) –2 , CY (t) = 108(1 – 3t) –3 2
E[(Y
Page 2
2
μ Y )2] = CY (0) = 18, E[(Y
μ Y )3] = CY (0) = 108Subject CT3 (Probability and Mathematical Statistics Core Technical) — September 2009 — Marking Schedule
coefficient of skewness of Y = 108/183/2 = 21/2 = 1.414
2
OR note that coefficient of skewness of Y = coefficient of skewness of X and
just work with X (some candidates may recognise X ~ Gamma(2,1) and
comment on the formula for coefficient of skewness (2/  ) given in the
Yellow Book).
4
(i)
f X ( x )
f ( x , y ) dy
0
f Y ( y )
e x y
dy e x
e x y
dx e y
e y
e x
0
0
f ( x , y ) dx
0
0
x y
Since f X , Y ( x , y ) e
0
e x . 1
e y [OR by symmetry]. 1
f X ( x ) f Y ( y ) , X and Y are independent.
1
x y
(ii)
F X , Y ( x , y )
e
u v
1
dvdu
00
y
x
u
F X , Y ( x , y )
e v dv
e du
0
e
u x
0
0
e
v y
0
1 e
x
1 e
y
1
