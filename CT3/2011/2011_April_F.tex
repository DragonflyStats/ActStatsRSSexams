\documentclass[a4paper,12pt]{article}

%%%%%%%%%%%%%%%%%%%%%%%%%%%%%%%%%%%%%%%%%%%%%%%%%%%%%%%%%%%%%%%%%%%%%%%%%%%%%%%%%%%%%%%%%%%%%%%%%%%%%%%%%%%%%%%%%%%%%%%%%%%%%%%%%%%%%%%%%%%%%%%%%%%%%%%%%%%%%%%%%%%%%%%%%%%%%%%%%%%%%%%%%%%%%%%%%%%%%%%%%%%%%%%%%%%%%%%%%%%%%%%%%%%%%%%%%%%%%%%%%%%%%%%%%%%%

\usepackage{eurosym}
\usepackage{vmargin}
\usepackage{amsmath}
\usepackage{graphics}
\usepackage{epsfig}
\usepackage{enumerate}
\usepackage{multicol}
\usepackage{subfigure}
\usepackage{fancyhdr}
\usepackage{listings}
\usepackage{framed}
\usepackage{graphicx}
\usepackage{amsmath}
\usepackage{chngpage}

%\usepackage{bigints}
\usepackage{vmargin}

% left top textwidth textheight headheight

% headsep footheight footskip

\setmargins{2.0cm}{2.5cm}{16 cm}{22cm}{0.5cm}{0cm}{1cm}{1cm}

\renewcommand{\baselinestretch}{1.3}

\setcounter{MaxMatrixCols}{10}

\begin{document}
\begin{enumerate}
%- 
% - 10
\item A life insurance company runs a statistical analysis of mortality rates. The company considers a population of 100,000 individuals. It assumes that the number of deaths X during one year has a Poisson distribution with expectation E[X] = \mu. Over four years
the company has observed the following realisations of X (number of deaths).
1
2
3
4
Year
Number of deaths (per 100,000 lives) 1,140 1,200 1,170 1,190
The maximum likelihood estimator for the parameter \mu of the Poisson distribution is
given by X .
(i)
Obtain the maximum likelihood estimate of the parameter $\mu$ using these data.

To obtain a more realistic model, it is proposed that the number of deaths should
depend on the age of the population. To this end the total population is divided into
four age groups of equal size and the number of deaths in each group during the
following year is counted. The observed values are given in the following table. The
total number of deaths is again per 100,000 lives.
Middle age (t) in group
Number of deaths (x) in age group
25
84
35
113
45
255
55
727
For these data we obtain: Σ t = 160, Σ t 2 = 6,900, Σ x = 1,179, Σ x 2 = 613,379 and
Σ xt = 57,515
(ii)
(a) Calculate the correlation coefficient between the middle age $t$ in a group and the number of deaths x in that group, and comment briefly on its value.
(b) Perform a linear regression of the number of deaths x as a function of the middle age t of the group.
%%%%%%%%%%%%%%%%%%%%%%%%%%
A statistician suggests using a Poisson distribution for the number of deaths per year in each group, where the parameter \mu depends on the middle age in that group. Under the suggested model the number of deaths in the group with middle age t i is given by X i ~ Poisson(\mu i ) with \mu i = wt i , where t i is the middle age of the group that the individual belongs to at the time of death.
(iii)
Derive a maximum likelihood estimator for the parameter w and estimate the value of w from the data in the above table.
\end{enumerate}

%%%%%%%%%%%%%%%%%%%%%%%%%%%%%%%%%%%%%%%%%%%%%%%%%%%%%%%%%%%%%%%%%%%%%%%%%%%%%%%
\newpage
10
(iii) Assuming a larger value of s results in a larger standard error, so a larger sample size is required to achieve the same width of confidence interval.

(i) \mu ˆ = X = 1,175
(ii)
(a)
160 2
S tt = 6900 −
= 500
4
S xx
1179 2
= 613379 −
= 265,868.75
4
S tx = 57515 −
Corr ( t , x ) =
160*1179
= 10,355
4
S tx
10355
=
= 0.898114
500* 265868.75
S tt * S xx
This implies that there is a strong linear relationship between age and number of deaths.
%%%%%%%%%%%%%%%%%%%%%%%%%%%%%%%%%%%%%%%
(b)
Model: \[x̂ = α ˆ + β ˆ t\]
S
10355
β ˆ = tx =
= 20.71,
S tt
500
1179
160
α ˆ = x − β ˆ t =
− 20.71*
= − 533.65
4
4
Estimated model: x ˆ = 20.71 t − 533.65
%% Page 7 
%%   — Examiners’ Report, April 2011
(iii)
p ( x i , w , t i ) =
exp( − wt i )( wt i ) x i
x i !
log p ( x i , w , t i ) = − wt i + x i (log w + log t i ) − log( x i !)
x
∂
log p ( x i , w , t i ) = − t i + i
w
∂ w
∂
1
Σ
log p ( x i , w , t i ) = −Σ t i + Σ x i = 0
w
∂ w
w ˆ =
Σ x i
Σ t i
(Second derivative gives −\sum x i / w 2 < 0 which confirms maximum.)
For the observed values we obtain w ˆ =
1179
= 7.36875
160
%%%%%%%%%%%%%%%%%%%%%%%%%%%%%%%%%%%%%%%%%%%%%%%%%%%%%%%%%%%%%%%%%%%%%%%%%%%%%%%%%%%%%%%%%%%%%
\end{document}
