
\documentclass[a4paper,12pt]{article}

%%%%%%%%%%%%%%%%%%%%%%%%%%%%%%%%%%%%%%%%%%%%%%%%%%%%%%%%%%%%%%%%%%%%%%%%%%%%%%%%%%%%%%%%%%%%%%%%%%%%%%%%%%%%%%%%%%%%%%%%%%%%%%%%%%%%%%%%%%%%%%%%%%%%%%%%%%%%%%%%%%%%%%%%%%%%%%%%%%%%%%%%%%%%%%%%%%%%%%%%%%%%%%%%%%%%%%%%%%%%%%%%%%%%%%%%%%%%%%%%%%%%%%%%%%%%

\usepackage{eurosym}
\usepackage{vmargin}
\usepackage{amsmath}
\usepackage{graphics}
\usepackage{epsfig}
\usepackage{enumerate}
\usepackage{multicol}
\usepackage{subfigure}
\usepackage{fancyhdr}
\usepackage{listings}
\usepackage{framed}
\usepackage{graphicx}
\usepackage{amsmath}
\usepackage{chngpage}

%\usepackage{bigints}
\usepackage{vmargin}

% left top textwidth textheight headheight

% headsep footheight footskip

\setmargins{2.0cm}{2.5cm}{16 cm}{22cm}{0.5cm}{0cm}{1cm}{1cm}

\renewcommand{\baselinestretch}{1.3}

\setcounter{MaxMatrixCols}{10}

\begin{document}

\begin{enumerate}
\item
The number of claims made by each policyholder in a certain class of business is modelled as having a Poisson distribution with mean $\lambda$.

\begin{enumerate}
\item Derive an expression for the probability, $p$, that a policyholder in this class has made at least one claim.
\item The claims records of 20 randomly chosen policyholders were examined and the number of policyholders that made at least one claim in a year, X, was recorded.
(ii)
(a) State the distribution of the random variable $X$ and its parameters.
(b) Derive an expression for the maximum likelihood estimator of the
probability p given in (i) using your answer in (ii)(a).
(iii)
\item Show that, in the case $X = 5$, the maximum likelihood estimate (MLE) of p is

$\hat{p} = 0.25$ and hence calculate the MLE of $\lambda$ .
\end{enumerate}
It is now found that of the five policyholders who had made at least one claim there were four who had made exactly one claim and one who had made two claims.

Calculate the MLE of $\lambda$ given this additional information.
\end{enumerate}

%%%%%%%%%%%%%%%%%%%%%%%%%%%%%%%%%%%%%%%%%%%%%%%%%%%%%%%%%%%%%%%%%%%%%%%%%%%%%%%%
\newpage

\begin{itemize}
\item (i) \begin{eqnarray*}p &=& Pr( N \geq 1) \\ &=& 1 − \Pr( N = 0) &=& 1 − e^{−\lambda} .\end{eqnarray*}
\item (ii) (a)
X ~ Bin(20,p)
(b)
\[L ( p ) \propto  p X (1 − p ) 20 − X \]
⇒ 

\[l ( p ) = log( L ( p )) = X log p + (20 − X ) log(1 − p )\]
and l '( p ) = 0 ⇒
(and l ''( p ) = −
\item (iii)
p ˆ =
X 20 − X
X
−
= 0 ⇒ X − 20 p ˆ = 0 ⇒ p ˆ =
p ˆ
1 − p ˆ
20
20 − X
X
−
≤ 0)
p ˆ 2 (1 − p ˆ ) 2
5
= 0.25
20
\item Then, using the invariance property of the MLE:
ˆ
p ˆ = 1 − e −\lambda  ⇒ \hat{\lambda} = − log(1 − p ˆ ) = − log(0.75) = 0.288
\item Likelihood function now is:

(iv)
L ( \lambda  ) \propto  P ( X = 0) 15 \times  P ( X = 1) 4 \times  P ( X = 2)
( ) ( \lambda  e ) ( \lambda  e )
\propto  e −\lambda 
15
−\lambda 
4
2 −\lambda 
⇒ 
\[ l ( \lambda  ) \propto − 15 \lambda  + 4 log \lambda  − 4 \lambda  + 2 log \lambda  − \lambda  = − 20 \lambda  + 6 log \lambda \]

and l ′ ( \lambda  ) = 0 ⇒ − 20 +
Page 4
6
= 0 ⇒ \hat{\lambda} = 0.3
ˆ \lambda 
%%-- Subject CT3 (Probability and Mathematical Statistics) — September 2011 — Examiners’ Report
(Also l ′′ ( \lambda  ) = −
6
< 0 , hence max)
\hat{\lambda} 2
\item Notice that part (ii)(b) requires the use of the binomial distribution from (ii)(a).
\item In part (iii)
the invariance property must be used and mentioned for full credit.
\end{itemize}
\newpage

\end{document}
