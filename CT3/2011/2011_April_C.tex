\documentclass[a4paper,12pt]{article}

%%%%%%%%%%%%%%%%%%%%%%%%%%%%%%%%%%%%%%%%%%%%%%%%%%%%%%%%%%%%%%%%%%%%%%%%%%%%%%%%%%%%%%%%%%%%%%%%%%%%%%%%%%%%%%%%%%%%%%%%%%%%%%%%%%%%%%%%%%%%%%%%%%%%%%%%%%%%%%%%%%%%%%%%%%%%%%%%%%%%%%%%%%%%%%%%%%%%%%%%%%%%%%%%%%%%%%%%%%%%%%%%%%%%%%%%%%%%%%%%%%%%%%%%%%%%

\usepackage{eurosym}
\usepackage{vmargin}
\usepackage{amsmath}
\usepackage{graphics}
\usepackage{epsfig}
\usepackage{enumerate}
\usepackage{multicol}
\usepackage{subfigure}
\usepackage{fancyhdr}
\usepackage{listings}
\usepackage{framed}
\usepackage{graphicx}
\usepackage{amsmath}
\usepackage{chngpage}

%\usepackage{bigints}
\usepackage{vmargin}

% left top textwidth textheight headheight

% headsep footheight footskip

\setmargins{2.0cm}{2.5cm}{16 cm}{22cm}{0.5cm}{0cm}{1cm}{1cm}

\renewcommand{\baselinestretch}{1.3}

\setcounter{MaxMatrixCols}{10}

\begin{document}
\begin{enumerate}
5
Let $X_1$ , $X_2$ , $X_3$ , $X_4$ , and $X_5$ be independent random variables, such that X i ~ gamma
5
with parameters i and \lambda for i = 1, 2, 3, 4, 5. Let S = 2 \lambda \sum X i .
i = 1
6
\begin{enumerate}[(i)]
\item (i) Derive the mean and variance of S using standard results for the mean and variance of linear combinations of random variables.
\item
(ii) Show that S has a chi-square distribution using moment generating functions and state the degrees of freedom of this distribution.
\item 
(iii) Verify the values found in part (i) using the results of part (ii). .
\end{enumerate}

%%%%%%%%%%%%%%%%%%%%%%%%%%%%%%%%%%%%%%%%%%%%%%%%%%%%%%%%%%%%%%%%%%%%%%%%%%%%%%%%%%%%%%%%%%%%%%%%%%%%%%%%%%%
\item Consider two random variables X and Y, for which the variances satisfy V[X] = 5V[Y]
and the covariance Cov[X,Y] satisfies Cov[X,Y] = V[Y].
Let S = X + Y and D = X − Y.
\begin{enumerate}[(i)]
\item (i)  Show that the covariance between S and D satisfies Cov[S,D] = 4V[Y].
\item (ii) Calculate the correlation coefficient between S and D.
\end{enumerate}
%%%%%%%%%%%%%%%%%%%%%%%%%%%%%%%%%%%%%%%%%%%%%%%%%%%%%%%%%%%%%%%%%%%%%%%%%%%%%%%%%%%%%%%%%%%%%%%%%%%%%%%%%%
5
(i)
E [ S ] = 2 \lambda \sum
i
= 30
i = 1 \lambda
5
V [ S ] = 4 \lambda 2 \sum
i = 1
(ii)
i
\lambda 2
t ⎞
⎛
M X i ( t ) = ⎜ 1 − ⎟
⎝ \lambda ⎠
= 60
− i
(from book of formulae)
5
⎡
⎛
⎞ ⎤ 5
M S ( t ) = E ⎡ ⎣ exp ( tS ) ⎤ ⎦ = E ⎢ exp ⎜ 2 \lambda t \sum X i ⎟ ⎥ = ∏ E ⎡ ⎣ exp ( 2 \lambda tX i ) ⎤ ⎦
⎜
⎟ ⎥
⎢ ⎣
i = 1
⎝
⎠ ⎦ i = 1
5 5
i = 1 i = 1
= ∏ M X i ( 2 \lambda t ) = ∏ ( 1 − 2 t )
(iii)
− i
= ( 1 − 2 t )
− 15
so S ~ χ 2 , with 30 df
2
χ 30
has mean 30 and variance 60 , as found in part (i).
% Page 3
%  — Examiners’ Report, April 2011
%%%%%%%%%%%%%%%%%%%%%%%%%%%%%%%%%%%%%%%%%%%%%%%%%%%%%%%%%%%%%%%%%%%%%%%%%%
6
(i) Cov[S,D] = Cov[X + Y , X − Y] = Cov[X,X] − Cov[X,Y] + Cov[Y,X] − Cov[Y,Y]
= V[X] − V[Y] = 4V[Y]
(ii) V[S] = V[X] + V[Y] + 2Cov[X,Y] = 8V[Y]
V[D] = V[X] + V[Y] − 2Cov[X,Y] = 4V[Y]
⇒ Corr[S,D] = 4V[Y]/{8V[Y] × 4V[Y]} 1/2 = +1/√2 = +0.707
\end{document}
