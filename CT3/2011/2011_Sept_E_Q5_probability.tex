\documentclass[a4paper,12pt]{article}

%%%%%%%%%%%%%%%%%%%%%%%%%%%%%%%%%%%%%%%%%%%%%%%%%%%%%%%%%%%%%%%%%%%%%%%%%%%%%%%%%%%%%%%%%%%%%%%%%%%%%%%%%%%%%%%%%%%%%%%%%%%%%%%%%%%%%%%%%%%%%%%%%%%%%%%%%%%%%%%%%%%%%%%%%%%%%%%%%%%%%%%%%%%%%%%%%%%%%%%%%%%%%%%%%%%%%%%%%%%%%%%%%%%%%%%%%%%%%%%%%%%%%%%%%%%%

\usepackage{eurosym}
\usepackage{vmargin}
\usepackage{amsmath}
\usepackage{graphics}
\usepackage{epsfig}
\usepackage{enumerate}
\usepackage{multicol}
\usepackage{subfigure}
\usepackage{fancyhdr}
\usepackage{listings}
\usepackage{framed}
\usepackage{graphicx}
\usepackage{amsmath}
\usepackage{chngpage}

%\usepackage{bigints}
\usepackage{vmargin}

% left top textwidth textheight headheight

% headsep footheight footskip

\setmargins{2.0cm}{2.5cm}{16 cm}{22cm}{0.5cm}{0cm}{1cm}{1cm}

\renewcommand{\baselinestretch}{1.3}

\setcounter{MaxMatrixCols}{10}

\begin{document}

\begin{enumerate}

%%%%%%%%%%%%%%%%%%%%%%%%%%%%%%%%%%%%%%%%%%%%%%%%%%%%%%%%%%%%%%%%%%%%%%%%%%%%%%%%%%%%%%%%%%%%%%%%555
% [3] 
5
\item Consider the random variable X taking the value X = 1 if a randomly selected person is a smoker, or X = 0 otherwise. The random variable Y describes the amount of physical exercise per week for this randomly selected person. It can take the values 0
(less than one hour of exercise per week), 1 (one to two hours) and 2 (more than two hours of exercise per week). The random variable R = (3 − Y) 2 (X + 1) is used as a risk index for a particular heart disease.
The joint distribution of X and Y is given by the joint probability function in the
following table.
X
0
1
6
0
0.2
0.1
Y
1
0.3
0.1
2
0.25
0.05
\begin{enumerate}[(i)]
\item Calculate the probability that a randomly selected person does more than two hours of exercise per week.
\item Decide whether X and Y are independent or not and justify your answer. 
\item Derive the probability function of $R$. 
\item Calculate the expectation of $R$.
\end{enumerate}
%%%%%%%%%%%%%%%%%%%%%%%%%%%%%%%%%%%%%%%%%%%%%%%%%%%%%%%%
\newpage

%%%%%%%%%%%%%%%%%%%%%%%%%%%%%%%%%%%%%%%%%%%%%%%%%%%%%%%%%%%%%%%%%%%%%%%%%%%%%%%%%%%%%%%%%%%%
5
(i) P[Y = 2] = 0.25 + 0.05 = 0.3
(ii) P[X = 0] = 0.75 and
\[ P[{X = 0} \cap {Y = 2}] = 0.25 ≠ 0.225 = 0.3 * 0.75 = P[X = 0] * P[Y = 2]\]
Therefore X and Y are not independent. (Any other joint probability can be used.)

(iii)
(iv)
The probability function is
r 1
2
4
8
9
18
P(R = r) 0.25 0.05 0.3 0.1 0.2 0.1
\begin{eqnarray*}
E[R] &=& 0.2 * 9 + 0.3 * 4 + 0.25 * 1 + 0.1 * 18 + 0.1 * 8 + 0.05 * 2\\
&=& 1.8 + 1.2 + 0.25 + 1.8 + 0.8 + 0.1 \\
&=& 5.95\\
\end{eqnarray*}

In part (ii) notice that one example of $P(XY) = P(X)P(Y)$ is not sufficient for showing
independence (it needs to hold for all cases). Also, some candidates failed to provide the
probability function in (iii).

\end{document}
