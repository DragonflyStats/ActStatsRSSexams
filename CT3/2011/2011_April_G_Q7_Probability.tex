\documentclass[a4paper,12pt]{article}

%%%%%%%%%%%%%%%%%%%%%%%%%%%%%%%%%%%%%%%%%%%%%%%%%%%%%%%%%%%%%%%%%%%%%%%%%%%%%%%%%%%%%%%%%%%%%%%%%%%%%%%%%%%%%%%%%%%%%%%%%%%%%%%%%%%%%%%%%%%%%%%%%%%%%%%%%%%%%%%%%%%%%%%%%%%%%%%%%%%%%%%%%%%%%%%%%%%%%%%%%%%%%%%%%%%%%%%%%%%%%%%%%%%%%%%%%%%%%%%%%%%%%%%%%%%%

\usepackage{eurosym}
\usepackage{vmargin}
\usepackage{amsmath}
\usepackage{graphics}
\usepackage{epsfig}
\usepackage{enumerate}
\usepackage{multicol}
\usepackage{subfigure}
\usepackage{fancyhdr}
\usepackage{listings}
\usepackage{framed}
\usepackage{graphicx}
\usepackage{amsmath}
\usepackage{chngpage}

%\usepackage{bigints}
\usepackage{vmargin}

% left top textwidth textheight headheight

% headsep footheight footskip

\setmargins{2.0cm}{2.5cm}{16 cm}{22cm}{0.5cm}{0cm}{1cm}{1cm}

\renewcommand{\baselinestretch}{1.3}

\setcounter{MaxMatrixCols}{10}

\begin{document}

%%- 
%%- 7
An insurance company distinguishes between three types of fraudulent claims:
\begin{description}
\item[Type 1:] legitimate claims that are slightly exaggerated
\item[Type 2:] legitimate claims that are strongly exaggerated
\item[Type 3:] false claims
\end{description}
%-----------------------------------------------%
Every fraudulent claim is characterised as exactly one of the three types. Assume that the probability of a newly submitted claim being a fraudulent claim of type 1 is 0.1.
For type 2 this probability is 0.02, and for type 3 it is 0.003.
\begin{enumerate}[(a)]
\item Calculate the probability that a newly submitted claim is not fraudulent.
The insurer uses a statistical software package to identify suspicious claims. If a claim is fraudulent of type 1, it is identified as suspicious by the software with probability 0.5. For a type 2 claim this probability is 0.7, and for type 3 it is 0.9.
Of all newly submitted claims, 20\% are identified by the software as suspicious.
\item 
Calculate the probability that a claim that has been identified by the software as suspicious is:
(a) a fraudulent claim of type 1,
(b) a fraudulent claim of any type.
\item 
Calculate the probability that a claim which has NOT been identified as
suspicious by the software is in fact fraudulent.
\end{enumerate} 
%%%%%%%%%%%%%%%%%%%%%%%%%%%%%%%%%%%%%%%%%%%%%%%%%%%%%%%%%%%%%%%%%%%%%%%%%%%%%%%%%%%%%%%%%%%%%%%%%%%%55
%%%%%%%%%%%%%%%%%%%%%%%%%%%%%%%%%%%%%%%%%%%%%%%%%%%%%%%%%%%%%%%%%%%%%%%%%%%%%%%%%%%%%%%%%%%%%%%%%5
7
(i) 
\begin{eqnarray*} 
1 − P[T1 \cup T2 \cup T3] &=& 1 − (0.1 + 0.02 + 0.003) \\ &=& 1 − 0.123 \\ &=& 0.877\\ 
\end{eqnarray*}
(ii) (a)
P[T1 | S] =
(b)
P[T1 \cup T2 \cup T3 | S] =
(iii)
(i)
\documentclass[a4paper,12pt]{article}

%%%%%%%%%%%%%%%%%%%%%%%%%%%%%%%%%%%%%%%%%%%%%%%%%%%%%%%%%%%%%%%%%%%%%%%%%%%%%%%%%%%%%%%%%%%%%%%%%%%%%%%%%%%%%%%%%%%%%%%%%%%%%%%%%%%%%%%%%%%%%%%%%%%%%%%%%%%%%%%%%%%%%%%%%%%%%%%%%%%%%%%%%%%%%%%%%%%%%%%%%%%%%%%%%%%%%%%%%%%%%%%%%%%%%%%%%%%%%%%%%%%%%%%%%%%%

\usepackage{eurosym}
\usepackage{vmargin}
\usepackage{amsmath}
\usepackage{graphics}
\usepackage{epsfig}
\usepackage{enumerate}
\usepackage{multicol}
\usepackage{subfigure}
\usepackage{fancyhdr}
\usepackage{listings}
\usepackage{framed}
\usepackage{graphicx}
\usepackage{amsmath}
\usepackage{chngpage}

%\usepackage{bigints}
\usepackage{vmargin}

% left top textwidth textheight headheight

% headsep footheight footskip

\setmargins{2.0cm}{2.5cm}{16 cm}{22cm}{0.5cm}{0cm}{1cm}{1cm}

\renewcommand{\baselinestretch}{1.3}

\setcounter{MaxMatrixCols}{10}

\begin{document}

%%- 
%%- 7
An insurance company distinguishes between three types of fraudulent claims:
\begin{description}
\item[Type 1:] legitimate claims that are slightly exaggerated
\item[Type 2:] legitimate claims that are strongly exaggerated
\item[Type 3:] false claims
\end{description}
%-----------------------------------------------%
Every fraudulent claim is characterised as exactly one of the three types. Assume that the probability of a newly submitted claim being a fraudulent claim of type 1 is 0.1.
For type 2 this probability is 0.02, and for type 3 it is 0.003.
\begin{enumerate}[(a)]
\item Calculate the probability that a newly submitted claim is not fraudulent.
The insurer uses a statistical software package to identify suspicious claims. If a claim is fraudulent of type 1, it is identified as suspicious by the software with probability 0.5. For a type 2 claim this probability is 0.7, and for type 3 it is 0.9.
Of all newly submitted claims, 20\% are identified by the software as suspicious.
\item 
Calculate the probability that a claim that has been identified by the software as suspicious is:
(a) a fraudulent claim of type 1,
(b) a fraudulent claim of any type.
\item 
Calculate the probability that a claim which has NOT been identified as
suspicious by the software is in fact fraudulent.
\end{enumerate} 
%%%%%%%%%%%%%%%%%%%%%%%%%%%%%%%%%%%%%%%%%%%%%%%%%%%%%%%%%%%%%%%%%%%%%%%%%%%%%%%%%%%%%%%%%%%%%%%%%%%%55
%%%%%%%%%%%%%%%%%%%%%%%%%%%%%%%%%%%%%%%%%%%%%%%%%%%%%%%%%%%%%%%%%%%%%%%%%%%%%%%%%%%%%%%%%%%%%%%%%5
7
(i) 
\begin{eqnarray*} 
1 − P[T1 \cup T2 \cup T3] 
&=& 1 − (0.1 + 0.02 + 0.003) \\ 
&=& 1 − 0.123 \\ 
&=& 0.877\\ 
\end{eqnarray*}

%--------------------------------------%
(ii) (a)
\begin{eqnarray*} 
P[T1 | S] &=& \frac{P[T1 \cap S]}{P[S]}\\
&=& \frac{P[S|T1) \times P[T1 ]}{P[S]}\\
&=& \frac{0.5 \times 0.1}{ 0.2}\\
&=& 0.25\\
\end{eqnarray*}


%--------------------------------------%
(b)
\begin{eqnarray*}
P[T1 \cup T2 \cup T3 | S] =
&=& \frac{(P[ T1 \cap S]  + P[T2 \cap S]  + P[ T3 \cap S] )}{P [ S ]}
\frac{(P[S | T1] P[T1] + P[S | T2] P[T2] + P[S | T3] P[T3])}{P [ S ]}
\\
& & \\
&=& \frac{(0.5\times 0.1) + (0.7 \times 0.02) + (0.9\times0.003)}{0.2}\\
&=& \frac{0.0667}{0.2}\\
&=&  0.3335 \\
\end{eqnarray*}

%------------------------------------------%
\begin{eqnarray*}
P[T1 \cup T2 \cup T3 | S] 
&=& \frac{(P[T1 \cup T 2 \cup T 3] − P [{ T 1 \cup T 2 \cup T 3} \cap S ])}{0.80}\\
&=& \frac{0.123 − 0.5\times 0.1 − 0.7 \times 0.02 − 0.9\times 0.003}{0.8}\\
&=& \frac{0.0563}{0.8}\\
&=& 0.0704\\
\end{eqnarray*}
\end{document}
%%%%%%%%%%%%%%%%%%%%%%%%%%%%%%%%%%%%%%%%%%%%%%%%%%%%%%%%%%%%%%%%%%%%%%%%%%%%%%555
(P[T1 \cap S] + P[T2 \cap S] + P[T3 \cap S])
P [ S ]
=
P [T1 \cup T 2 \cup T 3 | S C ] =
=
8
P [ T 1 \cap S ] P [ S | T 1] P [ T 1] 0.5*0.1
=
=
= 0.25
P [ S ]
P [ S ]
0.2
5(11630.67) + 5(8984.97)
= 10307.82
10
495.33 − 465.17 30.16
=
= 0.51 on 10 df
1 1 58.62
101.527
+
6 6
∴ s P = 101.527
%%%%%%%%%%%%%%%%%%%%%%%%%%%%%%%%%%%%%%%%%%%%%%%%%%%%%%%%%%%%%%%%%%%%%%%%%%%%%%%%%%%
without needing to look up tables (although candidates can do so, e.g.
t 10 (2.5\%) = 2.228)
there is clearly no evidence of a difference between medications A and B as regards their effectiveness for the relief of coughing.

\end{document}
