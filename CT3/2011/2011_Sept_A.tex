\documentclass[a4paper,12pt]{article}

%%%%%%%%%%%%%%%%%%%%%%%%%%%%%%%%%%%%%%%%%%%%%%%%%%%%%%%%%%%%%%%%%%%%%%%%%%%%%%%%%%%%%%%%%%%%%%%%%%%%%%%%%%%%%%%%%%%%%%%%%%%%%%%%%%%%%%%%%%%%%%%%%%%%%%%%%%%%%%%%%%%%%%%%%%%%%%%%%%%%%%%%%%%%%%%%%%%%%%%%%%%%%%%%%%%%%%%%%%%%%%%%%%%%%%%%%%%%%%%%%%%%%%%%%%%%

\usepackage{eurosym}
\usepackage{vmargin}
\usepackage{amsmath}
\usepackage{graphics}
\usepackage{epsfig}
\usepackage{enumerate}
\usepackage{multicol}
\usepackage{subfigure}
\usepackage{fancyhdr}
\usepackage{listings}
\usepackage{framed}
\usepackage{graphicx}
\usepackage{amsmath}
\usepackage{chngpage}

%\usepackage{bigints}
\usepackage{vmargin}

% left top textwidth textheight headheight

% headsep footheight footskip

\setmargins{2.0cm}{2.5cm}{16 cm}{22cm}{0.5cm}{0cm}{1cm}{1cm}

\renewcommand{\baselinestretch}{1.3}

\setcounter{MaxMatrixCols}{10}

\begin{document}
\begin{enumerate}
%%%%%%%%%%%%%%%%%%%%%%%%%%%%%%%%%%%%%%%%%%%%%%%%%%%%%%%%%%%%%%%%%%%%%%%%%%%%%%%%%
%%%%%%%%%%%%%%%%%%%%%%%%%%%%%%%%%%%%%%%%%%%%%%%%%%%%%%%%%%%%%%%%%%%%%%%%%%%%%%%%%%1
The first 20 claims that were paid out under a group of policies were for the following
amounts (in units of £1,000):
3.2
7.0
2.1
8.1
6.3
4.4
4.0
5.8
3.8
1.7
4.4
2.8
6.5
5.0
7.8
3.2
2.8
3.7
5.2
4.4
For these data \sumx = 92.2.
(i)
Calculate the mean of these 20 claim amounts.

The next 80 claims paid out had a mean amount of £5,025.
(ii)
2
Calculate the mean amount for the first 100 claims.

[Total 3]
The claims which arose in a sample of policies of a certain class gave the following
frequency distribution for the number of claims per policy in the last year:
Number of claims x 0 1 2 3 4 or more
0
Number of policies f 15 20 10 5
Calculate the third order moment about the origin for these data.
3

A random sample of 60 adult men who live in Leeds includes 21 who have visited Majorca. An independent random sample of 70 adult women who live in Leeds includes 28 who have visited Majorca.
Calculate a 98\% confidence interval for the proportion of adults who live in Leeds
who have visited Majorca.
\end{enumerate}
\newpage
%%%%%%%%%%%%%%%%%%%%%%%%%%%%%%%%%%%%%%%%%%%%%%%%%%%%%%%%%%%%%%%%%%%%%%%%%%%%%%%%%%%
1
(i) mean x = 4.61 or £4,610.
(ii) mean of the whole 100 =
20(4610) + 80(5025) 494200
=
= £4,942
100
100
Generally very well answered.
2
Required moment =
1
fx 3
\sum
\sum f
x
=
(
)
1
235
15 × 0 3 + 20 × 1 3 + 10 × 2 3 + 5 × 3 3 =
= 4.7
50
50
Many candidates calculated the third central moment (around the sample mean), rather than
the moment around zero as required in the question.
3
Sample proportion = 49/130
Upper 1% normal percentage point = 2.326
98% CI is
49/130 ± 2.326*[(49/130)(81/130)/130] 1/2 i.e. 0.3769 ± 0.0989 i.e. (0.278, 0.476)
Answers here were generally satisfactory. Some candidates erroneously computed CIs based
on men and women separately.
\end{document}
