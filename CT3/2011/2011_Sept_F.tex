\documentclass[a4paper,12pt]{article}

%%%%%%%%%%%%%%%%%%%%%%%%%%%%%%%%%%%%%%%%%%%%%%%%%%%%%%%%%%%%%%%%%%%%%%%%%%%%%%%%%%%%%%%%%%%%%%%%%%%%%%%%%%%%%%%%%%%%%%%%%%%%%%%%%%%%%%%%%%%%%%%%%%%%%%%%%%%%%%%%%%%%%%%%%%%%%%%%%%%%%%%%%%%%%%%%%%%%%%%%%%%%%%%%%%%%%%%%%%%%%%%%%%%%%%%%%%%%%%%%%%%%%%%%%%%%

\usepackage{eurosym}
\usepackage{vmargin}
\usepackage{amsmath}
\usepackage{graphics}
\usepackage{epsfig}
\usepackage{enumerate}
\usepackage{multicol}
\usepackage{subfigure}
\usepackage{fancyhdr}
\usepackage{listings}
\usepackage{framed}
\usepackage{graphicx}
\usepackage{amsmath}
\usepackage{chngpage}

%\usepackage{bigints}
\usepackage{vmargin}

% left top textwidth textheight headheight

% headsep footheight footskip

\setmargins{2.0cm}{2.5cm}{16 cm}{22cm}{0.5cm}{0cm}{1cm}{1cm}

\renewcommand{\baselinestretch}{1.3}

\setcounter{MaxMatrixCols}{10}

\begin{document}
\begin{enumerate}Consider a situation in which integer-valued responses (y) are recorded at ten values
of an integer-valued explanatory variable (x). The data are presented in the following
scatter plot:
10
10
(a)
x
x
x
x
x
x
x
x x
8 10
x
0
2
4
6
12
x
For these data: Σx = 58, Σx 2 = 420, Σy = 41, Σy 2 = 217, Σxy = 202
(i)
(a) Calculate the value of the coefficient of determination (R 2 ) for the data.
(b) Determine the equation of the fitted least-squares line of regression of
y on x.
[7]
(ii) Calculate a 95\% confidence interval for the slope of the underlying line of
regression of y on x.

(iii) (a)
Calculate an estimate of the expected response in the case x = 9.
(b)
Calculate the standard error of this estimate.

Suppose the observation (x = 10, y = 8) is added to the existing data. The coefficient
of determination is now R 2 = 0.07.
(iv)
Comment briefly on the effect of the new observation on the fit of the linear
model.

[Total 16]
END OF PAPER
CT3 S2011—7

%%%%%%%%%%%%%%%%%%%%%%%%%%%%%%%%%%%%%%%%%%%%%%%%%%%%%%%%%%%%%%%%%%%%%%%%%%%%%%555
10
(i)
(a)
S xx = 420 – 58 2 /10 = 83.6, S yy = 217 – 41 2 /10 = 48.9,
S xy = 202 – 58*41/10 = –35.8
SS T = 48.9, SS REG = (–35.8) 2 /83.6 = 15.3306
⇒ R 2 = 15.3306/48.9 = 0.3135 (or 31.4%)
[OR using R 2 = S xy 2 / S xx S yy ]
Page 8 (Probability and Mathematical Statistics) — September 2011 — Examiners’ Report
Fitted line y = α ˆ + β ˆ x :
(b)
β ˆ = − 35.8 / 83.6 = − 0.42823 , α ˆ = 4.1 − ( − 0.42823*5.8 ) = 6.58373
Fitted line is y = 6.5837 – 0.4282 x
(ii)
σ ˆ 2 = ( 48.9 − 15.3306 ) / 8 = 4.1962
()
1/2
s . e . β ˆ = ( 4.1962 / 83.6 ) = 0.2240
()
95\% confidence interval for β is β ˆ ± t 8 ∗ s . e . β ˆ
i.e. –0.42823 ± 2.306*0.2240
i.e. (–0.945, 0.088)
(iii)
(a) At x = 9, y ˆ = 6.5837 − 0.4282 ∗ 9 = 2.7299 i.e. 2.730
⎛ 1
(b) s . e . = ⎜
⎜ 10
⎝
2
(iv)
+
( 9 − 5.8 ) 2 ⎞ ⎟ 4.1962
83.6
⎟
⎠
= 0.93360 ⇒ s.e. = 0.9662
Addition of new observation makes data more randomly scattered. The
strength of the linear relationship is reduced from “weak” to “almost nothing”.
Generally well answered. Some problems were encountered in part (iii)(b), where the wrong
formula was used.
END OF EXAMINER’S REPORT
Page 9
