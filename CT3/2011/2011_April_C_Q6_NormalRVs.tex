\documentclass[a4paper,12pt]{article}

%%%%%%%%%%%%%%%%%%%%%%%%%%%%%%%%%%%%%%%%%%%%%%%%%%%%%%%%%%%%%%%%%%%%%%%%%%%%%%%%%%%%%%%%%%%%%%%%%%%%%%%%%%%%%%%%%%%%%%%%%%%%%%%%%%%%%%%%%%%%%%%%%%%%%%%%%%%%%%%%%%%%%%%%%%%%%%%%%%%%%%%%%%%%%%%%%%%%%%%%%%%%%%%%%%%%%%%%%%%%%%%%%%%%%%%%%%%%%%%%%%%%%%%%%%%%

\usepackage{eurosym}
\usepackage{vmargin}
\usepackage{amsmath}
\usepackage{graphics}
\usepackage{epsfig}
\usepackage{enumerate}
\usepackage{multicol}
\usepackage{subfigure}
\usepackage{fancyhdr}
\usepackage{listings}
\usepackage{framed}
\usepackage{graphicx}
\usepackage{amsmath}
\usepackage{chngpage}

%\usepackage{bigints}
\usepackage{vmargin}

% left top textwidth textheight headheight

% headsep footheight footskip

\setmargins{2.0cm}{2.5cm}{16 cm}{22cm}{0.5cm}{0cm}{1cm}{1cm}

\renewcommand{\baselinestretch}{1.3}

\setcounter{MaxMatrixCols}{10}

\begin{document}
\largen
\noindent In a large population, 35\% of voters intend to vote for party A at the next election. A random sample of 200 voters is selected from this population and asked which party they will vote for.
Calculate, approximately, the probability that 80 or more of the people in this sample intend to vote for party A.


%%%%%%%%%%%%%%%%%%%%%%%%%%%%%%%%%%%%%%%%%%%%%%%%%%%%%%%%%%%%%%%%%%%%%%%%%%%%%%%%%%%%%%%%%
3
If $X$ is the number of voters in the sample voting for party A, we have $X ~ Binomial(200, 0.35)$ and using the CLT X ~ N(70, 45.5) approximately.

Using continuity correction



\begin{eqnarray*}
P( X \geq 80 ) &=& P\left( Z \geq  \frac{79.5 \;-\;70}{\sqrt{45.5}} \right)\\
&=& P\left( Z \geq  \frac{9.5 }{6.745} \right)\\
&=& P\left( Z \geq  1.408 \right)\\
&=& 1 − P(Z < 1.408) \\
&=& 1 − 0.920 \\
&=& 0.08.\\
\end{eqnarray*}


%%%%%%%%%%%%%%%%%%%%%%%%%%%%%%%%%%%%%%%%%%%%%%%

\begin{framed}
\begin{verbatim}
>
> 1 - pnorm(1.408)
[1] 0.07956553
>
\end{verbatim}
\end{framed}





%%%%%%%%%%%%%%%%%%%%%%%%%%%%%%%%%%%%%%%%%%%%%%%%%%%%%%%%%%%%%%%%%%%%%%%%%%%5

\end{document}
