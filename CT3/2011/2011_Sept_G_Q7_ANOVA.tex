
\documentclass[a4paper,12pt]{article}

%%%%%%%%%%%%%%%%%%%%%%%%%%%%%%%%%%%%%%%%%%%%%%%%%%%%%%%%%%%%%%%%%%%%%%%%%%%%%%%%%%%%%%%%%%%%%%%%%%%%%%%%%%%%%%%%%%%%%%%%%%%%%%%%%%%%%%%%%%%%%%%%%%%%%%%%%%%%%%%%%%%%%%%%%%%%%%%%%%%%%%%%%%%%%%%%%%%%%%%%%%%%%%%%%%%%%%%%%%%%%%%%%%%%%%%%%%%%%%%%%%%%%%%%%%%%

\usepackage{eurosym}
\usepackage{vmargin}
\usepackage{amsmath}
\usepackage{graphics}
\usepackage{epsfig}
\usepackage{enumerate}
\usepackage{multicol}
\usepackage{subfigure}
\usepackage{fancyhdr}
\usepackage{listings}
\usepackage{framed}
\usepackage{graphicx}
\usepackage{amsmath}
\usepackage{chngpage}

%\usepackage{bigints}
\usepackage{vmargin}

% left top textwidth textheight headheight

% headsep footheight footskip

\setmargins{2.0cm}{2.5cm}{16 cm}{22cm}{0.5cm}{0cm}{1cm}{1cm}

\renewcommand{\baselinestretch}{1.3}

\setcounter{MaxMatrixCols}{10}

\begin{document}

\begin{enumerate}
\item
7
The total amounts y ij (in £ millions) paid out under a certain type of policy issued by four different companies A, B, C, D in each of six consecutive years were as follows:
Company
A
B
C
D
2.870
3.105
2.800
2.830
3.125
3.200
2.985
2.600
3.000
3.300
3.060
2.765
2.865
2.975
2.900
2.690
2.890
3.210
2.920
2.600
3.060
3.150
3.050
2.700
Total
17.810
18.940
17.715
16.185
For these data, Σ i Σ j y ij = 70.650 and Σ i Σ j y ij 2 = 208.828.
Consider the ANOVA model $Y ij = μ + τ i + e ij$ , i = 1, ..., 4, j = 1, ..., 6, where Y ij is the
jth amount paid out by company i, and e ij ~ N(0, σ 2 ) are independent errors.
The ANOVA table for these data is given below.

\begin{verbatim}
Source
Company (between treatments)
Residual
Total
DF
3
20
23
SS
0.640
0.212
0.852
MS
0.213
0.0106
\end{verbatim}

\begin{enumerate}[(i)]
\item (i) Test the hypothesis that there are no differences in the means of the amounts paid out under such policies by the four companies (the company means), stating your conclusions clearly.
\item 
(ii) Comment briefly on the validity of the test performed in (i), using the plot of the residuals given below.
\item
(iii) (a)
Calculate the least significant difference between pairs of company means using a 5\% significance level.
(b)
List the company means in order, illustrate the non-significant pairs using suitable underlining, and comment briefly.
\end{enumerate}
 \newpage
 7
 \begin{itemize}
     \item 

(i) F = 0.213/0.0106 = 20.094 and at the 5\% significance level,
F 3,20 (0.05) = 3.098.a
\item Since F = 20.094 > 3.098, there is strong evidence against the null hypothesis, and we conclude that there are differences in the mean amounts paid out by the companies.
\item (ii) The variance of the residuals seems to be similar for the four companies; this is consistent with the assumption of constant variance in the response variable.
Also there are no obvious patterns or outliers. The analysis seems valid.
\item (iii) (a)
LSD = t 20 (0.025)
= 2.086
(b)
⎛ 1 1 ⎞
σ 2 ⎜ + ⎟
⎝ 6 6 ⎠
0.0106 / 3 = 0.124
\item The four company (treatment) means are:
y 1 i = 17.810
18.940
17.715
= 2.968, y 2 i =
= 3.157, y 3 i =
= 2.953
6
6
6
y 4 i = 16.185
= 2.698
6
which are given in order and underlined as
y 4 i < y 3 i < y 1 i < y 2 i
\item Amounts paid out by companies 2 and 4 are significantly different
from those paid out by the other two companies. Company 4 seems to
pay out significantly lower amounts, with Company 2 paying
significantly higher.
\item 
Parts (i) and (ii) were generally well answered. In part (iii) many candidates did not use the
correct formula for LSD and then performed pair-wise comparisons using the wrong statistic.
 \end{itemize}
\end{document}
