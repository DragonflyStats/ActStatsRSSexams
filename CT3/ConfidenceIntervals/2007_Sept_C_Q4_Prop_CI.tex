
\documentclass[a4paper,12pt]{article}

%%%%%%%%%%%%%%%%%%%%%%%%%%%%%%%%%%%%%%%%%%%%%%%%%%%%%%%%%%%%%%%%%%%%%%%%%%%%%%%%%%%%%%%%%%%%%%%%%%%%%%%%%%%%%%%%%%%%%%%%%%%%%%%%%%%%%%%%%%%%%%%%%%%%%%%%%%%%%%%%%%%%%%%%%%%%%%%%%%%%%%%%%%%%%%%%%%%%%%%%%%%%%%%%%%%%%%%%%%%%%%%%%%%%%%%%%%%%%%%%%%%%%%%%%%%%

\usepackage{eurosym}
\usepackage{vmargin}
\usepackage{amsmath}
\usepackage{graphics}
\usepackage{epsfig}
\usepackage{enumerate}
\usepackage{multicol}
\usepackage{subfigure}
\usepackage{fancyhdr}
\usepackage{listings}
\usepackage{framed}
\usepackage{graphicx}
\usepackage{amsmath}
\usepackage{chngpage}

%\usepackage{bigints}
\usepackage{vmargin}

% left top textwidth textheight headheight

% headsep footheight footskip

\setmargins{2.0cm}{2.5cm}{16 cm}{22cm}{0.5cm}{0cm}{1cm}{1cm}

\renewcommand{\baselinestretch}{1.3}

\setcounter{MaxMatrixCols}{10}

\begin{document}
%%-- Question 4
 In a random sample of 200 policies from a company’s private motor business, there are 68 female policyholders and 132 male policyholders.
Calculate an approximate 99\% confidence interval for the proportion of policyholders who are female in the corresponding population of all policyholders.

%%%%%%%%%%%%%%%%%%%%%%%%%%%%%%%%%%%%%%%%%%%%%%%%%%%%%%%%%%%%%%%%%%%%%%%%%%%%%%%%%%%%%%
\begin{framed}
\noindent \textbf{General Structure of a Confidence Interval}
\[  \mbox{Point Estimate} \pm \left[ \mbox{Quantile} \times \mbox{Standard Error}\right]\]
\end{framed}


\begin{itemize}
\item Point Estimate: Sample Proportion ${ \displaystyle \hat{p} = \frac{68}{200} = 0.34 }$
\item Interval Size : 99\%
\item Sample Size : $n=200$
\item We consider this a large sample (i.e. $n \geq 30$), and assume the sampling distribution is approximated by the Normal distribution.
\end{itemize}
%--------------------%
\noindent \textbf{Quantiles for Confidence Intervals}
\\ \textit{ Assuming Normality based on large sample size.}\\
\begin{center}
\begin{tabular}{|c|c|c|} \hline
90\%     & $Z_{0.95}$ & 1.645 \\ \hline
95\%     & $Z_{0.975}$ & 1.96 \\ \hline
98\%    & $Z_{0.99}$ & 2.326    \\ \hline
99\%    & $Z_{0.995}$ &  2.576  \\ \hline
\end{tabular}
\end{center}
\noindent \textbf{Standard Error}
\begin{framed}
\[ S.E.(p) \;=\; \sqrt{ \frac{\hat{p} (1\;-\;\hat{p})  }{n}  } \]
\end{framed}

\begin{eqnarray*} S.E.(p) &=& \sqrt{ \frac{0.34 \times 0.66 }{200}  }\\
&=& \sqrt{ \frac{0.2244 }{200}  }\\
&=& \sqrt{ 0.001122 }\\
&=& 0.0335 \\
\end{eqnarray*}
\medskip

\noindent \textbf{Calculation of Confidence Interval}
\begin{eqnarray*}
\mbox{99\% Confidence Interval} &=& 0.34 \;\pm\; 2.576\;(0.0335)\\
&=& 0.34 \;\pm\; 0.086 \\
&=& (0.254, 0.426)\\
\end{eqnarray*}


\end{document}
