\documentclass[a4paper,12pt]{article}
	
%%%%%%%%%%%%%%%%%%%%%%%%%%%%%%%%%%%%%%%%%%%%%%%%%%%%%%%%%%%%%%%%%%%%%%%%%%%%%%%%%%%%%%%%%%%%%%%%%%%%%%%%%%%%%%%%%%%%%%%%%%%%%%%%%%%%%%%%%%%%%%%%%%%%%%%%%%%%%%%%%%%%%%%%%%%%%%%%%%%%%%%%%%%%%%%%%%%%%%%%%%%%%%%%%%%%%%%%%%%%%%%%%%%%%%%%%%%%%%%%%%%%%%%%%%%%

\usepackage{eurosym}
\usepackage{vmargin}
\usepackage{amsmath}
\usepackage{graphics}
\usepackage{epsfig}
\usepackage{enumerate}
\usepackage{multicol}
\usepackage{subfigure}
\usepackage{fancyhdr}
\usepackage{listings}
\usepackage{framed}
\usepackage{graphicx}
\usepackage{amsmath}
\usepackage{chngpage}

%\usepackage{bigints}
\usepackage{vmargin}

% left top textwidth textheight headheight

% headsep footheight footskip

\setmargins{2.0cm}{2.5cm}{16 cm}{22cm}{0.5cm}{0cm}{1cm}{1cm}

\renewcommand{\baselinestretch}{1.3}

\setcounter{MaxMatrixCols}{10}

\begin{document}
	\large
	\noindent An employment survey is carried out in order to determine the percentage, $p$, of
	unemployed people in a certain population in a way such that the estimation has a margin of error less than 0.5\% with probability at least 0.95. \\ \\
	\noindent In a similar study conducted a year ago it was found that the percentage of unemployed people in the
	population was 6\%.\\\\
	 Estimate the sample size, $n$, that is required to achieve this margin of error, by
	constructing an appropriate confidence interval (or otherwise).
	
	%%%%%%%%%%%%%%%%%%%%%%%%%%%%%%%%%%%%%%%%%%%%%%%%%%%%%%%%%%%%%%%%%%%%%%%%%%
	
	
	%%--- Question 7
	\begin{itemize}
		\item The 95\% confidence interval for the population percentage ${p}$ is
		\[ \hat{p} \pm 1.96 \sqrt{ \frac{ \hat{p} (1 - \hat{p} )}{n}}\]
		
		Re-arranging this gives us the margin of error.
		
		giving 
		\[| p - \hat{p} | \leq 1.96\sqrt{ \frac{ \hat{p} (1 - \hat{p} )}{n}}
		\]
		\item For the margin of error to be less than 0.5\% (i.e. 0.005) we need to solve
		\[0.005 \;=\; 1.96 \sqrt{ \frac{ \hat{p} (1 - \hat{p} )}{n}} \]
		Re-arranging this expression algebraically
		\[ n \;=\; 1.96^2 \left[ \frac{ \hat{p} (1 - \hat{p} )} {0.005^2} \right] \]
		
		\item Recall: In a similar study
		conducted a year ago it was found that the percentage of unemployed people in the
		population was 6\%.
		\item If not given any information, we would use $\hat{p} \;=\; 0.50$
		\medskip 
		\item Using the percentage from the previous study as the value for $\hat{p}$, i.e. $\hat{p} = 0.06$ , we obtain $n = 8,666.6$.
		
		\begin{eqnarray*} 
			n &=& 1.96^2 \times \frac{ (0.06 \times 0.94 )}{0.000025} \\
			& & \\
			&=& 3.841 \times \frac{0.0564}{0.000025}\\
			& & \\
			&=& 3.841 \times 2256 \\
			& & \\
			&=& 8665.296 \\
		\end{eqnarray*}
		
		\item So we need a sample of (at least) 8666 people.
		% ⎛ p ( 1 - p ) ⎞
		% \item (OR, solution can be based on \hat{p} ~ N ⎜ p ,
		% ⎟ and
		% n
		%
	\end{itemize}
	% \[P ( - 0.005 < \hat{p} - p < 0.005 ) > 0.95 \]without referring to the CI.)
	
	
	%%%%%%%%%%%%%%%%%%%%%%%%%%%%%%%%%%%%%%%%%%%%%%%%%%%%%%%%%%%%%%%%%%%%%%%%%%%%%%%%%%%%%%
	
	%%%%%%%%%%%%%%%%%%%%%%%%%%%%%%%%%%%%%%%%%%%%%%%%%%%%%%%%%%%%%%%%%%%%%%%%%%
\end{document}

\usepackage{subfigure}
\usepackage{fancyhdr}
\usepackage{listings}
\usepackage{framed}
\usepackage{graphicx}
\usepackage{amsmath}
\usepackage{chngpage}

%\usepackage{bigints}
\usepackage{vmargin}

% left top textwidth textheight headheight

% headsep footheight footskip

\setmargins{2.0cm}{2.5cm}{16 cm}{22cm}{0.5cm}{0cm}{1cm}{1cm}

\renewcommand{\baselinestretch}{1.3}

\setcounter{MaxMatrixCols}{10}

\begin{document}
	\large
	\noindent An employment survey is carried out in order to determine the percentage, $p$, of
	unemployed people in a certain population in a way such that the estimation has a margin of error less than 0.5\% with probability at least 0.95. 
	\medskip 
	\noindent In a similar study conducted a year ago it was found that the percentage of unemployed people in the
	population was 6\%. Calculate the sample size, $n$, that is required to achieve this margin of error, by
	constructing an appropriate confidence interval (or otherwise).
	
	%%%%%%%%%%%%%%%%%%%%%%%%%%%%%%%%%%%%%%%%%%%%%%%%%%%%%%%%%%%%%%%%%%%%%%%%%%
	
	
	%%--- Question 7
	\begin{itemize}
		\item The 95\% CI for the population percentage $\hat{p}$ is
		\[ \hat{p} \pm 1.96 \sqrt{ \frac{ \hat{p}\; (1 - \hat{p} )}{n}}\]
		
		Re-arranging this gives us the margin of error.
		
		giving 
		\[| p - \hat{p} | \leq 1.96\;\sqrt{ \frac{ \hat{p} (1 - \hat{p} )}{n}}
		\]
		\item For the margin of error to be less than 0.5\% (i.e. 0.005) we need to solve
		\[0.005 \;=\; 1.96 \sqrt{ \frac{ \hat{p} \;(1 - \hat{p} )}{n}} \]
		Re-arranging this expression algebraically
		\[ n \;=\; 1.96^2 \left[ \frac{ \hat{p} \;(1 - \hat{p} )} {0.005^2} \right] \]
		
		\item Recall: In a similar study
		conducted a year ago it was found that the percentage of unemployed people in the
		population was 6\%.
		\item If not given any information, we would use $\hat{p} \;=\; 0.50$
		\medskip 
		\item Using the percentage from the previous study as the value for $\hat{p}$, i.e. $\hat{p} = 0.06$ , we obtain $n = 8,666$.
		
		\begin{eqnarray*} 
			n &=& 1.96^2 \frac{ (0.06 \times 0.94 )} {0.000025} \\
			&=& 3.841 \times \frac{0.0564}{0.000025}\\
			&=& 3.841 \times 2256 \\
			&=& 8665.296 \\
		\end{eqnarray*}
		
		\item So we need a sample of (at least) 8666 people.
		% ⎛ p ( 1 - p ) ⎞
		% \item (OR, solution can be based on \hat{p} ~ N ⎜ p ,
		% ⎟ and
		% n
		%
	\end{itemize}
	% \[P ( - 0.005 < \hat{p} - p < 0.005 ) > 0.95 \]without referring to the CI.)
	
	
	%%%%%%%%%%%%%%%%%%%%%%%%%%%%%%%%%%%%%%%%%%%%%%%%%%%%%%%%%%%%%%%%%%%%%%%%%%%%%%%%%%%%%%
	
	%%%%%%%%%%%%%%%%%%%%%%%%%%%%%%%%%%%%%%%%%%%%%%%%%%%%%%%%%%%%%%%%%%%%%%%%%%
\end{document}
