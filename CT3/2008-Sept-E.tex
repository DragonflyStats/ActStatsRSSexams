PLEASE TURN OVER11
Male
4
6
10
1
1
A study was conducted to investigate lengths of stay, in days, of short-term stay
patients in a particular hospital. Independent random samples of 40 male patients and
35 female patients were selected and the lengths of stay of these patients are given in
the following tables:
8
6
7
7
8
2
8
5
2
1
6
7
7
8
3
9
3
7
11
9
6
5
6
5
3
4
6
9
6
1
Male: x = 215 x 2 = 1,481
10
1
2
2
3
Female
2
7
8
6
3
5
7
2
3
6
5
4
4
4
5
1
1
6
4
9
9
5
5
11
6
1
4
1
6
2
3
4
8
8
3
Female: x = 168 x 2 = 1,026
The male observations are assumed to be normally distributed with mean  1 and
standard deviation  1 , and independently the female observations are assumed to be
normally distributed with mean  2 and standard deviation  2 .
(i)
Suppose that it is known that  1 = 3.0 days and  2 = 2.5 days.
(a) Construct a 95% confidence interval for the difference between the
mean length of stay for males and the mean length of stay for females,
that is for  1   2 .
(b) Comment briefly on any implications of this confidence interval.
[6]
(ii)
Suppose now that  1 and  2 are unknown.
(a) Perform a two-sample t-test to investigate whether there is a difference
between the mean length of stay for males and the mean length of stay
for females, assuming that  1 and  2 are equal.
(b) Show that the variances in the male and female samples are not
significantly different at the 5% level, and comment briefly with
reference to the validity of the test conducted in (ii)(a).
(c) Suppose you are not prepared to assume more than you feel is
absolutely necessary – in particular you do not want to assume that
 1 and  2 are equal, nor that the observations necessarily come from
normal populations.
Perform an alternative (large-sample) test to that conducted in part
(ii)(a), “to investigate whether there is a difference between the mean
length of stay for males and the mean length of stay for females”, and
compare the results of the test with the results of the test obtained in
part (ii)(a).
[11]
[Total 17]
%%%%%%%%%%%%%%%%%%%%%%%%%%%%%%%%%%%%%%%%%%%%%%%%%%%%%%%%%%%%%%%%%%%%%%%%%%%%%%%%%%%
11
(i)
(a)
Males: n 1 = 40
x 1 = 215/40 = 5.375
Females: n 2 = 35 x 2 = 168/35 = 4.8
95% CI:
x 1 − x 2 ± z 0.025
σ 1 2 σ 2 2
+
n 1 n 2
= 5.375 – 4.8 ± 1.96
3 2 2.5 2
+
40 35
= 0.575 ± (1.96)(0.6353)
= 0.575 ± 1.245 or (–0.67, 1.82)
(b)
Page 6
As this CI includes the value 0 we would not eliminate the possibility
that the males and females have the same expected length of stay.Subject CT3 (Probability and Mathematical Statistics Core Technical) — September2008 — Examiners’ Report
(ii)
(a)
s 1 2 = (1481 – 215 2 /40)/39 = 8.34295
s 2 2 = (1026 – 168 2 /35)/34 = 6.45882
s 2 p =
( n 1 − 1) s 1 2 + ( n 2 − 1) s 2 2
(39)(8.34295) + (34)(6.45882)
=
= 7.46541
40 + 35 − 2
n 1 + n 2 − 2
Two sample t-test
x 1 − x 2
t =
⎛ 1 1 ⎞
s 2 p ⎜ + ⎟
⎝ n 1 n 2 ⎠
=
5.375 − 4.8
1 ⎞
⎛ 1
7.46541 ⎜ + ⎟
⎝ 40 35 ⎠
= 0.909
t 73 (0.025) = 1.996 (by interpolation of 2.000 and 1.980 for 60 df and
120 df)
[OR just quote the N(0,1) value 1.96 in place of the t 73 value.]
Therefore there is no evidence to reject the null hypothesis that the
means for males and females do not differ at the 5% significance level,
and we conclude that the mean lengths of stay are the same.
(b)
s 1 2
s 2 2
=
8.34295
= 1.29
6.45882
Comparing this to an F 39,34 distribution, which has a 5% critical point
between 2.075 and 1.717 (two-sided test), there is no evidence that the
population variances differ.
The assumption of common variance was made when conducting the
test in (ii)(a), and this seems valid given the result of the test in (ii)(b).
(c)
z =
=
x 1 − x 2
s 1 2 s 2 2
+
n 1 n 2
5.375 − 4.8
8.34295 6.45882
+
40
35
= 0.917
Page 7Subject CT3 (Probability and Mathematical Statistics Core Technical) — September 2008 — Examiners’ Report
Compare with N(0,1), e.g. 1.96 for 5% level test. Therefore we reach
exactly the same conclusion (as in (ii)(a) but without making the
assumptions of equal variances and normal distributions – we have
large samples and can rely on CLT).
