

\documentclass[a4paper,12pt]{article}

%%%%%%%%%%%%%%%%%%%%%%%%%%%%%%%%%%%%%%%%%%%%%%%%%%%%%%%%%%%%%%%%%%%%%%%%%%%%%%%%%%%%%%%%%%%%%%%%%%%%%%%%%%%%%%%%%%%%%%%%%%%%%%%%%%%%%%%%%%%%%%%%%%%%%%%%%%%%%%%%%%%%%%%%%%%%%%%%%%%%%%%%%%%%%%%%%%%%%%%%%%%%%%%%%%%%%%%%%%%%%%%%%%%%%%%%%%%%%%%%%%%%%%%%%%%%

\usepackage{eurosym}
\usepackage{vmargin}
\usepackage{amsmath}
\usepackage{graphics}
\usepackage{epsfig}
\usepackage{enumerate}
\usepackage{multicol}
\usepackage{subfigure}
\usepackage{fancyhdr}
\usepackage{listings}
\usepackage{framed}
\usepackage{graphicx}
\usepackage{amsmath}
\usepackage{chngpage}

%\usepackage{bigints}
\usepackage{vmargin}

% left top textwidth textheight headheight

% headsep footheight footskip

\setmargins{2.0cm}{2.5cm}{16 cm}{22cm}{0.5cm}{0cm}{1cm}{1cm}

\renewcommand{\baselinestretch}{1.3}

\setcounter{MaxMatrixCols}{10}

\begin{document}
%%-- \begin{enumerate}

%%[Total 19]
%%CT3 S2007—613
In a laboratory experiment a response variable (yield, y) is thought to be affected by a quantitative factor (percentage of catalyst, x). The experiment involved making four
observations of y at each of four values of x, being 12\%, 14\%, 16\% and 18\%, and
resulted in the following observed response data.
12%
46
51
47
42
14%
56
57
63
60
16%
56
63
60
64
18%
47
51
54
55
These data are analysed by two statisticians, A and B, who use an analysis of variance
approach and a linear regression approach, respectively.
\begin{enumerate}
\item (i)
Statistician A’s approach:
You are given the following data summaries:
sub-totals $\sum_y = 186$, 236, 243 and 207 at x = 12, 14, 16 and 18,
respectively, and overall totals $\sum_y = 872$ and $\sum_y^2 = 48,196$.
\begin{itemize}
\item (a) Apply a one-way analysis of variance to these data and obtain the
resulting F-value for the usual test.
\item (b) Show that the P-value for the test is substantially less than 0.01, by referring to tables of percentage points for the F distribution.
\item (c) The result of part (b) above shows that there is very strong evidence of
an effect on y due to the quantitative factor x. Suggest a suitable
diagram that statistician A could now use to describe the effect of x on
y. Draw this diagram and hence comment on the effect of x on y.
\item (d) The graph below shows the residuals plotted against the values of x:
Comment briefly on any implications of this graph.
\end{itemize}

%%%%%%%%%%%%%%%%%%%%%%%%%\item (ii)
\item Statistician B’s approach:
You are given the following data summaries:
\[\sum x = 240 \sum_y = 872 \sum x 2 = 3,680 \sum_y 2 = 48,196 \sum xy = 13,150.\]
(a) Perform a linear regression analysis on these data to show that the
fitted line is given by $y = 41.4 + 0.875x$.
(b) Perform the hypothesis test on the slope coefficient
$H 0 : \beta = 0 v H 1 : \beta \neq 0$
showing that the P-value is greater than 0.20.

\begin{itemize}
    \item Comment on what this implies about the relationship between x and y.

\item (c) The graph below shows the residuals plotted against the values of x:
Comment briefly on what this graph implies about the effect of x on y.
\item (d)
Suggest an additional analysis statistician B could now use to describe
the effect of x on y.
\end{itemize}
\end{enumerate}
\newpage

%%%%%%%%%%%%%%%%%%%%%%%%%%%%%%%%%%%%%%%%%%%%%%%%%%%%%55

13
\item (i)
(a)
\begin{itemize}
\item $ { \displaystyle SS_T = 48196 – 872^2 /16 = 48196 – 47524 = 672 }$
\item $ { \displaystyle SS_B = (186^2 + 236^2 + 243^2 + 207^2 )/4 – 47524 = 523.5 }$
\item $ { \displaystyle SS_R = 672 – 523.5 = 148.5 }$
\end{itemize} 
F =
523.5 / 3 174.5
=
= 14.10
148.5 /12 12.375
[or construct an ANOVA table]
\item (b)
$F_{3,12} (1\%) = 5.953$ from tables
since 14.10 >> 5.953, P-value << 0.01
\item (c)
Statistician A could plot either the individual y values or the four
means of y against x to see what “shape” the effect might take.
Here is a plot of the individual y values:
%%--Page 8  — September 2007 — %%%%%%%%%%%%%%%%%%%%%%%%%%%%%%%%
\item The shape of the effect seems to be curved, initially increasing, then
decreasing.
(d)
%% \item (ii)
(a)
\begin{itemize}
\item The implications are simply that there is nothing to invalidate the
assumptions required for the analysis.
240 2
= 3680 −
= 80
16
872 2
= 48196 −
= 672
16
S_{xx}
S_{yy}
[or could state it is the same as SS T from \item (i)]
S_{xy} = 13150 −
(240)(872)
= 70
16
70
\item $\hat{\beta}$ = = 0.875 as required
80
\alpha ˆ =

\item (b)
1
(872 − 0.875(240)) = 41.375 as required.
16
\sum ˆ 2 =
1
70 2
(672 −
) = 43.625
14
80
43.625
= 0.7385
s.e. ( \hat{beta} ) =
80
t =
0.875 − 0
= 1.185 on 14 d.f.
0.7385
P-value = 2 × P(t 14 >1.185)
\item As $P(t_{14} >1.345) = 0.10$ from tables, P-value > 2(0.10), i.e. > 0.20
\item This implies that there is no evidence against $H_0$ , and hence that there
is no linear relationship between x and y.
[not that there is no “relationship”]
(c) Residual plot suggests that there may be a curved, rather than linear,
relationship between x and y.
(d) Statistician B could try a quadratic regression (or some other curved
form) of y on x.
\end{itemize}
END OF %%%%%%%%%%%%%%%%%%%%%%%%%%%%%%%%
Page 9
\end{document}
