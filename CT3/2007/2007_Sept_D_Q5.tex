\documentclass[a4paper,12pt]{article}

%%%%%%%%%%%%%%%%%%%%%%%%%%%%%%%%%%%%%%%%%%%%%%%%%%%%%%%%%%%%%%%%%%%%%%%%%%%%%%%%%%%%%%%%%%%%%%%%%%%%%%%%%%%%%%%%%%%%%%%%%%%%

\usepackage{eurosym}
\usepackage{vmargin}
\usepackage{amsmath}
\usepackage{graphics}
\usepackage{epsfig}
\usepackage{enumerate}
\usepackage{multicol}
\usepackage{subfigure}
\usepackage{fancyhdr}
\usepackage{listings}
\usepackage{framed}
\usepackage{graphicx}
\usepackage{amsmath}
\usepackage{chngpage}

%\usepackage{bigints}
\usepackage{vmargin}

% left top textwidth textheight headheight

% headsep footheight footskip

\setmargins{2.0cm}{2.5cm}{16 cm}{22cm}{0.5cm}{0cm}{1cm}{1cm}

\renewcommand{\baselinestretch}{1.3}

\setcounter{MaxMatrixCols}{10}

\begin{document}


%%%%%%%%%%%%%%%%%%%%%%%%%%%%%%%%%%%%%%%%%%%%%%%%%%%%%%%%%%%%%%%%%%%%%%%%%%%%%%%%%%
%CT3 S2007—25
\item For a particular insurance company a sample of eight claim amounts (in units of
8
\$1,000) on household contents is taken. The data give
$\sum  X_{i} = 56.7$ and
i = 1
8
$$\sum  X_{i} 2 = 403.95$ .
The claim amounts are assumed to follow a normal distribution.
i = 1
6
\begin{enumerate}
\item (i) Calculate a 90\% confidence interval for the true mean claim amount.
\item (ii) Use the confidence interval calculated in (i) above to comment on an expert’s
assessment that the average claim amount for the company is \$6,500.
\end{enumerate}
\newpage


%%--- Subject CT3  — September 2007 — Examiners’ Report
5
(i)
56.7
= 7.0875.
8
1 ⎛
56.7 2 ⎞
s 2 = ⎜ 403.95 −
⎟ = 0.298 ⇒ s = 0.546 .
7 ⎜ ⎝
8 ⎟ ⎠
x =
90% CI for the true mean is given by:
x \pm t 7,0.05
s
0.546
= 7.0875 \pm 1.895
= 7.0875 \pm 0.3658
n
8
i.e. the 90% CI is (6.722, 7.453), or (\$6722, \$7453).
(ii)
\item The value \$6500 is not included in the CI above, and therefore we conclude
that the data are not consistent with the expert’s assessment at the 10%
significance level.

\end{document}
