\documentclass[a4paper,12pt]{article}

%%%%%%%%%%%%%%%%%%%%%%%%%%%%%%%%%%%%%%%%%%%%%%%%%%%%%%%%%%%%%%%%%%%%%%%%%%%%%%%%%%%%%%%%%%%%%%%%%%%%%%%%%%%%%%%%%%%%%%%%%%%%%%%%%%%%%%%%%%%%%%%%%%%%%%%%%%%%%%%%%%%%%%%%%%%%%%%%%%%%%%%%%%%%%%%%%%%%%%%%%%%%%%%%%%%%%%%%%%%%%%%%%%%%%%%%%%%%%%%%%%%%%%%%%%%%

\usepackage{eurosym}
\usepackage{vmargin}
\usepackage{amsmath}
\usepackage{graphics}
\usepackage{epsfig}
\usepackage{enumerate}
\usepackage{multicol}
\usepackage{subfigure}
\usepackage{fancyhdr}
\usepackage{listings}
\usepackage{framed}
\usepackage{graphicx}
\usepackage{amsmath}
\usepackage{chngpage}

%\usepackage{bigints}
\usepackage{vmargin}

% left top textwidth textheight headheight

% headsep footheight footskip

\setmargins{2.0cm}{2.5cm}{16 cm}{22cm}{0.5cm}{0cm}{1cm}{1cm}

\renewcommand{\baselinestretch}{1.3}

\setcounter{MaxMatrixCols}{10}

\begin{document}
\begin{enumerate}


12
An insurance company is investigating past data for two household claims assessors,
A and B, used by the company. In particular claims resulting from similar types of
water damage were extracted. The following table shows the assessors’ initial
estimates of the cost (in units of £100) of meeting each claim.
A:
B:
4.6
5.7
6.6
3.4
2.8
4.7
5.8
3.6
2.1
6.5
5.2
3.3
5.9
3.8
3.4
2.4
7.8
7.0
3.5
4.0
1.6
4.4
8.6
2.7
for the A data: n A = 13, \sum x = 60.6 and \sum x 2 = 340.92
for the B data: n B = 11, \sum x = 48.8 and \sum x 2 = 236.80
\item (i) Draw a suitable diagram to represent these data so that the initial estimates of
the two assessors can be compared.

\item (ii) You are required to perform an appropriate test to compare the means of the
assessors’ initial estimates for this type of water damage.
(a) State your hypotheses clearly.
(b) Use your diagram in part \item (i) to comment briefly on the validity of your
test.
(c) Calculate your test statistic and specify the resulting P-value.
(d) State your conclusion clearly.
%%%%%%%%%%%%%%%%%%%%%%%%%%%%%%%%%%%%%%%%%%%%%%%%%%%%
\item (iii)
You are required to perform an appropriate test to compare the variances of
the assessors’ initial estimates.
(a) State your hypotheses clearly.
(b) Use your diagram in part \item (i) to comment briefly on the validity of your
test.
(c) Calculate your test statistic and specify the resulting P-value.
(d) State your conclusion clearly and hence comment further on the validity of your test in part \item (ii).

(iv)
Use your answers in parts \item (i) to \item (iii) to comment on the overall comparison of the two assessors as regards their initial estimates for this type of water damage.

[Total 20]
%%%%%%%%%%%%%%%%%%%%%%%%%%%%%%%%%%%%%%%%%%%%%%%%%%%%%%%%%%%%%%%%%%%%%%%%%%%%%%%%%%%%%%%%%5
CT3 A2007—613
In a study of the relation between the amount of information available and use of
buses in eight comparable test cities, bus route maps were given to residents of the cities at the beginning of the test period. The increase in average daily bus use during the test period was recorded. The numbers of maps and the increase in bus use are
given in the table below (both in thousands).
Number of maps (x)
80 220 140 120 180 100 200 160
Increase in bus use (y) 0.60 6.70 5.30 4.00 6.55 2.15 6.60 5.75
For these data:
∑ x = 1, 200 , ∑ x 2 = 196,800 , ∑ y = 37.65 , ∑ y 2 = 213.4875 , ∑ xy = 6,378
\item (i) Construct a scatterplot of the data and comment on the relationship between the increase in bus use and the number of maps distributed.

\item (ii) The equation of the fitted linear regression is given by y = − 1.816 + 0.04348 x .
Perform an appropriate statistical test to assess the hypothesis that the slope in this fitted model suggests no relationship between the increase in bus use and the number of maps distributed. Any assumptions made should be clearly
stated.
[6]
\item (iii) The fitted responses and the residuals from the linear regression model fitted
in part \item (ii) are given below:
Fitted values ( y ˆ )
Residuals ( e ˆ )
1.66 7.75 4.27 3.40 6.01 2.53 6.88 5.14
- 1.06 - 1.05 1.03 0.60 0.54 - 0.38 - 0.28 0.61
Plot the residuals against the values of the fitted responses and comment on
the adequacy of the model.

(iv)
A new city is added to the study, and 250,000 maps are distributed to its
citizens.
Calculate the prediction of the increase in bus use in this city according to the model fitted in part \item (ii) and comment on the validity of this prediction.



%%%%%%%%%%%%%%%%%%%%%%%%%%%%%%%%%%%%%%%%%%%%%%%%%%%%%%%%%%%%%%%%%%%%%%%%%%%%%%%%%%
12
\item (i)
dotplots on same scale are most suitable
[alternatively boxplots are acceptable]
\item (ii)
(a)
Let μ A = mean initial estimate for this type of water damage for assessor A and μ B = mean initial estimate for this type of water damage
for assessor B .
H 0 : μ A = μ B v H 1 : μ A \neq μ B
(b) dotplots show that normality assumption is reasonably valid
dotplots perhaps cast doubt on equal variances assumption
(c) test statistic is t =
From data: x A = 60.6
1
60.6 2
) = 4.8692
= 4.662 , s 2 A = (340.92 −
13
12
13
x B = 48.8
1
48.8 2
) = 2.0305
= 4.436 , s B 2 = (236.80 −
11
10
11
s 2 p = 12(4.8692) + 10(2.0305)
= 3.5789 ∴ s p = 1.8918
22
and
observed t =
Page 8
x A − x B
∼ t n A + n B − 2 under H 0
1
1
s p
+
n A n B
4.662 − 4.436
0.226
=
= 0.29 on 22 df
0.775
1 1
1.8918
+
13 11Subject CT3  — April 2007 — %%%%%%%%%%%%%%%%%%%%%%%%%%%%%%%%
Clearly P-value is very large, or noting that t 22 (40%) = 0.2564, then
P-value is just a bit less than 0.8.
\item (iii)
(d) So there is no evidence at all of any difference between assessors A and B as regards their mean initial estimates for this type of water damage.
(a) H 0 : \sigma^2 A = \sigma^2 B v H 1 : \sigma^2 A \neq \sigma^2 B
(b) as in \item (i) dotplots show that normality assumption is reasonably valid
(c) test statistic is F =
observed F =
s 2 A
s B 2
∼ F n A − 1, n B − 1 under H 0
4.8692
= 2.40 on 12,10 df
2.0305
F 12,10 (10%) = 2.284 and F 12,10 (5\%) = 2.913
Thus P-value is between 0.10 and 0.20.
(d)
So there is no real evidence of any difference between assessors A and B as regards the variances of their initial estimates for this type of water damage.
This validates the possibly doubtful assumption required in part \item (ii).
(iv)
Overall there is no real evidence to distinguish any differences in the initial
estimates for this type of water damage for the two assessors A and B.
Page 9Subject CT3  — April 2007 — %%%%%%%%%%%%%%%%%%%%%%%%%%%%%%%%
\item (i)
The scatterplot is shown below.
13
80
100
120
140
160
180
200
220
maps
The plot suggests that there is a positive relationship between the increase in  bus use and the number of maps distributed. The increase seems to be reasonably linear up to around 180000 maps, after which point it seems to
level off (overall, relationship seems curved, possibly quadratic).
\item (ii)
S_{xx} = \sum x 2 − (\sum x) 2 /n = 196800 − (1200) 2 /8 = 16800
S_{yy} = \sigmay 2 − (\sigmay) 2 /n = 213.4875 − (37.65) 2 /8 = 36.29719
S_{xy} = \sum xy − (\sum x)( \sigmay)/n = 6378 − (1200)(37.65)/8 = 730.5
2 ⎞
S_{xy} 2 ⎞ 1 ⎛
1 ⎛ ⎜
⎟ = ⎜ 36.29719 − (730.5) ⎟ = 0.75558
\sigmâ =
S_{yy} −
n − 2 ⎜ ⎝
S_{xx} ⎟ ⎠ 6 ⎜ ⎝
16800 ⎟ ⎠
2
s.e.( \betâ ) =
0.75558
\sigma ˆ 2
=
= 0.006706
16800
S_{xx}
To test H 0 : \beta = 0 v H 1 : \beta \neq 0 , the test statistic is
0.04348
\hat{\beta} − 0
=
= 6.484 ,
s.e.( \hat{\beta} ) 0.006706
and under the assumption that the errors of the regression are i.i.d . N (0, \sigma^2 )
random variables, it has a t distribution with $n - 2 = 6$ df.
Page 10

From statistical tables we find t 6,0.025 = 2.447 (or, t 6,0.005 = 3.707 ).
Therefore, there is strong evidence against H 0 . We conclude that a straight line representation of the relationship between the increase in bus use and the number of maps distributed would have a non-zero slope.
The plot is shown below.
\item (iii)
2
3
4
5
6
7
Fitted
Negative residuals are associated with the fitted values at the two ends of the
data set, suggesting that the model is inadequate. Pattern suggests that a
quadratic model might be appropriate.
(iv)
Predicted value is y ˆ = − 1.816 + 0.04348 × 250 = 9.054
This uses extrapolation on the fitted regression line. The prediction is probably not valid, especially as the linear model does not seem adequate.
END OF %%%%%%%%%%%%%%%%%%%%%%%%%%%%%%%%
Page 11
\end{document}
