
\documentclass[a4paper,12pt]{article}

%%%%%%%%%%%%%%%%%%%%%%%%%%%%%%%%%%%%%%%%%%%%%%%%%%%%%%%%%%%%%%%%%%%%%%%%%%%%%%%%%%%%%%%%%%%%%%%%%%%%%%%%%%%%%%%%%%%%%%%%%%%%%%%%%%%%%%%%%%%%%%%%%%%%%%%%%%%%%%%%%%%%%%%%%%%%%%%%%%%%%%%%%%%%%%%%%%%%%%%%%%%%%%%%%%%%%%%%%%%%%%%%%%%%%%%%%%%%%%%%%%%%%%%%%%%%

\usepackage{eurosym}
\usepackage{vmargin}
\usepackage{amsmath}
\usepackage{graphics}
\usepackage{epsfig}
\usepackage{enumerate}
\usepackage{multicol}
\usepackage{subfigure}
\usepackage{fancyhdr}
\usepackage{listings}
\usepackage{framed}
\usepackage{graphicx}
\usepackage{amsmath}
\usepackage{chngpage}

%\usepackage{bigints}
\usepackage{vmargin}

% left top textwidth textheight headheight

% headsep footheight footskip

\setmargins{2.0cm}{2.5cm}{16 cm}{22cm}{0.5cm}{0cm}{1cm}{1cm}

\renewcommand{\baselinestretch}{1.3}

\setcounter{MaxMatrixCols}{10}

\begin{document}
The number of claims, X , which arise in a year on each policy of a particular class is to be modelled as a Poisson random variable with mean \lambda. Let X = ( X 1 , X 2 , ..., X n ) be
1 n
a random sample of size n from the distribution of X , and let X = ∑ X i .
n i = 1
Suppose that it is required to estimate \lambda , the mean number of claims on a policy.
\begin{enumerate}[(i)]
\item Show that $\hat{\lambdâ}$, the maximum likelihood estimator of $\lambda$, is given by $\hat{\lambda} = X $. 
\item Derive the Cramer-Rao lower bound (CRlb) for the variance of unbiased estimators of $\lambda$.
\item 

(a) Show that \lambdâ is unbiased for \lambda and that it attains the CRlb.
(b) Explain clearly why, in the case that n is large, the distribution of \lambdâ can be approximated by
⎛ \lambda ⎞
\hat{\lambda} ~ N ⎜ \lambda , ⎟ .
⎝ n ⎠

(iv)
(a)
Show that, in the case $n = 100$, an approximate 95\% confidence interval for $\lambda$ is given by
x \pm 0.196 x .
(b)
\end{enumerate}
Evaluate the confidence interval in (iv)(a) based on a sample with the
following composition:
observation
frequency
0
11
1
28
2
19
3
28
4
9
5
2
6
2
7
1
[6]


%%%%%%%%%%%%%%%%%%%%%%%%%%%%%%%%%%%%%%%%%%%%%%%%%%%%%%%%%%%%%%%%%%%%%%%%%%%%%%%%%%
11
\item (i)
L ( x ) =
( \lambda ) = log L ( \lambda ) = − n \lambda + ( ∑ x i ) log \lambda + constant
⇒
⇒
\item (ii)
e − n \lambda \lambda ∑ x i
∏ x i !
d
∑ x i = 0 ⇒ \hat{\lambda} =
= − n +
\lambda
d \lambda
d 2
d \lambda 2
=−
∑ X i = X
n
∑ x i
\lambda 2
⎡ d 2 ⎤ 1
n \lambda n
⇒ − E ⎢ 2 ⎥ = 2 E ⎡ ⎣ ∑ X i ⎤ ⎦ = 2 =
\lambda
\lambda
⎢ ⎣ d \lambda ⎥ ⎦ \lambda
⇒ CRlb =
\item (iii)
(a)
\lambda
.
n
E ⎡ ⎣ \hat{\lambda} ⎤ ⎦ = E ⎡ ⎣ X ⎤ ⎦ = E [ X ] = \lambda
V ⎡ ⎣ \hat{\lambda} ⎤ ⎦ = V ⎡ ⎣ X ⎤ ⎦ =
(iv)
V [ X ]
n
=
\lambda
,which is CRlb.
n
(b) The theory of asymptotic distributions of MLEs (and in this case the
⎛ \lambda ⎞
CLT) gives \hat{\lambda} ~ N approximately, for large n so \hat{\lambda} ~ N ⎜ \lambda , ⎟ ,
⎝ n ⎠
approximately.
(a) Large sample approximate 95\% CI for \lambda is given by
(
( ) )
\hat{\lambda} \pm 1.96 \times  s . e . \hat{\lambda}
i.e. x \pm ( 1.96 \times  s . e . ( x ) )
()
\lambda
s . e . \hat{\lambda} =
which we can estimate by using x to estimate \lambda , giving
n
()
the estimated standard error e . s . e . \hat{\lambda} =
x
n
With n = 100, we get the 95\% CI as
⎛
x ⎞
x \pm ⎜ ⎜ 1.96 \times 
⎟ i . e . x \pm 0.196 x .
100 ⎟ ⎠
⎝
Page 7Subject CT3 (Probability and Mathematical Statistics Core Technical) — April 2007 — %%%%%%%%%%%%%%%%%%%%%%%%%%%%%%%%
(b)
x = 215 /100 = 2.15
CI is 2.15 \pm 0.196(2.15) 1/2 i.e. 2.15 \pm 0.287 i.e. (1.86, 2.44)
\end{document}
