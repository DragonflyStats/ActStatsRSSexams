
\documentclass[a4paper,12pt]{article}

%%%%%%%%%%%%%%%%%%%%%%%%%%%%%%%%%%%%%%%%%%%%%%%%%%%%%%%%%%%%%%%%%%%%%%%%%%%%%%%%%%%%%%%%%%%%%%%%%%%%%%%%%%%%%%%%%%%%%%%%%%%%%%%%%%%%%%%%%%%%%%%%%%%%%%%%%%%%%%%%%%%%%%%%%%%%%%%%%%%%%%%%%%%%%%%%%%%%%%%%%%%%%%%%%%%%%%%%%%%%%%%%%%%%%%%%%%%%%%%%%%%%%%%%%%%%

\usepackage{eurosym}
\usepackage{vmargin}
\usepackage{amsmath}
\usepackage{graphics}
\usepackage{epsfig}
\usepackage{enumerate}
\usepackage{multicol}
\usepackage{subfigure}
\usepackage{fancyhdr}
\usepackage{listings}
\usepackage{framed}
\usepackage{graphicx}
\usepackage{amsmath}
\usepackage{chngpage}

%\usepackage{bigints}
\usepackage{vmargin}

% left top textwidth textheight headheight

% headsep footheight footskip

\setmargins{2.0cm}{2.5cm}{16 cm}{22cm}{0.5cm}{0cm}{1cm}{1cm}

\renewcommand{\baselinestretch}{1.3}

\setcounter{MaxMatrixCols}{10}

\begin{document}
\begin{enumerate}

[Total 5]
A random sample of insurance policies of a certain type was examined for each of
four insurance companies and the sums insured ( y ij , for companies i = 1, 2, 3, 4)
under each policy are given in the table below (in units of £100):
Company
1
2
3
4
For these data,
CT3 A2007—4
Total
58.2
56.3
50.1
52.9
57.2
54.5
54.2
49.9
58.4
57.0
55.5
50.0
55.8
55.3
51.7
54.9
284.5
223.1
159.8
204.5
∑ i ∑ j y ij = 871.9 and ∑ i ∑ j y ij 2 = 47, 633.53Consider the ANOVA model Y ij = μ + τ i + e ij , i = 1,..., 4, j = 1,..., n i , where Y ij is the j th
sum insured for company i , n i is the number of responses for company i ,
e ij ~ N (0, \sigma^2 ) are independent errors, and
11
∑ i = 1 n i τ i = 0 .
4
\item (i) Calculate estimates of the parameters μ and τ i , i = 1, 2, 3, 4 .
\item (ii) Test the hypothesis that there are no differences in the means of the sums
insured under such policies by the four companies.

[Total 8]


%%%%%%%%%%%%%%%%%%%%%%%%%%%%%%%%%%%
The number of claims, X , which arise in a year on each policy of a particular class is to be modelled as a Poisson random variable with mean \lambda. Let X = ( X 1 , X 2 , ..., X n ) be
1 n
a random sample of size n from the distribution of X , and let X = ∑ X i .
n i = 1
Suppose that it is required to estimate \lambda , the mean number of claims on a policy.
\item (i) Show that \lambdâ , the maximum likelihood estimator of \lambda, is given by \hat{\lambda} = X . 
\item (ii) Derive the Cramer-Rao lower bound (CRlb) for the variance of unbiased estimators of \lambda.
\item (iii)

(a) Show that \lambdâ is unbiased for \lambda and that it attains the CRlb.
(b) Explain clearly why, in the case that n is large, the distribution of \lambdâ can be approximated by
⎛ \lambda ⎞
\hat{\lambda} ~ N ⎜ \lambda , ⎟ .
⎝ n ⎠

(iv)
(a)
Show that, in the case n = 100, an approximate 95\% confidence interval for \lambda is given by
x \pm 0.196 x .
(b)
Evaluate the confidence interval in (iv)(a) based on a sample with the
following composition:
observation
frequency
0
11
1
28
2
19
3
28
4
9
5
2
6
2
7
1
[6]
[Total 16]
CT3 A2007—5

%%%%%%%%%%%%%%%%%%%%%%%%%%%%%%%%%%%%%%%%%%%%%%%%%%%%%%%%%%%%%%%%%%%%%%%%%%%%%%%%%%
Page 5Subject CT3 (Probability and Mathematical Statistics Core Technical) — April 2007 — %%%%%%%%%%%%%%%%%%%%%%%%%%%%%%%%
\item (i)
10
\item (ii)
μ ˆ = Y .. =
τ ˆ 1 = Y 1. − Y .. = 284.5 871.9
−
= 2.406
5
16
τ ˆ 2 = Y 2. − Y .. = 223.1 871.9
−
= 1.281
4
16
τ ˆ 3 = Y 3. − Y .. = 159.8 871.9
−
= − 1.227
3
16
τ ˆ 4 = Y 4. − Y .. = 204.5 871.9
−
= − 3.369
4
16
SS T = ∑∑ y ij 2 −
i
SS B =
871.9
= 54.494
16
j
Y .. 2
= 120.430
n
∑ n i ( Y i . − Y .. ) 2 = (5 × 2.406 2 ) + (4 ×1.281 2 ) + (3 × 1.227 2 ) + (4 × 3.369 2 ) = 85.425
i
⎛ Y 2 ⎞ Y 2
[OR : SS B = ∑ ⎜ i . ⎟ − .. = 85.428]
⎜
⎟ n
i ⎝ n i ⎠
SS R = SS T − SS B = 35.002
The ANOVA table is:
Source
DF
SS
MS
F
Company (between treatments) 3
85.428 28.476 9.763
Residual
12 35.002 2.917
Total
15 120.430
At the 5\% significance level, F 0.05,3,12 = 3.490 (or F 0.01,3,12 = 5.953 )
Since F = 9.763 > 3.490, there is evidence against the null hypothesis, and we conclude that there are differences in the mean sums insured by the companies.
% Page 6
% Subject CT3 (Probability and Mathematical Statistics Core Technical) — April 2007 — %%%%%%%%%%%%%%%%%%%%%%%%%%%%%%%%
11
\item (i)
L ( x ) =
( \lambda ) = log L ( \lambda ) = − n \lambda + ( ∑ x i ) log \lambda + constant
⇒
⇒
\item (ii)
e − n \lambda \lambda ∑ x i
∏ x i !
d
∑ x i = 0 ⇒ \hat{\lambda} =
= − n +
\lambda
d \lambda
d 2
d \lambda 2
=−
∑ X i = X
n
∑ x i
\lambda 2
⎡ d 2 ⎤ 1
n \lambda n
⇒ − E ⎢ 2 ⎥ = 2 E ⎡ ⎣ ∑ X i ⎤ ⎦ = 2 =
\lambda
\lambda
⎢ ⎣ d \lambda ⎥ ⎦ \lambda
⇒ CRlb =
\item (iii)
(a)
\lambda
.
n
E ⎡ ⎣ \hat{\lambda} ⎤ ⎦ = E ⎡ ⎣ X ⎤ ⎦ = E [ X ] = \lambda
V ⎡ ⎣ \hat{\lambda} ⎤ ⎦ = V ⎡ ⎣ X ⎤ ⎦ =
(iv)
V [ X ]
n
=
\lambda
,which is CRlb.
n
(b) The theory of asymptotic distributions of MLEs (and in this case the
⎛ \lambda ⎞
CLT) gives \hat{\lambda} ~ N approximately, for large n so \hat{\lambda} ~ N ⎜ \lambda , ⎟ ,
⎝ n ⎠
approximately.
(a) Large sample approximate 95\% CI for \lambda is given by
(
( ) )
\hat{\lambda} \pm 1.96 × s . e . \hat{\lambda}
i.e. x \pm ( 1.96 × s . e . ( x ) )
()
\lambda
s . e . \hat{\lambda} =
which we can estimate by using x to estimate \lambda , giving
n
()
the estimated standard error e . s . e . \hat{\lambda} =
x
n
With n = 100, we get the 95\% CI as
⎛
x ⎞
x \pm ⎜ ⎜ 1.96 ×
⎟ i . e . x \pm 0.196 x .
100 ⎟ ⎠
⎝
Page 7Subject CT3 (Probability and Mathematical Statistics Core Technical) — April 2007 — %%%%%%%%%%%%%%%%%%%%%%%%%%%%%%%%
(b)
x = 215 /100 = 2.15
CI is 2.15 \pm 0.196(2.15) 1/2 i.e. 2.15 \pm 0.287 i.e. (1.86, 2.44)
\end{document}
