
\documentclass[a4paper,12pt]{article}

%%%%%%%%%%%%%%%%%%%%%%%%%%%%%%%%%%%%%%%%%%%%%%%%%%%%%%%%%%%%%%%%%%%%%%%%%%%%%%%%%%%%%%%%%%%%%%%%%%%%%%%%%%%%%%%%%%%%%%%%%%%%%%%%%%%%%%%%%%%%%%%%%%%%%%%%%%%%%%%%%%%%%%%%%%%%%%%%%%%%%%%%%%%%%%%%%%%%%%%%%%%%%%%%%%%%%%%%%%%%%%%%%%%%%%%%%%%%%%%%%%%%%%%%%%%%

\usepackage{eurosym}
\usepackage{vmargin}
\usepackage{amsmath}
\usepackage{graphics}
\usepackage{epsfig}
\usepackage{enumerate}
\usepackage{multicol}
\usepackage{subfigure}
\usepackage{fancyhdr}
\usepackage{listings}
\usepackage{framed}
\usepackage{graphicx}
\usepackage{amsmath}
\usepackage{chngpage}

%\usepackage{bigints}
\usepackage{vmargin}

% left top textwidth textheight headheight

% headsep footheight footskip

\setmargins{2.0cm}{2.5cm}{16 cm}{22cm}{0.5cm}{0cm}{1cm}{1cm}

\renewcommand{\baselinestretch}{1.3}

\setcounter{MaxMatrixCols}{10}

\begin{document}
\begin{enumerate}

[Total 5]
A random sample of insurance policies of a certain type was examined for each of
four insurance companies and the sums insured ( y ij , for companies i = 1, 2, 3, 4)
under each policy are given in the table below (in units of £100):
Company
1
2
3
4
For these data,
CT3 A2007—4
Total
58.2
56.3
50.1
52.9
57.2
54.5
54.2
49.9
58.4
57.0
55.5
50.0
55.8
55.3
51.7
54.9
284.5
223.1
159.8
204.5
∑ i ∑ j y ij = 871.9 and ∑ i ∑ j y ij 2 = 47, 633.53Consider the ANOVA model Y ij = μ + τ i + e ij , i = 1,..., 4, j = 1,..., n i , where Y ij is the j th
sum insured for company i , n i is the number of responses for company i ,
e ij ~ N (0, σ 2 ) are independent errors, and
11
∑ i = 1 n i τ i = 0 .
4
(i) Calculate estimates of the parameters μ and τ i , i = 1, 2, 3, 4 .
(ii) Test the hypothesis that there are no differences in the means of the sums
insured under such policies by the four companies.
[5]
[Total 8]
[3]

%%%%%%%%%%%%%%%%%%%%%%%%%%%%%%%%%%%
The number of claims, X , which arise in a year on each policy of a particular class is to be modelled as a Poisson random variable with mean λ. Let X = ( X 1 , X 2 , ..., X n ) be
1 n
a random sample of size n from the distribution of X , and let X = ∑ X i .
n i = 1
Suppose that it is required to estimate λ , the mean number of claims on a policy.
(i) Show that λ̂ , the maximum likelihood estimator of λ, is given by λ ˆ = X . [3]
(ii) Derive the Cramer-Rao lower bound (CRlb) for the variance of unbiased estimators of λ.
(iii)
[4]
(a) Show that λ̂ is unbiased for λ and that it attains the CRlb.
(b) Explain clearly why, in the case that n is large, the distribution of λ̂ can be approximated by
⎛ λ ⎞
λ ˆ ~ N ⎜ λ , ⎟ .
⎝ n ⎠
[3]
(iv)
(a)
Show that, in the case n = 100, an approximate 95\% confidence interval for λ is given by
x ± 0.196 x .
(b)
Evaluate the confidence interval in (iv)(a) based on a sample with the
following composition:
observation
frequency
0
11
1
28
2
19
3
28
4
9
5
2
6
2
7
1
[6]
[Total 16]
CT3 A2007—5

%%%%%%%%%%%%%%%%%%%%%%%%%%%%%%%%%%%%%%%%%%%%%%%%%%%%%%%%%%%%%%%%%%%%%%%%%%%%%%%%%%
Page 5Subject CT3 (Probability and Mathematical Statistics Core Technical) — April 2007 — Examiners’ Report
(i)
10
(ii)
μ ˆ = Y .. =
τ ˆ 1 = Y 1. − Y .. = 284.5 871.9
−
= 2.406
5
16
τ ˆ 2 = Y 2. − Y .. = 223.1 871.9
−
= 1.281
4
16
τ ˆ 3 = Y 3. − Y .. = 159.8 871.9
−
= − 1.227
3
16
τ ˆ 4 = Y 4. − Y .. = 204.5 871.9
−
= − 3.369
4
16
SS T = ∑∑ y ij 2 −
i
SS B =
871.9
= 54.494
16
j
Y .. 2
= 120.430
n
∑ n i ( Y i . − Y .. ) 2 = (5 × 2.406 2 ) + (4 ×1.281 2 ) + (3 × 1.227 2 ) + (4 × 3.369 2 ) = 85.425
i
⎛ Y 2 ⎞ Y 2
[OR : SS B = ∑ ⎜ i . ⎟ − .. = 85.428]
⎜
⎟ n
i ⎝ n i ⎠
SS R = SS T − SS B = 35.002
The ANOVA table is:
Source
DF
SS
MS
F
Company (between treatments) 3
85.428 28.476 9.763
Residual
12 35.002 2.917
Total
15 120.430
At the 5% significance level, F 0.05,3,12 = 3.490 (or F 0.01,3,12 = 5.953 )
Since F = 9.763 > 3.490, there is evidence against the null hypothesis, and we conclude that there are differences in the mean sums insured by the companies.
% Page 6
% Subject CT3 (Probability and Mathematical Statistics Core Technical) — April 2007 — Examiners’ Report
11
(i)
L ( x ) =
( λ ) = log L ( λ ) = − n λ + ( ∑ x i ) log λ + constant
⇒
⇒
(ii)
e − n λ λ ∑ x i
∏ x i !
d
∑ x i = 0 ⇒ λ ˆ =
= − n +
λ
d λ
d 2
d λ 2
=−
∑ X i = X
n
∑ x i
λ 2
⎡ d 2 ⎤ 1
n λ n
⇒ − E ⎢ 2 ⎥ = 2 E ⎡ ⎣ ∑ X i ⎤ ⎦ = 2 =
λ
λ
⎢ ⎣ d λ ⎥ ⎦ λ
⇒ CRlb =
(iii)
(a)
λ
.
n
E ⎡ ⎣ λ ˆ ⎤ ⎦ = E ⎡ ⎣ X ⎤ ⎦ = E [ X ] = λ
V ⎡ ⎣ λ ˆ ⎤ ⎦ = V ⎡ ⎣ X ⎤ ⎦ =
(iv)
V [ X ]
n
=
λ
,which is CRlb.
n
(b) The theory of asymptotic distributions of MLEs (and in this case the
⎛ λ ⎞
CLT) gives λ ˆ ~ N approximately, for large n so λ ˆ ~ N ⎜ λ , ⎟ ,
⎝ n ⎠
approximately.
(a) Large sample approximate 95% CI for λ is given by
(
( ) )
λ ˆ ± 1.96 × s . e . λ ˆ
i.e. x ± ( 1.96 × s . e . ( x ) )
()
λ
s . e . λ ˆ =
which we can estimate by using x to estimate λ , giving
n
()
the estimated standard error e . s . e . λ ˆ =
x
n
With n = 100, we get the 95% CI as
⎛
x ⎞
x ± ⎜ ⎜ 1.96 ×
⎟ i . e . x ± 0.196 x .
100 ⎟ ⎠
⎝
Page 7Subject CT3 (Probability and Mathematical Statistics Core Technical) — April 2007 — Examiners’ Report
(b)
x = 215 /100 = 2.15
CI is 2.15 ± 0.196(2.15) 1/2 i.e. 2.15 ± 0.287 i.e. (1.86, 2.44)
\end{document}
