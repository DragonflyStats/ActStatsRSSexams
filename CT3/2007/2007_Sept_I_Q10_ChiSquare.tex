
\documentclass[a4paper,12pt]{article}

%%%%%%%%%%%%%%%%%%%%%%%%%%%%%%%%%%%%%%%%%%%%%%%%%%%%%%%%%%%%%%%%%%%%%%%%%%%%%%%%%%%%%%%%%%%%%%%%%%%%%%%%%%%%%%%%%%%%%%%%%%%%%%%%%%%%%%%%%%%%%%%%%%%%%%%%%%%%%%%%%%%%%%%%%%%%%%%%%%%%%%%%%%%%%%%%%%%%%%%%%%%%%%%%%%%%%%%%%%%%%%%%%%%%%%%%%%%%%%%%%%%%%%%%%%%%

\usepackage{eurosym}
\usepackage{vmargin}
\usepackage{amsmath}
\usepackage{graphics}
\usepackage{epsfig}
\usepackage{enumerate}
\usepackage{multicol}
\usepackage{subfigure}
\usepackage{fancyhdr}
\usepackage{listings}
\usepackage{framed}
\usepackage{graphicx}
\usepackage{amsmath}
\usepackage{chngpage}

%\usepackage{bigints}
\usepackage{vmargin}

% left top textwidth textheight headheight

% headsep footheight footskip

\setmargins{2.0cm}{2.5cm}{16 cm}{22cm}{0.5cm}{0cm}{1cm}{1cm}

\renewcommand{\baselinestretch}{1.3}

\setcounter{MaxMatrixCols}{10}

\begin{document}



A chi-square test of association for the frequency data in the following $2 × 3$ table

\begin{center}
\begin{tabular}{|c|c||c|c|c} \hline
 & Factor A & A1 & A2 & A3 \\   \hline 
Factor B &  B1  &  &  &  \\ \hline
&  B2  &  &  &  \\ \hline
&  B3  &  &  &  \\ \hline
\end{tabular}
\end{center}

B1
B2

40
30
50
80
30
70
produces a chi-square statistic with value 4.861 and associated P-value 0.089.
Consider a chi-square test of association for the data in the following 2 × 3 table, in
which all frequencies are twice the corresponding frequencies in the first table:
\begin{center}
\begin{tabular}{|c|c||c|c|c} \hline
 & Factor A & A1 & A2 & A3 \\   \hline 
Factor B &  B1  &  &  &  \\ \hline
&  B2  &  &  &  \\ \hline
&  B3  &  &  &  \\ \hline
\end{tabular}
\end{center}
Factor B
B1
B2
Factor A
A1
A2
A3
80
60
100
160
60
140
\begin{enumerate}[(a)]
\item (i) State, or calculate, the value of the chi-square test statistic for the second table.

\item(ii) Find the P-value associated with the test statistic in (i).
\item(iii) Comment on the results.
\end{enumerate}

%%%%%%%%%%%%%%%%%%%%%%%%
\newpage
10
\begin{itemize}
\item (i)
Chi-square statistic is doubled and has value 9.722
OR work it out
\item (ii)
(
)
P-value is given by P χ 22 > 9.722 = 0.0077
Note: answer = 0.008 is acceptable for the mark
\item (iii) Comment: With the first table we do not have strong enough evidence to
justify rejecting the hypothesis of no association. In the second table, we have
the same proportions in the columns, but based on more data, and now we do
have strong enough evidence (P-value < 1\%) to justify rejecting the
hypothesis of no association.
\end{itemize}

\end{document}
