\documentclass[a4paper,12pt]{article}

%%%%%%%%%%%%%%%%%%%%%%%%%%%%%%%%%%%%%%%%%%%%%%%%%%%%%%%%%%%%%%%%%%%%%%%%%%%%%%%%%%%%%%%%%%%%%%%%%%%%%%%%%%%%%%%%%%%%%%%%%%%%%%%%%%%%%%%%%%%%%%%%%%%%%%%%%%%%%%%%%%%%%%%%%%%%%%%%%%%%%%%%%%%%%%%%%%%%%%%%%%%%%%%%%%%%%%%%%%%%%%%%%%%%%%%%%%%%%%%%%%%%%%%%%%%%

\usepackage{eurosym}
\usepackage{vmargin}
\usepackage{amsmath}
\usepackage{graphics}
\usepackage{epsfig}
\usepackage{enumerate}
\usepackage{multicol}
\usepackage{subfigure}
\usepackage{fancyhdr}
\usepackage{listings}
\usepackage{framed}
\usepackage{graphicx}
\usepackage{amsmath}
\usepackage{chngpage}

%\usepackage{bigints}
\usepackage{vmargin}

% left top textwidth textheight headheight

% headsep footheight footskip

\setmargins{2.0cm}{2.5cm}{16 cm}{22cm}{0.5cm}{0cm}{1cm}{1cm}

\renewcommand{\baselinestretch}{1.3}

\setcounter{MaxMatrixCols}{10}

\begin{document}
8
A random sample of size n is taken from a distribution with probability density
function
α
f ( x ) =
(1 + x ) α+ 1
,
0 < x <∞
where α is a parameter such that α > 0.

\begin{enumerate}

\item (i)
Show by evaluating the appropriate integral that, in the case α > 1, the mean
1
.
of this distribution is given by
α − 1
[Hint: when integrating, write x = (1 + x ) - 1 and exploit the fact that the
integral of a density function is unity over its full range.]
\item (ii)
9
10
Determine the method of moments estimator of α.
\end{itemize}

\end{enumerate}
%%%%%%%%%%%%%%%%%%%%%%%%%%%%%%%%%%%%%%%%%%%%%%%%%%%%%%%%%%%%%%%%%%%%%%%%%
8
\begin{itemize}
\item (i)
Mean =
α
∫ x (1 + x ) α+ 1 dx
0
∞
=
∫ (1 + x )
0
α
(1 + x )
∞
dx − ∫ 1
α+ 1
0
α
(1 + x ) α+ 1
dx
∞
=
α
( α − 1)
dx − 1
∫
α − 1 (1 + x ) ( α− 1) + 1
0
=
\item (ii)
α
1
− 1 =
α − 1
α − 1
Equate population mean to sample mean:
Solve to get α = 1 +
\end{itemize}
%%%%%%%%%%%%%%%%%%%%%%%%%%%%%%%%%%%%%%%%%%%%%%%%%%%
\newpage


\end{document}
