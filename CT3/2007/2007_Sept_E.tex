
\documentclass[a4paper,12pt]{article}

%%%%%%%%%%%%%%%%%%%%%%%%%%%%%%%%%%%%%%%%%%%%%%%%%%%%%%%%%%%%%%%%%%%%%%%%%%%%%%%%%%%%%%%%%%%%%%%%%%%%%%%%%%%%%%%%%%%%%%%%%%%%%%%%%%%%%%%%%%%%%%%%%%%%%%%%%%%%%%%%%%%%%%%%%%%%%%%%%%%%%%%%%%%%%%%%%%%%%%%%%%%%%%%%%%%%%%%%%%%%%%%%%%%%%%%%%%%%%%%%%%%%%%%%%%%%

\usepackage{eurosym}
\usepackage{vmargin}
\usepackage{amsmath}
\usepackage{graphics}
\usepackage{epsfig}
\usepackage{enumerate}
\usepackage{multicol}
\usepackage{subfigure}
\usepackage{fancyhdr}
\usepackage{listings}
\usepackage{framed}
\usepackage{graphicx}
\usepackage{amsmath}
\usepackage{chngpage}

%\usepackage{bigints}
\usepackage{vmargin}

% left top textwidth textheight headheight

% headsep footheight footskip

\setmargins{2.0cm}{2.5cm}{16 cm}{22cm}{0.5cm}{0cm}{1cm}{1cm}

\renewcommand{\baselinestretch}{1.3}

\setcounter{MaxMatrixCols}{10}

\begin{document}
\begin{enumerate}

[Total 5]11
Suppose that the random variable X follows an exponential distribution with
probability density function
f ( x ) = λ e −λ x , 0 < x < ∞
Define a new random variable Y =
(i)
(a)
( λ > 0) .
1
X 3 .
Show that the cumulative density function of Y is given by
⎧ ⎪ 1 − exp( −λ y 3 ),
F Y ( y ) = ⎨
⎩ 0,
y ≥ 0
y < 0
and hence, or otherwise, find the probability density function of Y.
(ii)
(b) Explain how you would simulate a value of Y given a value u from the
uniform U(0,1) distribution.
[7]
(a) Find an expression for the maximum likelihood estimator of the
parameter λ, using a sample y 1 , y 2 , ..., y n , from the distribution of Y.
(b) Eight observed values of the random variable Y are given below:
0.72 1.15 1.26 1.03 1.69 1.30 1.42 1.15
Calculate the maximum likelihood estimate of λ using these values. [6]
(iii)
(a)
The hazard function of a continuous random variable T is defined
f ( t )
as h ( t ) =
, where f(t) denotes the probability density function and
S ( t )
S(t) denotes the survival function defined as S ( t ) = P ( T > t ) .
Derive the hazard functions of the random variables X and Y defined
above.
(b)
If a random variable T represents the lifetime of an individual, then the
hazard function h ( t ), as defined in part (iii)(a), gives the instantaneous
mortality rate (that is, the force of mortality) at time t for that
individual.
State (with reasons) which of the two random variables ( X and Y ) you
would use to model the lifetime of pensioners for a period of time
longer than one year, basing your answer on the form of the
corresponding hazard functions derived in part (iii)(a).
[5]
[Total 18]
CT3 S2007—5
%%%%%%%%%%%%%%%%%%%%%%%%%12
A series of n geomagnetic readings are taken from a meter, but the readings are
judged to be approximate and unreliable. The chief scientist involved does know
however that the true values are all positive and she suggests that an appropriate
model for the readings is that they are independent observations of a random variable
which is uniformly distributed on (0, θ), where θ > 1.
(i)
(ii)
Suppose that the chief scientist knows only that the number, M , of the readings
which are less than 1 is m , with the remaining n − m being greater than 1 and
that she adopts the model as suggested above.
1
.
θ
(a) Show that the probability that a single reading is less than 1 is
(b) n
Demonstrate that the maximum likelihood estimate of θ is θ ˆ = .
m
(c) Demonstrate that the Cramer-Rao lower bound (CRlb) for estimating
θ 2 ( θ − 1 )
and hence state the large sample distribution of θ̂ .
θ is
n
[10]
Suppose that exactly 45 readings in a random sample of 100 readings are less
than 1.
(a) Calculate an estimate of the standard error of θ̂ and hence calculate an
approximate two-sided 95\% confidence interval for θ.
(b) Use the large sample distribution of θ̂ to test the hypotheses
H 0 : θ = 3 v H 1 : θ < 3.
[9]

%%%%%%%%%%%%%%%%%%%%%%%%%%%%%%%%%%%%%%%%%%%%%%%%%%%%%%%%%%%%%%%%%%%%%%%%%%%%5
11
(iii) Comment: With the first table we do not have strong enough evidence to
justify rejecting the hypothesis of no association. In the second table, we have
the same proportions in the columns, but based on more data, and now we do
have strong enough evidence (P-value < 1%) to justify rejecting the
hypothesis of no association.
(i) (a)
Y = X
1
3
⇒ X = Y 3 , and range of Y is (0, ∞ ).
The cdf is given by
F Y ( y ) = P ( Y ≤ y ) = P ( X ≤ y 3 ) = F X ( y 3 )
⎪ ⎧ 1 − exp( −λ y 3 ), y ≥ 0
∴ F Y ( y ) = ⎨
y < 0
⎩ 0,
(using formulae or by integration).
Then, the pdf of Y can be derived as
f Y ( y ) =
(
)
d
F Y ( y ) = 3 λ y 2 exp −λ y 3 .
dy
[ OR , directly as
f Y ( y ) = f X ( x )
3
dx
= λ e −λ y 3 y 2
dy
(
)
⇒ f Y ( y ) = 3 λ y 2 exp −λ y 3 ,
OR , from formulae, identifying the cdf as that of a Weibull distribution
with c = λ , γ = 3. ]
Page 5Subject CT3  — September 2007 — Examiners’ Report
(b)
First simulate X ~ exp(λ) as
1
u = 1 − e −λ x ⇒ x = − log(1 − u ) ,
λ
then set y = x
1
3 .
[ OR , use cdf of Y directly, i.e. u = 1 − e
−λ y 3
⎧ 1
⎫
⇒ y = ⎨ − log(1 − u ) ⎬
⎩ λ
⎭
(ii)
n
(a)
n
{
(
L ( λ ) = ∏ f ( y i ; λ ) = ∏ 3 λ y i 2 exp −λ y i 3
i = 1
i = 1
) } = 3 λ ∏ y
n n
i
( λ ) = log L ( λ ) = n log( λ ) − λ ∑ y i 3 + constant
i
n
′ ( λ ) = − ∑ y i 3
λ i
′ ( λ ) = 0 ⇒ λ ˆ =
n
∑ y i 3
i
[Check that ′′ ( λ ) = −
(b)
n
λ 2
< 0. ]
For the given data we have
∑ y i 3 = 16.3952
i
∴λ ˆ =
n
∑ y i
3
=
8
= 0.488 .
16.3952
i
(iii)
(a)
For X ~ exp(λ) we have
h ( x ) =
f ( x )
λ e −λ x
=
=λ
S ( x ) 1 − 1 − e −λ x
(
)
For Y (using pdf and cdf derived above):
(
(
) = 3 λ y
) )
3 λ y 2 exp −λ y 3
f ( y )
h ( y ) =
=
S ( y ) 1 − 1 − exp −λ y 3
(
Page 6
i
2
.
2
⎛
⎞
exp ⎜ −λ ∑ y i 3 ⎟
⎜
⎟
i
⎝
⎠
1
3
]Subject CT3  — September 2007 — Examiners’ Report
(b)
12
(i)
X has a constant hazard rate h ( x ) = λ , and therefore should only be
used when the force of mortality can be assumed constant, e.g. over a
one-year period of time in mortality studies. For longer periods of
lifetime the r.v. Y is more suitable, as it gives an increasing hazard
function with time.
Let X be a reading and M be the number of readings which are less than 1
(a) Since X ~ U(0, θ) , P(X < 1) = length of [0,1]/ length of [0, θ] = 1/θ
(b) ⎛ 1 ⎞ ⎛ 1 ⎞
L ( θ ) ∝ ⎜ ⎟ ⎜ 1 − ⎟
⎝ θ ⎠ ⎝ θ ⎠
m
⇒
⇒
n − m
⇒
θ − 1 ⎞
⎟
⎝ θ ⎠
( θ ) = − m log θ + ( n − m ) log ⎛ ⎜
( θ ) = ( n − m ) log ( θ − 1 ) − n log θ
∂
n − m n
=
−
set to zero ⇒ θ ˆ = n / m
∂θ θ − 1 θ
OR Since M ~ bi(n,1/θ) , MLE of 1/θ is the sample proportion of
readings which are <1, namely m/n, so
( 1/ θ ) = m / n
(c)
⇒ 1/ θ ˆ = m / n ⇒ θ ˆ = n / m
( n − m ) − n
∂
n − m n
∂ 2
=
−
⇒− 2 =
∂θ θ − 1 θ
∂θ
( θ − 1 ) 2 θ 2
⎡ n − M ⎤ n n − n / θ n
⎡ ∂ 2 ⎤
n
⎥ −
=
− 2 = 2
∴ E ⎢ − 2 ⎥ = E ⎢
2
2
2
⎢ ⎣ ( θ − 1 ) ⎥ ⎦ θ
( θ − 1 ) θ θ ( θ − 1 )
⎣ ⎢ ∂θ ⎦ ⎥
∴ CRlb =
(ii)
(a)
θ 2 ( θ − 1 )
n
⎛ θ 2 ( θ − 1 ) ⎞
Large sample distribution of θ̂ is θ ˆ ∼ N ⎜ θ ,
⎟
⎜
⎟
n
⎝
⎠
ˆ
n = 100, m = 45, θ = 100 / 45 = 2.222
Estimate of standard error of
θ̂ = [(100/45) 2 (100/45 – 1)/100] 1/2 = 0.2457
⇒ approximate 95% CI for θ is given by 2.222 \pm 1.96 × 0.2457
i.e. 2.222 \pm 0.482 i.e. (1.74, 2.70).
Page 7Subject CT3  — September 2007 — Examiners’ Report
(b)
Under H 0 : θ ˆ ∼ N ( 3, 0.18 )
2.222 − 3 ⎞
⎛
P -value = P θ ˆ < 2.222 = P ⎜ Z <
⎟ = P ( Z < − 1.834 )
0.4243 ⎠
⎝
(
)
= 0.033 (by interpolation in the table)
We can reject H 0 (at levels of testing down to 3.3%) and conclude that
θ < 3.
[OR note that P(Z < −1.834) is less than 0.05, so “reject H 0 at 5% level”]
