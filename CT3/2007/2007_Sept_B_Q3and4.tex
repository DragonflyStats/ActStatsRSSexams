
\documentclass[a4paper,12pt]{article}

%%%%%%%%%%%%%%%%%%%%%%%%%%%%%%%%%%%%%%%%%%%%%%%%%%%%%%%%%%%%%%%%%%%%%%%%%%%%%%%%%%%%%%%%%%%%%%%%%%%%%%%%%%%%%%%%%%%%%%%%%%%%%%%%%%%%%%%%%%%%%%%%%%%%%%%%%%%%%%%%%%%%%%%%%%%%%%%%%%%%%%%%%%%%%%%%%%%%%%%%%%%%%%%%%%%%%%%%%%%%%%%%%%%%%%%%%%%%%%%%%%%%%%%%%%%%

\usepackage{eurosym}
\usepackage{vmargin}
\usepackage{amsmath}
\usepackage{graphics}
\usepackage{epsfig}
\usepackage{enumerate}
\usepackage{multicol}
\usepackage{subfigure}
\usepackage{fancyhdr}
\usepackage{listings}
\usepackage{framed}
\usepackage{graphicx}
\usepackage{amsmath}
\usepackage{chngpage}

%\usepackage{bigints}
\usepackage{vmargin}

% left top textwidth textheight headheight

% headsep footheight footskip

\setmargins{2.0cm}{2.5cm}{16 cm}{22cm}{0.5cm}{0cm}{1cm}{1cm}

\renewcommand{\baselinestretch}{1.3}

\setcounter{MaxMatrixCols}{10}

\begin{document}
\begin{enumerate}
%%3
\item It is known that 24\% of the customers in a bank holding a current account also have another type of account with the bank.
Calculate an approximate value for the probability that fewer than 50 customers in a random sample of 250 customers with a current account also have another type of
account.
%%[3]
%%-- Question 4
\item In a random sample of 200 policies from a company’s private motor business, there are 68 female policyholders and 132 male policyholders.
Calculate an approximate 99\% confidence interval for the proportion of policyholders who are female in the corresponding population of all policyholders.
%%[3]
\end{enumerate}
\newpage
%%%%%%%%%%%%%%%%
3
Let N be the number who have another type of account.
∴ N ~ binomial(250, 0.24) ≈ N (60, 45.6) = N (60, 6.753 2 )
P ( N < 50) → P ( N < 49.5) with continuity correction
= P ( Z <
4
49.5 − 60
= − 1.55) = 1 − 0.93943 = 0.061
6.753
Sample proportion P = 68/200 = 0.34
99% CI is 0.34 \pm 2.576
Page 2
0.34 × 0.66
i.e. 0.34 \pm 0.086 i.e. (0.254, 0.426)

\end{document}
