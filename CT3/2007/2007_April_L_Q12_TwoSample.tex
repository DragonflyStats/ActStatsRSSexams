\documentclass[a4paper,12pt]{article}

%%%%%%%%%%%%%%%%%%%%%%%%%%%%%%%%%%%%%%%%%%%%%%%%%%%%%%%%%%%%%%%%%%%%%%%%%%%%%%%%%%%%%%%%%%%%%%%%%%%%%%%%%%%%%%%%%%%%%%%%%%%%%%%%%%%%%%%%%%%%%%%%%%%%%%%%%%%%%%%%%%%%%%%%%%%%%%%%%%%%%%%%%%%%%%%%%%%%%%%%%%%%%%%%%%%%%%%%%%%%%%%%%%%%%%%%%%%%%%%%%%%%%%%%%%%%

\usepackage{eurosym}
\usepackage{vmargin}
\usepackage{amsmath}
\usepackage{graphics}
\usepackage{epsfig}
\usepackage{enumerate}
\usepackage{multicol}
\usepackage{subfigure}
\usepackage{fancyhdr}
\usepackage{listings}
\usepackage{framed}
\usepackage{graphicx}
\usepackage{amsmath}
\usepackage{chngpage}

%\usepackage{bigints}
\usepackage{vmargin}

% left top textwidth textheight headheight

% headsep footheight footskip

\setmargins{2.0cm}{2.5cm}{16 cm}{22cm}{0.5cm}{0cm}{1cm}{1cm}

\renewcommand{\baselinestretch}{1.3}

\setcounter{MaxMatrixCols}{10}

\begin{document}
\begin{enumerate}


12
An insurance company is investigating past data for two household claims assessors,
A and B, used by the company. In particular claims resulting from similar types of
water damage were extracted. The following table shows the assessors’ initial
estimates of the cost (in units of £100) of meeting each claim.

A:
B:
4.6
5.7
6.6
3.4
2.8
4.7
5.8
3.6
2.1
6.5
5.2
3.3
5.9
3.8
3.4
2.4
7.8
7.0
3.5
4.0
1.6
4.4
8.6
2.7
for the A data: n A = 13, \sum x = 60.6 and \sum x 2 = 340.92
for the B data: n B = 11, \sum x = 48.8 and \sum x 2 = 236.80
\begin{enumerate}[(a)]
\item (i) Draw a suitable diagram to represent these data so that the initial estimates of
the two assessors can be compared.

\item (ii) You are required to perform an appropriate test to compare the means of the
assessors’ initial estimates for this type of water damage.
(a) State your hypotheses clearly.
(b) Use your diagram in part \item (i) to comment briefly on the validity of your
test.
(c) Calculate your test statistic and specify the resulting P-value.
(d) State your conclusion clearly.
%%%%%%%%%%%%%%%%%%%%%%%%%%%%%%%%%%%%%%%%%%%%%%%%%%%%
\item (iii)
You are required to perform an appropriate test to compare the variances of
the assessors’ initial estimates.
(a) State your hypotheses clearly.
(b) Use your diagram in part \item (i) to comment briefly on the validity of your
test.
(c) Calculate your test statistic and specify the resulting P-value.
(d) State your conclusion clearly and hence comment further on the validity of your test in part \item (ii).

(iv)
Use your answers in parts \item (i) to \item (iii) to comment on the overall comparison of the two assessors as regards their initial estimates for this type of water damage.
\end{enumerate}


%%%%%%%%%%%%%%%%%%%%%%%%%%%%%%%%%%%%%%%%%%%%%%%%%%%%%%%%%%%%%%%%%%%%%%%%%%%%%%%%%%
12
\begin{itemize}
\item (i)
dotplots on same scale are most suitable
[alternatively boxplots are acceptable]
\item (ii)
(a)
Let \mu_A = mean initial estimate for this type of water damage for assessor A and \mu_{B} = mean initial estimate for this type of water damage
for assessor B .
H 0 : \mu_A = \mu_{B} v H 1 : \mu_A \neq \mu_{B}
(b) dotplots show that normality assumption is reasonably valid
dotplots perhaps cast doubt on equal variances assumption
(c) test statistic is t =
From data: x A = 60.6
1
60.6 2
) = 4.8692
= 4.662 , s 2 A = (340.92 −
13
12
13
x B = 48.8
1
48.8 2
) = 2.0305
= 4.436 , s B 2 = (236.80 −
11
10
11
s 2 p = 12(4.8692) + 10(2.0305)
= 3.5789 ∴ s p = 1.8918
22
and
observed t =
Page 8
x A − x B
∼ t n A + n B − 2 under H 0
1
1
s p
+
n A n B
4.662 − 4.436
0.226
=
= 0.29 on 22 df
0.775
1 1
1.8918
+
13 11Subject CT3  — April 2007 — %%%%%%%%%%%%%%%%%%%%%%%%%%%%%%%%
Clearly P-value is very large, or noting that t 22 (40%) = 0.2564, then
P-value is just a bit less than 0.8.
\item (iii)
(d) So there is no evidence at all of any difference between assessors A and B as regards their mean initial estimates for this type of water damage.
(a) $H 0 : \sigma^2 A = \sigma^2 B$ v $H 1 : \sigma^2 A \neq \sigma^2 B$
(b) as in \item (i) dotplots show that normality assumption is reasonably valid
(c) test statistic is F =
observed F =
s 2 A
s B 2
∼ F n A − 1, n B − 1 under H 0
4.8692
= 2.40 on 12,10 df
2.0305
F 12,10 (10%) = 2.284 and F 12,10 (5\%) = 2.913
Thus P-value is between 0.10 and 0.20.
(d)
So there is no real evidence of any difference between assessors A and B as regards the variances of their initial estimates for this type of water damage.
This validates the possibly doubtful assumption required in part \item (ii).
\item (iv)
Overall there is no real evidence to distinguish any differences in the initial
estimates for this type of water damage for the two assessors A and B.
\end{itemize}
\end{document}
