
\documentclass[a4paper,12pt]{article}

%%%%%%%%%%%%%%%%%%%%%%%%%%%%%%%%%%%%%%%%%%%%%%%%%%%%%%%%%%%%%%%%%%%%%%%%%%%%%%%%%%%%%%%%%%%%%%%%%%%%%%%%%%%%%%%%%%%%%%%%%%%%%%%%%%%%%%%%%%%%%%%%%%%%%%%%%%%%%%%%%%%%%%%%%%%%%%%%%%%%%%%%%%%%%%%%%%%%%%%%%%%%%%%%%%%%%%%%%%%%%%%%%%%%%%%%%%%%%%%%%%%%%%%%%%%%

\usepackage{eurosym}
\usepackage{vmargin}
\usepackage{amsmath}
\usepackage{graphics}
\usepackage{epsfig}
\usepackage{enumerate}
\usepackage{multicol}
\usepackage{subfigure}
\usepackage{fancyhdr}
\usepackage{listings}
\usepackage{framed}
\usepackage{graphicx}
\usepackage{amsmath}
\usepackage{chngpage}

%\usepackage{bigints}
\usepackage{vmargin}

% left top textwidth textheight headheight

% headsep footheight footskip

\setmargins{2.0cm}{2.5cm}{16 cm}{22cm}{0.5cm}{0cm}{1cm}{1cm}

\renewcommand{\baselinestretch}{1.3}

\setcounter{MaxMatrixCols}{10}

\begin{document}
\begin{enumerate}
%%--- [Total 5]
\item A random sample of insurance policies of a certain type was examined for each of
four insurance companies and the sums insured ( y ij , for companies i = 1, 2, 3, 4)
under each policy are given in the table below (in units of £100):
Company
1
2
3
4
For these data,
CT3 A2007—4
Total
58.2
56.3
50.1
52.9
57.2
54.5
54.2
49.9
58.4
57.0
55.5
50.0
55.8
55.3
51.7
54.9
284.5
223.1
159.8
204.5
\sum  i \sum  j y ij = 871.9 and \sum  i \sum  j y ij 2 = 47, 633.53Consider the ANOVA model Y ij = \mu  + \tau  i + e ij , i = 1,..., 4, j = 1,..., n i , where Y ij is the j th
sum insured for company i , n i is the number of responses for company i ,
e ij ~ N (0, \sigma^2 ) are independent errors, and
11
\sum  i = 1 n i \tau  i = 0 .
4

\begin{enumerate}[(i)]
\item (i) Calculate estimates of the parameters \mu  and \tau  i , i = 1, 2, 3, 4 .
\item (ii) Test the hypothesis that there are no differences in the means of the sums
insured under such policies by the four companies.
\end{enumerate}
% [Total 8]



%%%%%%%%%%%%%%%%%%%%%%%%%%%%%%%%%%%%%%%%%%%%%%%%%%%%%%%%%%%%%%%%%%%%%%%%%%%%%%%%%%
\item (i)
10
\item (ii)
\mu  ˆ = Y .. =
\tau  ˆ 1 = Y 1. − Y .. = 284.5 871.9
−
= 2.406
5
16
\tau  ˆ 2 = Y 2. − Y .. = 223.1 871.9
−
= 1.281
4
16
\tau  ˆ 3 = Y 3. − Y .. = 159.8 871.9
−
= − 1.227
3
16
\tau  ˆ 4 = Y 4. − Y .. = 204.5 871.9
−
= − 3.369
4
16
SS T = \sum \sum  y ij 2 −
i
SS B =
871.9
= 54.494
16
j
Y .. 2
= 120.430
n
\sum  n i ( Y i . − Y .. ) 2 = (5 \times  2.406 2 ) + (4 \times 1.281 2 ) + (3 \times  1.227 2 ) + (4 \times  3.369 2 ) = 85.425
i
⎛ Y 2 ⎞ Y 2
[OR : SS B = \sum  ⎜ i . ⎟ − .. = 85.428]
⎜
⎟ n
i ⎝ n i ⎠
SS R = SS T − SS B = 35.002
The ANOVA table is:
\begin{verbatim}
Source
DF
SS
MS
F
Company (between treatments) 3
85.428 28.476 9.763
Residual
12 35.002 2.917
Total
15 120.430
\end{verbatim}
At the 5\% significance level, F 0.05,3,12 = 3.490 (or F 0.01,3,12 = 5.953 )
Since F = 9.763 > 3.490, there is evidence against the null hypothesis, and we conclude that there are differences in the mean sums insured by the companies.
\end{document}
