\documentclass[a4paper,12pt]{article}

%%%%%%%%%%%%%%%%%%%%%%%%%%%%%%%%%%%%%%%%%%%%%%%%%%%%%%%%%%%%%%%%%%%%%%%%%%%%%%%%%%%%%%%%%%%%%%%%%%%%%%%%%%%%%%%%%%%%%%%%%%%%%%%%%%%%%%%%%%%%%%%%%%%%%%%%%%%%%%%%%%%%%%%%%%%%%%%%%%%%%%%%%%%%%%%%%%%%%%%%%%%%%%%%%%%%%%%%%%%%%%%%%%%%%%%%%%%%%%%%%%%%%%%%%%%%

\usepackage{eurosym}
\usepackage{vmargin}
\usepackage{amsmath}
\usepackage{graphics}
\usepackage{epsfig}
\usepackage{enumerate}
\usepackage{multicol}
\usepackage{subfigure}
\usepackage{fancyhdr}
\usepackage{listings}
\usepackage{framed}
\usepackage{graphicx}
\usepackage{amsmath}
\usepackage{chngpage}

%\usepackage{bigints}
\usepackage{vmargin}

% left top textwidth textheight headheight

% headsep footheight footskip

\setmargins{2.0cm}{2.5cm}{16 cm}{22cm}{0.5cm}{0cm}{1cm}{1cm}

\renewcommand{\baselinestretch}{1.3}

\setcounter{MaxMatrixCols}{10}

\begin{document}
\begin{enumerate}

A charity issues a large number of certificates each costing \$10 and each being repayable one year after issue. Of these certificates, 1\% are randomly selected to
receive a prize of \$10 such that they are repaid as \$20. The remaining 99\% are repaid at their face value of \$10.
\begin{enumerate}[(i)]
\item (i)
Show that the mean and standard deviation of the sum repaid for a single
purchased certificate are \$10.1 and \$0.995 respectively.

Consider a person who purchases 200 of these certificates.
\item (ii)
Calculate approximately the probability that this person is repaid more than
\$2,040 by using the Central Limit Theorem applied to the total sum repaid.

\item (iii)
An alternative approach to approximating the probability in (ii) above is based
on the number of prize certificates the person is found to hold. This number
will follow a binomial distribution.
Use a Poisson approximation to this binomial distribution to approximate the probability that this person is repaid more than \$2,040.

\item (iv)
Comment briefly on the comparison of the two approximations above given
that the exact probability using the binomial distribution is 0.0517.
\end{enumerate}

%%-- Page 3%%%%%%%%%%%%%%%%%%%%%%%%%%%%%%%%%%%5 — April 2007 — Examiners’ Report
7
\begin{itemize}
\item (i)
Let X = the sum repaid for a single certificate.
\[E ( X ) = 10(0.99) + 20(0.01) = 10.1\]
\[E ( X^2 ) = 10^2 (0.99) + 20^2 (0.01) = 103\]
\[ V ( X ) = 103 − 10.1^2 = 0.99\] 
\[sd ( X ) = 0.9950\]
\item (ii)
%%%%%%%%%%%%%%%%%%%%%%%%%%%%%%%%%%%%%%%%%%%%%%%%%%%%%%%%%%%%%%%%%%
Let S = the sum repaid for 200 certificates.
∴ $E ( S ) = 200(10.1) = 2020$, $V ( S ) = 200(0.99) = 198$ ∴ $sd ( X ) = 14.07$
P ( S > 2040) = P ( Z >
2040 − 2020
= 1.42)
14.07
= 1 − 0.9222 = 0.0778
\item (iii)
N ~ binomial(200, 0.01) ≈ Poisson(2)

\begin{eqnarraty*}
P ( S > 2040) &=& P ( N > 4)\\
&=& 1 − P ( N \leq 4) \\ 
&=& 1 − 0.94735 \\
&=& 0.0527\\
\end{eqnarray*}

\item (iv)
Clearly the Poisson approximation to the binomial is better than the Central
Limit Theorem approximation.

\item OR:
Since S is discrete and increases in steps of 10, one can argue for the use of a
continuity correction in (ii) above:
2045 − 2020 ⎞
⎛
P ( S > 2040 ) = P ⎜ Z ≥
⎟ = P ( Z > 1.78 )
14.07
⎝
⎠
= 1 − 0.96246 = 0.0375
%%\item (Either approach is acceptable for the marks.)
Page 4%%%%%%%%%%%%%%%%%%%%%%%%%%%%%%%%%%%5 — April 2007 — Examiners’ Report
∞
\end{itemize}
\end{document}
