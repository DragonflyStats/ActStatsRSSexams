

\documentclass[a4paper,12pt]{article}

%%%%%%%%%%%%%%%%%%%%%%%%%%%%%%%%%%%%%%%%%%%%%%%%%%%%%%%%%%%%%%%%%%%%%%%%%%%%%%%%%%%%%%%%%%%%%%%%%%%%%%%%%%%%%%%%%%%%%%%%%%%%%%%%%%%%%%%%%%%%%%%%%%%%%%%%%%%%%%%%%%%%%%%%%%%%%%%%%%%%%%%%%%%%%%%%%%%%%%%%%%%%%%%%%%%%%%%%%%%%%%%%%%%%%%%%%%%%%%%%%%%%%%%%%%%%

\usepackage{eurosym}
\usepackage{vmargin}
\usepackage{amsmath}
\usepackage{graphics}
\usepackage{epsfig}
\usepackage{enumerate}
\usepackage{multicol}
\usepackage{subfigure}
\usepackage{fancyhdr}
\usepackage{listings}
\usepackage{framed}
\usepackage{graphicx}
\usepackage{amsmath}
\usepackage{chngpage}

%\usepackage{bigints}
\usepackage{vmargin}

% left top textwidth textheight headheight

% headsep footheight footskip

\setmargins{2.0cm}{2.5cm}{16 cm}{22cm}{0.5cm}{0cm}{1cm}{1cm}

\renewcommand{\baselinestretch}{1.3}

\setcounter{MaxMatrixCols}{10}

\begin{document}


%%[Total 5]11
Suppose that the random variable X follows an exponential distribution with
probability density function
\[f ( x ) = \lambda e −\lambda x , 0 < x < ∞\]
Define a new random variable Y =


\begin{enumerate}
\item (i)
(a)
( \lambda > 0) .
1
X 3 .
Show that the cumulative density function of Y is given by
⎧ ⎪),


\[ F Y ( y ) = \begin{cases}  1 − exp( −\lambda y 3  &  y \leq 0 \\
 0 & y < 0 \\
 \end{cases}
 \]
and hence, or otherwise, find the probability density function of Y.

\item (b) Explain how you would simulate a value of Y given a value u from the
uniform U(0,1) distribution.
[7]
\item (ii)(a) Find an expression for the maximum likelihood estimator of the
parameter $\lambda$, using a sample y 1 , y 2 , ..., y n , from the distribution of Y.
\item (b) Eight observed values of the random variable Y are given below:
\[0.72 1.15 1.26 1.03 1.69 1.30 1.42 1.15\]
Calculate the maximum likelihood estimate of $\lambda$ using these values. [6]
\item (iii)
(a)
The hazard function of a continuous random variable T is defined
f ( t )
as h ( t ) =
, where $f(t)$ denotes the probability density function and
S ( t )
S(t) denotes the survival function defined as $S ( t ) = P ( T > t )$ .
Derive the hazard functions of the random variables X and Y defined
above.
\item (b)
If a random variable T represents the lifetime of an individual, then the
hazard function $h ( t )$, as defined in part (iii)(a), gives the instantaneous
mortality rate (that is, the force of mortality) at time t for that
individual.

State (with reasons) which of the two random variables ( X and Y ) you
would use to model the lifetime of pensioners for a period of time
longer than one year, basing your answer on the form of the
corresponding hazard functions derived in part (iii)(a).
\end{enumerate}
\newpage
%%%%%%%%%%%%%%%%%%%%%%%%%
%%%%%%%%%%%%%%%%%%%%%%%%%%%%%%%%%%%%%%%%%%%%%%%%%%%%%%%%%%%%%%\newpage


11
\begin{itemize}
    \item 
(i) (a)
Y = X
1
3
⇒ X = Y 3 , and range of Y is (0, ∞ ).
\item The cdf is given by
\[F Y ( y ) = P ( Y \leq y ) = P ( X \leq y 3 ) = F X ( y 3 )\]
% ⎪ ⎧ 1 − exp( −\lambda y 3 ), y ≥ 0
% ∴ F Y ( y ) = ⎨
% y < 0
% ⎩ 0,
(using formulae or by integration).
\item Then, the pdf of Y can be derived as
\[f Y ( y ) =
(
)
d
F Y ( y ) = 3 \lambda y 2 exp −\lambda y 3 .
dy\]
\item  [ OR , directly as
\[f Y ( y ) = f X ( x )
3
dx
= \lambda e −\lambda y 3 y 2
dy
(
)\]
\[ f Y ( y ) = 3 \lambda y 2 exp −\lambda y 3 ,\[
OR , from formulae, identifying the cdf as that of a Weibull distribution
with $c = \lambda$ , $γ = 3$. ]
Page 5Subject CT3  — September 2007 — Examiners’ Report
(b)
\item  First simulate X ~ exp(\lambda) as
1
\[u = 1 − e −\lambda x ⇒ x = − log(1 − u ) ,\]
%\lambda
then set y = x
1
3 .
[ OR , use cdf of Y directly, i.e. u = 1 − e
−\lambda y 3
⎧ 1
⎫
⇒ y = ⎨ − log(1 − u ) ⎬
⎩ \lambda
⎭
\item (ii)
n
(a)
n
{
\[(
L ( \lambda ) = ∏ f ( y i ; \lambda ) = ∏ 3 \lambda y i 2 exp −\lambda y i 3
i = 1
i = 1
) } = 3 \lambda ∏ y
n n
i
( \lambda ) = log L ( \lambda ) = n log( \lambda ) − \lambda ∑ y i 3 + constant\]
i
n
′ ( \lambda ) = − ∑ y i 3
\lambda i
′ ( \lambda ) = 0 ⇒ \lambda ˆ =
n
∑ y i 3
i
[Check that ′′ ( \lambda ) = −
(b)
n
\lambda 2
< 0. ]
\item For the given data we have
∑ y i 3 = 16.3952
i
∴\lambda ˆ =
n
∑ y i
3
=
8
= 0.488 .
16.3952
\item (iii)
(a)
For $X ~ exp(\lambda)$ we have
\begin{eqnarray*}
h ( x ) &=& f ( x ) \lambda e −\lambda x \\
&=& \lambda
S ( x ) 1 − 1 − e −\lambda x
\end{eqnarray*}
\item For Y (using pdf and cdf derived above):
(
(
) = 3 \lambda y
) )
3 \lambda y 2 exp −\lambda y 3
f ( y )
h ( y ) =
=
S ( y ) 1 − 1 − exp −\lambda y 3
(
Page 6
i
2
.
2
⎛
⎞
exp ⎜ −\lambda ∑ y i 3 ⎟
⎜
⎟
i
⎝
⎠
1
3
]

\end{itemize}
\end{document}
