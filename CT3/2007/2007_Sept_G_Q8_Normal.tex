
\documentclass[a4paper,12pt]{article}

%%%%%%%%%%%%%%%%%%%%%%%%%%%%%%%%%%%%%%%%%%%%%%%%%%%%%%%%%%%%%%%%%%%%%%%%%%%%%%%%%%%%%%%%%%%%%%%%%%%%%%%%%%%%%%%%%%%%%%%%%%%%%%%%%%%%%%%%%%%%%%%%%%%%%%%%%%%%%%%%%%%%%%%%%%%%%%%%%%%%%%%%%%%%%%%%%%%%%%%%%%%%%%%%%%%%%%%%%%%%%%%%%%%%%%%%%%%%%%%%%%%%%%%%%%%%

\usepackage{eurosym}
\usepackage{vmargin}
\usepackage{amsmath}
\usepackage{graphics}
\usepackage{epsfig}
\usepackage{enumerate}
\usepackage{multicol}
\usepackage{subfigure}
\usepackage{fancyhdr}
\usepackage{listings}
\usepackage{framed}
\usepackage{graphicx}
\usepackage{amsmath}
\usepackage{chngpage}

%\usepackage{bigints}
\usepackage{vmargin}
%%- Question 8 


% left top textwidth textheight headheight

% headsep footheight footskip

\setmargins{2.0cm}{2.5cm}{16 cm}{22cm}{0.5cm}{0cm}{1cm}{1cm}

\renewcommand{\baselinestretch}{1.3}

\setcounter{MaxMatrixCols}{10}

\begin{document}

8
Claim sizes in a certain insurance situation are modelled by a normal distribution with
mean $\mu = \$30,000$ and standard deviation $\sigma = \$4,000$.

 The insurer defines a claim to
be a large claim if the claim size exceeds \$35,000.
\begin{enumerate}[(a)]
\item (i)
Calculate the probabilities that the size of a claim exceeds
(a)
(b)
\$35,000, and
\$36,000

\item 
(iii)
Calculate the probability that the size of a large claim (as defined by the
insurer) exceeds \$36,000.
\item 
Calculate the probability that a random sample of 5 claims includes 2 which
exceed \$35,000 and 3 which are less than \$35,000.
\end{enumerate}


$X$ is normally distributed with mean $\mu = 30$ and $\sigma = 4$ (working in units of \$1000)

%%%%%%%%%%%%%%%%%%%%%%%%%%%%%%%%%%%%
\newpage
%%%%%%%%%%%%%%%%%%%%%%%%%%%%%%%%%%%%%%%%%%%%%%%%%%%%%%%%%%%%
%%--- Page 3Subject CT3  — September 2007 — Examiners’ Report
8
$X$ is normally distributed with mean $\mu = 30$ and $\sigma = 4$ (working in units of \$1000)

\begin{framed}
\noindent \textbf{Part (a)}\\ \large
\noindent Calculate the probabilities that the size of a claim exceeds
(a)
(b)
\$35,000, and
\$36,000


\end{framed}
\large
\noindent 
\begin{eqnarray*}
P ( X > 35 ) &=& P \left( Z >\frac{35 − 30}{4} \right) \\
&=& P \left( Z >\frac{5}{4} \right) \\
&=& P ( Z > 1.25 ) \\ 
&=& 1 - \Phi (1.25 ) \\ 
&=& 1 − 0.89435 \\
&=& 0.10565\\
\end{eqnarray*}
 

\begin{eqnarray*}
P ( X > 36 ) &=& P \left( Z >\frac{36 − 30}{4} \right) \\
&=& P \left( Z >\frac{6}{4} \right) \\
&=& P ( Z > 1.50 ) \\ 
&=& 1 - \Phi (1.50) \\ 
&=& 1 − 0.93319 \\
&=& 0.06681\\
\end{eqnarray*}
\newpage

\newpage
%%%%%%%%%%%%%%%%%%%%%%%%%%%%%%%%%%%%%%%%%%%%%%%%%%%%%%%%%%%%
\newpage
\begin{framed}
\noindent \textbf{Part (b)}\\ \large
\noindent 
Calculate the probability that the size of a large claim (as defined by the
insurer) exceeds \$36,000.
\end{framed}
\large
\noindent 
Calculate the probability that the size of a claim  exceeds \$36,000,\textbf{ \textit{given that it is defined as a large claim by the
insurer}}.

\begin{eqnarray*}
P(X > 36 | X > 35) &=& \frac{P(X > 36 \mbox{ and } X > 35)}{P(X > 35)}\\
&=& \frac{P(X > 36)}{P(X > 35)}\\
& & \\
&=& \frac{P(Z > 1.5)}{P(Z > 1.25)}\\
& & \\
&=& \frac{0.06681}{0.10565}\\
& & \\
&=& 0.632 \\
\end{eqnarray*}
%%%%%%%%%%%%%%%%%%%%%%%%%%%%%%%%%%%%%%%%%%%%%%%%%
\newpage
\begin{framed}
\noindent \textbf{Part (c)}\\ \large
\noindent Calculate the probability that a random sample of 5 claims includes 2 which
exceed \$35,000 and 3 which are less than \$35,000.
\end{framed}
\large
\noindent This can be described as a \textbf{ \textit{Binomial Experiment}} with 5 independent trials where a success is a claim that exceeds \$35,000.
\begin{itemize}
    \item n = 5
    \item p = 0.10565
    \item Number of Successes : X = 2
\end{itemize}
\begin{eqnarray*}
P(X = 2) &=& { 5 \choose 2} \times  (0.10565)^2 \times (1 \;-\;0.10565)^3 \\
&=& { 10} \times  0.10565^2 \times 0.89435^3) \\
&=& 10 \times 0.01116  \times 0.71535 \\
&=& 0.07983 \\
\end{document}
