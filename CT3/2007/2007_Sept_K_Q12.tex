
\documentclass[a4paper,12pt]{article}

%%%%%%%%%%%%%%%%%%%%%%%%%%%%%%%%%%%%%%%%%%%%%%%%%%%%%%%%%%%%%%%%%%%%%%%%%%%%%%%%%%%%%%%%%%%%%%%%%%%%%%%%%%%%%%%%%%%%%%%%%%%%%%%%%%%%%%%%%%%%%%%%%%%%%%%%%%%%%%%%%%%%%%%%%%%%%%%%%%%%%%%%%%%%%%%%%%%%%%%%%%%%%%%%%%%%%%%%%%%%%%%%%%%%%%%%%%%%%%%%%%%%%%%%%%%%

\usepackage{eurosym}
\usepackage{vmargin}
\usepackage{amsmath}
\usepackage{graphics}
\usepackage{epsfig}
\usepackage{enumerate}
\usepackage{multicol}
\usepackage{subfigure}
\usepackage{fancyhdr}
\usepackage{listings}
\usepackage{framed}
\usepackage{graphicx}
\usepackage{amsmath}
\usepackage{chngpage}

%\usepackage{bigints}
\usepackage{vmargin}

% left top textwidth textheight headheight

% headsep footheight footskip

\setmargins{2.0cm}{2.5cm}{16 cm}{22cm}{0.5cm}{0cm}{1cm}{1cm}

\renewcommand{\baselinestretch}{1.3}

\setcounter{MaxMatrixCols}{10}

\begin{document}
12
A series of n geomagnetic readings are taken from a meter, but the readings are
judged to be approximate and unreliable. The chief scientist involved does know
however that the true values are all positive and she suggests that an appropriate
model for the readings is that they are independent observations of a random variable
which is uniformly distributed on (0, θ), where θ > 1.
(i)
(ii)
Suppose that the chief scientist knows only that the number, M , of the readings
which are less than 1 is m , with the remaining n − m being greater than 1 and
that she adopts the model as suggested above.
1
.
θ
(a) Show that the probability that a single reading is less than 1 is
(b) n
Demonstrate that the maximum likelihood estimate of θ is θ ˆ = .
m
(c) Demonstrate that the Cramer-Rao lower bound (CRlb) for estimating
θ 2 ( θ − 1 )
and hence state the large sample distribution of θ̂ .
θ is
n
[10]
Suppose that exactly 45 readings in a random sample of 100 readings are less
than 1.
(a) Calculate an estimate of the standard error of θ̂ and hence calculate an
approximate two-sided 95\% confidence interval for θ.
(b) Use the large sample distribution of θ̂ to test the hypotheses
H 0 : θ = 3 v H 1 : θ < 3.
[9]

%%%%%%%%%%%%%%%%%%%%%%%%%%%%%%%%%%%%%%%%
12
(i)
X has a constant hazard rate h ( x ) = λ , and therefore should only be
used when the force of mortality can be assumed constant, e.g. over a
one-year period of time in mortality studies. For longer periods of
lifetime the r.v. Y is more suitable, as it gives an increasing hazard
function with time.
Let X be a reading and M be the number of readings which are less than 1
(a) Since X ~ U(0, θ) , P(X < 1) = length of [0,1]/ length of [0, θ] = 1/θ
(b) ⎛ 1 ⎞ ⎛ 1 ⎞
L ( θ ) ∝ ⎜ ⎟ ⎜ 1 − ⎟
⎝ θ ⎠ ⎝ θ ⎠
m
⇒
⇒
n − m
⇒
θ − 1 ⎞
⎟
⎝ θ ⎠
( θ ) = − m log θ + ( n − m ) log ⎛ ⎜
( θ ) = ( n − m ) log ( θ − 1 ) − n log θ
∂
n − m n
=
−
set to zero ⇒ θ ˆ = n / m
∂θ θ − 1 θ
OR Since M ~ bi(n,1/θ) , MLE of 1/θ is the sample proportion of
readings which are <1, namely m/n, so
( 1/ θ ) = m / n
(c)
⇒ 1/ θ ˆ = m / n ⇒ θ ˆ = n / m
( n − m ) − n
∂
n − m n
∂ 2
=
−
⇒− 2 =
∂θ θ − 1 θ
∂θ
( θ − 1 ) 2 θ 2
⎡ n − M ⎤ n n − n / θ n
⎡ ∂ 2 ⎤
n
⎥ −
=
− 2 = 2
∴ E ⎢ − 2 ⎥ = E ⎢
2
2
2
⎢ ⎣ ( θ − 1 ) ⎥ ⎦ θ
( θ − 1 ) θ θ ( θ − 1 )
⎣ ⎢ ∂θ ⎦ ⎥
∴ CRlb =
(ii)
(a)
θ 2 ( θ − 1 )
n
⎛ θ 2 ( θ − 1 ) ⎞
Large sample distribution of θ̂ is θ ˆ ∼ N ⎜ θ ,
⎟
⎜
⎟
n
⎝
⎠
ˆ
n = 100, m = 45, θ = 100 / 45 = 2.222
Estimate of standard error of
θ̂ = [(100/45) 2 (100/45 – 1)/100] 1/2 = 0.2457
⇒ approximate 95% CI for θ is given by 2.222 \pm 1.96 × 0.2457
i.e. 2.222 \pm 0.482 i.e. (1.74, 2.70).
Page 7Subject CT3  — September 2007 — Examiners’ Report
(b)
Under H 0 : θ ˆ ∼ N ( 3, 0.18 )
2.222 − 3 ⎞
⎛
P -value = P θ ˆ < 2.222 = P ⎜ Z <
⎟ = P ( Z < − 1.834 )
0.4243 ⎠
⎝
(
)
= 0.033 (by interpolation in the table)
We can reject H 0 (at levels of testing down to 3.3%) and conclude that
θ < 3.
[OR note that P(Z < −1.834) is less than 0.05, so “reject H 0 at 5% level”]


\end{document}
