
\documentclass[a4paper,12pt]{article}

%%%%%%%%%%%%%%%%%%%%%%%%%%%%%%%%%%%%%%%%%%%%%%%%%%%%%%%%%%%%%%%%%%%%%%%%%%%%%%%%%%%%%%%%%%%%%%%%%%%%%%%%%%%%%%%%%%%%%%%%%%%%%%%%%%%%%%%%%%%%%%%%%%%%%%%%%%%%%%%%%%%%%%%%%%%%%%%%%%%%%%%%%%%%%%%%%%%%%%%%%%%%%%%%%%%%%%%%%%%%%%%%%%%%%%%%%%%%%%%%%%%%%%%%%%%%

\usepackage{eurosym}
\usepackage{vmargin}
\usepackage{amsmath}
\usepackage{graphics}
\usepackage{epsfig}
\usepackage{enumerate}
\usepackage{multicol}
\usepackage{subfigure}
\usepackage{fancyhdr}
\usepackage{listings}
\usepackage{framed}
\usepackage{graphicx}
\usepackage{amsmath}
\usepackage{chngpage}

%\usepackage{bigints}
\usepackage{vmargin}

% left top textwidth textheight headheight

% headsep footheight footskip

\setmargins{2.0cm}{2.5cm}{16 cm}{22cm}{0.5cm}{0cm}{1cm}{1cm}

\renewcommand{\baselinestretch}{1.3}

\setcounter{MaxMatrixCols}{10}

\begin{document}
\begin{enumerate}
%© Institute of Actuaries1
\item Data collected on claim amounts (£) for two postcode regions give the following values for n (the number of claims), x (the mean claim amount) and s (the sample standard deviation) of the claim amounts.
Region 1
25
120.2
58.1
n
x
s
Region 2
18
142.7
62.2
Calculate, to one decimal place, the mean and sample standard deviation of the claim amounts for both regions combined.
[4]
%%-- Question 2
\item The random variable X has probability density function
\[f ( x ) =
2
x 3
,
x > 1\]
and cumulative distribution function
⎧ 0,
⎪
F ( x ) = ⎨
1
⎪ 1 − 2 ,
⎩ x
x < 1
x ≥ 1
.
%%%%%%%%%%%%%%%%5
Use the following uniform(0,1) random numbers
\[0.5719, 0.8612, 0.3028\]
to simulate three observations of $X$, explaining your method and calculations clearly.
[4]
\en{enumerate}
%%%%%%%%%%%%%%%%%%%%%%%%%%%5
1
x =
\sigma x 25(120.2) + 18(142.7) 5573.6
=
=
= 129.6
n
25 + 18
43
Using the fact that $\sigma_x^2$ = ( n − 1) s 2 + nx 2 , then for the combined set
\[ \sigma_x^2 = [24(58.1) 2 + 25(120.2) 2 ] + [17(62.2) 2 + 18(142.7) 2 ] = 874525.14\]
∴ s 2 =
2
874525.14 − (5573.6) 2 / 43
= 3621.02 ∴ s = 60.2
42
\begin{itemize}
    \item Method: set uniform(0,1) random number r = F ( x ) = 1 – 1/ x 2
⇒ simulated observation x = [1/(1 − r )] 1/2
\item Here we get $x = 1.528 , 2.684, 1.198.$
\item Note: We can do away with the step of subtracting r from 1 and use x = (1/ r ) 1/2 .
\item This gives $x = 1.322, 1.078, 1.817$.
\end{itemize}

\end{document}
