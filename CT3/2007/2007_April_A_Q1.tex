\documentclass[a4paper,12pt]{article}

%%%%%%%%%%%%%%%%%%%%%%%%%%%%%%%%%%%%%%%%%%%%%%%%%%%%%%%%%%%%%%%%%%%%%%%%%%%%%%%%%%%%%%%%%%%%%%%%%%%%%%%%%%%%%%%%%%%%%%%%%%%%%%%%%%%%%%%%%%%%%%%%%%%%%%%%%%%%%%%%%%%%%%%%%%%%%%%%%%%%%%%%%%%%%%%%%%%%%%%%%%%%%%%%%%%%%%%%%%%%%%%%%%%%%%%%%%%%%%%%%%%%%%%%%%%%

\usepackage{eurosym}
\usepackage{vmargin}
\usepackage{amsmath}
\usepackage{graphics}
\usepackage{epsfig}
\usepackage{enumerate}
\usepackage{multicol}
\usepackage{subfigure}
\usepackage{fancyhdr}
\usepackage{listings}
\usepackage{framed}
\usepackage{graphicx}
\usepackage{amsmath}
\usepackage{chngpage}

%\usepackage{bigints}
\usepackage{vmargin}

% left top textwidth textheight headheight

% headsep footheight footskip

\setmargins{2.0cm}{2.5cm}{16 cm}{22cm}{0.5cm}{0cm}{1cm}{1cm}

\renewcommand{\baselinestretch}{1.3}

\setcounter{MaxMatrixCols}{10}

\begin{document}
\begin{enumerate}

% © Institute of Actuaries1
\item Consider the following two random samples of ten observations which come from the
distributions of random variables which assume non-negative integer values only.
Sample 1: 7 4 6 11 5 9 8 3 5 5
sample mean = 6.3, sample variance = 6.01
Sample 2: 8 3 5 11 2 4 6 12 3 9
sample mean = 6.3, sample variance = 12.46
One sample comes from a Poisson distribution, the other does not.
State, with brief reasons, which sample you think is likely to be which.



\end{enumerate}
\newpage
%%%%%%%%%%%%%%%%%%%%%%%%%%%%%%%%%%%%%%%%%%%%%%%%%%%%%%%%%%%%%%%%%%%%%%%%%%%%%%%%%%%5
1
\begin{itemize}
    \item A Poisson random variable has mean = variance and this will be reflected in the
sample mean and variance for a random sample.
\item Sample 2 has a very much higher variance than mean, whereas sample 1 has mean
and variance approximately the same, so sample 1 is likely to be the one which comes
from a Poisson distribution.
\end{itemize}

\end{document}

