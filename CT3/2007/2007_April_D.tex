\documentclass[a4paper,12pt]{article}

%%%%%%%%%%%%%%%%%%%%%%%%%%%%%%%%%%%%%%%%%%%%%%%%%%%%%%%%%%%%%%%%%%%%%%%%%%%%%%%%%%%%%%%%%%%%%%%%%%%%%%%%%%%%%%%%%%%%%%%%%%%%%%%%%%%%%%%%%%%%%%%%%%%%%%%%%%%%%%%%%%%%%%%%%%%%%%%%%%%%%%%%%%%%%%%%%%%%%%%%%%%%%%%%%%%%%%%%%%%%%%%%%%%%%%%%%%%%%%%%%%%%%%%%%%%%

\usepackage{eurosym}
\usepackage{vmargin}
\usepackage{amsmath}
\usepackage{graphics}
\usepackage{epsfig}
\usepackage{enumerate}
\usepackage{multicol}
\usepackage{subfigure}
\usepackage{fancyhdr}
\usepackage{listings}
\usepackage{framed}
\usepackage{graphicx}
\usepackage{amsmath}
\usepackage{chngpage}

%\usepackage{bigints}
\usepackage{vmargin}

% left top textwidth textheight headheight

% headsep footheight footskip

\setmargins{2.0cm}{2.5cm}{16 cm}{22cm}{0.5cm}{0cm}{1cm}{1cm}

\renewcommand{\baselinestretch}{1.3}

\setcounter{MaxMatrixCols}{10}

\begin{document}
\begin{enumerate}

PLEASE TURN OVER8
A random sample of size n is taken from a distribution with probability density
function
α
f ( x ) =
(1 + x ) α+ 1
,
0 < x <∞
where α is a parameter such that α > 0.
(i)
Show by evaluating the appropriate integral that, in the case α > 1, the mean
1
.
of this distribution is given by
α − 1
[Hint: when integrating, write x = (1 + x ) - 1 and exploit the fact that the
integral of a density function is unity over its full range.]
(ii)
9
10
Determine the method of moments estimator of α.
[3]
[2]
[Total 5]
Consider three random variables X , Y , and Z with the same variance σ 2 = 4. Suppose
that X is independent of both Y and Z , but Y and Z are correlated, with correlation
coefficient ρ YZ = 0.5.
(i) Calculate the covariance between X and U , where U = Y+Z . [1]
(ii) Calculate the covariance between Z and V , where V = 3 X – 2 Y . [2]
(iii) Calculate the variance of W , where W = 3 X – 2 Y + Z .
[2]

%%%%%%%%%%%%%%%%%%%%%%%%%%%%%%%%%%%%%%%%%%%%%%%%%%%%%%%%%%%%%%%%%%%%%%%%%
8
(i)
Mean =
α
∫ x (1 + x ) α+ 1 dx
0
∞
=
∫ (1 + x )
0
α
(1 + x )
∞
dx − ∫ 1
α+ 1
0
α
(1 + x ) α+ 1
dx
∞
=
α
( α − 1)
dx − 1
∫
α − 1 (1 + x ) ( α− 1) + 1
0
=
(ii)
α
1
− 1 =
α − 1
α − 1
Equate population mean to sample mean:
Solve to get α = 1 +
9
(i)
1
= x
α − 1
1
1
, so MME = 1 +
x
X
Cov(X,Y+Z) = Cov(X,Y) + Cov(X,Z) = 0
[Note: The simple statement of the answer “0” is acceptable for the single
mark available.]
(ii) Cov(Z, 3X - 2Y) = 3Cov(Z,X) – 2Cov(Z,Y) = 0 – 2ρ YZ σ 2
= - 2 × 0.5 × 4 = - 4
(iii) V[3X – 2Y + Z] = (9 + 4 + 1)σ 2 – 12Cov(X,Y) + 6Cov(X,Z) – 4Cov(Y,Z)
= 14(4) – 4(0.5)(4) = 48


\end{document}
