\documentclass[a4paper,12pt]{article}

%%%%%%%%%%%%%%%%%%%%%%%%%%%%%%%%%%%%%%%%%%%%%%%%%%%%%%%%%%%%%%%%%%%%%%%%%%%%%%%%%%%%%%%%%%%%%%%%%%%%%%%%%%%%%%%%%%%%%%%%%%%%%%%%%%%%%%%%%%%%%%%%%%%%%%%%%%%%%%%%%%%%%%%%%%%%%%%%%%%%%%%%%%%%%%%%%%%%%%%%%%%%%%%%%%%%%%%%%%%%%%%%%%%%%%%%%%%%%%%%%%%%%%%%%%%%

\usepackage{eurosym}
\usepackage{vmargin}
\usepackage{amsmath}
\usepackage{graphics}
\usepackage{epsfig}
\usepackage{enumerate}
\usepackage{multicol}
\usepackage{subfigure}
\usepackage{fancyhdr}
\usepackage{listings}
\usepackage{framed}
\usepackage{graphicx}
\usepackage{amsmath}
\usepackage{chngpage}

%\usepackage{bigints}
\usepackage{vmargin}

% left top textwidth textheight headheight

% headsep footheight footskip

\setmargins{2.0cm}{2.5cm}{16 cm}{22cm}{0.5cm}{0cm}{1cm}{1cm}

\renewcommand{\baselinestretch}{1.3}

\setcounter{MaxMatrixCols}{10}

\begin{document}
\begin{enumerate}

% © Institute of Actuaries1
\item Consider the following two random samples of ten observations which come from the
distributions of random variables which assume non-negative integer values only.
Sample 1: 7 4 6 11 5 9 8 3 5 5
sample mean = 6.3, sample variance = 6.01
Sample 2: 8 3 5 11 2 4 6 12 3 9
sample mean = 6.3, sample variance = 12.46
One sample comes from a Poisson distribution, the other does not.
State, with brief reasons, which sample you think is likely to be which.
2
[2]
\item A random sample of 200 policy surrender values (in units of £1,000) yields a mean of
43.6 and a standard deviation of 82.2.
Determine a 99\% confidence interval for the true underlying mean surrender value for
such policies.
[3]
3
\item It is assumed that claims on a certain type of policy arise as a Poisson process with
claim rate λ per year.
For a group of 150 independent policies of this type, the total number of claims during
the last calendar year was recorded as 123.
Determine an approximate 95\% confidence interval for the true underlying annual
claim rate for such a policy.
[4]
\end{enumerate}
\newpage
%%%%%%%%%%%%%%%%%%%%%%%%%%%%%%%%%%%%%%%%%%%%%%%%%%%%%%%%%%%%%%%%%%%%%%%%%%%%%%%%%%%5
1
\begin{itemize}
    \item A Poisson random variable has mean = variance and this will be reflected in the
sample mean and variance for a random sample.
\item Sample 2 has a very much higher variance than mean, whereas sample 1 has mean
and variance approximately the same, so sample 1 is likely to be the one which comes
from a Poisson distribution.
\end{itemize}

%%%%%%%%%%%%%5
2
Approximate large sample confidence interval for the mean is given by
s
n
x ± z α / 2
for 99% CI, z α / 2 = 2.5758
leading to 43.6 ± 2.5758
or
3
(28.6, 58.6)
or
82.2
200
⇒ 43.6 ± 15.0
(£28,600, £58,600)
The mean number of claims per policy is X =
123
= 0.82
150
Using the normal approximation to the Poisson distribution approximate 95\% confidence interval for λ is X ± 1.96
→ 0.82 ± 1.96
0.82
150
X
n
→ 0.82 ± 1.96(0.0739)
→ 0.82 ± 0.145 or (0.675, 0.965)
%%%%%%%%%%%%%%%%%%
4
Answer = −0.982

(The relationship is now a negative one; the only change is the sign. An answer of
+0.982 gets 1 mark.)
% Page 2Subject CT3 (Probability and Mathematical Statistics Core Technical) — April 2007 — Examiners’ Report
%%%%%%%%%%%%%%%%%%%%%%
5
Let N = number of claims in the six months
Let X = a single claim size
Let S = sum of claim sizes for the six months
Then N ~ Poisson(6)
\[E ( S ) = E ( N ) E ( X ) = (6)(80) = £480\]
V ( S ) = E ( N ) V ( X ) + V ( N )[ E ( X )] 2 = (6)(80 2 ) + (6)(80 2 ) = 76800
∴ sd ( S ) = £277
\end{document}
