%%- B

\documentclass[a4paper,12pt]{article}

%%%%%%%%%%%%%%%%%%%%%%%%%%%%%%%%%%%%%%%%%%%%%%%%%%%%%%%%%%%%%%%%%%%%%%%%%%%%%%%%%%%%%%%%%%%%%%%%%%%%%%%%%%%%%%%%%%%%%%%%%%%%%%%%%%%%%%%%%%%%%%%%%%%%%%%%%%%%%%%%%%%%%%%%%%%%%%%%%%%%%%%%%%%%%%%%%%%%%%%%%%%%%%%%%%%%%%%%%%%%%%%%%%%%%%%%%%%%%%%%%%%%%%%%%%%%

\usepackage{eurosym}
\usepackage{vmargin}
\usepackage{amsmath}
\usepackage{graphics}
\usepackage{epsfig}
\usepackage{enumerate}
\usepackage{multicol}
\usepackage{subfigure}
\usepackage{fancyhdr}
\usepackage{listings}
\usepackage{framed}
\usepackage{graphicx}
\usepackage{amsmath}
\usepackage{chngpage}

%\usepackage{bigints}
\usepackage{vmargin}

% left top textwidth textheight headheight

% headsep footheight footskip

\setmargins{2.0cm}{2.5cm}{16 cm}{22cm}{0.5cm}{0cm}{1cm}{1cm}

\renewcommand{\baselinestretch}{1.3}

\setcounter{MaxMatrixCols}{10}

\begin{document}
\begin{enumerate}
\item 
Let the random variable Y denote the size (in units of £1,000) of the loss per claim sustained in a particular line of insurance. Suppose that Y follows a chi-square distribution with 2 degrees of freedom. Two such claims are randomly chosen and
their corresponding losses are assumed to be independent of each other.
(i) Determine the mean and the variance of the total loss from the two claims. [2]
(ii) Find the value of k such that there is a probability of 0.95 that the total loss from the two claims exceeds k.
%%%%%%%%%%%%%%%%%%%%%%%%%%%%%%%%5
\item 
The human resources department of a large insurance company currently estimates that 82\% of new employees recruited by their call centres will still be employed by the company after one year. A recent extension to the call centre business led to 280
new employees being recruited.

Calculate an approximate value for the probability that at least 240 of these new
employees will still be employed by the company after one year.
%%%%%%%%%%%%%%%%%%%%%%%%%%%%%%%%%%%%%%%%%5
% [3]6
% 7
\item The variables X 1 , X 2 , ..., X 40 give the size (in units of £100) of each of 40 claims in a random sample of claims arising from damage to cars by vandals. The size of each claim is assumed to follow a gamma distribution with parameters α = 4 and λ = 0.5
1 40
and each is independent of all others. Let X =
∑ X i be the random variable
40 i = 1
giving the mean size of such a sample.
(i) State the approximate sampling distribution of X and determine its parameters.

(ii) Determine approximately the median of X .

\end{enumerate}
%[1]
%%%%%%%%%%%%%%%%%%%%%%%%%%%%%%%%%%%%%%%%%%%%%%%%%%%%%%%%%%%%%%%%%%%%%%%%%%%%%%%%%%%%

4
(i)
2
E [ Y i ] = 2, Var [ Y i ] = 4
Therefore E [ Y 1 + Y 2 ] = 4 and (since Y 1 , Y 2 independent) Var [ Y 1 + Y 2 ] = 8.
So, for total loss, mean = £4,000 and variance = 8 ×10 6 (£ 2 ).
(OR from Y 1 + Y 2 ~ χ 2 4 )
(ii)
For the total loss we have Y 1 + Y 2 ~ χ 4 2 , so we want a constant k
(
)
2
such that P χ 4 > k = 0.95.
(
)
From tables of the χ 24 distribution P χ 24 > 0 . 7107 = 0.95.
∴The total losses will exceed £710.7 with probability 0.95.
%%%%%%%%%%%%%%%%%%%%%%%%%%%%%%%%%%%%%%%%%%%%%%%%%%%%%%%%%%%%%%%%%%%%%%%%%%%%%%%%%%%%%%%%5
5
Let X be the number still in employment after one year.
∴ X ~ bin (280, 0.82) ≈ N (229.6, 6.43 2 )
P ( X ≥ 240) = P ( X > 239.5) applying a continuity correction
= P ( Z >
6
(i)
239.5 − 229.6
) = P ( Z > 1.54) = 1 − 0.93822 = 0.062
6.43
E[X i ] = 4/0.5 = 8 and Var[X i ] = 4/0.5 2 = 16 (or by noting X i ~ χ 8 2 ).
Var [ X i ] ⎞
∑ X i ⎛
X = i = 1 ≈ N E [ X ],
, i.e. N(8, 0.4)
40
Using the CLT:
⎜
⎝
40
i
40
⎟
⎠
approximately.
[Note: The exact distribution of X is Gamma(160,20)]
(ii)
The symmetry of the distribution gives: median ⎡ ⎣ X ⎤ ⎦ = mean ⎡ ⎣ X ⎤ ⎦ = 8,
i.e. £800.
\end{document}
