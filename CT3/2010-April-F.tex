
CT3 A2010—612
As part of a project in a modelling module, a statistics student is required to submit a
report on the sums insured on home contents insurance policies based on samples of
such policies covering risks in five medium-sized towns in each of England, Wales,
and Scotland. Data are provided on the average sum insured (Y, in units of £1,000)
for each of the 15 towns and are as follows:
England
y 11.9 11.1 9.5 9.2
For these data:
∑ y
13.9
5.9
Wales
9.1 8.0 5.7
8.1
9.3
Scotland
9.1 7.7 8.2
10.4
= 55.6 (England), 36.8 (Wales), 44.7 (Scotland)
overall
∑ y = 137.1, ∑ y 2
= 1,316.63
The student decides to use an analysis of variance approach.
\item 
Suggest brief comments the student should make on the basis of the plot
below:
Individual and mean value plot of England, Wales, Scotland
14
13
12
11
10
9
8
7
6
5
England
Wales
Scotland
[2]
\item 
(a)
(b)
Carry out the analysis of variance on the average sums insured.
Comment on your conclusions.
[6]
The lecturer of the module decides to provide further information. It has been
suggested that the value of a UK index of the town’s prosperity (X) might also be a
useful explanatory variable (in addition to the country in which the town is situated).
The data on the index are as follows (for the towns in the same order as in the first
table):
x
23
27
England
14 19
For these data: overall
CT3 A2010—7
29
15
27
Wales
24 18
22
22
16
Scotland
20 25
28
∑ x = 329 , ∑ x 2 = 7,543 , ∑ xy = 3, 091.7
%%%%%%%%%%%%%%%%%%%%%%%%%%%%%%%%%%%%%%%%
A graph of average sum insured against index (with country identified) is given
below:
Average sum insured v index
*
* England
+ Scotland
% Wales
*
+
*
*
+
+
*
%
+
%
10
%
+
%
%
15
20
25
30
x
The student decides to add the results of a regression approach to her report, using
“index” as an explanatory variable, so she fits the regression model
Y = a + bX + e
using the least squares criterion.
Part of the output from fitting the model using a statistics package on a computer is as
follows:
Coefficients: Estimate
(Intercept) 3.46166
x
0.25889
Residual standard error:
R-Squared: 0.3449
\item 
Std. Error t value Pr(>|t|)
2.21919
1.560 0.1428
0.09896
2.616 0.0213*
1.789 on 13 degrees of freedom
Verify (by performing your own calculations) the following results for the
fitted model as given in the output above:
(a) the fitted regression line is y = 3.462 + 0.2589x
(b) the percentage of the variation in the response (y) explained by the
model (x) is 34.5%
(c) the standard error of the slope estimate is 0.09896
[8]
(iv) Comment briefly on the usefulness of “index” as a predictor for the average
sum insured.
[2]
(v) Suggest another model which you think might be more successful in
explaining the variability in the values of the average sum insured and provide
a better predictor.
[2]
[Total 20]
END OF PAPER
CT3 A2010—8



12
(i)
(ii)
• the three sets of points are positioned at different levels (the means are
shown), so there is a prima facie case for suggesting that the underlying
means are different (i.e. there are country effects)
• the means are in the order England (highest), Scotland, Wales (lowest)
• the variation in the data for Scotland is perhaps lower than that for
England, but with only 5 observations for each country, we cannot be sure
that there is a real underlying difference in variance
(a)
SS T = 1316.63 – 137.1 2 /15 = 63.536, SS B = (55.6 2 + 36.8 2 + 44.7 2 ) /
5 – 137.1 2 /15 = 35.644
∴ SS R = 63.536 – 35.644 = 27.892
Source of variation
Between countries
Residual
Total
Df
2
12
14
SS
MSS
35.644 17.82
27.892 2.324
Under H 0 : no country effects F = 17.82/2.324 = 7.67 on (2,12) df
P-value of F = 7.67 is less than 0.01, so we reject H 0 and conclude that
there are differences among the population means of the average sum
insured
(b)
Page 8
We have strong evidence that country effects exist − the means appear
to be in the order England (highest), Scotland, Wales (lowest).Subject CT3 (Probability and Mathematical Statistics Core Technical) — April 2010 — Examiners’ Report
(iii)
(a)
S xx = 7543 – 329 2 /15 = 326.9333, S yy = 63.536 (from (i)(b) above)
S xy = 3091.7 – 329×137.1/15 = 84.64
ˆ ( 329 /15 ) = 3.4617
β ˆ = 84.64 / 326.9333 = 0.25889 , α ˆ = 137.1/15 − β×
So fitted line is y = 3.462 + 0.2589 x
(b) R 2 = S xy 2 /( S xx S yy ) = 84.64 2 /(326.9333×63.536) = 0.34488 so 34.5%
(c) SSRES = S yy – S xy 2 / S xx = 63.536 – 84.64 2 /326.9333 = 41.62349
⇒ σ ˆ 2 = 41.62349 /13 = 3.201807
()
1/2
⇒ s . e . β ˆ = ( 3.201807 / 326.9333 ) = 0.09896
(iv)
From the plot we see that the relationship between “index” and “average sum
insured” is weak, positive (and possibly linear) – the percentage of the
variation in “average sum insured” explained by the relationship with “index”
is only 34.5%.
So “index” is of some, but limited, use as a predictor of “average sum
insured”.
(v)
We should try a “multiple regression” model which includes “country” and
“index” in the model.
[ Note: although not explicitly in the syllabus, a comment to the effect that
“Country” should be included as a qualitative variable (a “factor”) e.g. by
using a text vector (with entries “ E ”, “ W ”, “ S ” say) or a pair of (Bernoulli)
dummy variables, may attract a bonus for a borderline candidate.]
END OF EXAMINERS’ REPORT
Page 9
