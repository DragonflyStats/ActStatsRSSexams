\documentclass[a4paper,12pt]{article}

%%%%%%%%%%%%%%%%%%%%%%%%%%%%%%%%%%%%%%%%%%%%%%%%%%%%%%%%%%%%%%%%%%%%%%%%%%%%%%%%%%%%%%%%%%%%%%%%%%%%%%%%%%%%%%%%%%%%%%%%%%%%%%%%%%%%%%%%%%%%%%%%%%%%%%%%%%%%%%%%%%%%%%%%%%%%%%%%%%%%%%%%%%%%%%%%%%%%%%%%%%%%%%%%%%%%%%%%%%%%%%%%%%%%%%%%%%%%%%%%%%%%%%%%%%%%

\usepackage{eurosym}
\usepackage{vmargin}
\usepackage{amsmath}
\usepackage{graphics}
\usepackage{epsfig}
\usepackage{enumerate}
\usepackage{multicol}
\usepackage{subfigure}
\usepackage{fancyhdr}
\usepackage{listings}
\usepackage{framed}
\usepackage{graphicx}
\usepackage{amsmath}
\usepackage{chngpage}

%\usepackage{bigints}
\usepackage{vmargin}

% left top textwidth textheight headheight

% headsep footheight footskip

\setmargins{2.0cm}{2.5cm}{16 cm}{22cm}{0.5cm}{0cm}{1cm}{1cm}

\renewcommand{\baselinestretch}{1.3}

\setcounter{MaxMatrixCols}{10}

\begin{document}
\begin{enumerate}
Determine the following probabilities:
5
(i) P[S 2 > σ 2 ]
%%%%%%%%%%%%%%%%%%%%%%%%%%%
[2]
(ii) P[ X > μ ⎜ S 2 > σ 2 ]
%%%%%%%%%%%%%%%%%%%%%%%%%%%
[2]
(iii) P[ X – μ > σ]
%%%%%%%%%%%%%%%%%%%%%%%%%%%
[2]
(iv)

%%%%%%%%%%%%%%%%%%%%%%%%%%%
 P[ X – μ > S ]
%%%%%%%%%%%%%%%%%%%%%%%%%%%
[2]
[Total 8]
A random sample of size 49 from a normal distribution gives a 99% confidence
interval for the population mean as (30, 50).
Determine a 90% confidence interval for the population mean based on this
information.
%%%%%%%%%%%%%%%%%%%%%%%%%%%
[5]
6
A sample is drawn from the normal distribution with mean μ and variance σ 2 . Denote
the sampling variance by S 2 .
(i)
Show that the expected value of S 2 is σ 2 , using the sampling distribution of
S 2 .
%%%%%%%%%%%%%%%%%%%%%%%%%%%
[2]
Due to a spreadsheet error a scientist accidentally uses the following to estimate the
variance
(ii)
1 ⎛ 2
Σ x i − nx 2
n ⎝
⎛
⎝
G 2 =
(a) Determine the bias of G 2 as an estimator of σ 2 .
%%%%%%%%%%%%%%%%%%%%%%%%%%%
[2]
(b) Comment on how the bias behaves as n gets large.
%%%%%%%%%%%%%%%%%%%%%%%%%%%


%%%%%%%%%%%%%%%%%%%%%%%%%%%

CT3 A2018–3 
%%%%%%%%%%%%%%%%%%%%%%%%%%%
Q5
From the 99% CI we know that
x − 2.576
s
s
30 and x + 2.576 =
50 .
=
7
7
[1]
Solving these two equations we obtain
x = 40 and
=
s
70
= 27.17391
2.576
[2]
So 90% CI is
40 ± 1.645
27.17391
7
i.e. (33.614, 46.386).
[2]
[Total 5]
This question was very well answered in general. A common mistake in
wrong answers was using incorrect critical values.
Page 5Subject CT3 (Probability and Mathematical Statistics Core Technical) – April 2018 – Examiners’ Report
Q6
 ( n − 1 ) S 2 
E 
 = n − 1 (since ~ X n 2 − 1 )
2
  σ
 
(i)
[1]
σ 2
⇒ E  S 2  = × ( n − 1 ) =
σ 2
  n − 1
(ii)
(a)
n − 1
n − 1
G 2 = S 2 ⇒ E  G 2  = σ 2
 
n
n
⇒ bias =
(b)
[1]
[1]
n − 1 2
σ 2
σ − σ 2 = −
n
n
[1]
As n gets large the bias tends to zero.
[1]
[Total 5]
Answers in part (i) were mixed, with some candidates attempting to
answer it without using the sampling distribution of the variance. Part
(ii) was better answered, while responses in part (iii) were mixed as
many candidates did not relate the bias arrived in part (ii)(a) to their
answers in this part.
