\documentclass[a4paper,12pt]{article}

%%%%%%%%%%%%%%%%%%%%%%%%%%%%%%%%%%%%%%%%%%%%%%%%%%%%%%%%%%%%%%%%%%%%%%%%%%%%%%%%%%%%%%%%%%%%%%%%%%%%%%%%%%%%%%%%%%%%%%%%%%%%%%%%%%%%%%%%%%%%%%%%%%%%%%%%%%%%%%%%%%%%%%%%%%%%%%%%%%%%%%%%%%%%%%%%%%%%%%%%%%%%%%%%%%%%%%%%%%%%%%%%%%%%%%%%%%%%%%%%%%%%%%%%%%%%

\usepackage{eurosym}
\usepackage{vmargin}
\usepackage{amsmath}
\usepackage{graphics}
\usepackage{epsfig}
\usepackage{enumerate}
\usepackage{multicol}
\usepackage{subfigure}
\usepackage{fancyhdr}
\usepackage{listings}
\usepackage{framed}
\usepackage{graphicx}
\usepackage{amsmath}
\usepackage{chngpage}

%\usepackage{bigints}
\usepackage{vmargin}

% left top textwidth textheight headheight

% headsep footheight footskip

\setmargins{2.0cm}{2.5cm}{16 cm}{22cm}{0.5cm}{0cm}{1cm}{1cm}

\renewcommand{\baselinestretch}{1.3}

\setcounter{MaxMatrixCols}{10}

\begin{document}
\begin{enumerate}
%%-- Question 1

\item A scientist collects the following data sample on the number of plants grown on
newly fertilised plots of land.
Number of plants Frequency
2
1 2
2 6
3 3
4 8
5 1
(i) Calculate the mean, median and mode of the sample.
%%%%%%%%%%%%%%%%%%%%%%%%%%%

(ii)

%%%%%%%%%%%%%%%%%%%%%%%%%%%
 Calculate the standard deviation of the sample.
%%%%%%%%%%%%%%%%%%%%%%%%%%%

\item 
Consider the following data, and the corresponding sums derived from the data:
x i :10.0 6.9 11.4 12.6 10.3 12.4 9.8
∑ x i = 73.4; ∑ x i 2 = 792.22; ∑ x i 3 = 8,750.972.
\begin{enumerate}[(i)]
\item Determine the third moment about the mean for these data.
%%%%%%%%%%%%%%%%%%%%%%%%%%%
\item Write down the mathematical definition of the coefficient of skewness
of a set of data.
%%%%%%%%%%%%%%%%%%%%%%%%%%%
\item Determine the coefficient of skewness for the data above. 
%%%%%%%%%%%%%%%%%%%%%%%%%%%
\end{enumerate}

%%%%%%%%%%%%%%%%%%%%%%%%%%%

\end{enumerate}
\newpage

Q1
(i)
(ii)
mean
=
Σ f i x i 60
= = 3
20
Σ f i
[1]
20 observations so median is 10.5 th value = 3 [1]
mode = 4 [1]
Σ f i x i 2 =
206
s
=
[1]
206 − 20*3 2
= 1.170
19

%[1]
%[Total 5]
%This question was very well answered by most candidates. In part (ii), a common mistake was using n instead of n-1 in the denominator.
(
1 n
1 n 3
3
%%%%%%%%%%%%%%%%%%%%%%%%%%%%%%%%%%%%%%%%%%%%%%%%%%%%%%%%
\newpage
Q2 (i)
x i − 3 x i 2 x + 3 x i x 2 − x 3
( x i − x =
)
∑
∑
n
n
= i 1 = i 1
=
)
(
1
∑ x i 3 − 3 x ∑ x i 2 + 3 x 2 ∑ x i − nx 3
n
)
[1]
2
3
1 
73.4
 73.4 
 73.4  
792.22  3
73.4
7
=  8750.972  3
−
+ 
−


 
7  
7
 7 
 7   
29.316
=
    4.188    
−
=
−
7
[1]
∑ ( x i − x ) / n
3
(ii)
(a)
\begin{itemize}
    \item Coefficient of skewness =
( ∑ ( x − x ) / n )
2
3/2
[1]
i
(b)
(
)
∑ ( x i − x ) / n = ∑ x i 2 − ( ∑ x i ) / n / n = (792.22 −73.4 2 /7)/7 = 3.224
2
2
[1]
\item Coefficient of skewness = − 4.188 / 3.224 1.5 = −0.723
\end{itemize}


%%Page 3Subject CT3 (Probability and Mathematical Statistics Core Technical) – April 2018 – Examiners’ Report

%Answers in part (i) were mixed, with many candidates struggling with the formula and failing to arrive at the correct answer. Parts (ii) and
%(iii) were answered well. Some candidates used n instead of n-1 in the formula for the variance.
\end{document}
