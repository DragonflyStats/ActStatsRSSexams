\documentclass[a4paper,12pt]{article}

%%%%%%%%%%%%%%%%%%%%%%%%%%%%%%%%%%%%%%%%%%%%%%%%%%%%%%%%%%%%%%%%%%%%%%%%%%%%%%%%%%%%%%%%%%%%%%%%%%%%%%%%%%%%%%%%%%%%%%%%%%%%%%%%%%%%%%%%%%%%%%%%%%%%%%%%%%%%%%%%%%%%%%%%%%%%%%%%%%%%%%%%%%%%%%%%%%%%%%%%%%%%%%%%%%%%%%%%%%%%%%%%%%%%%%%%%%%%%%%%%%%%%%%%%%%%

\usepackage{eurosym}
\usepackage{vmargin}
\usepackage{amsmath}
\usepackage{graphics}
\usepackage{epsfig}
\usepackage{enumerate}
\usepackage{multicol}
\usepackage{subfigure}
\usepackage{fancyhdr}
\usepackage{listings}
\usepackage{framed}
\usepackage{graphicx}
\usepackage{amsmath}
\usepackage{chngpage}

%\usepackage{bigints}
\usepackage{vmargin}

% left top textwidth textheight headheight

% headsep footheight footskip

\setmargins{2.0cm}{2.5cm}{16 cm}{22cm}{0.5cm}{0cm}{1cm}{1cm}

\renewcommand{\baselinestretch}{1.3}

\setcounter{MaxMatrixCols}{10}

\begin{document}

7
An analyst wishes to model the number of Initial Public Offerings (IPOs) on the
local stock exchange each calendar month. Let X i be a random variable denoting the
number of IPOs in month i, where all X i ’s are independent. The analyst wishes to
model X i using a Poisson distribution but is aware that there is less activity during the
summer so uses the following rates:
{
l,
\lambda,
i \neq July, August
\lambda
l i =
		
ul,
i = July, August
u\lambda,
(i)
Write down the probability function of X i . .

The analyst wishes to estimate u and \lambda using data from the last 12 months.
(ii)
Show that the log likelihood is given by
l l ( ( x x ; ; \lambda , , u u ) ) = ∑ x x i i ln
ln \lambda − (10
(10 + 2 2 u u ) ) \lambda + ( ( x x Jul
x Aug
ln u u + C
C
Jul + x
Aug ) ) ln
i i
where C is a constant, independent of u and \lambda.
\lambda.

(iii)

S2018–4
CT3 X18–4
Derive the maximum likelihood estimators for u and \lambda. You are not required to
function.
confirm that these estimators maximise the likelihood function.
[5]


Q7
(i)
pp ii (xx) =
(ii)
⎧
⎪
\lambda\lambda xx ee −\lambda\lambda
0, xx < 0
, xx \geq 0, ii \neq JJJJJJJJ, AAAAAAAAAAAA
xx!
⎨ (uuuu) xx ee −uuuu
⎪
, xx \geq 0, ii = JJJJJJJJ, AAAAAAAAAAAA
⎩ xx!
LL(xx; \lambda\lambda, uu) =
�
ii\neqJJJJJJ,AAAAAA
= �
ii
xx ii
\lambda\lambda xx ii ee −\lambda\lambda
xx ii !


�
ii=JJJJJJ,AAAAAA
(uuuu) xx ii ee −uuuu
xx ii !
\lambda\lambda
× �ee −(10+2uu)\lambda\lambda uu xx JJJJJJ +xx AAAAAA �
xx ii !
ll(xx; \lambda\lambda, uu) = � xx ii ln \lambda\lambda − (10 + 2uu)\lambda\lambda + �xx JJJJJJ + xx AAAAAA � ln uu + CC
ii
(iii)
xx JJJJJJ + xx AAAAAA
δδδδ
= −2\lambda\lambda +
=0
δδδδ
uu
⟹ 2\lambda\lambda\lambda\lambda = (xx JJJJJJ + xx AAAAAA )





δδδδ ΣΣxx ii
=
− 10 − 2uu = 0
δδδδ
\lambda\lambda
⟹ ΣΣxx ii − 10\lambda\lambda = 2\lambda\lambda\lambda\lambda = (xx JJJJJJ + xx AAAAAA )

Page 7Subject CT3  %%%%%%%%%%%%%%%%%%%%%%%%%%%%%%%%%%%%%%%%%%%%%%%%%%55
⟹ \lambda\lambdâ =
�
ii\neqJJJJJJ,AAAAAA
xx ii /10

⟹ uu� = (xx JJJJJJ + xx AAAAAA )/2\lambda\lambdâ
= (xx JJJJJJ + xx AAAAAA )/(2
= (xx JJJJJJ + xx AAAAAA )/(
�
ii\neqJJJJJJ,AAAAAA
�
ii\neqJJJJJJ,AAAAAA
xx ii /10)
xx ii /5)

[Total 10]
There were mixed answers in parts (i) and (ii), often with poor notation
for the likelihood. Part (iii) was well answered.
\end{document}
