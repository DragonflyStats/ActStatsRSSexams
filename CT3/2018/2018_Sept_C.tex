\documentclass[a4paper,12pt]{article}

%%%%%%%%%%%%%%%%%%%%%%%%%%%%%%%%%%%%%%%%%%%%%%%%%%%%%%%%%%%%%%%%%%%%%%%%%%%%%%%%%%%%%%%%%%%%%%%%%%%%%%%%%%%%%%%%%%%%%%%%%%%%%%%%%%%%%%%%%%%%%%%%%%%%%%%%%%%%%%%%%%%%%%%%%%%%%%%%%%%%%%%%%%%%%%%%%%%%%%%%%%%%%%%%%%%%%%%%%%%%%%%%%%%%%%%%%%%%%%%%%%%%%%%%%%%%

\usepackage{eurosym}
\usepackage{vmargin}
\usepackage{amsmath}
\usepackage{graphics}
\usepackage{epsfig}
\usepackage{enumerate}
\usepackage{multicol}
\usepackage{subfigure}
\usepackage{fancyhdr}
\usepackage{listings}
\usepackage{framed}
\usepackage{graphicx}
\usepackage{amsmath}
\usepackage{chngpage}

%\usepackage{bigints}
\usepackage{vmargin}

% left top textwidth textheight headheight

% headsep footheight footskip

\setmargins{2.0cm}{2.5cm}{16 cm}{22cm}{0.5cm}{0cm}{1cm}{1cm}

\renewcommand{\baselinestretch}{1.3}

\setcounter{MaxMatrixCols}{10}

\begin{document}

%%%%%%%%% 6
A poll was conducted with respect to a future referendum, where voters will answer either “yes” or “no” to a question on a political issue. A random sample of 1,106 eligible voters were asked in the poll and 608 of them answered “no”. Let p denote
the true population proportion of “no” voters.
We wish to predict the result of the referendum, by testing the hypotheses H 0 : p = 0.5 against H 1 : p > 0.5.
\begin{enumerate}[(a)]
\item (i) Perform a suitable test of these hypotheses at the 5% level of significance,
stating your conclusion in terms of a predicted result for the referendum.  
\item (ii) (a) Determine a 90% central confidence interval for p.
(b) Explain the effect of using a larger sample on the confidence interval in
part (ii)(a). 

(a) Give the definition of the P-value of a hypothesis test.
(b) Derive the P-value of the test in part (i) stating your conclusion in
terms of a predicted result for the referendum.
(c) Comment, with reference to your answers in parts (i) and (iii)(b), on
the advantages of using a P-value approach for testing, compared with
level.
a fixed significance level.

[Total 14]
\end{itemize}
%%--(iii)

%%%%%%%%%%%%%%%%%%%%%%%%%%%%%%%%%%%%%%%%%%%
7
An analyst wishes to model the number of Initial Public Offerings (IPOs) on the local stock exchange each calendar month. Let X i be a random variable denoting the number of IPOs in month i, where all X i ’s are independent. The analyst wishes to
model X i using a Poisson distribution but is aware that there is less activity during the
summer so uses the following rates:
{
l,
\lambda,
i ≠ July, August
\lambda
l i =
		
ul,
i = July, August
u\lambda,
\begin{itemize}
\item (i)
Write down the probability function of X i . .

The analyst wishes to estimate u and \lambda using data from the last 12 months.
\item (ii)
Show that the log likelihood is given by
l l ( ( x x ; ; \lambda , , u u ) ) = ∑ x x i i ln
ln \lambda − (10
(10 + 2 2 u u ) ) \lambda + ( ( x x Jul
x Aug
ln u u + C
C
Jul + x
Aug ) ) ln
i i
where C is a constant, independent of u and \lambda.
\lambda.

\item (iii)

S2018–4
CT3 X18–4
Derive the maximum likelihood estimators for u and \lambda. You are not required to
function.
confirm that these estimators maximise the likelihood function.
\end{itemize}

%%%%%%%%%%%%%%%%%%%%%%%%%%%%%%%%%%%%%%%%%%%%%%%%%%%%%%%%%%%%%%%%%%%%%%%%%%%%%%%%%%%%%
\newpage

Q6
(i)
If X denotes the “no” voters, under H 0 we have
X ~ Binomial(1106, 0.5), or approximately X ~ N(553, 276.5)

Using a continuity correction, the z statistic is given as
zz =
607.5−553
√276.5
= 3.28
[1.5]
Critical point for tables is z 0.05 = 1.6449.
[0.5]
So we reject H 0 at the 5% level in favour of H 1 ,
which means that we have evidence of a “no” vote.
(ii)
pp� (1−pp� )
(a) 90% CI is given by pp̂ ± 1.6449 �
This gives:
608
(
± 1.6449 �
1106
nn

608
)(1−608/1106)
1106
1106

, i.e. (0.525, 0.574)

(b) A larger sample would reduce the standard error of pp̂ and would therefore give a
narrower interval.

(iii)
(a) The P-value is the probability, assuming H 0 is true, of observing a test statistic at
least as “extreme” as the value observed (or, it is the lowest significance level at
which H 0 can be rejected).

(b) PP-value = PP(XX ≥ 608) = PP �ZZ >
607.5−553
√276.5
� = PP(ZZ > 3.28) = 0.00052

We have very strong evidence against H 0 , which means that we have very strong
evidence of a “no” vote.

(c) Using a fixed level does not provide clear detailed information on the strength of
the evidence against H 0 , whereas using a P-value is more informative about the
Page 6Subject CT3  – September 2018 – Examiners’ Report
strength of this evidence.
Here, using the P-value approach clearly tells us about how strong the evidence
against H 0 is, which means we can put our conclusion in stronger terms.

[Total 14]
Generally well answered. Notice that a continuity correction is needed
in this question for full marks. In part (i) many candidates mistakenly
used the sample estimate of p in the variance.
Q7
(i)
pp ii (xx) =
(ii)
⎧
⎪
\lambda\lambda xx ee −\lambda\lambda
0, xx < 0
, xx ≥ 0, ii ≠ JJJJJJJJ, AAAAAAAAAAAA
xx!
⎨ (uuuu) xx ee −uuuu
⎪
, xx ≥ 0, ii = JJJJJJJJ, AAAAAAAAAAAA
⎩ xx!
LL(xx; \lambda\lambda, uu) =
�
ii≠JJJJJJ,AAAAAA
= �
ii
xx ii
\lambda\lambda xx ii ee −\lambda\lambda
xx ii !


�
ii=JJJJJJ,AAAAAA
(uuuu) xx ii ee −uuuu
xx ii !
\lambda\lambda
× �ee −(10+2uu)\lambda\lambda uu xx JJJJJJ +xx AAAAAA �
xx ii !
ll(xx; \lambda\lambda, uu) = � xx ii ln \lambda\lambda − (10 + 2uu)\lambda\lambda + �xx JJJJJJ + xx AAAAAA � ln uu + CC
ii
(iii)
xx JJJJJJ + xx AAAAAA
δδδδ
= −2\lambda\lambda +
=0
δδδδ
uu
⟹ 2\lambda\lambda\lambda\lambda = (xx JJJJJJ + xx AAAAAA )





δδδδ ΣΣxx ii
=
− 10 − 2uu = 0
δδδδ
\lambda\lambda
⟹ ΣΣxx ii − 10\lambda\lambda = 2\lambda\lambda\lambda\lambda = (xx JJJJJJ + xx AAAAAA )

Page 7Subject CT3  –September 2018 – Examiners’ Report
⟹ \lambda\lambdâ =
�
ii≠JJJJJJ,AAAAAA
xx ii /10

⟹ uu� = (xx JJJJJJ + xx AAAAAA )/2\lambda\lambdâ
= (xx JJJJJJ + xx AAAAAA )/(2
= (xx JJJJJJ + xx AAAAAA )/(
�
ii≠JJJJJJ,AAAAAA
�
ii≠JJJJJJ,AAAAAA
xx ii /10)
xx ii /5)

[Total 10]
There were mixed answers in parts (i) and (ii), often with poor notation
for the likelihood. Part (iii) was well answered.
