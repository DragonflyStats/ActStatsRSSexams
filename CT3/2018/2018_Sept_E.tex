\documentclass[a4paper,12pt]{article}

%%%%%%%%%%%%%%%%%%%%%%%%%%%%%%%%%%%%%%%%%%%%%%%%%%%%%%%%%%%%%%%%%%%%%%%%%%%%%%%%%%%%%%%%%%%%%%%%%%%%%%%%%%%%%%%%%%%%%%%%%%%%%%%%%%%%%%%%%%%%%%%%%%%%%%%%%%%%%%%%%%%%%%%%%%%%%%%%%%%%%%%%%%%%%%%%%%%%%%%%%%%%%%%%%%%%%%%%%%%%%%%%%%%%%%%%%%%%%%%%%%%%%%%%%%%%

\usepackage{eurosym}
\usepackage{vmargin}
\usepackage{amsmath}
\usepackage{graphics}
\usepackage{epsfig}
\usepackage{enumerate}
\usepackage{multicol}
\usepackage{subfigure}
\usepackage{fancyhdr}
\usepackage{listings}
\usepackage{framed}
\usepackage{graphicx}
\usepackage{amsmath}
\usepackage{chngpage}

%\usepackage{bigints}
\usepackage{vmargin}

% left top textwidth textheight headheight

% headsep footheight footskip

\setmargins{2.0cm}{2.5cm}{16 cm}{22cm}{0.5cm}{0cm}{1cm}{1cm}

\renewcommand{\baselinestretch}{1.3}

\setcounter{MaxMatrixCols}{10}

\begin{document}

%%%%%%%%% 9
For an investigation into drinking habits a random sample of men aged 16–90 is
obtained. The following data are reported for men belonging to different age groups:
Age group
16–24 25–44 45–64 65 and over
Average units per week 3.5 4.8 5.1 4.2
Sample standard deviation 2.3 1.8 1.6 1.1
Sample size 50 65 60 35
(i)
Calculate a 95% confidence interval for the expected value of the average units
of alcohol per week consumed by men aged 16–24 based on the sample above.


(ii) Calculate the overall average units of alcohol per week consumed by men aged
above.
16–90 in the sample above.

(iii) Test the hypothesis, using an analysis of variance, that the mean number of
units of alcohol per week is the same for all age groups. 
[8]
(iv) Calculate a 95% confidence interval for the expected units of alcohol per week
above.
consumed by all men aged 16–90 based on the sample above.

(v) Comment on your results in parts (iii) and (iv), in particular, whether the result
in part (iv) should be used to draw inference about the drinking habits of an
individual.
individual.

[Total 17]

S2018–6

%%%%%%%%%%%%%%%%%%%%%%%%%%%%%%%%%%%%%%%%%%%%%%
Q9
(i)
Using quantiles of the tt 50 -distribution as an approximation to the required tt 49 -
distribution.
2.3
2.3
�3.5 − 2.009
, 3.5 + 2.009 � = [2.8465, 4.1535]
√50
(ii)
(iii)
√50
Total sample size: 50+65+60+35=210

Total units: 50 × 3.5 + 65 × 4.8 + 60 × 5.1 + 35 × 4.2 = 940
940
Overall average: 210 = 4.476
(iv)

ANOVA:
SSSS BB = 50 × (3.5 − 4.476) 2 + 65 × (4.8 − 4.476) 2
+60 × (5.1 − 4.476) 2 + 35 × (4.2 − 4.476) 2
= 80.48
SSSS RR = 49 × 2.3 2 + 64 × 1.8 2 + 59 × 1.6 2 + 34 × 1.1 2 = 658.75
80.48/3
Test statistic: FF = 658.75/206 = 8.3891
This compares to a 1% quantile of a FF 3,206 distribution.






This quantile is between3.782 and 3.949, and we therefore have sufficient evidence to reject the null hypothesis that the average number of units of alcohol per week is the same for all age groups.

%%-- Page 9Subject CT3  –September 2018 – Examiners’ Report
(v)
Overall variance in sample:
1
1
1
SSSS TT = 209 (SSSS RR + SSSS BB ) = 209 (658.75 + 80.48) =3.54
209
3.54
3.54
95% C.I.: �4.476 − 1.96� 210 , 4.476 + 1.96� 210 � = [4.222,4.73]


[Alternative solution: σσ� 2 = SSSS RR /(nn − kk). Then CI is (4.234,4.718).]
(vi)
The results in part (iii) indicate that age has an impact on drinking habits,
and therefore, the overall average of units per week and the corresponding
confidence interval in part (iv) might not be meaningful to describe the drinking
habits of any specific individual.

[Total 17]

% Generally well answered. In part (ii) a few candidates calculated a simple mean instead of a weighted average. There were mixed answers in part (v) with many candidates failing to comment meaningfully on their results.

\end{document}
