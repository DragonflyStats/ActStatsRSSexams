\documentclass[a4paper,12pt]{article}

%%%%%%%%%%%%%%%%%%%%%%%%%%%%%%%%%%%%%%%%%%%%%%%%%%%%%%%%%%%%%%%%%%%%%%%%%%%%%%%%%%%%%%%%%%%%%%%%%%%%%%%%%%%%%%%%%%%%%%%%%%%%%%%%%%%%%%%%%%%%%%%%%%%%%%%%%%%%%%%%%%%%%%%%%%%%%%%%%%%%%%%%%%%%%%%%%%%%%%%%%%%%%%%%%%%%%%%%%%%%%%%%%%%%%%%%%%%%%%%%%%%%%%%%%%%%

\usepackage{eurosym}
\usepackage{vmargin}
\usepackage{amsmath}
\usepackage{graphics}
\usepackage{epsfig}
\usepackage{enumerate}
\usepackage{multicol}
\usepackage{subfigure}
\usepackage{fancyhdr}
\usepackage{listings}
\usepackage{framed}
\usepackage{graphicx}
\usepackage{amsmath}
\usepackage{chngpage}

%\usepackage{bigints}
\usepackage{vmargin}

% left top textwidth textheight headheight

% headsep footheight footskip

\setmargins{2.0cm}{2.5cm}{16 cm}{22cm}{0.5cm}{0cm}{1cm}{1cm}

\renewcommand{\baselinestretch}{1.3}

\setcounter{MaxMatrixCols}{10}

\begin{document}

CT3 X18–610
A statistician has a series of bivariate data {(x 1 ,y 1 ), (x 2 ,y 2 ), ... (x n ,y n )} and wishes to
perform a linear regression on these data.
(i) State the equation that must be minimised to give the least squares estimates of
coefficients.
the regression coefficients.

(ii) Derive the least squares estimate of the slope coefficient from the equation in
(i).
part (i).

For a sample of 44 fish, the age (days) and length (millimetres) of each fish are
measured. Denote age by X and length by Y. The following summary data are
obtained:
3,660, ∑ x x i i 2 2 = 389,684,
389,684, ∑ y y i i = 136,727,
136,727, ∑ y y i i 2 2 = 500,813,951,
500,813,951,
∑ x x i i = 3,660,
13,609,918
∑ x x i i y y i i = 13,609,918
(iii) Determine the coefficients for a linear regression of Y on X.
X.

(iv)
 Calculate the sample correlation coefficient between x and y.
y.
\end{enumerate}
%[Total 13]
%END OF PAPER
%S2018–7 
CT3 X18–7
%%%%%%%%%%%%%%%%%%%%%%%%%%%%%%%%%%%%%%%%%%%%%%%%%%%%%%%%%%%%%%%%%%%%%%%%%%
\newpage
Q10
(i) ∑ nnii=1 ee ii2 = ∑ nnii=1 �yy ii − (αα + ββxx ii ) �

2
(ii) Partially differentiate w.r.t. each parameter and equate to zero gives:
2 ∑ nnii=1 [yy ii − (αα� + ββ̂ xx ii )] = 0 ⇒ ∑ nnii=1 yy ii = nnαα� + ββ̂ ∑ nnii=1 xx ii

2 ∑ nnii=1 xx ii [yy ii − (αα� + ββ̂ xx ii )] = 0 ⇒ ∑ nnii=1 xx ii yy ii = αα� ∑ nnii=1 xx ii + ββ̂ ∑ nnii=1 xx ii2
Eliminate αα� from simultaneous equations:
nn
nn nn nn
ii=1 ii=1
nn ii=1
nn
�� xx ii � � yy ii = nnαα� � xx ii + ββ̂ �� xx ii �
ii=1
nn
2
nn � xx ii yy ii = nnαα� � xx ii + nnββ̂ � xx ii2
ii=1
ii=1
ii=1
⇒ nn ∑ nnii=1 xx ii yy ii − (∑ nnii=1 xx ii )(∑ nnii=1 yy ii ) = ββ̂ (nn ∑ nnii=1 xx ii2 − [∑ nnii=1 xx ii ] 2 )
⇒ ββ̂ = �nn ∑ nnii=1 xx ii yy ii − (∑ nnii=1 xx ii )(∑ nnii=1 yy ii )� � (nn ∑ nnii=1 xx ii2 − [∑ nnii=1 xx ii ] 2 )
Page 10



Subject CT3  – September 2018 – Examiners’ Report
\begin{itemize}
\item (iii) SS xxxx = 389,684 − (3,660) 2 /44 = 85,238.55
SS xxxx = 13,609,918 − (3,660 × 136,727)/44 = 2,236,718
xx̅ =
3,660
44
= 83.1818, yy� = 136,727/44 = 3,107.43
ββ̂ = SS xxxx /SS xxxx = 2,236,718/85,238.55 = 26.241
αα� = yy� − ββ̂ xx̅ = 3,107.43 − 26.241 × 83.1818 = 924.68
\item (iv) SS yyyy = 500,813,951 − (136,727) 2 /44 = 75,944,121
rr = SS xxxx / �SS xxxx SS yyyy = 2,236,718/�85,238.55 × 75,944,121 = 0.879
\end{itemize}






%% [Total 13]
%% Parts (i) and (ii) were more focussed on methodology compare to other questions and were not well answered. Parts (iii) and (iv) were very well answered.
%% END OF EXAMINERS’ REPORT
%% Page 11
\end{document}
