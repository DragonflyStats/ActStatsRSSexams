\documentclass[a4paper,12pt]{article}

%%%%%%%%%%%%%%%%%%%%%%%%%%%%%%%%%%%%%%%%%%%%%%%%%%%%%%%%%%%%%%%%%%%%%%%%%%%%%%%%%%%%%%%%%%%%%%%%%%%%%%%%%%%%%%%%%%%%%%%%%%%%%%%%%%%%%%%%%%%%%%%%%%%%%%%%%%%%%%%%%%%%%%%%%%%%%%%%%%%%%%%%%%%%%%%%%%%%%%%%%%%%%%%%%%%%%%%%%%%%%%%%%%%%%%%%%%%%%%%%%%%%%%%%%%%%

\usepackage{eurosym}
\usepackage{vmargin}
\usepackage{amsmath}
\usepackage{graphics}
\usepackage{epsfig}
\usepackage{enumerate}
\usepackage{multicol}
\usepackage{subfigure}
\usepackage{fancyhdr}
\usepackage{listings}
\usepackage{framed}
\usepackage{graphicx}
\usepackage{amsmath}
\usepackage{chngpage}

%\usepackage{bigints}
\usepackage{vmargin}

% left top textwidth textheight headheight

% headsep footheight footskip

\setmargins{2.0cm}{2.5cm}{16 cm}{22cm}{0.5cm}{0cm}{1cm}{1cm}

\renewcommand{\baselinestretch}{1.3}

\setcounter{MaxMatrixCols}{10}

\begin{document}
[Total 10]8
An insurance company believes that claim amounts in a certain portfolio of policies
follow a normal distribution. An analyst chose 61 policies at random which gave a
sample mean of £523 and a sample standard deviation of £81.

\begin{enumerate}[(a)]
\item (i) Determine a 95\% confidence interval for the mean claim amount in the
portfolio.
portfolio.
\item 
(ii) Determine a 95\% confidence interval for the variance of claim amounts in the
portfolio.
portfolio.
\item 
The company assumes that the true mean and standard deviation of claim amounts are
the same as those in the sample.
The number of claims per month for the portfolio follows a Poisson process with
mean 250.
(iii)
(iv)
\item 
Determine the approximate probability that the number of claims in a
use.
particular month exceeds 270, justifying any assumptions you use.
\item 
Determine the mean and standard deviation for the total annual amount of
portfolio.
claims in the portfolio.
\end{enumerate}
%%%%%%%%%%%%%%%%%%%%%%%%%%%%%%%%%%%%%%%
The company has changed its loss assessment processes in order to reduce claim sizes
on average, targeting a reduction of £20 compared to the current mean. It does not
expect a change to the variability of claim amounts. The company intends to verify
whether the target has been met by using a sample of claims to test the null hypothesis
that there is no change, against a one-sided alternative hypothesis. Company policy is
to perform statistical tests at a significance level of 5%.
(v)
Determine the smallest number of claims that would need to be sampled under
the new processes for a £20 reduction to be statistically significant in the test.



[Total 20]
S2018–5 
CT3 X18–5

%%%%%%%%%%%%%%%%%%%%
Q8
C. I. = 523 ±
(i)
√61
= (502.3,543.7)
nn−1;0.025
60×81 2 60×81 2
= � 40.48 , 83.30 �
= (4726, 9725)
(iii)


(nn−1)ss 2
C. I. = � ΧΧ 2

√nn
81
= 523 ± 2.000 ×
(ii)
tt 60;.975 ss
(nn−1)ss 2
, ΧΧ 2
nn−1;0.975
�
By the CLT and as \lambda\lambda is large NN~NN(\lambda\lambda, \lambda\lambda) = NN(250,250)
PP(NN > 270) = PP(NN > 270.5) continuity correction
(iv)
= PP �ZZ >
270.5−250
√250
� = PP(ZZ > 1.297) = 1 − 0.903 = 0.097
Now the rate of claims is 12 × \lambda\lambda = 3000





= √3000 ∗ √81 2 + 523 2 =28987 
Want smallest nn such that PP(XX� < 503) ≤ 0.05 under H 0
XX�−μμ
i.e. PP � σσ/
Page 8


Mean = \lambda\lambda nnnnnn ∗ μμ cccccccccccc = 3000 ∗ 523 = 1,569,000
2
Standard deviation = �\lambda\lambda aaaaaaaaaaaa ∗ �(σσ cccccccccccc
+ μμ 2 )
(v)

√
<
nn
503−μμ
σσ/ √nn
� ≤ 0.05

%%-- Subject CT3  – September 2018 – Examiners’ Report
−1.6449 ≥
503−523
81/ √nn
1.6449×81 2
= −
20√nn
81

⟹ nn ≥ � 20 � = 44.38
i.e. nn is at least 45 claims.


%%-- [Total 20]
%%-- Parts (i) and (ii) were very well answered. Part (iii) was generally well answered, although many candidates failed to justify the normal approximation and/or to apply a continuity correction. In part (iv) many candidates performed the calculations for the monthly amounts rather than annual.
\end{document}
