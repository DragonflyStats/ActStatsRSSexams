\documentclass[a4paper,12pt]{article}

%%%%%%%%%%%%%%%%%%%%%%%%%%%%%%%%%%%%%%%%%%%%%%%%%%%%%%%%%%%%%%%%%%%%%%%%%%%%%%%%%%%%%%%%%%%%%%%%%%%%%%%%%%%%%%%%%%%%%%%%%%%%%%%%%%%%%%%%%%%%%%%%%%%%%%%%%%%%%%%%%%%%%%%%%%%%%%%%%%%%%%%%%%%%%%%%%%%%%%%%%%%%%%%%%%%%%%%%%%%%%%%%%%%%%%%%%%%%%%%%%%%%%%%%%%%%

\usepackage{eurosym}
\usepackage{vmargin}
\usepackage{amsmath}
\usepackage{graphics}
\usepackage{epsfig}
\usepackage{enumerate}
\usepackage{multicol}
\usepackage{subfigure}
\usepackage{fancyhdr}
\usepackage{listings}
\usepackage{framed}
\usepackage{graphicx}
\usepackage{amsmath}
\usepackage{chngpage}

%\usepackage{bigints}
\usepackage{vmargin}

% left top textwidth textheight headheight

% headsep footheight footskip

\setmargins{2.0cm}{2.5cm}{16 cm}{22cm}{0.5cm}{0cm}{1cm}{1cm}

\renewcommand{\baselinestretch}{1.3}

\setcounter{MaxMatrixCols}{10}

\begin{document}
\begin{enumerate}
[1]
[Total 5]
PLEASE TURN OVER7
We consider a random sample X 1 , ..., X n from a normal distribution with expectation
1
E [ X i ] = θ 2 and variance V(X i ) = σ 2 for all i.
2
(i)
(ii)
(iii)

%%%%%%%%%%%%%%%%%%%%%%%%%%%

8
Derive an estimator θ̂ for the unknown parameter θ using the method of
moments.
%%%%%%%%%%%%%%%%%%%%%%%%%%%
[3]
Estimate θ for a sample of size 200 for which
derived in part (i).
%%%%%%%%%%%%%%%%%%%%%%%%%%%

200
∑ x i = 900 using the estimator
i = 1
[1]
Comment on the suitability of the estimator in part (i). In particular, consider
n
n

%%%%%%%%%%%%%%%%%%%%%%%%%%%

the two cases ∑ X i < 0 and ∑ X i ≥ 0.
[1]
i = 1
i = 1
[Total 5]
Consider a one-way analysis of variance.
(i)
List the assumptions required for an analysis of variance to be valid.
[2]
A study involving three different types of treatment for a medical condition gave the
following sums on a particular improvement score (y):
n Σ y i Σ y i 2
Treatment 1 8 193.03 4,697.80
Treatment 2 9 259.49 7,508.30
Treatment 3 9 263.08 7,730.34
In the above table, n denotes the number of patients receiving each of the three types
of treatment.
(ii)

%%%%%%%%%%%%%%%%%%%%%%%%%%%

CT3 A2018–4
Test the hypothesis, using an analysis of variance, that the mean improvement
score for each type of treatment is the same.
%%%%%%%%%%%%%%%%%%%%%%%%%%%

[6]
[Total 8]9
The random variables X and Y have joint probability density function (pdf)
⎧ 24 x 3 y for 0 < x < y < 1,
f X,Y (x,y) = ⎨
otherwise.
⎩ 0
(i)
(a)
Show that the marginal pdf of X is
f X   (x) = 12x 3 (1 – x 2 ), 0 <
x < 1.
(b)
Show that the marginal pdf of Y is f Y   (y) = 6y 5 , 0 < y < 1.

%%%%%%%%%%%%%%%%%%%%%%%%%%%
[2]
(ii) Determine the covariance cov (X, Y).
%%%%%%%%%%%%%%%%%%%%%%%%%%%
[5]
(iii) Determine the conditional pdf f X |Y (x| y) together with the range of X for which
it is defined. 
%%%%%%%%%%%%%%%%%%%%%%%%%%%

[2]
(vi)

%%%%%%%%%%%%%%%%%%%%%%%%%%%

10
⎛
⎝
(v)
1
1
Determine the conditional probability P ⎛⎝ X > | Y = .
%%%%%%%%%%%%%%%%%%%%%%%%%%%
[2]
3
2
1
Determine the conditional expectation E ⎛⎝ X | Y = .
%%%%%%%%%%%%%%%%%%%%%%%%%%%
[3]
4   
⎛
⎝
(iv)
Verify that E[E[X |Y ]] = E[X ] by evaluating each side of the equation.
%%%%%%%%%%%%%%%%%%%%%%%%%%%

[3]
[Total 17]
A large pension scheme regularly investigates the lifestyle of its pensioners using surveys. In successive surveys it draws a random sample from all pensioners in the
scheme and it obtains the following data on whether the pensioners smoke.
Survey 1: Of 124 pensioners surveyed, 36 were classed as smokers.
Survey 2: Of 136 pensioners surveyed, 25 were classed as smokers.
An actuary wants to investigate, using statistical testing at a 5\% significance level,
whether there have been significant changes in the proportion of pensioners, p, who
smoke in the entire pension scheme.
(i)
Perform a statistical test, without using a contingency table, to determine if the
proportion p has changed from the first survey to the second.
%%%%%%%%%%%%%%%%%%%%%%%%%%%

[7]
When a third survey is performed it is found that 26 out of the 141 surveyed
pensioners, are smokers.
(ii)
(iii)
Perform a statistical test using a contingency table to determine if the
proportion p is different among the three surveys.
%%%%%%%%%%%%%%%%%%%%%%%%%%%

(a)
[7]
Calculate the proportion of smokers in the third survey.
(b)
Comment on your answers to parts (i) and (ii).

%%%%%%%%%%%%%%%%%%%%%%%%%%%
[2]

%%%%%%%%%%%%%%%%%%%%%%%%%%%

[Total 16]
CT3 A2018–5 
%%%%%%%%%%%%%%%%%%%%%%%%%%%

Q7
We have X =
(i)
1 2
θ
2
[1]
and solving for θ we obtain the estimator θ ˆ = ± 2X [1]
only if ∑ x i ≥ 0 . [1]
9
ˆ = − 3 .
For the given sample we obtain xx̅ = 2 and θ ˆ = 3 or θ
(ii)
[1]
n
(iii)
Since we consider a normal distribution it is possible that
∑ X i < 0 in which
i = 1
case the estimator in part (i) is not defined.
Part (i) was answered well by most candidates, but only a relatively
small number achieved full marks. A common mistake was not
including the negative solution, while only a small number of
Page 6
[1]
[Total 5]Subject CT3 (Probability and Mathematical Statistics Core Technical) – April 2018 – Examiners’ Report
candidates noted that the solution was only valid for a non-negative
sum. Parts (ii) and (iii) were answered well in general.
Q8
(i)
The variances of the errors are common to all treatments.
The errors are independently distributed
with a N(0,σ 2 ) distribution.
(ii)
[1]
[0.5]
[0.5]
y .. = 193.03 + 259.49 + 263.08 = 715.6
715.6 2
= 240.93
26
SST = 4697.8 + 7508.3 + 7730.34 −
 193.03 2 259.49 2 263.08 2  715.6 2
+
+
= 133.85
SSB = 
  −

8
9
9
26


df
SS
MSS
F stat
Between
Residual
Treatment
2
23
25
133.85
107.08
240.93
66.93
4.66
[1]
[1]
14.36
[2]
F 2,23;0.95
= 3.422 < F stat
[1]
Therefore reject H 0 and conclude that the means are not the same at 5% level
[1]
[Total 8]
Generally very well answered. There were some calculation errors in
part (ii).
