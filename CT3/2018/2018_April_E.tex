\documentclass[a4paper,12pt]{article}

%%%%%%%%%%%%%%%%%%%%%%%%%%%%%%%%%%%%%%%%%%%%%%%%%%%%%%%%%%%%%%%%%%%%%%%%%%%%%%%%%%%%%%%%%%%%%%%%%%%%%%%%%%%%%%%%%%%%%%%%%%%%%%%%%%%%%%%%%%%%%%%%%%%%%%%%%%%%%%%%%%%%%%%%%%%%%%%%%%%%%%%%%%%%%%%%%%%%%%%%%%%%%%%%%%%%%%%%%%%%%%%%%%%%%%%%%%%%%%%%%%%%%%%%%%%%

\usepackage{eurosym}
\usepackage{vmargin}
\usepackage{amsmath}
\usepackage{graphics}
\usepackage{epsfig}
\usepackage{enumerate}
\usepackage{multicol}
\usepackage{subfigure}
\usepackage{fancyhdr}
\usepackage{listings}
\usepackage{framed}
\usepackage{graphicx}
\usepackage{amsmath}
\usepackage{chngpage}

%\usepackage{bigints}
\usepackage{vmargin}

% left top textwidth textheight headheight

% headsep footheight footskip

\setmargins{2.0cm}{2.5cm}{16 cm}{22cm}{0.5cm}{0cm}{1cm}{1cm}

\renewcommand{\baselinestretch}{1.3}

\setcounter{MaxMatrixCols}{10}

\begin{document}
\begin{enumerate}
%%%%%%%%%%%%%%%%%%%%%%%%%%%%%%%%%%%%%%%%%%%%%%%%%%%%%%%%%%%%%%%%%%%%%%%%%%%%%%%%%%%%%%%%%%%%%%%%%%%%%%%%%%%%%%%%%%%%%%%%%%%%%5
Q9
(i)
(a)
f X =
( x )
1
\int^{\infty}_{0} 24 x
3
(
)
y =
dy 12 x 3 1 − x 2 ,      0 < x < 1

x
(b)
f =
Y ( y )
y
y dx
\int^{\infty}_{0} 24 x =
3
6 y 5 ,     0 < y < 1

0
%%% Page 7Subject CT3  – April 2018 – Examiners’ Report
E [ X ] =
(ii)
1
1
\[\int^{\infty}_{0} xf X ( x ) dx = \int^{\infty}_{0} x 12 x ( 1 − x )\]
3
0
=
E [ Y ]
2
0
1
 x 5 x 7 
dx =
  1
2  −  = 24 / 35    
7  
  5
0

1
y 6  
y dy
\int^{\infty}_{0} =
5

6/7
0
1 y
3
=
E [ XY ]   24
y   
dx dy
\int^{\infty}_{0}\int^{\infty}_{0} xy x =
00
1
24 7
=
y dy 3 / 5
5 \int^{\infty}_{0}
0
\[cov ( X , Y ) = E ( XY ) − E ( X ) E ( Y ) \] =
%%%%%%%%%%%%%%%%%%%%%%%%%%%%%%%%%%%%%%%%%%%%%%%%%%%%%%%%%%%%%%%%%%%%%%%%%
(iii)
(iv)
3 24 6
3
− × =
5 35 7 245
y
y
\[\int^{\infty}_{0} xf X | Y ( x | y ) dx = \int^{\infty}_{0} x 4 x\]
0
0
3 − 4
=
y dx
4  5  y 4
x =
y
5 y 4   0 5
So, E ( X
=
| Y 1/ =
4 ) 1/ 5 .
(vi)

f X , Y ( x , y ) 24 x 3 y
f X | Y =
= =
4 x 3 y − 4 for 0 < x < y, or 0 otherwise.
( x | y )   
5
f Y ( y )
6 y

1/2
1/2
1/2
1
1 
65

 1 
P  X > | Y =  = \int^{\infty}_{0} f X | Y  x |  dx = \int^{\infty}_{0} 4 x 3 2 4 dx = 16   x 4   =

1/3
3
2  1/3
81

 2 
1/3
= y =
E ( X | Y
)
(v)



4
From (v) we have E [ X | Y ] =    Y .
5
 4 
Therefore, E   E [ X | Y ]   = E  Y  and using (ii):
 5 
E   E [ X | Y =
]  
4 6
= 24 / 35 .
5 7
This is the same as E [ X ] from (ii).
Part (i) was answered correctly by the majority of candidates. Part (ii)
was well answered, although many candidates used incorrect limits of
integration. Part (iii) was generally answered well. Parts (iv)-(vi) were
not answered particularly well, while a number of candidates did not
attempt them at all.
Page 8
%%%%%%%%%%%%%%%%%%%%%%%%%%%%%%%%%%%%%%%%%%%%%%%%%%%%%%%%%%%%%%%%%%%%%%%%%%%%%%%%%%%%%%%%%%%%%%%%%%%%%%%%%%%5
%%[Total 17]Subject CT3  – April 2018 – Examiners’ Report
In part (iv) many candidates did not use the correct expression, and in
part (v) candidates struggled with the calculation and the correct limits
of integration. In part (vi) some candidates provided a general proof of
the equation, which is not what was required here.
Q10
(i)
=
\hat{p}_1
36
25
= 0.290, =
\hat{p}_2 = 0.184
124
136

common \hat{p} = ( 36 + 25 ) / ( 124 + 136 ) = 0.23 5

\begin{itemize}
\item As the sample is large we can use a normal approximation.
\[Test statistic =
( \hat{p}_1 − \hat{p}_2 )
\hat{p} (1 − \hat{p} ) /124 + \hat{p} (1 − \hat{p} ) /136\]
= ( 0.290 − 0.184 )
0.235 ( 1 − 0.235 ) /124 + 0.235 ( 1 − 0.235 ) /136
= 0.106
0 .00145 + 0.00132
= 2.014

\item As Z 0.975 = 1.96 is less than the test statistic we reject H 0 : p 1 = p 2 at a 5%
significance level.

\item (ii)
H 0 : Proportions are the same.
Calculate totals
Survey 1 2 3
Y
N
Total 36
88
124 25
111
136 26
115
141
87
314
401

Contingency table
Survey 1 2 3
Y 26.90 29.51 30.59
87
%%-- Page 9Subject CT3  – April 2018 – Examiners’ Report
N
Total
97.10
124.00
106.49
136.00
110.41
141.00
314
401

\item Test statistic
2
e i − a i )
(
=
∑
e i
3.078 + 0.688 + 0.689 + 0.852 + 0.191 + 0.191 =
5.69
=
d.f. = (3 – 1)(2 – 1) = 2


\item The 95\%  point of X 2 2 = 5.991 . As test statistic is lower, do not reject that the
proportion of smokers is equal.

(iii)
\hat{p} 3 26
=
/141 0 .184
(a) =
(b)

\item In the first case the test rejected that the proportions were the same, but
in the second it did not reject that they were, as the proportion in the
third survey is almost identical to that in the second.
\end{itemize}
[Total 16]
The answers in part (i) were generally good. Common errors included not calculating a common proportion p and calculation mistakes when
computing the test statistic. Part (ii) was also well answered in general, although there were some calculation errors. Part (iii) (a) was well
answered by most candidates. Answers in (iii)(b) were mixed, with most candidates making the correct observation. However, many candidates
failed to identify appropriate reasoning.
%%%%%%%%%%%%%%%%%%%%%%%%%%%%%%%%%%%%%%%%%%%%%%%%
\newpage
Q11
(i)
There are 100 observations for age 50 and we can therefore use the normal distribution:


1.69
1.69 
,14.1 + 1.96
[ 13.8452, 1 4.3548 ]
 14.1 − 1.96
 =
100
100



(Alternative solution: using the t 99 distribution and interpolation to obtain the
critical value 1.987, gives confidence interval of (13.84, 14.36).)
(ii)
H 0 : μ 40 =
μ 50
15 − 14.1
Test
statistic: z 10
=
= 4.534
2.25 + 1.69
Page 10

% Subject CT3  – April 2018 – Examiners’ Report
which is approximately standard normal under H 0 due to the large sample
size (100 drivers per age).

The p -value is therefore very close to 0,

and the null hypothesis is rejected. We conclude that the average annual
mileage at age 50 is not the same as the average annual mileage at age 40. 
(iii)
460 2
S xx = 27,500 −
= 1, 050
8
S yy
105.3 2
= 1,398.23 −
= 12.21875
8 
105.3
=
− 112.75
8 
5,942 − 460
S xy =
r =
(iv)

− 112.75
= − 0.99542
1, 050 × 12.21875

For the correlation coefficient in part (iii) the variation amongst drivers of the
same age is ignored.

(Therefore there seems to be a stronger linear relationship between age and
annual mileage than for the case where variations amongst drivers of the same
age is considered (part (iv)).
(v) Only if the variance in each group is zero will the two coefficients coincide.

(vi) We have
S xy =∑ x i y i − ( ∑ x i ) ( ∑ y i ) / n = 100 × 5,942 −
( 100 × 460 )( 100 × 105.3 )
= − 11, 275
S xx =
∑ x i 2
800

2
∑ x i )
(
−
n
2
100 × 460 )
(
100 × 27,500 −
=
= 2, 750, 000 − 2, 645, 000 = 105, 000
800

Therefore,
Page 11Subject CT3  – April 2018 – Examiners’ Report
S xy
11, 275
= − 0.10738
105, 000 
105.3
460
α ˆ = y − β ˆ x =
+ 0.10738 ×
= 19.33 7
8
8 
β ˆ =
S xx
=−
=
y ˆ 19.337 − 0.10738 x

%Part (i) was very well answered. Answers in part (ii) were mostly correct. Some candidates used a t99 distribution with pooled variance, in which case an assumption of equal variances needs to be made and explicitly mentioned. Part (iii) was well answered in general. Answers in parts (iv)-(v) were mixed, with many students failing to state in (iv) that the variation was ignored. Part (vi) was not particularly well answered, with many candidates using the Sxy and Sxx sums from a previous part without further explanation regarding why they would be the same here.
%%%%%%%%%%%%%%%%%%%%%%%%%%%%%%%%%%%%%%%%%%%%%%%%%%%%%%%%%%%%%%%%%%%%%%%%%%%%%%%%
%END OF EXAMINERS’ REPORT
%Page 12

%[Total 21]
\end{document}
