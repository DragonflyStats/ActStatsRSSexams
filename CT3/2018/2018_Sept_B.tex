[Total 6]4
We consider three groups of policyholders: A, B and C. We denote by X A the random
variable for the number of claims that a randomly chosen policyholder in group A
submits during a calendar year. X B and X C denote the corresponding random variables
for policyholders in groups B and C. We assume that X A , X B and X C have Poisson
distributions with parameters \lambda A = 0.2, \lambda B = 0.1 and \lambda C = 0.05 depending on the
group. Each policyholder belongs to exactly one group and group membership does
not change during the lifetime of a policyholder. It is assumed that any individual
policyholder submits claims during any year independently of claims submitted by
other policyholders.
An insurance company has a portfolio of policies with 20% of policyholders
belonging to group A, 20% belonging to group B and the remaining policyholders
belonging to group C.
The insurance company chooses a policyholder at random.
(i)
Determine the probability that this policyholder will submit at least two claims
A.
during a year given that he belongs to group A.

The insurance company chooses another policyholder at random but does not know to
which group he belongs.
(ii)
(iii)

5
Show that the probability this policyholder will submit exactly one claim
0.0794.
during a year is approximately 0.0794.

Calculate the probability that this policyholder belongs to group A given that
year.
he submitted exactly one claim in the previous year.

[Total 7]
In a small empirical study 100 male and 100 female workers in a company are asked
about their body weight and then classified into the following three categories:
not overweight, overweight and obese. The observed numbers of workers in each
category are shown in the following table.
Females
Males
Total
Not overweight Overweight Obese Total
45
33
78 32
41
73 23
26
49 100
100
200
Test the null hypothesis that weight classification is independent of gender, using a 5%
level.
significance level.

S2018–3 
CT3 X18–3
%%%%%%%%%%%%%%%%%%%%%%%%%%%%
Q4
(i)
(ii)
Page 4
PP[XX AA ≥ 2] = 1 − FF AA (1) = 1 − 0.98248 = 0.01752 using tables
PP[XX AA = 1]PP[AA] + PP[XX BB = 1]PP[BB] + PP[XX CC = 1]PP[CC]

Subject CT3  – September 2018 – Examiners’ Report
= ee −0.2 ∗ 0.2 ∗ 0.2 + ee −0.1 ∗ 0.1 ∗ 0.2 + ee −0.05 ∗ 0.05 ∗ 0.6 = 0.07938286 = 0.0794 
(iii)
Let XX 0 be the number of claims submitted last year
PP[XX AA =1]PP[AA]
0.2
PP[AA|XX 0 = 1] =
=ee −0.2 ∗ 0.2 ∗ 0.07938286 = 0.4125479 = 0.4125 
0 =1]
PP[XX
[Total 7]
Generally well answered. In part (i) some candidates mistakenly
divided by the probability of being in group A, effectively conditioning
on being in group A twice.
Q5
Observed
Females
Males
Not overweight Overweight Obese
45
32
23 100
33
41
26 100
78
73
49 200
Expected
Females
Males
Not overweight Overweight Obese
39
36.5
24.5 100
39
36.5
24.5 100
78
73
49 200

Squared Difference
Not overweight Overweight Obese
Females
36
20.25
2.25
Males
36
20.25
2.25

Test statistic:
36
20.25
2.25
2 × � 39 + 36.5 + 24.5 � = 3.14 
2 degrees of freedom, Chi-Squared critical value is 5.991 at a 5% significance level. 
Do not reject null hypothesis that weight and gender are independent 
[Total 5]
Page 5Subject CT3  –September 2018 – Examiners’ Report
Very well answered by most candidates.
