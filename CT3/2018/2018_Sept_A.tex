© Institute and Faculty of Actuaries1
A data set of 20 observations has mean 45 and standard deviation 25.4. The data set is
reviewed and one observation which was incorrectly recorded as 130 is now corrected
to 30.
data.
Determine the mean and standard deviation of the corrected data.
2

A random variable, X, has the probability generating function G X (t) where
G X (t) = 0.4096 + 0.4096t + 0.1528t 2 + 0.0256t 3 + 0.0016t 4
(i)
(t).
Determine the probability P(X = 3) using G X (t).

You are now given that X follows a binomial distribution.
(ii)
X.
Determine the parameter values of the distribution of X.

3

[Total 4]
A sports scientist is building a statistical model to describe the number of attempts
a high jump athlete will have to make until she succeeds in clearing a certain height
for the first time during an indoor sports event. For this model the scientist considers
a geometric distribution with probability of success p. The cumulative distribution
function of the geometric distribution is given as
F X (x) = 1 − (1 − p) x , x = 1, 2, 3, ...
(i)
(a) State the assumptions that the scientist needs to make for considering
this distribution.
(b) Comment on the validity of the assumptions in part (i)(a). 

The athlete has tried n jumps without success.
(ii)

S2018–2
CT3 X18–2
(a) Determine the probability that the athlete will require more than x
additional jumps to succeed in clearing the height.
(b) athlete. 
Comment on what the answer in part (ii)(a) means for the athlete.
%%%%%%%%%%%%%%%%%%%%%%%%%%%%%%%%%5
Solutions
Q1
For original data: ΣΣxx ii = nnxx̅ = 20 × 45 = 900
Corrected data:
xx̅ =
900+30−130
20
Original data:
= 40

ΣΣxx ii2 = (nn − 1)ss 2 + nnxx̅ 2 = 19 × 25.4 2 + 20 × 45 2 = 52758.04 
ΣΣxx ii2 = 52758.04 + 30 2 − 130 2 = 36758.04 
Corrected data:
36758.04−20∗40 2
ss = �
19
= 15.825

[Total 4]
The question was very well answered by most candidates. A common
mistake was using n instead of n-1 for the standard deviation.
Q2
(i)
(ii)
PP(XX = 3) = 0.0256
n=4 (since PP(XX > 4) = 0)
PP(XX = 4) = pp 4 = 0.0016
⟹ pp = 0.2

and PP[XX = 4] = 0.0016 > 0



Therefore X ~ Binomial(4,0.2)
Alternative solutions, e.g. PP(XX = 0) = (1 − pp) 4 = 0.4096 ⟹ pp = 0.2
[The coefficient for tt 2 in the question is incorrect (the correct coefficient value is 0.1536). If
EE[XX] = GG XX′ (1) = nnnn is used then an answer of pp = 0.1996 is obtained.]
[Total 4]
Page 3Subject CT3  –September 2018 – Examiners’ Report
Generally well answered. In part (i), most candidates identified the
polynomial coefficients correctly. However, in part (ii) some
candidates attempted a complicated route relating to finding
derivatives, leading to errors in many cases.
In cases where the incorrect coefficient for tt 2 was used following the
alternative solution with EE[XX] = GG XX′ (1) = nnnn, full credit was given to
candidates providing the slightly different answer under this approach.
%%%%%%%%%%%%%%%%%%%%%%%%%%%%%%%%%%%%%%%%%%%%%%%%%%%%
Q3
(i)
(a) Needs to assume that each time the athlete tries she independently has the same
probability p of passing the height, i.e. that attempts here are iid.

(b) Given that the attempts are at the same event and on the same day,
it is reasonable to assume that conditions are the same (independence) and that
probability of success does not change.
(ii)

(a) If X is the corresponding random variable, we want:
PP(XX > xx + nn | XX > nn)
=
PP(XX>xx+nn)
PP(XX>nn)
=
(1−pp) xx+nn
(1−pp) nn
[0.5]
= (1 − pp) xx = PP(XX > xx)
[1.5]
(b) The lack of success on the first n jumps is irrelevant – under this model the
chances of success are not any better because there have been n attempts already. 
[Total 6]
Answers were mixed, with many candidates in part (i) failing to
describe the assumptions or give justifications for them. All reasonable
comments were given credit here. In part (ii), many candidates failed to
express the conditional probability as required.
