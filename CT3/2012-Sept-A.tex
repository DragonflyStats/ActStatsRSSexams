 88 lines (72 sloc) 2.45 KB
\documentclass[a4paper,12pt]{article}

%%%%%%%%%%%%%%%%%%%%%%%%%%%%%%%%%%%%%%%%%%%%%%%%%%%%%%%%%%%%%%%%%%%%%%%%%%%%%%%%%%%%%%%%%%%%%%%%%%%%%%%%%%%%%%%%%%%%%%%%%%%%%%%%%%%%%%%%%%%%%%%%%%%%%%%%%%%%%%%%%%%%%%%%%%%%%%%%%%%%%%%%%%%%%%%%%%%%%%%%%%%%%%%%%%%%%%%%%%%%%%%%%%%%%%%%%%%%%%%%%%%%%%%%%%%%

\usepackage{eurosym}
\usepackage{vmargin}
\usepackage{amsmath}
\usepackage{graphics}
\usepackage{epsfig}
\usepackage{enumerate}
\usepackage{multicol}
\usepackage{subfigure}
\usepackage{fancyhdr}
\usepackage{listings}
\usepackage{framed}
\usepackage{graphicx}
\usepackage{amsmath}
\usepackage{chngpage}

%\usepackage{bigints}
\usepackage{vmargin}

% left top textwidth textheight headheight

% headsep footheight footskip

\setmargins{2.0cm}{2.5cm}{16 cm}{22cm}{0.5cm}{0cm}{1cm}{1cm}

\renewcommand{\baselinestretch}{1.3}

\setcounter{MaxMatrixCols}{10}

\begin{document}
\begin{enumerate}
© Institute and Faculty of Actuaries1
Calculate the mean, the median and the mode for the data in the following frequency
table.
Observation 0 1 2 3 4
Frequency 20 54 58 28 0

2
The following data are sizes of claims (ordered) for a random sample of 20 recent
claims submitted to an insurance company:
174
487
3
214
490
264
564
298
644
335
686
368
807
381
1092
395
1328
402
1655
442
2272
(i) Calculate the interquartile range for this sample of claim sizes.
(ii) Give a brief interpretation of the interquartile range calculated in part (i). [1]
[Total 4]
Let X be a discrete random variable with the following probability distribution:
X
P(X = x)
0
0.4
1
0.3
2
0.2
3
0.1
Calculate the variance of Y, where Y = 2X + 10.

%%%%%%%%%%%%
Page 2%%%%%%%%%%%%%%%%%%%%%%%%%%%%%%%%%%%%5 – September 2012 – %%%%%%%%%%%%%%%%%%%%%%%%%%%%%%%%
1
mean =
1
254
= 1.5875
( 54 + 2*58 + 3* 28 ) =
160
160
1
Median = value between 80 th and 81 st observation = 2 1
Mode = 2 1
Generally well answered. Note that the median is NOT the 80 th observation, as some
candidates quoted.
2
(i)
⎛ n + 2 ⎞
Q 1 = ⎜
⎟ th observation counting from below = 5.5th observation
⎝ 4 ⎠
=
335 + 368
= 351.5
2
1
⎛ n + 2 ⎞
Q 3 = ⎜
⎟ th observation counting from above = 5.5th observation from above
⎝ 4 ⎠
807 + 686
= 746.5
2 1
IQR = Q 3 − Q 1 = 395 1
=
[With alternative definition:
⎛ n + 1 ⎞
Q 1 = ⎜
⎟ th observation counting from below = 343.25 ,
⎝ 4 ⎠
⎛ n + 1 ⎞
Q 3 = ⎜
⎟ th observation counting from above = 776.75 , IQR = 433.5 .]
⎝ 4 ⎠
(ii)
The length of the interval containing the central half of the claim sizes is 395.
1
The vast majority of candidates calculated the quartiles correctly, although some were
confused with their definition. Part (ii) was not very well answered.
Page 3%%%%%%%%%%%%%%%%%%%%%%%%%%%%%%%%%%%%5 – September 2012 – %%%%%%%%%%%%%%%%%%%%%%%%%%%%%%%%
3
E[X] = 1\times0.3 + 2\times0.2 + 3\times0.1 = 1
⇒ V[X] = (0 –1) 2 \times0.4 + (1 – 1) 2 \times0.3 + (2 – 1) 2 \times0.2 + (3 – 1) 2 \times0.1
= 0.4 + 0.2 + 0.4 = 1
(OR via E[X 2 ] = 2)
V[Y] = 4V[X] = 4
[OR: Directly from the distribution of Y, which is Y = 10, 12, 14, 16 with
probabilities 0.4, 0.3, 0.2, 0.1 respectively.]
No particular problems encountered here. There are a variety of different methods for
obtaining the correct answer.

