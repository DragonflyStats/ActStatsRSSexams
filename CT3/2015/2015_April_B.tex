\documentclass[a4paper,12pt]{article}



%%%%%%%%%%%%%%%%%%%%%%%%%%%%%%%%%%%%%%%%%%%%%%%%%%%%%%%%%%%%%%%%%%%%%%%%%%%%%%%%%%%%%%%%%%%%%%%%%%%%%%%%%%%%%%%%%%%%%%%%%%%%%%%%%%%%%%%%%%%%%%%%%%%%%%%%%%%%%%%%%%%%%%%%%%%%%%%%%%%%%%%%%%%%%%%%%%%%%%%%%%%%%%%%%%%%%%%%%%%%%%%%%%%%%%%%%%%%%%%%%%%%%%%%%%%%
\usepackage{eurosym}
\usepackage{vmargin}
\usepackage{amsmath}
\usepackage{graphics}
\usepackage{epsfig}
\usepackage{enumerate}
\usepackage{multicol}
\usepackage{subfigure}
\usepackage{fancyhdr}
\usepackage{listings}
\usepackage{framed}
\usepackage{graphicx}
\usepackage{amsmath}
\usepackage{chngpage}

%\usepackage{bigints}

\usepackage{vmargin}
% left top textwidth textheight headheight
% headsep footheight footskip
\setmargins{2.0cm}{2.5cm}{16 cm}{22cm}{0.5cm}{0cm}{1cm}{1cm}
\renewcommand{\baselinestretch}{1.3}
\setcounter{MaxMatrixCols}{10}
\begin{document}
\begin{enumerate}

%%%%%%%%%%%%%%%%%%%%%%%%%%%%%%%%%%%%%%%%%%%%%%%%%%%%%%%%%%%%%%%%%%%%%%%%%%%%%%%%%

4 An insurance company experiences claims from 290 insurance policies in a year on a portfolio of 900 policies. Only one claim can be made on a policy in a year. The company assumes that all policies are independent of each other.
Determine a 90% confidence interval for the proportion of policies on which a claim is made in a year. 

5 A random sample of 30 observations is drawn from a normal distribution with unknown variance.
\begin{enumerate}[(i)]
\item (i) Write down an expression for the distribution of S, the population standard deviation. 
The sample standard deviation, s, is 7.5.
\item (ii) Calculate a 95% confidence interval for the population standard deviation. 
\end{enumerate}

%%%%%%%%%%%%%%% CT3 A2015–3 PLEASE TURN OVER

6 Let $X_1, X_2, \ldots, X_6$ be a random sample from a population following a Gamma(2,1)
distribution. Consider the following two estimators of the mean of this distribution:
  ˆ1  X and 2 1 2 3 4 5 6 ˆ 9 ( 1
                                  3
                                  ( )
                                  3 0
  )
0
  X  X  X  X  X  X
where X is the mean of the sample.
\begin{enumerate}[(i)]
\item (i) Determine the sampling distribution of X using moment generating functions. [5]
\item (ii) Derive the bias of each estimator 1 ˆ and 2 ˆ . [2]
\item (iii) Derive the mean square error of each estimator 1 ˆ and 2 ˆ . [5]
\item (iv) Compare the efficiency of the two estimators 1 ˆ and 2 ˆ . [1]
\end{enumerate}
%%%%%%%%%%%%%%%%%%%%%%%%%%%%%%%%%%%%%%%%%%%%
  4 No. claims ~Bin(900,p)
pˆ  290 / 900  0.322
􀝌̂~􀜰􁈺􀝌, 􀝌􁈺1 􀵆 􀝌􁈻/􀝊􁈻approximately
C.I.
      0.95
1 0.322 1 0.322
* 0.322 1.644
ˆ ˆ
ˆ 9* 0.296,0.348
900
p p
p Z
n
 
    
Very well answered.
5 (i)
  2
2
2 1
1
~ n
n S




(ii) 2 2
0.025;29  45.72, 0.975;29 16.05
variance C.I. =
       
2 2 2 2
2 2
0.025;29 0.975;29
1 1 7.5 7.5 , 29* , 29* 35.679,101.64
45.72 16.05
 n  S n  S   
             
95% C.I. for S is  35.679, 101.64 =(5.97,10.08)
Generally well answered.
Subject CT3 (Probability and Mathematical Statistics) – April 2015 – Examiners’ Report
%%%%%%%%%%%%%%%%%%%%%%%%%%%%%%%%%%%%%%%%%%%%%%%%%%%%%%%%%%%%%%%%%%%%%%%%%%%%%%%%%%%%%%%%%%%%%%%%%%%%%%5
6 (i) Using MGFs,
X   X1 X2 X6
M t M t M t M t
n n n
            
     
 since Xi are independent
1
6
X 6
M t
        
   
since Xi are identically distributed
2 6 12
1 1
6 6
t t             
    
i.e. a Gamma(12, 6) distribution.
(ii) Eˆ1   E X   E Xi   2. So bias = 0.
 2       
9 3 1 3 2.
30 30
ˆ E   E Xi  E Xi  E Xi  So, again, bias = 0.
(iii) Since bias = 0 for both, we have:
       
1
MSE 2 0.333
6
ˆ
6
V Xi
 V X   
       
2 2
2 2
MSE 9 3 1 3
30 30
ˆ ˆ V V Xi V Xi           
   
(independence)
81 3 3
2
900
 
 = 0.547
(iv) 1 ˆ has smaller MSE, and therefore is more efficient than 2 ˆ . nswers here were generally good. In part (ii) a number of candidates were confused with the estimators and the estimated parameter, but in part (iv) most candidates were familiar
with the concept of efficiency.
\end{document}
