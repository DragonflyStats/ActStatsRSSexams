\documentclass[a4paper,12pt]{article}

%%%%%%%%%%%%%%%%%%%%%%%%%%%%%%%%%%%%%%%%%%%%%%%%%%%%%%%%%%%%%%%%%%%%%%%%%%%%%%%%%%%%%%%%%%%%%%%%%%%%%%%%%%%%%%%%%%%%%%%%%%%%%%%%%%%%%%%%%%%%%%%%%%%%%%%%%%%%%%%%%%%%%%%%%%%%%%%%%%%%%%%%%%%%%%%%%%%%%%%%%%%%%%%%%%%%%%%%%%%%%%%%%%%%%%%%%%%%%%%%%%%%%%%%%%%%

\usepackage{eurosym}
\usepackage{vmargin}
\usepackage{amsmath}
\usepackage{graphics}
\usepackage{epsfig}
\usepackage{enumerate}
\usepackage{multicol}
\usepackage{subfigure}
\usepackage{fancyhdr}
\usepackage{listings}
\usepackage{framed}
\usepackage{graphicx}
\usepackage{amsmath}
\usepackage{chngpage}

%\usepackage{bigints}
\usepackage{vmargin}

% left top textwidth textheight headheight

% headsep footheight footskip

\setmargins{2.0cm}{2.5cm}{16 cm}{22cm}{0.5cm}{0cm}{1cm}{1cm}

\renewcommand{\baselinestretch}{1.3}

\setcounter{MaxMatrixCols}{10}

\begin{document}



%%%%%%%%%%%%%%%%%%%%%%%%%%%%%%%%%%%%%%%%%%%%%%%%%%%%%%%%%%%%%%%%%%%%%%%%%%%%%%%%%
9 An insurance company has calculated premiums assuming that the average claim size per claim for a certain class of insurance policies does not exceed \$20,000 per annum.
An actuary analyses 25 such claims that have been randomly selected. She finds that the average claim size in the sample is \$21,000 and the sample standard deviation is
\$2,500. Assume that the size of a single claim is normally distributed with unknown expectation  and variance 2.
\begin{enumerate}
\item (i) Calculate a 95\%  confidence interval for  based on the sample of 25 claims.

\item (ii) Perform a test for the null hypothesis that the expected claim size is not greater
than \$20,000 at a 5\% significance level. 
\item (iii) Discuss whether your answers to parts (i) and (ii) are consistent. 
\item (iv) Calculate the largest expected claim size, 0, for which the hypothesis  ≤0
can be rejected at a 5\% significance level based on the sample of 25 claims. 
The insurer is also concerned about the number of claims made each year. It is found that the average number of claims per policy was 0.5 during the year 2011. When the
analysis was repeated in 2012 it was found that the average number of claims per policy had increased to 0.6. These averages were calculated on the basis of random
samples of 100 policies in each of the two years. Assume that the number of claims per policy per year has a Poisson distribution with unknown expectation  and is
independent from the number of claims in any other year or for any other policy. 
\item (v) Perform a test at 5\% significance level for the null hypothesis that = 0.6
during the year 2011. 
\item (vi) Perform a test to decide whether the average number of claims has increased
from 2011 to 2012. 
\end{enumerate}
%%%%%%%%%%%%%%%%%%%%%%%%%%%%%%5
\newpage
9 
begin{itemize}
\item In thousands:
  (i) 0.025,24 0.025,24
21 2.5 , 21 2.5 21 2.064 1 , 21 2.064 1
5 5 2 2
 t  t          
19.968, 22.032
\item (ii) H0 :   20 v H1: \alpha > 20
(or, H0: \alpha = 20v H1: \alpha > 20)
Test statistic: 0
0.05,24
21 20 2 1.711
/ 2.5 0.2
t x t
s n
  
    

0.05,24
21 20 2 1.711
2.5 0.2
t

  

We reject the null hypothesis.
\item (iii) The confidence interval in part (i) corresponds to a two-sided test. We found
in part (i) that 20 is contained in the confidence interval, and we can therefore
not reject the null hypothesis H0 :   20 at a 5\% significance level.
However, the one-sided test rejects H0 :   20 since only positive differences
X 0 are considered. Answers are consistent.
\item (iv) 0
0 0
21 1.711, 21 0.8555, 20.1445
2.5 0.2
 
    

\item (v) Test H0: \lambda = 0.6 v H1: \lambda \neq0.6
Test statistic (based on normal approximation to Poisson) is:
  0  
0
0.5 0.6 0.1 1.29 1.96,1 .96
/ 0.6 /100 0.077
z x
n
  
    


0.5 0.6 0.1 1.29  1.96,1 .96
0.6 /100 0.077
 
  
(or, with continuity correction
 0.5 0.5 0.6
 100 1.226
 0.6 /100
 z
  
   )
 – April 2015 – %%%%%%%%%%%%%%%%%%%%%%%%%%%%%%%%%%
Page 8
The null hypothesis H0 :   0.6 cannot be rejected for the year 2011.
\item (vi) Test H0 : 2012  2011 v H1 : 2012  2011
(or, H0 : 2012  2011 v H1 : 2012  2011 )
Overall sample mean ˆ = 0.55
Test statistics now is: 2012 2011
1 2
0.6 0.5 0.1 0.9535 1.64
ˆ ˆ 1.1/100 0.104
z
n n
   
    
  
\item The null hypothesis H0 : 2012  2011 cannot be rejected at the 5\% level.
Therefore, we do not have empirical evidence to suggest that the alternative
2012  2011 is true.
Generally well answered, although in part (ii) many candidates did not use the correct
hypotheses for the test.
\end{itemize}
\end{document}
