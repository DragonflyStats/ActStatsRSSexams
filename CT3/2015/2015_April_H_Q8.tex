\documentclass[a4paper,12pt]{article}

%%%%%%%%%%%%%%%%%%%%%%%%%%%%%%%%%%%%%%%%%%%%%%%%%%%%%%%%%%%%%%%%%%%%%%%%%%%%%%%%%%%%%%%%%%%%%%%%%%%%%%%%%%%%%%%%%%%%%%%%%%%%%%%%%%%%%%%%%%%%%%%%%%%%%%%%%%%%%%%%%%%%%%%%%%%%%%%%%%%%%%%%%%%%%%%%%%%%%%%%%%%%%%%%%%%%%%%%%%%%%%%%%%%%%%%%%%%%%%%%%%%%%%%%%%%%

\usepackage{eurosym}
\usepackage{vmargin}
\usepackage{amsmath}
\usepackage{graphics}
\usepackage{epsfig}
\usepackage{enumerate}
\usepackage{multicol}
\usepackage{subfigure}
\usepackage{fancyhdr}
\usepackage{listings}
\usepackage{framed}
\usepackage{graphicx}
\usepackage{amsmath}
\usepackage{chngpage}

%\usepackage{bigints}
\usepackage{vmargin}

% left top textwidth textheight headheight

% headsep footheight footskip

\setmargins{2.0cm}{2.5cm}{16 cm}{22cm}{0.5cm}{0cm}{1cm}{1cm}

\renewcommand{\baselinestretch}{1.3}

\setcounter{MaxMatrixCols}{10}

\begin{document}
8
The random variables X and Y have a joint probability distribution with density
function:

\[
f xy ( x , y ) = \begin{cases}
3x & x,y >0 1\\
0 & otherwise\\
\end{cases}
\]
\begin{enumerate}[(a)]
\item Determine the marginal densities of X and Y.
\item State, with reasons, whether X and Y are independent.
\item Determine $E[X]$ and $E[Y]$.
\end{enumerate}

%%%%%%%%%%%%%%%%%%%%%%%%%5
\newpage
\item 8 (i)     2
  0
  0
  3 3 3
  x
  y x
  X y f x xdy xy x 
       for 0 <x< 1
      1 1
  3 3 2 3 1 2
  2 2
  x
  Y
  y x y
  f y xdx x y
  
  
           for $0 <y< 1$
 \item  (ii) Not independent because fX x fY  y  fXY x, y
  (iii)    
  1 1
  4
  0 0
  3 0.75
  X 4 E X  xf x dx   x     
     
  1 2 4 1
  0 0
  3 3
  Y 2 2 4 8
  E Y yf y dy y y
    
           
  
  % Subject CT3 (Probability and Mathematical Statistics) – April 2015 – Examiners’ Report
  % Page 7
% \item In part (i) many candidates could not identify correctly the range of integration, but part (ii) was well answered. In part (iii) again the wrong range of the variables was often used.
\end{document}
