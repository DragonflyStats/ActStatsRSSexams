\documentclass[a4paper,12pt]{article}

%%%%%%%%%%%%%%%%%%%%%%%%%%%%%%%%%%%%%%%%%%%%%%%%%%%%%%%%%%%%%%%%%%%%%%%%%%%%%%%%%%%%%%%%%%%%%%%%%%%%%%%%%%%%%%%%%%%%%%%%%%%%%%%%%%%%%%%%%%%%%%%%%%%%%%%%%%%%%%%%%%%%%%%%%%%%%%%%%%%%%%%%%%%%%%%%%%%%%%%%%%%%%%%%%%%%%%%%%%%%%%%%%%%%%%%%%%%%%%%%%%%%%%%%%%%%
    
\usepackage{eurosym}
\usepackage{vmargin}
\usepackage{amsmath}
\usepackage{graphics}
\usepackage{epsfig}
\usepackage{enumerate}
\usepackage{multicol}
\usepackage{subfigure}
\usepackage{fancyhdr}
\usepackage{listings}
\usepackage{framed}
\usepackage{graphicx}
\usepackage{amsmath}
\usepackage{chngpage}

%\usepackage{bigints}

\usepackage{vmargin}

% left top textwidth textheight headheight
% headsep footheight footskip

\setmargins{2.0cm}{2.5cm}{16 cm}{22cm}{0.5cm}{0cm}{1cm}{1cm}
\renewcommand{\baselinestretch}{1.3}
\setcounter{MaxMatrixCols}{10}
\begin{document}
\large 
%% Question 3 
\noindent Assume that in a large portfolio of insurance contracts the claim size is a normally distributed random variable with expected value 1000. Also assume that the number of claims is a random variable following a Poisson distribution with parameter
 = 400.

\begin{enumerate}[(a)]
\item Calculate the expected value of the total claim amount from contracts in this portfolio. 
\item Calculate a lower limit for the standard deviation of the total amount of claims from contracts in this portfolio. 
\end{enumerate}
\end{enumerate}
%%%%%%%%%%%%%%%%%%%%%%%%%%%%%%%%%%%%%%%%%%%%%%%%%%%%%%%%%%%%%%%%%%%%%%%%%%%%%%%%%%%%%%%%%%%
\newpage

\newpage

3 Let X be the size of an individual claim, and N be the number of claims.
\begin{itemize}
    \item (i) Expected total amount is EX EN 1,000  400  400,000

    \item (ii) Var total amount  ENV X V NEX 2
\item A lower bound for the variance is then obtained by assuming V X   0 , that
is,
STDtotal amount  V N EX 2  201000 = 20,000
\end{itemize}

%The first part was answered very well. In part (ii) many candidates failed to recognise that the answer relies on the variance being equal to zero.
\end{document}
