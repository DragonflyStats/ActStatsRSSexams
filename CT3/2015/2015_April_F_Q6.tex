\documentclass[a4paper,12pt]{article}



%%%%%%%%%%%%%%%%%%%%%%%%%%%%%%%%%%%%%%%%%%%%%%%%%%%%%%%%%%%%%%%%%%%%%%%%%%%%%%%%%%%%%%%%%%%%%%%%%%%%%%%%%%%%%%%%%%%%%%%%%%%%%%%%%%%%%%%%%%%%%%%%%%%%%%%%%%%%%%%%%%%%%%%%%%%%%%%%%%%%%%%%%%%%%%%%%%%%%%%%%%%%%%%%%%%%%%%%%%%%%%%%%%%%%%%%%%%%%%%%%%%%%%%%%%%%
\usepackage{eurosym}
\usepackage{vmargin}
\usepackage{amsmath}
\usepackage{graphics}
\usepackage{epsfig}
\usepackage{enumerate}
\usepackage{multicol}
\usepackage{subfigure}
\usepackage{fancyhdr}
\usepackage{listings}
\usepackage{framed}
\usepackage{graphicx}
\usepackage{amsmath}
\usepackage{chngpage}

%\usepackage{bigints}

\usepackage{vmargin}
% left top textwidth textheight headheight
% headsep footheight footskip
\setmargins{2.0cm}{2.5cm}{16 cm}{22cm}{0.5cm}{0cm}{1cm}{1cm}
\renewcommand{\baselinestretch}{1.3}
\setcounter{MaxMatrixCols}{10}
\begin{document}
%%%%%%%%%%%%%%% CT3 A2015–3 PLEASE TURN OVER

6 Let $X_1, X_2, \ldots, X_6$ be a random sample from a population following a Gamma(2,1)
distribution. Consider the following two estimators of the mean of this distribution:
$\hat{\theta}_1 = \bar{X}$

\[ \hat{\theta}_1 = \frac{9}{30} \left( X_1 + X_2 + X_3 \right) \;+\; \frac{1}{30} \left( X_4 + X_5 + X_6 \right) 


where $\bar{X}$ is the mean of the sample.
\begin{enumerate}[(i)]
\item (i) Determine the sampling distribution of X using moment generating functions. 
\item (ii) Derive the bias of each estimator $\hat{\theta}_1$  and $\hat{\theta}_2$ . 
\item (iii) Derive the mean square error of each estimator $\hat{\theta}_1$ and $\hat{\theta}_2$ . 
\item (iv) Compare the efficiency of the two estimators $\hat{\theta}_1$ and $\hat{\theta}_2$. 
\end{enumerate}
%%%%%%%%%%%%%%%%%%%%%%%%%%%%%%%%%%%%%%%%5
6 (i) Using MGFs,
X   X1 X2 X6
M t M t M t M t
n n n
            
     
 since Xi are independent
1
6
X 6
M t
        
   
since Xi are identically distributed
2 6 12
1 1
6 6
t t             
    
i.e. a Gamma(12, 6) distribution.

(ii) \[ E(\hat{\theta}_{1} ) \;=\; E( \bar{X} ) \;=\; E( Xi ) \;=\; 2. \]So bias = 0.


%%%%%%%%%%%%%%%%%%%%%%%%%%%%%%%%%%%%%

\[ E(\hat{\theta}_{2} ) \;=\; \frac{9}{30} \times 3 E(X_i) \;+\;\frac{1}{30} \times 3 E(X_i) \;=\; E(X_i) \;=\; 2 \]

So, again, bias = 0.

%%%%%%%%%%%%%%%%%%%%%%%%%%%%%%%%%%%%%%%%%%%%%%%%%%

(iii) Since bias = 0 for both, we have:
\[
MSE (\hat{\theta}_{1} )  \;=\;\operatorname{Var}( \bar{X} ) \;=\; \frac{\operatorname{Var}(X_i)}{6} \;=\; \frac{2}{6} \;=\; 0.3333=
\]

\[
MSE (\hat{\theta}_{2} )  \;=\;\operatorname{Var}( \hat{\theta}_{2}  ) 
\;=\; \left( \frac{9}{30} \right)^2 \times 3 \operatorname{Var}(X_i) \;+\; \left(\frac{1}{30}\right)^2 \times 3 \operatorname{Var}(X_i)
\]

(independence)

81 3 3
2
900
 
 = 0.547
(iv) 1 ˆ\theta has smaller MSE, and therefore is more efficient than 2 \hat{\theta} . ns
Answers here were generally good. In part (ii) a number of candidates were confused with the estimators and the estimated parameter, but in part (iv) most candidates were familiar
with the concept of efficiency.
%%%%%%%%%%%%%%%%%%%%%%%%%%%%%%%%%%
\end{document}
