\documentclass[a4paper,12pt]{article}

%%%%%%%%%%%%%%%%%%%%%%%%%%%%%%%%%%%%%%%%%%%%%%%%%%%%%%%%%%%%%%%%%%%%%%%%%%%%%%%%%%%%%%%%%%%%%%%%%%%%%%%%%%%%%%%%%%%%%%%%%%%%%%%%%%%%%%%%%%%%%%%%%%%%%%%%%%%%%%%%%%%%%%%%%%%%%%%%%%%%%%%%%%%%%%%%%%%%%%%%%%%%%%%%%%%%%%%%%%%%%%%%%%%%%%%%%%%%%%%%%%%%%%%%%%%%
 
  
 \usepackage{eurosym}
\usepackage{vmargin}
\usepackage{amsmath}
\usepackage{graphics}
\usepackage{epsfig}
\usepackage{enumerate}
\usepackage{multicol}
\usepackage{subfigure}
\usepackage{fancyhdr}
\usepackage{listings}
\usepackage{framed}
\usepackage{graphicx}
\usepackage{amsmath}

\usepackage{chngpage}

%\usepackage{bigints}

\usepackage{vmargin}
% left top textwidth textheight headheight
% headsep footheight footskip
\setmargins{2.0cm}{2.5cm}{16 cm}{22cm}{0.5cm}{0cm}{1cm}{1cm}
\renewcommand{\baselinestretch}{1.3}
\setcounter{MaxMatrixCols}{10}
\begin{document}
\begin{enumerate}

%%%%%%%%%%%%%%%%%%%%%%%%%%%%%%%%%%%%%%%%%%%%%%%%%%%%%%%%%%%%%%%%%%%%%%%%%%%%%%%%%
11 Consider the set of paired data $(x1, y1), (x2 , y2 ),,(xn , yn )$ to which we fit the linear regression model:
  2
Yi ~ N(  xi , ) ,
where the Yi are independent random variables, and ,  and 2 are unknown
parameters.
\item \item (i) (a) Show that the least squares estimator ˆ is an unbiased estimator for
parameter .
(b) Show that the variance of ˆ is given by:
  2
(ˆ)
xx
V
S

 \;=\;
where Sxx \;=\;(xi  x )2 .
You are given that the covariance of the least squares estimators ˆ  and ˆ is
given by
2
cov( ˆ , ˆ)
xx
x
S

%%%%%%%%%%%%%%%%%%%%%%%%%%%%%%%%%%%%%%%%%%%%%%%%%%%%%%%%%%%%%%%%%%%%%%%%%%%%%%%%%%%%%%%%5
11 \item \item (i) (a)
ˆ ( )  ( )
i i i i
xx xx
x x y y x x y
S S
  
 \;=\; \;=\;  
    ( ) ( ) ( )
ˆ i i i i
i
xx xx xx
x x x x x x x
E EY
S S S
  
 \;=\; \;=\;   \;=\;    
    ( ) ( ) ( )
ˆ i i i i
i
xx xx xx
x x x x x x x
E EY
S S S
  
 \;=\; \;=\;   \;=\;    
(b)    
   
2 2
2 2
( ) ( )
ˆ i i i
xx xx
V x xY x x
V
S S
  
 \;=\; \;=\;   using independence of Yi
and   2
ˆ
xx
V
S

 \;=\;
(c) With zi \;=\; xi  x we have z \;=\; 0 and   2
cov ˆ , ˆ 0
zz
z
S

  \;=\; \;=\;
i.e. the two estimators are uncorrelated which implies a better model for estimation.
%% – April 2015 – %%%%%%%%%%%%%%%%%%%%%%%%%%%%%%%%
%%Page 10
\itemThere seems to be a decreasing relationship. However it is not clear if it is linear.
\item (iii) (a) H0: \beta = 0 v. H1: $\beta \neq 0$
t \;=\; 0.2455 / 0.1015 \;=\; 2.419 t \;=\; 0.2455 / 0.1015 \;=\; 2.419
with t8(2.5\%) = 2.306 and t8(0.5\%) = 3.355
At the 5\% level we would reject the null in favour of the hypothesis that there is linear relationship between crawling age and average
temperature (but we would not reject H0 at the 1\% level – or any
             level < 4.2\%)
Alternative solutions:
  Test H0: ρ = 0 v H1: ρ \neq 0
Use R2 = r2, giving r = – 0.65.
` Under H0:
  2
2 0.65 8 2.419
1 1 0.4226
t r n
r
 
\;=\; \;=\; \;=\;
 
Then same as above.
Or, using Fisher’s transformation:
  1 log 1 0.7753
2 1
w r
r
   \;=\;   \;=\;    
Under H0 :
  ~ 0, 1
3
W N
n
 
    
or 7W ~ N 0,1
7w \;=\; 2.051, so conclusion is similar as before.
(b) The coefficient of determination R2 is rather low, so the fit of the
model does not seem good.
(iv) (a) Under the transformed data we have
yˆi \;=\; ˆ  ˆ zi \;=\; ˆ  ˆ xi  x  \;=\; ˆ ˆx  ˆxi
which is the same as yˆi \;=\;ˆ  ˆ xi with ˆ \;=\; ˆ  ˆx  ˆ \;=\; ˆ  ˆx and
ˆ \;=\; ˆ .
%% – April 2015 – %%%%%%%%%%%%%%%%%%%%%%%%%%%%%%%%
%%Page 11
Alternative solution:
  With zi \;=\; xi  x we have
Szz \;=\;(zi  z )2 \;=\;zi2 \;=\;xi  x 2 \;=\; Sxx
Szy \;=\;(zi  z )( yi  y) \;=\;(xi  x)( yi  y) \;=\; Sxy
ˆ zy xy ˆ
zz xx
S S
S S
\;=\; \;=\; \;=\;
ˆ \;=\; y  ˆz \;=\; y \;=\; ˆ  ˆ x
(b) Fitted model:
  yˆi \;=\; 34.5501 0.245511.2  0.2455 zi \;=\; 31.8005  0.2455 zi.
yˆi \;=\; 34.5501 0.245511.2  0.2455 xi \;=\; 31.8005 0.2455 xi .
%%-- Part \item (i) was very poorly answered. The answers in this part can be derived using direct application of known results on statistics and probability that are explicitly given in the Core Reading, combined with basic algebraic skills. Parts \item \item (ii) and \item (iii) were well answered, while the performance in part (iv) was mixed.
%%%%%%%%%%%%%%%%%%%%%%%%%%%%%%%%

  \;=\;  .
(c) Explain why a data transformation of the form zi \;=\; xi  x would result
in a better model.
[7]
A study investigated whether there is an association between babies’ first crawling
age and the average temperature during the month they first try to crawl (about 6 months after birth). It is thought that in colder months heavier clothing may restrict a
baby’s movement more than in warmer months and so the age at which the baby first crawls would be expected to be greater. A random sample of babies was taken and the results of the study are given in following table:
\begin{verbatim}
  Baby 1 2 3 4 5 6 7 8 9 10
Average temperature , C (x) 16 2 10 20 12 16 14 9 3 10
Crawling Age, weeks (y) 32 34 34 30 31 29 29 31 33 35
\end{verbatim}
%%--- CT3 A2015–7
A scatterplot of the data is given below:
  \item \item (ii) Comment on the relationship between crawling age and average temperature based on these observations. 
The linear regression model 2
Yi ~ N(  xi , ) was fitted to these data, and some of
the results are given below:
  Estimate Standard error
\alpha 34.5501 1.2617
\beta −0.2455 0.1015
\sigma1.733
Also, R2 = 0.4226.
Based on these results:
  \item (iii) (a) Perform a statistical test to investigate the hypothesis that there is no linear relationship between crawling age and average temperature.
(b) Comment on the fit of the model. 
Consider the data transformation mentioned in part (i)(c), giving the model
2
Yi ~ N(  zi , ) .
(iv) (a) Show that the estimators of the parameters in this model are given by:
  ˆ \;=\; ˆ ˆ x, ˆ \;=\; ˆ
where ˆ  and ˆ are the parameter estimators of the model using the
original data.
(b) Write down the fitted model under this transformation. 
[Total 18]

5 10 15 20
29 31 33 35
temperature
age
\end{document}
