\documentclass[a4paper,12pt]{article}

%%%%%%%%%%%%%%%%%%%%%%%%%%%%%%%%%%%%%%%%%%%%%%%%%%%%%%%%%%%%%%%%%%%%%%%%%%%%%%%%%%%%%%%%%%%%%%%%%%%%%%%%%%%%%%%%%%%%%%%%%%%%%%%%%%%%%%%%%%%%%%%%%%%%%%%%%%%%%%%%%%%%%%%%%%%%%%%%%%%%%%%%%%%%%%%%%%%%%%%%%%%%%%%%%%%%%%%%%%%%%%%%%%%%%%%%%%%%%%%%%%%%%%%%%%%%
    
\usepackage{eurosym}
\usepackage{vmargin}
\usepackage{amsmath}
\usepackage{graphics}
\usepackage{epsfig}
\usepackage{enumerate}
\usepackage{multicol}
\usepackage{subfigure}
\usepackage{fancyhdr}
\usepackage{listings}
\usepackage{framed}
\usepackage{graphicx}
\usepackage{amsmath}
\usepackage{chngpage}

%\usepackage{bigints}

\usepackage{vmargin}

% left top textwidth textheight headheight
% headsep footheight footskip

\setmargins{2.0cm}{2.5cm}{16 cm}{22cm}{0.5cm}{0cm}{1cm}{1cm}
\renewcommand{\baselinestretch}{1.3}
\setcounter{MaxMatrixCols}{10}
\begin{document}
\begin{enumerate}

%%%%%%%%%%%%%%%%%%%%%%%%%%%%%%%%%%%%%%%%%%%%%%%%%%%%%%%%%%%%%%
%%%%%%%%%%%%%%%%%%%%%%%%%%%%%%%%%%%%%%%%%%%%%%%%%%%%%%%%%%%%%%%%%%%%%%%%%%%%%%%%%%%%%%%%%%%%5
\item 
%% Question 2
A random sample of size n consists of k distinct observations $x_1, x_2, \ldots, x_k$ which have
been observed with frequencies $f_1, f_2, \ldots, f_k$ where k 1
n i fi . Consider the
deviations of x from a constant A, giving the observations di= xi A for i = 1, …, k.
Show that the sample variance of the xi values is given by:
  2 2 2
( 1 ( 1 ) / ) / ( 1) k k
sx  i fidi  i fidi n n  .
\end{enumerate}
%%%%%%%%%%%%%%%%%%%%%%%%%%%%%%%%%%%%%%%%%%%%%%%%%%%%%%%%%%%%%%%%%%%%%%%%%%%%%%%%%%%%%%%%%%%
\newpage

Generally well answered, although some problems were encountered with the variance.
2 2  2  
1
/ 1
k
x i i
i
s f x x n

  
and xi  x  xi  x  A A  (xi  A)  x  A  di  d
since
( )
i i i i f d f x A
d xA
n n

     
This gives      
2
2 2 2
1 1 1
/ 1 / / 1
k k k
x i i ii ii
i i i
s f d d n fd fd n n
  
                    
  
% Overall performance was poor. Many answers completely ignored the involved frequencies.
% This is an example of a question that is not examined frequently and candidates found challenging.
%%%%%%%%%%%%%%%%%%%%%%

\end{document}
