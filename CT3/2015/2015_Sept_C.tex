\documentclass[a4paper,12pt]{article}

%%%%%%%%%%%%%%%%%%%%%%%%%%%%%%%%%%%%%%%%%%%%%%%%%%%%%%%%%%%%%%%%%%%%%%%%%%%%%%%%%%%%%%%%%%%%%%%%%%%%%%%%%%%%%%%%%%%%%%%%%%%%%%%%%%%%%%%%%%%%%%%%%%%%%%%%%%%%%%%%%%%%%%%%%%%%%%%%%%%%%%%%%%%%%%%%%%%%%%%%%%%%%%%%%%%%%%%%%%%%%%%%%%%%%%%%%%%%%%%%%%%%%%%%%%%%
  \usepackage{eurosym}
\usepackage{vmargin}
\usepackage{amsmath}
\usepackage{graphics}
\usepackage{epsfig}
\usepackage{enumerate}
\usepackage{multicol}
\usepackage{subfigure}
\usepackage{fancyhdr}
\usepackage{listings}
\usepackage{framed}
\usepackage{graphicx}
\usepackage{amsmath}
\usepackage{chngpage}
%\usepackage{bigints}
\usepackage{vmargin}

% left top textwidth textheight headheight

% headsep footheight footskip

\setmargins{2.0cm}{2.5cm}{16 cm}{22cm}{0.5cm}{0cm}{1cm}{1cm}

\renewcommand{\baselinestretch}{1.3}

\setcounter{MaxMatrixCols}{10}

\begin{document}

\begin{enumerate}
CT3 S2015–4
7 X and Y are discrete random variables with joint distribution given below.
Y = −1 Y = 0 Y = 1
X = 1 0 1/4 0
X = 0 1/4 1/4 1/4
\begin{enumerate}[(i)]
\item (i) Determine the conditional expectation E[Y|X = 1]. [1]
\item (ii) Determine the conditional expectation E[X|Y = y] for each value of y. [3]
\item(iii) Determine the expected value of X based on your conditional expectation results from part (ii).
\end{enumerate}
%%%%%%%%%%%%%%%%%%%%%%%%%%%%%%%%%%%%%%5
8 Consider three groups of policyholders: A, B and C. Denote by XA the random variable for the number of claims that a randomly chosen policyholder in group A submits during any particular calendar year. XB and XC denote the corresponding random variables for policyholders in groups B and C.
Assume that XA, XB and XC have Poisson distributions with parameters A = 0.2,
B = 0.1 and C = 0.05, depending on the group.
Each policyholder belongs to exactly one group and group membership does not change during the lifetime of a policyholder.
Assume that:
   any individual policyholder submits a claim during any year independently of claims submitted by other policyholders.
 the number of claims a policyholder submits during a year depends on the group the policyholder belongs to, but given which group the policyholder is a member of, the number of claims submitted during a year is independent of the number of claims the policyholder submitted in previous years.
An insurance company has a portfolio of policies with 20\% of policyholders belonging to group A, 20\% belonging to group B and the remaining policyholders belonging to group C.

The insurance company randomly chooses one of its policyholders.
\begin{enumerate}[(i)]
\item (i) Calculate the probability that this policyholder will submit at least two claims in a particular year given that he belongs to group A. [2]
Now assume that the insurance company does not know to which group the randomly selected policyholder belongs.
\item (ii) Show that the probability that the randomly selected policyholder submits exactly one claim in any particular year is approximately 0.0794. [3]
%%------------------CT3 S2015–5 PLEASE TURN OVER
\item (iii) Determine the probability that the randomly selected policyholder belongs to
group A given that the policyholder submitted exactly one claim in the
previous year. 
\item (iv) Determine the probability that the randomly chosen policyholder will submit
one claim during the current year given that he submitted one claim in the
previous year. [5]
\end{enumerate}

%%%Q7 (i)  
1 0 0 1 1 0
| 1 4 0 1
4
E Y X
     
  
(ii)  
1 0 0 1
| 1 4 0 1
4
E X Y
  
    ,  
1 1 0 1 | 0 4 4 1 1 2
2
E X Y
  
  
 
1 0 0 1
| 1 4 0 1
4
E X Y
  
  
(iii) E X   E EX |Y  EX |Y  1 PY  1
EX |Y  0 PY  0
EX |Y 1 PY 1
and   0 1 1 1 0 1 1
4 2 2 4 4
E X       
Some reasonable answers, but generally a mixed performance. Note that the
question asks candidates to “determine” the various expectations, so working
needs to be shown to gain full marks.
Subject CT3 (Probability and Mathematical Statistics Core Technical) – September 2015 – Examiners’ Report
Page 7
%%%%%%%%%%%%%%%%%%%%%%%%%%%%%%
\newpage
Q8 (i) PXA  2 1 FA 1 1 0.98248  0.01752 using tables
(ii) PX A 1PA PXB 1PB PXC 1PC
 e0.2 *0.2*0.2  e0.1 *0.1*0.2  e0.05 *0.05*0.6  0.07938286 .
(iii) Let X 0 be the number of claims submitted last year
0    
0
1
| 1
1
P X A P A
P A X
P X
           
= 0.2 *0.2* 0.2 0.4125
0.0794
e 
(iv) Let X 0 be the number of claims submitted last year, and X1 be the number of
claims that will be submitted in the current year.
P[X1 1 |X 0 1]  P[{X1 1 A | X 0 1} P{X1 1 B | X 0 1}]
 P[{X1 1 C | X 0 1}]
The first probability is given as
P[{X1 1} A|X 0 1]  P[X1 1| AX 0 1P A| X 0 1]
 P[X1 1| AP A| X 0 1]
where the last equality follows from conditional independence of X1 from
X 0 given group membership. Then
1   0      
0
1
[ 1| | 1] 1
1
A
A
P X P A
P X A P A X P X
P X

   
    
 (e0.2 *0.2)2 *0.2 / 0.0794  0.06754
Similarly
1 0  2  
0 [{ 1} | 1] 1 0.02062
1 B
P B
P X BX P X
P X
     
    
1 0  2  
0 [{ 1} | 1] 1 0.01709
1 C
P C
P X CX P X
P X
     
    
%%Subject CT3 (Probability and Mathematical Statistics Core Technical) – September 2015 – Examiners’ Report
%%Page 8
Thus:
  P[X1 1 |X 0 1]  0.06754 0.02062  0.01709  0.10521
Part (i) Well answered, although some candidates over-complicated the
answer.
Part (ii) Generally well answered.
Part (iii) Reasonably well answered.
Part (iv) This was not well answered. It is a more challenging question, with
other parts leading up to this. Many candidates did not attempt it.
\end{document}

