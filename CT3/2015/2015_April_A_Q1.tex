\documentclass[a4paper,12pt]{article}

%%%%%%%%%%%%%%%%%%%%%%%%%%%%%%%%%%%%%%%%%%%%%%%%%%%%%%%%%%%%%%%%%%%%%%%%%%%%%%%%%%%%%%%%%%%%%%%%%%%%%%%%%%%%%%%%%%%%%%%%%%%%%%%%%%%%%%%%%%%%%%%%%%%%%%%%%%%%%%%%%%%%%%%%%%%%%%%%%%%%%%%%%%%%%%%%%%%%%%%%%%%%%%%%%%%%%%%%%%%%%%%%%%%%%%%%%%%%%%%%%%%%%%%%%%%%
    
\usepackage{eurosym}
\usepackage{vmargin}
\usepackage{amsmath}
\usepackage{graphics}
\usepackage{epsfig}
\usepackage{enumerate}
\usepackage{multicol}
\usepackage{subfigure}
\usepackage{fancyhdr}
\usepackage{listings}
\usepackage{framed}
\usepackage{graphicx}
\usepackage{amsmath}
\usepackage{chngpage}

%\usepackage{bigints}

\usepackage{vmargin}

% left top textwidth textheight headheight
% headsep footheight footskip

\setmargins{2.0cm}{2.5cm}{16 cm}{22cm}{0.5cm}{0cm}{1cm}{1cm}
\renewcommand{\baselinestretch}{1.3}
\setcounter{MaxMatrixCols}{10}
\begin{document}
\begin{enumerate}

%%%%%%%%%%%%%%%%%%%%%%%%%%%%%%%%%%%%%%%%%%%%%%%%%%%%%%%%%%%%%%%%%%%%%%%%%%%%%%%%%
\item 
%% Question 1 
Two groups of students sat the same exam. The marks in the first group of 64 students had an average of 52 and a standard deviation of 9. The marks in the second group of 42 students had an average of 45 and a standard deviation of 8.
Calculate the average and standard deviation of the combined data set of 106 students.

%%%%%%%%%%%%%%%%%%%%%%%%%%%%%%%%%%%%%%%%%%%%%%%%%%%%%%%%%%%%%%%%%%%%%%%%%%%%%%%%%%%%%%%%%%%
\newpage

Page 3
1 i1 1 1 3328
i
x  n x  and i2 2 2 1890
i
x  n x  giving i 5218
i
x 
5218 49.23
106
x 
2
2 2
i1 ( 1 1) 1 i1 / 1 178159
i i
x n s x n
 
    
 
 
2
2 2
i2 ( 2 1) 2 i2 / 2 87674
i i
x n s x n
 
    
 
  giving i2 265833
i
x 
2
2
265833 5218
106
85.4244
105
s
 
  
    and s = 9.243
Generally well answered, although some problems were encountered with the variance.
wered very well. In part (ii) many candidates failed to recognise that the answer relies on the variance being equal to zero.
\end{document}
