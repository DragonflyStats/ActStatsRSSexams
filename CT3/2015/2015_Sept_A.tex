\documentclass[a4paper,12pt]{article}

%%%%%%%%%%%%%%%%%%%%%%%%%%%%%%%%%%%%%%%%%%%%%%%%%%%%%%%%%%%%%%%%%%%%%%%%%%%%%%%%%%%%%%%%%%%%%%%%%%%%%%%%%%%%%%%%%%%%%%%%%%%%%%%%%%%%%%%%%%%%%%%%%%%%%%%%%%%%%%%%%%%%%%%%%%%%%%%%%%%%%%%%%%%%%%%%%%%%%%%%%%%%%%%%%%%%%%%%%%%%%%%%%%%%%%%%%%%%%%%%%%%%%%%%%%%%
  \usepackage{eurosym}
\usepackage{vmargin}
\usepackage{amsmath}
\usepackage{graphics}
\usepackage{epsfig}
\usepackage{enumerate}
\usepackage{multicol}
\usepackage{subfigure}
\usepackage{fancyhdr}
\usepackage{listings}
\usepackage{framed}
\usepackage{graphicx}
\usepackage{amsmath}
\usepackage{chngpage}
%\usepackage{bigints}
\usepackage{vmargin}

% left top textwidth textheight headheight

% headsep footheight footskip

\setmargins{2.0cm}{2.5cm}{16 cm}{22cm}{0.5cm}{0cm}{1cm}{1cm}

\renewcommand{\baselinestretch}{1.3}
\setcounter{MaxMatrixCols}{10}
\begin{document}
\begin{enumerate}

% CT3 S2015–2
1 A random sample of 20 claim amounts (x, in £) made on a certain type of travel insurance policy (type A) was selected and gave the following data summaries:
  x 1 1,860
x2  8, 438, 200
sample mean = 593
sample standard deviation = 271.95.
It was later discovered that two of these 20 claims were made on a different incorrect type of policy, and the corresponding amounts were £770 and £510.
These claims are going to be replaced by two claims made on the correct type of policy, with corresponding amounts £1,000 and £280.
(i) Determine the sample mean and standard deviation of the revised sample. [4]
(ii) Comment on how your answers compare with the original sample mean and standard deviation. [3]
[Total 7]
%%%%%%%%%%%%%%%%%
2 Consider the following measure of skewness for a unimodal distribution:
  mean mode
standard deviation

  .
(i) Determine the value of  for a gamma distribution with parameters α = 1.6 and
 = 0.2. [5]
(ii) Comment on why  is a suitable measure of skewness for distributions with one mode. [3]
[Total 8]
3 Random samples are drawn from two different normally distributed populations with
variances 2
1 and 22
 . From the first population a sample of 25 measurements is taken
with a sample variance of 2
1 s  2.4 . For the second population the sample size is 13
with a sample variance of 2
s2 1.5.
Determine a 95% confidence interval for the ratio of the true variances 2 2
1 / 2. [3]

Solutions
Q1 (i) New sum is (11860 – 770 – 510 + 1000 + 280) = 11860
New sum of squares: (8438200 – 7702 – 5102 + 10002 + 2802) = 8663600
Therefore:
  New sample mean = 11860/20 = 593
New standard deviation (sd) = {(8663600 – 118602/20)/19}0.5 = 292.95
(ii) Since the sum of the two new claims is the same as those replaced, the mean is the same.
However the sd has increased as the two new claims are further away from the
mean as compared to the two claims in the first sample.
Part (i) was generally well answered. In part (ii) the explanation about the sd was not always convincing.
Q2 (i) We have mean = 1.6/0.2 = 8 and sd = (1.6/0.22)0.5 = 6.325
For the mode we need to maximise the probability density function (pdf):
     
           
%%%%%%%%%%%%%%%%%%%%%%%%%%
1
log log log 1 log
f y y e y
fy yy

  

 
           
and
d log f  y 1 0 y 1
dy y
   
   

.
Subject CT3 (Probability and Mathematical Statistics Core Technical) – September 2015 – Examiners’ Report
Page 4
   2
2 2
Also d log f y 1 0.
dy y
   
    
 
So mode is at y 1 3  
 

.
Therefore, 8 3 0.791
6.325

   .
(ii) For unimodal symmetrical distributions, the mode will coincide with the mean and therefore deviations of this measure from 0 will indicate asymmetry.
Also, the measure is standardised by dividing with the standard deviation to make it scale-free.
Many candidates failed to work out the mode correctly. Note that this is a typical calculus maximisation exercise. Part (ii) was not well answered, with many candidates failing to comment on the relationship between the mean and the mode, and very few mentioning the standardisation.
%%%%%%%%%%%%%%%%%%%%%%%%%%%%%%
Q3
2 2
1 1
2 2 24,12
2 2
S / ~ F
S


so the confidence interval is given by
 
2 2
1 1
2 2 12,24
2 24,12 2
* 1 , * 2.4 * 1 , 2.4 *2.541 0.530, 4.066
1.5 3.019 1.5
s s F
s F s
             
Generally very well answered.


\end{document}
