\documentclass[a4paper,12pt]{article}

%%%%%%%%%%%%%%%%%%%%%%%%%%%%%%%%%%%%%%%%%%%%%%%%%%%%%%%%%%%%%%%%%%%%%%%%%%%%%%%%%%%%%%%%%%%%%%%%%%%%%%%%%%%%%%%%%%%%%%%%%%%%%%%%%%%%%%%%%%%%%%%%%%%%%%%%%%%%%%%%%%%%%%%%%%%%%%%%%%%%%%%%%%%%%%%%%%%%%%%%%%%%%%%%%%%%%%%%%%%%%%%%%%%%%%%%%%%%%%%%%%%%%%%%%%%%
  \usepackage{eurosym}
\usepackage{vmargin}
\usepackage{amsmath}
\usepackage{graphics}
\usepackage{epsfig}
\usepackage{enumerate}
\usepackage{multicol}
\usepackage{subfigure}
\usepackage{fancyhdr}
\usepackage{listings}
\usepackage{framed}
\usepackage{graphicx}
\usepackage{amsmath}
\usepackage{chngpage}
%\usepackage{bigints}
\usepackage{vmargin}

% left top textwidth textheight headheight

% headsep footheight footskip

\setmargins{2.0cm}{2.5cm}{16 cm}{22cm}{0.5cm}{0cm}{1cm}{1cm}

\renewcommand{\baselinestretch}{1.3}

\setcounter{MaxMatrixCols}{10}

\begin{document}

\begin{enumerate}

CT3 S2015–7
11 A property agent carries out a study on the relationship between the age of a building and the maintenance costs, X, per square metre per annum based on a sample of 86 buildings. In the sample denote by xi the annual maintenance costs per square metre
for building i. In a first step the sample is divided into new and old buildings. The
maintenance costs are summarised in the following table:
  sample size n xi  2xi
new buildings 25 100 800
old buildings 61 300 2200
\item (i) Perform a test for the null hypothesis that the variance of the maintenance costs of new buildings is equal to the variance of the maintenance costs for old buildings, against the alternative that the variance of the maintenance costs of new buildings is larger. Use a significance level of 5\%. [6]

\item (ii) Perform a test of the null hypothesis that the mean of the maintenance costs of new buildings is equal to the mean of the maintenance costs for old buildings, against the alternative of different means. Use a significance level of 5\%. 
To obtain further insight into the relationship between age and maintenance costs for old buildings the agent wishes to carry out a linear regression analysis. Let A denote the age of a building and X denote the annual maintenance costs per square metre.
The agent uses the model E[X ]  A \beta. The agent has the following summary data
for the age ai and costs xi of the 61 old buildings in the sample.
61
1
i 4,500
i
a

  ,
61
1
30,000

 i i 
i
a x and
61
2
1
i 506,400
i
a

  .
\item (iii) Estimate the correlation coefficient (A, X) between age A and maintenance
costs X. 
(iv) Estimate the parameters γ and \beta. 
%%--- [Total 16]
%%--- END OF PAPER
\newpage
%%%%%%%%%%%%%%%%%%%%%%%%%%%%%%%%%%%%%%%%%%%%%%%%%%%
Q11 \item (i) Test statistic:
  2
new
2 24,60
old
S ~ F
S
2 2
2 2
new
1 25 1 800 25*4 16.67
24 i 25 i 24 S x x
                  
 
2
2
old
2, 200 300 / 61
60
S 
 = 12.08
The 95\% quantile of F24,60 is 1.7 and observed value is
16.67 1.38 1.7
12.08
F  
Therefore, there is no evidence (at 5\% level) to suggest that the variance for new buildings is larger.
\item (ii) Assuming that the two population variances are equal, we have:
  2 24 16.67 60 12.08 13.39
p 84 s
  
 
100 300
25 61 1.06
13.39 1 1
25 61
t

 
    
 
The 0.975 quantile (2-sided test) of the t84 distribution is between 1.98 and
2.00.
There is no evidence to suggest that the mean maintenance costs of new buildings are different from mean maintenance costs of old buildings.
%%%%%%%%%%%
[Alternatively, if we samples are considered large, we can use the z statistic:
    4 4.92 0.987
  16.67 12.05
  25 61
  z
  
   
  
  ]
\item (iii) 1 30,000 1 4,500*300 7,869
ax i i 61 i i 61 S a x  a x   
Sxx  2, 200  3002 / 61  724.6
%%-- Subject CT3 (Probability and Mathematical Statistics Core Technical) – September 2015 – %%%%%%%%%%%%%%%%%%%%%%%%%%%%%%%%
%%-- Page 12
Saa  506, 400  4,5002 / 61 174, 433
 ,  7,869 0.7
174,433*724 6
ˆ
.
ax
aa xx
A X S
S S
   
(iv) 2 ˆ 61*30,000 4,500*300 480,000 0.04511
61*506, 400 4,500 10,640, 400

   

Or, using the results in part \item (iii): 7,869 0.04511
174,
ˆ
433
  
300 0.04511*4,500 1.59
61
ˆ 
  
Or: ˆ 4.92 0.04511* 4500 1.59
61
 
 
  


%% Generally well answered, with some errors in the calculations.
%%END OF %%%%%%%%%%%%%%%%%%%%%%%%%%%%%%%%

\end{document}
