
\documentclass[a4paper,12pt]{article}

%%%%%%%%%%%%%%%%%%%%%%%%%%%%%%%%%%%%%%%%%%%%%%%%%%%%%%%%%%%%%%%%%%%%%%%%%%%%%%%%%%%%%%%%%%%%%%%%%%%%%%%%%%%%%%%%%%%%%%%%%%%%%%%%%%%%%%%%%%%%%%%%%%%%%%%%%%%%%%%%%%%%%%%%%%%%%%%%%%%%%%%%%%%%%%%%%%%%%%%%%%%%%%%%%%%%%%%%%%%%%%%%%%%%%%%%%%%%%%%%%%%%%%%%%%%%

\usepackage{eurosym}
\usepackage{vmargin}
\usepackage{amsmath}
\usepackage{graphics}
\usepackage{epsfig}
\usepackage{enumerate}
\usepackage{multicol}
\usepackage{subfigure}
\usepackage{fancyhdr}
\usepackage{listings}
\usepackage{framed}
\usepackage{graphicx}
\usepackage{amsmath}
\usepackage{chngpage}

%\usepackage{bigints}
\usepackage{vmargin}

% left top textwidth textheight headheight

% headsep footheight footskip

\setmargins{2.0cm}{2.5cm}{16 cm}{22cm}{0.5cm}{0cm}{1cm}{1cm}

\renewcommand{\baselinestretch}{1.3}

\setcounter{MaxMatrixCols}{10}

\begin{document}

10
Let X denote a random variable with a continuous uniform (0, 1000) distribution.
Define a random variable Y as the minimum of X and 800.

(i) Show that the conditional distribution of X given X < 800 is a continuous uniform (0, 800) distribution.
(ii) Verify (giving clear reasons) that the expectation of the random variable Y is 480.
(iii) Suppose that Y 1 , , Y n are independent and identically distributed, each with the same distribution as Y.

In the case that n is large, determine the approximate distribution of
1 n
Y =
Y i , stating its expectation. (You are not required to derive or state
n i 1
the variance of Y .)
[1]
(iv)

Comment on the comparison of the conditional expectation of X given X < 800
with the expectation of Y.

%%%%%%%%%%%%%%%%%%%%%%%%%%%%%%%%%%%%%%%%%%%%%%%%%%%%%%%%%%%%%%%%%%%%%%%%%%%%%%%%%%%%%%%%%
%% Question 11
Consider the following simple model for the number of claims, N, which occur in a
year on a policy:
0
0.55
n
P(N = n)
1
0.25
2
0.15
3
0.05
(a) Explain how you would simulate an observation of N using a number r, an
observation of a random variable which is uniformly distributed on (0, 1).
(b) Illustrate your method described in (i) by simulating three observations of N
using the following random numbers between 0 and 1:
0.6221, 0.1472, 0.9862.
[4]

%%%%%%%%%%%%%%%%%%%%%%%%%%%%%%%%%%%%%%%%%%%%%%%%%%%%%%%%%%%%%%%%%%%%%%%%%%%%%%%%
X ~ U(0,1000) , Y = min(X,800)
(i)
P ( X
x | X
800)
P ( X
x and X 800)
P ( X 800)
P ( X x )
for 0 < x < 800
800 /1000
x /1000
800 /1000
x
for 0
800
x 800
so the conditional distribution is U(0,800)

[other reasonable arguments were given credit, e.g. the conditional distribution is simply a scaled version of the original uniform distribution on a restricted range .]

%%%%%%%%%%%%%%%%%%%%%%%%%%%%%%%%%%

(ii)

E[Y] = E[X|X < 800] P(X < 800) + 800 P(X
400
= 480
Page 4
800
1000
800
200
1000
800)

(iii) Y is approximately normal with expectation 480 by Central Limit Theorem
(iv) E[X | X < 800] = 400 whereas E[Y] = 480.
The higher value for E[Y] results from 20% of the Y values being 800 (and
80% being between 0 and 800).
11
(a)
For 0 r < 0.55
0.55 r < 0.8
0.8 r < 0.95
0.95 r 1
n = 0
n = 1
n = 2
n = 3
[OR any equivalent allocation which reflects the probabilities of the 4 values
of N.]
\end{document}
