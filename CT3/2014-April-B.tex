Let X be a random variable with probability density function:
⎧ 1 x
⎪ e
f ( x ) = ⎨ 2
⎪ 1 e − x
⎩ 2
(i)
; x ≤ 0
; x > 0
Show that the moment generating function of X is given by:
M X ( t ) = (1 − t 2 ) − 1 ,
for t < 1 .
(ii)
CT3 A2014–2
[3]
Hence find the mean and the variance of X using the moment generating
function in part (i).
%%%%%%%%%%%%%%%%%%%%%%%%%%%%%%%%%%%%%%%%%%%%%%%%%%%%%%%%%%%%%
5
Consider ten independent random variables X 1 , ... , X 10 which are identically
distributed with an exponential distribution with expectation 4.
10
(i)
Specify the approximate distribution of X = ∑ X i , including all parameters,
i = 1
using the central limit theorem.
6
[2]
(ii) Calculate the approximate value of the probability P [ X < 40] using the result
in part (i).
[1]
(iii) Calculate the exact probability P [ X < 40] .
(iv) Comment on the answers in parts (ii) and (iii).
[3]
[1]
[Total 7]

%%%%%%%%%%%%%%%%%%%%%%%%%%%%%%%%%%%%%%%%%%%%%%%%%%%%%%%%%%%%%%%%%%%%%%%%%%%%%%%%%
4
(i)
1
M X ( t )  E ( e ) 
2
tX
1  e ( t  1) x 
 

2   t  1  
0

0
 e
( t  1) x

1  e ( t  1) x 
 

2   t  1  

1
dx   e ( t  1) x dx
2
0

0
and for  t  1
Page 3Subject CT3 (Probability and Mathematical Statistics) – April 2014 – Examiners’ Report
M X ( t ) 
(ii)
1  1
1 
1


 
2  t  1 t  1  1  t 2

M  X ( t )  (1  t 2 )  1   (1  t 2 )  2 (  2 t )  2 t (1  t 2 )  2


 E ( X )  M  X (0)  0

M  X ( t )  2 t (1  t 2 )  2

  2(1  t
2  2
)
 2 t (  2)(1  t 2 )  3 (  2 t )
 2(1  t 2 )  2  8 t 2 (1  t 2 )  3
 E ( X 2 )  M  X (0)  2
V ( X )  E ( X 2 )  E 2 ( X )  2
(Alternatively, based on a series expansion:
M X ( t )  1  t 2  t 4  ...  E ( X )  0 and E ( X 2 )  2 and the variance follows.)
Generally well answered. In part (ii) most candidates were familiar with the method, but
some showed poor differentiation skills.
5


(i) X ~ N  ,  2 with   10  4  40 and  2  10   4   160
(ii) X is symmetric so P  X  40   0.5
(iii) The exact distribution of X is gamma(10, 1⁄4)
2
 1

P  X  40   P  2   X  20   P  Y  20  where Y has a  2 distribution
4


with 20 d.f.
P  X  40   P  Y  20   0.5421
(iv)
Although the sample size here is small, the CLT gives an answer which is
close to the exact probability.
Mostly well answered. There were a few problems with the distributions in part (iii). In part
(iv) comments should refer to the use of CLT with small samples for full marks.
Page 4Subject CT3 (Probability and Mathematical Statistics) – April 2014 – Examiners’ Report
