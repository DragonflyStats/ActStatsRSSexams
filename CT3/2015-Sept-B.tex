\documentclass[a4paper,12pt]{article}

%%%%%%%%%%%%%%%%%%%%%%%%%%%%%%%%%%%%%%%%%%%%%%%%%%%%%%%%%%%%%%%%%%%%%%%%%%%%%%%%%%%%%%%%%%%%%%%%%%%%%%%%%%%%%%%%%%%%%%%%%%%%%%%%%%%%%%%%%%%%%%%%%%%%%%%%%%%%%%%%%%%%%%%%%%%%%%%%%%%%%%%%%%%%%%%%%%%%%%%%%%%%%%%%%%%%%%%%%%%%%%%%%%%%%%%%%%%%%%%%%%%%%%%%%%%%
  \usepackage{eurosym}
\usepackage{vmargin}
\usepackage{amsmath}
\usepackage{graphics}
\usepackage{epsfig}
\usepackage{enumerate}
\usepackage{multicol}
\usepackage{subfigure}
\usepackage{fancyhdr}
\usepackage{listings}
\usepackage{framed}
\usepackage{graphicx}
\usepackage{amsmath}
\usepackage{chngpage}
%\usepackage{bigints}
\usepackage{vmargin}

% left top textwidth textheight headheight

% headsep footheight footskip

\setmargins{2.0cm}{2.5cm}{16 cm}{22cm}{0.5cm}{0cm}{1cm}{1cm}

\renewcommand{\baselinestretch}{1.3}

\setcounter{MaxMatrixCols}{10}

\begin{document}

\begin{enumerate}
%%%%%%%%%%%%%%%%%%%%%%%%%%%%%%%%%%%%%%%%%%%%%%%%%%%%%%%%%%%%%%%%%
CT3 S2015–3 PLEASE TURN OVER

%%%%%%%%%%%%%%%%%%%%%%%%%%%%%%%%%%%%%%%%%%%%%%%%%%%%%%%%%%%%%%%%%%%%
  
4 During a particular year, it was found that 83 claims were made in a sample of 500
insurance policies. Policies are assumed to be independent from each other and the
number of claims per policy is identically distributed for all policies according to a
Poisson distribution, with parameter λ denoting the claim rate per policy per year.
(i) Calculate an approximate 95% confidence interval for λ. [3]
(ii) Comment on the validity of this confidence interval. [1]

%%%%%%%%%%%%%%%%%%%%%%%%%%%%%%%%%%%%%%%%%%%%%%%%%%%%%%%%%%%%%%%%%%%%
  
5 An insurance company is accused of delaying payments for large claims. To
investigate this accusation a sample of 25 claims is considered. In each case the
claim size xi (in £) and the time yi (in days) taking to pay the claim are recorded.
Assume that the claim size and the time taken to pay the claim are normally
distributed. In the sample the following statistics have been observed:
  2 2
1 1
25 25
( i ) 5,116,701, ( i ) 61.44
i i
x x y y
 
      .
25
1
( i )( i ) 2,606.96
i
x x y y

    .
(i) Calculate the correlation coefficient between the claim sizes, xi, and the times
taken to pay the claim, yi. [1]
(ii) Perform a statistical test of the hypothesis that the correlation between claim
size and time until payment is zero against the alternative that the correlation
is different from zero. [3]

%%%%%%%%%%%%%%%%%%%%%%%%%%%%%%%%%%%%%%%%%%%%%%%%%%%%%%%%%%%%%%%%%%%%
6 Consider a survey of alcohol consumption in three different locations in the UK. In
each of the three locations 50 men are asked about the units of alcohol they
consumed during the week preceding the survey. The results are summarised in the
following table:
  Location code A B C
Average number of units 26 22 27
Sample standard deviation 7 6 9
Perform a one-way analysis of variance test to test the hypothesis that the location has
no impact on alcohol consumption. [6]
%%%%%%%%%%%%%%%%%%%%%%%%%%%%%%%%%%%%%%%%%%%%%%%%%%%%%%%%%%%%%%%%%%%%
  Q4 (i) We have  
ˆ
N 0,1
n
  

 
approximately, and the confidence interval is
given by
ˆ 1.96 ˆ / 500 with 83 0.166
500
ˆ   
i.e. 0.166 1.96 0.166 / 500 which gives (0.130, 0.202).
Subject CT3 (Probability and Mathematical Statistics Core Technical) – September 2015 – Examiners’ Report
Page 5
(ii) The sample size is large here, so normal approximation is valid.
[Equivalently nλ is large.]
Generally well answered. Some candidates failed to properly justify the use
of the normal approximation.
Q5 (i) 2606.96 0.1470326
5116701 61.44
r 

(ii)
2
25 2 0.147 23 0.71
1 1 0.0216
t r
r
 
  
 
t has t-distribution with 23 d.f. The 95% quantile is 1.714.
Since this is a two-sided test and 0.71 is within the interval [–1.714, 1.714] the
null hypothesis cannot be rejected at 10% level of significance.
(Note that other significance level may also be used.)
[Alternatively, Fisher’s transformation gives z = 0.695, and conclusion is the
  same as above.]
Well answered. Note that the test in part (ii) is two-sided.
Q6 SSR  49 72  62  92   8,134
26 22 27 25
3
Y  
 
 2  2  2  50 26 25 22 25 27 25 700 B SS       
2,147
2 700 147 6.325
2 8134
147
B
R
SS
F SS
  
This is clearly a rather large value since the 1% point from a F2,120 distribution is
4.787, so the null hypothesis is rejected. We conclude that alcohol consumption is
different in different areas.
Subject CT3 (Probability and Mathematical Statistics Core Technical) – September 2015 – Examiners’ Report
Page 6
[Alternatively, the following sums can be computed:
    1,300  yA   yB 1,100  yC 1,350  y  3,750
   y2A  36, 201  yB2  25,964  yC2  40,419  y2 102,584
  SST = 8,834 SSB = 700 ]
Mixed answers. Candidates who were able to calculate correctly the various
sums of squares did well. Note that the main answer provided here is
computationally more efficient than the alternative answer (which most
                                                            candidates preferred).
