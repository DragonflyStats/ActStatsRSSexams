\documentclass[a4paper,12pt]{article}

%%%%%%%%%%%%%%%%%%%%%%%%%%%%%%%%%%%%%%%%%%%%%%%%%%%%%%%%%%%%%%%%%%%%%%%%%%%%%%%%%%%%%%%%%%%%%%%%%%%%%%%%%%%%%%%%%%%%%%%%%%%%%%%%%%%%%%%%%%%%%%%%%%%%%%%%%%%%%%%%%%%%%%%%%%%%%%%%%%%%%%%%%%%%%%%%%%%%%%%%%%%%%%%%%%%%%%%%%%%%%%%%%%%%%%%%%%%%%%%%%%%%%%%%%%%%

\usepackage{eurosym}
\usepackage{vmargin}
\usepackage{amsmath}
\usepackage{graphics}
\usepackage{epsfig}
\usepackage{enumerate}
\usepackage{multicol}
\usepackage{subfigure}
\usepackage{fancyhdr}
\usepackage{listings}
\usepackage{framed}
\usepackage{graphicx}
\usepackage{amsmath}
\usepackage{chngpage}

%\usepackage{bigints}
\usepackage{vmargin}

% left top textwidth textheight headheight

% headsep footheight footskip

\setmargins{2.0cm}{2.5cm}{16 cm}{22cm}{0.5cm}{0cm}{1cm}{1cm}

\renewcommand{\baselinestretch}{1.3}

\setcounter{MaxMatrixCols}{10}

\begin{document}
\large
\noindent Sixty per cent of new drivers in a particular country have had additional driving education. During their first year of driving, new drivers who have not had additional
driving education have a probability 0.09 of having an accident, while new drivers who have had additional driving education have a probability 0.05 of having an
accident.
\begin{enumerate}[(a)]
\item Calculate the probability that a new driver does not have an accident during
their first year of driving.
\item Calculate the probability that a new driver has had additional driving
education, given that the driver had no accidents in the first year.
\end{enumerate}

%%%%%%%%%%%%%%%%%%%%%%%%%%%%%%%%%%%%%%%%%%%%%%%%%%%%%%%%%%%%%%%%%%%%%%%%%%%%%%%5
\newpage
%% Question 3
\noindent \textbf{Notation}\\
\noindent Consider the following events:
\begin{description}
\item[A:] Driver has had additional education
\item[B:] Driver has not had additional education
\item[C:] Driver has not had accident in the first year.
\end{description}

\medskip 

\noindent \textbf{Part (a)}\\
\noindent Calculate the probability that a new driver does not have an accident during
their first year of driving.

\begin{eqnarray*}
P(C)  &=& P(C \mbox{ and } A)  \;+\; P(C \mbox{ and } B)\\ 
&=& P(C|A)P(A)  \;+\; P(C|B)P(B)\\ 
&=& (0.95 \times 0.6 ) \;+\; (0.91 \times 0.4 ) \\
&=& 0.934 \\
\end{eqnarray*}


\noindent \textbf{Part (b)}\\
\noindent Calculate the probability that a new driver has had additional driving
education, given that the driver had no accidents in the first year.


\[ P(A|C)  \;=\; \frac{ P(C|A)P(A) }{P(C)} \;=\; \frac{0.95 \times 0.6}{0.934} \;=\;0.610\]


%Reasonably well answered. Some candidates did not realise that the answer from part (a) could be used in part (b).
\end{document}
