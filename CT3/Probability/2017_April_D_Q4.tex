\documentclass[a4paper,12pt]{article}

%%%%%%%%%%%%%%%%%%%%%%%%%%%%%%%%%%%%%%%%%%%%%%%%%%%%%%%%%%%%%%%%%%%%%%%%%%%%%%%%%%%%%%%%%%%%%%%%%%%%%%%%%%%%%%%%%%%%%%%%%%%%%%%%%%%%%%%%%%%%%%%%%%%%%%%%%%%%%%%%%%%%%%%%%%%%%%%%%%%%%%%%%%%%%%%%%%%%%%%%%%%%%%%%%%%%%%%%%%%%%%%%%%%%%%%%%%%%%%%%%%%%%%%%%%%%

\usepackage{eurosym}
\usepackage{vmargin}
\usepackage{amsmath}
\usepackage{graphics}
\usepackage{epsfig}
\usepackage{enumerate}
\usepackage{multicol}
\usepackage{subfigure}
\usepackage{fancyhdr}
\usepackage{listings}
\usepackage{framed}
\usepackage{graphicx}
\usepackage{amsmath}
\usepackage{chngpage}

%\usepackage{bigints}
\usepackage{vmargin}

% left top textwidth textheight headheight

% headsep footheight footskip

\setmargins{2.0cm}{2.5cm}{16 cm}{22cm}{0.5cm}{0cm}{1cm}{1cm}

\renewcommand{\baselinestretch}{1.3}

\setcounter{MaxMatrixCols}{10}

\begin{document}
	\large 
	\noindent An insurance company calculates car insurance premiums based on the age of the policyholder according to three age groups:
	\begin{description}
	\item[Group A] consists of drivers younger than 22 years old; 
	\item[Group B] consists of drivers 22 to 33 years old;
	\item[Group C] consists of
	drivers older than 33 years.
\end{description}
\medskip
	\begin{itemize}
		\item Its portfolio consists of 10\% Group A policyholders, 38\% Group B policyholders and 52\% Group C policyholders.
		\item 
		The probability of a claim in any 12-month period for a policyholder belonging to Group A, B or C is 13\%, 3\% and 2\%, respectively.
	\end{itemize}
	
	\begin{enumerate}[(a)]
		\item %(i)
		Calculate the probability that a randomly chosen policyholder from this
		portfolio will make a claim during a 12-month period.
		
		
		\item %(ii)
		One of the company’s policyholders has just made a claim. Calculate the probability that the policyholder is younger than 22 years.
	\end{enumerate}
	%%%%%%%%%%%%%%%%%%%%%5
	\newpage \begin{itemize}
		\item Denote by $A$, $B$, $C$ the event that policyholder belongs to the corresponding group.
		\item Also let $F$ be the event that a policyholder makes a claim.
	\end{itemize}
	\medskip
	
	\noindent \textbf{Part (a) } 
	\noindent Calculate the probability that a randomly chosen policyholder from this
	portfolio will make a claim during a 12-month period.
	
	
	
	
	\begin{eqnarray*}
		P(F) &=& P( F \mbox{ and } A) \;+\; P( F \mbox{ and } B) + P( F \mbox{ and } C) \\
		& & \\
		&=& P(F|A)P(A) + P(F|B)P(B) + P(F|C)P(C)\\
		& & \\
		&=& 0.13\times 0.1 + 0.03\times 0.38 + 0.02\times 0.52 \\
		& & \\
		&=& 0.0348\\
	\end{eqnarray*}
	\newpage
	
	
	\noindent \textbf{Part (b) } \\
	\noindent
	One of the company’s policyholders has just made a claim. Calculate the probability that the policyholder is younger than 22 years.
	
	\begin{eqnarray*}
		P ( A | F ) 
		&=& \frac{P(F \mbox{ and } A)}{P(F)}\\
		& & \\
		&=& \frac{P (F | A) P (A) }{P(F)}\\
		& & \\
		&=& \frac{0.13 \times 0.1 }{0.0348}\\
		& & \\
		&=& 0.374\\
	\end{eqnarray*}
	
	%% Generally very well answered, with no particular issues.
	
	%%%%%%%%%%%%%%%%%%%%%%%%%%%%%%%%
	
\end{document}

%%%%%%%%%%%%%%%%%%%%%%%%%%%%%%%
