\documentclass[a4paper,12pt]{article}

%%%%%%%%%%%%%%%%%%%%%%%%%%%%%%%%%%%%%%%%%%%%%%%%%%%%%%%%%%%%%%%%%%%%%%%%%%%%%%%%%%%%%%%%%%%%%%%%%%%%%%%%%%%%%%%%%%%%%%%%%%%%%%%%%%%%%%%%%%%%%%%%%%%%%%%%%%%%%%%%%%%%%%%%%%%%%%%%%%%%%%%%%%%%%%%%%%%%%%%%%%%%%%%%%%%%%%%%%%%%%%%%%%%%%%%%%%%%%%%%%%%%%%%%%%%%

\usepackage{eurosym}
\usepackage{vmargin}
\usepackage{amsmath}
\usepackage{graphics}
\usepackage{epsfig}
\usepackage{enumerate}
\usepackage{multicol}
\usepackage{subfigure}
\usepackage{fancyhdr}
\usepackage{listings}
\usepackage{framed}
\usepackage{graphicx}
\usepackage{amsmath}
\usepackage{chngpage}

%\usepackage{bigints}
\usepackage{vmargin}

% left top textwidth textheight headheight

% headsep footheight footskip

\setmargins{2.0cm}{2.5cm}{16 cm}{22cm}{0.5cm}{0cm}{1cm}{1cm}

\renewcommand{\baselinestretch}{1.3}

\setcounter{MaxMatrixCols}{10}

\begin{document}
\begin{enumerate}
9
Perform an analysis of variance to test the hypothesis that there is no
difference among the three groups as regards coughing.
Comment briefly on any difference among the three groups.
[8]
[1]
[Total 22]
Claims on a certain type of policy are such that the claim amounts are approximately
normally distributed.
(i)
A sample of 101 such claim amounts (in £) yields a sample mean of £416 and
sample standard deviation of £72. For this type of policy:
(a) Obtain a 95% confidence interval for the mean of the claim amounts.
(b) Obtain a 95% confidence interval for the standard deviation of the
claim amounts.
[8]
The company makes various alterations to its policy conditions and thinks that these
changes may result in a change in the mean, but not the standard deviation, of the
claim amounts. It wants to take a random sample of claims in order to estimate the
new mean amount with a 95% confidence interval equal to
sample mean ± £10.
(ii)
Determine how large a sample must be taken, using the following as an
estimate of the standard deviation:
(a) The sample standard deviation from part (i).
(b) The upper limit of the confidence interval for the standard deviation
from part (i)(b).
[6]
(iii)
Comment briefly on your two answers in (ii)(a) and (ii)(b).
CT3 A2011—5
[2]
[Total 16]
%%%%%%%%%%%%%%%%%%%%%%%%%%%%%%%%%%%%%%%%%%%%%%%%%%%%%%%%%%%%%%%%%%%%%%%%%%%%%%%%%%%%%%%%%%%%%%%%%%%%%%%%%%%%%%%%%%%%

(i) (a)
With n large we use normal approximation to t 100 .
416 ± 1.96
72
101
= 416 ± 14.04 = (402.0, 430.0)
(b)
Using
( n − 1) S 2
σ
2
a 95% CI for
σ 2
2
~ χ 100
is
( n − 1) S 2
2
χ 100
(0.025)
2
<σ <
( n − 1) S 2
2
χ 100
(0.975)
⎛ 100 × 72 2 100 × 72 2 ⎞
,
which gives ⎜
⎟ = (4000, 6985).
⎜ 129.6
74.22 ⎟ ⎠
⎝
95% CI for standard deviation σ is therefore
(
Page 6
)
4000, 6985 = (63.2, 83.6).Subject CT3 (Probability and Mathematical Statistics Core Technical) — Examiners’ Report, April 2011
(ii)
(a)
95% CI is x ± 1.96
s
and with s = 72 we have
n
1.96 × 72
= 10 ⇒ n = 199.15
n
So n ≥ 200.
(b)
Taking s = 83.57 gives
1.96 × 83.57
= 10 ⇒ n = 268.30, , so n ≥ 269.
n
\end{document}
