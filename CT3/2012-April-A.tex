 88 lines (72 sloc) 2.45 KB
\documentclass[a4paper,12pt]{article}

%%%%%%%%%%%%%%%%%%%%%%%%%%%%%%%%%%%%%%%%%%%%%%%%%%%%%%%%%%%%%%%%%%%%%%%%%%%%%%%%%%%%%%%%%%%%%%%%%%%%%%%%%%%%%%%%%%%%%%%%%%%%%%%%%%%%%%%%%%%%%%%%%%%%%%%%%%%%%%%%%%%%%%%%%%%%%%%%%%%%%%%%%%%%%%%%%%%%%%%%%%%%%%%%%%%%%%%%%%%%%%%%%%%%%%%%%%%%%%%%%%%%%%%%%%%%

\usepackage{eurosym}
\usepackage{vmargin}
\usepackage{amsmath}
\usepackage{graphics}
\usepackage{epsfig}
\usepackage{enumerate}
\usepackage{multicol}
\usepackage{subfigure}
\usepackage{fancyhdr}
\usepackage{listings}
\usepackage{framed}
\usepackage{graphicx}
\usepackage{amsmath}
\usepackage{chngpage}

%\usepackage{bigints}
\usepackage{vmargin}

% left top textwidth textheight headheight

% headsep footheight footskip

\setmargins{2.0cm}{2.5cm}{16 cm}{22cm}{0.5cm}{0cm}{1cm}{1cm}

\renewcommand{\baselinestretch}{1.3}

\setcounter{MaxMatrixCols}{10}

\begin{document}
\begin{enumerate}
1
The following 24 observations give the length of time (in hours, ordered) for which a
specific fully charged laptop computer will operate on battery before requiring
recharging.
1.2 1.4 1.5 1.6 1.7 1.7 1.8 1.8 1.9 1.9 2.0 2.0
2.1 2.1 2.1 2.2 2.3 2.4 2.4 2.5 3.1 3.6 3.7 4.5
The owner of this computer is about to watch a film on his fully charged laptop.
Calculate from these data the longest showing time for a film that he can watch, so
that the probability that the battery's lifetime will be sufficient for watching the entire
film is 0.75.

2
3
Data were collected on the time (in days) until each of 200 claims is settled by the
insurer in a certain insurance portfolio. A boxplot of the data is shown below.
(i) Calculate the median and the interquartile range of the data using the plot. 
(ii) Comment on the distribution of the data as shown in the plot.

[Total 4]
Two students are selected at random without replacement from a group of 100
students, of whom 64 are male and 36 are female.
Calculate the probability that the two selected students are of different genders.
4

Claim amounts arising under a particular type of insurance policy are modelled as
having a normal distribution with standard deviation £35. They are also assumed to be
independent from each other.
Calculate the probability that two randomly selected claims differ by more than £100.
[4]
CT3 A2012–25
6
%%%%%%%%%%%%%%%%%%%%%%%%%%%%%%
Page 2%%%%%%%%%%%%%%%%%%%%%%%%%%%%%%%%%%%%5 – April 2012 – %%%%%%%%%%%%%%%%%%%%%%%%%%%%%%%%
1
We want the first quartile of the data.
⎛ n + 2 ⎞
Q 1 = ⎜
⎟ th observation counting from below = 6.5th observation
⎝ 4 ⎠
=
1.7 + 1.8
= 1.75 hours.
2
[With alternative definition:
⎛ n + 1 ⎞
Q 1 = ⎜
⎟ th observation counting from below = 1.725 ]
⎝ 4 ⎠
Most answers were quite poor. Many candidates tried to work with a normal or t distribution,
when this was not justified (or required). Only a small number of candidates realised that
quartiles were required – but then a large proportion of them used the wrong quartile.
2
(i) From plot median = 60.5 days, IQR = 112.5 – 26 = 86.5 days.
(ii) The distribution is skewed to the right and a number of values appear to be
outliers.
Well answered.
3
P(1st selected is male and 2 nd selected is female) = 64 36
.
100 99
P(1st selected is female and 2 nd selected is male) = 36 64
.
100 99
⇒ P(selected students are of different genders) = 2.
64 36 128
. =
= 0.465
100 99 275
⎛ 64 ⎞ ⎛ 36 ⎞
⎜ ⎟⎜ ⎟
1
1
64 \times 36 \times 2
= 0.465 ]
[OR P(selected students are of different genders) = ⎝ ⎠ ⎝ ⎠ =
100 \times 99
⎛ 100 ⎞
⎜
⎟
⎝ 2 ⎠
Very well managed by most candidates. Some tried to calculate the probabilities with
replacement.
% Page 3
%%%%%%%%%%%%%%%%%%%%%%%%%%%%%%%%%%%%5 – April 2012 – %%%%%%%%%%%%%%%%%%%%%%%%%%%%%%%%
4
Claim amount ~ N(\mu, 35 2 ) ⇒ difference between 2 claim amounts D ~ N(0, 2\times35 2 )
i.e. D ~ N(0, 2450)
⇒ P(|D| > 100) = P(|Z| > 100/2450 1/2 ) = P(|Z| > 2.020) = 2 * 0.0217 = 0.043
Performance was mixed here. There were some problems with specifying the correct
variance, and a number of answers gave a one-sided probability.
%%%%%%%%%%%%%%%%%%%%%%%%%%%%%%%%%%%%%%%%%%%%%%%%%%%%%%%%%%%%%%%%%%%%%%%%%%%%%%%%%%%%%%%%%%%%%%%%
5
(i)
number of claims in a month X ~ Poisson(2)
from tables: P(X=0) = 0.1353
[alternative: P(X=0) = e −2 ]
(ii)
number of claims in a year X ~ Poisson(24)
from tables: P ( X > 30) = 1 − P ( X \leq 30) = 1 − 0.9042 = 0.0958
[alternative: use normal approximation with continuity correction which
gives 0.0923]
Well answered by majority of candidates.
%%%%%%%%%%%%%%%%%%
© Institute and Faculty of Actuaries1
Calculate the mean, the median and the mode for the data in the following frequency
table.
Observation 0 1 2 3 4
Frequency 20 54 58 28 0

2
The following data are sizes of claims (ordered) for a random sample of 20 recent
claims submitted to an insurance company:
174
487
3
214
490
264
564
298
644
335
686
368
807
381
1092
395
1328
402
1655
442
2272
(i) Calculate the interquartile range for this sample of claim sizes.
(ii) Give a brief interpretation of the interquartile range calculated in part (i). [1]
[Total 4]
%%%%%%%%%%%%%%%%%%%%%%%%%%%%%%%%%%%%%%%%%%%%%%%%%%%%%%%%%%%%%%%%%%%%%%%%%%%%%%%%%%%%%%%%%%%%%
Let X be a discrete random variable with the following probability distribution:
X
P(X = x)
0
0.4
1
0.3
2
0.2
3
0.1
Calculate the variance of Y, where Y = 2X + 10.
4


Consider a random variable U that has a uniform distribution on (0,1) and a random
variable X that has a standard normal distribution. Assume that U and X are
independent.
Determine an expression for the probability density function of the random variable Z
= U + X in terms of the cumulative distribution function of X.
[4]
\end{document}
