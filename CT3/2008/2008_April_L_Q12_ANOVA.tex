\documentclass[a4paper,12pt]{article}

%%%%%%%%%%%%%%%%%%%%%%%%%%%%%%%%%%%%%%%%%%%%%%%%%%%%%%%%%%%%%%%%%%%%%%%%%%%%%%%%%%%%%%%%%%%%%%%%%%%%%%%%%%%%%%%%%%%%%%%%%%%%%%%%%%%%%%%%%%%%%%%%%%%%%%%%%%%%%%%%%%%%%%%%%%%%%%%%%%%%%%%%%%%%%%%%%%%%%%%%%%%%%%%%%%%%%%%%%%%%%%%%%%%%%%%%%%%%%%%%%%%%%%%%%%%%
  \usepackage{eurosym}
\usepackage{vmargin}
\usepackage{amsmath}
\usepackage{graphics}
\usepackage{epsfig}
\usepackage{enumerate}
\usepackage{multicol}
\usepackage{subfigure}
\usepackage{fancyhdr}
\usepackage{listings}
\usepackage{framed}
\usepackage{graphicx}
\usepackage{amsmath}
\usepackage{chngpage}
%\usepackage{bigints}
\usepackage{vmargin}

% left top textwidth textheight headheight

% headsep footheight footskip

\setmargins{2.0cm}{2.5cm}{16 cm}{22cm}{0.5cm}{0cm}{1cm}{1cm}

\renewcommand{\baselinestretch}{1.3}

\setcounter{MaxMatrixCols}{10}

\begin{document}

\begin{enumerate}
%%PLEASE TURN OVER12
\item The members of the computer games clubs of three neighbouring schools decide to take part in a light-hearted competition. Each club selects five of its members at random under a procedure agreed and supervised by the clubs. Each selected student
then plays a particular game at the end of which the score he/she has attained is displayed and recorded – the standard set by the games designers is such that reasonably competent players should score about 100.
The results are as follows:
School 1
School 2
School 3
(i)
105
103
137
134
81
115
96
91
105
147
100
123
116
110
149
An analysis of variance is conducted on these results and gives the following
\begin{verbatim}
ANOVA table:
Source of variation
Between schools
Residual
Total
(a)
d.f. SS MSS
2
12
14 2,298
3,468
5,766 1,149
289
\end{verbatim}
Test the hypothesis that there are no school effects against a general alternative.
You should quote a narrow range of values within which the probability-value of the data lies, and state your conclusion clearly.
(b)
CT3 A2008—6
Calculate a 95\% confidence interval for the underlying mean score for club members in School 1, using the information available from all
three schools.
[7](ii)
The members of the computer games club of a nearby fourth school hear about
the competition and ask to be included in the overall comparison. Scores for a random sample of five of the club members at this school (School 4) are
obtained and are:
112 140 88 103 123.
The scores obtained by all twenty students are shown in the display below:
(a)
Carry out an analysis of variance on the results for all four schools together − you should construct the ANOVA table and test the
hypothesis that there are no school effects against a general alternative.
You should quote an approximate value for the probability-value of the data, and state your conclusion clearly.
(b) Comment briefly on the comparison of the results of the analysis
involving Schools 1−3 only conducted in part (i)(a) and the results here of the analysis involving all four schools.
(c) Calculate a 95\% confidence interval for the difference in the underlying mean scores for club members in Schools 1 and 2, using the
information available from all four schools.

(iv)
\newpage
12
(i)
(a)
F = 1149/289 = 3.98 on (2, 12) degrees of freedom

From Yellow Tables pages 172/3, P-value of the data is between 0.05 and 0.025.
We can reject H_0 (the “no schools effects” hypothesis) at the 5\% level of testing but not at the 1% level. We have some evidence against the “no schools effects” hypothesis − and conclude that there are school
effects (i.e. differences among the underlying means).
(b)
School 1 mean = 598/5 = 119.6
t 12 (0.025) = 2.179
95\% CI for school 1 mean is 119.6 \pm 2.179×(289/5) 1/2
i.e. 119.6 \pm 16.6 or
(ii)
(a)
(103.0, 136.2)
y 1• = 598, y 2• = 485, y 3• = 629, y 4• = 566
y •• = 2278,\sigmay 2 = 266,788
SS_{T}= 266788 – 2278 2 /20 = 7323.8
SS B = (598 2 + 485 2 + 629 2 + 566 2 )/5 − 2278 2 /20 = 2301
⇒ SS R = 7324 − 2301 = 5023

\begin{verbatim}
    Source of variation d.f.
Between schools
Residual
Total 3
16
19
SS MSS
2301
5023
7324 767
314
\end{verbatim}

F = 767/314 = 2.44 on (3, 16) degrees of freedom
%Page 7Subject CT3 (Probability and Mathematical Statistics Core Technical) — April 2008 — Examiners’ Report
From Yellow Tables pages 172/3, P-value of the data is just more than
0.1 (>10%)
\begin{itemize}
    \item We do not have sufficiently strong evidence against the “no schools effects” hypothesis, which can stand.
\item (b) With only three schools involved, the results from one of them (School 2) are sufficiently different from those of the other two to allow us to detect a difference among underlying means.
\item However, the results for the fourth school range across the results for the original three schools
− with all four schools in the comparison, the “between schools” sum of squares is no longer so high relative to the residual and we fail to detect differences.
\end{itemize}

(c) t 16 (0.025) = 2.120
0.5
⎧
⎛ 1 1 ⎞ ⎫
95\% CI is ( 119.6 − 97 ) \pm 2.120 ⎨ 314 ⎜ + ⎟ ⎬
⎝ 5 5 ⎠ ⎭
⎩
i.e. 22.6 \pm 23.76 or (−1.2, 46.4)

\end{document}
