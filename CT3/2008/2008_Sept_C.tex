\documentclass[a4paper,12pt]{article}

%%%%%%%%%%%%%%%%%%%%%%%%%%%%%%%%%%%%%%%%%%%%%%%%%%%%%%%%%%%%%%%%%%%%%%%%%%%%%%%%%%%%%%%%%%%%%%%%%%%%%%%%%%%%%%%%%%%%%%%%%%%%%%%%%%%%%%%%%%%%%%%%%%%%%%%%%%%%%%%%%%%%%%%%%%%%%%%%%%%%%%%%%%%%%%%%%%%%%%%%%%%%%%%%%%%%%%%%%%%%%%%%%%%%%%%%%%%%%%%%%%%%%%%%%%%%

\usepackage{eurosym}
\usepackage{vmargin}
\usepackage{amsmath}
\usepackage{graphics}
\usepackage{epsfig}
\usepackage{enumerate}
\usepackage{multicol}
\usepackage{subfigure}
\usepackage{fancyhdr}
\usepackage{listings}
\usepackage{framed}
\usepackage{graphicx}
\usepackage{amsmath}
\usepackage{chngpage}

%\usepackage{bigints}
\usepackage{vmargin}

% left top textwidth textheight headheight

% headsep footheight footskip

\setmargins{2.0cm}{2.5cm}{16 cm}{22cm}{0.5cm}{0cm}{1cm}{1cm}

\renewcommand{\baselinestretch}{1.3}

\setcounter{MaxMatrixCols}{10}

\begin{document}
\begin{enumerate}

7
Let N be the number of claims arising on a group of policies in a period of one week and suppose that N follows a Poisson distribution with mean 60.
Let $\{X_1 , X_2 ,  \ldots , X_N \}$ be the corresponding claim amounts and suppose that, independently of N, these are independent and identically distributed with mean £500
and standard deviation £400.
N
Let S   X i be the total claim amount for the period of one week.
i  1
\begin{enumerate}[(i)]
\item (i) Determine the mean and the standard deviation of S.
\item (ii) Explain why the distribution of S can be taken as approximately normal, and hence calculate, approximately, the probability that S is greater than £40,000.
\end{enumerate}
%%%%%%%%%%%%%%%%%%%%%%%%%%%%%%%%%%%%%%%%%%[Total 5]
CT3 S2008—3

8
\begin{enumerate}[(i)]
\item (i)
Use the following uniform(0,1) random numbers
0.9236 , 0.2578
and a suitable table of probabilities to simulate two observations of the random
variable X, where $X \sim N(200,100)$.
\item (ii)
Use the following uniform(0,1) random numbers
0.3287 , 0.9142
to simulate two observations of the random variable Y, where Y has an exponential distribution with mean 100.
\end{enumerate}

%%%%%%%%%%%%%%%%%%%%%%%%%%%%%%%%%%%%%%%%%%%%%%%%%%%%%%%%%%%%%%%%%%%%%%%%%%%%%%%%%%%%%%%%%
7
(i)
\[E ( S ) = E ( N ) E ( X ) = (60)(500) = £30, 000\]
\[V ( S ) = E ( N ) V ( X ) + V ( N )[ E ( X )] 2\]
\[= (60)(400 2 ) + (60)(500 2 ) = 24, 600, 000 ∴ sd ( S ) = £4,960\]
(ii)
As S is the sum of a large number of i.i.d. variables, then the central limit theorem gives an approximate normal distribution for S.
%%%%%%%%%%%%%%
P ( S > 40000) = P ( Z >
40000 − 30000
= 2.016)
4960
= 1 − 0.9781 = 0.0219
[Note: 2.02 leading to 0.0217 is also acceptable.]
8
(i)
From Yellow Book Table
$P(Z < 1.43) = 0.9236$ giving x value $(10*1.43) + 200 = 214.3$
$P(Z < −0.65) = 0.2578$ giving x value $(10*(−0.65)) + 200 = 193.5$

(ii)
Setting r = P(Y < y) = 1 – exp(−y/100) ⇒ y = −100*log(1 − r)
$r = 0.3287$ therefore $y = −100log(0.6713) = 39.85$
$r = 0.9142$ therefore $y = −100log(0.0858) = 245.6$
Note: We can do away with the step of subtracting r from 1 and use.
y = −100*log(r). This gives y = 111.3, 8.971.
\end{document}
