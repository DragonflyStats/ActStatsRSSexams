\documentclass[a4paper,12pt]{article}

%%%%%%%%%%%%%%%%%%%%%%%%%%%%%%%%%%%%%%%%%%%%%%%%%%%%%%%%%%%%%%%%%%%%%%%%%%%%%%%%%%%%%%%%%%%%%%%%%%%%%%%%%%%%%%%%%%%%%%%%%%%%%%%%%%%%%%%%%%%%%%%%%%%%%%%%%%%%%%%%%%%%%%%%%%%%%%%%%%%%%%%%%%%%%%%%%%%%%%%%%%%%%%%%%%%%%%%%%%%%%%%%%%%%%%%%%%%%%%%%%%%%%%%%%%%%

\usepackage{eurosym}
\usepackage{vmargin}
\usepackage{amsmath}
\usepackage{graphics}
\usepackage{epsfig}
\usepackage{enumerate}
\usepackage{multicol}
\usepackage{subfigure}
\usepackage{fancyhdr}
\usepackage{listings}
\usepackage{framed}
\usepackage{graphicx}
\usepackage{amsmath}
\usepackage{chngpage}

%\usepackage{bigints}
\usepackage{vmargin}

% left top textwidth textheight headheight

% headsep footheight footskip

\setmargins{2.0cm}{2.5cm}{16 cm}{22cm}{0.5cm}{0cm}{1cm}{1cm}

\renewcommand{\baselinestretch}{1.3}

\setcounter{MaxMatrixCols}{10}

\begin{document}
%[Total 6]

%%[2]
%%[Total 8]
%%CT3 S2007—410
When a new claim comes into an office it is screened at a first stage and has a probability  of being cleared for progress, otherwise it is rejected. If it clears the first stage, it is then independently screened at a second stage and has the same probability
 of being cleared for progress, otherwise it is rejected.
\bgin{enumerate}[(i)]
\item (i) Explain clearly why the probability of a claim being rejected at the first stage
is 1 - , of being rejected at the second stage is  (1 - ) and of progressing after the two stages is  2 .
[3]
\item (ii) For a sample of n independent claims which came into the office x_1 were rejected at the first stage, x 2 were rejected at the second stage and x 3
progressed after the two stages (x_1 + x 2 + x 3 = n).
(a)
Write down the likelihood L() for this sample and hence show that the derivative of the log-likelihood is given by
x  2 x 3 x_1  x 2

.
log L (  )  2

1  


(b)
Show that the maximum likelihood estimator (MLE) is given by
 ˆ 
x 2  2 x 3
.
x_1  2 x 2  2 x 3
[7]
\item (iii)
(iv)

2
(a) log L (  ) of the log-likelihood in
 2
part (ii) above and hence show that the Cramer-Rao lower bound
 (1   )
(CRlb) is given by
.
n (1   )
\end{enumerate}
(b) Use the asymptotic distribution for the MLE ̂ with the CRlb
evaluated at ̂ to obtain an approximate large-sample 95\% confidence
interval for  expressing it simply in terms of ̂ and n.
[7]
Determine the second derivative
For a sample of 1,000 independent claims, 110 were rejected at the first stage, 96 were rejected at the second stage and 794 progressed after the two stages.
Calculate the MLE ̂ together with an approximate 95\% confidence interval
for .
[3]
\end{enumerate}
%%%%%%%%%%%%%%%%%%%%%%%%%%%%%%%%%%%%%%%%%%%%%%%%%%%%%%%%%%%%%%%%%%%%%%%%%%%%%%%%%%%%%%%%%
\newpage 

10

(i) P(rejected at 1 st ) = 1 – P(cleared at 1 st ) = 1 – \theta
P(rejected at 2 nd ) = P(cleared at 1 st )P(rejected at 2 nd | cleared at 1 st )
= \theta (1 – \theta)
P(progressing after two) = P(cleared at 1 st ) P(cleared at 2 nd ) = \theta 2
(ii)
(a)
L ( \theta ) = [(1 − \theta )] x_1 [ \theta (1 − \theta )] x 2 [ \theta 2 ] x 3
= \theta x 2 + 2 x 3 (1 − \theta ) x_1 + x 2
∴ log L ( \theta ) = ( x 2 + 2 x 3 ) log \theta + ( x_1 + x 2 ) log(1 − \theta )
∴
%%%%%%%%%%%%%%%%%%%%%%%%%%%%%
(b)
x + 2 x 3 x_1 + x 2
\frac{\partial}{\partial} 
log L ( \theta ) = 2
−
1 − \theta
\frac{\partial}{\partial} \theta
\theta
equate to zero for MLE
∴\theta ( x_1 + x 2 ) = (1 − \theta )( x 2 + 2 x 3 )
∴\theta ( x_1 + 2 x 2 + 2 x 3 ) = x 2 + 2 x 3
x 2 + 2 x 3
x_1 + 2 x 2 + 2 x 3
∴\hat{\theta} =
(iii)
%%%%%%%%%%%%%%%%%%%%%%%%%%%%%%%%
(a)
\frac{\partial}{\partial}  2 x 2 + 2 x 3
\frac{\partial}{\partial} \theta \theta 2
log L ( \theta ) = −
2
E {
−
x_1 + x 2
(1 − \theta ) 2
\frac{\partial}{\partial}  2 n \theta (1 − \theta ) + 2 n \theta 2
\frac{\partial}{\partial} \theta \theta 2
log L ( \theta )} = −
2
−
n (1 − \theta ) + n \theta (1 − \theta )
(1 − \theta ) 2
%%-- Page 5Subject CT3 (Probability and Mathematical Statistics Core Technical) — September 2008 — Examiners’ Report
n
n
n (1 + \theta )
1
1
= − (1 + \theta ) −
(1 + \theta ) = − n (1 + \theta )( +
) =−
\theta
\theta 1 − \theta )
\theta (1 − \theta )
(1 − \theta )
1
CRlb =
− E [
\frac{\partial}{\partial} 
log L ( \theta )]
\frac{\partial}{\partial} \theta 2
2
=
\theta (1 − \theta )
n (1 + \theta )
\hat{\theta} ≈ N ( \theta , CRlb ) for large n
(b)
using CRlb =
\hat{\theta} (1 − \hat{\theta} )
\hat{\theta} (1 − \hat{\theta} )
, then \hat{\theta} \approx N ( \theta ,
)
n (1 + \hat{\theta} )
n (1 + \hat{\theta} )
\hat{\theta} (1 − \hat{\theta} )
%%%%%%%%%%%%%%%%%%%%%%%%%%%%% 
95\% CI is \hat{\theta} \pm  1.96
n (1 + \hat{\theta} )
(iv)
\hat{\theta} =
96 + 2(794)
1684
=
= 0.8910
110 + 2(96) + 2(794) 1890
CRlb \approx 
0.8910(1 − 0.8910)
= 0.0000514 ∴ CRlb = 0.00717
1000(1 + 0.8910)
∴95\% CI is 0.8910 \pm  1.96(0.00717)
⇒ 0.891 \pm 0.014 or (0.877, 0.905)
\end{document}
