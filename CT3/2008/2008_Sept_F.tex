\documentclass[a4paper,12pt]{article}

%%%%%%%%%%%%%%%%%%%%%%%%%%%%%%%%%%%%%%%%%%%%%%%%%%%%%%%%%%%%%%%%%%%%%%%%%%%%%%%%%%%%%%%%%%%%%%%%%%%%%%%%%%%%%%%%%%%%%%%%%%%%%%%%%%%%%%%%%%%%%%%%%%%%%%%%%%%%%%%%%%%%%%%%%%%%%%%%%%%%%%%%%%%%%%%%%%%%%%%%%%%%%%%%%%%%%%%%%%%%%%%%%%%%%%%%%%%%%%%%%%%%%%%%%%%%

\usepackage{eurosym}
\usepackage{vmargin}
\usepackage{amsmath}
\usepackage{graphics}
\usepackage{epsfig}
\usepackage{enumerate}
\usepackage{multicol}
\usepackage{subfigure}
\usepackage{fancyhdr}
\usepackage{listings}
\usepackage{framed}
\usepackage{graphicx}
\usepackage{amsmath}
\usepackage{chngpage}

%\usepackage{bigints}
\usepackage{vmargin}

% left top textwidth textheight headheight

% headsep footheight footskip

\setmargins{2.0cm}{2.5cm}{16 cm}{22cm}{0.5cm}{0cm}{1cm}{1cm}

\renewcommand{\baselinestretch}{1.3}

\setcounter{MaxMatrixCols}{10}

\begin{document}
\begin{enumerate}

%%-- CT3 S2007—612
\item Consider a situation in which the data consist of two responses at each of five values
of an explanatory variable (x = 1, 2, 3, 4, 5), so we have a data set with ten responses
(y), as in the following table:
1
12
x
y
1
19
2
18
2
35
3
19
3
44
4
32
4
53
5
44
5
65
For these data x = 30, y = 341, x 2 = 110, y 2 = 14,345 , xy = 1,211
(i)
(ii)
You are asked to carry out a linear regression analysis using these data.
(a) Draw a plot of the data to show the relationship between the response and explanatory values.
(b) Calculate the total, regression, and residual sums of squares for a least- squares linear regression analysis of y on x, and hence calculate the value of $R^2$ , the coefficient of determination.
(c) Determine the equation of the fitted regression line.
(d) Calculate a 95\% confidence interval for the slope of the underlying regression line.

A colleague suggests that it will be simpler and will produce the same results if we use the following reduced data, in which the two responses at each x
value are replaced by their mean:
x
y
1
2
3
4
5
15.5 26.5 31.5 42.5 54.5
The details of the regression analysis for these data are given in the box below.
Regression equation: y = 5.90 + 9.40 x
Coef Stdev t-ratio p-val
Intercept
x 5.900
9.400 2.233
0.673 2.64
13.96 0.078
0.001
s = 2.129 R-sq = 98.5%
%%%%%%%%%%%%%%%%%%%%%%%%%%%%%%%%%%%%%%%
\begin{verbatim}
Analysis of Variance
Source
df
SS
Regression 1 883.60
Error
3
13.60
Total
4 897.20
MS
F
p-val
883.60 194.91 0.001
4.53
\end{verbatim}
Discuss the similarities and the differences between the two approaches and their
results, in particular addressing the claim by the colleague that the two analyses will
produce “the same results”.

\end{enumerate}
\newpage
%%%%%%%%%%%%%%%%%%%%%%%%%%%%%%%%%%%%%%%%%%%%%%%%%%%%%%%%%%%%%%%%%%%%%%%%%%%%%%%%%%%
12
(i)
(a)
(b)
SSTOT = S yy = 14345 – 341 2 /10 = 2716.9
Σx = 30, Σx 2 = 110 so S xx = 110 – 30 2 /10 = 20
S xy = 1211 − 30*341/10 = 188
∴SSREG = 188 2 /20 = 1767.2
SSRES = 2716.9 − 1767.2 = 949.7
R 2 = 1767.2/2716.9 = 0.650 (65.0%)
(c)
y = a + bx:
b ˆ = 188 / 20 = 9.4
a ˆ = 341/10 − 9.4 × (30 /10) = 5.9
Fitted line is y = 5.9 + 9.4x
(d)
()
t 8 (0.025) = 2.306
Page 8
1/ 2
⎛ 949.7 / 8 ⎞
s . e . b ˆ = ⎜
⎟
⎝ 20 ⎠
= 2.4363

%%---- Subject CT3 (Probability and Mathematical Statistics Core Technical) — September2008 — Examiners’ Report
95% confidence interval for b is given by $9.4 \pm 2.306 × 2.4363$
i.e. 9.4 ± 5.62 i.e. $(3.78, 15.02)$
(ii)
When we replace the pair of responses by their mean:
the equation of the fitted line remains the same
but otherwise the analyses do not produce “equivalent results”
the “fit” of the line is very much better [the goodness-of-fit measure R 2 increases
to a very high value − from 65% to 98.5%]
plus, for example:
the estimate of the slope has a much lower standard error (2.436 drops to 0.6733)
the SSTOT drops hugely (from 2716.9 on 9df to 897.2 on 4df)
the residual error (SSRES) drops hugely [from 949.7 on 8df (error variance
estimate 118.7 ) to 13.60 on 3 df (error variance estimate 4.53)]
BUT we lose all information on the variation of the response for a given value of
the explanatory variable
Note: these and other relevant comments will receive credit.
\end{document}
