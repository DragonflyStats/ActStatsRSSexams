\documentclass[a4paper,12pt]{article}

%%%%%%%%%%%%%%%%%%%%%%%%%%%%%%%%%%%%%%%%%%%%%%%%%%%%%%%%%%%%%%%%%%%%%%%%%%%%%%%%%%%%%%%%%%%%%%%%%%%%%%%%%%%%%%%%%%%%%%%%%%%%%%%%%%%%%%%%%%%%%%%%%%%%%%%%%%%%%%%%%%%%%%%%%%%%%%%%%%%%%%%%%%%%%%%%%%%%%%%%%%%%%%%%%%%%%%%%%%%%%%%%%%%%%%%%%%%%%%%%%%%%%%%%%%%%
  \usepackage{eurosym}
\usepackage{vmargin}
\usepackage{amsmath}
\usepackage{graphics}
\usepackage{epsfig}
\usepackage{enumerate}
\usepackage{multicol}
\usepackage{subfigure}
\usepackage{fancyhdr}
\usepackage{listings}
\usepackage{framed}
\usepackage{graphicx}
\usepackage{amsmath}
\usepackage{chngpage}
%\usepackage{bigints}
\usepackage{vmargin}

% left top textwidth textheight headheight

% headsep footheight footskip

\setmargins{2.0cm}{2.5cm}{16 cm}{22cm}{0.5cm}{0cm}{1cm}{1cm}

\renewcommand{\baselinestretch}{1.3}

\setcounter{MaxMatrixCols}{10}

\begin{document}

\begin{enumerate}
%%CT3 A2008—411
\item In an investigation about the duration of insurance policies of a certain type, a sample of n policies is studied. All n policies have been initiated at the same time, which is also the time of the start of the investigation. For each policy, the time T (in months)
until the policy expires can be modelled as an exponential random variable with parameter \lambda , independently of the times for all other policies.
\begin{enumerate}[(i)]
\item %(i)
Suppose that the investigation is terminated as soon as k policies have expired, where k is a known (predetermined) constant. The observed policy expiry times are denoted by $t_1 , t_2 , \ldots, t_k$ with $0 < k \leq n$ and t 1 < t 2 < ... < t k .
(a) Show that the probability that any randomly selected policy is still in force at the time of the termination of the investigation is e −\lambda  t k .

(b) Show that the likelihood function of the parameter $\lambda$, using information from all n policies, is given by
k
L ( \lambda  ) = \lambda  k e
−\lambda  \sum  t i
i = 1
e − ( n − k ) \lambda  t k .
Hence find the maximum likelihood estimate (MLE) of \lambda .
(c)
Consider an investigation on 20 policies which is terminated when five policies have expired, giving the following observed expiry times (in
months):
\[1.03 6.67 12.70 12.88 21.54\]
Calculate the MLE of $\lambda$ based on this sample.

\item (ii)
Suppose instead that the investigation is terminated after a fixed length of time t_{0}  . The number of policies that have expired by time t_{0}  is considered to be a random variable, denoted by $K$.
(a) Explain clearly why the distribution of K is binomial and determine its parameters.
(b) Hence find the MLE of $\lambda$ in this case.
(c) Consider an investigation on 20 policies that is terminated after 24 months. By the time of termination five policies have expired.
Use this information to calculate the MLE of $\lambda$ in this case.

\end{enumerate}
\end{enumerate}
\newpage
%%%%%%%%%%%%%%%%%%%%%%%%%%%%%%%%%5
11
(i)
(a)
The required probability is
\begin{eqnarray*}
P ( T > t k ) &=& 1 − P ( T ≤ t k ) \\
&=& 1 − F T ( t k )\\
&=& 1 − (1 − e −\lambda  t k ) \\
&=& e − \lambda  t k (using formulae or by integration).\\
\end{eqnarray*}


(b)
The likelihood function is given by:
k
L ( \lambda  ) = ∏ f ( t i )
i = 1
k
(
= ∏ \lambda  e −\lambda  t i
i = 1
n
∏
j = k + 1
P ( T > t k )
) ∏ ( e ) = \lambda  e
n
−\lambda  t k
k
k
−\lambda  \sum  t i
i = 1
e − ( n − k ) \lambda  t k
j = k + 1
%Page 5Subject CT3 (Probability and Mathematical Statistics Core Technical) — April 2008 — Examiners’ Report
For the MLE:
k
l ( \lambda  ) = \log L ( \lambda  ) = k log( \lambda  ) − \lambda  \sum  t i − ( n − k ) \lambda  t k
i = 1
l ′ ( \lambda  ) =
k k
− \sum  t i − ( n − k ) t k
\lambda  i = 1
l ′ ( \lambda  ) = 0 ⇒ \hat{\lambda} =
k
.
k
\sum  t i + ( n − k ) t k
i = 1
[And l ′′ ( \lambda  ) = −
(c)
k
< 0 ]
\lambda  2
%-------------------------------------------------%
For the observed data,
k

$n = 20$, $k = 5$, $t_k = 21.54$,
\sum  t i = 54.82 .
i = 1
\hat{\lambda} =
k
=
k
\sum  t i + ( n − k ) t k
5
= 0.0132 .
54.82 + 15 × 21.54
i = 1
$-----------------------------------------------$
(ii)
\begin{itemize}
\item (a)
We have n policies with independent durations, and each will have expired by the time of termination with probability
p = P ( T ≤ t_{0}  ) = 1 − e^{−\lambda t_{0}}  ,
or will have not expired with probability 1 - p.
(
\item Therefore, K ~ bin n , 1 − e^{−\lambda t_{0}} 
(b)
(
L ( \lambda  ) ∝ 1 − e^{−\lambda t_{0}} 
)
) ( e )
k
−\lambda  t_{0}  n − k
(
)
l ( \lambda  ) = \log L ( \lambda  ) = k \log 1 − e^{−\lambda t_{0}}  − ( n − k ) \lambda  t_{0} 
l ′ ( \lambda  ) =
kt_{0}  e^{−\lambda t_{0}} 
1 − e^{−\lambda t_{0}} 
− ( n − k ) t_{0} 
l ′ ( \lambda  ) = 0 ⇒ e^{−\lambda t_{0}}  =

n − k
1
⎛ k ⎞
⇒ \hat{\lambda} = − \log ⎜ 1 − ⎟
n
t_{0} 
⎝ n ⎠S

%ubject CT3 (Probability and Mathematical Statistics Core Technical) — April 2008 — Examiners’ Report
\item [OR, observed proportion (k/n) is the MLE of corresponding
proportion/probability ${1 − exp(−\lambda t_{0}  )}$; solving for $\lambda$  leads to same
estimate as above.]
\end{itemize}
%--------------------------------------------%
(c)
Now t_{0}  = 24 and all other involved quantities are as before.
1
1
5 ⎞
⎛ k ⎞
⎛
\hat{\lambda} = − \log ⎜ 1 − ⎟ = − \log ⎜ 1 − ⎟ = 0.0120 .
t_{0} 
24
⎝ n ⎠
⎝ 20 ⎠

\end{document}
