\documentclass[a4paper,12pt]{article}

%%%%%%%%%%%%%%%%%%%%%%%%%%%%%%%%%%%%%%%%%%%%%%%%%%%%%%%%%%%%%%%%%%%%%%%%%%%%%%%%%%%%%%%%%%%%%%%%%%%%%%%%%%%%%%%%%%%%%%%%%%%%%%%%%%%%%%%%%%%%%%%%%%%%%%%%%%%%%%%%%%%%%%%%%%%%%%%%%%%%%%%%%%%%%%%%%%%%%%%%%%%%%%%%%%%%%%%%%%%%%%%%%%%%%%%%%%%%%%%%%%%%%%%%%%%%

\usepackage{eurosym}
\usepackage{vmargin}
\usepackage{amsmath}
\usepackage{graphics}
\usepackage{epsfig}
\usepackage{enumerate}
\usepackage{multicol}
\usepackage{subfigure}
\usepackage{fancyhdr}
\usepackage{listings}
\usepackage{framed}
\usepackage{graphicx}
\usepackage{amsmath}
\usepackage{chngpage}

%\usepackage{bigints}
\usepackage{vmargin}

% left top textwidth textheight headheight

% headsep footheight footskip

\setmargins{2.0cm}{2.5cm}{16 cm}{22cm}{0.5cm}{0cm}{1cm}{1cm}

\renewcommand{\baselinestretch}{1.3}

\setcounter{MaxMatrixCols}{10}

\begin{document}
\begin{enumerate} 
%% Institute of Actuaries1
\item The mean of a sample of 30 claim amounts arising from a certain kind of insurance
policy is £5,200. Six of these claim amounts have mean £8,000 while ten others have mean £3,100.
Calculate the mean of the remaining claim amounts in this sample.
2
[3]
\item Five years ago a financial institution issued a specialised type of investment bond and investors had the option to cash in after 1, 2, 3, 4 or 5 years. The following table gives a frequency distribution showing the numbers of those investors who cashed in
at each stage.
duration (length of time held before being cashed in)
1 year
2 years
3 years
4 years
5 years
130
151
97
64
98
Calculate the sample mean and standard deviation of the duration of these bonds before being cashed in.

\item 3
(i)
[4]
Let Y be the sum of two independent random variables $X_1$ and $X_2$ , that is,
Y  X 1  X 2 .
Show that the moment generating function (mgf) of Y is the product of the
mgfs of X 1 and X 2 .
[2]
(ii)
\item Let $X_1$ and $X_2$ be independent gamma random variables with parameters
(  1 ,  ) and (  2 ,  ) , respectively .
Use mgfs to show that Y  X 1  X 2 is also a gamma random variable and
specify its parameters.
[2]
[Total 4]

\end{enumerate}
%%%%%%%%%%%%%%%%%%%%%%%%%%%%%%%%%%%%%%%%%%%%%%%%%%%%%%%%%%%%%%%%%%%%%%%%%%%%%%%5

\newpage
1
n = 30, x = 5200
n 1 = 6, x 1 = 8000
n 2 = 10, x 2 = 3100
n 3 = 14
x =
∑ x = n 1 x 1 + n 2 x 2 + n 3 x 3
n 1 + n 2 + n 3
n
⇒ x 3 =
%%%%%%%%%%%%%%%%%%%%%%%%%%%%%%%%%%%%%%%%%%%%%%%%%%%%%%%%%%%%%%%%%%%%%%%%%%%%
2
nx − n 1 x 1 − n 2 x 2 30 × 5200 − 6 × 8000 − 10 × 3100 77000
=
=
= 5500
n 3
14
14
data: Σ f = 540, Σ fx = 1469, Σ fx 2 = 5081
mean =
1469
= 2.72 years
540
variance =
%%%%%%%%%%%%%%%%%%%%%%%%%%%%%%%%%%%%%%%%%%%%%%%%%%%%%%%%%%%%%%%%%%%%%%%%%%%%
3
(i)
1
1469 2
(5081 −
) = 2.0126
539
540
∴ s.d. = 1.42 years
\begin{eqnarray*}
M Y ( t ) &=& E ( e tY ) \\ &=& E ( e t ( X 1 + X 2 ) )
\\ &=& E ( e tX 1 ) E ( e tX 2 ) \\ &=& M X 1 ( t ) M X 2 ( t )\\
\end{eqnarray*}
(ii)
t
M X i ( t ) = (1 − ) −α i
λ
t
∴ M Y ( t ) = (1 − ) − ( α 1 +α 2 )
λ
so that Y is a gamma r.v. with parameters ( α 1 + α 2 , λ ) .


%Subject CT3 (Probability and Mathematical Statistics Core Technical) — September2008 — Examiners’ Report
\end{document}
