\documentclass[a4paper,12pt]{article}

%%%%%%%%%%%%%%%%%%%%%%%%%%%%%%%%%%%%%%%%%%%%%%%%%%%%%%%%%%%%%%%%%%%%%%%%%%%%%%%%%%%%%%%%%%%%%%%%%%%%%%%%%%%%%%%%%%%%%%%%%%%%%%%%%%%%%%%%%%%%%%%%%%%%%%%%%%%%%%%%%%%%%%%%%%%%%%%%%%%%%%%%%%%%%%%%%%%%%%%%%%%%%%%%%%%%%%%%%%%%%%%%%%%%%%%%%%%%%%%%%%%%%%%%%%%%
  \usepackage{eurosym}
\usepackage{vmargin}
\usepackage{amsmath}
\usepackage{graphics}
\usepackage{epsfig}
\usepackage{enumerate}
\usepackage{multicol}
\usepackage{subfigure}
\usepackage{fancyhdr}
\usepackage{listings}
\usepackage{framed}
\usepackage{graphicx}
\usepackage{amsmath}
\usepackage{chngpage}
%\usepackage{bigints}
\usepackage{vmargin}

% left top textwidth textheight headheight

% headsep footheight footskip

\setmargins{2.0cm}{2.5cm}{16 cm}{22cm}{0.5cm}{0cm}{1cm}{1cm}

\renewcommand{\baselinestretch}{1.3}

\setcounter{MaxMatrixCols}{10}

\begin{document}

\begin{enumerate}
%%-- [Total 3]
\item Consider two events A and B, such that $P ( A ) = 0.3$ and $P ( A \cap B ) = 0.1$ .
Find the minimum and maximum possible values of the conditional probability
P ( A | B ) .
CT3 A2008—2
5
\item An insurance company covers claims from four different non-life portfolios, denoted as G 1 , G 2 , G 3 and G 4 . The number of policies included in each portfolio is given
below:
G 1
G 2
G 3
G 4
Portfolio
No. of policies 4,000 7,000 13,000 6,000
It is estimated that the percentages of policies that will result in a claim in the following year in each of the portfolios are 8\%, 5\%, 2\% and 4\% respectively.

Suppose a policy is chosen at random from the group of 30,000 policies comprising the four portfolios after one year and it is found that a claim did arise on this policy during the year. Calculate the probability that the selected policy comes from
portfolio G 3 .
\end{enumerate}
\newpage
%%%%%%%%%%%%%%
4
The median should be preferred, as it is not sensitive to the extreme observed
claim of \$4320.
\[P ( A | B ) = \frac{P ( A \cap B )}{P ( B )}\]
Maximum value of P(B) is 0.8 in which case P ( A | B ) =
0.1
= 0.125
0.8
Minimum value of $P(B)$ is 0.1, in the case $B \subset A$. Then $P ( A | B ) = 1$
%%---- Page 2Subject CT3  — April 2008 — Examiners’ Report
5
Define the following events:
C: Policy results in a claim;
B i : Policy comes from portfolio i, i = 1, 2, 3, 4.
Then the required probability is P(B 3 |C), and using Bayes’ theorem:
P ( B 3 | C ) =
P ( C | B 3 ) P ( B 3 )
P ( C | B 3 ) P ( B 3 )
,
=
P ( C )
∑ P ( C | B i ) P ( B i )
i
which gives
P ( B 3 | C ) =
0.02 ×
0.08 ×
13
30
4
7
13
6
+ 0.05 × + 0.02 × + 0.04 ×
30
30
30
30
=
0.26 / 30 0.26
=
= 0.222 .
1.17 / 30 1.17
[OR It is possible to argue straight to
260/(320 + 350 + 260 + 240) = 260/1170 = 0.222
which is correct and gets full marks.]


\end{document}
