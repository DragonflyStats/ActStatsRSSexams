\documentclass[a4paper,12pt]{article}

%%%%%%%%%%%%%%%%%%%%%%%%%%%%%%%%%%%%%%%%%%%%%%%%%%%%%%%%%%%%%%%%%%%%%%%%%%%%%%%%%%%%%%%%%%%%%%%%%%%%%%%%%%%%%%%%%%%%%%%%%%%%%%%%%%%%%%%%%%%%%%%%%%%%%%%%%%%%%%%%%%%%%%%%%%%%%%%%%%%%%%%%%%%%%%%%%%%%%%%%%%%%%%%%%%%%%%%%%%%%%%%%%%%%%%%%%%%%%%%%%%%%%%%%%%%%
  \usepackage{eurosym}
\usepackage{vmargin}
\usepackage{amsmath}
\usepackage{graphics}
\usepackage{epsfig}
\usepackage{enumerate}
\usepackage{multicol}
\usepackage{subfigure}
\usepackage{fancyhdr}
\usepackage{listings}
\usepackage{framed}
\usepackage{graphicx}
\usepackage{amsmath}
\usepackage{chngpage}
%\usepackage{bigints}
\usepackage{vmargin}

% left top textwidth textheight headheight

% headsep footheight footskip

\setmargins{2.0cm}{2.5cm}{16 cm}{22cm}{0.5cm}{0cm}{1cm}{1cm}

\renewcommand{\baselinestretch}{1.3}

\setcounter{MaxMatrixCols}{10}

\begin{document}

\begin{enumerate}
© Institute of Actuaries1
The number of claims which arose during the calendar year 2005 on each of a group
of 80 private motor policies was recorded and resulted in the following frequency
distribution:
Number of claims x
Number of policies f
0
64
1
12
2
3
3
0
4
1
For these data\sigmafx = 22,\sigmafx 2 = 40
Calculate the sample mean and standard deviation of the number of claims per policy.

2
Data on a sample of 29 claim amounts give a sample mean of \$461.5 and a sample
standard deviation of \$618.8.
One claim amount of \$3,657.50 is identified as an outlier and after investigation is
found to be in error. Calculate the revised sample mean and standard deviation if this
erroneous amount is removed.

3
The following sample contains claim amounts (\$) on a particular class of insurance
policies:
1,717
1,614
4
1,595
1,524
1,764
4,320
1,464
1,626
1,854
1,440
1,560
1,602
1,698
(i) Determine the mean and the median of the claim amounts.

(ii) State, with reasons, which of the two measures considered above you would
prefer to use to estimate the central point of the claim amounts.

%%%%%%%%%%%%%%%%%%%%%%%%%%%%%%%%%%%%%%%%
© Faculty of Actuaries
© Institute of ActuariesSubject CT3 (Probability and Mathematical Statistics Core Technical) — April 2008 — Examiners’ Report
1
n = 80,\sigmafx = 22,\sigmafx 2 = 40
x = 22 / 80 = 0.275
s 2 =
2
1 ⎛
22 2 ⎞ 33.95
= 0.42975 ⇒ s = 0.656
⎜ ⎜ 40 −
⎟ =
79 ⎝
80 ⎟ ⎠
79
Σ x = nx = 29(461.5) = 13383.5
Σ x 2 = ( n − 1) s 2 + nx 2 = 28(618.8) 2 + 29(461.5) 2 = 16898062
removing the outlier of 3657.5 gives
Σ x = 13383.5 − 3657.5 = 9726
Σ x 2 = 16898062 − 3657.5 2 = 3520756
Therefore x =
9726
= \$347.4
28
1
9726 2
s = [3520756 −
] = 5272.6 Therefore s = \$72.6
27
28
2
3
(i)
x =
23778
= 1829.08.
13
Median = 7 th ordered observation = 1614.
(ii)
