\documentclass[a4paper,12pt]{article}

%%%%%%%%%%%%%%%%%%%%%%%%%%%%%%%%%%%%%%%%%%%%%%%%%%%%%%%%%%%%%%%%%%%%%%%%%%%%%%%%%%%%%%%%%%%%%%%%%%%%%%%%%%%%%%%%%%%%%%%%%%%%%%%%%%%%%%%%%%%%%%%%%%%%%%%%%%%%%%%%%%%%%%%%%%%%%%%%%%%%%%%%%%%%%%%%%%%%%%%%%%%%%%%%%%%%%%%%%%%%%%%%%%%%%%%%%%%%%%%%%%%%%%%%%%%%
  \usepackage{eurosym}
\usepackage{vmargin}
\usepackage{amsmath}
\usepackage{graphics}
\usepackage{epsfig}
\usepackage{enumerate}
\usepackage{multicol}
\usepackage{subfigure}
\usepackage{fancyhdr}
\usepackage{listings}
\usepackage{framed}
\usepackage{graphicx}
\usepackage{amsmath}
\usepackage{chngpage}
%\usepackage{bigints}
\usepackage{vmargin}

% left top textwidth textheight headheight

% headsep footheight footskip

\setmargins{2.0cm}{2.5cm}{16 cm}{22cm}{0.5cm}{0cm}{1cm}{1cm}

\renewcommand{\baselinestretch}{1.3}

\setcounter{MaxMatrixCols}{10}

\begin{document}

\begin{enumerate}
\item % Question 6
Consider two random variables X and Y with joint probability density function (pdf)
\[f ( x , y ) =
4
(1 − xy ) , 0 < x < 1, 0 < y < 1 .
3\]
The marginal pdf of X is given by
\[f ( x ) =
2
(2 − x ) , 0 < x < 1
3\]
with a corresponding marginal pdf for Y by symmetry (you are not asked to verify
these marginal densities).
\begin{enumerate}
    \item (i)
Show that the conditional pdf of Y given X = x is given by
f ( y | x ) = 2
\item (ii)
(1 − xy )
, 0 < y < 1.
(2 − x )

(a) Determine the conditional expectation $E ( Y | X = x )$ as a function of x
and hence determine $E ( Y )$ .
(b) Verify your answer in part (a) by determining $E ( Y )$ directly from the
marginal pdf of Y.
\end{enumerate}
\end{enumerate}

\newpage
%%%%%%%%%%%%%%%%%%%%%%%%%%%%%%%%%%%%%%%%%%%%%%%
6
(i)
f ( y | x ) =
f ( x , y )
f ( x )
4
(1 − xy )
(1 − xy )
3
, 0 < y < 1
=
= 2
2
(2 − x )
(2 − x )
3
(ii)
(a)
E ( Y | X = x ) =
1
2
y (1 − xy ) dy
(2 − x ) ∫ 0
2
y 2
y 3 1
2
1 x
(3 − 2 x )
=
[ − x ] 0 =
( − ) =
(2 − x ) 2
3
(2 − x ) 2 3 3(2 − x )
E ( Y ) = ∫
=
1 (3 − 2 x )
2
(2 − x ) dx
0 3(2 − x ) 3
2 1
2
4
2 1
−
=
−
=
(3
2
x
)
dx
[3
x
x
]
0
9 ∫ 0
9
9

%%%%%%%%%%%%%%%%%%%%%%%%%%%%%%%%%%5 — April 2008 — %%%%%%%%%%%%%%%%%%%%%%%%%%%%%%%%%%%%%%%%%

7
(i)
E ( Y ) =
2 1
2 2 y 3 1 2 2 4
y
(2
−
y
)
dy
=
[ y − ] 0 = ⋅ =
3 ∫ 0
3
3
3 3 9
\[M_X ( t ) = E ( e tX ) = (1 − 4 t ) − 3\] from yellow book
Let Y =
1
X .
2
Therefore \[M Y ( t ) = E ( e tY ) = E ( e tX / 2 ) = M X ( t / 2) = (1 − 2 t ) − 6 / 2\]
which is the m.g.f. of a gamma(3,1/2) or χ 6 2 variable
(ii)
P ( X > 20) = P ( Y > 10)
= 1 – 0.8753 = 0.1247
\end{document}
