
\documentclass[a4paper,12pt]{article}

%%%%%%%%%%%%%%%%%%%%%%%%%%%%%%%%%%%%%%%%%%%%%%%%%%%%%%%%%%%%%%%%%%%%%%%%%%%%%%%%%%%%%%%%%%%%%%%%%%%%%%%%%%%%%%%%%%%%%%%%%%%%%%%%%%%%%%%%%%%%%%%%%%%%%%%%%%%%%%%%%%%%%%%%%%%%%%%%%%%%%%%%%%%%%%%%%%%%%%%%%%%%%%%%%%%%%%%%%%%%%%%%%%%%%%%%%%%%%%%%%%%%%%%%%%%%
  \usepackage{eurosym}
\usepackage{vmargin}
\usepackage{amsmath}
\usepackage{graphics}
\usepackage{epsfig}
\usepackage{enumerate}
\usepackage{multicol}
\usepackage{subfigure}
\usepackage{fancyhdr}
\usepackage{listings}
\usepackage{framed}
\usepackage{graphicx}
\usepackage{amsmath}
\usepackage{chngpage}
%\usepackage{bigints}
\usepackage{vmargin}

% left top textwidth textheight headheight

% headsep footheight footskip

\setmargins{2.0cm}{2.5cm}{16 cm}{22cm}{0.5cm}{0cm}{1cm}{1cm}

\renewcommand{\baselinestretch}{1.3}

\setcounter{MaxMatrixCols}{10}

\begin{document}
 % Question 7
The claim amount X in units of \$1,000 for a certain type of industrial policy is modelled as a gamma variable with parameters $\alpha = 3$ and $\lambda = 1⁄4$.

\begin{enumerate}[(a)]
    \item (i) Use moment generating functions to show that $\frac{1}{2}X \sim \chi^{2}_{6}$.
\item (ii) Hence use tables to find the probability that a claim amount exceeds \$20,000.
\end{enumerate}


%%%%%%%%%%%%%%%%%%%%%%%%%%%%%%%%%%5 — April 2008 — %%%%%%%%%%%%%%%%%%%%%%%%%%%%%%%%%%%%%%%%%

%%- Question 7
\begin{itemize}
\item (i)


\[M_X ( t ) = E(e^{tX}) = (1 -4t)^{-3}\] from yellow book
\item 

Let $Y = \frac{1}{2} X$

Therefore 
\begin{eqnarray*}
M_{Y}(t) &=& E(e^{tY}) \\
&=& E(e^{tX/2})\\ 
&=& M_{X}(t/2) \\
&=& (1\;-\; 2t) ^{-6/2}\\
\end{eqnarray*}
which is the m.g.f. of a gamma(3,1/2) or $\chi^{2}_{6}$ variable
\item (ii)
\begin{eqnarray*}
P ( X > 20) &=& P ( Y > 10)\\
&=& 1 – 0.8753 \\ &=& 0.1247\\
\end{itemize}

\end{document}



