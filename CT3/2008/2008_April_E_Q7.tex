
\documentclass[a4paper,12pt]{article}

%%%%%%%%%%%%%%%%%%%%%%%%%%%%%%%%%%%%%%%%%%%%%%%%%%%%%%%%%%%%%%%%%%%%%%%%%%%%%%%%%%%%%%%%%%%%%%%%%%%%%%%%%%%%%%%%%%%%%%%%%%%%%%%%%%%%%%%%%%%%%%%%%%%%%%%%%%%%%%%%%%%%%%%%%%%%%%%%%%%%%%%%%%%%%%%%%%%%%%%%%%%%%%%%%%%%%%%%%%%%%%%%%%%%%%%%%%%%%%%%%%%%%%%%%%%%
  \usepackage{eurosym}
\usepackage{vmargin}
\usepackage{amsmath}
\usepackage{graphics}
\usepackage{epsfig}
\usepackage{enumerate}
\usepackage{multicol}
\usepackage{subfigure}
\usepackage{fancyhdr}
\usepackage{listings}
\usepackage{framed}
\usepackage{graphicx}
\usepackage{amsmath}
\usepackage{chngpage}
%\usepackage{bigints}
\usepackage{vmargin}

% left top textwidth textheight headheight

% headsep footheight footskip

\setmargins{2.0cm}{2.5cm}{16 cm}{22cm}{0.5cm}{0cm}{1cm}{1cm}

\renewcommand{\baselinestretch}{1.3}

\setcounter{MaxMatrixCols}{10}

\begin{document}

\begin{enumerate}
\item % Question 6
The claim amount X in units of \$1,000 for a certain type of industrial policy is modelled as a gamma variable with parameters $\alpha = 3$ and $\lambda = 1⁄4$.
1
X ~ χ 6 2 .
2
\begin{enumerate}
    \item (i) Use moment generating functions to show that
\item (ii) Hence use tables to find the probability that a claim amount exceeds \$20,000.
\end{enumerate}

\end{enumerate}


%%%%%%%%%%%%%%%%%%%%%%%%%%%%%%%%%%5 — April 2008 — %%%%%%%%%%%%%%%%%%%%%%%%%%%%%%%%%%%%%%%%%
\newpage
7
\begin{itemize}
\item (i)
E ( Y ) =
2 1
2 2 y 3 1 2 2 4
y
(2
−
y
)
dy
=
[ y − ] 0 = ⋅ =
3 ∫ 0
3
3
3 3 9
\[M_X ( t ) = E ( e tX ) = (1 − 4 t ) − 3\] from yellow book
\item Let Y =
1
X .
2
Therefore \[M Y ( t ) = E ( e tY ) = E ( e tX / 2 ) = M X ( t / 2) = (1 − 2 t ) − 6 / 2\]
which is the m.g.f. of a gamma(3,1/2) or χ 6 2 variable
\item (ii)
\begin{eqnarray*}
P ( X > 20) &=& P ( Y > 10)\\
&=& 1 – 0.8753 \\ &=& 0.1247\\
\end{itemize}

\end{document}



