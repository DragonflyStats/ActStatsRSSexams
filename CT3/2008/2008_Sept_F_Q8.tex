\documentclass[a4paper,12pt]{article}

%%%%%%%%%%%%%%%%%%%%%%%%%%%%%%%%%%%%%%%%%%%%%%%%%%%%%%%%%%%%%%%%%%%%%%%%%%%%%%%%%%%%%%%%%%%%%%%%%%%%%%%%%%%%%%%%%%%%%%%%%%%%%%%%%%%%%%%%%%%%%%%%%%%%%%%%%%%%%%%%%%%%%%%%%%%%%%%%%%%%%%%%%%%%%%%%%%%%%%%%%%%%%%%%%%%%%%%%%%%%%%%%%%%%%%%%%%%%%%%%%%%%%%%%%%%%

\usepackage{eurosym}
\usepackage{vmargin}
\usepackage{amsmath}
\usepackage{graphics}
\usepackage{epsfig}
\usepackage{enumerate}
\usepackage{multicol}
\usepackage{subfigure}
\usepackage{fancyhdr}
\usepackage{listings}
\usepackage{framed}
\usepackage{graphicx}
\usepackage{amsmath}
\usepackage{chngpage}

%\usepackage{bigints}
\usepackage{vmargin}

% left top textwidth textheight headheight

% headsep footheight footskip

\setmargins{2.0cm}{2.5cm}{16 cm}{22cm}{0.5cm}{0cm}{1cm}{1cm}

\renewcommand{\baselinestretch}{1.3}

\setcounter{MaxMatrixCols}{10}

\begin{document}
\begin{enumerate}



8
\begin{enumerate}[(i)]
\item (i)
Use the following uniform(0,1) random numbers
0.9236 , 0.2578
and a suitable table of probabilities to simulate two observations of the random
variable X, where $X \sim N(200,100)$.
\item (ii)
Use the following uniform(0,1) random numbers
0.3287 , 0.9142
to simulate two observations of the random variable Y, where Y has an exponential distribution with mean 100.
\end{enumerate}

%%%%%%%%%%%%%%%%%%%%%%%%%%%%%%%%%%%%%%%%%%%%%%%%%%%%%%%%%%%%%%%%%%%%%%%%%%%%%%%%%%%%%%%%%

8
(i)
From Yellow Book Table
$P(Z < 1.43) = 0.9236$ giving x value $(10*1.43) + 200 = 214.3$
$P(Z < −0.65) = 0.2578$ giving x value $(10*(−0.65)) + 200 = 193.5$

(ii)
Setting r = P(Y < y) = 1 – exp(−y/100) ⇒ y = −100*log(1 − r)
$r = 0.3287$ therefore $y = −100log(0.6713) = 39.85$
$r = 0.9142$ therefore $y = −100log(0.0858) = 245.6$
Note: We can do away with the step of subtracting r from 1 and use.
y = −100*log(r). This gives y = 111.3, 8.971.
\end{document}
