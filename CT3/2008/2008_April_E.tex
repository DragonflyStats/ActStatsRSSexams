\documentclass[a4paper,12pt]{article}

%%%%%%%%%%%%%%%%%%%%%%%%%%%%%%%%%%%%%%%%%%%%%%%%%%%%%%%%%%%%%%%%%%%%%%%%%%%%%%%%%%%%%%%%%%%%%%%%%%%%%%%%%%%%%%%%%%%%%%%%%%%%%%%%%%%%%%%%%%%%%%%%%%%%%%%%%%%%%%%%%%%%%%%%%%%%%%%%%%%%%%%%%%%%%%%%%%%%%%%%%%%%%%%%%%%%%%%%%%%%%%%%%%%%%%%%%%%%%%%%%%%%%%%%%%%%
  \usepackage{eurosym}
\usepackage{vmargin}
\usepackage{amsmath}
\usepackage{graphics}
\usepackage{epsfig}
\usepackage{enumerate}
\usepackage{multicol}
\usepackage{subfigure}
\usepackage{fancyhdr}
\usepackage{listings}
\usepackage{framed}
\usepackage{graphicx}
\usepackage{amsmath}
\usepackage{chngpage}
%\usepackage{bigints}
\usepackage{vmargin}

% left top textwidth textheight headheight

% headsep footheight footskip

\setmargins{2.0cm}{2.5cm}{16 cm}{22cm}{0.5cm}{0cm}{1cm}{1cm}

\renewcommand{\baselinestretch}{1.3}

\setcounter{MaxMatrixCols}{10}

\begin{document}

\begin{enumerate}
%%CT3 A2008—411
\item In an investigation about the duration of insurance policies of a certain type, a sample
of n policies is studied. All n policies have been initiated at the same time, which is
also the time of the start of the investigation. For each policy, the time T (in months)
until the policy expires can be modelled as an exponential random variable with
parameter λ, independently of the times for all other policies.
\begin{enumerate}[(i)]
\item %(i)
Suppose that the investigation is terminated as soon as k policies have expired,
where k is a known (predetermined) constant. The observed policy expiry
times are denoted by t 1 , t 2 , ..., t k with 0 < k ≤ n and t 1 < t 2 < ... < t k .
(a) Show that the probability that any randomly selected policy is still in
force at the time of the termination of the investigation is e −λ t k .
(b) Show that the likelihood function of the parameter λ, using information
from all n policies, is given by
k
L ( λ ) = λ k e
−λ ∑ t i
i = 1
e − ( n − k ) λ t k .
Hence find the maximum likelihood estimate (MLE) of λ.
(c)
Consider an investigation on 20 policies which is terminated when five
policies have expired, giving the following observed expiry times (in
months):
\[1.03 6.67 12.70 12.88 21.54\]
Calculate the MLE of λ based on this sample.
[9]
\item (ii)
Suppose instead that the investigation is terminated after a fixed length of time
t 0 . The number of policies that have expired by time t 0 is considered to be a
random variable, denoted by K.
(a) Explain clearly why the distribution of K is binomial and determine its
parameters.
(b) Hence find the MLE of λ in this case.
(c) Consider an investigation on 20 policies that is terminated after 24
months. By the time of termination five policies have expired.
Use this information to calculate the MLE of λ in this case.
[9]
\end{enumerate}
\end{enumerate}
\newpage
%%%%%%%%%%%%%%%%%%%%%%%%%%%%%%%%%5
11
(i)
(a)
The required probability is
P ( T > t k ) = 1 − P ( T ≤ t k ) = 1 − F T ( t k )
= 1 − (1 − e −λ t k ) = e −λ t k (using formulae or by integration).
(b)
The likelihood function is given by:
k
L ( λ ) = ∏ f ( t i )
i = 1
k
(
= ∏ λ e −λ t i
i = 1
n
∏
j = k + 1
P ( T > t k )
) ∏ ( e ) = λ e
n
−λ t k
k
k
−λ ∑ t i
i = 1
e − ( n − k ) λ t k
j = k + 1
%Page 5Subject CT3 (Probability and Mathematical Statistics Core Technical) — April 2008 — Examiners’ Report
For the MLE:
k
l ( λ ) = log L ( λ ) = k log( λ ) − λ ∑ t i − ( n − k ) λ t k
i = 1
l ′ ( λ ) =
k k
− ∑ t i − ( n − k ) t k
λ i = 1
l ′ ( λ ) = 0 ⇒ λ ˆ =
k
.
k
∑ t i + ( n − k ) t k
i = 1
[And l ′′ ( λ ) = −
(c)
k
< 0 ]
λ 2
For the observed data,
k
n = 20, k = 5, t k = 21.54,
∑ t i = 54.82 .
i = 1
λ ˆ =
k
=
k
∑ t i + ( n − k ) t k
5
= 0.0132 .
54.82 + 15 × 21.54
i = 1
(ii)
(a)
We have n policies with independent durations, and each will have
expired by the time of termination with probability
p = P ( T ≤ t 0 ) = 1 − e −λ t 0 ,
or will have not expired with probability 1 - p.
(
Therefore, K ~ bin n , 1 − e −λ t 0
(b)
(
L ( λ ) ∝ 1 − e −λ t 0
)
) ( e )
k
−λ t 0 n − k
(
)
l ( λ ) = log L ( λ ) = k log 1 − e −λ t 0 − ( n − k ) λ t 0
l ′ ( λ ) =
kt 0 e −λ t 0
1 − e −λ t 0
− ( n − k ) t 0
l ′ ( λ ) = 0 ⇒ e −λ t 0 =
Page 6
n − k
1
⎛ k ⎞
⇒ λ ˆ = − log ⎜ 1 − ⎟
n
t 0
⎝ n ⎠S

%ubject CT3 (Probability and Mathematical Statistics Core Technical) — April 2008 — Examiners’ Report
[OR, observed proportion (k/n) is the MLE of corresponding
proportion/probability {1 − exp(−λt 0 )}; solving for λ leads to same
estimate as above.]
(c)
Now t 0 = 24 and all other involved quantities are as before.
1
1
5 ⎞
⎛ k ⎞
⎛
λ ˆ = − log ⎜ 1 − ⎟ = − log ⎜ 1 − ⎟ = 0.0120 .
t 0
24
⎝ n ⎠
⎝ 20 ⎠
12
(i)
(a)
F = 1149/289 = 3.98 on (2, 12) degrees of freedom
From Yellow Tables pages 172/3, P-value of the data is between 0.05
and 0.025.
We can reject H 0 (the “no schools effects” hypothesis) at the 5% level
of testing but not at the 1% level. We have some evidence against the
“no schools effects” hypothesis − and conclude that there are school
effects (i.e. differences among the underlying means).
(b)
School 1 mean = 598/5 = 119.6
t 12 (0.025) = 2.179
95\% CI for school 1 mean is 119.6 \pm 2.179×(289/5) 1/2
i.e. 119.6 \pm 16.6 or
(ii)
(a)
(103.0, 136.2)
y 1• = 598, y 2• = 485, y 3• = 629, y 4• = 566
y •• = 2278,\sigmay 2 = 266,788
SS T = 266788 – 2278 2 /20 = 7323.8
SS B = (598 2 + 485 2 + 629 2 + 566 2 )/5 − 2278 2 /20 = 2301
⇒ SS R = 7324 − 2301 = 5023

\begin{verbatim}
    Source of variation d.f.
Between schools
Residual
Total 3
16
19
SS MSS
2301
5023
7324 767
314
\end{verbatim}

F = 767/314 = 2.44 on (3, 16) degrees of freedom
%Page 7Subject CT3 (Probability and Mathematical Statistics Core Technical) — April 2008 — Examiners’ Report
From Yellow Tables pages 172/3, P-value of the data is just more than
0.1 (>10%)
\begin{itemize}
    \item We do not have sufficiently strong evidence against the “no schools
effects” hypothesis, which can stand.
\item (b) With only three schools involved, the results from one of them (School
2) are sufficiently different from those of the other two to allow us to
detect a difference among underlying means.
\item However, the results for
the fourth school range across the results for the original three schools
− with all four schools in the comparison, the “between schools” sum
of squares is no longer so high relative to the residual and we fail to
detect differences.
\end{itemize}

(c) t 16 (0.025) = 2.120
0.5
⎧
⎛ 1 1 ⎞ ⎫
95\% CI is ( 119.6 − 97 ) \pm 2.120 ⎨ 314 ⎜ + ⎟ ⎬
⎝ 5 5 ⎠ ⎭
⎩
i.e. 22.6 \pm 23.76 or (−1.2, 46.4)
13
(i)
A clearly labelled scatterplot:
There seems to be a positive linear relationship between blood flow and
auricular pressure.
%Page 8Subject CT3 (Probability and Mathematical Statistics Core Technical) — April 2008 — Examiners’ Report
(ii)
n = 12
S xx = 1381.85 −
S yy = 1251 −
113 2
= 186.9167
12
S xy = 1272.2 −
\beta ˆ =
S xy
S xx
=
126.5 2
= 48.3292
12
(126.5)(113)
= 80.9917
12
80.9917
= 1.676
48.3292
1
α ˆ = y − \beta ˆ x = (113 − 1.6758*126.5) = − 8.249
\end{document}
