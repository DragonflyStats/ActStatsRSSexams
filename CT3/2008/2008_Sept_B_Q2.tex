\documentclass[a4paper,12pt]{article}

%%%%%%%%%%%%%%%%%%%%%%%%%%%%%%%%%%%%%%%%%%%%%%%%%%%%%%%%%%%%%%%%%%%%%%%%%%%%%%%%%%%%%%%%%%%%%%%%%%%%%%%%%%%%%%%%%%%%%%%%%%%%%%%%%%%%%%%%%%%%%%%%%%%%%%%%%%%%%%%%%%%%%%%%%%%%%%%%%%%%%%%%%%%%%%%%%%%%%%%%%%%%%%%%%%%%%%%%%%%%%%%%%%%%%%%%%%%%%%%%%%%%%%%%%%%%

\usepackage{eurosym}
\usepackage{vmargin}
\usepackage{amsmath}
\usepackage{graphics}
\usepackage{epsfig}
\usepackage{enumerate}
\usepackage{multicol}
\usepackage{subfigure}
\usepackage{fancyhdr}
\usepackage{listings}
\usepackage{framed}
\usepackage{graphicx}
\usepackage{amsmath}
\usepackage{chngpage}

%\usepackage{bigints}
\usepackage{vmargin}

% left top textwidth textheight headheight

% headsep footheight footskip

\setmargins{2.0cm}{2.5cm}{16 cm}{22cm}{0.5cm}{0cm}{1cm}{1cm}

\renewcommand{\baselinestretch}{1.3}

\setcounter{MaxMatrixCols}{10}

\begin{document}
Five years ago a financial institution issued a specialised type of investment bond and investors had the option to cash in after 1, 2, 3, 4 or 5 years. The following table gives a frequency distribution showing the numbers of those investors who cashed in
at each stage.
duration (length of time held before being cashed in)
1 year
2 years
3 years
4 years
5 years
130
151
97
64
98
\begin{enumerate} 

\item Calculate the sample mean and standard deviation of the duration of these bonds before being cashed in.

\end{enumerate}
%%%%%%%%%%%%%%%%%%%%%%%%%%%%%%%%%%%%%%%%%%%%%%%%%%%%%%%%%%%%%%%%%%%%%%%%%%%%%%%5

\newpage

2
nx − n 1 x 1 − n 2 x 2 30 × 5200 − 6 × 8000 − 10 × 3100 77000
=
=
= 5500
n 3
14
14
data: Σ f = 540, Σ fx = 1469, Σ fx 2 = 5081
mean =
1469
= 2.72 years
540
variance =
%%%%%%%%%%%%%%%%%%%%%%%%%%%%%%%%%%%%%%%%%%%%%%%%%%%%%%%%%%%%%%%%%%%%%%%%%%%%
3
(i)
1
1469 2
(5081 −
) = 2.0126
539
540
∴ s.d. = 1.42 years

\end{document}
