\documentclass[a4paper,12pt]{article}

%%%%%%%%%%%%%%%%%%%%%%%%%%%%%%%%%%%%%%%%%%%%%%%%%%%%%%%%%%%%%%%%%%%%%%%%%%%%%%%%%%%%%%%%%%%%%%%%%%%%%%%%%%%%%%%%%%%%%%%%%%%%%%%%%%%%%%%%%%%%%%%%%%%%%%%%%%%%%%%%%%%%%%%%%%%%%%%%%%%%%%%%%%%%%%%%%%%%%%%%%%%%%%%%%%%%%%%%%%%%%%%%%%%%%%%%%%%%%%%%%%%%%%%%%%%%

\usepackage{eurosym}
\usepackage{vmargin}
\usepackage{amsmath}
\usepackage{graphics}
\usepackage{epsfig}
\usepackage{enumerate}
\usepackage{multicol}
\usepackage{subfigure}
\usepackage{fancyhdr}
\usepackage{listings}
\usepackage{framed}
\usepackage{graphicx}
\usepackage{amsmath}
\usepackage{chngpage}

%\usepackage{bigints}
\usepackage{vmargin}

% left top textwidth textheight headheight

% headsep footheight footskip

\setmargins{2.0cm}{2.5cm}{16 cm}{22cm}{0.5cm}{0cm}{1cm}{1cm}

\renewcommand{\baselinestretch}{1.3}

\setcounter{MaxMatrixCols}{10}

\begin{document}

Let Y be the sum of two independent random variables $X_1$ and $X_2$ , that is,
Y  X 1  X 2 .
\begin{enumerate} 
\item Show that the moment generating function (mgf) of Y is the product of the
mgfs of X 1 and X 2 .

(ii)
\item Let $X_1$ and $X_2$ be independent gamma random variables with parameters
(  1 ,  ) and (  2 ,  ) , respectively .
Use mgfs to show that Y  X 1  X 2 is also a gamma random variable and
specify its parameters.

\end{enumerate}
%%%%%%%%%%%%%%%%%%%%%%%%%%%%%%%%%%%%%%%%%%%%%%%%%%%%%%%%%%%%%%%%%%%%%%%%%%%%%%%5

\newpage


\begin{eqnarray*}
M Y ( t ) &=& E ( e tY ) \\ &=& E ( e t ( X 1 + X 2 ) )
\\ &=& E ( e tX 1 ) E ( e tX 2 ) \\ &=& M X 1 ( t ) M X 2 ( t )\\
\end{eqnarray*}
(ii)
t
M X i ( t ) = (1 − ) −\alpha  i
\lambda 
t
∴ M Y ( t ) = (1 − ) − ( \alpha_{1} +\alpha_{2} )
\lambda 
so that Y is a gamma r.v. with parameters $( \alpha_{1} + \alpha_{2} , \lambda  )$ .


%Subject CT3 (Probability and Mathematical Statistics Core Technical) — September2008 — Examiners’ Report
\end{document}
