\documentclass[a4paper,12pt]{article}

%%%%%%%%%%%%%%%%%%%%%%%%%%%%%%%%%%%%%%%%%%%%%%%%%%%%%%%%%%%%%%%%%%%%%%%%%%%%%%%%%%%%%%%%%%%%%%%%%%%%%%%%%%%%%%%%%%%%%%%%%%%%%%%%%%%%%%%%%%%%%%%%%%%%%%%%%%%%%%%%%%%%%%%%%%%%%%%%%%%%%%%%%%%%%%%%%%%%%%%%%%%%%%%%%%%%%%%%%%%%%%%%%%%%%%%%%%%%%%%%%%%%%%%%%%%%
  \usepackage{eurosym}
\usepackage{vmargin}
\usepackage{amsmath}
\usepackage{graphics}
\usepackage{epsfig}
\usepackage{enumerate}
\usepackage{multicol}
\usepackage{subfigure}
\usepackage{fancyhdr}
\usepackage{listings}
\usepackage{framed}
\usepackage{graphicx}
\usepackage{amsmath}
\usepackage{chngpage}
%\usepackage{bigints}
\usepackage{vmargin}

% left top textwidth textheight headheight

% headsep footheight footskip

\setmargins{2.0cm}{2.5cm}{16 cm}{22cm}{0.5cm}{0cm}{1cm}{1cm}

\renewcommand{\baselinestretch}{1.3}

\setcounter{MaxMatrixCols}{10}

\begin{document}

\begin{enumerate}
PLEASE TURN OVER7
8
The claim amount X in units of \$1,000 for a certain type of industrial policy is
modelled as a gamma variable with parameters \alpha = 3 and \lambda = 1⁄4.
1
X ~ χ 6 2 .
2
(i) Use moment generating functions to show that
(ii) Hence use tables to find the probability that a claim amount exceeds \$20,000.

[Total 5]

A woodcutter has to cut 100 fence posts of a standard length and he has a metal bar of
the required length to act as the standard. The woodcutter decides to vary his
procedure from post to post − he cuts the first post using the metal standard, then uses
this post as his standard for the cut of the next post. He continues in a similar manner,
each time using the most recently cut post as the standard for the next cut.
Each time the woodcutter cuts a post there is an error in the length cut relative to the
standard being employed for that cut − you should assume that the errors are
independent observations of a random variable with mean 0 and standard deviation
3mm.
Calculate, approximately, the probability that the length of the final post differs from
the length of the original metal standard by more than 15mm.

9
10
A researcher wishes to investigate whether a coin is balanced or not, that is if
P ( heads ) = 0.5 . She throws the coin four times and decides to accept the hypothesis
H 0 : P ( heads ) = 0.5 in a test against the alternative H 1 : P ( heads ) \neq 0.5 , if the number
of times that the coin lands “heads” is 1, 2, or 3.
(i) Calculate the probability of the type I error of this test.

(ii) Calculate the probability of the type II error of this test, if the true probability
that the coin lands “heads” is 0.7.

[Total 6]
Pressure readings are taken regularly from a meter. It transpires that, in a random
sample of 100 such readings, 45 are less than 1, 35 are between 1 and 2, and 20 are
between 2 and 3.
Perform a \chi^2 goodness of fit test of the model that states that the readings are
independent observations of a random variable that is uniformly distributed on (0, 3).

%%%%%%%%%%%%%%%%%%%%%%%%%%%%%%%%%%%%
8
Let L be the length of the metal bar and Z i be the error that arises at the i th cut.
Length of 1 st post cut = L + Z 1
Length of 2 nd post cut = L + Z 1 + Z 2
Length of 100 th post cut = L + Z 1 + Z 2 + ... + Z 100
Error in length of last post cut is E = Z 1 + Z 2 + ... + Z 100
E ~ N(0,900) approximately, by CLT
P(|E| <15) ≈ P(|Z| < 15/30) = P(|Z| < 0.5) = 2 × 0.1915 = 0.383
So P(error exceeds 15mm) ≈ 1 – 0.383 = 0.617
9
(i)
Probability of type I error is
\alpha = P (reject H 0 H 0 is true) = P ( X = 0 or X = 4 H 0 is true) ,
which gives
\alpha = P { X = 0 P ( Heads ) = 0.5)} + P { X = 4 P ( Heads ) = 0.5)}
4
4
⎛ 1 ⎞ ⎛ 1 ⎞
= ⎜ ⎟ + ⎜ ⎟ = 0.125.
⎝ 2 ⎠ ⎝ 2 ⎠
Page 4Subject CT3 (Probability and Mathematical Statistics Core Technical) — April 2008 — %%%%%%%%%%%%%%%%%%%%%%%%%%%%%%%%%%
(ii)
Probability of type II error of the test at P(Heads) = 0.7 is
\beta = P (accept H 0 H 1 is true) = 1 − P { X = 0 or X = 4 P ( Heads ) = 0.7}
(
)
⇒ \beta = 1 − 0.3 4 + 0.7 4 = 0.7518.
[OR using P { X = 1 or X = 2 or X = 3 | P ( heads ) = 0.7}]
10
range
observed frequency
expected frequency
0–1
45
100/3
1–2
35
100/3
2–3
20
100/3
\chi^2 = [(45 – 100/3) 2 + (35 – 100/3) 2 + (20 – 100/3) 2 ]/(100/3) = 9.50 on 2df
P-value = P ( \chi^22 > 9.50) < 0.01
Reject model (at the 1% level of testing) as not providing a good fit to the data.
OR 5\% point of \chi^22 is 5.991, so we reject model at 5\%
OR 1% point of \chi^22 is 9.210, so we reject model at 1%
