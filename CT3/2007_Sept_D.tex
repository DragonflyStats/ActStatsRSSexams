
\documentclass[a4paper,12pt]{article}

%%%%%%%%%%%%%%%%%%%%%%%%%%%%%%%%%%%%%%%%%%%%%%%%%%%%%%%%%%%%%%%%%%%%%%%%%%%%%%%%%%%%%%%%%%%%%%%%%%%%%%%%%%%%%%%%%%%%%%%%%%%%%%%%%%%%%%%%%%%%%%%%%%%%%%%%%%%%%%%%%%%%%%%%%%%%%%%%%%%%%%%%%%%%%%%%%%%%%%%%%%%%%%%%%%%%%%%%%%%%%%%%%%%%%%%%%%%%%%%%%%%%%%%%%%%%

\usepackage{eurosym}
\usepackage{vmargin}
\usepackage{amsmath}
\usepackage{graphics}
\usepackage{epsfig}
\usepackage{enumerate}
\usepackage{multicol}
\usepackage{subfigure}
\usepackage{fancyhdr}
\usepackage{listings}
\usepackage{framed}
\usepackage{graphicx}
\usepackage{amsmath}
\usepackage{chngpage}

%\usepackage{bigints}
\usepackage{vmargin}

% left top textwidth textheight headheight

% headsep footheight footskip

\setmargins{2.0cm}{2.5cm}{16 cm}{22cm}{0.5cm}{0cm}{1cm}{1cm}

\renewcommand{\baselinestretch}{1.3}

\setcounter{MaxMatrixCols}{10}

\begin{document}
\begin{enumerate}

7
Let N denote the number of claims which arise in a portfolio of business and let X_{i} be
the amount of the ith claim. Let each of the X_{i} ’s be independently modelled as a
normal variable with mean £10,000 and standard deviation £2,000 and let N be
independently modelled as a Poisson variable with parameter 20.
Calculate the mean and standard deviation of the total claim amount S = X 1 +...+X N .
[3]
8
Claim sizes in a certain insurance situation are modelled by a normal distribution with
mean μ = £30,000 and standard deviation \sigma = £4,000 The insurer defines a claim to
be a large claim if the claim size exceeds £35,000.
(i)
Calculate the probabilities that the size of a claim exceeds
(a)
(b)
£35,000, and
£36,000

(ii)
(iii)
Calculate the probability that the size of a large claim (as defined by the
insurer) exceeds £36,000.

Calculate the probability that a random sample of 5 claims includes 2 which
exceed £35,000 and 3 which are less than £35,000.

[Total 6]
CT3 S2007—3
%%%%%%%%%%%%%%%%%%%%%%%%%9
10
For a certain class of policies issued by a large insurance company it is believed that
the probability of each policy giving rise to any claims is 0.5, independently of all
other policies. A random sample of 250 such policies is selected.
(i) Determine approximately the probability that at least 139 of the policies in the
sample will each give rise to any claims.
[4]
(ii) Suppose we do observe that 139 policies in our sample give rise to at least one
claim. Use your answer to part (i) to determine whether this suggests at the
1% level of significance that the probability of any claims arising from a
policy of this certain class is greater than initially believed.
[3]
[Total 7]
A chi-square test of association for the frequency data in the following 2 × 3 table
Factor B
B1
B2
Factor A
A1
A2
A3
40
30
50
80
30
70
produces a chi-square statistic with value 4.861 and associated P-value 0.089.
Consider a chi-square test of association for the data in the following 2 × 3 table, in
which all frequencies are twice the corresponding frequencies in the first table:
Factor B
B1
B2
Factor A
A1
A2
A3
80
60
100
160
60
140
(i) State, or calculate, the value of the chi-square test statistic for the second table.

(ii) Find the P-value associated with the test statistic in (i).
(iii) Comment on the results.
CT3 S2007—4
[1]

%%%%%%%%%%%%%%%%%%%%%%%%%%%%%%%%%%%%%%%%%%%%%%%%%%%%%%%%%%%%
Page 3Subject CT3  — September 2007 — Examiners’ Report
8
X ~ N with mean μ = 30 and \sigma = 4 (working in units of £1000)
(i)
(a)
(b)
(ii)
35 − 30 ⎞
⎛
P ( X > 35 ) = P ⎜ Z >
⎟ = P ( Z > 1.25 ) = 1 − 0.89435 = 0.10565
4 ⎠
⎝
36 − 30 ⎞
⎛
P ( X > 36 ) = P ⎜ Z >
⎟ = P ( Z > 1.5 ) = 1 − 0.93319 = 0.06681
4 ⎠
⎝
P(X > 36 | X > 35) = P(X > 36 and X > 35) / P(X > 35)
= P(X > 36) / P(X > 35)
= P(Z > 1.5)/P(Z > 1.25) = 0.06681/0.10565 = 0.632
(iii)
9
(i)
⎛ 5 ⎞
2
3
⎜ ⎟ × 0.1056 × 0.8944 = 0.0798
2
⎝ ⎠
If X_{i}s the random variable denoting the number of policies giving a claim,
then X ~ binomial(250,0.5).
Using the normal approximation (CLT), X ≈ N (125, 62.5) .
Using the appropriate continuity correction we have:
P ( X ≥ 139) = P ( X > 138.5)
138.5 − 125 ⎞
⎛
= P ⎜ Z >
⎟ = 1 − \Phi ( 1.7076 ) = 0.044 .
62.5 ⎠
⎝
(ii)
This is a one-sided test of H 0 : p = 0.5 v H 1 : p > 0.5 .
P-value of the test is 0.044 from part (i).
The evidence against the hypothesis that p = 0.5 (and in favour of
p > 0.5) is not strong enough to justify rejecting it at the 1% level of testing
— we cannot conclude “p > 0.5”.
Page 4Subject CT3  — September 2007 — Examiners’ Report
10
(i)
Chi-square statistic is doubled and has value 9.722
OR work it out
(ii)
(
)
P-value is given by P χ 22 > 9.722 = 0.0077
Note: answer = 0.008 is acceptable for the mark
