%%- A

\documentclass[a4paper,12pt]{article}

%%%%%%%%%%%%%%%%%%%%%%%%%%%%%%%%%%%%%%%%%%%%%%%%%%%%%%%%%%%%%%%%%%%%%%%%%%%%%%%%%%%%%%%%%%%%%%%%%%%%%%%%%%%%%%%%%%%%%%%%%%%%%%%%%%%%%%%%%%%%%%%%%%%%%%%%%%%%%%%%%%%%%%%%%%%%%%%%%%%%%%%%%%%%%%%%%%%%%%%%%%%%%%%%%%%%%%%%%%%%%%%%%%%%%%%%%%%%%%%%%%%%%%%%%%%%

\usepackage{eurosym}
\usepackage{vmargin}
\usepackage{amsmath}
\usepackage{graphics}
\usepackage{epsfig}
\usepackage{enumerate}
\usepackage{multicol}
\usepackage{subfigure}
\usepackage{fancyhdr}
\usepackage{listings}
\usepackage{framed}
\usepackage{graphicx}
\usepackage{amsmath}
\usepackage{chngpage}

%\usepackage{bigints}
\usepackage{vmargin}

% left top textwidth textheight headheight

% headsep footheight footskip

\setmargins{2.0cm}{2.5cm}{16 cm}{22cm}{0.5cm}{0cm}{1cm}{1cm}

\renewcommand{\baselinestretch}{1.3}

\setcounter{MaxMatrixCols}{10}

\begin{document}
\begin{enumerate}
%%%%%%%%%%%%%%%%%%%%%%%%%%%%%%%%%%%%%%%%%%%%%%%%%%%%%%%%%%%%%%%%%%%%%%%%%%5
1
A random sample of 12 claim amounts (in units of £1,000) on a general insurance
portfolio is given by:
14.9
12.4
19.4
3.1
17.6
21.5
15.3
20.1
18.8
11.4
46.2
16.2
For these data: Σx = 216.9, Σx 2 = 5,052.13, sample mean x = £18,075
sample median = £16,900, sample standard deviation s = £10,143.
Calculate the sample mean, median, and standard deviation of the sample (of size 10)
which remains after we remove the claim amounts 3.1 and 46.2 from the original
sample (you should show intermediate working and/or give justifications for your
answers).
[6]
%%%%%%%%%%%%%%%%%%%%%%%%%%%%%%%%%%%%%%%%%%%%%%%%%%%%%%%%%%%%%%%%%%%%%%%%%%%%%%%%%%%%%%%%%%%
2
Consider three events A, B, and C for which A and C are independent, and B and C are
mutually exclusive. You are given the probabilities P(A) = 0.3, P(B) = 0.5, P(C) = 0.2
and P(A∩B) = 0.1.
Find the probability that none of A, B, or C occurs.
3
[3]
The random variable X has probability density function
\[f ( x ) = k (1 − x )(1 + x ),
\qquad 0 < x < 1 ,\]
where k is a positive constant.
4
5
(i) Show that k = 1.5.
(ii) Calculate the probability P(X > 0.25).
[2]
[2]
[Total 4]

%%%%%%%%%%%%%%%%%%%%%%%%%%%%%%%%%%%%%%%%%%%%%%%%%%%%%%%%%%%%%%%%%%%%%%%%%%%%%
1
Revised mean = (216.9 – 3.1 – 46.2)/10 = 16.76 i.e. £16,760
Revised median = original median = £16,900
Revised Σx 2 = 5052.13 – 3.1 2 – 46.2 2 = 2908.08
1/ 2
⎧ ⎪ 1 ⎛
167.6 2 ⎞ ⎫ ⎪
so revised standard deviation = ⎨ ⎜ 2908.08 −
⎟ ⎬
⎜
10 ⎟ ⎠ ⎪ ⎭
⎪ ⎩ 9 ⎝
2
= 3.31837 i.e. £3,318.37
P ( A ∪ B ∪ C ) = P ( A ) + P ( B ) + P ( C ) – P ( A ∩ B ) – P ( A ∩ C ) – P ( B ∩ C ) + P ( A ∩ B ∩ C )
= 0.3 + 0.5 + 0.2 – 0.1 – (0.3 × 0.2) = 0.84
(OR via a Venn diagram)
So P (none occur) = 1 – 0.84 = 0.16
1
3
%%%%%%%%%%%%%%%%%%%%%%%%%%%%%%%%%%%%%%%%%%%%%%%%%%%%%%%%%%%%%%%%
(i)
∫
0
(ii)
1
1
⎡
x 3 ⎤
⎛ 1 ⎞
f ( x ) dx = 1 ⇒ ∫ k (1 − x 2 ) dx = 1 ⇒ k ⎢ x − ⎥ = 1 ⇒ k ⎜ 1 − ⎟ = 1 ⇒ k = 1.5
3 ⎥ ⎦
⎝ 3 ⎠
⎣ ⎢
0
0
P ( X > 0.25) =
1
∫ 0.25 f ( x ) dx
1
⎡
x 3 ⎤
= 1.5 × 0.422 = 0.633.
= ∫ 1.5(1 − x ) dx = 1.5 ⎢ x − ⎥
0.25
3 ⎥ ⎦
⎢ ⎣
0.25
1
%%%%%%%%%%%%%%%%%%%%%%%%%%%%%%%%%%%%%%%%%%%%%%%%%%%%%%%%%%%%%%%%%%%%%%%%%%%%%%%%%%%%%
\end{document}
