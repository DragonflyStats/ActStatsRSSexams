
\documentclass[a4paper,12pt]{article}

%%%%%%%%%%%%%%%%%%%%%%%%%%%%%%%%%%%%%%%%%%%%%%%%%%%%%%%%%%%%%%%%%%%%%%%%%%%%%%%%%%%%%%%%%%%%%%%%%%%%%%%%%%%%%%%%%%%%%%%%%%%%%%%%%%%%%%%%%%%%%%%%%%%%%%%%%%%%%%%%%%%%%%%%%%%%%%%%%%%%%%%%%%%%%%%%%%%%%%%%%%%%%%%%%%%%%%%%%%%%%%%%%%%%%%%%%%%%%%%%%%%%%%%%%%%%

\usepackage{eurosym}
\usepackage{vmargin}
\usepackage{amsmath}
\usepackage{graphics}
\usepackage{epsfig}
\usepackage{enumerate}
\usepackage{multicol}
\usepackage{subfigure}
\usepackage{fancyhdr}
\usepackage{listings}
\usepackage{framed}
\usepackage{graphicx}
\usepackage{amsmath}
\usepackage{chngpage}

%\usepackage{bigints}
\usepackage{vmargin}

% left top textwidth textheight headheight

% headsep footheight footskip

\setmargins{2.0cm}{2.5cm}{16 cm}{22cm}{0.5cm}{0cm}{1cm}{1cm}

\renewcommand{\baselinestretch}{1.3}

\setcounter{MaxMatrixCols}{10}

\begin{document}
\begin{enumerate}
%%%%%%%%%%%%%%%%%%%%%%%%%%%%%%%%%%%%%%%%%%%%%%%%%%%%%%%%%%%%%%%%%%%%%%%%%%%%%%%%%
%%%%%%%%%%%%%%%%%%%%%%%%%%%%%%%%%%%%%%%%%%%%%%%%%%%%%%%%%%%%%%%%%%%%%%%%%%%
\item In an opinion poll, a sample of 100 people from a large town were asked which
candidate they would vote for in a forthcoming national election with the following
results:
Candidate
Supporters
\item (i)
A
32
B
47
C
21
\begin{enumerate}[\item (i)]
\item Determine the approximate probability that candidate B will get more than
50% of the vote.

\item A second opinion poll of 150 people was conducted in a different town with the
following results:
Candidate
Supporters
\item (ii)
7
A
57
B
56
C
37
Use an appropriate test to decide whether the two towns have significantly
different voting intentions.
\end{enumerate}
\newpage

%%%%%%%%%%%%%%%%%%%%%%%%%%%%%%%%%%%%%%%%%%%%%%%%%%%%%%%%%%%%%%%%%%%%%%%%%%%%%%%%%%%%
6
\begin{itemize}
\item (i) 

B  0.47
0.5  0.47
P  B  0.5   P 


1  0.47
1  0.47
0.47 *
 0.47 *
100
100

\item (ii) H 0 = Towns have the same voting intentions


  1  Φ  0.601   0.274



Actual Candidate
Town 1
Town 2 A
32
57
89 B
47
56
103 C
21
37
58 Sum
100
150
250
Expected Candidate
Town 1
Town 2 A
35.6
53.4
89 B
41.2
61.8
103 C
23.2
34.8
58 100
150
250
0.364 0.817 0.209 0.243 0.544 0.139
( f  e ) 2
e
Test statistic = 2.315
\item Degrees of freedom = (3  1)*(2  1) = 2. Approximate p-value of X 2 2
distribution is between 0.30 and 0.32 (0.314 from interpolation.)
\begin{itemize}
\item Therefore we fail to reject H 0 that towns have the same voting intentions.
\item The wording in part \item (i) of the question was not entirely clear, as the question should in fact refer to the probability that the candidate will get more than 50\% of the vote in a different sample of the same size. However, there was very little evidence that candidates were confused by this, and marking was generous in cases where answers seemed to be affected. 
\item In general the question was very well answered. Answers including a continuity correction
were also given full marks in part \item (i).
\end{itemize}
\end{itemize}
\end{document}
