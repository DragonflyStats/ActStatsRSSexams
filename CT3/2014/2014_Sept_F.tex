
10 An insurer has collected data on average alcohol consumption (units per week) and
cigarette smoking (average number of cigarettes per day) in eight regions in the UK.
Region, i 1 2 3 4 5 6 7 8 Average
Alcohol units per week, xi 15 25 21 29 13 18 21 17 19.875
Cigarettes per day, yi 4 8 8 10 6 9 7 5 7.125
For these observations we obtain:
  Σxi yi =1,190; Σxi2 = 3,355; Σyi2 = 435
(i) Calculate the coefficient of correlation between alcohol consumption and
cigarette smoking. [4]
(ii) Calculate a 95% confidence interval for the true correlation coefficient. You
may assume that the joint distribution of the two random variables is a
bivariate normal distribution. [6]
(iii) Fit a linear regression model to the data, by considering alcohol consumption
as the explanatory variable. You should write down the model and estimate
the values of the intercept and slope parameters. [3]
(iv) Calculate the coefficient of determination R2 for the regression model in
part (iii). [1]
(v) Give an interpretation of R2 calculated in part (iv). [1]
[Total 15]
END OF PAPER
10 (i) Sxx  3,3558*19.8752 194.875,
Syy  435 8*7.1252  28.875,
Sxy 1190 8*19.875*7.125  57.125
0.76153
xy
xx yy
S
r
S S
  [4]
(ii) 1 log1
2 1
W r
r



is normally distributed with mean 1 log 1
2 1
 
 
and standard
deviation 1/ n 3
Confidence interval for the mean of W: W 1.96 / n  3
Subject CT3 (Probability and Mathematical Statistics) – September 2014 – Examiners’ Report
Page 11
Using r from part (i), the estimated value of W is 0.999848.
This gives a confidence interval of
0.999848 1.96 0.123309176, 1 .87638647
5
  forW.
Since
2
2
1
1
W
W
r e
e



we obtain the C.I. for the true correlation 
 
2 0.123309176 2 1.87638647
2 0.123309176 2 1.87638647
1, 1 0.122688 , 0.95417
1 1
x x
x x
e e
e e
   
  
   
[6]
(iii) Yi  a bXi  i
/ 57.125 0.293137
194.875
ˆ b  Sxy Sxx  
ˆ 1  ˆ  1.29891
8 i i a  y bx  [3]
(iv) R2  0.761532  0.58 [1]
(v) About 58% of the total variability of the response “cigarettes per day” is
statistically explained by alcohol consumption. [1]
[Total 15]
Generally well answered with some problems in part (ii), which involves the more demanding
(and less frequently examined) CI for the correlation coefficient.
END OF EXAMINERS’ REPORT
