\documentclass[a4paper,12pt]{article}

%%%%%%%%%%%%%%%%%%%%%%%%%%%%%%%%%%%%%%%%%%%%%%%%%%%%%%%%%%%%%%%%%%%%%%%%%%%%%%%%%%%%%%%%%%%%%%%%%%%%%%%%%%%%%%%%%%%%%%%%%%%%%%%%%%%%%%%%%%%%%%%%%%%%%%%%%%%%%%%%%%%%%%%%%%%%%%%%%%%%%%%%%%%%%%%%%%%%%%%%%%%%%%%%%%%%%%%%%%%%%%%%%%%%%%%%%%%%%%%%%%%%%%%%%%%%

\usepackage{eurosym}
\usepackage{vmargin}
\usepackage{amsmath}
\usepackage{graphics}
\usepackage{epsfig}
\usepackage{enumerate}
\usepackage{multicol}
\usepackage{subfigure}
\usepackage{fancyhdr}
\usepackage{listings}
\usepackage{framed}
\usepackage{graphicx}
\usepackage{amsmath}
\usepackage{chngpage}

%\usepackage{bigints}
\usepackage{vmargin}

% left top textwidth textheight headheight

% headsep footheight footskip

\setmargins{2.0cm}{2.5cm}{16 cm}{22cm}{0.5cm}{0cm}{1cm}{1cm}

\renewcommand{\baselinestretch}{1.3}

\setcounter{MaxMatrixCols}{10}

\begin{document}
\begin{enumerate}

1
The following sample shows the durations x i in minutes for 20 journeys from
Edinburgh to Glasgow:
51 53 54 55 59 59 60 60 60 69 71 72 74 90 97 104 107 108 115 167
20
with
∑ x i =1,585 and
i = 1
2
20
∑ x i 2 = 142,127.
i = 1
(i) Calculate the mean and the median of this sample.
(ii) Calculate the standard deviation of this sample.
[2]
[2]
%%%%%%%%%%%%%%%%%

A set of data has mean 62 and standard deviation 6.

Derive a linear transformation for these data that will result in the new data having mean 50 and standard deviation 12.
[3]
3
4
Sixty per cent of new drivers in a particular country have had additional driving education. During their first year of driving, new drivers who have not had additional
driving education have a probability 0.09 of having an accident, while new drivers who have had additional driving education have a probability 0.05 of having an
accident.
(a) Calculate the probability that a new driver does not have an accident during
their first year of driving.
(b) Calculate the probability that a new driver has had additional driving
education, given that the driver had no accidents in the first year.
[5]
%%%%%%%%%%%%%%%%%%%%%%%%%%%%%%%%%%%%%%%%%%%%%%%%%%%%%%%%%%%%%%%%%%%%%%%%%%%%%%%%%
1
(i)
Mean =
1585
 79.25
20
Median = 70
142,127 
(ii)
Var =
19
1585 2
20  869.25 ,
SD  869.25  29.48
Well answered.
2
We want to find a and b for y = a + bx such that
y  a  bx  50 and s y  b s x
These give b = 2 and a = 50 – 124 = – 74
or, b = – 2 and a = 50 + 124 = 174
Generally well answered.
3
Consider the following events:
A: Driver has had additional education
B: Driver has not had additional education
C: Driver has not had accident in the first year.
(a)
\[P(C) = P(C|A) Pr(A) + P(C|B) Pr(B) = 0.95*0.6 + 0.91*0.4
= 0.934\]
(b)
P  A  C  
P  C | A  P  A 
P  C 

0.95  0.6
 0.610
0.934
Reasonably well answered. Some candidates did not realise that the answer from part (a)
could be used in part (b).
\end{document}
