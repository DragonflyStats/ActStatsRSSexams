
\documentclass[a4paper,12pt]{article}

%%%%%%%%%%%%%%%%%%%%%%%%%%%%%%%%%%%%%%%%%%%%%%%%%%%%%%%%%%%%%%%%%%%%%%%%%%%%%%%%%%%%%%%%%%%%%%%%%%%%%%%%%%%%%%%%%%%%%%%%%%%%%%%%%%%%%%%%%%%%%%%%%%%%%%%%%%%%%%%%%%%%%%%%%%%%%%%%%%%%%%%%%%%%%%%%%%%%%%%%%%%%%%%%%%%%%%%%%%%%%%%%%%%%%%%%%%%%%%%%%%%%%%%%%%%%

\usepackage{eurosym}
\usepackage{vmargin}
\usepackage{amsmath}
\usepackage{graphics}
\usepackage{epsfig}
\usepackage{enumerate}
\usepackage{multicol}
\usepackage{subfigure}
\usepackage{fancyhdr}
\usepackage{listings}
\usepackage{framed}
\usepackage{graphicx}
\usepackage{amsmath}
\usepackage{chngpage}

%\usepackage{bigints}
\usepackage{vmargin}

% left top textwidth textheight headheight

% headsep footheight footskip

\setmargins{2.0cm}{2.5cm}{16 cm}{22cm}{0.5cm}{0cm}{1cm}{1cm}

\renewcommand{\baselinestretch}{1.3}

\setcounter{MaxMatrixCols}{10}

\begin{document}
\begin{enumerate}

CT3 S2014–3 
5 Consider two random variables X and Y with E[X] = 2, V[X] = 4, E[Y] = −3, V[Y] = 1,
and Cov[X, Y] = 1.6.
Calculate:
(a) the expected value of 5X + 20Y.
(b) the correlation coefficient between X and Y.
(c) the expected value of the product XY.
(d) the variance of X − Y.

6 In a medical study conducted to test the suggestion that daily exercise has the effect of lowering blood pressure, a sample of eight patients with high blood pressure was
selected. Their blood pressure was measured initially and then again a month later after they had participated in an exercise programme. The results are shown in the
table below:
Patient 1 2 3 4 5 6 7 8
Before 155 152 146 153 146 160 139 148
After 145 147 123 137 141 142 140 138
\begin{enumerate}[\item (i)]
\item \item (i) Explain why a standard two-sample t-test would not be appropriate in this
investigation to test the suggestion that daily exercise has the effect of
lowering blood pressure. 
\item \item (ii) Perform a suitable t-test for this medical study. You should clearly state the
null and alternative hypotheses. [7]
\end{enumerate}
%%%%%%%%%%%%%%%%%%%%%%%%%%%%%%%%%%%%%%%%%%%%%%%%%%%%%%%%%%%%%%%%%%%

5 (a) E5X  20Y   5*2  20*3 10  60   50
(b)  ,  1.6 0.8
2
Corr X Y  
(c) EXY   Cov  X ,Y   EX EY  1.6  2*3  4.4
(d) V  X Y  V  X  V Y   2Cov  X ,Y   4 1 3.2 1.8

Generally very well answered. There were only a few problems with using the correct
expression for the variance (taking into account the covariance).
 (Probability and Mathematical Statistics) – September 2014 

%%%%%%%%%%%%%%%%%%%%%%%%%%%%%%%%
Page 6
6 \item \item (i) The two samples are from the same patients, so they are clearly not
independent. 
\item \item (ii) First calculate differences d = measurement before – measurement after :
  d : 10 5 23 16 5 18 −1 10
For these we have d  86,d 2 1,360
giving d  86 / 8 10.75 and sd(d)  (1360 862 / 8) / 7  7.8876
H0: mean difference = 0 v H1: mean difference > 0
10.75 3.855
( ) 7.8876 8
t d
sd d n
  
From tables, t7 (0.005)  3.499 and t7 (0.001)  4.785
Therefore, we have strong evidence against H0 (P-value < 0.5\%), and
conclude that daily exercise has the effect of lowering blood pressure. 
[Total 8]
The question clearly indicates that a standard two-sample t
test is not appropriate here, and candidates should recognise the need for a paired test. In
some cases, although the correct test was identified, its application was wrong.
\end{document}
