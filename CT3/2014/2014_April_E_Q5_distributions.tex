\documentclass[a4paper,12pt]{article}

%%%%%%%%%%%%%%%%%%%%%%%%%%%%%%%%%%%%%%%%%%%%%%%%%%%%%%%%%%%%%%%%%%%%%%%%%%%%%%%%%%%%%%%%%%%%%%%%%%%%%%%%%%%%%%%%%%%%%%%%%%%%%%%%%%%%%%%%%%%%%%%%%%%%%%%%%%%%%%%%%%%%%%%%%%%%%%%%%%%%%%%%%%%%%%%%%%%%%%%%%%%%%%%%%%%%%%%%%%%%%%%%%%%%%%%%%%%%%%%%%%%%%%%%%%%%

\usepackage{eurosym}
\usepackage{vmargin}
\usepackage{amsmath}
\usepackage{graphics}
\usepackage{epsfig}
\usepackage{enumerate}
\usepackage{multicol}
\usepackage{subfigure}
\usepackage{fancyhdr}
\usepackage{listings}
\usepackage{framed}
\usepackage{graphicx}
\usepackage{amsmath}
\usepackage{chngpage}

%\usepackage{bigints}
\usepackage{vmargin}

% left top textwidth textheight headheight

% headsep footheight footskip

\setmargins{2.0cm}{2.5cm}{16 cm}{22cm}{0.5cm}{0cm}{1cm}{1cm}

\renewcommand{\baselinestretch}{1.3}

\setcounter{MaxMatrixCols}{10}

\begin{document}

%%%%%%%%%%%%%%%%%%%%%%%%%%%%%%%%%%%%%%%%%%%%%%%%%%%%%%%%%%%%%
%%--- Question 5
Consider ten independent random variables X 1 , ... , X 10 which are identically
distributed with an exponential distribution with expectation 4.
%%-- 10
\begin{enumerate}
\item (i)
Specify the approximate distribution of $X = \sum  X i$ , including all parameters,
i = 1
using the central limit theorem.
6

\item (ii) Calculate the approximate value of the probability $P [ X < 40]$ using the result
in part (i).

\item (iii) Calculate the exact probability P $[ X < 40]$ .
\item (iv) Comment on the answers in parts (ii) and (iii).
\end{enumerate}

%%%%%%%%%%%%%%%%%%%%%%%%%%%%%%%%%%%%%%%%%%%%%%%%%%
\newpage
5


\begin{itemize}
\item (i) X ~ N  ,  2 with   10  4  40 and  2  10   4   160
\item (ii) X is symmetric so P ( X  40   0.5
\item (iii) The exact distribution of X is gamma(10, 1⁄4)
2
 1

P ( X  40   P  2   X  20   P ( Y  20  where Y has a  2 distribution
4


with 20 d.f.
P ( X  40   P ( Y  20   0.5421
\item (iv)
Although the sample size here is small, the CLT gives an answer which is
close to the exact probability.
Mostly well answered. There were a few problems with the distributions in part (iii). In part
\item (iv) comments should refer to the use of CLT with small samples for full marks.
\end{itemize}
%%Page 4 (Probability and Mathematical Statistics) – April 2014 – 
\end{document}
