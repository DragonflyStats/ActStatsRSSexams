\documentclass[a4paper,12pt]{article}

%%%%%%%%%%%%%%%%%%%%%%%%%%%%%%%%%%%%%%%%%%%%%%%%%%%%%%%%%%%%%%%%%%%%%%%%%%%%%%%%%%%%%%%%%%%%%%%%%%%%%%%%%%%%%%%%%%%%%%%%%%%%%%%%%%%%%%%%%%%%%%%%%%%%%%%%%%%%%%%%%%%%%%%%%%%%%%%%%%%%%%%%%%%%%%%%%%%%%%%%%%%%%%%%%%%%%%%%%%%%%%%%%%%%%%%%%%%%%%%%%%%%%%%%%%%%

\usepackage{eurosym}
\usepackage{vmargin}
\usepackage{amsmath}
\usepackage{graphics}
\usepackage{epsfig}
\usepackage{enumerate}
\usepackage{multicol}
\usepackage{subfigure}
\usepackage{fancyhdr}
\usepackage{listings}
\usepackage{framed}
\usepackage{graphicx}
\usepackage{amsmath}
\usepackage{chngpage}

%\usepackage{bigints}
\usepackage{vmargin}

% left top textwidth textheight headheight

% headsep footheight footskip

\setmargins{2.0cm}{2.5cm}{16 cm}{22cm}{0.5cm}{0cm}{1cm}{1cm}

\renewcommand{\baselinestretch}{1.3}

\setcounter{MaxMatrixCols}{10}

\begin{document}
\begin{enumerate}
9 The following data (x) give the acidity (in appropriate units) of three different
varieties of grape.
Variety Mean Variance
A 8 7 18 15 12.0 28.7
B 90 74 200 122 121.5 3137.0
C 897 493 812 365 641.8 64284.9
A wine maker wants to test whether there are differences in the mean acidity level of
the three varieties and wishes to use analysis of variance (ANOVA) methodology.
(i) Explain why ANOVA should not be used for the data as given in the table
above. 
A statistician suggests two transformations of the original data:
  • the natural logarithm, y = ln(x),
• and the square root, z = x.
These give the following summary statistics:
  y =ln(x) z = x
Variety Mean Variance Mean Variance
A 2.4075 0.2136 3.3975 0.6046
B 4.7250 0.1892 10.8200 5.9242
C 6.4000 0.1800 24.9425 26.4567
The wine maker then decides to use the natural logarithm transformation (y) of the
original data.
(ii) Justify the wine maker’s choice of data transformation for performing the
analysis. 
(iii) Perform ANOVA on the transformed data, y, to investigate possible
differences in the mean acidity level of the three grape varieties and state your
conclusions. 
(iv) Calculate 95% confidence intervals for the mean values of each of the three
varieties on the original scale, based on the ANOVA performed on the
transformed values. 
(v) Comment on the intervals obtained in part (iv) in relation to your conclusion
in part (iii). 

\newpage

9 (i) The original values vary in scale among the 3 varieties, resulting in large
differences in the variances of the 3 groups. This violates the ANOVA
requirement that the error variance should not depend on the treatment
concerned. 
(ii) The logarithm transformation gives very similar variances for the 3 groups, as
opposed to the square root which still produces large differences. 
(iii) First calculate relevant sums:
  2.4075 4 9.63, 2 3 0.2136 4 2.40752 23.825 yA    yA     
4.725 4 18.9, 2 3 0.1892 4 4.7252 89.870 yB    yB     
6.4 4 25.6, 2 3 0.18 4 6.42 164.38 yC    yC     
SST = 23.825 + 89.87 + 164.38 – (9.63+18.9+25.6)2 / 12 = 33.9036
SSB = (9.632 +18.92 +25.62)/4 – (9.63+18.9+25.6)2 / 12 = 32.1553
SSR = SST – SSB = 1.7483
ANOVA table:
  Source of variation df SS MSS
Between groups 2 32.1553 16.0777
Residual 9 1.7483 0.1943
Total 11 33.9036
 (Probability and Mathematical Statistics) – September 2014 – 
Page 10
16.0777 82.75
0.1943
F  on 2, 9 df
F2,9(1%) = 8.022, so P-value << 0.01
There is overwhelming evidence against the null hypothesis. We conclude that
there are differences in the mean level of acidity of the three grape varieties.

(iv) The CIs are given by
yi  t9,0.975 ˆ / ni witht9,0.975 2.262 and ˆ  MSSR  0.44079
For A: 2.4075  2.262 0.44079 / 2 i.e. (1.909, 2.906)
and on the original scale: e1.909, e2.906  = (6.75, 18.28)
For B: 4.725  2.262 0.44079 / 2 i.e. (4.226, 5.224)
and on the original scale (68.44, 185.68)
For C: 6.4  2.262 0.44079 / 2 i.e. (5.901, 6.899)
and on the original scale (365.40, 990.28) 
(v) The CIs do not overlap. This agrees with the ANOVA conclusion, and in
addition shows differences between all 3 pairs of means. 

There were no problems with the ANOVA part of this question. However, the explanation in
part (i) was often unclear. In part (iv) some candidates failed to transform back to the
original scale.
