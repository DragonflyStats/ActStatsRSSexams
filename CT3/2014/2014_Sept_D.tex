8 The promoter of a touring dance show wishes to analyse how the price per ticket
affects the size of its audiences. She tests two prices, £14 and £16, over 10 shows
each which give rise to the following attendances.
Price
£14 120 115 130 127 124 110 121 129 118 122
£16 111 107 101 115 111 105 99 104 110 98
(i) Calculate the mean and standard deviation of the attendance for each sample.
[4]
(ii) Perform a statistical test to determine whether the variances of the attendance
are equal under the two prices. [3]
(iii) Perform a t-test to determine whether the mean attendance is the same under
the two prices. [4]
(iv) Calculate a 95% confidence interval for the difference between the mean show
revenue under each price. [4]
(v) Comment on which price the promoter should choose. [3]
[Total 18]
CT3 S2014–6


8 (i) Let X1,, X10 denote the sample at £14 and Y1,,Y10 the sample at £16.
xi 1216, xi2 148220
1216 148220 121.62 *10 121.6, 6.275
10 x 9 x s 
    
yi 1061, xi2 112863
1061 112863 106.12 *10 106.1, 5.685
10 y 9 y s 
     [4]
(ii) 2 2 2 2
H0 : x  y , H1 : x  y
Under 2 2
H0 sx / sy ~ F9,9
sx2 / s2y  6.2752 / 5.6852 1.22
9,9;0.975 9,9;0.025
1 0.25, 4.026
4.026
F   F  so we fail to reject H0 . [3]
(iii) Given (ii) we can assume that standard deviations are equal.
2 1 9*6.2752 9*5.6852  35.847
P 10 10 2 s   
 
test statistic = 121.6 106.1 15.5 5.789
2 5.987 1
P 10 5 s

 
test statistic ~ t10102  t18 = 2.101 at 2.5%.
So reject H0 : there is a significant difference between the means at 5%
significance level. [4]
Subject CT3 (Probability and Mathematical Statistics) – September 2014 – Examiners’ Report
Page 9
(iv) Difference in means = 14*121.6-16*106.1=4.8
2 1 9*142 *6.2752 9*162 *5.6852  7995.7
P 10 10 2 s   
 
Using t18 as before the confidence interval is
4.8 2.101* 7995.7 1 1  79.22,88.82
10 10
       
 
[4]
(v) There is a significant lower attendance with the higher price but, as the
confidence interval contains zero, no significant difference in revenues.
Financially it doesn’t matter which price the promoter chooses, but the lower
price would get more people to see the show. [3]
[Total 18]
Parts (i) – (iii) were well answered. In part (iv) some candidates did not realise that the
required CI referred to revenue. In the same part, there were also many errors in calculating
the common variance correctly. In part (v) other sensible comments were also given credit
as appropriate.
