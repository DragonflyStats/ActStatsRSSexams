
\documentclass[a4paper,12pt]{article}

%%%%%%%%%%%%%%%%%%%%%%%%%%%%%%%%%%%%%%%%%%%%%%%%%%%%%%%%%%%%%%%%%%%%%%%%%%%%%%%%%%%%%%%%%%%%%%%%%%%%%%%%%%%%%%%%%%%%%%%%%%%%%%%%%%%%%%%%%%%%%%%%%%%%%%%%%%%%%%%%%%%%%%%%%%%%%%%%%%%%%%%%%%%%%%%%%%%%%%%%%%%%%%%%%%%%%%%%%%%%%%%%%%%%%%%%%%%%%%%%%%%%%%%%%%%%

\usepackage{eurosym}
\usepackage{vmargin}
\usepackage{amsmath}
\usepackage{graphics}
\usepackage{epsfig}
\usepackage{enumerate}
\usepackage{multicol}
\usepackage{subfigure}
\usepackage{fancyhdr}
\usepackage{listings}
\usepackage{framed}
\usepackage{graphicx}
\usepackage{amsmath}
\usepackage{chngpage}

%\usepackage{bigints}
\usepackage{vmargin}

% left top textwidth textheight headheight

% headsep footheight footskip

\setmargins{2.0cm}{2.5cm}{16 cm}{22cm}{0.5cm}{0cm}{1cm}{1cm}

\renewcommand{\baselinestretch}{1.3}

\setcounter{MaxMatrixCols}{10}

\begin{document}
4 Consider six life policies, each on one of six independent lives. Each of four of the
policies has a probability of 2/3 of giving rise to a claim within the next five years,
and each of the other two policies has a probability of 1/3 of giving rise to a claim
within the next five years. It is assumed that only one claim can arise from each
policy.
\item (i) Calculate the expected number of claims which will arise from the six policies
within the next five years. 
\item (ii) Calculate the probability that exactly one claim will arise from the six policies
within the next five years. 
(iii) Calculate the probability that two policies chosen at random from the six
policies will both give rise to claims within the next five years. 
[Total 8]

%%%
%%%%%%%%%%%%%%%%%%%%%%%%%%%%%%%%%%%%%%%%%%%%%
4 If X is the total number of claims, with X1 from group 1 (G1, with probability 2/3) and
X2from group 2 (G2, with probability 1/3), we have
\item (i) X1~ Bin(4, 2/3) and X2~ Bin(2, 1/3) .
E(X )  E(X1  X2 )  E(X1)  E(X2 )
 4(2 / 3)  2(1/ 3) 10 / 3  3.333 
\item (ii) P(X 1)  P(X1 1, X2  0)  P(X1  0, X2 1)
3 0 2 0 4 1 1 4 2 4 2
(2/ 3)(1/ 3) (1/ 3) (2/ 3) (2/ 3) (1/ 3) (1/ 3) (2/ 3)
1 0 0 1
       
         
       
 4 / 81  0.0494 
(iii) P(two randomly selected policies giving claims) =
  P(both give claims | both from G1) * P(both from G1)
+P(both give claims | both from G2) * P(both from G2)
+ 2*P(both give claims | one from G1, one from G2) * P(one from G1, one
                                                       from G2)
2 2 4 3 1 2 2 1 2 1 4 2 41 2 0.3037
3 6 5 3 6 5 3 3 6 5 135
                   
      

[Total 8]
Mixed performance. Parts (i) and (ii) were answered well, but there were many inadequate
attempts in part (iii). In many cases candidates failed to see the different combinations
resulting in the required event, while there were also problems in calculating the correct
probability for each combination.
\end{document}
