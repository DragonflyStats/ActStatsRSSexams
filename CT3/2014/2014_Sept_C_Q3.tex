
\documentclass[a4paper,12pt]{article}

%%%%%%%%%%%%%%%%%%%%%%%%%%%%%%%%%%%%%%%%%%%%%%%%%%%%%%%%%%%%%%%%%%%%%%%%%%%%%%%%%%%%%%%%%%%%%%%%%%%%%%%%%%%%%%%%%%%%%%%%%%%%%%%%%%%%%%%%%%%%%%%%%%%%%%%%%%%%%%%%%%%%%%%%%%%%%%%%%%%%%%%%%%%%%%%%%%%%%%%%%%%%%%%%%%%%%%%%%%%%%%%%%%%%%%%%%%%%%%%%%%%%%%%%%%%%

\usepackage{eurosym}
\usepackage{vmargin}
\usepackage{amsmath}
\usepackage{graphics}
\usepackage{epsfig}
\usepackage{enumerate}
\usepackage{multicol}
\usepackage{subfigure}
\usepackage{fancyhdr}
\usepackage{listings}
\usepackage{framed}
\usepackage{graphicx}
\usepackage{amsmath}
\usepackage{chngpage}

%\usepackage{bigints}
\usepackage{vmargin}

% left top textwidth textheight headheight

% headsep footheight footskip

\setmargins{2.0cm}{2.5cm}{16 cm}{22cm}{0.5cm}{0cm}{1cm}{1cm}

\renewcommand{\baselinestretch}{1.3}

\setcounter{MaxMatrixCols}{10}

\begin{document}

3 Let N be a random variable describing the number of withdrawals from a bank branch
each day. It is assumed that N is Poisson distributed with mean μ. Let Xi, the random
variable describing the amount of each withdrawal, be exponentially distributed with
mean 1/λ. All Xi are independent and identically distributed. Let S denote the total
amount withdrawn from that branch in a day i.e.
1
N
i
i
S X
=
  = Σ
with S = 0 if N = 0.
\item (i) Derive the moment generating function of S. 
\item (ii) Calculate the mean and variance of S if μ = 100 and λ = 0.025. 
[Total 7]

%%%%%%%%%%%%%%%%%%%%%%%%%%%%%%%%%%%%%%%%

3 \item (i) tS | exp\left{  (  1 2 ) \right}|
  E \left[e N \;=;\ n\right]  \;=;\ E \left[ t X  X  XN N \;=;\ n\right] 
\left{  (  1 2 ) \right} (  )  \left{  (  ) \right}
1
exp exp 1
n n n
n i X
i
E t X X X E tX M t t

\;=;\
\;=;\ \left[    \right] \;=;\ \left[  \right] \;=;\ \;=;\             
(  )  (  tS ) 

%%%%%%%%%%%%%%%%%%%%%%%%%%%%%
Therefore

\begin{eqnarray*}
M_S(t) 
&=&  E etS\\
&=&  E \left[ E( etS |N) \right]\\
&=&  E \left[ \left{  (MX t  ) \right}N \right]\\
&=&  E \left[exp\left{ N logMX ( t ) \right}\right] \\
&=&  M_{N} \left{  -\log \left( 1 - \frac{t}{\lambda} right)  \right}\\
&=& exp \left[ \mu \left{ \left{  -\log \left( 1 - \frac{t}{\lambda} right)^{-1} \;-\;1 \right} \right] \\  
\end{eqnarray*}

%%%%%%%%%%%%%%%%%%%%%%%%%%%%%%%%%%%%%%%%%%%%%%%5

\item (ii)
\[E Xi \;=;\  \frac{1}{0.025} \;=;\ 40 \]
\[E \left[Xi \right]  \;=;\ \frac{1}{(0.025)^2} +40^2 \;=;\  3200 \]
\[ E(S)  \;=;\ \mu E\[ Xi ] \;=;\100*40 \;=;\ 4000 \]
\[V (S)  \;=;\\mu E \left[Xi^2 \right]  \;=;\100*3200 \;=;\ 320,000\]

(OR
  2 2
  2
  100* 1 100*40 320,000
  0.025
  V S \;=;\ E N V Xi V N \left[E X \right]  \;=;\  \;=;\ )

[Total 7]
Part \item (i) required careful and precise derivation of the result, and many candidates struggled
with it. Answers in questions involving work with MGF expressions have also been
problematic in the past – more practice and better understanding is needed.
Subject CT3 (Probability and Mathematical Statistics) – September 2014 
