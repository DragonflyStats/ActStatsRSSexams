\documentclass[a4paper,12pt]{article}

%%%%%%%%%%%%%%%%%%%%%%%%%%%%%%%%%%%%%%%%%%%%%%%%%%%%%%%%%%%%%%%%%%%%%%%%%%%%%%%%%%%%%%%%%%%%%%%%%%%%%%%%%%%%%%%%%%%%%%%%%%%%%%%%%%%%%%%%%%%%%%%%%%%%%%%%%%%%%%%%%%%%%%%%%%%%%%%%%%%%%%%%%%%%%%%%%%%%%%%%%%%%%%%%%%%%%%%%%%%%%%%%%%%%%%%%%%%%%%%%%%%%%%%%%%%%

\usepackage{eurosym}
\usepackage{vmargin}
\usepackage{amsmath}
\usepackage{graphics}
\usepackage{epsfig}
\usepackage{enumerate}
\usepackage{multicol}
\usepackage{subfigure}
\usepackage{fancyhdr}
\usepackage{listings}
\usepackage{framed}
\usepackage{graphicx}
\usepackage{amsmath}
\usepackage{chngpage}

%\usepackage{bigints}
\usepackage{vmargin}

% left top textwidth textheight headheight

% headsep footheight footskip

\setmargins{2.0cm}{2.5cm}{16 cm}{22cm}{0.5cm}{0cm}{1cm}{1cm}

\renewcommand{\baselinestretch}{1.3}

\setcounter{MaxMatrixCols}{10}

\begin{document}
\begin{enumerate}
7 Consider the following discrete distribution with an unknown parameter p for the
distribution of the number of policies with 0, 1, 2, or more than 2 claims per year in a
portfolio of n independent policies.
number of claims 0 1 2 more than 2
probability 2 p p 0.25 p 1− 3.25 p
We denote by X0 the number of policies with no claims, by X1 the number of policies
with one claim and by X2 the number of policies with two claims per year. The
random variable X = X0 + X1 + X2 is then the number of policies with at most two
claims.
(i) Derive an expression for the maximum likelihood estimator ˆp of parameter p
in terms of X and n. [5]
(ii) Show that the estimator obtained in part (i) is unbiased. 
The following frequencies are observed in a portfolio of n = 200 policies during the
year 2012:
  number of claims 0 1 2 more than 2
observed frequency 123 58 13 6
A statistician proposes that the parameter p can be estimated by 􀀄p= 58/200 = 0.29
since p is the probability that a randomly chosen policy leads to one claim per year.
(iii) Estimate the parameter p using the estimator derived in part (i). 
(iv) Explain why your answer to part (iii) is different from the proposed estimated
value of 0.29. 
An alternative model is proposed where the probability function has the form
number of claims 0 1 2 more than 2
probability p 2 p 0.25 p 1− 3.25 p
(v) Explain how the maximum likelihood estimator suggested in part (i) needs to
be adapted to estimate the parameter p in this new model. 
(vi) Suggest a suitable test to use to make a decision about which of the two
models should be used based on empirical data. 
[Total 13]
CT3 S2014–5 
\newpage


7 We denote by X0 the number of policies with no claims, by X1 the number of
policies with one claim and by X2 the number of policies with two claims per year.
Let X  X0 + X1  X2
(i) Likelihood function
L p 2 pX0 pX1 0.25 pX2 1 3.25 pn X   
Log-likelihood
l  p  X0 log 2 p  X1 log  p  X2 log 0.25 p  n  X log 1 3.25 p
+ constant
    0 1 2 3.25 3.25
1 3.25 1 3.25
dl X X X n X X n X
dp p p p p p p
 
     
 
 – September 2014 – %%%%%%%%%%%%%%%%%%%%%%%%%%%%%%%%%%%%%%%%%
Page 7
dl 0
dp
 gives X 1 3.25 p  3.25n  X  p =0
X  3.25Xp  3.25np  3.25Xp  X  3.25np  0
3.25
pˆ X
n

[Alternative solution:
   Set   3.25 p to be the probability of at most two claims.
 L  X 1 n X       andl   X ln  n  X ln 1   constant
  
 1
 dl X n X
 d
 
  
   
 and setting equal to zero: ˆ X
 n
   .
 Using the invariance property of the MLE we obtain:
   ˆ 3.25
 3 25
 ˆ ˆ
 .
 p p X
 n
     ] [5]
(ii) 􀜧􁈾 ] 1  
3.25
ˆp E X
n
 , X has Binomial dist. with parameters n and
2 p  p  0.25 p
EX   n2 p  p  0.25 p  3.25 pn
and therefore E pˆ   p 
(iii) X  194 , ˆ 194 0.2985
3.25 200
p 


(iv) The MLE in part (iii) takes the structure of the entire probability function into
account while the estimator 58/200 only considers the number of policies with
one claim. 
(v) No change required, since the MLE ˆp turns out to dependent only on the total
number of policies with less than three claims. 
(vi) 2 -test 
[Total 13]
The later parts of the question were well answered. However there was a considerable
number of poor answers in parts (i) and (ii). Part (i) particularly, deals with the likelihood
concept which is fundamental in statistics. The setting does not refer explicitly to a usual
 – September 2014 – %%%%%%%%%%%%%%%%%%%%%%%%%%%%%%%%%%%%%%%%%
Page 8
distribution, but involves a standard model, and candidates at this level need to make sure
that they can work with the likelihood function in a variety of standard models.a
\end{document}
