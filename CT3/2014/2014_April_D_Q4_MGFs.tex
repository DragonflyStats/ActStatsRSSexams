\documentclass[a4paper,12pt]{article}

%%%%%%%%%%%%%%%%%%%%%%%%%%%%%%%%%%%%%%%%%%%%%%%%%%%%%%%%%%%%%%%%%%%%%%%%%%%%%%%%%%%%%%%%%%%%%%%%%%%%%%%%%%%%%%%%%%%%%%%%%%%%%%%%%%%%%%%%%%%%%%%%%%%%%%%%%%%%%%%%%%%%%%%%%%%%%%%%%%%%%%%%%%%%%%%%%%%%%%%%%%%%%%%%%%%%%%%%%%%%%%%%%%%%%%%%%%%%%%%%%%%%%%%%%%%%

\usepackage{eurosym}
\usepackage{vmargin}
\usepackage{amsmath}
\usepackage{graphics}
\usepackage{epsfig}
\usepackage{enumerate}
\usepackage{multicol}
\usepackage{subfigure}
\usepackage{fancyhdr}
\usepackage{listings}
\usepackage{framed}
\usepackage{graphicx}
\usepackage{amsmath}
\usepackage{chngpage}

%\usepackage{bigints}
\usepackage{vmargin}

% left top textwidth textheight headheight

% headsep footheight footskip

\setmargins{2.0cm}{2.5cm}{16 cm}{22cm}{0.5cm}{0cm}{1cm}{1cm}

\renewcommand{\baselinestretch}{1.3}

\setcounter{MaxMatrixCols}{10}

\begin{document}
\begin{enumerate}Let $X$ be a random variable with probability density function:

\[ f(x) =  \begin{cases}  
\frac{1}{2} e^{x} & x \leq 0 \\
\frac{1}{2} e^{-x}  & x > 0 \\     \end{cases}\]


(i)

Show that the moment generating function of X is given by:
\[M_X ( t ) = (1 − t^{2}  )^{ − 1} ,\]
for $ |t| < 1$ .

(ii)

Hence find the mean and the variance of X using the moment generating
function in part (i).

%%%%%%%%%%%%%%%%%%%%%%%%%%%%%%%%%%%%%%%%%%%%%%%%%%%%%%%%%%%%%%%%%%%%%%%%%%%%%%%%%
4
(i)
1
M_X ( t ) \;=\; E ( e^{tx} ) \;=\;
2
tX
1  e ( t \;-\; 1) x 
\;=\; 

2   t \;-\; 1  
0
\;-\;
0
 e
( t \;-\; 1) x
\;-\;
1  e ( t \;-\; 1) x 
\;-\; 

2   t \;-\; 1  

1
dx \;-\;  e ( t \;-\; 1) x dx
2
0

0
and for  t  1
M X ( t ) \;=\;
%%%%%%%%%%%%%%%%%%%%%%%%%%%%%%%%%%%%%%%%%%%%%%%%%%%%%%%%%%%%%%%%%%%%%
%% --- Page 3 (Probability and Mathematical Statistics) – April 2014 – 
%%%%%%%%%%%%%%%%%%%%%%%%%%%%%%%%%%%%%%%%%%%%%%%%%%%%%%%%%%%%%%%%%%%%%%
(ii)
1  1
1 
1
\;-\;

 \;=\;
2  t \;-\; 1 t \;-\; 1  1 \;-\; t^{2} 

M  X ( t ) \;=\; (1 \;-\; t^{2}  ) \;-\; 1 \;=\; \;-\; (1 \;-\; t^{2}  ) \;-\; 2 ( \;-\; 2 t ) \;=\; 2 t (1 \;-\; t^{2}  ) \;-\; 2


\[ E ( X ) \;=\; M  X (0) \;=\; 0 \]


%--------------------------------%
M^{\prime\prime} X ( t ) 
&=&  2 t (1 \;-\; t^{2}  ) \;-\; 2  \\
&=&  2(1 \;-\; t2 \;-\; 2)   \\
\;-\; 2 t ( \;-\; 2)(1 \;-\; t^{2}  ) \;-\; 3 ( \;-\; 2 t )
\;=\; 2(1 \;-\; t^{2}  ) \;-\; 2 \;-\; 8 t^{2}  (1 \;-\; t^{2}  ) \;-\; 3

\[ E ( X 2 ) \;=\; M^{\prime\prime} X (0) \;=\; 2  \]
\[V ( X ) \;=\; E ( X^2 ) \;-\; E^2 ( X ) \;=\; 2\]
(Alternatively, based on a series expansion:
M X ( t ) \;=\; 1 \;-\; t^{2}  \;-\; t 4 \;-\; ...  E ( X ) \;=\; 0 and E ( X 2 ) \;=\; 2 and the variance follows.)

%%-- Generally well answered. In part (ii) most candidates were familiar with the method, but some showed poor differentiation skills.
%%%%%%%%%%%%%%%%%%%%%%%%%%%%%%%%%%%%%%%%%%%%%%%%%%

\end{document}
