\documentclass[a4paper,12pt]{article}

%%%%%%%%%%%%%%%%%%%%%%%%%%%%%%%%%%%%%%%%%%%%%%%%%%%%%%%%%%%%%%%%%%%%%%%%%%%%%%%%%%%%%%%%%%%%%%%%%%%%%%%%%%%%%%%%%%%%%%%%%%%%%%%%%%%%%%%%%%%%%%%%%%%%%%%%%%%%%%%%%%%%%%%%%%%%%%%%%%%%%%%%%%%%%%%%%%%%%%%%%%%%%%%%%%%%%%%%%%%%%%%%%%%%%%%%%%%%%%%%%%%%%%%%%%%%

\usepackage{eurosym}
\usepackage{vmargin}
\usepackage{amsmath}
\usepackage{graphics}
\usepackage{epsfig}
\usepackage{enumerate}
\usepackage{multicol}
\usepackage{subfigure}
\usepackage{fancyhdr}
\usepackage{listings}
\usepackage{framed}
\usepackage{graphicx}
\usepackage{amsmath}
\usepackage{chngpage}

%\usepackage{bigints}
\usepackage{vmargin}

% left top textwidth textheight headheight

% headsep footheight footskip

\setmargins{2.0cm}{2.5cm}{16 cm}{22cm}{0.5cm}{0cm}{1cm}{1cm}

\renewcommand{\baselinestretch}{1.3}

\setcounter{MaxMatrixCols}{10}

\begin{document}
\begin{enumerate}
%%-- 10
\item An analyst is instructed to investigate the relationship between the size of a bond issue and its trading volumes (value traded). The data for 33 bonds are plotted in the following chart.

\begin{enumerate}[(i)]
\item (i)
Comment on the relationship between issue size and value traded. 
The analyst denotes issue size by s and monthly value traded by v . He calculates the following from the data:
∑ s i = 2,843.7, ∑ s i 2 = 397, 499.8, ∑ v i = 115.34, ∑ v i 2 = 689.37, ∑ s i v i = 15, 417.75
\item (ii)
(a) Determine the correlation coefficient between s and v .
%%%%%%%%%%%%%%%%%%%%%%%%%%%%%%%%%%%%%%%%%%%%%%%%%%%%%%%%%%%%%%%%%%%%%%%%%%%%%%%%%%%%%%%%%%%%%%%%%%%%%%%%%%%%%%
(b) Perform a statistical test to determine if the correlation coefficient is significantly different from 0.

\item
(iii) Determine the parameters of a linear regression of v on s and state the fitted model equation.

\item
(iv) State the outcome of a statistical test to determine whether the slope parameter in part (iii) differs significantly from zero, justifying your answer. A colleague suggests that the central part of the data, with issue sizes between £50m
and £150m, seem to have a greater spread of value traded and without the bonds in
the upper and lower tails the linear relationship would be much weaker.
\item
(v) Comment on the colleague’s observation.
%%%%%%%%%%%%%%%%%%%%%%%%%%%%%%%%%%%%%%%%%%%%%%%%%%%%%%%%%%%%%%%%%%%%%%%%%%%%%%%%%%%%%%%%%%%%%%%%%%%%%%%%%%%%%
\end{enumerate}

\newpage
10
(i) There appears to be a positive linear relationship
(ii) (a)
S ss   s i 2   
   s i 
S vv   v i 2   
   v i 
S vs   v i s i     v  s   / n  15417.75   2843.7  115.34  / 33  5478.6
r 
(ii)
(b)
2 
 / n  397499.8   2843.7  / 33  152450.4

2
2 
 / n  689.37   115.34  / 33  286.24

i
2
i
S vs
5478.6

 0.8294
152450.4  286.24
S ss S vv
H 0 : r  0, H 1 : r  0
Test statistic = r
n  2
1  r
2

0.8294 33  2
1  0.8294
2
 8.266
At 0.5% level t 31  2.744 which << test statistic
So reject H 0 .
(iii)

S vs
5478.6

 0.0359
S ss 152450.4
  v   s 
115.34
2843.7
 0.0359
 0.398
33
33
v i  0.398  0.0359 s i

(iv)
Testing whether  is significantly different from zero is mathematically the same as testing whether the correlation coefficient is significantly different from zero.
As H 0 was rejected in (ii)(b), we can conclude testing H 0 :   0 would give
the same result.

(v)
It is true that extreme observations can determine the strength of a linear relationship. However, there are many more bonds in the central part of the data and we would consequently expect a greater range of value traded.
Generally well answered. In part (ii)(b) Fisher’s z transformation method was also allowed.
In part (v) other possible reasonable comments were given credit.
\end{document}
