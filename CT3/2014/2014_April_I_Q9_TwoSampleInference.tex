\documentclass[a4paper,12pt]{article}

%%%%%%%%%%%%%%%%%%%%%%%%%%%%%%%%%%%%%%%%%%%%%%%%%%%%%%%%%%%%%%%%%%%%%%%%%%%%%%%%%%%%%%%%%%%%%%%%%%%%%%%%%%%%%%%%%%%%%%%%%%%%%%%%%%%%%%%%%%%%%%%%%%%%%%%%%%%%%%%%%%%%%%%%%%%%%%%%%%%%%%%%%%%%%%%%%%%%%%%%%%%%%%%%%%%%%%%%%%%%%%%%%%%%%%%%%%%%%%%%%%%%%%%%%%%%

\usepackage{eurosym}
\usepackage{vmargin}
\usepackage{amsmath}
\usepackage{graphics}
\usepackage{epsfig}
\usepackage{enumerate}
\usepackage{multicol}
\usepackage{subfigure}
\usepackage{fancyhdr}
\usepackage{listings}
\usepackage{framed}
\usepackage{graphicx}
\usepackage{amsmath}
\usepackage{chngpage}

%\usepackage{bigints}
\usepackage{vmargin}

% left top textwidth textheight headheight

% headsep footheight footskip

\setmargins{2.0cm}{2.5cm}{16 cm}{22cm}{0.5cm}{0cm}{1cm}{1cm}

\renewcommand{\baselinestretch}{1.3}

\setcounter{MaxMatrixCols}{10}

\begin{document}

%% CT3 A2014–4
%% Question 9
The weekly amount spent on childcare for one child is believed to depend on the age of the child. 
\begin{itemize}
    \item We denote by X the random variable describing the cost per child for a randomly selected child of age one year, Y being the cost for a three year old child, and Z the cost for a five year old child. 
    \item It is assumed that X, Y, and Z are normally distributed and that childcare costs are independent between children.
\end{itemize} Random samples of children of different ages are taken and the weekly childcare costs are
recorded during the year 2012. A summary of the data is given in the following table:
\begin{center}
\begin{tabular}{cccc}
Random variable &  X &  Y &  Z \\ \hline
Age of child & 1 & 3 &  5 \\ \hline
Average cost per week per child & 200 & 170 & 155 \\ \hline
Sample standard deviation &  30 & 30 & 20 \\ \hline
Sample size & 25 & 25 & 25 \\ \hline
\end{tabular}
\end{center}

\begin{enumerate}[(a)]
\item (i)
Calculate the overall average weekly cost of childcare per child for the
children in these samples.

\item (ii) Calculate a 95\% confidence interval for the expected childcare cost for a child
aged one year.

\item (iii) Calculate a 95\% confidence interval for the expected childcare cost for a child
aged five years.

\item (iv) Calculate a 95\% confidence interval for the ratio of the variances of X and Z .

\item (v) Perform a test at 5\% significance level for the null hypothesis that the
variances of X and Z are equal based on your answer to part (iv).

\item (vi) Calculate an approximate 95\% confidence interval for the difference between
the average weekly childcare cost per child for children aged one and for
children aged five. Justify any assumptions that you make and explain any
approximate values you use.

\item (vii) Perform a test to decide if there is a difference between the expected weekly
childcare cost per child spent for children aged one and for children aged five
based on your answer to part (vi).

\item (viii) Perform an analysis of variance to decide if the age of a child has an impact on
the weekly amount spent on childcare.
\end{enumerate}

%%%%%%%%%%%%%%%%%%%%%%%%%%%%%%%%%%%%%%%%%%%%%%%%%%%%%%%%%%%%%%%%%%%%%%%%%%%%%%%%%%%%%%%%%%%%%%%%%%%%%%%%%%%%%
\newpage
%%- Quesiton 9
\begin{itemize}
\item (i) Overall average is (200 + 170 + 155)/3 = 175 since sample sizes are all equal.
\item (ii) X  t 0.025, 24
30
  200  2.064*6, 200  2.064*6    187.62, 212.38 
5
\item (iii) Z  t 0.025, 24
20
  155  2.064 * 4,1 55  2.064 * 4    146.74, 1 63.26 
5
\item (iv)  S X 2
  2.25
S X 2
1



,
, 2.25  2.269    0.992, 5.105 
F
 2
  
24,24
2

  S Z F 24,24 S Z
   2.269
(v)
\item (vi)
The ratio 1 is contained in the confidence interval, therefore the null
hypothesis  2 X   2 Z cannot be rejected.
Pooled variance:
s 2 p


24  30 2  20 2
48
  650 .
Difference: 200  155 = 45

2
2 
, 45  t 0.025,48 650
 45  t 0.025,48 650

25
25 

 45  2.01  7.21, 45  2.01  7.21    30.51, 59.49 
where we have used the approximation t 0.025,48  2.01 (see tables, value for
t 0.025,50  2.009 )
\item We made the assumption  2 X   2 Z which is justified by the result in parts (iv)
and (v).
\item (vii) The confidence interval does not contain 0, so there is a difference.
\item (viii) SS R  24  30 2  30 2  20 2  52800


%%%%%%%%%%%%%%%%%%%%%%%%%%%%%%%%%%%%%%%%%%%%%%%%%%%%%%%%%%%%%%%%%%%%%%%%%%%%
\item Alternative solution possible
2
2
2
SS B  25     200  175    170  175    155  175     26250


\end{itemize}
Page 8Subject CT3 (Probability and Mathematical Statistics) – April 2014 
SS B / 2 13125

 17.9
SS R / 72 733.33
This is clearly a very large value compared to F 2,72  F 2,60  4.977 at the 1%
level, so the age of the child has an impact on childcare cost.
Generally well answered. In part (iv) calculation of the ratio of the variance of Z over the
variance of X was given full credit. In part (viii) many candidates attempted to calculate the
SS values using the original data, rather than the “quick” formulae given in the answer. This
was given full marks where appropriate, but was not the best use of time in the exam.
\end{document}
