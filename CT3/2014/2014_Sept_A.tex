1 A sample of marks from an exam has median 49 and interquartile range 19. The
marks are rescaled by multiplying by 1.2 and adding 6.
Calculate the new median and interquartile range. [4]
2 Consider an insurer that offers two types of policy: home insurance and car insurance.
70% of all customers have a home insurance policy, and 80% of all customers have a
car insurance policy. Every customer has at least one of the two types of policies.
Calculate the probability that a randomly selected customer:
  (i) does not have a car insurance policy. [1]
(ii) has car insurance and home insurance. [1]
(iii) has home insurance, given that he has car insurance. [2]
(iv) does not have car insurance, given that he has home insurance. [2]
[Total 6]
3 Let N be a random variable describing the number of withdrawals from a bank branch
each day. It is assumed that N is Poisson distributed with mean μ. Let Xi, the random
variable describing the amount of each withdrawal, be exponentially distributed with
mean 1/λ. All Xi are independent and identically distributed. Let S denote the total
amount withdrawn from that branch in a day i.e.
1
N
i
i
S X
=
  = Σ
with S = 0 if N = 0.
(i) Derive the moment generating function of S. [4]
(ii) Calculate the mean and variance of S if μ = 100 and λ = 0.025. [3]
[Total 7]
4 Consider six life policies, each on one of six independent lives. Each of four of the
policies has a probability of 2/3 of giving rise to a claim within the next five years,
and each of the other two policies has a probability of 1/3 of giving rise to a claim
within the next five years. It is assumed that only one claim can arise from each
policy.
(i) Calculate the expected number of claims which will arise from the six policies
within the next five years. [2]
(ii) Calculate the probability that exactly one claim will arise from the six policies
within the next five years. [2]
(iii) Calculate the probability that two policies chosen at random from the six
policies will both give rise to claims within the next five years. [4]
[Total 8]

%%%
1 Let X1,, Xn be the existing marks and Y1,,Yn denote the transformed marks.
Then Yi 1.2Xi  6.
Median is n 1 / 2 th observation so same transformation applies to median.
New median = 49*1.2 + 6 = 64.8.
As for the median, each transformed quartile, QYi 1.2QXi  6. Then the new
interquartile range is, IQRY  QY3 QY1 1.2QX3  6  1.2QX1  6 1.2IQRX .
New IQR = 19*1.2 = 22.8. [4]
Generally well answered.
2 Note that each customer has at least one contract, that is, PCar Home 1.
(i) P CarC  1  PCar  0.2  20% [1]
(ii) PCar  Home  PCar PHome PCar Home [1]
 0.8  0.7 1  0.5
(iii)    
 
Car Home 0.5 Home|Car 0.625
Car 0.8
P
P
P
    [2]
(iv) P  CarC Home  PHome  PCarHome  0.7  0.5  0.2
 
Car Home 0.2 Car |Home 0.2857
Home 0.7
C
P C
P
P
            [2]
[Total 6]
Reasonably well done, with the exception of part (iv). Note that events are not independent
here. Alternative ways to arrive at the correct answer were given full credit.
Subject CT3 (Probability and Mathematical Statistics) – September 2014 – Examiners’ Report
Page 4
3 (i) tS | exp  1 2 |
  E e N  n  E  t X  X  XN N  n
  1 2      
1
exp exp 1
n n n
n i X
i
E t X X X E tX M t t


                      
   tS 
MS t  E e
 E EetS |N
 
  N
E MX t   
 
 E expN logMX t 
N log 1
M t
           
1
exp 1 t 1               
[4]
(ii)  
 
2 2
2
1 40, 1 40 3200
0.025 0.025
E Xi   E Xi    
ES EXi  100*40  4000
V S E Xi2  100*3200  320,000
(OR
           
   
  2 2
  2
  100* 1 100*40 320,000
  0.025
  V S  E N V Xi V N E X     )
[3]
[Total 7]
Part (i) required careful and precise derivation of the result, and many candidates struggled
with it. Answers in questions involving work with MGF expressions have also been
problematic in the past – more practice and better understanding is needed.
Subject CT3 (Probability and Mathematical Statistics) – September 2014 – Examiners’ Report
Page 5
4 If X is the total number of claims, with X1 from group 1 (G1, with probability 2/3) and
X2from group 2 (G2, with probability 1/3), we have
(i) X1~ Bin(4, 2/3) and X2~ Bin(2, 1/3) .
E(X )  E(X1  X2 )  E(X1)  E(X2 )
 4(2 / 3)  2(1/ 3) 10 / 3  3.333 [2]
(ii) P(X 1)  P(X1 1, X2  0)  P(X1  0, X2 1)
3 0 2 0 4 1 1 4 2 4 2
(2/ 3)(1/ 3) (1/ 3) (2/ 3) (2/ 3) (1/ 3) (1/ 3) (2/ 3)
1 0 0 1
       
         
       
 4 / 81  0.0494 [2]
(iii) P(two randomly selected policies giving claims) =
  P(both give claims | both from G1) * P(both from G1)
+P(both give claims | both from G2) * P(both from G2)
+ 2*P(both give claims | one from G1, one from G2) * P(one from G1, one
                                                       from G2)
2 2 4 3 1 2 2 1 2 1 4 2 41 2 0.3037
3 6 5 3 6 5 3 3 6 5 135
                   
      
[4]
[Total 8]
Mixed performance. Parts (i) and (ii) were answered well, but there were many inadequate
attempts in part (iii). In many cases candidates failed to see the different combinations
resulting in the required event, while there were also problems in calculating the correct
probability for each combination.
