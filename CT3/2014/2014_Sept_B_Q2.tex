
\documentclass[a4paper,12pt]{article}

%%%%%%%%%%%%%%%%%%%%%%%%%%%%%%%%%%%%%%%%%%%%%%%%%%%%%%%%%%%%%%%%%%%%%%%%%%%%%%%%%%%%%%%%%%%%%%%%%%%%%%%%%%%%%%%%%%%%%%%%%%%%%%%%%%%%%%%%%%%%%%%%%%%%%%%%%%%%%%%%%%%%%%%%%%%%%%%%%%%%%%%%%%%%%%%%%%%%%%%%%%%%%%%%%%%%%%%%%%%%%%%%%%%%%%%%%%%%%%%%%%%%%%%%%%%%

\usepackage{eurosym}
\usepackage{vmargin}
\usepackage{amsmath}
\usepackage{graphics}
\usepackage{epsfig}
\usepackage{enumerate}
\usepackage{multicol}
\usepackage{subfigure}
\usepackage{fancyhdr}
\usepackage{listings}
\usepackage{framed}
\usepackage{graphicx}
\usepackage{amsmath}
\usepackage{chngpage}

%\usepackage{bigints}
\usepackage{vmargin}

% left top textwidth textheight headheight

% headsep footheight footskip

\setmargins{2.0cm}{2.5cm}{16 cm}{22cm}{0.5cm}{0cm}{1cm}{1cm}

\renewcommand{\baselinestretch}{1.3}

\setcounter{MaxMatrixCols}{10}

\begin{document}

70% of all customers have a home insurance policy, and 80% of all customers have a
car insurance policy. Every customer has at least one of the two types of policies.
Calculate the probability that a randomly selected customer:
  \item (i) does not have a car insurance policy. 
\item (ii) has car insurance and home insurance. 
\item (iii) has home insurance, given that he has car insurance. 
\item (iv) does not have car insurance, given that he has home insurance. 

%%%

2 Note that each customer has at least one contract, that is, PCar Home 1.
\item (i) P CarC  1  PCar  0.2  20% 
\item (ii) PCar  Home  PCar PHome PCar Home 
 0.8  0.7 1  0.5
(iii)    
 
Car Home 0.5 Home|Car 0.625
Car 0.8
P
P
P
    
(iv) P  CarC Home  PHome  PCarHome  0.7  0.5  0.2
 
Car Home 0.2 Car |Home 0.2857
Home 0.7
C
P C
P
P
            
[Total 6]
Reasonably well done, with the exception of part (iv). Note that events are not independent
here. Alternative ways to arrive at the correct answer were given full credit.
Subject CT3 (Probability and Mathematical Statistics) – September 2014 
