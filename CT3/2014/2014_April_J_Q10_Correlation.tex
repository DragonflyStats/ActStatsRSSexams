\documentclass[a4paper,12pt]{article}

%%%%%%%%%%%%%%%%%%%%%%%%%%%%%%%%%%%%%%%%%%%%%%%%%%%%%%%%%%%%%%%%%%%%%%%%%%%%%%%%%%%%%%%%%%%%%%%%%%%%%%%%%%%%%%%%%%%%%%%%%%%%%%%%%%%%%%%%%%%%%%%%%%%%%%%%%%%%%%%%%%%%%%%%%%%%%%%%%%%%%%%%%%%%%%%%%%%%%%%%%%%%%%%%%%%%%%%%%%%%%%%%%%%%%%%%%%%%%%%%%%%%%%%%%%%%

\usepackage{eurosym}
\usepackage{vmargin}
\usepackage{amsmath}
\usepackage{graphics}
\usepackage{epsfig}
\usepackage{enumerate}
\usepackage{multicol}
\usepackage{subfigure}
\usepackage{fancyhdr}
\usepackage{listings}
\usepackage{framed}
\usepackage{graphicx}
\usepackage{amsmath}
\usepackage{chngpage}

%\usepackage{bigints}
\usepackage{vmargin}

% left top textwidth textheight headheight

% headsep footheight footskip

\setmargins{2.0cm}{2.5cm}{16 cm}{22cm}{0.5cm}{0cm}{1cm}{1cm}

\renewcommand{\baselinestretch}{1.3}

\setcounter{MaxMatrixCols}{10}

\begin{document}
\begin{enumerate}
%%-- 10
\item An analyst is instructed to investigate the relationship between the size of a bond issue and its trading volumes (value traded). The data for 33 bonds are plotted in the following chart.

\begin{enumerate}[(i)]
\item (i)
Comment on the relationship between issue size and value traded. 
The analyst denotes issue size by s and monthly value traded by v . He calculates the following from the data:
\[\sum  s i = 2,843.7, \sum  s i 2 = 397, 499.8, \sum  v i = 115.34, \sum  v i 2 = 689.37, \sum  s i v i = 15, 417.75\]
\item (ii)
(a) Determine the correlation coefficient between s and v .
%%%%%%%%%%%%%%%%%%%%%%%%%%%%%%%%%%%%%%%%%%%%%%%%%%%%%%%%%%%%%%%%%%%%%%%%%%%%%%%%%%%%%%%%%%%%%%%%%%%%%%%%%%%%%%
(b) Perform a statistical test to determine if the correlation coefficient is significantly different from 0.

\item
(iii) Determine the parameters of a linear regression of v on s and state the fitted model equation.

\item
(iv) State the outcome of a statistical test to determine whether the slope parameter in part (iii) differs significantly from zero, justifying your answer. A colleague suggests that the central part of the data, with issue sizes between \$50m
and \$150m, seem to have a greater spread of value traded and without the bonds in
the upper and lower tails the linear relationship would be much weaker.
\item
(v) Comment on the colleague’s observation.
%%%%%%%%%%%%%%%%%%%%%%%%%%%%%%%%%%%%%%%%%%%%%%%%%%%%%%%%%%%%%%%%%%%%%%%%%%%%%%%%%%%%%%%%%%%%%%%%%%%%%%%%%%%%%
\end{enumerate}

\newpage
%%- Question 10
\begin{itemize}
\item (i) There appears to be a positive linear relationship
\item (ii) (a)
\begin{eqnarray*}
S_{ss} &=& \sum s_i^2 - \frac{(\sum s_i)^2 }{n} \\&=& 397499.8 - \frac{(2843.7)^2}{33} \\
&=& 152450.4 \\
\end{eqnarray*}

\begin{eqnarray*}
S_{vv} &=& \sum v_i^2 - \frac{(\sum v_i)^2 }{n} \\&=& 689.37 - \frac{(115.34 )^2}{33} \\
&=& 286.24 \\
\end{eqnarray*}



\begin{eqnarray*}
S_{vs} &=& \sum v_i\;s_i - \frac{(\sum v_i)(\sum s_i) }{n} \\
&=& 15417.75 - \frac{2843.7 \times  115.34 }{33} \\
&=& 5478.6\\
\end{eqnarray*}

\item (ii)
(b)
2 
 / n \;=\; 397499.8   2843.7  / 33 \;=\; 152450.4

2
2 
 / n \;=\; 689.37   115.34  / 33 \;=\; 286.24

i
2
i
S vs
5478.6
\;=\;
\;=\; 0.8294
152450.4  286.24

\begin{eqnarray*}
r &=&  \frac{S_{vs}}{\sqrt{S_{vv} \times S_{ss} } }\\
&=& \frac{5478.6}{\sqrt{152450.4 \times 286.24 }}\\
&=& 0.8294\\
\end{eqnarray*}

S ss S vv
H_0 : r \;=\; 0, H_{1} : r \neq  0

\begin{itemize}
    \item Test statistic 
    
\begin{eqnarray*} 
\mbox{Test Statistic} &=& r \frac{ \sqrt{n-2} }{\sqrt{1-r^2}} \\
&=& 0.8294 \frac{ \sqrt{33-2} }{\sqrt{1-(0.8294)^2}} \\
&=& 8.266 \\
\end{eqnarray*}

\item At 0.5\% level t 31 \;=\; 2.744 which << test statistic
\item So reject H_0 .
\end{itemize}
\item (iii)
\;=\;
S vs
5478.6
\;=\;
\;=\; 0.0359
S ss 152450.4
 \;=\; v   s \;=\;
115.34
2843.7
 0.0359
\;=\; 0.398
33
33
v i \;=\; 0.398  0.0359 s i

\item (iv)
Testing whether  is significantly different from zero is mathematically the same as testing whether the correlation coefficient is significantly different from zero.
As H_0 was rejected in (ii)(b), we can conclude testing H_0 :  \;=\; 0 would give
the same result.
\item 
(v)
It is true that extreme observations can determine the strength of a linear relationship. However, there are many more bonds in the central part of the data and we would consequently expect a greater range of value traded.

\item Generally well answered. In part (ii)(b) Fisher’s z transformation method was also allowed.
In part (v) other possible reasonable comments were given credit.
\end{itemize}
\end{document}
