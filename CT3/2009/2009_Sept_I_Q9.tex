\documentclass[a4paper,12pt]{article}

%%%%%%%%%%%%%%%%%%%%%%%%%%%%%%%%%%%%%%%%%%%%%%%%%%%%%%%%%%%%%%%%%%%%%%%%%%%%%%%%%%%%%%%%%%%%%%%%%%%%%%%%%%%%%%%%%%%%%%%%%%%%%%%%%%%%%%%%%%%%%%%%%%%%%%%%%%%%%%%%%%%%%%%%%%%%%%%%%%%%%%%%%%%%%%%%%%%%%%%%%%%%%%%%%%%%%%%%%%%%%%%%%%%%%%%%%%%%%%%%%%%%%%%%%%%%

\usepackage{eurosym}
\usepackage{vmargin}
\usepackage{amsmath}
\usepackage{graphics}
\usepackage{epsfig}
\usepackage{enumerate}
\usepackage{multicol}
\usepackage{subfigure}
\usepackage{fancyhdr}
\usepackage{listings}
\usepackage{framed}
\usepackage{graphicx}
\usepackage{amsmath}
\usepackage{chngpage}

%\usepackage{bigints}
\usepackage{vmargin}

% left top textwidth textheight headheight

% headsep footheight footskip

\setmargins{2.0cm}{2.5cm}{16 cm}{22cm}{0.5cm}{0cm}{1cm}{1cm}

\renewcommand{\baselinestretch}{1.3}

\setcounter{MaxMatrixCols}{10}

\begin{document}
\begin{enumerate}

%%%%%%%%%%%%%%%%%%%%%%%%%%%%%%%%%%%%%%%%%%%%%%%%%%%%%%%%%%%%%%%%%%%%%%%%%%%%%%%%%%%%%%%%%%%%%%
\item Consider a population in which a proportion $\theta$ of members have some specified characteristic. Let $P$ denote the corresponding proportion of members in a random sample of size n from the population.
\begin{enumerate}
\item (i)
Explain clearly why the mean and standard error of P are given by
E [ P ] = θ ,
s . e . [ P ] =
θ ( 1 − θ )
n
.
[3]
An insurance company uses a questionnaire to monitor the satisfaction of its customers.
In one part customers are asked to answer “yes” or “no” to a particular question.
Suppose that a random sample of 200 responses is examined.
\item (ii)
Calculate the approximate probability that at least 150 “yes” answers are found in the sample, on the assumption that the true (population) proportion of “yes” answers is 0.7.

Suppose the true (population) proportion of “yes” answers ($\theta$) is unknown, and for a random sample of 200 responses, the number of “yes” answers is found to be 146.
\item (iii)
\begin{description}
\item[(a)] Calculate an upper (one-sided) 95\% confidence interval of the form $(0, L)$ for $\theta$.
\item[(b)] Calculate a lower (one-sided) 95\% confidence interval of the form $(L, 1)$ for $\theta$.
\end{description}

(c) A test of the hypotheses:
H 0 : θ = 0.7 v H 1 : θ > 0.7
results in a P-value of 0.198.
Comment on how this result relates to the confidence interval in part
(iii)(b).
[9]
\end{enumerate}
\end{enumerate}
%%%%%%%%%%%%%%%%%%%%%%%%%%%%%%%%%%%%%%%%%%%%%%%%%%%%%%%%%%%%%%%%%%%%%%%%%%%%%%5
\newpage
9
(i) E(amount) = 0.8(1650) + 0.2(625) = 1,320 + 125 = \$1,445 1
(ii) $E(number of claims) = 150000(0.15) = 22,500$ 1
E(total claim amount) = 22500(1445) = \$32,512,500
\end{document}
