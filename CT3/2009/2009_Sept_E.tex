\documentclass[a4paper,12pt]{article}

%%%%%%%%%%%%%%%%%%%%%%%%%%%%%%%%%%%%%%%%%%%%%%%%%%%%%%%%%%%%%%%%%%%%%%%%%%%%%%%%%%%%%%%%%%%%%%%%%%%%%%%%%%%%%%%%%%%%%%%%%%%%%%%%%%%%%%%%%%%%%%%%%%%%%%%%%%%%%%%%%%%%%%%%%%%%%%%%%%%%%%%%%%%%%%%%%%%%%%%%%%%%%%%%%%%%%%%%%%%%%%%%%%%%%%%%%%%%%%%%%%%%%%%%%%%%

\usepackage{eurosym}
\usepackage{vmargin}
\usepackage{amsmath}
\usepackage{graphics}
\usepackage{epsfig}
\usepackage{enumerate}
\usepackage{multicol}
\usepackage{subfigure}
\usepackage{fancyhdr}
\usepackage{listings}
\usepackage{framed}
\usepackage{graphicx}
\usepackage{amsmath}
\usepackage{chngpage}

%\usepackage{bigints}
\usepackage{vmargin}

% left top textwidth textheight headheight

% headsep footheight footskip

\setmargins{2.0cm}{2.5cm}{16 cm}{22cm}{0.5cm}{0cm}{1cm}{1cm}

\renewcommand{\baselinestretch}{1.3}

\setcounter{MaxMatrixCols}{10}

\begin{document}
\begin{enumerate}
%%-- PLEASE TURN OVER
%%-- 11
\item Three insurance company colleagues had just completed an investigation which
involved the application of a two-sample t-test to compare two independent samples, each of size 11. They were concerned about the validity of the equal variance assumption required for this test. Their data were as follows.
A:
B:
21
19
22
18
28
38
27
33
20
24
23
39
26
22
32
20
25
28
21
26
30
30
\sigma  x A = 275, \sigma  x 2 A = 7, 033, \sigma  x B = 297, \sigma  x B 2 = 8,559
\begin{enumerate}[(i)]
\item (i)
One of the colleagues suggested a graphical approach for the comparison of the variances.
\item Draw a suitable diagram to represent these data so that the variability of the samples can be compared, and comment briefly on that comparison.
[3]
(ii)
(iii)
\item Another colleague suggested using an F-test for the comparison of variances.
(a) Perform this F-test at the 5% level to compare the variances and express your conclusion clearly.
(b) In addition obtain an approximate value of the P-value for this test by linearly interpolating between suitable entries in the tables.
\item 
The third colleague suggested another procedure using a two-sample t-test in the following way:
“For each sample calculate the absolute values of the deviations of the observations from the mean of that sample;
then apply a two-sample t-test to the two sets of absolute deviations.”
(a) Discuss the possible reasoning behind this suggested procedure by considering the potential values of such absolute deviations when the assumption of equal variances is valid and when it is not valid.
(b) (1) Calculate the required sets of absolute deviations for the given
data.
(2) Perform the suggested two-sample t-test at the 5\% level stating your conclusion clearly.
(3) Obtain, in addition, an approximate P-value for this test by linearly interpolating between suitable entries in the tables.
%%-- [11]
\item (iv)
Comment briefly on the conclusions that may have been reached by the three
colleagues.
\end{enumerate}
\newpage
%%%%%%%%%%%%%%%%%%%%%%%%%%%%%%%%%%%%%%%%%%%%%%%%%%%%%%%%%%%%%%%%%%%%%%%%%%%%%%%%%%%%%%%%%%%%%%

11
(i)
Dotplots on same scale are most suitable
[alternatively boxplots or histograms are acceptable]
2
The spread of the B data appears to be greater than that of the A data and so casts some doubt on the equal variance assumption.
1
(ii)
(a)
s 2 A 1
275 2
7033
10
11 15.8 s B 2 1
297 2
8559
10
11 54.0 1
3.418 on 10, 10 df 1
F
s B 2
s 2 A
54.0
15.8
For a two-sided test at the 5\% level, critical value is
F 10,10(2.5\%) = 3.717
So we accept H 0 :
(b)
2
A
1
2
B
at the 5% level.
1
F 10,10(2.5\%) = 3.717 and F 10,10(5%) = 2.978
So P-value is between 0.05 and 0.10
1
\begin{itemize}
\item By interpolation: P-value is
0.05
(iii)
(a)
3.717 3.418
(0.10 0.05) 0.05 (0.405)(0.05) 0.070
3.717 2.978
\item If the samples have equal variances, then the absolute deviations will be similar in size for both samples; if one sample has a larger variance than the other, then the deviations will be more extreme such that the
absolute deviations will be larger for that sample.
\item A two-sided two-sample t-test applied to these absolute deviations will test for a difference in the means of these absolute deviations and hence for a difference in the variances in the original samples.
(b)
(1)
x A
275
11
2
25
So the deviations for sample A , i.e. d A
4 3 3 2 5 2 1 7 0 4 5
Page 6
1
| x A x A | , are
1

x B
297
11
27
So the deviations for sample B , i.e. d B
| x B
x B | are
8 9 11 6 3 12 5 7 1 1 3
(2)
1
Calculations:
d A
36,
d A 2 158 and
d B
66,
d B 2
540
d A 36
11 2
3.273 and s dA 1
36 2
158
10
11 4.018
d B 66
11 2
6.000 and s dB 1
66 2
540
10
11 14.400
1
2
s dp
10(4.018) 10(14.400)
20
obs. t =
3.273 6.000
1 1
3.035
11 11
3.035 1
2.107 on 20 df 1
9.209
2.727
1.294
s dp
\item For the two-sided test at the 5\% level, critical value is
t 20(2.5\%) = 2.086
So we just reject H 0 :
at the 5% level.
(3)
dA
dB
and hence H 0 :
2
A
1
2
B
1
t 20(2.5\%) = 2.086 and t 20(1\%) = 2.528
\item So P-value is between 0.02 and 0.05
1
\item By interpolation: P-value is
0.02
(iv)
2.528 2.107
(0.03) 0.02 (0.952)(0.03) 0.049
2.528 2.086
1
Tests in (ii) and (iii) give different results at the 5\% level, but in fact have
quite similar P-values.
Graphical approach in (i) casts doubt on H 0 .
So all three are fairly consistent.
2
\end{itemize}
\end{document}
