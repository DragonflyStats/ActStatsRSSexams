%%- C

\documentclass[a4paper,12pt]{article}

%%%%%%%%%%%%%%%%%%%%%%%%%%%%%%%%%%%%%%%%%%%%%%%%%%%%%%%%%%%%%%%%%%%%%%%%%%%%%%%%%%%%%%%%%%%%%%%%%%%%%%%%%%%%%%%%%%%%%%%%%%%%%%%%%%%%%%%%%%%%%%%%%%%%%%%%%%%%%%%%%%%%%%%%%%%%%%%%%%%%%%%%%%%%%%%%%%%%%%%%%%%%%%%%%%%%%%%%%%%%%%%%%%%%%%%%%%%%%%%%%%%%%%%%%%%%

\usepackage{eurosym}
\usepackage{vmargin}
\usepackage{amsmath}
\usepackage{graphics}
\usepackage{epsfig}
\usepackage{enumerate}
\usepackage{multicol}
\usepackage{subfigure}
\usepackage{fancyhdr}
\usepackage{listings}
\usepackage{framed}
\usepackage{graphicx}
\usepackage{amsmath}
\usepackage{chngpage}

%\usepackage{bigints}
\usepackage{vmargin}

% left top textwidth textheight headheight

% headsep footheight footskip

\setmargins{2.0cm}{2.5cm}{16 cm}{22cm}{0.5cm}{0cm}{1cm}{1cm}

\renewcommand{\baselinestretch}{1.3}

\setcounter{MaxMatrixCols}{10}

\begin{document}
\begin{enumerate}

A survey is undertaken to investigate the frequency of motor accidents at a certain intersection.
It is assumed that, independently for each week, the number of accidents follows a Poisson distribution with mean $\lambda$.
\begin{enumerate}[(i)]
\item (i)
%%%%%%%%%%%%%%%%%%%%%%%%%%%%%%%%%%%%%%%%%%%%%%%%%%%%%%%%%%%%%%%%%%%%%%%%
8
In a single week of observation two accidents occur. Determine a 95\% confidence interval for $\lambda$, using tables of “Probabilities for the Poisson distribution”.
\item 
(ii) In an observation period of 30 weeks an average of 2.4 accidents is recorded.
Determine a 95\% confidence interval for $\lambda$, using a normal approximation. 
\item (iii) Comment on your answers in parts (i) and (ii) above.
\end{enumerate}
\item 
%%-Question 7
A random sample of 25 recent claim amounts in a general insurance context is taken from a population that you may assume is normally distributed. In units of \$1,000, the sample mean is x = 9.416 and the sample standard deviation is s = 2.105.
Calculate a 95\% one-sided upper confidence limit (that is, the upper limit k of a confidence interval of the form (0,k)) for the standard deviation of the claim amounts in the population.
%%%%%%%%%%%%%%%%%%%%%%%%%%%%%%%%%%%%%%%%%%%%%%%%%%%%%%%%%%%%%%%%%%%%%%%%

7
(i)
For a single observation x from Poisson( \lambda ) a 95\% confidence interval for $\lambda$ is
( \lambda 1 , \lambda 2 ) where
∞
\sum p ( r ; \lambda 1 ) = 0.025
x
and
r = x
\sum p ( r ; \lambda 2 ) = 0.025
r = 0
So for x = 2
\lambda 1 is s.t.
∞
\sum p ( r ; \lambda 1 ) = 0.025
r = 2
1
i.e.
\sum p ( r ; \lambda 1 ) = 0.975
r = 0
From tables \lambda 1 is between 0.20 and 0.30 being about 0.24.
\lambda 2 is s.t.
2
\sum p ( r ; \lambda 2 ) = 0.025
r = 0
From tables \lambda 2 is between 7.00 and 7.25 being about 7.23.
95\% confidence interval for \lambda is $(0.24, 7.23)$.
%%%%%%%%%%%%%%%%%%%%%%%%%%%%%%%%%%%%%%%%%%%%%%%%%%%%%%%%%%%%%%%%%%%%%%%%%%%%%%%5
(ii)
For a sample of n with observed mean x
X −\lambda
≈ N (0,1)
ˆ \lambda
where \lambda ˆ = X
[OR: could use
\sum  X − n \lambda
≈ N (0,1) ]
ˆ
n \lambda
n
giving an approximate 95\% confidence interval as X ± 1.96
X
n
So for n = 30, x =2.4
95\% CI is 2.4 ± 1.96
(iii)
2.4
⇒ 2.4 ± 0.55 ⇒ (1.85, 2.95)
30
The CI in (ii) is much narrower due to having more data.
(It is also centred higher due to the larger estimate.)
%%%%%%%%%%%%%%%%%%%%%%%%%
8
Let \sigma  2 be the population variance.
24 S 2
\sum  2
⎛ 24 S 2
⎞
~ χ 224 ⇒ P ⎜ 2 > 13.85 ⎟ = 0.95
⎜ \sum 
⎟
⎝
⎠
⎛ 2 24 S 2 ⎞
2
2
⇒ P ⎜ \sum  <
⎟ ⎟ = 0.95 ⇒ k = 24 × 2.105 / 13.85 = 7.678
⎜
13.85 ⎠
⎝
⇒ k = 2.771 so upper confidence limit for \sum  is 2.771 i.e. \$2771
(OR: CI is (0, 2.771) i.e. (0, \$2771)).


\end{document}
