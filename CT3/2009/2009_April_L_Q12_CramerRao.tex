\documentclass[a4paper,12pt]{article}

%%%%%%%%%%%%%%%%%%%%%%%%%%%%%%%%%%%%%%%%%%%%%%%%%%%%%%%%%%%%%%%%%%%%%%%%%%%%%%%%%%%%%%%%%%%%%%%%%%%%%%%%%%%%%%%%%%%%%%%%%%%%%%%%%%%%%%%%%%%%%%%%%%%%%%%%%%%%%%%%%%%%%%%%%%%%%%%%%%%%%%%%%%%%%%%%%%%%%%%%%%%%%%%%%%%%%%%%%%%%%%%%%%%%%%%%%%%%%%%%%%%%%%%%%%%%

\usepackage{eurosym}
\usepackage{vmargin}
\usepackage{amsmath}
\usepackage{graphics}
\usepackage{epsfig}
\usepackage{enumerate}
\usepackage{multicol}
\usepackage{subfigure}
\usepackage{fancyhdr}
\usepackage{listings}
\usepackage{framed}
\usepackage{graphicx}
\usepackage{amsmath}
\usepackage{chngpage}

%\usepackage{bigints}
\usepackage{vmargin}

% left top textwidth textheight headheight

% headsep footheight footskip

\setmargins{2.0cm}{2.5cm}{16 cm}{22cm}{0.5cm}{0cm}{1cm}{1cm}

\renewcommand{\baselinestretch}{1.3}

\setcounter{MaxMatrixCols}{10}

\begin{document}

%%%%%%%%%%%%%%%%%%%%%%%%%%%%%%%%%%%%%%%%%%%%%%%%5

%%- April 2009 Question 12

\begin{center
\begin{tabular}{ccccc}
class:   & A & B & C & D \\
frequency:&  1071 & 62 & 68 & 299 \\
\end{tabular}
\end{center}

A genetic model specifies that the probability that an individual plant belongs to each
class is given by:

\begin{tabular}{ccccc}
class:   & A & B & C & D \\

probability: & 
${ \displaystyle \frac{1}{4} (2 + \theta  )  }$ & 
${ \displaystyle \frac{1}{4} (1 − \theta  )  }$ & 
${ \displaystyle \frac{1}{4} (1 − \theta  )  }$ & 
${ \displaystyle \frac{1}{4} (\theta )     }$  \\
\end{center}
\end{tabular}

%%%%%%%%%%%%%%%%%%%%%%%%%%%%%%



%%%%%%%%%%%%%%%%%%%%%%%%%%%%%%%%%%%%%%%%%%%%%%%%%%%%%%%%%%%%%%%%%%%%%%%%%%%%%%%%%%%%%%%%%5
12
\begin{itemize}
\item (i)
(a)

\[L ( \theta ) = [\frac{1}{4} (2 + \theta )]^{1071} [\frac{1}{4} (1 − \theta )]^{62} [ \frac{1}{4}(1 − \theta )]^{68} [\frac{1}{4} \theta ]^{299} (\times \mbox{constant})\]

\[L ( \theta ) \propto (2 + \theta )^{1071} (1 − \theta ) ^{130} \theta^{299}
\[log L ( \theta ) = const + 1071\log(2 + \theta ) + 130 \log(1 − \theta ) + 299 \log \theta\]
%------------------%
(b)


\begin{eqnarray*}
\frac{d}{d \theta} \log L ( \theta ) &=& \frac{1071}{2 + \theta } \;-\; \frac{130}{1 − \theta} ;\+\; \frac{299}{\theta} \\
& & \\
&=& \frac{
1071 \theta (1 − \theta ) − 130 \theta (2 + \theta ) + 299(2 + \theta )(1 − \theta )}{
(2 + \theta )(1 − \theta ) \theta} \\
\end{eqnarray}

\[numerator = 1071 \theta − 1071 \theta^2 − 260 \theta − 130 \theta"2 + 598 − 299 \theta − 299 \theta^2\]
equate to zero: \[750 \theta 2 − 256 \theta − 299 = 0\]

%------------------%

Therefore
\[\hat{\theta} = \frac{256 \pm \sqrt{256 2 − 4(750)( − 299)}}{2(750)}\]
\[\hat{\theta} = 0.17067 \pm 0.65406\]

So MLE$ \hat{\theta} = 0.82473$ (or 0.825 to 3dp) as other root is negative.
(ii)
%--------------------%
(a)

\begin{eqnarray*}
\frac{d^2}{d^2 \theta} \log L ( \theta ) &=& -\frac{1071}{(2 + \theta )^2 } \;-\; \frac{130}{(1 − \theta )^2} ;\-\; \frac{299}{\theta^2} \\
\end{eqnarray}


at $\hat{\theta} = 0.825$ ,
\[ \frac{d^2}{d^2 \theta}  \log L ( \theta ) = − 134.20 − 4244.90 − 439.30 = − 4818.4  \]

%--------------------%
\[CRlb =  \frac{1}{-E \left \frac{d^2}{d^2 \theta}  \log L ( \theta )right] } \approx \frac{1}{4818.4 }
= 0.0002075\]

%-----------------------------% 

\[\hat{\theta} ≈ N ( \theta , CRlb )\] for large samples
and so an approximate 95\% CI for $\theta$ is $\hat{\theta} \pm 1.96 CRlb$
Here: $0.825 \pm 1.96 0.0002075$ which is equivalent to $0.825 \pm 0.028 or (0.797, 0.853)$

%-----------------------------% 
 (iii)
(a)
\begin{itemize}
    \item With $\theta = 0.775$ the four probabilities are 0.69375, 0.05625, 0.05625, 0.19375 respectively
and the corresponding expected frequencies are 1040.625, 84.375, 84.375, 290.625.


\item The test statistic is
\begin{eqnarray*}
\chi^{2}_{TS} = \sum \frac{( o \;-\; e ) 2}{e}\\
&=& 0.887 + 5.934 + 3.178 + 0.241\\ 
&=& 10.24 \\
\end{eqnarray*}

on 3 df
\[ P-value = P ( \chi^{2}_{3} > 10.24) = 1 − 0.983 = 0.017\]
\item These data do not support the model with the value $\theta = 0.775$ in that the probability of observing these data when $\theta = 0.775$ is only 0.017.
[OR: could say “do not support ate 5\% level, but do support at 1\% level”]
(b)
\item This is consistent with the fact that $\theta = 0.775$ is well outside the approximate 95\% CI.
\end{itemize}


\end{document}
