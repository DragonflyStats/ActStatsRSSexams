\documentclass[a4paper,12pt]{article}

%%%%%%%%%%%%%%%%%%%%%%%%%%%%%%%%%%%%%%%%%%%%%%%%%%%%%%%%%%%%%%%%%%%%%%%%%%%%%%%%%%%%%%%%%%%%%%%%%%%%%%%%%%%%%%%%%%%%%%%%%%%%%%%%%%%%%%%%%%%%%%%%%%%%%%%%%%%%%%%%%%%%%%%%%%%%%%%%%%%%%%%%%%%%%%%%%%%%%%%%%%%%%%%%%%%%%%%%%%%%%%%%%%%%%%%%%%%%%%%%%%%%%%%%%%%%

\usepackage{eurosym}
\usepackage{vmargin}
\usepackage{amsmath}
\usepackage{graphics}
\usepackage{epsfig}
\usepackage{enumerate}
\usepackage{multicol}
\usepackage{subfigure}
\usepackage{fancyhdr}
\usepackage{listings}
\usepackage{framed}
\usepackage{graphicx}
\usepackage{amsmath}
\usepackage{chngpage}

%\usepackage{bigints}
\usepackage{vmargin}

% left top textwidth textheight headheight

% headsep footheight footskip

\setmargins{2.0cm}{2.5cm}{16 cm}{22cm}{0.5cm}{0cm}{1cm}{1cm}

\renewcommand{\baselinestretch}{1.3}

\setcounter{MaxMatrixCols}{10}

\begin{document}

3
[3]
The random variable X has probability density function
\[f ( x ) = k (1 − x )(1 + x ),
\qquad 0 < x < 1 ,\]
where k is a positive constant.
4
5
\begin{enumerate}[(i)]
\item Show that $k = 1.5$.
\item Calculate the probability $P(X > 0.25)$.
\end{enumerate}

%%%%%%%%%%%%%%%%%%%%%%%%%%%%%%%%%%%%%%%%%%%%%%%%%%%%%%%%%%
3
%%%%%%%%%%%%%%%%%%%%%%%%%%%%%%%%%%%%%%%%%%%%%%%%%%%%%%%%%%%%%%%%
(i)
∫
0
(ii)
1
1
⎡
x 3 ⎤
⎛ 1 ⎞
f ( x ) dx = 1 ⇒ ∫ k (1 − x 2 ) dx = 1 ⇒ k ⎢ x − ⎥ = 1 ⇒ k ⎜ 1 − ⎟ = 1 ⇒ k = 1.5
3 ⎥ ⎦
⎝ 3 ⎠
⎣ ⎢
0
0
P ( X > 0.25) =
1
∫ 0.25 f ( x ) dx
1
⎡
x 3 ⎤
= 1.5 × 0.422 = 0.633.
= ∫ 1.5(1 − x ) dx = 1.5 ⎢ x − ⎥
0.25
3 ⎥ ⎦
⎢ ⎣
0.25
1
%%%%%%%%%%%%%%%%%%%%%%%%%%%%%%%%%%%%%%%%%%%%%%%%%%%%%%%%%%%%%%%%%%%%%%%%%%%%%%%%%%%%%
\end{document}
