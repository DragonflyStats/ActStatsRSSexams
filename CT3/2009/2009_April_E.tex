%%- E

\documentclass[a4paper,12pt]{article}

%%%%%%%%%%%%%%%%%%%%%%%%%%%%%%%%%%%%%%%%%%%%%%%%%%%%%%%%%%%%%%%%%%%%%%%%%%%%%%%%%%%%%%%%%%%%%%%%%%%%%%%%%%%%%%%%%%%%%%%%%%%%%%%%%%%%%%%%%%%%%%%%%%%%%%%%%%%%%%%%%%%%%%%%%%%%%%%%%%%%%%%%%%%%%%%%%%%%%%%%%%%%%%%%%%%%%%%%%%%%%%%%%%%%%%%%%%%%%%%%%%%%%%%%%%%%

\usepackage{eurosym}
\usepackage{vmargin}
\usepackage{amsmath}
\usepackage{graphics}
\usepackage{epsfig}
\usepackage{enumerate}
\usepackage{multicol}
\usepackage{subfigure}
\usepackage{fancyhdr}
\usepackage{listings}
\usepackage{framed}
\usepackage{graphicx}
\usepackage{amsmath}
\usepackage{chngpage}

%\usepackage{bigints}
\usepackage{vmargin}

% left top textwidth textheight headheight

% headsep footheight footskip

\setmargins{2.0cm}{2.5cm}{16 cm}{22cm}{0.5cm}{0cm}{1cm}{1cm}

\renewcommand{\baselinestretch}{1.3}

\setcounter{MaxMatrixCols}{10}

\begin{document}
\begin{enumerate}
10
\item For a group of policies the probability distribution of the total number of claims, N,
arising during a period of one year is given by
P(N = 0) = 0.70, P(N = 1) = 0.15, P(N = 2) = 0.10, P(N = 3) = 0.05.
Each claim amount, X (in units of £1,000), follows a gamma distribution with parameters \alpha = 2 and \lambda = 0.1 independently of each other claim amount and of the number of claims.
Calculate the expected value and the standard deviation of the total of the claim amounts for a period of one year.
%%------ CT3 A2009—4


11
\item The number of claims, $X$, which arise in a year on each policy of a particular class is to be modelled as a Poisson random variable with mean \lambda. Let $X = (X_1 , X_2 , \ldots, X_n )$ be a random sample from the distribution of X, and let X =
1 n
∑ X i .
n i = 1
n
\begin{enumerate}[(i)]
\item (i)
(a)
Use moment generating functions to show that
∑ X i has a Poisson
i = 1
distribution with mean $n\lambda$.
(b) State, with a brief reason, whether or not the variable 2X 1 + 5 has a Poisson distribution.
(c) State, with a brief reason, whether or not X has a Poisson distribution in the case that n = 2.
(d) State the approximate distribution of X in the case that n is large.
%%-- [8]
\item An actuary is interested in the level of claims being experienced and wants in particular to test the hypotheses
H 0 : \lambda = 1 v H 1 : \lambda > 1 .
He decides to use a random sample of size n = 100 and the best (most powerful) available test. You may assume that this test rejects H 0 for x > k , for some constant k.
\item (ii)
(a) Show that the value of k for the test with level of significance 0.01 is k = 1.2326.
(b) Calculate the power of the test in part \item (ii)(a) in the case \lambda = 1.2 and then in the case \lambda = 1.5.
(c) Comment briefly on the values of the power of the test obtained in part
\item (ii)(b).
\end{enumerate}
\end{enumerate}
%%%%%%%%%%%%%%%%%%%%%%%%%%%%%%%%%%%%%%%%%%%%%%%%%%%%%%%%%%%%%%%%%%%%%%%%%%%%%%%%%%%%%%%%%%%%%%%%%%%%%%%%%%%%%%%%%%%%%%%
\newpage

11
\item (i)
(a)
Let S = ∑ X i
i = 1
{ (
) }
M X ( t ) = exp \lambda e t − 1
{ (
) }
n
{ (
) }
n
⇒ \[M S ( t ) = { M X ( t ) } = ⎡ exp \lambda e t − 1 ⎤ = exp n \lambda e t − 1\]

⇒ S ~ Poisson(n\lambda)
(b)
No
One reason is that E[2X 1 + 5] = 2\lambda + 5, which is not equal to V[2X 1 + 5] = 4\lambda
[Note: another obvious reason is that 2X 1 + 5 can only takes values 5,
7, 9, ... , not 0, 1, 2, 3,... ]
%%%%%%%%%%%%%%%%%%%%%%%%%%%%%%%%%%%%%%%%%%%%%%%%%%%%%%%%%%%%%%%%%%%%%%%%%%%%%%%%%%%%%%%%%%%%%%%%%%%%%%%%%%%%%%%%%
(c)
No
One reason is that E ⎣ ⎡ X ⎦ ⎤ = \lambda , which is not equal to V ⎡ ⎣ X ⎤ ⎦ = \lambda / 2
[Note: another obvious reason is that X can take values 0.5, 1.5, 2.5, ... , which a Poisson variable cannot.]

(d)
\item (ii)
(a)
(b)
⎛ \lambda ⎞
X ≈ N ⎜ \lambda , ⎟
⎝ n ⎠
1 ⎞
⎛

Under H 0 , X ~ N ⎜ 1 ,
⎟ approximately
⎝ 100 ⎠
k − 1

k is such that P ( X > k | H 0 ) = 0.01 so
= 2.3263
0.1
⇒ k = 1.2326
\[Power(\lambda) = P(reject H 0 |\lambda)\]
\[Power(\lambda = 1.2) = P ( X > 1.2326 ) where X ~ N (1.2, 0.012)\]
= P(Z > 0.298) = 0.383
Power(\lambda = 1.5) = P ( X > 1.2326 ) where X ~ N (1.5, 0.015)
= P(Z > −2.183) = 0.985
(c)

%%%%%%%%%%%%%%%%%%%%%%%%%%%%%%%%%%%%%%%%%%%%%%%%%%%%%%%%%%%%%%%%%%%%%%%%%%%%%%%%%%%%%%%%%5
12
\item (i)
(a)
Power of test increases as the value of \lambda increases further away from
\lambda = 1.
1
1
1
1
L ( \theta ) = [ (2 + \theta )] 1071 [ (1 − \theta )] 62 [ (1 − \theta )] 68 [ \theta ] 299 (× constant)
4
4
4
4
∝ (2 + \theta ) 1071 (1 − \theta ) 130 \theta 299
log L ( \theta ) = const + 1071log(2 + \theta ) + 130 log(1 − \theta ) + 299 log \theta
(b)
d
1071 130 299
log L ( \theta ) =
−
+
d \theta
2 + \theta 1 − \theta
\theta
=
1071 \theta (1 − \theta ) − 130 \theta (2 + \theta ) + 299(2 + \theta )(1 − \theta )
(2 + \theta )(1 − \theta ) \theta

\[numerator = 1071 \theta − 1071 \theta 2 − 260 \theta − 130 \theta 2 + 598 − 299 \theta − 299 \theta^2\]
equate to zero: 750 \theta 2 − 256 \theta − 299 = 0
Page 6
1

256 \pm 256 2 − 4(750)( − 299)
ˆ
∴\theta =
= 0.17067 \pm 0.65406
2(750)
So MLE \hat{\theta} = 0.82473 (or 0.825 to 3dp) as other root is negative.
(ii)
(a)
d 2
d \theta
2
log L ( \theta ) = −
1071
(2 + \theta )
2
−
130
(1 − \theta )
2
−
299
\theta 2
d 2
at $\hat{\theta} = 0.825$ ,
log L ( \theta ) = − 134.20 − 4244.90 − 439.30 = − 4818.4
d \theta 2
CRlb =
(b)
1
⎡ d 2
⎤
− E ⎢ 2 log L ( \theta ) ⎥
⎢ ⎣ d \theta
⎥ ⎦
≈
1
= 0.0002075
4818.4
\[\hat{\theta} ≈ N ( \theta , CRlb )\] for large samples
and so an approximate 95\% CI for \theta is \hat{\theta} \pm 1.96 CRlb
Here: 0.825 \pm 1.96 0.0002075 ⇒ 0.825 \pm 0.028 or (0.797, 0.853)

 (iii)
(a)
\begin{itemize}
    \item With $\theta = 0.775$ the four probabilities are 0.69375, 0.05625, 0.05625, 0.19375 respectively
and the corresponding expected frequencies are 1040.625, 84.375, 84.375, 290.625.
∴\item χ 2 = ∑
( o − e ) 2
= 0.887 + 5.934 + 3.178 + 0.241 = 10.24 on 3 df
e
P-value = P ( χ 3 2 > 10.24) = 1 − 0.983 = 0.017
\item These data do not support the model with the value \theta = 0.775 in that the probability of observing these data when \theta = 0.775 is only 0.017.
[OR: could say “do not support at 5\% level, but do support at 1\% level”]
(b)
\item This is consistent with the fact that \theta = 0.775 is well outside the approximate 95\% CI.
\end{itemize}


\end{document}
