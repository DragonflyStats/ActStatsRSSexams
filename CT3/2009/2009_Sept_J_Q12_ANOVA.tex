\documentclass[a4paper,12pt]{article}

%%%%%%%%%%%%%%%%%%%%%%%%%%%%%%%%%%%%%%%%%%%%%%%%%%%%%%%%%%%%%%%%%%%%%%%%%%%%%%%%%%%%%%%%%%%%%%%%%%%%%%%%%%%%%%%%%%%%%%%%%%%%%%%%%%%%%%%%%%%%%%%%%%%%%%%%%%%%%%%%%%%%%%%%%%%%%%%%%%%%%%%%%%%%%%%%%%%%%%%%%%%%%%%%%%%%%%%%%%%%%%%%%%%%%%%%%%%%%%%%%%%%%%%%%%%%

\usepackage{eurosym}
\usepackage{vmargin}
\usepackage{amsmath}
\usepackage{graphics}
\usepackage{epsfig}
\usepackage{enumerate}
\usepackage{multicol}
\usepackage{subfigure}
\usepackage{fancyhdr}
\usepackage{listings}
\usepackage{framed}
\usepackage{graphicx}
\usepackage{amsmath}
\usepackage{chngpage}

%\usepackage{bigints}
\usepackage{vmargin}

% left top textwidth textheight headheight

% headsep footheight footskip

\setmargins{2.0cm}{2.5cm}{16 cm}{22cm}{0.5cm}{0cm}{1cm}{1cm}

\renewcommand{\baselinestretch}{1.3}

\setcounter{MaxMatrixCols}{10}

\begin{document}
\begin{enumerate}

%%-- CT3 S2009—612
\item A bank has a free telephone number for its customer services department. Often the call volume is heavy and customers are placed on hold until a staff member is available to answer. The bank hopes that a caller remains on hold until the call is
answered, so as not to upset or lose an existing or potential customer.

A survey was conducted to analyse whether callers would remain on hold longer (on average), if they heard a recorded message containing: (A) an advertisement about the bank’s products; (B) “easy listening” music; or (C) classical music. The bank
randomly selected a sample of five unanswered calls under each recorded message, and the length of time (in minutes) that the caller remained on hold before hanging up is given in the table below.
\begin{verbatim}
Recorded message
Time
Total
5 1 11 2 8
A: advertisement
B: easy listening music 0 1 4 6 3
C: classical music
13 9 8 15 7
27
14
52
\end{verbatim}
For these data \sigma y = 93, \sigma y 2 = 865
Let \mu  A , \mu  B , \mu  C denote the mean telephone holding times under recorded message A, B and C respectively.
\item (i)
(a) Perform an analysis of variance to test the hypothesis that the nature of the recorded message has no effect on the length of time that callers remain on hold. You should construct an appropriate ANOVA table and state your conclusion clearly.
(b) Calculate a 95\% confidence interval for $\mu_A − \mu_C$ , using information available from all three samples.
An equivalent approach for analysing the effects of the recorded messages on holding
time is the following:
consider the regression model E [ Y i ] = a + b 1 x i 1 + b 2 x i 2 , i = 1, 2, ..., 15, where Y i is the telephone holding time and x i 1 , x i 2 are indicator variables such that x i 1 = 1 if the message for caller i contains an advertisement (and 0 otherwise), and x i 2 = 1 if the message contains easy listening music (and 0 otherwise).

The results from fitting this model are given below:
Coef.
Intercept
x 1
x 2
s = 3.406
%%%%%%%%%%%%%%%%%%%%%%%%%%%%%%%%%%%
10.400
−5.000
−7.600
Std. Error t-value
1.523
2.154
2.154
p-value
6.828 1.8 * 10 -5
−2.321
0.039
−3.528
0.004
R-sq = 51.7%
%-------------------------------------------------------------------------------%
\item (ii)
Using the fitted model:
(a) Calculate the predicted value for the telephone holding time when the message contains classical music.
(b) Test the hypothesis H 0 : b 1 = 0 against H 1 : b 1 \neq 0 at the 5\% level of significance.
(c) Derive an expression relating b 1 with \mu  A and \mu  C , and hence verify your result from the test in \item (ii)(b) using the confidence interval
obtained in (i)(b).

\end{enumerate}
\newpage
%%%%%%%%%%%%%%%%%%%%%%%%%%%%%%%%%%%%%%%%%%%%%%%%%%%%%%%%%%%%%%%%%%%%%%%%%%%%%%%%%%%%%%%%%%5
12
\item (i)
(a)
2
y A = 27, y B = 14, y C = 52 , y = 93, y = 865
SST = 865 – 932/15 = 288.4

2
2
2
2
SSB = (27 + 14 + 52 )/5
SS R = 288.4
93 /15 = 149.2
2
149.2 = 139.2
\begin{verbatim}
Source of variation
d.f. SS MSS
Between 2 149.2 74.6
Residual 12 139.2 11.6
Total 14 288.4
\end{verbatim}
2
(b)
F = 74.6/11.6 = 6.431 on (2,12) degrees of freedom. 1
%%%%%%%%%%%%%%%%%%%%%%%%%%%%%%
\begin{itemize}
\item From yellow tables, $F 2,12 (0.05) = 3.885$ and $F 2,12 (0.01) = 6.927$. 1
\item We can reject the hypothesis of “no message effect” at the 5\% significance level, but not at the 1\% level. We have some evidence
against the “no message effect” hypothesis and conclude that there is a message effect. 1
t 12 (0.025) = 2.179 1
1
95\% CI is (5.4 10.4) 2.179 11.6
5
i.e. 5 4.694
\item (ii)
(a)
or ( 9.694, 0.306)
10.4
2
1
2
\item The P-value is 0.039. We have evidence to reject the hypothesis that $b_1 = 0$ at the 5\% level of significance.
(c)
0.5
\item We have y ˆ i 10.4 5.0 x i 1 7.6 x i 2 and for classical music message we need x i 1 x i 2 0 .

\item This gives y ˆ i
(b)
1
5
For x i 1 1, x i 2
and x i 1
b 1
x i 2
A
0 we have
0 gives
C
A
1
a b 1
a
C
1
1
\item The 95\% CI for A C in \item (i)(b) can be used for testing H 0: 0 and equivalently H 0: b 1 = 0. The interval does not include
A
C
the value 0, and thus we reject H 0 at the 5\% level.
2
\end{itemize}
\end{document}
