\documentclass[a4paper,12pt]{article}

%%%%%%%%%%%%%%%%%%%%%%%%%%%%%%%%%%%%%%%%%%%%%%%%%%%%%%%%%%%%%%%%%%%%%%%%%%%%%%%%%%%%%%%%%%%%%%%%%%%%%%%%%%%%%%%%%%%%%%%%%%%%%%%%%%%%%%%%%%%%%%%%%%%%%%%%%%%%%%%%%%%%%%%%%%%%%%%%%%%%%%%%%%%%%%%%%%%%%%%%%%%%%%%%%%%%%%%%%%%%%%%%%%%%%%%%%%%%%%%%%%%%%%%%%%%%

\usepackage{eurosym}
\usepackage{vmargin}
\usepackage{amsmath}
\usepackage{graphics}
\usepackage{epsfig}
\usepackage{enumerate}
\usepackage{multicol}
\usepackage{subfigure}
\usepackage{fancyhdr}
\usepackage{listings}
\usepackage{framed}
\usepackage{graphicx}
\usepackage{amsmath}
\usepackage{chngpage}

%\usepackage{bigints}
\usepackage{vmargin}

% left top textwidth textheight headheight

% headsep footheight footskip

\setmargins{2.0cm}{2.5cm}{16 cm}{22cm}{0.5cm}{0cm}{1cm}{1cm}

\renewcommand{\baselinestretch}{1.3}

\setcounter{MaxMatrixCols}{10}

\begin{document}
\begin{enumerate}
%%[Total 8]
\item The table below shows a bivariate probability distribution for two discrete random variables X and Y:
Y = 1
Y = 2
Find the value of E[X|Y = 2].
9
2
3
21.7
X = 0
0.15
0.05
X = 1
0.20
0.15
X = 2
0.25
0.20
[3]
In a group of motor insurance policies issued by a company, 80\% of claims are made on comprehensive policies and 20\% are made on third-party-only policies.
\end{enumerate}
\item (i) Calculate the average amount paid out on a claim, given that the average amount paid out by the company on a comprehensive policy claim is \$1,650, and the average amount paid out on a third-party-only policy claim is \$625.
[1]
\item (ii) Calculate the total expected amount paid out in claims by the company in one year, given that the total number of policies is 150,000 and, on average, the claim rate is 0.15 claims per policy per year.

\end{enumerate}
%%%%%%%%%%%%%%%%%%%%%%%%%%%%%%%%%%%%%%%%%%%%%%%%%%%%%%%%%%%%%%%%%%%%%%%%%%%%%%%%%%%%%%%%%%%%%%
\item Consider a population in which a proportion $\theta$ of members have some specified characteristic. Let $P$ denote the corresponding proportion of members in a random sample of size n from the population.
\begin{enumerate}
\item (i)
Explain clearly why the mean and standard error of P are given by
E [ P ] = \theta  ,
s . e . [ P ] =
\theta  ( 1 − \theta  )
n
.
[3]
An insurance company uses a questionnaire to monitor the satisfaction of its customers.
In one part customers are asked to answer “yes” or “no” to a particular question.
Suppose that a random sample of 200 responses is examined.
\item (ii)
Calculate the approximate probability that at least 150 “yes” answers are found in the sample, on the assumption that the true (population) proportion of “yes” answers is 0.7.

Suppose the true (population) proportion of “yes” answers ($\theta$) is unknown, and for a random sample of 200 responses, the number of “yes” answers is found to be 146.
\item (iii)
\begin{description}
\item[(a)] Calculate an upper (one-sided) 95\% confidence interval of the form $(0, L)$ for $\theta$.
\item[(b)] Calculate a lower (one-sided) 95\% confidence interval of the form $(L, 1)$ for $\theta$.
\end{description}

(c) A test of the hypotheses:
H 0 : \theta  = 0.7 v H 1 : \theta  > 0.7
results in a P-value of 0.198.
Comment on how this result relates to the confidence interval in part
(iii)(b).
[9]
\end{enumerate}
\end{enumerate}
%%%%%%%%%%%%%%%%%%%%%%%%%%%%%%%%%%%%%%%%%%%%%%%%%%%%%%%%%%%%%%%%%%%%%%%%%%%%%%5
\newpage
10
(i)
(ii)
(iii)
Number of sample members with the characteristic X ~ bi(n,) with mean
n  and variance n (1 – ). P = X/n. 1
E[P] = n  / n = 1

\[s.e.[P] = {V[P]}1/2 = {V[X]/n 2}1/2 = {n (1 – )/n 2 }1/2 = { (1 – )/n}1/2 1\]
X ~ bi(200, 0.7) with mean 140 and variance 42 2
P Z
149.5 140
42
P ( X 150)
P ( Z
1.466) 0.071
%---------------------------------------------------------------------------------------%
(a) Sample proportion P ~ N(  (1 – )/200)
2
Observed P = 146/200 = 0.73
\begin{itemize}
\item Estimated standard error(P) = (0.73 \times 0.27/200)1/2 = 0.03139
Pr
P
e . s . e . P
Pr
1.645
1
0.95
P 1.645 e . s . e . P
0.95
2
\item Upper 95\% CI is given by (0, 0.73 + 1.645 \times 0.03139)
i.e. (0, 0.782)
(b)
1
By analogy with (i),
\item Lower 95\% CI is given by (0.73 – 1.645 \times  0.03139, 1)
(c)
i.e. (0.678, 1)  2
\item The P–value indicates that the null hypothesis “ = 0.7” can stand and we do not have to conclude that > 0.7. 1
\item The CI in (iii)(b) includes values down to 0.678, so all such values, including 0.7, are consistent with the data when considering how low a value of is reasonable. 1
\item The two results complement each other. 1

\end{itemize}

\end{document}
