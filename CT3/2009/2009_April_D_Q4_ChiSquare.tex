%%- B

\documentclass[a4paper,12pt]{article}

%%%%%%%%%%%%%%%%%%%%%%%%%%%%%%%%%%%%%%%%%%%%%%%%%%%%%%%%%%%%%%%%%%%%%%%%%%%%%%%%%%%%%%%%%%%%%%%%%%%%%%%%%%%%%%%%%%%%%%%%%%%%%%%%%%%%%%%%%%%%%%%%%%%%%%%%%%%%%%%%%%%%%%%%%%%%%%%%%%%%%%%%%%%%%%%%%%%%%%%%%%%%%%%%%%%%%%%%%%%%%%%%%%%%%%%%%%%%%%%%%%%%%%%%%%%%

\usepackage{eurosym}
\usepackage{vmargin}
\usepackage{amsmath}
\usepackage{graphics}
\usepackage{epsfig}
\usepackage{enumerate}
\usepackage{multicol}
\usepackage{subfigure}
\usepackage{fancyhdr}
\usepackage{listings}
\usepackage{framed}
\usepackage{graphicx}
\usepackage{amsmath}
\usepackage{chngpage}

%\usepackage{bigints}
\usepackage{vmargin}

% left top textwidth textheight headheight

% headsep footheight footskip

\setmargins{2.0cm}{2.5cm}{16 cm}{22cm}{0.5cm}{0cm}{1cm}{1cm}

\renewcommand{\baselinestretch}{1.3}

\setcounter{MaxMatrixCols}{10}

\begin{document}
\begin{enumerate}
\item 
Let the random variable $Y$ denote the size (in units of \$1,000) of the loss per claim sustained in a particular line of insurance. Suppose that $Y$ follows a chi-square distribution with 2 degrees of freedom. Two such claims are randomly chosen and
their corresponding losses are assumed to be independent of each other.
\begin{enumerate}[(i)]
\item (i) Determine the mean and the variance of the total loss from the two claims. 
\item (ii) Find the value of $k$ such that there is a probability of 0.95 that the total loss from the two claims exceeds $k$.
\end{enumerate}
%%%%%%%%%%%%%%%%%%%%%%%%%%%%%%%%5

\end{enumerate}
%[1]
%%%%%%%%%%%%%%%%%%%%%%%%%%%%%%%%%%%%%%%%%%%%%%%%%%%%%%%%%%%%%%%%%%%%%%%%%%%%%%%%%%%%
\newpage
4
(i)
2
\begin{itemize}
\item $E [ Y_i ] = 2$, $Var [ Y_i ] = 4$
Therefore $E [ Y_1 + Y_2 ] = 4$ and (since Y_1 , Y_2 independent) $Var [ Y_1 + Y_2 ] = 8$.
\item So, for total loss, mean = \$4,000 and variance = 8 ×10 6 (\$ 2 ).
(OR from Y_1 + Y_2 ~ χ 2 4 )
\item (ii)
For the total loss we have $Y_1 + Y_2 \sim \chi^2_4$ , so we want a constant k
(
)
2
such that P χ 4 > k = 0.95.
(
)
\item From tables of the χ 24 distribution P χ 24 > 0 . 7107 = 0.95.
∴The total losses will exceed \$710.7 with probability 0.95.
\end{itemize}
%%%%%%%%%%%%%%%%%%%%%%%%%%%%%%%%%%%%%%%%%%%%%%%%%%%%%%%%%%%%%%%%%%%%%%%%%%%%%%%%%%%%%%%%


\end{document}
