%%- C

\documentclass[a4paper,12pt]{article}

%%%%%%%%%%%%%%%%%%%%%%%%%%%%%%%%%%%%%%%%%%%%%%%%%%%%%%%%%%%%%%%%%%%%%%%%%%%%%%%%%%%%%%%%%%%%%%%%%%%%%%%%%%%%%%%%%%%%%%%%%%%%%%%%%%%%%%%%%%%%%%%%%%%%%%%%%%%%%%%%%%%%%%%%%%%%%%%%%%%%%%%%%%%%%%%%%%%%%%%%%%%%%%%%%%%%%%%%%%%%%%%%%%%%%%%%%%%%%%%%%%%%%%%%%%%%

\usepackage{eurosym}
\usepackage{vmargin}
\usepackage{amsmath}
\usepackage{graphics}
\usepackage{epsfig}
\usepackage{enumerate}
\usepackage{multicol}
\usepackage{subfigure}
\usepackage{fancyhdr}
\usepackage{listings}
\usepackage{framed}
\usepackage{graphicx}
\usepackage{amsmath}
\usepackage{chngpage}

%\usepackage{bigints}
\usepackage{vmargin}

% left top textwidth textheight headheight

% headsep footheight footskip

\setmargins{2.0cm}{2.5cm}{16 cm}{22cm}{0.5cm}{0cm}{1cm}{1cm}

\renewcommand{\baselinestretch}{1.3}

\setcounter{MaxMatrixCols}{10}

\begin{document}

A survey is undertaken to investigate the frequency of motor accidents at a certain intersection.
It is assumed that, independently for each week, the number of accidents follows a Poisson distribution with mean $\lambda$.
\begin{enumerate}[(i)]
\item (i)

In a single week of observation two accidents occur. Determine a 95\% confidence interval for $\lambda$, using tables of “Probabilities for the Poisson distribution”.
\item 
(ii) In an observation period of 30 weeks an average of 2.4 accidents is recorded.
Determine a 95\% confidence interval for $\lambda$, using a normal approximation. 
\item (iii) Comment on your answers in parts (i) and (ii) above.
\end{enumerate}
\item 

%%%%%%%%%%%%%%%%%%%%%%%%%%%%%%%%%%%%%%%%%%%%%%%%%%%%%%%%%%%%%%%%%%%%%%%%

7
(i)
For a single observation x from Poisson( $\lambda$ ) a 95\% confidence interval for $\lambda$ is
$( \lambda 1 , \lambda 2 )$ where
∞
\[\sum_{r = x} p ( r ; \lambda_1 ) = 0.025\]
x
and

\[\sum_{r = 0} p ( r ; \lambda_2 ) = 0.025\]

So for x = 2
$\lambda_1$ is s.t.
∞
\[\sum_{r = 2} p ( r ; \lambda_1 ) = 0.025 \]
%------------------------------%
1
i.e.
\[\sum_{r = 0} p ( r ; \lambda_1 ) = 0.975\]

From tables $\lambda_1$ is between 0.20 and 0.30 being about 0.24.
$\lambda_2$ is s.t.
2
\[\sum_{r = 0} p ( r ; \lambda 2 ) = 0.025\]
\smallskip
From tables $\lambda_2$ is between 7.00 and 7.25 being about 7.23.
95\% confidence interval for $\lambda$ is $(0.24, 7.23)$.
%%%%%%%%%%%%%%%%%%%%%%%%%%%%%%%%%%%%%%%%%%%%%%%%%%%%%%%%%%%%%%%%%%%%%%%%%%%%%%%5
(ii)
For a sample of n with observed mean x
X −\lambda
≈ N (0,1)
ˆ \lambda
where $\hat{\lambda}= X$
%% [OR: could use \sum  X − n \lambda ≈ N (0,1) ]
ˆ
n \lambda
n
giving an approximate 95\% confidence interval as X ± 1.96
X
n
So for n = 30, x =2.4
95\% CI is $2.4 \pm 1.96 $
(iii)
2.4
⇒ 
\[2.4 \pm 0.55 = (1.85, 2.95)\]
30
The CI in (ii) is much narrower due to having more data.
(It is also centred higher due to the larger estimate.)



\end{document}
