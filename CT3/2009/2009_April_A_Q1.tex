\documentclass[a4paper,12pt]{article}

%%%%%%%%%%%%%%%%%%%%%%%%%%%%%%%%%%%%%%%%%%%%%%%%%%%%%%%%%%%%%%%%%%%%%%%%%%%%%%%%%%%%%%%%%%%%%%%%%%%%%%%%%%%%%%%%%%%%%%%%%%%%%%%%%%%%%%%%%%%%%%%%%%%%%%%%%%%%%%%%%%%%%%%%%%%%%%%%%%%%%%%%%%%%%%%%%%%%%%%%%%%%%%%%%%%%%%%%%%%%%%%%%%%%%%%%%%%%%%%%%%%%%%%%%%%%

\usepackage{eurosym}
\usepackage{vmargin}
\usepackage{amsmath}
\usepackage{graphics}
\usepackage{epsfig}
\usepackage{enumerate}
\usepackage{multicol}
\usepackage{subfigure}
\usepackage{fancyhdr}
\usepackage{listings}
\usepackage{framed}
\usepackage{graphicx}
\usepackage{amsmath}
\usepackage{chng%%-- Page}

%\usepackage{bigints}

\usepackage{vmargin}

% left top textwidth textheight headheight

% headsep footheight footskip

\setmargins{2.0cm}{2.5cm}{16 cm}{22cm}{0.5cm}{0cm}{1cm}{1cm}
\renewcommand{\baselinestretch}{1.3}
\setcounter{MaxMatrixCols}{10}
\begin{document}
\begin{enumerate}

%%%%%%%%%%%%%%%%%%%%%%%%%%%%%%%%%%%%%%%%%%%%%%%%%%
  1 An actuary is simulating claims Xi on a portfolio of insurance policies.
For each i, let Yi be 1 if Xi exceeds a given amount M and 0 if not. The variance of
Yi is 0.12.
The actuary wishes to estimate the proportion of claims that exceed M.
Calculate the number of simulations that the actuary will have to perform in order
to estimate the true proportion to within 0.01 with 99% confidence. 



\noindent \textbf{Solutions}\\
Q1
2 2
/2
2
n z ˆ


n > 2.57582 * 2
0.12
0.01
= 7,961.87
so n = 7,962
%%-- Many candidates scored well on this straightforward question, although a disappointing number of candidates were unfamiliar with the theory and so scored poorly.
%%%%%%%%%%%%%%%%%%%%%%%%%%%%%%%%%%%%%%%%%%%%%%%%%
\new%%-- Page 
Q2

\begin{itemize}
\item (i) The model assumes that the mean and standard deviation of the claim amounts
are known with certainty.
\item Model assumes that claims are settled as soon as the incident occurs, with no
delays.

\item No allowance for expenses is made.
No allowance for interest.
(ii) Car insurance, contents insurance (or other similar examples)
\item Part (i) was typically poorly answered, as the majority of candidates gave the
characteristics of insurable risks in general, rather than focusing on short term
contracts.
\item 
Part (ii) was generally well answered.
\end{itemize}
%%%%%%%%%%%%%%%%%%%%%%%%%%%%%%%%%%%%%%%%%%%%%%% – September 2015 – Examiners’ Report

\end{document}
