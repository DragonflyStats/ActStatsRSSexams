\documentclass[a4paper,12pt]{article}

%%%%%%%%%%%%%%%%%%%%%%%%%%%%%%%%%%%%%%%%%%%%%%%%%%%%%%%%%%%%%%%%%%%%%%%%%%%%%%%%%%%%%%%%%%%%%%%%%%%%%%%%%%%%%%%%%%%%%%%%%%%%%%%%%%%%%%%%%%%%%%%%%%%%%%%%%%%%%%%%%%%%%%%%%%%%%%%%%%%%%%%%%%%%%%%%%%%%%%%%%%%%%%%%%%%%%%%%%%%%%%%%%%%%%%%%%%%%%%%%%%%%%%%%%%%%

\usepackage{eurosym}
\usepackage{vmargin}
\usepackage{amsmath}
\usepackage{graphics}
\usepackage{epsfig}
\usepackage{enumerate}
\usepackage{multicol}
\usepackage{subfigure}
\usepackage{fancyhdr}
\usepackage{listings}
\usepackage{framed}
\usepackage{graphicx}
\usepackage{amsmath}
\usepackage{chngpage}

%\usepackage{bigints}
\usepackage{vmargin}

% left top textwidth textheight headheight

% headsep footheight footskip

\setmargins{2.0cm}{2.5cm}{16 cm}{22cm}{0.5cm}{0cm}{1cm}{1cm}

\renewcommand{\baselinestretch}{1.3}

\setcounter{MaxMatrixCols}{10}

\begin{document}
\begin{enumerate}
%%-- Question 7#
\item A scientific investigation involves a linear regression with the usual assumptions that the response variable y follows a normal distribution with mean α + β x and variance 
$\sigma^2$ . Twenty data points were recorded, corresponding to four observations of y at
x = 1, three observations of y at x = 2, six observations of y at x = 3, and seven observations of y at x = 4. The resulting means of these sets of y observations are
given in the table below.
x
no. of y's
mean of y's
8
1
4
18.6
3
6
23.2
4
7
27.1
\begin{enumerate}[(i)]
\item (i) Determine the fitted regression line of y on x.
\item (ii) Suppose that you have been asked to provide a 95% confidence interval for
the slope coefficient.

(a) Comment briefly on any problems you might encounter in the computation of the required confidence interval.
(b) Indicate briefly any further information that you would need in order to overcome these problems.
\end{enumerate}

%%%%%%%%%%%%%%%%%%%%%%%%%%%%%%%%%%%%%%%%%%%%%%%%%%%%%%%%%%%%%%%%%%%%%%%%%%%%%%
7
(i)
We need to calculate the basics sums ∑x, ∑x 2, ∑y, ∑xy
\begin{itemize}
\item n = 20
\item ∑x = 4(1) + 3(2) + 6(3) + 7(4) = 56
\item ∑x 2 = 4(12) + 3(22) + 6(32) + 7(42) = 182
\item ∑y = 4(18.6) + 3(21.7) + 6(23.2) + 7(27.1) = 468.4
\item ∑xy = 1(4)(18.6) + 2(3)(21.7) + 3(6)(23.2) + 4(7)(27.1) = 1381.0
\end{itemize}
%%%%%%%%%%%%%%%%%%%%%%
2
S xy
69.48
25.2
ˆ
ˆ
y
1
(56)(468.4) 69.48 and S xx
20
1381.0
ˆ x
182
1
(56) 2
20
25.2
1
1
2.757
1
[468.4 (2.757)56] 15.7
20
1
y ˆ 15.7 2.757 x
(ii)
(a)
95% CI for β is ˆ t 0.025,18
ˆ 2
S xx
So we need to calculate
ˆ 2
1
n 2
( y i
ˆ
ˆ x ) 2
i
or
1
n 2
( S yy
2
S xy
S xx
) .
1
Problem as we do not have the individual yi values, only means of sets of them.
(b)
We would need these individual yi values (or the s.d. or set).
y 2 for each
2
8
X|Y = 2 takes values 0, 1, 2 with probabilities 1/8, 3/8, 4/8
Page 4
(being in the ratios 1:3:4) 2
So E[X|Y = 2] = 1(3/8) + 2(4/8) = 11/8 = 1.375 1
%% Subject CT3 (Probability and Mathematical Statistics Core Technical) — September 2009 — Marking Schedule
\end{document}
