%%- E

\documentclass[a4paper,12pt]{article}

%%%%%%%%%%%%%%%%%%%%%%%%%%%%%%%%%%%%%%%%%%%%%%%%%%%%%%%%%%%%%%%%%%%%%%%%%%%%%%%%%%%%%%%%%%%%%%%%%%%%%%%%%%%%%%%%%%%%%%%%%%%%%%%%%%%%%%%%%%%%%%%%%%%%%%%%%%%%%%%%%%%%%%%%%%%%%%%%%%%%%%%%%%%%%%%%%%%%%%%%%%%%%%%%%%%%%%%%%%%%%%%%%%%%%%%%%%%%%%%%%%%%%%%%%%%%

\usepackage{eurosym}
\usepackage{vmargin}
\usepackage{amsmath}
\usepackage{graphics}
\usepackage{epsfig}
\usepackage{enumerate}
\usepackage{multicol}
\usepackage{subfigure}
\usepackage{fancyhdr}
\usepackage{listings}
\usepackage{framed}
\usepackage{graphicx}
\usepackage{amsmath}
\usepackage{chngpage}

%\usepackage{bigints}
\usepackage{vmargin}

% left top textwidth textheight headheight

% headsep footheight footskip

\setmargins{2.0cm}{2.5cm}{16 cm}{22cm}{0.5cm}{0cm}{1cm}{1cm}

\renewcommand{\baselinestretch}{1.3}

\setcounter{MaxMatrixCols}{10}

\begin{document}
\begin{enumerate}
%%10
\item For a group of policies the probability distribution of the total number of claims, N,
arising during a period of one year is given by
P(N = 0) = 0.70, P(N = 1) = 0.15, P(N = 2) = 0.10, P(N = 3) = 0.05.
Each claim amount, X (in units of £1,000), follows a gamma distribution with parameters $\alpha = 2$ and $\lambda = 0.1$ independently of each other claim amount and of the number of claims.
Calculate the expected value and the standard deviation of the total of the claim amounts for a period of one year.
%%------ CT3 A2009—4


%%11
\item The number of claims, $X$, which arise in a year on each policy of a particular class is to be modelled as a Poisson random variable with mean $\lambda$. Let $X = (X_1 , X_2 , \ldots, X_n )$ be a random sample from the distribution of X, and let X =
\[1 n
∑ X i .
n i = 1
n\]
\begin{enumerate}[(i)]
\item (i)
(a)
Use moment generating functions to show that
∑ X i has a Poisson
i = 1
distribution with mean $n\lambda$.
(b) State, with a brief reason, whether or not the variable 2X 1 + 5 has a Poisson distribution.
(c) State, with a brief reason, whether or not X has a Poisson distribution in the case that n = 2.
(d) State the approximate distribution of X in the case that n is large.
%%-- [8]
\item An actuary is interested in the level of claims being experienced and wants in particular to test the hypotheses
\[H 0 : \lambda = 1 v H 1 : \lambda > 1 .\]
He decides to use a random sample of size n = 100 and the best (most powerful) available test. You may assume that this test rejects H 0 for $x > k$ , for some constant k.
\item (ii)
(a) Show that the value of k for the test with level of significance 0.01 is $k = 1.2326$.
(b) Calculate the power of the test in part (ii)(a) in the case $\lambda = 1.2$ and then in the case $\lambda = 1.5$.
(c) Comment briefly on the values of the power of the test obtained in part
\item (ii)(b).
\end{enumerate}
\end{enumerate}
%%%%%%%%%%%%%%%%%%%%%%%%%%%%%%%%%%%%%%%%%%%%%%%%%%%%%%%%%%%%%%%%%%%%%%%%%%%%%%%%%%%%%%%%%%%%%%%%%%%%%%%%%%%%%%%%%%%%%%%
\newpage

11
\item (i)
(a)
Let S = ∑ X i
i = 1
{ (
) }
\[M X ( t ) = exp \lambda e t − 1
{ (
) }
n
{ (
) }
n\]
⇒ \[M_S ( t ) = { M_X ( t ) } = ⎡ exp \lambda e t − 1 ⎤ = exp n \lambda e t − 1\]

⇒ S ~ Poisson(n\lambda)
(b)
No
One reason is that $E[2X 1 + 5] = 2\lambda + 5$, which is not equal to $V[2X 1 + 5] = 4\lambda$
[Note: another obvious reason is that 2X 1 + 5 can only takes values 5,
7, 9, ... , not 0, 1, 2, 3,... ]
%%%%%%%%%%%%%%%%%%%%%%%%%%%%%%%%%%%%%%%%%%%%%%%%%%%%%%%%%%%%%%%%%%%%%%%%%%%%%%%%%%%%%%%%%%%%%%%%%%%%%%%%%%%%%%%%%
(c)
No
One reason is that $E \left[ X ⎦\right = \lambda$ , which is not equal to V ⎡ ⎣ X ⎤ ⎦ = \lambda / 2
[Note: another obvious reason is that X can take values 0.5, 1.5, 2.5, ... , which a Poisson variable cannot.]

(d)
\item (ii)
(a)
(b)
⎛ \lambda ⎞
X ≈ N ⎜ \lambda , ⎟
⎝ n ⎠
1 ⎞
⎛

Under H 0 , X ~ N ⎜ 1 ,
⎟ approximately
⎝ 100 ⎠
k − 1

k is such that P ( X > k | H 0 ) = 0.01 so
= 2.3263
0.1
⇒ k = 1.2326
\[Power(\lambda) = P(reject H 0 |\lambda)\]
\[Power(\lambda = 1.2) = P ( X > 1.2326 ) where X ~ N (1.2, 0.012)\]
= P(Z > 0.298) = 0.383

\begin{eqnarray*}
Power(\lambda = 1.5) &=& P ( X > 1.2326 ) where X ~ N (1.5, 0.015)\\
&=& P(Z > −2.183)\\ &=& 0.985\\
\end{eqnarray*}
(c)

\end{document}
