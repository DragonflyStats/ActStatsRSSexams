\documentclass[a4paper,12pt]{article}

%%%%%%%%%%%%%%%%%%%%%%%%%%%%%%%%%%%%%%%%%%%%%%%%%%%%%%%%%%%%%%%%%%%%%%%%%%%%%%%%%%%%%%%%%%%%%%%%%%%%%%%%%%%%%%%%%%%%%%%%%%%%%%%%%%%%%%%%%%%%%%%%%%%%%%%%%%%%%%%%%%%%%%%%%%%%%%%%%%%%%%%%%%%%%%%%%%%%%%%%%%%%%%%%%%%%%%%%%%%%%%%%%%%%%%%%%%%%%%%%%%%%%%%%%%%%

\usepackage{eurosym}
\usepackage{vmargin}
\usepackage{amsmath}
\usepackage{graphics}
\usepackage{epsfig}
\usepackage{enumerate}
\usepackage{multicol}
\usepackage{subfigure}
\usepackage{fancyhdr}
\usepackage{listings}
\usepackage{framed}
\usepackage{graphicx}
\usepackage{amsmath}
\usepackage{chngpage}

%\usepackage{bigints}
\usepackage{vmargin}

% left top textwidth textheight headheight

% headsep footheight footskip

\setmargins{2.0cm}{2.5cm}{16 cm}{22cm}{0.5cm}{0cm}{1cm}{1cm}

\renewcommand{\baselinestretch}{1.3}

\setcounter{MaxMatrixCols}{10}

\begin{document}
\begin{enumerate}
\item 1
In a sample of 100 households in a specific city, the following distribution of number of people per household was observed:
Number of people x
Number of households f x
1
7
2
f 2
3
20
4
f 4
5
18
6
10
7
5
The mean number of people per household was found to be 4.0. However, the
frequencies for two and four members per household (f 2 and f 4 respectively) are
missing.
2
3

\begin{enumerate}[(a)]
\item Calculate the missing frequencies f 2 and f 4 .
\item Find the median of these data, and hence comment on the symmetry of the
data.
\end{enumerate}

%%%%%%%%%%%%%%%%%%%%%%%%%%%%%%%%%%%%%%%%%%%%%%%%%%%%%%%%%%%%%%%%%%%%%%%%%%%%%%%%%%%%
\newpage
1
\item (i)
We have 60
These give f 2
f 4 100 and
f 2
40
4 .
1
f 4 and 2 f 2 4 f 4 148
from which we obtain f 2
\item (ii)
7 2 f 2 60 4 f 4 90 60 35
100
6 and f 4
34 .
1
Median is equal to the midpoint between the 50 th and 51 st ordered observations, i.e. median = 4. 1
We have mean = median, suggesting that the distribution of these data is roughly symmetric. 1
%%%%%%%%%%%%%%%%%%%%%%%%%%%%%%%%%%%%%%%%%%%%%%%%%%%%%%%%%%%%%%%%%%%%%%%%%%%%%%%%%%%%
\end{document}
