13
The following table gives the scores (out of 100) that 10 students obtained on a
midterm test ( x ) and the final examination ( y ) in a course in statistics.
65
44
Midterm x
Final y
62
49
50
54
82
59
80
66
68
67
88
71
67
81
90
89
92
98
For these data you are given: S xx = 1,760.4, S_{yy} = 2,737.6, S xy = 1,529.8
(i)
(a) Draw a scatterplot of the data and comment briefly on the relationship
between the score in the final examination and that in the midterm test.
(b) The equation of the line of best fit is given by y = 3.146 + 0.869 x .
Perform a suitable test involving the slope parameter \beta , to test the null
hypothesis H 0 : \beta  = 0 against H 1 : \beta  > 0.
(c) Calculate a 95% confidence interval for the mean final examination
score for a midterm score of 75.
(d) Consider now that we require a 95% confidence interval for an
individual predicted final examination score for a midterm score of 75.
State (giving reasons) whether this interval will be narrower or wider
than the one calculated in part (i)(c) above. (You are not asked to
calculate the interval.)
[13]
The lecturer of this course decides to assess the linear relationship between the score
in the final examination and that in the midterm test, by using the sample correlation
coefficient r .
The hypothesis H 0 : ρ = 0 (where ρ denotes the population correlation coefficient) can
be tested against H 1 : ρ > 0, by using the result that under H 0 the sampling distribution
of the statistic
(ii)
r n − 2
1 − r 2
is the t n-2 distribution (where n is the size of the sample).
(a) Show algebraically, that is without referring to the specific data given
here, that in general the above statistic and the statistic involving \beta  that
you used in (i)(b) produce equivalent tests.
(b) Calculate the value of r for the given data and hence verify numerically
the result of part (ii)(a) above.

%%%%%%%%%%%%%%%%%%%%%%%%%%%%%%%%%%%%%%%%%%%%%%%%%%%%%%%%%%%%%%%%%%%%%%%%%%%%%%%%%%%%%%%%%%%%%%%%%%%%%5




Page 7Subject CT3 (Probability and Mathematical Statistics Core Technical) — April 2009 — Examiners’ Report
(i)
(a)
The plot is given below.
13
50
60
70
80
90
Midterm score
There seems to be a positive relationship between final and midterm
score. However it is not clear if this relationship is linear.
[Following comment also valid: the relationship looks linear but with
substantial scatter.]
(b)
2
S xy
1 ⎛
⎜ S_{yy} −
σ ˆ =
n − 2 ⎜
S xx
⎝
2
s.e. ( \beta  ˆ ) =
⎞ 1 ⎛
(1529.8) 2 ⎞
⎟ = ⎜ 2737.6 −
⎟ = 176.0241
⎟ 8 ⎜ ⎝
1760.4 ⎟ ⎠
⎠
σ ˆ 2
176.0241
=
= 0.3162
1760.4
S xx
To test H 0 : \beta  = 0 v H 1 : \beta  > 0 , the test statistic is
0.869
\beta  ˆ − 0
=
= 2.748 ,
s.e.( \beta  ˆ ) 0.3162
and under the assumption that the errors of the regression are
i.i.d . N (0, σ 2 ) random variables, it has a t distribution
with n - 2 = 8 df.
Page 8Subject CT3 (Probability and Mathematical Statistics Core Technical) — April 2009 — Examiners’ Report
From statistical tables we find t 8,0.05 = 1.860 , and t 8,0.01 = 2.896 .
We reject the hypothesis H 0 : \beta  = 0 in favour of H 1 : \beta  > 0 at the 5%
level (but not at the 1% level).
(c)
The mean response, y ˆ new for x ˆ new = 75 is
y ˆ new = 3.146 + 0.869 × 75 = 68.321 .
Its standard error is calculated as
1 ( x new − x ) 2
+
n
S xx
s.e.( y ˆ new ) = σ ˆ
= 13.26741
1 (75 − 74.4) 2
+
= 13.26741 × 0.31655
10
1760.4
= 4.1998
The 95% CI is given by y ˆ new ± t 0.025,8 × s.e.( y ˆ new ) ,
i.e. 68.321 ± 2.306 × 4.1998 = 68.321 ± 9.6847
or (58.636, 78.006)
(ii)
(d) The CI for the individual predicted score will be wider than the CI for
the mean score in (i)(c), because the variance for the individual
predicted value is larger.
(a) Both statistics follow a t n - 2 distribution under the null hypothesis.
In addition
\beta  ˆ − 0
σ ˆ 2
S xx
(b)
r =
Then
S xy
S xx
=
2
⎛
S xy
1
⎜ S_{yy} −
( n − 2) S xx ⎜
S xx
⎝
S xy
S xx S_{yy}
1 − r
2
S_{yy} 1 −
S xy
S xx
2
S xy
=
r n − 2
1 − r 2
S xx S_{yy}
1529.8
= 0.6969.
1760.4 × 2737.6
=
r n − 2
⎞
⎟
⎟
⎠
=
( n − 2) S xx
=
0.6969 × 8
1 − 0.6969
2
= 2.748 , same as in (i)(b).
END OF EXAMINERS’ REPORT
Page 9
