%%- D

\documentclass[a4paper,12pt]{article}

%%%%%%%%%%%%%%%%%%%%%%%%%%%%%%%%%%%%%%%%%%%%%%%%%%%%%%%%%%%%%%%%%%%%%%%%%%%%%%%%%%%%%%%%%%%%%%%%%%%%%%%%%%%%%%%%%%%%%%%%%%%%%%%%%%%%%%%%%%%%%%%%%%%%%%%%%%%%%%%%%%%%%%%%%%%%%%%%%%%%%%%%%%%%%%%%%%%%%%%%%%%%%%%%%%%%%%%%%%%%%%%%%%%%%%%%%%%%%%%%%%%%%%%%%%%%

\usepackage{eurosym}
\usepackage{vmargin}
\usepackage{amsmath}
\usepackage{graphics}
\usepackage{epsfig}
\usepackage{enumerate}
\usepackage{multicol}
\usepackage{subfigure}
\usepackage{fancyhdr}
\usepackage{listings}
\usepackage{framed}
\usepackage{graphicx}
\usepackage{amsmath}
\usepackage{chngpage}

%\usepackage{bigints}
\usepackage{vmargin}

% left top textwidth textheight headheight

% headsep footheight footskip

\setmargins{2.0cm}{2.5cm}{16 cm}{22cm}{0.5cm}{0cm}{1cm}{1cm}

\renewcommand{\baselinestretch}{1.3}

\setcounter{MaxMatrixCols}{10}

\begin{document}
\begin{enumerate}
\item 
An analysis of variance investigation with samples of size eight for each of four
treatments results in the following ANOVA table.
\begin{verbatim}
Source of variation d.f. SS MSS
Between treatments
Residual
Total 3
28
31 6716
3362
10078 2239
120
\end{verbatim}
\begin{enumerate}[(i)]
\item (i) Calculate the observed F statistic, specify an interval in which the resulting P-value lies, and state your conclusion clearly.
\item (ii) The four treatment means are:
y 1 = 85.0, y 2 = 66.5, y 3 = 59.0, and y 4 = 95.5 .
(a) Calculate the least significant difference between pairs of means using a 5\% level.
(b) List the means in order, illustrate the non-significant pairs using suitable underlining, and comment briefly.
\end{enumerate}
%%%%%%%%%%%%%%%%%%%%%%%%%%%%%%%%%%%%%%%%%%%%%%%%%%%%%%%%%%%%%%%

9
(i)
F =
2239
= 18.66 on 3,28 d.f.
120
from tables F 3,28 (1%) = 4.568
∴ P-value < 0.01 or 0 < P-v < 0.01
So there is overwhelming evidence of a difference between the underlying treatment means.
%%%%%%%%%%%%%%%%%%%%%%%%%%%%%%%%%%%%%%%%%
(ii)
(a) 1 1
1 1
LSD = t 28 (2.5%) \sigma  ˆ 2 ( + ) = 2.048 120( + ) = 11.2
8 8
8 8
(b) means in order
y 3. < y 2. < y 1. < y 4.
underlined thus:
Treatments 2 & 3 are separate from treatments 1 & 4 which have significantly higher means.
%--------------------------------------%
10
\[E [ N ] = 0.7(0) + 0.15(1) + 0.1(2) + 0.05(3) = 0.5\]
\[E ⎡ N 2 ⎤ = 0.15(1) + 0.1(4) + 0.05(9) = 1.00]\ ∴ Var [ N ] = 1.00 − 0.5 2 = 0.75

E [ X ] =
2
2
= 20, Var [ X ] = 2 = 200
0.1
0.1
Let S = total of the claim amounts.
\[E [ S ] = E [ N ] E [ X ] = (0.5)(20) = 10 ,\] i.e. \$10,000.
V [ S ] = E [ N ] Var [ X ] + Var [ N ] [ E ( X )] 2 = (0.5)(200) + (0.75)(20) 2 = 400
∴ sd ( S ) = 20 , i.e. \$20,000.
n
%--------------------------------------%
\end{document}
