
\documentclass[a4paper,12pt]{article}

%%%%%%%%%%%%%%%%%%%%%%%%%%%%%%%%%%%%%%%%%%%%%%%%%%%%%%%%%%%%%%%%%%%%%%%%%%%%%%%%%%%%%%%%%%%%%%%%%%%%%%%%%%%%%%%%%%%%%%%%%%%%%%%%%%%%%%%%%%%%%%%%%%%%%%%%%%%%%%%%%%%%%%%%%%%%%%%%%%%%%%%%%%%%%%%%%%%%%%%%%%%%%%%%%%%%%%%%%%%%%%%%%%%%%%%%%%%%%%%%%%%%%%%%%%%%

\usepackage{eurosym}
\usepackage{vmargin}
\usepackage{amsmath}
\usepackage{graphics}
\usepackage{epsfig}
\usepackage{enumerate}
\usepackage{multicol}
\usepackage{subfigure}
\usepackage{fancyhdr}
\usepackage{listings}
\usepackage{framed}
\usepackage{graphicx}
\usepackage{amsmath}
\usepackage{chngpage}

%\usepackage{bigints}
\usepackage{vmargin}

% left top textwidth textheight headheight

% headsep footheight footskip

\setmargins{2.0cm}{2.5cm}{16 cm}{22cm}{0.5cm}{0cm}{1cm}{1cm}

\renewcommand{\baselinestretch}{1.3}

\setcounter{MaxMatrixCols}{10}

\begin{document}
\large
\noindent In a large portfolio 65\% of the policies have been in force for more than five years. An investigation considers a random sample of 500 policies from the portfolio.


 \large
\noindent Calculate an approximate value for the probability that fewer than 300 of the policies in the sample have been in force for more than five years.
\medskip
%%%%%%%%%%%%%%%%%%%%%

\noindent \textbf{Solution}\\
%%- question 4
Let $X$ be the number in force for more than five years
then $X \sim \mbox{ binomial}(500,0.65)$
Using a normal approximation, $X \sim N(325, 10.665^2 )$
\begin{framed}
\noindent Continuity Corrections:
\begin{itemize}
\item If ${ \displaystyle P(X \;=\;n) }$ use ${ \displaystyle P(n\;-\;0.5 \;<\; X \;<\;n\;+\;0.5) }$
\item If ${ \displaystyle P(X \;>\;n) }$ use ${ \displaystyle P(X \;>\;n\;+\;0.5) }$
\item If ${ \displaystyle P(X \;\leq\;n) }$ use ${ \displaystyle P(X \;<\;n\;+\;0.5) }$
\item If ${ \displaystyle P(X \;<\;n) }$ use  ${ \displaystyle P(X \;<\;n\;-\;0.5) }$
\item If ${ \displaystyle P(X \;\geq\;n) }$ use $ { \displaystyle P(X \;>\;n \;-\; 0.5) }$
\end{itemize}
\end{framed}
$P(X < 300)$ becomes $P(X < 299.5)$ using continuity correction

%%-- ) where Z ~ N(0,1)
\begin{eqnarray*} 
P(X < 299.5) &=& P \left( Z < \frac{299.5 \;-\; 325}{10.665} \right)\\
&=& P ( Z < \;-\; 2.39)\\
&=& 1 \;-\; 0.99158 \\ 
&=& 0.0084\\
\end{eqnarray*}

%%-- Page 4  — September 2006 — 

%%%%%%%%%%%%%%%%%%%%%%%%%%%%%%%%%%%%%%%%%%%%%%%%%%%%%%%%%%%%%%%%%%%%%%%%%%%%%%%
\end{document}
