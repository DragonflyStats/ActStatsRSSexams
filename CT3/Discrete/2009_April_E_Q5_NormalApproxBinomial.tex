

\documentclass[a4paper,12pt]{article}

%%%%%%%%%%%%%%%%%%%%%%%%%%%%%%%%%%%%%%%%%%%%%%%%%%%%%%%%%%%%%%%%%%%%%%%%%%%%%%%%%%%%%%%%%%%%%%%%%%%%%%%%%%%%%%%%%%%%%%%%%%%%%%%%%%%%%%%%%%%%%%%%%%%%%%%%%%%%%%%%%%%%%%%%%%%%%%%%%%%%%%%%%%%%%%%%%%%%%%%%%%%%%%%%%%%%%%%%%%%%%%%%%%%%%%%%%%%%%%%%%%%%%%%%%%%%

\usepackage{eurosym}
\usepackage{vmargin}
\usepackage{amsmath}
\usepackage{graphics}
\usepackage{epsfig}
\usepackage{enumerate}
\usepackage{multicol}
\usepackage{subfigure}
\usepackage{fancyhdr}
\usepackage{listings}
\usepackage{framed}
\usepackage{graphicx}
\usepackage{amsmath}
\usepackage{chngpage}

%\usepackage{bigints}
\usepackage{vmargin}

% left top textwidth textheight headheight

% headsep footheight footskip

\setmargins{2.0cm}{2.5cm}{16 cm}{22cm}{0.5cm}{0cm}{1cm}{1cm}

\renewcommand{\baselinestretch}{1.3}

\setcounter{MaxMatrixCols}{10}

\begin{document}
\large

\begin{framed}
\noindent A continuity correction is an adjustment that is made when a discrete distribution is approximated by a continuous distribution.
\\ 
\medskip

\noindent If a random variable X has a binomial distribution with parameters $n$ and $p$, i.e., $X$ is distributed as the number of ``successes" in n independent Bernoulli trials with probability $p$ of success on each trial, then

    \[ {\displaystyle P(X\leq x) \;\approx \; P(X<x+1)} \]

\noindent for any $x \in \{0, 1, 2, \ldots n\}$. If $np$ and $np(1 - p)$ are large (sometimes taken to mean $\geq 5$), then we can say

    \[ {\displaystyle P(X\leq x) \;\approx \; P(Y\leq x \;+\;1/2)} \]
    \[ {\displaystyle P(X\geq x) \;\approx \; P(Y\geq x \;-\;1/2)} \]

\noindent where $Y$ is a normally distributed random variable with the same expected value and the same variance as $X$, i.e., $E(Y) = np$ and $var(Y) = np(1 \;-\; p)$. This addition (or subtraction) of 1/2 to $x$ is the continuity correction.
\end{framed}
\newpage

\noindent The human resources department of a large insurance company currently estimates that 82\% of new employees recruited by their call centres will still be employed by the company after one year. A recent extension to the call centre business led to 280
new employees being recruited.
\medskip 

\noindent Calculate an approximate value for the probability that at least 240 of these new employees will still be employed by the company after one year.
%%%%%%%%%%%%%%%%%%%%%%%%%%%%%%%%%%%%%%%%%5

%%%%%%%%%%%%%%%%%%%%%%%%%%%%%%%%%%%%%%%%%%%%%%%%%%%%%%%%%%%%%%%%%%%%%%%%%%%%%%%%%%%%%%%%
\medskip 
%%- Question 5
\begin{itemize}
\item Let X be the number still in employment after one year.

\item $X$ is a binomial random variables with parameters $n \;=\; 280$, $p \;=\;0.82$ 

\[X \sim  \mbox{bin} (280, 0.82) \]
\begin{itemize}
\item[$\bullet$] The binomial mean is $np= 280 \times 0.82 \;=\; 229.6$ .
\item[$\bullet$] The  variance is $np(1-p) \;=\;280 \times 0.82 \times 0.18 \;=\; 41.328$.
\item[$\bullet$] The standard deviation is therefore $\sqrt{41.328} = 6.43$
\end{itemize}

\item Using the normal distribution to approximate $X$
\[X \sim N (229.6, 6.43^2 )\]


\item  applying a continuity correction
\[P ( X \geq 240) = P ( X > 239.5)\]
\end{itemize}
%%%%%%%%%%%%%%%%%%%%%%%%%%%%%

\begin{eqnarray*}
P ( X > 239.5) &=& P\left(Z > \left[ \frac{239.5 \;-\; 229.6}{6.43} \right] \right) \\
&=&  P ( Z > 1.54) \\ 
&=&  1 \;-\; \Phi(1.54) \\
&=& 1 \;-\; 0.93822 \\
&=& 0.062 \\
\end{eqnarray*}


\end{document}
