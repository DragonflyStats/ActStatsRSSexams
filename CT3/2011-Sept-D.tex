
PLEASE TURN OVER7
The total amounts y ij (in £ millions) paid out under a certain type of policy issued by
four different companies A, B, C, D in each of six consecutive years were as follows:
Company
A
B
C
D
2.870
3.105
2.800
2.830
3.125
3.200
2.985
2.600
3.000
3.300
3.060
2.765
2.865
2.975
2.900
2.690
2.890
3.210
2.920
2.600
3.060
3.150
3.050
2.700
Total
17.810
18.940
17.715
16.185
For these data, Σ i Σ j y ij = 70.650 and Σ i Σ j y ij 2 = 208.828.
Consider the ANOVA model Y ij = μ + τ i + e ij , i = 1, ..., 4, j = 1, ..., 6, where Y ij is the
jth amount paid out by company i, and e ij ~ N(0, σ 2 ) are independent errors.
The ANOVA table for these data is given below.
Source
Company (between treatments)
Residual
Total
DF
3
20
23
SS
0.640
0.212
0.852
MS
0.213
0.0106
(i) Test the hypothesis that there are no differences in the means of the amounts
paid out under such policies by the four companies (the company means),
stating your conclusions clearly.
[2]
(ii) Comment briefly on the validity of the test performed in (i), using the plot of
the residuals given below.
[2]
(iii) (a)
Calculate the least significant difference between pairs of company
means using a 5% significance level.
(b)
List the company means in order, illustrate the non-significant pairs
using suitable underlining, and comment briefly.
[6]
[Total 10]
CT3 S2011—4

%%%%%%%%%%%%%%%%%%%%%%%%%%%%%%%%%%%%%%%%%%%%%%%
8
(i)
(ii)
1
∑ E [ X i ] = λ
n i
1
1
V [ λ ˆ ] = 2 ∑ i V [ X i ] = λ (using independence of X i )
n
n
E [ λ ˆ ] =
P [0.2 ≤ λ ˆ ≤ 0.3] = P [2 ≤ 10 λ ˆ ≤ 3]
= F(3; λ = 2.5) − F(1; λ = 2.5) = 0.75758 − 0.28730 = 0.47028
(iii)
(a)
10 λ ˆ = ∑ i = 1 X i ~ N (2.5, 2.5) approximately.
10
With continuity correction:
P [0.2 ≤ λ ˆ ≤ 0.3] = P [2 ≤ 10 λ ˆ ≤ 3] ≈ P [2 − 0.5 ≤ 10 λ ˆ ≤ 3 + 0.5]
3.5 − 2.5 ⎤
⎡ 1.5 − 2.5
⎛ 1 ⎞
= P ⎢
≤ Z ≤
⎟ − 1
⎥ = 2* F Z ⎜
2.5 ⎦
⎣ 2.5
⎝ 2.5 ⎠
= 2* F Z (0.63246) − 1 = 2 * 0.73565 − 1 = 0.4713
(b)
10 λ ˆ = ∑ i = 1 X i = Y ≈ N (2.5, 2.5) approximately.
10
Without continuity correction:
3 − 2.5 ⎤
⎡ 2 − 2.5
P [0.2 ≤ λ ˆ ≤ 0.3] = P [2 ≤ 10 λ ˆ ≤ 3] ≈ P ⎢
≤ Z ≤
⎥
2.5 ⎦
⎣ 2.5
⎛ 0.5 ⎞
= 2*F Z ⎜
⎟ − 1 = 2*F Z (0.32) − 1 = 2*0.62552 − 1 = 0.2510
⎝ 2.5 ⎠
(iv)
When compared to the exact probability in (ii) the results in (iii) (a) and (b)
show that the continuity correction reduces the approximation error
significantly for this small sample size.
(v)
0.3 − 0.25 ⎤
⎡ 0.2 − 0.25
⎛ 0.05 ⎞
P [0.2 ≤ λ ˆ ≤ 0.3] ≈ P ⎢
≤ z ≤
⎟ − 1 = 0.95
⎥ = 2* F Z ⎜
0.25 / n ⎦
⎣ 0.25 / n
⎝ 0.25 / n ⎠
1.95
0.05
n , and
= F ( z ), then z = 1.96 =
2
0.25
Page 6
n = 1.96
0.5
= 1.96, and n ≈ 384
0.05Subject CT3 (Probability and Mathematical Statistics) — September 2011 — Examiners’ Report
(vi)
Using the normal approximation we find:
0.27 ± z 0.975
λ ˆ
0.27
= 0.27 ± 1.96
= 0.27 ± 0.05092 = [021908, 0.32092]
n
20
In part (i) independence must be mentioned for full marks in the derivation of the variance. In
(ii) most candidates either went straight to a normal approximation, or incorrectly calculated
the Poisson probability. In part (iii) many candidates applied the continuity correction
wrongly.
