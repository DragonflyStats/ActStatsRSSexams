
\documentclass[a4paper,12pt]{article}

%%%%%%%%%%%%%%%%%%%%%%%%%%%%%%%%%%%%%%%%%%%%%%%%%%%%%%%%%%%%%%%%%%%%%%%%%%%%%%%%%%%%%%%%%%%%%%%%%%%%%%%%%%%%%%%%%%%%%%%%%%%%%%%%%%%%%%%%%%%%%%%%%%%%%%%%%%%%%%%%%%%%%%%%%%%%%%%%%%%%%%%%%%%%%%%%%%%%%%%%%%%%%%%%%%%%%%%%%%%%%%%%%%%%%%%%%%%%%%%%%%%%%%%%%%%%

\usepackage{eurosym}
\usepackage{vmargin}
\usepackage{amsmath}
\usepackage{graphics}
\usepackage{epsfig}
\usepackage{enumerate}
\usepackage{multicol}
\usepackage{subfigure}
\usepackage{fancyhdr}
\usepackage{listings}
\usepackage{framed}
\usepackage{graphicx}
\usepackage{amsmath}
\usepackage{chngpage}

%\usepackage{bigints}
\usepackage{vmargin}

% left top textwidth textheight headheight

% headsep footheight footskip

\setmargins{2.0cm}{2.5cm}{16 cm}{22cm}{0.5cm}{0cm}{1cm}{1cm}

\renewcommand{\baselinestretch}{1.3}

\setcounter{MaxMatrixCols}{10}

\begin{document}
\begin{enumerate}
11
An actuary has been advised to use the following positively-skewed claim size distribution as a model for a particular type of claim, with claim sizes measured in units of £100,
f ( x ; )
x 2
2
3
exp
x
: 0
x
,
with moments given by E[X] = 3 , E[X 2 ] = 12
0
2
and E[X 3 ] = 60 3 .
\begin{enumerate}
\item (i) Determine the variance of this distribution and calculate the coefficient of skewness.
\item $(ii) Let $\{X_1 , X 2 ,, X n\}$ be a random sample of n claim sizes for such claims. Show that the maximum likelihood estimator (MLE) of is given by
and show that it is unbiased for .
\item (iii)
A sample of $n = 50$ claim sizes yields x i
X
3

313.6 and x i 2
2, 675.68 .
(a) Calculate the MLE.
(b) Calculate the sample variance and comment briefly on its comparison with the variance of the distribution evaluated at .
(c) Given that the sample coefficient of skewness is 1.149, comment briefly on its comparison with the coefficient of skewness of the distribution.
%
\item (iv)
12
(a) Write down a large-sample approximate 95\% confidence interval for the mean of the distribution in terms of the sample mean x and the sample variance $s^2$ . Hence obtain an approximate 95\% confidence interval for and evaluate this for the data in part (iii) above.
(b) Evaluate the variance of the distribution at both the lower and upper limits of this confidence interval and comment briefly with reference to your answer in part (iii)(b) above.
\end{itemize}

%%%%%%%%%%%%%%%%%%%%%%%%%%%%%%%%%%%%%%%%%%%%%%%%%%%%%%%%%%%%%%%%%%%%%%%%%%%%%%%%%%%%%%%%%%%%%%%%%
%%Page 8Subject CT3 (Probability and Mathematical Statistics Core Technical)
\newpage 
11
(i)
2
= E[X 2 ]
[E(X)] 2 = 12
2
= E[X 3 ] 3 E[X 2 ] + 2
3
= 60
(3 ) 2 = 3
%-----------------%
2
3
3(3 )(12 2 ) + 2(3 ) 3
= (60
108 + 54)
3
3
= 6
coefficient of skewness =
3
3
3
6
1.155
2 3
( 3
)
[OR: note that X ~ gamma(3,1/ ) and use formulae in tables
2
so var = 3 2 and coef. of skew. =
]
3
n
(ii)
L ( )
i 1
x i 2
2
exp(
3
x i
x i 2
)
2
n 3 n
log L ( ) log( x i 2 ) n log 2 3 n log
x i
x i
3 n
log L ( )
x i
exp
2
equate to zero:
x i
3 n
x i
3 n
2
2
this clearly maximises L( )
X i
3 n
So MLE is
(iii)
E 1
E X
3
(a) x
(b) s 2
2
2
log L ( ) ]
X
3
1
E X
3
313.6
50
[or consider
6.272
1
3
3
unbiased
6.272
3
2.091
1
313.6 2
(2675.68
) 14.465
49
50
3
2
and 3
2
13.117
s 2 is a bit larger but still quite close

%------------------------------------------------%
(iv)

(c) sample coefficient 1.149 is very close to the distribution value 1.155
(a) s 2
is x 1.96
n
approximate 95\% CI for
as
= 3 , divide by 3 for an approximate 95\% CI for
1
s 2
x 1.96
3
n
for data:
1
14.465
6.272 1.96
3
50
2.091 0.351
(b)
2
= 3
2
or
(1.740, 2.442)
= 9.083 at lower limit of 1.740
= 17.890 at upper limit of 2.442

$s^2 = 14.465$ is well within these values confirming that $s^2$ is quite close to 3 2 .

\end{document}
