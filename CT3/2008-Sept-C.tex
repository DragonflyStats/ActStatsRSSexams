7
Let N be the number of claims arising on a group of policies in a period of one week
and suppose that N follows a Poisson distribution with mean 60.
Let X 1 , X 2 , . . , X N be the corresponding claim amounts and suppose that,
independently of N, these are independent and identically distributed with mean £500
and standard deviation £400.
N
Let S   X i be the total claim amount for the period of one week.
i  1
(i) Determine the mean and the standard deviation of S.
(ii) Explain why the distribution of S can be taken as approximately normal, and
hence calculate, approximately, the probability that S is greater than £40,000.
[3]
[Total 5]
CT3 S2008—3
[2]
PLEASE TURN OVER8
(i)
Use the following uniform(0,1) random numbers
0.9236 , 0.2578
and a suitable table of probabilities to simulate two observations of the random
variable X, where X ~ N(200,100).
[3]
(ii)
Use the following uniform(0,1) random numbers
0.3287 , 0.9142
to simulate two observations of the random variable Y, where Y has an
exponential distribution with mean 100.
[3]


%%%%%%%%%%%%%%%%%%%%%%%%%%%%%%%%%%%%%%%%%%%%%%%%%%%%%%%%%%%%%%%%%%%%%%%%%%%%%%%%%%%%%%%%%
7
(i)
E ( S ) = E ( N ) E ( X ) = (60)(500) = £30, 000
V ( S ) = E ( N ) V ( X ) + V ( N )[ E ( X )] 2
= (60)(400 2 ) + (60)(500 2 ) = 24, 600, 000 ∴ sd ( S ) = £4,960
(ii)
As S is the sum of a large number of i.i.d. variables, then the central limit
theorem gives an approximate normal distribution for S.
P ( S > 40000) = P ( Z >
40000 − 30000
= 2.016)
4960
= 1 − 0.9781 = 0.0219
[Note: 2.02 leading to 0.0217 is also acceptable.]
8
(i)
From Yellow Book Table
P(Z < 1.43) = 0.9236 giving x value (10*1.43) + 200 = 214.3
P(Z < −0.65) = 0.2578 giving x value (10*(−0.65)) + 200 = 193.5
(ii)
Setting r = P(Y < y) = 1 – exp(−y/100) ⇒ y = −100*log(1 − r)
r = 0.3287 ⇒ y = −100log(0.6713) = 39.85
r = 0.9142 ⇒ y = −100log(0.0858) = 245.6
Note: We can do away with the step of subtracting r from 1 and use.
y = −100*log(r). This gives y = 111.3, 8.971.
