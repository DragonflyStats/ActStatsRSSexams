\documentclass[a4paper,12pt]{article}



%%%%%%%%%%%%%%%%%%%%%%%%%%%%%%%%%%%%%%%%%%%%%%%%%%%%%%%%%%%%%%%%%%%%%%%%%%%%%%%%%%%%%%%%%%%%%%%%%%%%%%%%%%%%%%%%%%%%%%%%%%%%%%%%%%%%%%%%%%%%%%%%%%%%%%%%%%%%%%%%%%%%%%%%%%%%%%%%%%%%%%%%%%%%%%%%%%%%%%%%%%%%%%%%%%%%%%%%%%%%%%%%%%%%%%%%%%%%%%%%%%%%%%%%%%%%
  
  
  
\usepackage{eurosym}
\usepackage{vmargin}
\usepackage{amsmath}
\usepackage{graphics}
\usepackage{epsfig}
\usepackage{enumerate}
\usepackage{multicol}
\usepackage{subfigure}
\usepackage{fancyhdr}
\usepackage{listings}
\usepackage{framed}
\usepackage{graphicx}
\usepackage{amsmath}
\usepackage{chngpage}


%\usepackage{bigints}

\usepackage{vmargin}



% left top textwidth textheight headheight



% headsep footheight footskip



\setmargins{2.0cm}{2.5cm}{16 cm}{22cm}{0.5cm}{0cm}{1cm}{1cm}



\renewcommand{\baselinestretch}{1.3}



\setcounter{MaxMatrixCols}{10}



\begin{document}

\begin{enumerate}

%%%%%%%%%%%%%%%%%%%%%%%%%%%%%%%%%%%%%%%%%%%%%%%%%%%%%%%%%%%%%%%%%%%%%%%%%%%%%%%%%
\item 1 Two groups of students sat the same exam. The marks in the first group of 64 students had an average of 52 and a standard deviation of 9. The marks in the second group of 42 students had an average of 45 and a standard deviation of 8.
Calculate the average and standard deviation of the combined data set of 106 students.

%%%%%%%%%%%%%%%%%%%%%%%%%%%%%%%%%%%%%%%%%%%%%%%%%%%%%%%%%%%%%%%%%%%%%%%%%%%%%%%%%%%%%%%%%%%%5
\item 2 A random sample of size n consists of k distinct observations $x_1, x_2, \ldots, x_k$ which have
been observed with frequencies f1, f2, …, fk where k 1
n i fi . Consider the
deviations of x from a constant A, giving the observations di= xi A for i = 1, …, k.
Show that the sample variance of the xi values is given by:
  2 2 2
( 1 ( 1 ) / ) / ( 1) k k
sx  i fidi  i fidi n n  .

%%%%%%%%%%%%%%%%%%%%%%%%%%%%%%%%%%%%%%%%%%%%%%%%%%%%%%%%%%%%%%%%%%%%%%%%%%%%%%%%%%%%%%%%%%%%
\item 3 Assume that in a large portfolio of insurance contracts the claim size is a normally distributed random variable with expected value 1000. Also assume that the number of claims is a random variable following a Poisson distribution with parameter
 = 400.

\begin{enumerate}[(i)]
\item (i) Calculate the expected value of the total claim amount from contracts in this portfolio. 
\item (ii) Calculate a lower limit for the standard deviation of the total amount of claims from contracts in this portfolio. 
\end{enumerate}
\end{enumerate}
%%%%%%%%%%%%%%%%%%%%%%%%%%%%%%%%%%%%%%%%%%%%%%%%%%%%%%%%%%%%%%%%%%%%%%%%%%%%%%%%%%%%%%%%%%%
\newpage

Page 3
1 i1 1 1 3328
i
x  n x  and i2 2 2 1890
i
x  n x  giving i 5218
i
x 
5218 49.23
106
x 
2
2 2
i1 ( 1 1) 1 i1 / 1 178159
i i
x n s x n
 
    
 
 
2
2 2
i2 ( 2 1) 2 i2 / 2 87674
i i
x n s x n
 
    
 
  giving i2 265833
i
x 
2
2
265833 5218
106
85.4244
105
s
 
  
    and s = 9.243
Generally well answered, although some problems were encountered with the variance.
2 2  2  
1
/ 1
k
x i i
i
s f x x n

  
and xi  x  xi  x  A A  (xi  A)  x  A  di  d
since
( )
i i i i f d f x A
d xA
n n

     
This gives      
2
2 2 2
1 1 1
/ 1 / / 1
k k k
x i i ii ii
i i i
s f d d n fd fd n n
  
                    
  
Overall performance was poor. Many answers completely ignored the involved frequencies.
This is an example of a question that is not examined frequently and candidates found
challenging.
%%%%%%%%%%%%%%%%%%%%%%
\newpage

3 Let X be the size of an individual claim, and N be the number of claims.
(i) Expected total amount is EX EN 1,000  400  400,000

Page 4
(ii) Var total amount  ENV X V NEX 2
A lower bound for the variance is then obtained by assuming V X   0 , that
is,
STDtotal amount  V N EX 2  201000 = 20,000
The first part was answered very well. In part (ii) many candidates failed to recognise that
the answer relies on the variance being equal to zero.
\end{document}
