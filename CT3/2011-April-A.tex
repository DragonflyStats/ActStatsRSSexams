\documentclass[a4paper,12pt]{article}

%%%%%%%%%%%%%%%%%%%%%%%%%%%%%%%%%%%%%%%%%%%%%%%%%%%%%%%%%%%%%%%%%%%%%%%%%%%%%%%%%%%%%%%%%%%%%%%%%%%%%%%%%%%%%%%%%%%%%%%%%%%%%%%%%%%%%%%%%%%%%%%%%%%%%%%%%%%%%%%%%%%%%%%%%%%%%%%%%%%%%%%%%%%%%%%%%%%%%%%%%%%%%%%%%%%%%%%%%%%%%%%%%%%%%%%%%%%%%%%%%%%%%%%%%%%%

\usepackage{eurosym}
\usepackage{vmargin}
\usepackage{amsmath}
\usepackage{graphics}
\usepackage{epsfig}
\usepackage{enumerate}
\usepackage{multicol}
\usepackage{subfigure}
\usepackage{fancyhdr}
\usepackage{listings}
\usepackage{framed}
\usepackage{graphicx}
\usepackage{amsmath}
\usepackage{chngpage}

%\usepackage{bigints}
\usepackage{vmargin}

% left top textwidth textheight headheight

% headsep footheight footskip

\setmargins{2.0cm}{2.5cm}{16 cm}{22cm}{0.5cm}{0cm}{1cm}{1cm}

\renewcommand{\baselinestretch}{1.3}

\setcounter{MaxMatrixCols}{10}

\begin{document}
\begin{enumerate}
1
The numbers of claims which have arisen in the last twelve years on 60 motor policies
(continuously in force over this period) are shown (sorted) below:
0 0 0 0 0 0 0 0 0 0 0 0 1 1 1 1 1 1 1 1
1 1 1 1 1 1 2 2 2 2 2 2 2 2 2 2 2 3 3 3
3 3 3 3 3 3 3 4 4 4 4 4 5 5 5 5 6 6 6 7
Derive:
(i)
(ii)
(iii)
2
The sample median, mode and mean of the number of claims.
The sample inter-quartile range of the number of claims.
The sample standard deviation of the number of claims.
[3]
[2]
[3]
[Total 8]
A random sample of size n = 36 has sample standard deviation s = 7.
Calculate, approximately, the probability that the mean of this sample is greater than
[3]
44.5 when the mean of the population is μ = 42 .


1
(i)
30 th and 31 st observations in order are both 2 ⇒ median = 2
mode = value with highest frequency = 1
Σx = 1(14) + 2(11) + 3(10) + 4(5) + 5(4) + 6(3) + 7(1) = 131
⇒ mean = 131/60 = 2.18
(ii)
Lower quartile is 15.5 th observation counting from below = 1
Upper quartile is 15.5 th observation counting from above = 3
⇒ IQR = 2
(iii)
Σx 2 = 1(14) + 4(11) + 9(10) + 16(5) + 25(4) + 36(3) + 49(1) = 485
⇒ standard deviation = [(485 – 131 2 /60)/59] 1/2 = 3.3726 1/2 = 1.84
2
⎛ X − μ 44.5 − 42 ⎞
P ( X > 44.5) = P ⎜
>
⎟
7 / 36 ⎠
⎝ S / n
⇒ P ( X > 44.5) ≈ P ( Z > 2.143) , where Z ~ N (0,1) ,
and from tables,
P ( X > 44.5) = 1 − 0.984 = 0.016
(A t 35 distribution can also be used if a normal distribution is assumed for the data.)
Page 2Subject CT3 (Probability and Mathematical Statistics Core Technical) — Examiners’ Report, April 2011
