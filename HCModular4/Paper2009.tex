1 HC Module 3 2015
This examination paper consists of 8 printed pages.
This front cover is page 1.
Question 1 starts on page 2.
There are 4 questions altogether in the paper.
© RSS 2015
EXAMINATIONS OF THE HONG KONG STATISTICAL SOCIETY
HIGHER CERTIFICATE IN STATISTICS, 2015
MODULE 3 : Basic statistical methods
Time allowed: One and a half hours
Candidates should answer THREE questions.
Each question carries 20 marks.
The number of marks allotted for each part-question is shown in brackets.
Graph paper and Official tables are provided.
Candidates may use calculators in accordance with the regulations published in
the Society's "Guide to Examinations" (document Ex1).
The notation log denotes logarithm to base e.
Logarithms to any other base are explicitly identified, e.g. log10.
Note also that
 
n
r
is the same as
nCr
.
2
1. On a factory production line it is important that the time taken to assemble a
component is within certain limits. Assembly times (in minutes) for a single
component are recorded for a random sample of twelve factory workers and the values
are as follows.
11.7 12.8 9.9 10.6 11.6 10.6 13.1 11.2 11.6 11.9 10.9 12.7
(i) Calculate the mean and standard deviation of these observed times.
(2)
(ii) Assuming that the underlying distribution of assembly times is Normal,
calculate 95% confidence intervals for the mean and for the standard deviation
of the assembly times.
(10)
(iii) In order to facilitate smooth operation of the entire production process, the
assembly times must satisfy certain conditions. Test at the 5% significance
level
(a) the hypothesis that the mean assembly time in the factory is 11 minutes
against the hypothesis that it is greater than 11 minutes,
(b) the hypothesis that the standard deviation of the assembly times in the
factory is 0.7 minutes against the hypothesis that it is greater than
0.7 minutes.
(8)
3
2. The number of births at a small maternity unit over a randomly selected three-week
period is shown in the table below, classified by the day of the week on which each of
the 294 births occurred.
Day Mon Tue Wed Thu Fri Sat Sun
Number of births 47 51 48 48 45 30 25
(i) Perform a test to investigate whether the numbers of births are uniformly
distributed across the seven days of the week. State your null hypothesis and
conclusions clearly.
(7)
(ii) Obtain a 95% confidence interval for the proportion of births that take place at
weekends (Saturday and Sunday). Use this interval to comment briefly on the
suggestion that fewer births occur at weekends than on other days of the week.
(6)
(iii) A review of medical procedures is undertaken and it is desired to investigate
whether, after this review, there has been a change in the proportion of births at
weekends. For a three-week period randomly chosen from the six months
following this review, it is observed that 68 out of 317 births are at weekends.
Test the null hypothesis that the proportion of births at weekends is unchanged
after the review against the alternative hypothesis that this proportion has risen.
(7)
4
3. An educational psychologist wishes to investigate the effect that the order of
examination questions on a paper has on anxiety levels in candidates. An examination
paper is prepared using identical questions in two versions. In version A the questions
are presented in order of difficulty with the easiest question first, whereas in version B
the questions are in reverse order with the easiest question last.
The 20 students in the class are assigned randomly to take the two different versions of
the examination paper, 10 taking each version. The following are measurements of an
anxiety index for the 20 students in suitable units, where low values of the index
indicate lower anxiety levels.
Version A 24.6 39.3 16.3 32.8 28.0 20.6 21.1 26.7 24.2 32.9
Version B 38.6 34.0 23.6 30.3 35.9 22.9 29.5 39.2 42.9 33.5
Summary values for the above are as follows.
Version A Version B
Sample mean = 26.65 Sample mean = 33.04
Sample variance = 47.05 Sample variance = 43.04
(i) The population variances for the anxiety indices for candidates taking the two
papers can be assumed to be equal. Assuming these populations to be
Normally distributed, calculate a 95% confidence interval for the difference in
mean anxiety levels for candidates taking the two versions of the examination.
Comment briefly on what this suggests about the effect on anxiety levels of the
two versions.
(10)
(ii) A statistician advises the educational psychologist that the scoring method used
to produce the anxiety index measurements may not produce values which are
Normally distributed. Analyse the data again, at the 5% significance level,
using a two-sided Wilcoxon rank sum test. State your null and alternative
hypotheses and your conclusions clearly.
(8)
(iii) Discuss the advice to use the Wilcoxon rank sum test, in particular the
advantages and disadvantages of doing so.
(2)
5
4. In a study of supermarket checkout equipment it is found that, although checkout
prices are often correct, customers can sometimes be charged more or less than the
prices posted on the shelves. It is suggested that discrepancies in prices may be
associated with whether or not items are on special offer compared with their normal
prices. A random sample of 819 items is investigated and for each item it is noted
whether the checkout equipment is undercharging, overcharging or charging the
correct price. The results are shown in the contingency table below.
Normal-priced items Special offer items Total
Undercharged 20 7 27
Overcharged 15 29 44
Correct price 384 364 748
Total 419 400 819
(i) Investigate whether or not price accuracy is associated with items being on
special offer or not. State your null and alternative hypotheses and report your
conclusions clearly.
(10)
(ii) Adapt the table to classify the normal-priced and special offer items according
to whether or not they are charged correctly. Calculate a 95% confidence
interval for the difference in the proportions of items which are charged
incorrectly in the normal-priced and special offer categories.
(8)
(iii) Comment briefly on your answers to parts (i) and (ii).
(2)
BLANK PAGE
6
7
BLANK PAGE
8
BLANK PAGE
